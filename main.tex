%%%%%%%%%%%%%%%%%%%%%%%%%%%%%%%%%%%%%%%%%
% Masters/Doctoral Thesis 
% LaTeX Template
% Version 2.5 (27/8/17)
%
% This template was downloaded from:
% http://www.LaTeXTemplates.com
%
% Version 2.x major modifications by:
% Vel (vel@latextemplates.com)
%
% This template is based on a template by:
% Steve Gunn (http://users.ecs.soton.ac.uk/srg/softwaretools/document/templates/)
% Sunil Patel (http://www.sunilpatel.co.uk/thesis-template/)
%
% Template license:
% CC BY-NC-SA 3.0 (http://creativecommons.org/licenses/by-nc-sa/3.0/)
%
%%%%%%%%%%%%%%%%%%%%%%%%%%%%%%%%%%%%%%%%%

%----------------------------------------------------------------------------------------
%	PACKAGES AND OTHER DOCUMENT CONFIGURATIONS
%----------------------------------------------------------------------------------------

\documentclass[
11pt, % The default document font size, options: 10pt, 11pt, 12pt
oneside, % Two side (alternating margins) for binding by default, uncomment to switch to one side
english, % ngerman for German
onehalfspacing, % Single line spacing, alternatives: onehalfspacing or doublespacing
%draft, % Uncomment to enable draft mode (no pictures, no links, overfull hboxes indicated)
nolistspacing, % If the document is onehalfspacing or doublespacing, uncomment this to set spacing in lists to single
%liststotoc, % Uncomment to add the list of figures/tables/etc to the table of contents
%toctotoc, % Uncomment to add the main table of contents to the table of contents
parskip, % Uncomment to add space between paragraphs
%nohyperref, % Uncomment to not load the hyperref package
headsepline, % Uncomment to get a line under the header
%chapterinoneline, % Uncomment to place the chapter title next to the number on one line
%consistentlayout, % Uncomment to change the layout of the declaration, abstract and acknowledgements pages to match the default layout
]{MastersDoctoralThesis} % The class file specifying the document structure

\usepackage{hyperref}
\usepackage[utf8]{inputenc} % Required for inputting international characters
\usepackage[T1]{fontenc} % Output font encoding for international characters

\usepackage{mathpazo} % Use the Palatino font by default
\usepackage{marginnote}
\usepackage{todonotes}

\usepackage[T1]{fontenc}
% \usepackage{siunitx}
\usepackage{graphics, subcaption}
\usepackage{booktabs}
\usepackage{adjustbox}
\usepackage{hhline}
\usepackage{pdflscape}
\usepackage{array}
\usepackage{rotating}
\usepackage{tabularx,ragged2e}
\newcolumntype{C}{>{\Centering\arraybackslash}X}
% \PassOptionsToPackage{latin1}{inputenc}
\usepackage{tabularray}
\definecolor{whitesmoke}{rgb}{0.96, 0.96, 0.96}

% \newenvironment{ctabularx}[1]{% or whatever name you like
%    \rowcolors{1}{}{lightgray}%
%    \tabularx{#1}%
%   }{%
%    \endtabularx
% }

% \usepackage[style=authoryear]{biblatex}

% \usepackage[backend=biblatex,style=authoryear,natbib=true]{biblatex} % Use the bibtex
% \usepackage[backend=biber,style=authoryear, citestyle=apa,natbib=true, sorting=anyt]{biblatex} % Use the bibtex backend with the authoryear citation style (which resembles APA)
% \usepackage[backend=biber,style=authoryear]{biblatex}
% \usepackage{apacite}
\usepackage[backend=biber,style=apa, citestyle=apa, natbib=true, sorting=anyt]{biblatex}

\addbibresource{C:\\Users\\bosse\\Documents\\UNIVERSITY\\masterthesis\\01_Masterarbeit.bib} % The filename of the bibliography

\usepackage[autostyle=true]{csquotes} % Required to generate language-dependent quotes in the bibliography

\usepackage{url}

\usepackage[acronym]{glossaries} %for no numbers: nonumberlist

% \addbibresource{01_Masterarbeit.bib} % The filename of the bibliography

\usepackage{enumitem}
\usepackage{wrapfig}
\usepackage{pdflscape}
\usepackage{float}
% \PassOptionsToPackage{table}{xcolor}
% \usepackage[table]{xcolor}


\newlist{questions}{enumerate}{2}
\setlist[questions,1]{label=RQ\arabic*.,ref=RQ\arabic*}
\setlist[questions,2]{label=(\alph*),ref=\thequestionsi(\alph*)}

%----------------------------------------------------------------------------------------
%	MARGIN SETTINGS
%----------------------------------------------------------------------------------------

\geometry{
	paper=a4paper, % Change to letterpaper for US letter
	inner=2.5cm, % Inner margin
	outer=2.5cm, % Outer margin
	bindingoffset=0.0cm, % Binding offset - was 0.5
	top=1.5cm, % Top margin
	bottom=1.5cm, % Bottom margin
	%showframe, % Uncomment to show how the type block is set on the page
}

%----------------------------------------------------------------------------------------
%	THESIS INFORMATION
%----------------------------------------------------------------------------------------

\thesistitle{Super duper fancy title. DONT FORGET TO CHANGE THIS!} % Your thesis title, this is used in the title and abstract, print it elsewhere with \ttitle
\supervisor{apl. Prof. Dr. Sven \textsc{Lautenbach}} % Your supervisor's name, this is used in the title page, print it elsewhere with \supname
\examiner{Prof. Dr. Alexander \textsc{Zipf}} % Your examiner's name, this is not currently used anywhere in the template, print it elsewhere with \examname
\degree{Master of Science} % Your degree name, this is used in the title page and abstract, print it elsewhere with \degreename
\author{Bosse \textsc{Sottmann}} % Your name, this is used in the title page and abstract, print it elsewhere with \authorname
\addresses{} % Your address, this is not currently used anywhere in the template, print it elsewhere with \addressname

\subject{Geography Sciences} % Your subject area, this is not currently used anywhere in the template, print it elsewhere with \subjectname
\keywords{} % Keywords for your thesis, this is not currently used anywhere in the template, print it elsewhere with \keywordnames
\university{\href{https://www.uni-heidelberg.de/en}{Heidelberg University}} % Your university's name and URL, this is used in the title page and abstract, print it elsewhere with \univname
\department{\href{https://www.geog.uni-heidelberg.de/index_en.html}{Institute of Geography}} % Your department's name and URL, this is used in the title page and abstract, print it elsewhere with \deptname
\group{\href{https://www.geog.uni-heidelberg.de/gis/index_en.html}{GIScience}} % Your research group's name and URL, this is used in the title page, print it elsewhere with \groupname
\faculty{\href{https://www.uni-heidelberg.de/fakultaeten/chemgeo/}{Faculty of Chemistry and Earth Sciences}} % Your faculty's name and URL, this is used in the title page and abstract, print it elsewhere with \facname

\AtBeginDocument{
\hypersetup{pdftitle=\ttitle} % Set the PDF's title to your title
\hypersetup{pdfauthor=\authorname} % Set the PDF's author to your name
\hypersetup{pdfkeywords=\keywordnames} % Set the PDF's keywords to your keywords
% change link color examples
% \hypersetup{urlcolor=red}
% \hypersetup{citecolor=green}
% \hypersetup{allcolor=blue}

% hide links
% \hypersetup{allcolors=.}, or even better:
\hypersetup{hidelinks}
% If you want to have obvious links in the PDF but not the printed text, use:
% \hypersetup{colorlinks=false}.
}


\makeglossaries

\newglossaryentry{formula}
{
    name=formula,
    description={A mathematical expression}
}

% in alphabetical order
\newacronym{aa}{AA}{Anticipatory Action}
\newacronym{brcis}{BRCiS}{Building Resilient Communities in Somalia Consortium}
\newacronym{cbm}{CBM}{Community-based monitoring}
\newacronym{cbwm}{CBWM}{Community-based Water Monitoring}
\newacronym{cbs}{CBS}{Community-based Surveillance}
\newacronym{cdrmc}{CDRMC}{Somaliland Community Disaster Risk Management Committees}
\newacronym{cocorahs}{CoCoRaHS}{Community Collaborative Rain, Hail and Snow Network}
\newacronym{crtrms}{CRTRMS}{Community Real-Time Risk Monitoring Systems}
\newacronym{cs}{CS}{Citizen Science}
\newacronym{dpia}{DPIA}{Data Protection Impact Assessment}
\newacronym{drm}{DRM}{Disaster Risk Management}
\newacronym{drr}{DRR}{Disaster Risk Reduction}
\newacronym{ea}{EA}{Early Action}
\newacronym{eap}{EAP}{Early Action Protocol}
\newacronym{ecsa}{ECSA}{European Citizen Science Association}
\newacronym{fao}{FAO}{Food and Agriculture Organization of the United Nation}
\newacronym{fba}{FbA}{Forecast based Action}
\newacronym{fbf}{FbF}{Forecast based Financing}
\newacronym{fewsnet}{FEWSNET}{Famine Early Warning Systems Network}
\newacronym{fsnau}{FSNAU}{Food Security and Nutrition Analysis Unit}
\newacronym{grc}{GRC}{German Red Cross}
\newacronym{gwp}{GWP}{Global Water Partnership}
\newacronym{heigit}{HeiGIT}{Heidelberg Institute for Geoinformation Technology}
\newacronym{idm}{IDM}{Integrated Drought Management}
\newacronym{idmp}{IDMP}{Integrated Drought Management Programme}
\newacronym{idp}{IDP}{Internally Displayed Person}
\newacronym{ifrc}{IFRC}{International Federation of Red Cross and Red Crescent Societies}
\newacronym{iwmi}{IWMI}{International Water Management Institute}
\newacronym{iwrm}{IWRM}{Integrated Water Resource Management}
\newacronym{mcs}{MCS}{Mobile Crowdsensing}
\newacronym{moad}{MoAD}{Ministry of Agricultural Development}
\newacronym{moh}{MoH}{Ministry of Health}
\newacronym{mowr}{MoWR}{Ministry of Water Resources}
\newacronym{ngo}{NGO}{Non-Governmental Organization}
\newacronym{nrc}{NRC}{Norwegian Red Cross}
\newacronym{nadfor}{NADFOR}{National Disaster Preparedness and Food Reserve Authority}
\newacronym{oam}{OAM}{Open Accessible and Mappable}
\newacronym{ocha}{OCHA}{United Nations Office for Coordination of Humanitarian Affairs}
\newacronym{prc}{PRC}{Project-Requirements-Catalogue}
\newacronym{qc}{QC}{Quality Control}
\newacronym{qa}{QA}{Quality Assurance}
\newacronym{rcrc}{RCRC}{Red Cross Red Crescent}
\newacronym{rcrccc}{RCRCCC}{Red Cross Red Crescent Climate Centre}
\newacronym{ssf}{SSF}{Six-Stage-Framework}
\newacronym{slmc}{SLMC}{Seven-layer model of collaboration}
\newacronym{ssdr}{SSDR}{Six-Stage-Design-Roadmap}
\newacronym{srcs}{SRCS}{Somalia Red Crescent Society}
\newacronym{swalim}{SWALIM}{Somali Water and Land Information Management}
\newacronym{undrr}{UNDRR}{United Nations Office for Disaster Risk Reduction}
\newacronym{un}{UN}{United Nations}
\newacronym{wfp}{WFP}{United Nations World Food Programme}
\newacronym{who}{WHO}{World Health Organization}
\newacronym{wmo}{WMO}{World Meteorological Organization}
\newacronym{wpi}{WPI}{Water Poverty Index}



\begin{document}

\frontmatter % Use roman page numbering style (i, ii, iii, iv...) for the pre-content pages

\pagestyle{plain} % Default to the plain heading style until the thesis style is called for the body content

%----------------------------------------------------------------------------------------
%	TITLE PAGE
%----------------------------------------------------------------------------------------

\begin{titlepage}
\begin{center}

\vspace*{.06\textheight}
{\scshape\LARGE \univname\par}\vspace{1.5cm} % University name
\textsc{\Large Master Thesis}\\[0.5cm] % Thesis type

\HRule \\[0.4cm] % Horizontal line
{\huge \bfseries \ttitle\par}\vspace{0.4cm} % Thesis title
\HRule \\[1.5cm] % Horizontal line
 
\begin{minipage}[t]{0.4\textwidth}
\begin{flushleft} \large
\emph{Author:}\\
{\authorname}
%\href{http://www.johnsmith.com}{\authorname} % Author name - remove the \href bracket to remove the link
\end{flushleft}
\end{minipage}
\begin{minipage}[t]{0.4\textwidth}
\begin{flushright} \large
\emph{Supervisor:} \\
{\supname}\\
%\href{http://www.jamessmith.com}{\supname} % Supervisor name - remove the \href bracket to remove the link  
\vspace{0.5cm}
\emph{Examiner:}\\
{\examname}
\end{flushright}
\end{minipage}\\[3cm]
 
\vfill

\large \textit{A thesis submitted in partial fulfillment of the requirements\\ for the degree of \degreename}\\[0.3cm] % University requirement text
\textit{in the}\\[0.4cm]
\deptname\\\facname\\[2cm] % Research group name and department name
 
\vfill

{\large \today}\\[4cm] % Date
%\includegraphics{Logo} % University/department logo - uncomment to place it
 
\vfill
\end{center}
\end{titlepage}

%----------------------------------------------------------------------------------------
%	DECLARATION PAGE
%----------------------------------------------------------------------------------------

\begin{declaration}
\addchaptertocentry{\authorshipname} % Add the declaration to the table of contents
\noindent I, \authorname, declare that this thesis titled, \enquote{\ttitle} and the work presented in it are my own. I confirm that:

\begin{itemize} 
\item This work was done wholly while in candidature for a research degree at this University.
\item Where any part of this thesis has previously been submitted for a degree or any other qualification at this University or any other institution, this has been clearly stated.
\item Where I have consulted the published work of others, this is always clearly attributed.
\item Where I have quoted from the work of others, the source is always given. With the exception of such quotations, this thesis is entirely my own work.
\item I have acknowledged all main sources of help.
\item Where the thesis is based on work done by myself jointly with others, I have made clear exactly what was done by others and what I have contributed myself.\\
\end{itemize}
 
\noindent Signed:\\
\rule[0.5em]{25em}{0.5pt} % This prints a line for the signature
 
\noindent Date:\\
\rule[0.5em]{25em}{0.5pt} % This prints a line to write the date
\end{declaration}

\cleardoublepage

%----------------------------------------------------------------------------------------
%	QUOTATION PAGE
%----------------------------------------------------------------------------------------

\vspace*{0.2\textheight}

\centerline{\noindent\enquote{\itshape"Can't A Guy Make One Mistake?"}}\bigbreak

\hfill {\small Baloo, The Jungle Book}

%----------------------------------------------------------------------------------------
%	ABSTRACT PAGE
%----------------------------------------------------------------------------------------

\begin{abstract}
\addchaptertocentry{\abstractname} % Add the abstract to the table of contents

Ensuring water security is considered as one of the major challenges of the twenty-first century. The trend of increasing demand and diminishing supplies is putting pressure on the availability of water worldwide. Particularly in the Horn of Africa, drought impacts determine the life of millions of people. Somaliland is in the midst of a years-long drought and water sources become more important than ever. Yet, information particularly about the most important water source type of berkads is incomplete and outdated. 

Insufficient data availability can severely hamper disaster risk reduction measures, especially with regard to Forecast-based Financing (FbF), a proactive natural disaster response approach that has recently become increasingly widespread. Triggered by predicted disaster impacts, Anticipatory Actions (AAs) attempt to counteract impacts before the disaster occurs, rather than responding to post-disaster impacts. However, drought is a relatively novel application focus for this approach and is highly dependent on relevant information about local impacts.\newline
One way to gather these information can be Citizen Science (CS), which has successfully been applied to provide data for acting on environmental issues primarily in North America and Europe. In addition, sub-categories of CS such as community-based monitoring, together with mobile crowdsensing, already form the conceptual backbone for the Somalia Red Crescent Society's health-related Community-based Surveillance project.\newline
Building on the combination of these concepts, the aim of this study is to first develop a new and transferable approach for community-based participatory mapping and monitoring of water sources for water-scarce and resource-limited settings to facilitate relevant AAs in the context of FbF. This framework will subsequently be applied to create an implementation roadmap for the SRCS, ultimately aiming to improve water governance and information availability to address water scarcity in Somaliland.

The work is embedded in a primarily inductive design of an exploratory, iterative case study, and guided by a mixed-methods approach combining literature analysis and expert consultations. The results indicate that it is conceptually possible to integrate the concepts of FbF and CS for monitoring water sources in resource scarce settings to eventually trigger AAs within one framework. Moreover, in the case of Somaliland, it can also reasonably be assumed that the practical feasibility of this integrated framework is given. On this basis, future work will be able to integrate and assess local information in a pilot study, thereby overcoming the main limitations of this work due to resource, time and information constraints.


\end{abstract}

%----------------------------------------------------------------------------------------
%	ACKNOWLEDGEMENTS
%----------------------------------------------------------------------------------------

\begin{acknowledgements}
\addchaptertocentry{\acknowledgementname} % Add the acknowledgements to the table of contents
\vspace{1cm}

I would like to express my deep gratitude to \examname\,\,for providing me with the opportunity to write my thesis in the frame of the GIScience at the University Heidelberg.\newline
To \supname\,\,and \textit{Dr. Michael Auer} for their time, continuous guidance and great humour during our numerous meetings.\newline
To \textit{Melanie Eckle-Elze} and the \textit{HeiGIT Disaster Team} for their willingness to involve me in the project, practical help and encouraging support.

I would also like to express my gratitude to all interviewees for their valuable time and shared perspectives and experiences. Their contribution has been indispensable to the success of this study.

Special thanks to my \textit{family}, \textit{flatmates} and \textit{friends} for providing me with unfailing support and continuous encouragement throughout my years of study and through the process of researching and writing this thesis. This accomplishment would not have been possible without you. Thank you.

\end{acknowledgements}

%----------------------------------------------------------------------------------------
%	LIST OF CONTENTS/FIGURES/TABLES PAGES
%----------------------------------------------------------------------------------------
% \printglossary[type=\acronymtype]
% \printglossary

\tableofcontents % Prints the main table of contents

\listoffigures % Prints the list of figures

\listoftables % Prints the list of tables

%----------------------------------------------------------------------------------------
%	ABBREVIATIONS
%----------------------------------------------------------------------------------------

\begin{abbreviations}{ll} % Include a list of abbreviations (a table of two columns)

% \textbf{LAH} & \textbf{L}ist \textbf{A}bbreviations \textbf{H}ere\\
% \textbf{WSF} & \textbf{W}hat (it) \textbf{S}tands \textbf{F}or\\
\printglossary[type=\acronymtype, title=List of Abbreviations]

% \printglossary[type=\acronymtype]
% \printglossary[type=\acronymtype]
% \printglossary
% \printglossary[title=Special Terms, toctitle=List of terms]
\end{abbreviations}

%----------------------------------------------------------------------------------------
%	PHYSICAL CONSTANTS/OTHER DEFINITIONS
%----------------------------------------------------------------------------------------

% \begin{constants}{lr@{${}={}$}l} % The list of physical constants is a three column table

% % The \SI{}{} command is provided by the siunitx package, see its documentation for instructions on how to use it

% Speed of Light & $c_{0}$ & \SI{2.99792458e8}{\meter\per\second} (exact)\\
% %Constant Name & $Symbol$ & $Constant Value$ with units\\

% \end{constants}

%----------------------------------------------------------------------------------------
%	SYMBOLS
%----------------------------------------------------------------------------------------

% \begin{symbols}{lll} % Include a list of Symbols (a three column table)

% $a$ & distance & \si{\meter} \\
% $P$ & power & \si{\watt} (\si{\joule\per\second}) \\
% %Symbol & Name & Unit \\

% \addlinespace % Gap to separate the Roman symbols from the Greek

% $\omega$ & angular frequency & \si{\radian} \\

% \end{symbols}

%----------------------------------------------------------------------------------------
%	DEDICATION
%----------------------------------------------------------------------------------------

% \dedicatory{For/Dedicated to/To my\ldots} 

%----------------------------------------------------------------------------------------
%	THESIS CONTENT - CHAPTERS
%----------------------------------------------------------------------------------------

\mainmatter % Begin numeric (1,2,3...) page numbering

\pagestyle{thesis} % Return the page headers back to the "thesis" style

% Include the chapters of the thesis as separate files from the Chapters folder
% Uncomment the lines as you write the chapters

% % Chapter Template

\chapter{Introduction} % Main chapter title

\label{Chapter1} % Change X to a consecutive number; for referencing this chapter elsewhere, use \ref{ChapterX}

%  possibly just one research question with some sub-questions
%----------------------------------------------------------------------------------------
%	SECTION 0 Proposal
%----------------------------------------------------------------------------------------
% The introduction should clearly establish the focus and purpose of the literature review.
% highlight research gap + emphasize the timeliness
% Locate your own research within the context of existing literature [very important!]. 
% reiterate central problem
% focus and purpose of literature review
% brief summery scholarly context
% highlight research gap
% emphasize timeliness
% isn't this all too much?

https://gradcoach.com/research-aims-objectives-questions/

What
Why
How
--> starting point
% mayyybeeee just mayyybe write that the first tow objectives and research questions are interim results and are addressed during the study, whereas the third and fourth objective/question is only discussed at the end.

Overview: 7 Common Introduction Mistakes
1. Not providing sufficient context for the study
-> sufficient contextual foundation (literature and real world)
-> what, where, who, when
-> what the fk is going on! -> foundation for research justification
write for the intelligent layman (intellectual curious but not an expert)
start from the bottom

2. Not presenting a strong justification for the research topic
justify research aims (not only: it hasn't been done before -> originality is important but not the only thing)
-> discuss novelty of the project
-> also the practical and theoretical importance (more research in the global south)
(a) What are you researching (and how is this novel/original)?
(b) Why is it important (will add value to the field)?
(c) Who is going to benefit from the research or who will struggle without it?

3. Having a research topic that’s too broad
-> not too broad
-> tightly define the research \textbf{aim}


4. Having poorly defined research aims, objectives and research questions
-> aim, objectives and questions


5. Having misaligned research aims, objectives and research questions


6. Not having well-defined and/or justified scope
narrow scope and specify boundaries
where, when, who --> focused, manageable, replicable
-> justify scope
-> mention how much research has already been undertaken in the research area of interest
-> say one it hasn't been done (could have been done, but as I know from I2, publications are rare)

7. Not providing a clear structural outline of the document
-> detailed and clear outline of the thesis, -> document structure -> orient the reader -> where to find stuff
does not need to be lengthy -> a line or two for each chapter

% research aims 

% This study sets out to assess and apply community-based participatory water source mapping and monitoring frameworks, designs and applications in a resource restricted environment in cooperation with a national non-government organisation.

% The aim of this study is to explore, potentially adapt and apply community-based participatory mapping and monitoring guidelines and frameworks for water sources in a water-scarce and resource-limited setting in collaboration with a national non-governmental organisation.

% research aim
% The study sets out to design an approach for community-based participatory mapping and monitoring of water sources in a water-scarce and resource-limited setting in collaboration with a national non-governmental organisation to facilitate respective \acrlongpl{aa} in the context of \acrlong{fbf}.

% The aim of this study is to design and test an approach for community-based participatory mapping and monitoring of water sources in a water-scarce and resource-limited setting in collaboration with a national non-governmental organization to facilitate respective \acrlongpl{aa} in the context of \acrlong{fbf}, with the goal of improving water management and accessibility in underserved communities.

% % research objectives
% 1. To understand the context and explore principles, guidelines and best practices for the application of a \acrlong{cs} programme.
% 2. To assess the feasibility of the \acrlong{cs} approach in the given context.
% 3. To potentially adjust and expand those frameworks and recommendations to the research aim and overall case study context.
% 4. To apply the framework(s) in order to create a roadmap for the design approach in regard to the prevailing conditions. 

% % research questions
% 1. What guidelines and best practices exist for the design of a \acrlong{cs} project in the thematic direction of the research aim.
% 2. Which combination of frameworks best suits this approach and how can they further be adjusted to the research aim?
% 3. How can a \acrlong{cs} project be conceptualised for the application of a community-based participatory water source monitoring programme with special attention to its feasibility for \acrlongpl{aa} in a resource restricted environment?
% 3.1. What information needs to be known about the source initially and continuously?

% you are an experienced grad coach with prolonged experience in designing and writing Master Theses. Give your detailed feedback to the following chain of research aim, objectives and questions and outline potential improvements by providing several reformulations with explanations why those are superior.

as: preparatory study (??)


%----------------------------------------------------------------------------------------
%	SECTION 1 topic and context / background motivation
%----------------------------------------------------------------------------------------
background, timely, importance
% Background and motivation for the research / Topic and Context
% topic and context (academic debate, practical problem)

% Begin by introducing your dissertation topic and giving any necessary background information. It’s important to contextualize your research and generate interest. Aim to show why your topic is timely or important. You may want to mention a relevant news item, academic debate, or practical problem.




% climate change

“Climate variability, climate change and global trends in drought hazard” ([“Special report on drought 2021”, 2021, p. 29](zotero://select/groups/4773535/items/RAAM9PVS)) ([pdf](zotero://open-pdf/groups/4773535/items/7AK5QVBL?page=31&annotation=7NYZ8XDU))
“Figure 1.3. Change in meteorological drought frequency (events/decade) from recent past (1981–2010) to 2100 for four projected warming levels of global surface air temperature (left) and number of drought events with stronger severity than ever recorded in the recent past (1981–2010) (right)” ([“Special report on drought 2021”, 2021, p. 32](zotero://select/groups/4773535/items/RAAM9PVS)) ([pdf](zotero://open-pdf/groups/4773535/items/7AK5QVBL?page=34&annotation=D7VGP69I))

% IPCC
“9.5.5 East Africa” ([Adelekan et al., 2022, p. 1327](zotero://select/groups/4773535/items/8HKQ6EZ3)) ([pdf](zotero://open-pdf/groups/4773535/items/9H76NWQ2?page=43&annotation=U8RXJHIH))

“Climate change has increased heat waves (high confidence) and drought (medium confidence) on land, and doubled the probability of marine heatwaves around most of Africa (high confidence). Multi-year droughts have become more frequent in west Africa, and the 2015–2017 Cape Town drought was three times more likely2 due to human-caused climate change. {9.5.3–7, 9.5.10} Increases in drought frequency and duration are projected over large parts of southern Africa above 1.5°C global warming (high confidence), with decreased precipitation in North Africa at 2°C global warming (high confidence), and above 3°C global warming, meteorological drought frequency will increase, and duration will double from approximately 2 months to 4 months in parts of North Africa, the western Sahel and southern Africa (medium confidence). {9.5.2, 9.5.3, 9.5.6.}” ([Adelekan et al., 2022, p. 1290](zotero://select/groups/4773535/items/8HKQ6EZ3)) ([pdf](zotero://open-pdf/groups/4773535/items/9H76NWQ2?page=6&annotation=DTHH5TGR))


% Hazards and Drought

With the projection of rising frequency and intensity of droughts throughout vast parts of the African continent, measures for prediction, monitoring and evidence based anticipatory actions and management become ever more important \autocite{abdulkadirAssessmentDroughtRecurrence2017,adelekanAfricaClimateChange2022,vereintenationenSpecialReportDrought2021}.

“known to have the most far-reaching impacts of all natural disasters” ([Abdulkadir, 2017, p. 104172233225941010000](zotero://select/groups/4773535/items/G2RYLAC4)) ([pdf](zotero://open-pdf/groups/4773535/items/5PPIPUKD?page=1&annotation=C6R6KIS4))


% talk about water security and water scarcity
“Water scarcity, as a supply/demand-driven and natural and/or human-made phenomenon, is one of the greatest challenges of the twenty-first century” ([pdf](zotero://open-pdf/groups/4773535/items/JM82W3ZF?page=13&annotation=WZB8I8FY))

https://www.unwater.org/our-work/integrated-monitoring-initiative-sdg-6

whatever source of water they can find is what they have. (2023-03-04_Beledi, Pos. 20)


“Countries in which less than 50\% of the population uses improved drinking water sources are all located in sub-Saharan Africa and Oceania 91-100\% 76-90\% 50-75\% <50\% insufficient data or not applicable Proportion of the population using improved drinking water sources in 2015 ■ 91–100\% ■ 76–90\% ■ 50–75\% ■ <50\% ■ INSUFFICIENT DATA OR NOT APPLICABLE” ([World Health Organization, 2016, p. 15](zotero://select/groups/4773535/items/KVAKZ9ZT)) ([pdf](zotero://open-pdf/groups/4773535/items/4STYK52H?page=14\&annotation=FBURDS4T))



% from global to regional (horn of Africa) to local (Somalia - Somaliland)

“Somaliland is characterized by drought” ([Abdulkadir, 2017, p. 104172233225941010000](zotero://select/groups/4773535/items/G2RYLAC4)) ([pdf](zotero://open-pdf/groups/4773535/items/5PPIPUKD?page=1&annotation=PTQSFYGD))






% case study - would fit better in the other way around (as the TH)
Somaliland, being no exception to the above mentioned climatic trend, is characterized by droughts with far reaching impacts on ecological, economic, and social aspects. Defined by a semi-arid, four-season climate with two extensive dry seasons and an economic backbone of pastoralism and rain-fed agriculture, water accessibility is of key importance in Somaliland \autocite{abdulkadirAssessmentDroughtRecurrence2017,petrucciLandscapeLandformsNorthern2022,republicofsomalilandSomalilandCountryProfile2021}.
In addition, the final report of the FbF feasibility study identifies five other hazards besides droughts, namely flood, cyclone, disease, locusts and conflict. Of all these hazards, drought was ranked as the greatest threat due to its increasing frequency, severity and wide-ranging consequences.\newline

% historic events -> from the past to the present (response) to the future (anticipatory)
“Spotlight on Somalia: Can we learn from failure? In Somalia in 2011, a famine was declared that, along with the complexity of the conflict situation, was responsible for thousands of deaths. At its peak, almost 4 in every 10 children in Southern Somalia were acutely malnourished, and 4 million people were estimated to be without basic food. The horror of this tragedy has since haunted the international community, who received 11 months of early warnings before a famine was declared. Beginning with La Nina forecasts almost one year in advance, FEWS-NET and others provided briefing notes and warning information to humanitarian actors in the region. Several months later, these alerts explained that rainy seasons had already failed, and that major impacts were extremely likely (Hillbruner and Moloney 2012). There has been a great deal of analysis of this event, in which several conclusions have come to light. One is that funding needs to be more readily available based on forecasted information (Lautze et al. 2012). The below graph (Hillbruner and Moloney 2012) demonstrates how large-scale funding was mobilized in the aftermath of the famine declaration, and was ultimately available after the most vulnerable had died. Secondly, the humanitarian community needs to clearly take responsibility for acting in advance of a disaster, even in complex cases like the Somali context. At the moment, such organizations are not held accountable for failure to act on early warning, as disaster response is considered business-asusual. Shouldering the responsibility to act in this critical moment between a warning and a disaster could avoid such impacts in the future (Lautze et al. 2012).” ([pdf](zotero://open-pdf/groups/4773535/items/WKFPUZCW?page=7&annotation=LIY2ABLI))

% also in case study
“The number of affected people will be 1,200,420 persons across all the six main regions in Somaliland. The top priority needs of the people affected to date are mainly water (70{\%), Food (21\%) and Health (9\%).” ([National Drought Committee, 2022, p. 3](zotero://select/groups/4773535/items/7XJRE6LM)) ([pdf](zotero://open-pdf/groups/4773535/items/2F59E4UZ?page=3&annotation=8JZVBSM6))


%----------------------------------------------------------------------------------------
%	SECTION 2 Focus and Scope
%----------------------------------------------------------------------------------------
narrow, pin point
% After a brief introduction to your general area of interest, narrow your focus and define the scope of your research.

% You can narrow this down in many ways, such as by:

% Geographical area
% Time period
% Demographics or communities
% Themes or aspects of the topic
% transition: this will help to tackle worsening situation

% ground truthing
% local impact forecasts
% local knowledge (while it could integrate local knowledge more, this work will only touch this topic)
% Volunteersensing
% mapping and monitoring with Crowdsensing
% EAP & AAs & trigger



% Early Actions / Anticipatory Actions / FbF

In order to meet this challenge, the Red Cross Red Crescent Movement together with the Red Cross Red Crescent Climate Center started the Forecast-based Financing (FbF) programme in 2007 to facilitate Anticipatory Actions instead of post-disaster reactions. Together with their local partners, the International Federation of Red Cross Red Crescent Societies (IFRC) is working on establishing so called Early Action Protocols (EAPs) to ensure better organization and coordination of Anticipatory Actions in the face of an incoming hazard. These actions are based on a predefined interplay of forecast, trigger and financing mechanisms to ensure rapid, scientific based responses.

% based on forecasts -> Forecasts
% drought forecasts
Existing drought forecasts and trigger indicators are mostly based on physical indicators and cover especially macro- and international levels, e.g. EDDI, SPI, SPEI \todo{look them up + which forecast is defined by the SRCS?}. Fine grained up-to-date forecasts which not only include physical but also social circumstances and knowledge on local levels are often scarce or even non-existent \todo[fancyline]{sources}. "However, assessments focused only on physical variables and processes fail to capture why drought matters [...]."\autocite[3]{lackstromBackyardHydroclimatologyCitizen2022}, how it impacts communities and which mitigation strategies are locally taken \todo{source?}.\newline

% impact of drought on the community
% --> impact forecasts
% --> mapping & monitoring of water source type Berkad

“The tools also tend to ignore the at risk community who happen to be host to very crucial traditional knowledge on droughts” ([Masinde and Bagula, 2010, p. 390](zotero://select/groups/4773535/items/JNC4ACZS)) ([pdf](zotero://open-pdf/groups/4773535/items/IWMKDQYV?page=1&annotation=VUYYP2LA))

% local knowledge
% ahh well.. maybe cut this down a bit. Local knowledge was not that important in the end..  
Besides the further development of more fine grained technical solutions the integration of local knowledge is another way forward. Engaging local people and communities and giving them an active voice in defining and co-producing anticipatory actions and knowledge can be of multiple benefit to communities and enrich the data generated \autocite{somaliredcrescentsocietyFeasibilityStudyPotential2022, njambi-szlapkaIntegratingCommunityVoices}\todo{source}. On the one hand, citizens can help to fill data gaps of categorized measurements such as simple assessments of dry-to-wet conditions which correspond to the above mentioned technical drought indicators \autocite{lackstromBackyardHydroclimatologyCitizen2022}. On the other hand, citizens can contribute their local knowledge which can potentially draw on years of experience, encompass a wide range of aspects and give them a feeling of co-owning which in itself can already bring many advantages \autocite{njambi-szlapkaIntegratingCommunityVoices}\todo{source}. The IFRC states, that the "community engagement and accountability (CEA) is essential […] to build acceptance and trust” \autocite{ifrcCommunityEngagementAccountability}(IFRC n.d.) for effective and sustainable outcomes. 
Nonetheless, direct contributions and communication from and with volunteers or community members remain a challenge in the joint management of hazards and risks. The tasks are numerous and need to take into consideration different aspects, ranging from cultural differences to different background knowledge and technical capabilities and capacities.



% specify, thought that this would be about the methodology
Qualitative interviews can capture a wide range of local contribution, knowledge and information but are very labour intensive, often limited in time and hence not feasible on larger scale for monitoring purposes.

% CS
“Involving stakeholders in the co-creation of WS assessment tools [38] could help increase the legitimacy and salience of data. Local knowledge should be incorporated into citizen science strategies: people are sources of information but they also generate a common understanding of water dynamics in the context of the socio-ecological processes involved. Co-creation will facilitate the communication of WS data in a harmonised and aggregated way that improves decision making based on real scenarios with actively involved and engaged stakeholders [13,22,91,108].” ([Butte et al., 2022, p. 10](zotero://select/groups/4773535/items/QB97YZ2M)) ([pdf](zotero://open-pdf/groups/4773535/items/Q936I2JN?page=10&annotation=JNYCQVSA))

“Stakeholders should not be seen only as information receivers but also as active players in the identification of information needs [150].” ([Butte et al., 2022, p. 13](zotero://select/groups/4773535/items/QB97YZ2M)) ([pdf](zotero://open-pdf/groups/4773535/items/Q936I2JN?page=13&annotation=9GZ2FNKW))

“With the advance of citizen science over the last decade, amateurs and non-experts from local communities have contributed in the various steps of the data cycle, from collection to analysis and interpretation” ([Butte et al., 2022, p. 7](zotero://select/groups/4773535/items/QB97YZ2M)) ([pdf](zotero://open-pdf/groups/4773535/items/Q936I2JN?page=7&annotation=E5NG5FU6))
“Previous studies have demonstrated the advantages of citizen science in: capturing data during episodic flooding events [56], gathering water related data over data-poor regions, especially in the Global South [57], and collecting information related to previous years, where historical data might be missing.” ([Butte et al., 2022, p. 7](zotero://select/groups/4773535/items/QB97YZ2M)) ([pdf](zotero://open-pdf/groups/4773535/items/Q936I2JN?page=7&annotation=EPZWNPJ6))

citizen science in the field of water monitoring: rivers, lakes, groundwater but hardly ever any direct drinking sources for early actions
for the first part: compare introduction of: “Citizen science projects in freshwater monitoring. From individual design to clusters?” ([Kirschke et al., 2022, p. 1](zotero://select/groups/4773535/items/GPC3LDT5)) ([pdf](zotero://open-pdf/groups/4773535/items/AI7HRQYC?page=1&annotation=6IJERY2C))


“This shift is reflected globally through policy initiatives of the United Nations such as Agenda 21 or the Aarhus Convention which emphasize that “the serious environmental, social, and economic challenges faced by societies worldwide cannot be addressed by public authorities alone” (UNECE, 2008).” ([Weston and Conrad, 2015, p. 1](zotero://select/groups/4773535/items/49HXDHSH)) ([pdf](zotero://open-pdf/groups/4773535/items/CCHM5SNH?page=1&annotation=DEWMV6FY))

% + potentially showing how to advance water security data gathering strategies with citizen science
“Figure 4. The figur” ([Butte et al., 2022, p. 6](zotero://select/groups/4773535/items/QB97YZ2M)) ([pdf](zotero://open-pdf/groups/4773535/items/Q936I2JN?page=6&annotation=LGYJQ9CS))

This works follows the “general public, typically as part of a collaborative project with professional scientists” ([Kirschke et al., 2022, p. 2](zotero://select/groups/4773535/items/GPC3LDT5)) ([pdf](zotero://open-pdf/groups/4773535/items/AI7HRQYC?page=2&annotation=BTE2JFKV)) which is basically “community-based monitoring, which also explicitly focuses on the public ’s involvement in monitoring processes” ([Kirschke et al., 2022, p. 2](zotero://select/groups/4773535/items/GPC3LDT5)) ([pdf](zotero://open-pdf/groups/4773535/items/AI7HRQYC?page=2&annotation=K4URWWLV))

\autocite[20]{gualazziniEWEAEarlyWarning2021} highlights the value of \acrshort{cbm} triangulated data as it ensures the consideration of local perspective in "high-level aggregated data and early action decision-making processes" and decentralises disaster management. 


“Furthermore it is an interesting communication tool in the light of science communication. Correspondingly water managers should be interested in participatory monitoring in the light of integrated water management.” ([Minkman, 2015, p. 199](zotero://select/groups/4773535/items/ZKLE6CPT)) ([pdf](zotero://open-pdf/groups/4773535/items/QMAPCSZG?page=199&annotation=GHI9KSDA))


“Local citizens are seen as essential participants in collaborative environmental management because they can provide vital information about the area’s natural and sociopolitical systems as well as support for measures to address non–point source pollution” ([Koehler and Koontz, 2008, p. 143](zotero://select/groups/4773535/items/2PKW9CYX)) ([pdf](zotero://open-pdf/groups/4773535/items/YYG4JK3R?page=1&annotation=MV6JAERB))


“Within these citizen science projects, monitoring mainly focuses on rivers (48%) but also considers lakes (26%), groundwater (12.57%), and estuaries (13.33%)” ([Kirschke et al., 2022, p. 4](zotero://select/groups/4773535/items/GPC3LDT5)) ([pdf](zotero://open-pdf/groups/4773535/items/AI7HRQYC?page=4&annotation=QGL5N2BA))
% while there are lots of hydrological monitoring stuff out there (mostly crowdsensing in the case of water + focused on health by CBS), there is none, that looks at drinking water source monitoring direktly by applying the proven sets of methods on a new topic in a new environment (not North America but resource scarce environments from the perspective of an NGO (SRCS)) 
“Citizen scientists measure physical (34.22%, e.g., temperature, turbidity, color), chemical (34.76%, e.g., pH, nitrates, phosphates, dissolved oxygen), biological (22.99%, e.g., fecal coliform bacteria, algae, aquatic macroinvertebrates), and some other parameters (8.02%, e.g., macro/microplastic pollution, riparian habitat), demonstrating the diverse tasks citizen scientists engage in. Citizen scientists further measure these parameters rather ‘regularly’ (82.93%) than ‘sometimes only’ (17.07%), indicating continuity and thus a substantial contribution of citizens to the monitoring process.” ([Kirschke et al., 2022, p. 4](zotero://select/groups/4773535/items/GPC3LDT5)) ([pdf](zotero://open-pdf/groups/4773535/items/AI7HRQYC?page=4&annotation=6GC8UE8I))

% benefits
Providing a technical solution that enables mutual contact between communities and the Somalia Red Crescent Society (SRCS) in a simple and accessible way would make it possible to collect this most helpful information. By ensuring that sovereignty over the collected data rests with the community, the decision-making power and benefits remain with those concerned. Therefore, in the development and application of an Early Action Protocol (EAP) in the context of Forecast based Action (FbA) in Somalia, the technical aspect in particular holds great potential to benefit numerous communities over the long term.

While there is a comparatively good basis with regard to the general availability of data on social, economic and natural information and conditions, as well as their spatial and qualitative characteristics, the timeliness of this information varies greatly. Especially in the context of Anticipatory Action, time is of the essence and a swift action based on up-to-date data is crucial. Gathering and incorporating local knowledge through manual surveys by a central organisation can be of great value, as shown above, but is time-consuming and slow.


--> highlight the pro of this work
"The highly localized information provided by observers can fill drought monitoring gaps by ground-truthing quantitative indicators and offering information in places where other monitoring tools may not exist. Overall, the research team found that strategic investments in time and funding can help fill in geographic and temporal gaps in drought monitoring information through volunteer observations."
https://www.drought.gov/news/research-confirms-role-citizen-science-contributions-drought-detection-and-monitoring and https://doi.org/10.1175/BAMS-D-21-0157.1



%----------------------------------------------------------------------------------------
%	SECTION 3 Relevance and Importance --> justification
%----------------------------------------------------------------------------------------
motivation, relation, insights, reasoning
% It’s essential to share your motivation for doing this research, as well as how it relates to existing work on your topic. Further, you should also mention what new insights you expect it will contribute.

% Start by giving a brief overview of the current state of research. You should definitely cite the most relevant literature, but remember that you will conduct a more in-depth survey of relevant sources in the literature review section, so there’s no need to go too in-depth in the introduction.

% Depending on your field, the importance of your research might focus on its practical application (e.g., in policy or management) or on advancing scholarly understanding of the topic (e.g., by developing theories or adding new empirical data). In many cases, it will do both.

% Ultimately, your introduction should explain how your thesis or dissertation:

% Helps solve a practical or theoretical problem
% Addresses a gap in the literature
% Builds on existing research
% Proposes a new understanding of your topic
% transition: this will help to tackle worsening situation

% state of the art (!!!!!!!!!!)
%technical feasibility: important here? --> make this the state of the research
The benefits of overcoming the high labour costs and the temporal handicap by providing a simple and accessible tool for the SRCS’s large network of volunteers can already be seen in other comparable contexts using the Community Based Surveillance (CBS) programme, its successor NYSS, the Ushahidi platform or tools like Social.Water, ITIKI and CoCoRaHS \autocite{fienenSocialWaterCrowdsourcing2012a} \todo{sources}. The CBS is a pioneering technical approach to disease outbreak surveillance in Somalia and was developed as part of the SRCS Health Strategy 2019-2023. In this process, “volunteers report health risks through their respective locations and send to the SRCS data platform […]” \autocite[57]{somaliredcrescentsocietyFeasibilityStudyPotential2022} through coded SMS. So far, 315 volunteers from 115 villages have specifically been trained. In at least two cases, an outbreak could be contained thanks to early detection (SRCS 2021, 2022). On top of this already proven application, further functions could be added as needed.
Social.Water, ITIKI and CoCoRaHS are environmental data collection and drought monitoring implementations. These approaches are based on Crowdsensing approaches and partly employ more sophisticated technical solutions via internet connections, wireless sensor networks or photographs and have been implemented and tested in recent studies or ongoing monitoring programmes. While suggesting that the method of Crowdsensing for data collection and monitoring is fit for purpose, these approaches are not feasible to this context due to internet connection and technical equipment requirements, lack of categorization or their focus on just environmental data acquisition without taking sufficient account of the impact on the community impact and social realities.

“Citizen science programmes are promising cost-efficient methods to monitor environmental resources, which make them especially suitable for low-income countries to overcome their sparse data resolution.” ([Weeser et al., 2018, p. 1598](zotero://select/groups/4773535/items/SFA2MLHC)) ([pdf](zotero://open-pdf/groups/4773535/items/GP79FHFC?page=9&annotation=4E9JCTQ5))
“Since today's citizen science studies are mostly located in high-income countries, we are enthusiastic to motivate the scientific community to conduct citizen science studies in low-income countries.” ([Weeser et al., 2018, p. 1598](zotero://select/groups/4773535/items/SFA2MLHC)) ([pdf](zotero://open-pdf/groups/4773535/items/GP79FHFC?page=9&annotation=TYD7Q2ZD))

% keeping data up to date is crucial in ensuring correct vulnerability and exposure data 
“Vulnerability and exposure changes over time, particularly after an extreme weather or climate event. Datasets must be kept up to date to ensure the impact-based forecast or warning using this data is reliable. Recognise that many official governmental data sources, such as a national census or demographic and health surveys, are updated infrequently – every five or ten years.” ([Harrowsmith et al., 2020, p. 28](zotero://select/groups/4773535/items/QJ397Y54)) ([pdf](zotero://open-pdf/groups/4773535/items/2GS362N5?page=28&annotation=5XVAQCTY))

“Intervening early to respond to spikes in need – i.e. before negative coping strategies are employed - can deliver significant gains and should be prioritized.” ([USAID, 2018, p. 6](zotero://select/groups/4773535/items/LGRWAU43)) ([pdf](zotero://open-pdf/groups/4773535/items/MBXSCVWR?page=6&annotation=C47BGB9V))

% current limitations


% in regard to Somalia and BRCiS rationale to perform more monitoring
“Prior to 2019, BRCiS consortium members did not have a single primary data collection mechanism for making informed decisions and bridging the gap between the national-level early warning systems and community information needs. Although there was substantive commonality between the members’ individual systems, key challenges remained.” ([Gualazzini, 2021, p. 4](zotero://select/groups/4773535/items/BWDYDL8T)) ([pdf](zotero://open-pdf/groups/4773535/items/8U5XVU5K?page=4&annotation=P963PKKK))
“Data inaccuracy, access limitations, a lack of thresholds and inadequate resources for data collection resulted in non-standardised information, preventing comparison across areas and weakening strategic decision-making” ([Gualazzini, 2021, p. 4](zotero://select/groups/4773535/items/BWDYDL8T)) ([pdf](zotero://open-pdf/groups/4773535/items/8U5XVU5K?page=4&annotation=L5SELDEJ))
“Even when receiving timely information on the ground, Members often had to wait for validation from national sources such as the Food Security and Nutrition Analysis Unit (FSNAU) to make decisions, and the resources for anticipatory action were very limited.” ([Gualazzini, 2021, p. 4](zotero://select/groups/4773535/items/BWDYDL8T)) ([pdf](zotero://open-pdf/groups/4773535/items/8U5XVU5K?page=4&annotation=CZST8S2L))
% now its better (?) but still rather slow compared to MCS

% takes long (25 days) is super resource intensive (key informant interviews -> 3 communities, 4 people each, lots of work.) + trust + triangulation + barely any automatization and rather coarse -> no detailed risk and vulnerability assessments (though this point may not be too valid.. -> the people know the area)
“Real-Time Risk Monitoring (RTRM)” ([Gualazzini, 2021, p. 4](zotero://select/groups/4773535/items/BWDYDL8T)) ([pdf](zotero://open-pdf/groups/4773535/items/8U5XVU5K?page=4&annotation=HZFMU84X))} % not too sure if I wanna include this or where.. maybe only in the discussion part? -> in comparison to Crowdsensing/MCS this is slow and relatively coarse. -> OCHA 2022 learning as well

% Community-led early warning and anticipatory action in Somalia
https://www.sparc-knowledge.org/news-features/features/community-led-early-warning-and-anticipatory-action-somalia
The \acrshort*{brcis} network 



% reasoning --> leading to research aim



more and more severe droughts in Somalia

but Forecasts primarily global scale
thus far, AAs are often limited by data availability
reparation and water trucking as mayor actions/response -> detailed information necessary

“The handbook of drought indicators (Svoboda et al. 2016) lists more than 50 drought indices. Not a single one of these indices connects climate anomalies to socioeconomic vulnerabilities,” ([Enenkel et al., 2020, p. 1163](zotero://select/groups/4773535/items/RX575C79)) ([pdf](zotero://open-pdf/groups/4773535/items/XD499UNK?page=3&annotation=EDKZFHJX))
--> there were more indicators in the following year, but those were limited due to availibiltiy of socioeconomic data

+ embeddedness of citizen projects -> more benefits (see goals)
https://www.unwater.org/our-work/integrated-monitoring-initiative-sdg-6
"Stronger accountability: Data can communicate that work is being done and progress is happening. Data can enable greater transparency, which reduces inefficiency and corruption.
Attracting commitment and investments: Data can quantify problems and make it easier to communicate needs for political commitment and public and private investments.
Evidence-based decision-making: Data can inform policy- and decision-makers of where to focus efforts and which solutions are most effective, to ensure the greatest possible gains with existing resources.
Leaving no one behind: Disaggregated data can help identify specific groups or areas with unmet needs and higher levels of risk, to which interventions can be targeted."
https://www.unwater.org/our-work/integrated-monitoring-initiative-sdg-6/background

“Experts in trigger methodology have indicated a more appropriate strategy may be to build on tools that currently exist at the government level such as national drought monitoring systems. As such, the ideal is an iterative process with the ground level along with a technology push that creates new ways to analyse drought and drought risk.” ([RCRC, 2020, p. 28](zotero://select/groups/4773535/items/UESIQTRJ)) ([pdf](zotero://open-pdf/groups/4773535/items/P5JPVZ97?page=28&annotation=977VS8FC))

% DRM Strategic Plan General and Specific Objectives
“2.1.1 Specific Objective 1 Vulnerable communities’ resilience at SRCS target areas strengthened through anticipatory actions, response, recovery, and disaster risk reduction, and they can anticipate and effectively respond to and recover from disasters and crisis by 2026.” ([“SRCS DRM Strategic Plan_final 9thNovember 2021-converted.pdf”, p. 15](zotero://select/groups/4773535/items/LFCBRZLD)) ([pdf](zotero://open-pdf/groups/4773535/items/6IL6K72G?page=15&annotation=ZFICKZRA))

% --> even the RCRC is still looking for good triggers -> maybe water levels are a good way -> reasoning for this study (see background identical text)



The \textit{Drought Monitoring Tool for Somalia} FAOSWALIM is currently also looking for alternative indicators for their Combined Drought Index (CDI) since the current proxy for soil moisture, NDVI does not correlate well with ground information anymore.  https://cdi.faoswalim.org/index/cdi

“For external experts, they may also benefit from community support to inform scientific processes, such as collecting data that spans across a large geographic region and having an enhanced understanding of community interests.” ([Huang et al., 2020, p. 144](zotero://select/groups/4773535/items/9CSBLJNJ)) ([pdf](zotero://open-pdf/groups/4773535/items/G5BEZQ7C?page=9&annotation=QPIKDKB9))
“Management of Drinking Water Source in Rural Communities under Climate Change” ([Huang et al., 2020, p. 136](zotero://select/groups/4773535/items/9CSBLJNJ)) ([pdf](zotero://open-pdf/groups/4773535/items/G5BEZQ7C?page=1&annotation=92FTSBXQ))
“For local communities, their needs of safe drinking water could be met and their abilities to manage and maintain water supply could be enhanced.” ([Huang et al., 2020, p. 144](zotero://select/groups/4773535/items/9CSBLJNJ)) ([pdf](zotero://open-pdf/groups/4773535/items/G5BEZQ7C?page=9&annotation=AL2NB5DK))
“Furthermore, this model could help increase scientific awareness among community members and engage the community with the environment.” ([Huang et al., 2020, p. 144](zotero://select/groups/4773535/items/9CSBLJNJ)) ([pdf](zotero://open-pdf/groups/4773535/items/G5BEZQ7C?page=9&annotation=FYU4BIKP))
“During the management of rural drinking water sources, a hybrid modality in which community management is the mainstay with supplement from external support from other organizations is highly recommended.” ([Huang et al., 2020, p. 147](zotero://select/groups/4773535/items/9CSBLJNJ)) ([pdf](zotero://open-pdf/groups/4773535/items/G5BEZQ7C?page=12&annotation=SQ6P8UBN)) % may also be a good transition to CBWM
% --> water related stuff --> subject is generally well researched and tried out in other circumstances and contexts

“Connecting top-down weather and climate data with bottom-up socioeconomic data via machine learning” ([Enenkel et al., 2020, p. 1166](zotero://select/groups/4773535/items/RX575C79)) ([pdf](zotero://open-pdf/groups/4773535/items/XD499UNK?page=6&annotation=BCGJNKZB))

% thus nice in theory, but not useful in practice
% in regard to drought and water scarcity
% and while there is an extensive body of literature about these topics, the minor details are not of great interest to this work but the general conclusion, that physical, large scale drought or water scarcity indicators do not capture the required level of detail and impact that is needed to operationally act upon. Also, while the complexity of these concepts is due to the level of complexity of the surveyed phenomenon, its application and comparison is hindered. Thus a method to assess local impact, that builds and incorporates these concepts in a practically applicable manner is needed to adequately address this detrimental topic. 

“A major weakness of the existing tools is the emphasis on macro/international level information.” ([Masinde and Bagula, 2010, p. 390](zotero://select/groups/4773535/items/JNC4ACZS)) ([pdf](zotero://open-pdf/groups/4773535/items/IWMKDQYV?page=1&annotation=7Z44Z9L7))

“The methods by which the Joint Monitoring Programme (JMP) of WHO and UNICEF” ([Bartram et al., 2014, p. 8137](zotero://select/groups/4773535/items/6AWUJTW5)) ([pdf](zotero://open-pdf/groups/4773535/items/BFNSQGWS?page=1&annotation=UL4Q2I4V))
“substantial limitations: current methods do not address water quality, equity of access, or extra-household services.” ([Bartram et al., 2014, p. 8137](zotero://select/groups/4773535/items/6AWUJTW5)) ([pdf](zotero://open-pdf/groups/4773535/items/BFNSQGWS?page=1&annotation=TIPCEXGG))

current challenges for utilisation of forecasting systems: scarse coverage of weather stations and poor utilisation by the farmers often due to bad dissemination channels  (too coarse, too unreliable)

“Community cultures, economies, and environments differ across countries and regions. These differences should be considered when designing hybrid management strategies, so that all actors can be appropriately enabled and the mechanism which is most effective for the given community can be identified.” ([Huang et al., 2020, p. 147](zotero://select/groups/4773535/items/9CSBLJNJ)) ([pdf](zotero://open-pdf/groups/4773535/items/G5BEZQ7C?page=12&annotation=WV5DXV5I))

“On this basis, it is essential to expand research area to study the various threats from climate variability to rural drinking water safety, and then to develop corresponding measures to address those threats to water security.” ([Huang et al., 2020, p. 147](zotero://select/groups/4773535/items/9CSBLJNJ)) ([pdf](zotero://open-pdf/groups/4773535/items/G5BEZQ7C?page=12&annotation=HCQHR7YR))

%%%%%%%%%%%%%%%%%%%%%%%% 
%local on the ground impact assessment not possible with current forecast abilities --> possibly link to that via CBS and the feasibility study --> 
(“Early Warning/Early Action Mechanisms: EWEA is working well in cases of health emergencies/epidemics through community-based surveillance (CBS); this allows the N” ([Somali Red Crescent Society, 2022, p. 51](zotero://select/groups/4773535/items/FZ6BJHJA)) ([pdf](zotero://open-pdf/groups/4773535/items/RJKNZZZ2?page=51&annotation=4C3HL8ES))) 
thus, this and comparable approaches are investigated in the next chapter

but: 
“Scale is critical in assessing water security [31]. National level assessments make it difficult to take action at operationalization level.” ([Mishra et al., 2021, p. 8](zotero://select/groups/4773535/items/MD2Z2HTF)) ([pdf](zotero://open-pdf/groups/4773535/items/366Z36U7?page=8&annotation=6IAXCXUB))

and: “Creating and using indicators for water security has to be directed towards some management control or assessment action.” ([Mishra et al., 2021, p. 8](zotero://select/groups/4773535/items/MD2Z2HTF)) ([pdf](zotero://open-pdf/groups/4773535/items/366Z36U7?page=8&annotation=P72LT9Y8))

“There has been little effort to align the spatiotemporal granularity of socioeconomic assessments with the granularity of weather or climate monitoring.” ([Enenkel et al., 2020, p. 1161](zotero://select/groups/4773535/items/RX575C79)) ([pdf](zotero://open-pdf/groups/4773535/items/XD499UNK?page=1&annotation=QBTLFCXM))

“but questions related to coping capacities, migration, poverty, water supply, access to food and markets, or political conflict remain unanswered or are even decoupled from routine drought risk assessments” ([Enenkel et al., 2020, p. 1162](zotero://select/groups/4773535/items/RX575C79)) ([pdf](zotero://open-pdf/groups/4773535/items/XD499UNK?page=2&annotation=HE48ZWFA))

“Our results indicate that using the phones to transmit more than just water quality data will likely improve the effectiveness and sustainability of this type of intervention.” ([Kumpel et al., 2015, p. 10846](zotero://select/groups/4773535/items/GPM4C7RJ)) ([pdf](zotero://open-pdf/groups/4773535/items/7VXVKEXK?page=1&annotation=4DJIADX2))


--> problem: conclusion of existing sources, tools and forecasts - only macro/international level? or are there meso/micro forecasts available?
better understanding the forecasting and its implications on the ground are crucial. --> local information. tons of Volunteers but even more water sources. Continue with Crowdsensing? Implications?

request of the SRCS -> practically wanted
understanding the full scope and knowing which water sources are at what level and quality can help with management decisions and trigger certain events very locally
current challenges (?) what do I want to address? (number 1,2,3)
outline of the thesis/project




“Thinking outside the box in terms of both hydro-meteorological and socio-economic indicators could be particularly useful” ([RCRC, 2020, p. 31](zotero://select/groups/4773535/items/UESIQTRJ)) ([pdf](zotero://open-pdf/groups/4773535/items/P5JPVZ97?page=31&annotation=GNZJ3FR5))

% research aim - high level terms
The aim of this study is to adapt and apply an approach for community-based participatory mapping and monitoring of water sources in a water-scarce and resource-limited setting in collaboration with a national non-governmental organization to facilitate respective \acrlongpl{aa} in the context of \acrlong{fbf}, with the goal of improving water management and availability to address water shortages.

% fuuuuuuuuuuse
The aim of this study is therefore to incorporate local knowledge on the availability of water sources into the monitoring of drought impacts in an effort to support triggering and Anticipatory Actions under the Early Action Protocol. For this purpose, already proven methods are combined. First, semi-structured (expert-) interviews will be conducted to generate in-depth local knowledge. Based on these findings on local (pre-)conditions, needs and limitations, a monitoring tool based on a volunteersensing approach will then be conceptualized and subsequently discussed with local decision-makers. A prototype development based on the Social.Water tool is a further possibility. In addition, the question of the equality of local to scientific knowledge will be raised and further influences of such a contribution on social conditions will be investigated \todo{too much?}.


--> mapping -> refers to the entire initial phase - not only the geographical location but to the gathering of all key features of a water source. 

%----------------------------------------------------------------------------------------
%	SECTION 4 Questions and objectives
%----------------------------------------------------------------------------------------
% Perhaps the most important part of your introduction is your questions and objectives, as it sets up the expectations for the rest of your thesis or dissertation. How you formulate your research questions and research objectives will depend on your discipline, topic, and focus, but you should always clearly state the central aim of your research.

% If your research aims to test hypotheses, you can formulate them here. Your introduction is also a good place for a conceptual framework that suggests relationships between variables.

“Firstly, there is a need for coordinated institutional responses in addressing matters related to water scarcity.” ([Leal Filho et al., 2022, p. 11](zotero://select/groups/4773535/items/CBFHWKLP)) ([pdf](zotero://open-pdf/groups/4773535/items/CG4XVTAI?page=11&annotation=PVHEGLHC))

“Secondly, it is important to correlate water availability – including groundwater – with an efficient water use to maximise and safeguard sustainable access for all users.” ([Leal Filho et al., 2022, p. 11](zotero://select/groups/4773535/items/CBFHWKLP)) ([pdf](zotero://open-pdf/groups/4773535/items/CG4XVTAI?page=11&annotation=RVPH3I3J))

“One of main challenges for adaptation for the coming decade is to extend planned adaptation at the local level and better integrate projected risk of climate change and variability into local autonomous responses.” ([Leal Filho et al., 2022, p. 11](zotero://select/groups/4773535/items/CBFHWKLP)) ([pdf](zotero://open-pdf/groups/4773535/items/CG4XVTAI?page=11&annotation=K4DUGJBX))



overcome limitations / incorporating recommendations: e.g. increased support/engagement of poeple who actually use the reports (e.g. SRCS officials) 
fill geographic and information gap ->

its about drought forecasting and early trigger but at the same time highly local and practical information where and which water sources are good and functioning and which are not. -> highly practical information. Some data exist but (mostly) outdated.
about getting local knowledge from SRCS Volunteers and their community as well as returning information about the bigger picture
in order to enhance the quality of data for managing severe droughts in Somaliland. (one short paragraph -> motivation)
provide number of weather stations in the area


This work is thus based on the question \textit{"In what way can a collaborative Volunteer Sensing Approach be conceptualized to facilitate community water source monitoring in Somaliland in order to enhance early drought triggers and following Anticipatory Actions?"} and explores following hypotheses:
1. Local knowledge can enhance early drought impact triggers in the context of Anticipatory Actions in Somaliland.
2. Water source availability, accessibility and quality are feasible ecological and social indicators for monitoring early drought impacts and guide Anticipatory Actions in Somaliland.
3. The combination of qualitative interviews and Volunteered Crowdsensing is a promising approach to qualitatively and quantitatively monitor early drought impacts on communities in Somaliland.



% research objectives
1. To conduct a comprehensive review of existing literature and guidelines related to the design and implementation of \acrlong{cs} programmes, and identify relevant work in regard to the research aim and overall case study context. (of water scarcity and FbF)

2. To assess the feasibility of the \acrlong{cs} approach in the given context by identifying potential challenges and opportunities for successful implementation, and to propose recommendations for addressing these challenges.

3. if feasible, develop a replicable and adaptable framework for community-based participatory water source mapping and monitoring in the context of \acrlong{fbf}, based on the principles and recommendations identified in objectives 1 and 2.

4. To apply the adapted and developed frameworks in order to create a roadmap for the implementation of the proposed project, including specific products, actions, and stakeholders involved.

% research question
1. What specific guidelines and best practices exist for the design and implementation of community-based participatory water source mapping and monitoring programmes in resource-limited and water-scarce settings, and how can they be applied to the thematic direction of \acrlong{fbf}? % none -> wider angle -> FbF

2. Based on the identified frameworks and principles, how can a \acrlong{fbf} oriented, replicable and adaptable framework be developed for community-based participatory water source mapping and monitoring in the context of water-scarce and resource-limited settings? 

3. In the specific context of this case study, how can the developed framework be applied to create a tailored roadmap for the implementation of a community-based participatory water source mapping and monitoring project, including specific products, activities, and stakeholders involved in the project?








%----------------------------------------------------------------------------------------
%	SECTION 5 Overview of the structure
%----------------------------------------------------------------------------------------
1-2 sentences outline, brief summery of each chapter
% To help guide your reader, end your introduction with an outline of the structure of the thesis or dissertation to follow. Share a brief summary of each chapter, clearly showing how each contributes to your central aims. However, be careful to keep this overview concise: 1-2 sentences should be enough.

transition: the way forward...
more or less standard shizzle

% somewhat in the result chapter (-> the how? -> it currently can't)
possibly merge results with discussion: based on this: https://libguides.usc.edu/writingguide/assignments/casestudy and these studies/guidelines/Principles



Case Studies. Writing@CSU. Colorado State University; Gerring, John. Case Study Research: Principles and Practices. New York: Cambridge University Press, 2007; Merriam, Sharan B. Qualitative Research and Case Study Applications in Education. Rev. ed. San Francisco, CA: Jossey-Bass, 1998; Miller, Lisa L. “The Use of Case Studies in Law and Social Science Research.” Annual Review of Law and Social Science 14 (2018): TBD; Mills, Albert J., Gabrielle Durepos, and Eiden Wiebe, editors. Encyclopedia of Case Study Research. Thousand Oaks, CA: SAGE Publications, 2010; Putney, LeAnn Grogan. "Case Study." In Encyclopedia of Research Design, Neil J. Salkind, editor. (Thousand Oaks, CA: SAGE Publications, 2010), pp. 116-120; Simons, Helen. Case Study Research in Practice. London: SAGE Publications, 2009; Kratochwill, Thomas R. and Joel R. Levin, editors. Single-Case Research Design and Analysis: New Development for Psychology and Education. Hilldsale, NJ: Lawrence Erlbaum Associates, 1992; Swanborn, Peter G. Case Study Research: What, Why and How? London : SAGE, 2010; Yin, Robert K. Case Study Research: Design and Methods. 6th edition. Los Angeles, CA, SAGE Publications, 2014; Walo, Maree, Adrian Bull, and Helen Breen. “Achieving Economic Benefits at Local Events: A Case Study of a Local Sports Event.” Festival Management and Event Tourism 4 (1996): 95-106.


%----------------------------------------------------------------------------------------
%	Notes:
%----------------------------------------------------------------------------------------


Checklist: Introduction 0 / 7
I have introduced my research topic in an engaging way.

I have provided necessary context to help the reader understand my topic.

I have clearly specified the focus of my research.

I have shown the relevance and importance of the dissertation topic.

I have clearly stated the problem or question that my research addresses.

I have outlined the specific objectives of the research.

I have provided an overview of the dissertation’s structure.
% % Chapter Template

\chapter{Theoretical Background} % Main chapter title

\label{chapter2} % \ref{Chapter2}
%----------------------------------------------------------------------------------------
%	SECTION 1
%----------------------------------------------------------------------------------------

\section{Introduction}

This chapter provides an overview of the theoretical background of this thesis, including key concepts, theories and literature in which this thesis is embedded. The chapter starts with a discussion of broad concepts such as \textit{water security}, \textit{water scarcity}, and \textit{drought}, along with their characteristics and differences in section \ref{sec:fundamental_concepts}. Building on this foundation, the section \ref{sec:fbf} introduces the approach to measure and monitor these wide concepts through indicators and indices together with the ideas of risk, vulnerability, and impact. Extending the previously prevailing idea of the \acrfull{drr} cycle of mitigation, preparation, response and recovery, the rather recently emerged concept and operationalisation of \textit{\acrfull{fbf}} is described in detail. The details cover aspects of structure, forecast based decision-making, setting strict thresholds for when to act, and finally what to do when thresholds are exceeded.\newline
Drawing on the realisation that the current data basis for predictions is too coarse for precise measures, another broad field \textit{\acrfull{cs}} is introduced in section \ref{sec:cs}. Following this, sub-concepts of \acrfull{cbm} and \acrfull{mcs} are further introduced and with \acrfull{cbs} and \acrfull{cbwm} concise examples for successful implementation in local context and thematic transferability of the approach are given, respectively.\newline
Section \ref{sec:case_area} anchors the concepts mentioned so far in the local context and addresses local specifics. The geographical and climatic conditions, the historical and current economic and socio-cultural context, and ongoing implementation efforts for anticipatory measures further describe the case study area. The chapter concludes with a summary of the most important points and establishes a link between the findings and the further thesis.

%----------------------------------------------------------------------------------------
%	SECTION 2 Water security, drought & water scarcity/quality/access
%----------------------------------------------------------------------------------------
\gls*{cbm}

\section{Fundamental concepts - Water Security, Water Scarcity and Drought}\label{sec:fundamental_concepts}
% human induced water shortage component
% what about water security? “Water Security: A Complex Concept” ([Butte et al., 2022, p. 1](zotero://select/groups/4773535/items/QB97YZ2M)) Q936I2JN
% and insecurity? “Progress in household water insecurity metrics: a crossdisciplinary approach” ([Jepson et al., 2017, p. 1] HWX5JRS4

Water security is a theoretical construct that has emerged in the 21st century to frame the overall water objectives and goals to guide local to global water management and policy development \autocite{sadoffWaterSecurity2020a}. It "links together the web of food, energy, climate, economic growth, and human security challenges that the world economy faces over the next two decades" \autocite[5]{wefBubbleCloseBursting2009}. In more detail, it is about "the availability of an acceptable quantity and quality of water for health, livelihoods, ecosystems and production, coupled with an acceptable level of water-related risks to people, environments and economies."\autocite{greySinkSwimWater2007}.
Water security integrates therefore economic, social and environmental dimensions into an interconnected and complex network of human and natural relations by addressing risks of too much, too little or poor quality water \autocite{vanbeekWaterSecurityPutting2014, mishraWaterSecurityChanging2021}. Due to the focus of this work, emphasis is placed on factors that decrease water security due to too little water availability. Besides other factors, natural disasters such as droughts, and water scarcity are the main drivers for insufficient quantities of water \autocite{caretta2022water}. Water quality and access are briefly addressed in addition to provide a more comprehensive understanding of water security for the following chapters.

% TODO: maybe switch the order here.. first water scarcity, then drought. -> would make more sense and it's only a small paragraph --> but takes some effort in the second paragraph of water scarcity.. maybe later on. Not that important

% nice source for water security: “A Framework for Water Security Data Gathering Strategies” ([Butte et al., 2022, p. 1](zotero://select/groups/4773535/items/QB97YZ2M)) ([pdf](zotero://open-pdf/groups/4773535/items/Q936I2JN?page=1&annotation=5G7DUDDP))

%-----------------------------------
%	SUBSECTION 2.1
%-----------------------------------

\subsection{About Drought}\label{subsec:about_drought}

Drought as highly complex and severe climate-related multi-hazard has far reaching, cascading and interconnected consequences affecting natural ecosystems, societies and economies (see figure \ref*{TODO:})\autocite{vereintenationenSpecialReportDrought2021}. Historically, droughts are a recurring feature that can occur in all climates. They can geographically extend over small areas to entire sub-continents and are slow onset events that can persist for a few weeks to several years. These high spatial and temporal variabilities make drought not only challenging to define but due to its slow onset, droughts are often only recognized when they are well advanced \autocite{idmpDroughtWaterScarcity2022,vereintenationenSpecialReportDrought2021}. While some drought conditions over large areas can be associated to some low-frequency changes in atmospheric conditions such as the El Niño, accurate cause identification can be rather challenging on smaller scales and requires many different parameters \autocite{botaiAnalysisDroughtProgression2019, vereintenationenSpecialReportDrought2021}.


% noice figurrrrreee
%“Figure 1.6. Schematic representation of potential interconnections among different sectors affected by droughts” ([“Special report on drought 2021”, 2021, p. 47](zotero://select/groups/4773535/items/RAAM9PVS)) ([pdf](zotero://open-pdf/groups/4773535/items/7AK5QVBL?page=49&annotation=TNA9J8ZS))

\missingfigure{sick figure y<o}

%In order to approach this complexity, drought is most often defined from four different perspectives, focussing on different manifestations and stages. These definitions are outlined in the coming sub-chapter \ref*{subsec:drought_definitions}, followed by a section addressing the necessary indicators  currently employed in practice for these definitions.
%Generally, droughts are commonly characterized by deviations or the complete failure of climate and weather systems that drive the hydrological cycle compared to normal conditions\autocite{botaiAnalysisDroughtProgression2019,idmpDroughtWaterScarcity2022,vanloonDroughtHumanmodifiedWorld2016,vereintenationenSpecialReportDrought2021}. A more in depth definition can be found in the sub-chapter \ref*{subsec:drought_definitions}.
This complex conglomeration of interrelated causes and effects of multiple temporal, spatial and thematic dimensions makes the definition of \textit{drought} a fairly multi-layered undertaking \autocite{balintMonitoringDroughtCombined2013}. Several well-known definitions are, for example, from the \autocite{theamericanheritagedictionaryoftheenglishlanguageDrought2022} defining drought as "a long period of abnormally low rainfall, especially one that adversely affects growing or living conditions". \autocite[2]{palmerMeteorologicalDrought1965} defines drought as "a prolonged and abnormal moisture deficiency." or \autocite{vanloonDroughtHumanmodifiedWorld2016} defines droughts simply as "an exceptional lack of water compared to normal conditions". Other drought definitions emphasize its natural and/or human origin, its special characteristics, impact and temporal duration or even understand "drought as a system of causality where the link between causes and effects is random in nature \autocite{balintMonitoringDroughtCombined2013, baltiReviewDroughtMonitoring2020, idmpDroughtWaterScarcity2022,loonDroughtAnthropocene2016, wangPropagationDroughtMeteorological2016, wilhiteUnderstandingDroughtPhenomenon1985}. Already in the 1980s, \autocite{wilhiteUnderstandingDroughtPhenomenon1985} found more than 150 published definitions of drought. Besides the categorization into a conceptual or operational category , \autocite{wilhiteUnderstandingDroughtPhenomenon1985} proposed a clustering of these definitions into four types, namely meteorological drought, agricultural drought, hydrological drought and socio-economic drought. This classification is still widespread today \autocite{balintMonitoringDroughtCombined2013, baltiReviewDroughtMonitoring2020,idmpDroughtWaterScarcity2022,vereintenationenSpecialReportDrought2021}.

The conceptual category refers to a general formulation of an idea of drought to understand its concept and identify its boundaries and is often formulated in relative terms \autocite{wilhiteUnderstandingDroughtPhenomenon1985}. Definitions in the operational category try to define how drought functions in terms of its onset, duration, severity and spatial coverage also covering how this can be measured via indices \autocite{balintMonitoringDroughtCombined2013, nationaldroughtmitigationcenterWhatDrought, wilhiteUnderstandingDroughtPhenomenon1985}. With these definitions, the current situation is usually compared to a historical average, which is usually based on a 30-year period,  presupposing the development and continuous measurement of indicators and indices that can be used. \autocite{vereintenationenSpecialReportDrought2021,wilhiteUnderstandingDroughtPhenomenon1985}.

The four types of drought are commonly conceptually defined and brought into practice by operational specifications. They can be understood as different, but complementary stages of the same process and are generally cascading in reason and time but can overlap and are difficult to completely unravel. Table \todo{TODO: see RCRC 2020 p.11} displays the four types at a glance and figure \todo{TODO:, see https://drought.unl.edu/Education/DroughtIn-depth/TypesofDrought.aspx} shows an overview about the different types, their succession and cascading elements.

\missingfigure{table RCRC 2020 p.11}

The \textit{meteorological drought} is usually characterized by the duration and the degree of dryness in comparison to the normal average amount and tries to conceptually understand how weather patterns can impact water availability. Definitions in this category are specific for a regions atmospheric conditions. That is to say that regions with a year-round precipitations regime such as tropical rainforest need different definitions and thresholds than e.g. climates characterized by seasonal rainfall patterns \autocite{nationaldroughtmitigationcenterTypesDrought}. Operational classification mostly uses rainfall, moisture, temperature and wind indicators to determine the onset, severity and duration of drought.

\textit{Agricultural drought} definitions establish a connection between different features of meteorological drought with their impacts on agriculture. Soil-moisture, differences between actual and potential evapotranspiration and soil water deficits are some of the operationalized indicators for monitoring this type of drought \autocite{baltiReviewDroughtMonitoring2020,nationaldroughtmitigationcenterTypesDrought,wilhiteUnderstandingDroughtPhenomenon1985}.

The type of \textit{hydrological drought} is associated with the impact of meteorological drought on surface or subsurface water resources such as rivers, lakes, and groundwater. Hydrological drought occurs when these indicators drop below normal levels \autocite{palmerMeteorologicalDrought1965}. The fastest responding indicator of this type of drought is most often the variability of streamflow. The water levels of lakes and groundwater usually lag behind the occurrence of the meteorological or agricultural drought which is why the hydrological drought is often out of phase with the previously mentioned types. The hydrological drought is commonly defined on the basis of watershed or river basin scale \autocite{baltiReviewDroughtMonitoring2020,nationaldroughtmitigationcenterTypesDrought,wilhiteUnderstandingDroughtPhenomenon1985}.

The \textit{socioeconomic drought} differs from the aforementioned types as it can also incorporate features of these types of drought to associate them with the demand and supply of some social or economic good. It therefore relates the impact of all other types of droughts on human population and its various sectors of society such as food security, health, and the economy. It is therefore sometimes also interchangeably used with drought impacts. Operational categorization involves using socioeconomic indicators such as unemployment rates and food prices to assess the severity and duration of the drought \autocite{nationaldroughtmitigationcenterTypesDrought,wilhiteUnderstandingDroughtPhenomenon1985}.

\missingfigure{https://drought.unl.edu/Education/DroughtIn-depth/TypesofDrought.aspx}

The shown economic, social and environmental impacts of drought in figure \todo{TODO:} depend on the severity of, and the risk to drought. These three concepts of impact, severity and risk are interrelated concepts used to assess and understand the effects of drought on various sectors. Thereby, in alignment with the definition of \autocite{vanloonDroughtHumanmodifiedWorld2016} it is the exceptional severity of the water shortage that distinguishes drought from aridity, an ordinarily recurrent or fully dry climate, and from water scarcity as a long-term "supply/demand and natural and/or human-made phenomenon" \autocites[7]{idmpDroughtWaterScarcity2022}{vereintenationenSpecialReportDrought2021, vanClimatologicalRiskDroughts2017}. Water scarcity is described in more detail in the following chapter.

%----------------------------------------------------------------------------------------
%	SUBSECTION 2.2 Water Scarcity
%----------------------------------------------------------------------------------------

\subsection{Water Scarcity}\label{subsec:water_scarcity}

Water scarcity, as for water security or drought, is defined in many different ways. The sixth IPCC Assessment Report defines water scarcity broadly as "a mismatch between the demand for fresh water and its availability, quantified in physical terms" \autocite[560]{caretta2022water}. Here, social and economic components are outsourced to the broader concept of water security and insecurity, focussing primarily on physical water scarcity \autocite{caretta2022water}. In contrast, the \acrfull{fao} defines water scarcity as "a gap between available supply and expressed demand of freshwater in a specified domain, under prevailing institutional arrangements (including both resource ‘pricing’ and retail charging arrangements) and infrastructural conditions" \autocite[5]{faoCopingWaterScarcity2012} further summarizing that water security is "an excess of water demand over available supply" \autocite[6]{faoCopingWaterScarcity2012}. Thus, highlighting the human dimension of this interactive and relative concept of physical and economic water scarcity. Hereby, physical water scarcity refers to a situation in which there is not enough water available in quantitative terms to meet demand whereas economic water scarcity occurs when inadequate infrastructure, institutional or financial capital obstructs access to water resources "even though water in nature is available to meet human demands" \autocites{idmpDroughtWaterScarcity2022}[11]{moldenWaterFoodWater2007}.
Water scarcity and drought are in a complex interrelationship with each other. A short overview about the key differences between water scarcity and drought are given in table \todo{TODO:“Table 1. Characteristics and impacts of water scarcity and drought” ([IDMP, 2022, p. 3](zotero://select/groups/4773535/items/LNSL8VD2)) ([pdf](zotero://open-pdf/groups/4773535/items/JM82W3ZF?page=9&annotation=QNC4A3FG))}. 

\missingfigure{differences water scarcity to drought}
% Distinctions between water scarcity and drought:

% also in Joplin
%“Table 1. Characteristics and impacts of water scarcity and drought Water scarcity Drought Length Long-term to permanent Temporary (weeks to multiyear) Driving forces Demand–supply imbalance, human-driven, and/or natural (overexploitation, pollution). Climate change can impact both supply and demand Natural climate variability which can be modified/amplified by climate change Potential impacts Restricted water availability, environmental degradation, desertification, exacerbated inequalities in access to water resources, potential competition Water shortages, competition, environmental degradation Measures Long-term IWRM to bring supply and demand back into sustainable balance Integrated drought management, including: (1) monitoring and early warning; (2quality) vulnerability and impact assessment; and (3) risk mitigation, preparedness and response Source: adapted from Hohenwallner et al. (2011) DROUGHT AND WATER SCARCITY – DEFINITIONS AND CHARACTERISTICS” ([pdf](zotero://open-pdf/groups/4773535/items/JM82W3ZF?page=9&annotation=E3EQRILA))

Furthermore, potential mutual reinforcements, climate change, increased water use and poor water management can make it sometimes difficult to clearly separate these concepts \autocite{idmpDroughtWaterScarcity2022,lealfilhoUnderstandingResponsesClimaterelated2022,liuWaterScarcityAssessments2017,rcrcFORECASTBASEDFINANCINGEARLY2020}. Nonetheless, following the definition of \autocite{faoCopingWaterScarcity2012} the concept of water scarcity always gives water shortage, understood as absolute lack of water in the current situation, a human dimension in particular on the demand side. Here, the quality of policies, planning and management is considered as critical to the overall severity of the impact of water scarcity \autocite{idmpDroughtWaterScarcity2022,faoCopingWaterScarcity2012,vereintenationenSpecialReportDrought2021}. The supply side can be influenced by human activities, but it is not a mandatory prerequisite. \autocite{idmpDroughtWaterScarcity2022}. 
Besides the already mentioned water scarcity on the basis of physical quantity and economical factors, water scarcity can also be caused by water of unacceptable quality and lack of access to water services \autocite{faoCopingWaterScarcity2012}. The recognition that insufficient water quality is an additional contributing factor to water scarcity is a relatively recent development in the literature \autocite{liuThreedimensionalWaterScarcity2020} but together with inadequate access highlights further challenges in ensuring water security \autocite{caretta2022water, mishraWaterSecurityChanging2021}. 

%-----------------------------------
%	SUBSECTION 2.3 Water Access & Water Quality
%-----------------------------------

\subsection{Water Quality \& Access}\label{subsec:water_quality}
% + evtl. human health related water borne diseases and CBS

As could be seen in the previous chapter, besides the quantitative availability of water, its accessibility and quality are crucial. Inadequate water quality can be related to numerous health and environmental issues and can further limit the availability of water for given uses \autocite{rcrcFORECASTBASEDFINANCINGEARLY2020, faoCopingWaterScarcity2012}. Unlike the previous concepts, water quality has mostly fixed indicators by which the condition can be determined but historically, and still today, water quality assessment is primarily carried out in laboratories with preceding water sampling activities. This procedure not only makes the determination of water quality a laborious and costly process, but also places high demands on equipment and personnel, so that it is not viable for large-scale rural assessments in low-income areas. \autocite{tariqOpenSourceWater2021,worldmeteorologicalorganizationPlanningWaterqualityMonitoring2013}. While simpler methods for in situ water quality monitoring exist, they are either insufficient or often still need too much investment and knowledge to conduct for widespread and frequent monitoring \autocite{worldmeteorologicalorganizationPlanningWaterqualityMonitoring2013}. Nonetheless, new solutions are being developed to simplify and scale affordable water quality assessments to rural areas \autocite{ighaloComprehensiveReviewWater2020,tariqOpenSourceWater2021}. While the direct assessment of water quality might be challenging, poor water quality can be linked to other factors. Environmental awareness, poor sanitation and hygiene conditions of people in rural areas were for example considered as major causes for contamination of water at the source \autocite{zamxakaMicrobiologicalPhysicochemicalAssessment2004}.

The definition of water access is again a rather challenging undertaking. The \autocite[254]{worldbankWorldDevelopmentReport1997} defined water access in rural areas by "access implies members of the household do not have to spend a disproportionate part of the day fetching water." While both time and distance still play a crucial role in literature when investigating water access \autocite{cassiviDrinkingWaterAccessibility2019,cassiviEvaluatingSelfreportedMeasures2021,emenikeAccessingSafeDrinking2017}, the term also gained a social component \autocite{emenikeAccessingSafeDrinking2017,mitlinUnaffordableUndrinkable}. \autocite{obeng-odoomAccessWater2012} adds four additional factors namely, affordability, quality, equitable distribution to the definition of water access to fully understand if users have access to water in daily live. \autocite{unitednations/developmentprogrammeDeepeningDemocracyFragmented2002} links these parameters to the access to an improved water source which should provide safe drinking water.
The access to improved water sources is therefore generally considered as crucial in the reaching of water security \autocite{cdcAssessingAccessWater2022}. Proactive measures to drought and water scarcity can not only potentially minimize or even neutralize impacts and are considerably more cost-efficient, early warning and anticipatory actions for drought and water scarcity impacts become ever more important \autocite{faoandun-waterProgressLevelWater2021,idmpDroughtWaterScarcity2022,worldbankHighDryClimate2016}.

%----------------------------------------------------------------------------------------
%	SECTION 4 FbF, EAP, AA & Early Warning
%----------------------------------------------------------------------------------------

\subsection{Measurement and estimation approaches for impacts}\label{subsec:indicators} % or Measuring and estimating impacts

Indicators and Indices are often used to translate complex matters into easier to explain numbers and scales that can be measured, tracked and reasonably compared \autocite{blauveltSystematizingEnvironmentalIndicators2014,williamsUsingIndicatorsExplain2017}. This can range from capturing simple measurements to complex and detailed issues that can not only depict ecological conditions but its interactions with societies \autocite{blauveltSystematizingEnvironmentalIndicators2014,mishraWaterSecurityChanging2021}. Indicators and Indices can thus establish a clear and common understanding of a concept or parts of it in a quantifiable and more objective way.
Here, an indicator is understood as a measurable parameter that provides information on the state or trend of an issue or problem. It can be a physical, chemical, biological, or socio-economic variable, such as temperature, soil moisture or streamflow and can be measured locally or remotely. An index is a composite measure that aggregates multiple indicators into a single value or score \autocite{unitednationsuniversityTooManyIndicators2017,williamsUsingIndicatorsExplain2017, svobodaHandbookDroughtIndicators2016}. Indices are commonly developed on regional or national level to account for the specific circumstances \autocite{unitednationsuniversityTooManyIndicators2017}. This case specification, together with different measurement and aggregation methods, partial inconsistency of definitions and differently focussed objectives on qualitative, quantitative, risk or impact scenarios can constrain their practical application and intercomparability \autocite{svobodaHandbookDroughtIndicators2016,unitednationsuniversityTooManyIndicators2017}. 
Since there is no one definition of drought, water scarcity or security, there is no one best solution to the choice between the many indicators and indices for either of those.

% possibly shorten this.. naming all these indices might be a little overkill, though naming none is also not feasible.
Precipitation, evapotranspiration, soil moisture, lake and groundwater levels, streamflow and vegetation water stress are among the most prominent drought indicators \autocite{europeandroughtobservatoryDroughtIndicators2017}. In order to adequately account for the different drought stages different drought indices, that aggregate these and other indicators, are applied. Among the most prominent meteorological drought indics are the Standardized Precipitation Index \textit{SPI} together with its extension the Standardized Precipitation-Evapotranspiration Index \textit{SPEI} \autocite{europeandroughtobservatoryDroughtIndicators2017,ncarStandardizedPrecipitationEvapotranspiration,ncarStandardizedPrecipitationIndex}. Agricultural drought indices like the Soil Moisture Anomaly \textit{SMA} or the Anomaly of Vegetation Condition \textit{FAPAR Anomaly} are based on soil moisture indicators and absorbed radiation fractions, respectively. By quantifying water flow volumina, the Low Flow Index \textit{LFI} belongs to the hydrological drought indices \autocite{europeandroughtobservatoryDroughtIndicators2017, svobodaHandbookDroughtIndicators2016}. In addition to these and other types of indices, such as Combined Drought Indices, the \textit{Handbook for Drought Indicators and Indices} lists over 50 drought indicators and indices. For further and more in-depth information, please refer to the interactive website of the \arcfull{IDMP} launched by the \acrfull{wmo} and \acrfull*{gwp} \autocite{idmpIndicatorsIndicesIntegrated2021}. 

All of these drought indices give a good impression about the physical side of climate anomalies, but none of the above mentioned indices link those climate anomalies to socioeconomic vulnerabilities \autocite{enenkelWhyPredictClimate2020}. \autocite{mishraWaterSecurityChanging2021} argue, that the framing of water security challenges extends beyond singular indicators. \autocite{lackstromBackyardHydroclimatologyCitizen2022} argue further, that assessments that only consider physical factors overlook the broader impact of drought on social, economic, and ecological systems.
The simple but widely used Falkenmark Indicator (Falkenmark et al. 1989) incorporates human factors by calculating a ratio between the given amount of water and the number of people living within that domain. By further categorizing this ratio to a level of water scarcity, the Falkenmark Indicator indicates the supply sides effects of water scarcity but variabilities, demand and socioeconomic factors are not represented. More dedicated indices like the \acrfull*{iwmi} Indicator and the \acrfull*{wpi} as well as other indices measuring water security give a more extensive representation of the overall situation \autocite{arreguin-cortesMunicipalLevelWater2019,liuWaterScarcityAssessments2017}. The \arcshort{wpi} for example represents the weighted average of five pre-standardized components namely, water availability, access, capacity, use and environment \autocite{sullivanWaterPovertyIndex2003}.

Determining the right set of indicators and indices for a given region to e.g. assess hazard severity depends on the objective and available data and is often a balancing act between many factors and circumstances \autocite{svobodaHandbookDroughtIndicators2016}. Besides the pure description of what certain natural or social circumstances \textit{are}, there is an growing interest to understand what these conditions will \textit{do} \autocite{boultDroughtImpactbasedForecasting2022, lackstromBackyardHydroclimatologyCitizen2022}.
The effects of these conditions on the ground are most often called the \textit{impact} of a certain weather phenomenon or climate development such as a drought hazard. Impacts can be direct or indirect and a generally difficult to quantify economically \autocite{vereintenationenSpecialReportDrought2021}. The level of impact is commonly determined based on the severity of the hazard, the exposure of the investigated elements and their respective vulnerabilities \autocite{harrowsmithFutureForecastImpact2020,svobodaHandbookDroughtIndicators2016,vereintenationenSpecialReportDrought2021}.
This concept is generally expressed by the risk equation

        \[Risk = f(Hazard, Exposure, Vulnerability)\]

    where

        \[Vulnerability = f(Level of Coping Capacity, Level of Adaptive Capacity)\]

\autocite{boultDroughtImpactbasedForecasting2022,harrowsmithFutureForecastImpact2020,vereintenationenSpecialReportDrought2021}. Drought hazard can be evaluated and described by the above mentioned indicators and indices with difficulties lying in the contextualization and setting of the threshold levels to separate between fluctuations within the normal range and extreme events. Exposure is commonly defined as social, economic, cultural or natural assets, services or resources in places that could be adversely affected by a hazard \autocite{ipccClimateChange20142014}. Exposed elements can be more ore less vulnerable to the hazard. Vulnerability conditions are determined by the sensitivity or susceptibility of a system, community or individual to physical, social, economic or environmental factors or processes \autocite{ipccClimateChange20142014}. These conditions are often further described as the level of coping and adaptive capacities. Coping capacities refer to available skills and resources of systems, organizations or individuals to address, manage and overcome unfavourable circumstances \autocite{ipccGlossaryTerms2012}. In the same manner, adaptive capacities relate to preparation, reduction and moderation of those impacts.

The establishment of a functional relationship between the hazard, exposure and vulnerability to its impact can be rather difficult for numerous reasons and is further discussed by \autocite{boultDroughtImpactbasedForecasting2022} for interested readers. Moreover, all these factors change over time, so that the quality of the calculations depends strongly on the timeliness of the data basis \autocite{harrowsmithFutureForecastImpact2020}. 

Relatively recent approaches argue for numerous benefits and reasons for greater inclusion of local knowledge and community integration in these approaches \autocite{balehegnIndigenousWeatherClimate2019,dubeFrameworkIntegrationTraditional2016,ebhuomaFrameworkIntegratingScientific2020,giordanoIntegrationLocalScientific2013a,greyIntegratingLocalIndigenous2020,hermansExploringIntegrationLocal2022a,mercerCultureDisasterRisk2012,mutasaKnowledgeApartheidDisaster2015,nyetanyaneIntegrationIndigenousKnowledge2020,nyongValueIndigenousKnowledge2007}. Another emerging area in scientific interest is the gender inequality of drought impacts \autocite{acharyaWhenRiverTalks2019,fanningDroughtDisplacementLivelihoods2018,hiwasakiLocalIndigenousKnowledge2015,mustafaGenderingFloodEarly2015,sachsRoutledgeHandbookGender2020,saniGenderOtherVulnerabilities2022}. Although these topics are of great interest, they fall largely outside the scope of this particular work.

An understanding of the severity of droughts and their current impacts enables targeted responses, as well as to allow for the development of future predictions based on current conditions. In this context, recent efforts have increasingly emphasized proactive and forward-looking measures in disaster relief initiatives. The forthcoming chapter will explore this relatively recent shift in approach and its implications for improving drought management strategies.


\section{FbF, EAP, AA & Early Warning + trigger}\label{sec:fbf}

Traditionally, disaster management efforts have primarily focused on long-term preparedness or post-disaster response, thus only providing assistance and relief to affected communities after a disaster has occurred (TODO: policy overview, hyogo framework [UNISDR], coughland et al 2015). The lack of standardized procedures for forecast-based actions led to disaster warnings often going unheard \autocite{kolenImpactsStormXynthia2013}. In the context of increasing frequency and severity of natural disasters, coupled with the impacts of climate change, the need for a more proactive approach that can reduce the impact of disasters on vulnerable communities became apparent \autocite{coughlandeperezForecastbasedFinancingApproach2015,trisosAfrica2022}. Nonetheless, financial resources were for the time being strongly directed towards post-disaster response and incentives to invest in new and complex scientific developments including relatively high uncertainties were limited \autocite{coughlandeperezActionbasedFloodForecasting2016}. This changed with the development and successful integration of several new forecast-based financing systems that utilized the opportunity gap between a forecast and the disaster to successfully reduce corresponding impact. Based on this, to "substantially increase the availability of and access to multi-hazard early warning systems and disaster risk information and assessments to people by 2030" became one of seven global targets of the Sendai for Disaster Risk Reduction 2015-2030 framework \autocites{coughlandeperezActionbasedFloodForecasting2016}[12]{undrrSendaiFrameworkDisaster}. Today, large institutions have now specialized sections for the financing of Early Actions such as the Climate Risk and Early Warning Systems Initiative \textit{(CREWS)} and the Global Risk Financing Facility \textit{(GRiF)} to support and backup \acfp{ea} \autocite{crewsClimateRiskEarly,GlobalRiskFinancing}. Forecast-based Financing \textit{(FbF)} has thus emerged as a promising approach to disaster management that enables proactive, timely, and cost-effective responses to disasters \autocite{coughlandeperezForecastbasedFinancingApproach2015} (TODO: add “FORECAST-BASED FINANCING An innovative approach” ([pdf](zotero://open-pdf/groups/4773535/items/3C2CE7BS?page=1&annotation=UKWEKCTA))).

The \acrfull*{ifrc} together with the \acrfull*{rccc} and \acrfull*{grc} have developed and improved the FbF programme to fund EAs since 2007 \autocite{ifrcForecastbasedFinancingNew2019}. 

\missingfigure[options]{“Figure 1 - FbF Diagram” ([RCRC, 2020, p. 3](zotero://select/groups/4773535/items/UESIQTRJ)) ([pdf](zotero://open-pdf/groups/4773535/items/P5JPVZ97?page=3&annotation=W3UC7H26))}

Following \autocite{coughlandeperezForecastbasedFinancingApproach2015, coughlandeperezActionbasedFloodForecasting2016} the structure of FbF can be distilled down to:
    “When forecast states that an agreed-upon probability threshold is exceeded for a hazard of a designated magnitude, then an action with an associated cost must be taken that has a desired effect and is carried out by a designated organisation.” \autocite[2]{coughlandeperezActionbasedFloodForecasting2016}.
Thus, the FbF approach involves three key components (1) triggering (2) pre-defined EAs and securing a (3) financing mechanism in advance (compare \ref{TODO: figure fbf}) (TODO: “Forecast-based Financing A new era for the humanitarian system” ([pdf](zotero://open-pdf/groups/4773535/items/KQZXSWVN?page=1&annotation=3BW2ZYST))). These components are summarized in an Early Action Protocol \textit{(EAP)} (TODO: cite policy overview “These three components are summarized in Early Action Protocols (EAPs).” ([“Forecast-based financing: A policy overview”, p. 2](zotero://select/groups/4773535/items/35XBEGJ7)) ([pdf](zotero://open-pdf/groups/4773535/items/8YZAQB5L?page=2&annotation=58UQZK6T))). 

\missingfigure{“FBF has three” ([pdf](zotero://open-pdf/groups/4773535/items/K89MIG2V?page=3&annotation=PF946AET))}

%-----------------------------------
%	SUBSECTION 4.1 Early Action Protocol
%-----------------------------------

\subsection{EAP and drought specifics}\label{subsec:eap}

In the \acrfull*{eap} triggers, actions to be taken and financing mechanisms are clearly outlined, thus summarizing and explicitly assigning responsibilities to the involved actors, ensuring that everyone understands their role and task in the event of activation \autocite{ruthForecastbasedFinancingPolicy2017}. This results in clear accountability and full commitment from all stakeholders, facilitating the timely and efficient implementation of the predetermined actions \autocite{ruthForecastbasedFinancingPolicy2017}.
Two types of analyses, namely the identification of forecasts and the risk assessment, form the basis for specifying the trigger, affected regions, and selected actions in the \acshort*{eap} (see Figure \ref*{TODO:}). 

\missingfigure{“Figure 2 - EAP Validation Steps” ([RCRC, 2020, p. 4](zotero://select/groups/4773535/items/UESIQTRJ)) ([pdf](zotero://open-pdf/groups/4773535/items/P5JPVZ97?page=4&annotation=6WWVYXLT))}%TODO:

Both assessments are primarily based on historic data and experiences. To identify suitable forecast(s), various forecasts are compared and analysed in terms of their capacities and performance to predict hazards. This is done mainly through a historically grounded analysis. Ultimately, a specific impact threshold based on one or a combination of several impact-based forecasts becomes the basis for triggering actions. This trigger also depends on the outcome of the risk assessment, as the impact of the hazard is highly influenced by the risk on site \autocite{ifrcFbFPractitionersManual2023,ifrcForecastbasedFinancingNew2019}.

The risk assessment is a complex analysis that takes numerous factors on scales of the hazard, and its sub-hazards, exposure, vulnerability and together with its coping and adaptive capacities, into account \autocite{ifrcFbFPractitionersManual2023}. Potential inputs depend strongly on the respective hazard and can range from records of historical events, housing location and building structures in the case of hurricanes and floods to social factors like income, demographics and school attendance. The objective being the identification of corresponding impact levels, thus determining the most effective actions and allocating resources as objectively as possible. Nonetheless, most of these parameters are proxies, as direct information about localized impact is seldom, outdated, of low accuracy or quality \autocite{ifrcFbFPractitionersManual2023}.

Due to the majority of the implemented \acrshortpl{eap} concentrating on fast onset disasters such as floods, hurricanes or strong rains, the FbF concept were primarily focussed and developed in this regard. Here, typically a sole trigger and its associated set of actions are established, emphasizing rapid responses, given that there is often less than 48 hours between the activation and the occurrence of the disaster. \autocite{rcrcFORECASTBASEDFINANCINGEARLY2020}. Drought as a usually slow-onset hazard, on the other hand, pose unique structural challenges to the process of determining thresholds to trigger actions as impact builds up over time and is highly dependent on the context \autocite{boultDroughtImpactbasedForecasting2022}. These challenges of identifying a forecast, determining a trigger and seclting actions are further outlined in the coming chapters.

%TODO: checken ob das auch wirklich so ist..
The specification of the financing mechanism as one of the three key components will not be covered in any further detail in this work, as the \acrshort{ifrc} has extended their Disaster Relief Emergency Fund \textit{(DREF)} with Forecast-based Action as dedicated mechanism to adequately support their increased numbers of FbF projects. Once the forecast-based trigger is met and the EAP is activated, the financing mechanism automatically assigns resources, which solves the issue of financing to a large extent and is therefore no longer of great interest to this piece of work.

%-----------------------------------
%	SUBSECTION 4.2
%-----------------------------------

\subsection{Forecasts}

Indicators and indices as discussed in chapter \ref*{subsec:indicators} measure the severity, duration and spatial coverage of hazard conditions based on historical and current weather data. They provide a snapshot of current conditions and serve as an indicator of the overall situation. Forecasts, on the other hand, use these indices together with climate models and weather data to predict future conditions and provide early warning of potential hazard events. Thus, forecasts extend the retrospective and current measures of indices to future prediction.
Similar to the indices, a single forecast usually only covers certain facets of a hazard. In the case of droughts, the thematic orientation commonly follows its definition classification into meteorological, hydrological and agricultural subdivisions. Furthermore, forecasts can additionally be categorized into global, continental or regional spatial scales with coarser scaling predictions mostly correlating with longer time spans and vice versa \autocite{baltiReviewDroughtMonitoring2020}. Global to continental meteorological drought forecasts with the focus on seasonal or inter-seasonal predictions are often based on same scale phenomenons such the Julian-Madden Oscillation, the ENSO cycles or the Indian Ocean Dipole \autocite{andersonMaddenJulianOscillationAffects2022,goreUnderstandingInfluenceENSO2020,yuanInfluencesIndianOcean2008}. These conditions are mostly collected through satellite and weather data often utilizing drought indices such as the SPI, SPEI and EDDI indices \autocite{kimIntegratedDroughtMonitoring2021}. Further drought prediction services such as the National Integrated Drought Information System of the US government, the European Drought Observatory (EDO) or its adaptation, the East African Drought Watch, utilize a wide range of different indices to predict hazard development and their impacts \autocite{europeandroughtobservatoryDroughtIndicators2017,icpacDroughtIndicators2023, nidisOutlooksForecasts2023}. These institutions also produce timely forecasts, but their data sources are usually based on the same remote evaluations mostly predicting what the weather and climate will be, and not what its implications on the ground will look like \autocite{enenkelWhyPredictClimate2020}. 

The transition to impact-based forecasts represents a radical shift in the way these forecasts are produced and opperationalized \autocite{ifrcFbFPractitionersManual2023}. Practically, this would change the information that a forecast would provide from predicting e.g. precipitation patterns to e.g. the magnitude and spatial coverage of crop failure \autocite{harrowsmithFutureForecastImpact2020}. The challenges of functional relationships, complex interconnected cause and effect networks and data availability mentioned in chapter \ref*{indicator} are also applicable here, but the change to impact-based information results in multiple benefits to practitioners nonetheless. Impact-based forecasts help with the identification and prioritization of areas and communities most severely impacted. They do this by supporting a transparent, evidence-based, sector- and context-specific decision-making process directly focussing the population at risk \autocite{ifrcFbFPractitionersManual2023}.

\autocite{boultDroughtImpactbasedForecasting2022} argue even further for an adapting and dynamic impact assessment process, as decadal shifts in climate variabilities, changing exposure and vulnerabilities are not incorporated in a pre-defined system. They propose a hybrid framework of multi-hazard forecasts interlinked with static vulnerability and dynamically adjusted with real-time expert vulnerability assessments. Threshold triggers are lower, where static vulnerabilities are higher. However, both the regular pre-defined impact forecast and the dynamic impact forecast must be preceded by a selection and definition of triggers and actions.

\subsection*{Trigger definition}\label{subsec:trigger}

“Triggers are mainly combination of hydro-meteorological forecast combined with exposure and vulnerability data” \autocite[19]{rcrcFORECASTBASEDFINANCINGEARLY2020}. There are commonly two ways to define a trigger for early actions. On one side, triggers can be consensus-based, meaning experts make real-time judgements by synthesizing information from multiple sources, or on the other side, triggers are data-driven, peer-reviewed and validated well in advance of a potential event \autocite{rcrcFORECASTBASEDFINANCINGEARLY2020}. Drought with its different layers of complexity may also benefit from a combination of these mechanisms, as e.g. the framework of \autocite{boultDroughtImpactbasedForecasting2022} proposed above shows. Generally, good conditions for effective trigger development are sufficient historical data, knowledge about local livelihoods and how diverse parts of communities are influenced differently, thorough identification of differentiated impact drivers and their correlation to magnitudes as well as trustworthy forecasts \autocite{coughlandeperezForecastbasedFinancingApproach2015,coughlandeperezActionbasedFloodForecasting2016,elisabethstephensFORECASTBASEDACTION2015,harrowsmithFutureForecastImpact2020,rcrcFORECASTBASEDFINANCINGEARLY2020}. 
Furthermore, the framing and definition of the underlying forecast, indices and indicators are paramount as data-driven triggers are "specific values of an indicator or index that initiate and/or terminate each level of a drought plan and associated mitigation and emergency management responses.” \autocites{rcrcFORECASTBASEDFINANCINGEARLY2020}[13]{svobodaHandbookDroughtIndicators2016}. This specification is highly context specific and e.g. in the case of flood can be defined as the level when the river breaches its banks and inundates the surrounding area. Though, in another area this overflow may only inundate open space and thus lead to no impact at all \autocite{elisabethstephensFORECASTBASEDACTION2015}. This circumstance is relatively easy to grasp, has a single trigger and one set of specified actions such as evacuation, transportation and early warning and is therefore well integrable and implementable (see upper illustration in figure \ref*{TODO: RCRC p.20 Figure 5}) \autocite{siahaanForecastbasedActionDREF2018}. 

\missingfigure{RCRC p.20 Figure 5}

Drought, due to its slow-onset and potentially cascading impacts that only builds up over time complexifies the process of trigger definition as \acrfullpl{aa} to some impacts may go hand in hand with active responses in some areas and be to early in others. Furthermore, forecast certainty, granularity and accuracy all decrease the more one looks into the future \autocite{rcrcFORECASTBASEDFINANCINGEARLY2020}. Deciding when to trigger is therefore a critical and challenging aspect of conceptualizing a drought \acrshort{eap} (see bottom illustration in figure TODO: RCRC p.20 Figure 5). Practitioners and experts interviewed by the \autocite{rcrcFORECASTBASEDFINANCINGEARLY2020} advocate for a staggering triggering system \autocite. Here, multiple triggers with different sets of \acrshortpl{aa} would extend the single trigger mechanism and give the opportunity to account for the different phases and the inherent complexity of the phenomenon drought. Moreover, the \autocite[30]{rcrcFORECASTBASEDFINANCINGEARLY2020} calls for the development of "unconventional triggers for \acrfull*{FbA}" as the trigger development is not yet complete.

\subsection{Anticipatory Actions}

% keep it concise. It is not complicated. No reason to blow it up.

Anticipatory Actions are at the heart of every EAP and their execution is what everything is working towards. The goal of every Anticipatory Action is to help people and communities at risk to reduce negative impacts of a hazard. The final execution is preceded by some conceptual and practical steps. The establishment process begins with the identification of contextually meaningful, suitable and locally realisable actions with special focus on stakeholders, resources and available lead-time. These are further prioritized and selected based on the risk assessment, type and magnitude of hazard, and forecasting capabilities. When a first set of \acrshort*{aa} is defined, they are worked through in detail, reflected on with stakeholders and ultimately finalised. Together with an evaluation phase, this process is often a simultaneous and iterative process which also does not stop with the operationalisation of the \acrshort{eap} \autocite{elisabethstephensFORECASTBASEDACTION2015,ifrcGlossaryTermsForecastbased2023,ifrcFbFPractitionersManual2023a,rcrcFORECASTBASEDFINANCINGEARLY2020}.
In practice, \acrlongpl{aa} are commonly split into a preparation and an activation phase. The preparation phase builds on the process described above, but also extends to actions that prepare for rapid activation, such as the prepositioning of water tablets before the rainy season \autocite{elisabethstephensFORECASTBASEDACTION2015}. The activation phase requires a constant operation of forecast monitoring and is initiated when the trigger is reached. Timely information dissemination, releasing and receiving funds, implementing of the \acrshortpl{aa} and subsequent evaluation are part of this phase \autocite{elisabethstephensFORECASTBASEDACTION2015,ifrcFbFPractitionersManual2023a}. Often, \acrshortpl{aa} are not very different from response actions except of their predictive and proactive nature. However, this foresight comes with the cost of uncertainty and forecasts may not always be accurate. The simultaneous implementation of \acrshortpl{aa} in the absence of the disaster is commonly referred to as \textit{to act in vain} \autocite{coughlandeperezForecastbasedFinancingApproach2015}. Besides financial costs, this may also manifest in reputational costs in e.g. the case of Early Warning and evacuation if false alarms occur too frequently \autocite{elisabethstephensFORECASTBASEDACTION2015}. Albeit, a growing body of evidence suggests that the benefits of AAs outweigh the costs substantially \autocite{cabotventonEconomicsResilienceDrought2018,coughlandeperezForecastbasedFinancingApproach2015,gualazziniEWEAEarlyWarning2021}. Furthermore, the issue of \textit{acting in vain} can be lessen by staggering triggers and adjusting AAs in accordance with long-term resilience building \autocite{wfpMonitoringEvaluationAnticipatory2021}. This can allocate the actions more precisely and increases the general benefits. \autocite{ifrcGlossaryTermsForecastbased2023} makes these design adjustments the basis of its definition of \textit{acting in vain} and thus argues for the abolition of this term, since the benefits of acting should always outweigh not acting at all.


\section{Citizen Science, Crowdsensing, Volunteersensing, VGI, alternatively satellite image interpretation}\label{sec:cs}

The inclusion of local knowledge in the system of Early Warning and Anticipatory Action can result in many benefits as already mentioned in the end of chapter \ref*{subsec:indicators}. Adapting knowledge and policies to local conditions and people as well as learning from them, strengthening autonomous responses and involving local stakeholders in all stages of the processes are just some of the potential ways to improve implementations \autocite{giordanoIntegrationLocalScientific2013a,idmpDroughtWaterScarcity2022,lackstromBackyardHydroclimatologyCitizen2022,lealfilhoRoleIndigenousKnowledge2022,lealfilhoUnderstandingResponsesClimaterelated2022}. One way to include local knowledge is through Citizen Science, very broadly defined as "public participation in scientific research and knowledge production" \autocite{fraislCitizenScienceEnvironmental2022} .
Historically, the first citizen science project was possibly the Christmas Bird Count run by the National Audubon Society in the USA every year since 1900 \autocite{linkHierarchicalModelRegional2006,silvertownNewDawnCitizen2009}. Since around 2000, the number of publications in regard to Citizen Science has risen substantially and has established itself as a vibrant area of scientific interest \autocite{kirschkeCitizenScienceProjects2022}. As more and new thematic fields joined this area of interest, numerous approaches have been made to define Citizen Science more precisely \autocite{haklayWhatCitizenScience2021}. Over 30 definitions were selected by \autocite{haklayWhatCitizenScience2021} to explore their ambiguity and extend the best practice principles and characteristics of citizen science established by the European Citizen Science Association (ESCA) \autocite{escaTenPrinciplesCitizen2015,escaECSACharacteristicsCitizen2020}. Different political, scientific or societal lenses along with a variety of focal points such as (1) biology, conservation and ecology, (2) geographic data and (3) social sciences and health related issues have all contributed to the concept of Citizen Science \autocite{haklayWhatCitizenScience2021,kirschkeCitizenScienceProjects2022}.
The first, natural research and conservation, is the orientation most frequently related to Citizen Science with overlapping concepts to community-based, volunteer and participatory monitoring. It has common interests with the second category of Volunteered Geographic Information (VGI) in topics such as crowdsourcing and data quality whereas the the third category mostly resolves around public engagement with intersections to CS in public participation \autocite{kullenbergWhatCitizenScience2016}. In order to highlight the core of Citizen Science alongside the different disciplinary orientations of the research, different frameworks, guidelines and levels of participation have been designed.\autocite{kirschkeCitizenScienceProjects2022} created a three cluster framework of design principles around \textit{citizen} and \textit{institutional} characteristics, together with their \textit{forms of interaction}. Within these categories \autocite{kirschkeCitizenScienceProjects2022} highlight various qualities and skills such as age, social status, motivation, knowledge and education of the contributing citizens, financial and human resources on the institutional side and the method and density of communication and feedback practices as important parts of interactions. Guidelines and principles further specify, expand and structure these broad topics to make them practically applicable in various contexts \autocite{citizenscience.govBasicStepsYour,escaTenPrinciplesCitizen2015,escaECSACharacteristicsCitizen2020,EUCitizenScience2023,fraislCitizenScienceEnvironmental2022,garciaFindingWhatYou2021,minkmanCitizenScienceWater2015,pocockStrategicFrameworkSupport,skarlatidouWhatVolunteersWant2019}. Citizen science projects can also be differentiated according to how engagement with participants is designed. This is referred to as the \textit{levels of participation} and is commonly structured into four levels. Increasing in participation intensity, \autocite{buckinghamshumGlobalParticipatoryPlatform2012} categorize them into (1) Crowdsourcing, (2) Distributed Intelligence, (3) Participation Science and (4) Extreme Citizen Science. Following this categorization, participants can be (1) 'sensors', (2) 'interpreters', (3) engaged in problem definition and data collection or even (4) part of the analysis. 
Depending on the level of participation and thematic orientation, Citizen Science is related to concepts of classic monitoring practices (1), transdisciplinary research emphasizing engagement of the public along the entire process (2 \& 3) and participation involving "groups that are or perceive themselves as being affected by the decision" (3 \& 4) \autocites{buckinghamshumGlobalParticipatoryPlatform2012}{conradReviewCitizenScience2011}{minkmanCitizenScienceWater2015}[1]{rennParticipatoryProcessesDesigning2006}. 
Current challenges and limitations in Citizen Science projects are the complex demands in the conceptualization and design process with a wide range of required skills and resources, recruiting participants and sustaining their motivation, data quality and accuracy considerations, biases in collection and analysis as well as privacy regulations \autocite{fraislCitizenScienceEnvironmental2022}. Furthermore, research and CS projects are currently unevenly distributed on a global scale with an over representation of North American countries resulting in less experiences and guidelines for other areas and contexts \autocite{kirschkeCitizenScienceProjects2022, zhengCrowdsourcingMethodsData2018}. Nonetheless, numerous studies suggest promising developments and application possibilities addressing all of the above mentioned challenges in design, participants and data related issues \autocite{buckinghamshumGlobalParticipatoryPlatform2012,buddeParticipatorySensingParticipatory2017,escaECSACharacteristicsCitizen2020,fraislCitizenScienceEnvironmental2022,lowryGrowingPainsCrowdsourced2019,pocockStrategicFrameworkSupport,ruttenHowGetKeep2017,weeserCitizenSciencePioneers2018a}. 
% come back to this in the discussion --> data quality e.g. can be 'solved' by trainings and supervision

\subsection{Community-based monitoring}\label{subsec:cbm}

% maybe add that there are more related concepts: participatory monitoring instead? or: citizen observatories, community based monitoring and participatory monitoring

\acrfull{cbm} is a sub-concept of citizen science and can be allocated to different layers of participation, depending on its definition, aspects and final implementation \autocite{westonCommunityBasedWaterMonitoring2015}. \acrshort{cbm} can encompass "a process where concerned citizens, government agencies, industry, academia, community groups and local institutions collaborate to monitor, track and respond to issues of common community concern" \autocite[410]{whitelawEstablishingCanadianCommunity2003}. The focus of \acrshort*{cbm} on monitoring is fundamental, but the monitored subject, further handling of the data and the involvement of the participants can vary widely \autocite{baptisteCommunityLedMonitoringWhen2020,conradReviewCitizenScience2011,koehlerCitizenParticipationCollaborative2008,muhamadkhairCommunitybasedMonitoringEnvironmental2021,shirkPublicParticipationScientific2012,westonCommunityBasedWaterMonitoring2015}. Within this work, \acrshort*{cbm} is understood as a combination of two main aspects. The collection part often refers to concepts of \textit{Crowdsourcing} or \textit{Crowdsensing} (see next Chapter \ref{subsec:mcs}) and a management aspect which promotes the incorporation of the generated information into community decision-making processes \autocite{conradCommunitybasedMonitoringScience2007, keoughAchievingIntegrativeCollaborative2006}. 
\acrlong*{cbm} can serve many purposes but its implementation and application is not always recommended. Therefore, many guidelines precede the design with an assessment of the feasibility of this approach \autocite{associationTenPrinciplesCitizen2015,citizenscience.govBasicStepsYour,fraislCitizenScienceEnvironmental2022,minkmanCitizenScienceWater2015, pettiboneCitizenScienceAll2016}. Here, the challenges, benefits and capabilities of the \acrshort*{cbm} approach are compared with the problem and core objectives of the project. It is emphasized that \acrshort*{cbm} should not be the goal itself, but only a means to fulfil the project goals \autocite{minkmanCitizenScienceWater2015}. Nonetheless, the diversity of this approach means that other goals can be pursued and achieved apart from the main interests (see Chapter \ref*{subsec:guidelines}). For example, enriching participants by addressing their needs, advancing their knowledge or teaching them new skills is considered as fundamental and important to achieving the main objective as it is to a successful project \autocite{fraislCitizenScienceEnvironmental2022}. 
In the following, a short overview about challenges, benefits and recommendations of \acrshort*{cbm} is given, broken down in the design phase, incorporation of participants and data concerns.

%%%%%%%%%%%%%%%%%%%%%%%%%%%
% possibly subsubsec ??
% design 
The conceptualization of CBM projects on the level of participation or the tripartite division according to characteristics of citizens, institutions and their forms of interaction have already been mentioned in connection with the broader concept of Citizen Science and are also applicable here. More concrete design factors and variables were synthesized by \autocite{kirschkeCitizenScienceProjects2022} but the systematic understanding of their influences on the success of remained unclear for now. A selection of subjects outside of the original research itself could be overall project management, communication in its various forms and with all stakeholders, community and participant recruitment, training and management, data management and analysis as well as the final implementation and operation of the project. Moreover, there is agreement that no \textit{one-size-fits-all} solution exists and different goals, resources, and contexts have considerable influence on the design from project to project \autocite{fraislCitizenScienceEnvironmental2022}. In order to account for the variety of challenges and to maximize the benefits, staged frameworks have been developed to guide the design \autocite{citizenscience.govBasicStepsYour, fraislCitizenScienceEnvironmental2022,garciaFindingWhatYou2021,minkmanCitizenScienceWater2015}. Yet, these frameworks can be relatively coarse and imprecise and are often partly tailored to specific goals and contexts, making a combination of several such frameworks and the inclusion of further guidelines and recommendations potentially necessary to tailor the design to the specific situation. 

%%%%%%%%%%%%%%%%%%%%%%%%%%%%%%%
% participants
Participants can take many roles in a \acrshort*{cbm} project based on the level of participation chosen but regardless of this, their adequate integration is seen as a cornerstone of any \acrshort*{cbm} project \autocite{land-zandstraParticipantsCitizenScience2021}. Knowledge and skills as well as other socio-economic variables can vary widely between participants and it is important to account for this to inspire and keep participants motivated to contribute \autocite{minkmanCitizenScienceWater2015,whitelawEstablishingCanadianCommunity2003}. One mayor drawback of online collaborative initiatives is often that a considerable proportion of contributors only participate once and with minimal effort while a relatively small number of participants are responsible for the majority of the work \autocite{sauermannCrowdScienceUser2015}. Understanding and thus sustaining the motivation of participants is therefore central to a successful project. The subject of what drives individuals to participate in citizen science projects has been extensively explored in literature \autocite{land-zandstraParticipantsCitizenScience2021,minkmanCitizenScienceWater2015,mloza-bandaCrowdsensingSuccessfulWater2018,ruttenHowGetKeep2017,tipaldoCitizenScienceCommunitybased2017,walkerBenefitsNegativeImpacts2021a,walkerBenefitsNegativeImpacts2021}. Motivation can be intrinsic or extrinsic and spans from the will to contribute to science and conservation over meeting and helping other potentially like minded people to learning new skills and financial compensation \autocite{minkmanCitizenScienceWater2015,rotmanDynamicChangesMotivation2012 ruttenHowGetKeep2017}. According to \autocite{rotmanDynamicChangesMotivation2012}  study, egocentric motives tended to drive new participants, whereas established participants were more motivated by altruistic reasons, such as helping others. Furthermore, the individual adaptation of the task's difficulty to each participant was suggested to positively influence motivation in order to neither bore nor overwhelm \autocite{minkmanCitizenScienceWater2015}. Other factors to inspire and sustain motivations are, among others, the expected benefits, acknowledgement and feedback culture and its perceived usefulness and integration into further processes \autocite{land-zandstraParticipantsCitizenScience2021,minkmanCitizenScienceWater2015,pettiboneCitizenScienceAll2016}. In addition to strengthening motivation, breaking down barriers to participation can also prove helpful. For this, understanding the background and circumstances of the participants is important. In their work for hydrological monitoring in Kenya, \autocite{weeserCitizenSciencePioneers2018a} could partly attribute low participation rates to the transmitting costs of 0.01 USD per text message at some station. Offsetting these costs could subsequently increase the overall participation rate significantly. \autocite{weeserCitizenSciencePioneers2018a} further discovered, that actual compensation or incentives appeared unnecessary as the intrinsic motivation of the participants proved to be adequate once financial constraints were addressed. Besides financial and resource restrictions, lack of knowledge and skills can be addressed by providing adequate training \autocite{fraislCitizenScienceEnvironmental2022,lackstromBackyardHydroclimatologyCitizen2022}.

% data
Supervision, external or mutual feedback and preceding training of participants can also address common data quality concerns \autocite{albusAccuracyLongtermVolunteer2020,baalbakiCitizenScienceLebanon2019,fraislCitizenScienceEnvironmental2022}. Besides the characteristics of the participant, the difficulty of the measurement task itself influences the quality. Simpler tasks such as gauging e.g. water levels provided high data quality in \autocite{weeserCitizenSciencePioneers2018a} study. \autocite{baalbakiCitizenScienceLebanon2019} has further found that most of the data collected by citizen scientists is comparable to that of university scientists when it comes to chemical or physical qualities of water. \autocite{albusAccuracyLongtermVolunteer2020} could support this finding, by analyzing data from the Texas Stream Team (TST) citizen science program and found an agreement of 80\% up to 90\% for DO, pH and conductivity parameters. However, \autocite{baalbakiCitizenScienceLebanon2019} also noted a disparity in the bacteriological test results between citizen and university scientists, to which they remarked, that it may be explained by the complexity of the testing process and the quality of the testing materials employed. \autocite{aceves-buenoCitizenScienceApproach2015} evaluated over 80 peer-reviewed studies of which only 11\% reported no data accuracy issues but only one study reported, that the data was unusable. Based on the aforementioned findings, ensuring data quality and accuracy through appropriate quality assurance and control measures is crucial. However, despite the reliability and accuracy challenges associated with \acrshort*{cbm} data, \autocite{aceves-buenoCitizenScienceApproach2015} noted, that these issues typically do not have a significant impact on the data's overall usefulness.

Besides the more specific challenges and benefits mentioned above, \acrlong*{cbm} approaches can benefit scientists, decision-makers, communities and participants in multiple ways. In addition to achieving the main objectives, raising awareness of the issue, the needs and the problems at hand, as well as increasing knowledge among all project stakeholders, can lead to changes in behaviour, improved management, reduced risks and a better representation of local conditions in the regional, national and international context. \autocite{huangManagementDrinkingWater2020,walkerBenefitsNegativeImpacts2021}. Output quality can be enhanced when the objective is clear, participant involvement is recognized as a high priority, enough resources for design, implementation, operation and analysis are available and the monitoring protocol is not too complex \autocite{butteFrameworkWaterSecurity2022, pocockStrategicFrameworkSupport}. 
In an attempt to scale this concept across regions or even an entire country with many physical, social and economic differences, the \acrshort*{cbm} concept has been increasingly explored with mobile, network-enabled devices. This is, together with practical examples and projects, presented in the coming chapters.


% % \subsection{VGI} % keep it short or completely out of this -> will come up in CS anyway

% % “In the field of geography, the mapping of features such as buildings, road networks, and land cover can now be undertaken by citizens as a result of advances in Web 2.0 and global positioning system (GPS)-enabled mobile technology, which has blurred the once clear-cut distinction between map producer and consumer (Coleman et al., 2009).” ([Zheng et al., 2018, p. 703](zotero://select/groups/4773535/items/LJU68CG4)) ([pdf](zotero://open-pdf/groups/4773535/items/U8QNZLI6?page=6&annotation=W66QRY3C))

% % “In a seminal paper published in 2007, Goodchild (2007) coined the phrase Volunteered Geographic Information (VGI). Similar to the idea of crowdsourcing, VGI refers to the idea of citizens as sensors, collecting vast amounts of georeferenced data.” ([Zheng et al., 2018, p. 703](zotero://select/groups/4773535/items/LJU68CG4)) ([pdf](zotero://open-pdf/groups/4773535/items/U8QNZLI6?page=6&annotation=73UDWLM6))

% % “OpenStreetMap (OSM) is an example of a highly successful VGI application (Neis & Zielstra, 2014),” ([Zheng et al., 2018, p. 703](zotero://select/groups/4773535/items/LJU68CG4)) ([pdf](zotero://open-pdf/groups/4773535/items/U8QNZLI6?page=6&annotation=LJBW7EY4))

% %-----------------------------------
% %	SUBSECTION 6.2
% %-----------------------------------

% \subsection{local knowledge (???)} %only when I still have time there are just tons of information maaaan
% These \textit{unconventional triggers} could be based on local or indigeneous knowledge and data from the ground.

% “External stakeholders’ attitudes towards and engagement with local knowledge in 1 disaster risk reduction: are we only paying lip service?” ([Šakić Trogrlić et al., 2021, p. 1](zotero://select/groups/4773535/items/BTRX6EIG)) ([pdf](zotero://open-pdf/groups/4773535/items/52GEZZFV?page=1&annotation=TLIV6XS7)). 

% % --> local context
% “Deep understanding of the local context, and the needs and wants of the targeted community would allow us to identify which drought impacts are most strongly felt by different groups of the community.” ([RCRC, 2020, p. 28](zotero://select/groups/4773535/items/UESIQTRJ)) ([pdf](zotero://open-pdf/groups/4773535/items/P5JPVZ97?page=28&annotation=EI2UMB2H))

% -->
% “Where they exist, these systems may be even more important for slow-onset hazards like drought, necessary but not sufficient - layers of additional (ideally) local indicators must be added to these in order to form an appropriate FbA trigger.” ([RCRC, 2020, p. 29](zotero://select/groups/4773535/items/UESIQTRJ)) ([pdf](zotero://open-pdf/groups/4773535/items/P5JPVZ97?page=29&annotation=YK99RIZB))


% local on the ground impact assessment not possible with current forecast abilities --> possibly link to that via CBS and the feasibility study --> (“Early Warning/Early Action Mechanisms: EWEA is working well in cases of health emergencies/epidemics through community-based surveillance (CBS); this allows the N” ([Somali Red Crescent Society, 2022, p. 51](zotero://select/groups/4773535/items/FZ6BJHJA)) ([pdf](zotero://open-pdf/groups/4773535/items/RJKNZZZ2?page=51&annotation=4C3HL8ES))) 
% thus, this and comparable approaches are investigated in the next chapter

% % nonetheless, difficult to scale
% “The Problem of Scale in Indigenous Knowledge: a Perspective from Northern Australia” ([Wohling, 2009, p. 1](zotero://select/groups/4773535/items/HIFJDYSG)) ([pdf](zotero://open-pdf/groups/4773535/items/BUPU6DGS?page=1&annotation=LPCCJN8Z))

% different trigger --> different set of actions --> interdependent of what the actions should look like (can be an interative forth and back coming process)--> can also differ for different groups

% % local knowledge --> same as with indicators, forecasting is also possible on the basis of local knowledge -> overview table
% “Table 1. Comparisons between indigenous knowledge-based seasonal forecasts and seasonal climate forecasts (adopted from Ziervogel and Opere 2010). Indigenous knowledge-based seasonal forecasts Seasonal climate forecasts Use biophysical indicators of the environment as well as spiritual methods Use of weather and climate models of measurable meteorological data Forecast methods are seldom documented Forecast methods are more developed and documented Up-scaling and down-scaling are usually complex Up-scaling and down-scaling are relatively simple Application of forecast output is less developed Application of forecast output is more developed Communication is usually oral Communication is usually written Explanation is based on spiritual and social values Explanation is theoretical Taught by observation and experience Taught through lectures and readings” ([Masinde and Bagula, 2012, p. 280](zotero://select/groups/4773535/items/EW9XSSZP)) ([pdf](zotero://open-pdf/groups/4773535/items/3WQ4S9PE?page=7&annotation=6XCISBM2))

% “Indigenous knowledge within an early warning system for droughts” ([Masinde and Bagula, 2012, p. 282](zotero://select/groups/4773535/items/EW9XSSZP)) ([pdf](zotero://open-pdf/groups/4773535/items/3WQ4S9PE?page=9&annotation=8Z9A9AW8))

% “The Best of Both Worlds: A Decision-Making Framework for Combining Traditional and Contemporary Forecast Systems” ([Plotz et al., 2017, p. 1](zotero://select/groups/4773535/items/3SBLBZEA)) ([pdf](zotero://open-pdf/groups/4773535/items/VAUJGIFB?page=1&annotation=RRFY2UWE))

% “B. Drought Forecasting in Sub-Saharan Africa” ([Masinde and Thothela, 2019, p. 304](zotero://select/groups/4773535/items/6D52T883)) ([pdf](zotero://open-pdf/groups/4773535/items/KLLQKDG2?page=2&annotation=ZQSDUEMX))

% there are more! Look into it.!

% “Researchers ([1], [19] and [20]) today concur that IK and modern science weather forecasts complement each other;” ([Masinde et al., 2013, p. 2](zotero://select/groups/4773535/items/M45MLGWC)) ([pdf](zotero://open-pdf/groups/4773535/items/LG6E76P4?page=2&annotation=JHQV2GYT))


% “In theory, focusing on what the weather will do, rather than what the weather will be, enables decision makers to plan and implement targeted preparatory actions to better reduce hazard impacts (Harrowsmith et al., 2020).” ([Boult et al., 2022, p. 2](zotero://select/groups/4773535/items/B2AQSTYL)) ([pdf](zotero://open-pdf/groups/4773535/items/W9TFLH43?page=2&annotation=NSLE7NL6))

% “Improving early warning of drought-driven food insecurity in southern Africa using operational hydrological monitoring and forecasting products” ([Shukla et al., 2020, p. 1187](zotero://select/groups/4773535/items/TE5NMA3T)) ([pdf](zotero://open-pdf/groups/4773535/items/9TNUGXSJ?page=1&annotation=TLBHC7BS))

% “Moving from drought hazard to impact forecasts” ([Sutanto et al., 2019, p. 1](zotero://select/groups/4773535/items/EUC5RV7N)) ([pdf](zotero://open-pdf/groups/4773535/items/9DI9EVBF?page=1&annotation=FCQCKY5P))


% %--> even though it is generally not recommend by the RCRC for a National Society to collect these local indicators by themselves

% “It is important to note that local indicators cannot be collected specifically for the FbF system by RCRC national societies.” ([RCRC, 2020, p. 30](zotero://select/groups/4773535/items/UESIQTRJ)) ([pdf](zotero://open-pdf/groups/4773535/items/P5JPVZ97?page=30&annotation=LRDPV2M7))

% “Indeed, collecting data on local indicators would require from the national society a team of enumerators that work continually to collect and process that information in all places where the program could possibly trigger (e.g. collect food price information for every village market). This would have extensive cost implications and likely over-burden the national society staff and volunteers.” ([RCRC, 2020, p. 30](zotero://select/groups/4773535/items/UESIQTRJ)) ([pdf](zotero://open-pdf/groups/4773535/items/P5JPVZ97?page=30&annotation=2YIIK6ZY))

% “As such, the inclusion of local indicators into an FbA trigger must involve assessing what indicators are relevant for the impacts the program is trying to anticipate and identify which of those indicators are already collected (e.g. the ministry of agriculture's food price bulletin) and are available at the time they would be needed to inform a possible trigger.” ([RCRC, 2020, p. 30](zotero://select/groups/4773535/items/UESIQTRJ)) ([pdf](zotero://open-pdf/groups/4773535/items/P5JPVZ97?page=30&annotation=7X3RFGVB))


% “2 Local knowledge in drought monitoring: an introduction to the literature review” ([Giordano et al., 2013, p. 526](zotero://select/groups/4773535/items/B7LM5ZR4)) ([pdf](zotero://open-pdf/groups/4773535/items/7I66DBIK?page=4&annotation=Z33M5FLQ))

%----------------------------------------------------------------------------------------
%	SECTION 7 MCS & other tools + water related monitoring (excel)
%----------------------------------------------------------------------------------------

\subsection{Mobile Crowdsensing (MCS)}\label{subsec:mcs} % practical applications of MCS, CBS & other tools + water related monitoring (excel)

Originating in 2006 from an article by \autocite{howeRiseCrowdsourcing} and Mark Robinson describing Crowdsourcing as a new internet based business model in the terms of "It's not outsourcing; it's crowdsourcing", by harnessing "the creative solutions of a distributed network of individuals through what amounts to an open call for proposals" \autocite[76]{brabhamCrowdsourcingModelProblem2008}. Nowadays crowdsourcing in scientific contexts is often applied as e.g. act of "collecting data without a direct integration into the scientific process" by a generally large audience \autocite[1591]{weeserCitizenSciencePioneers2018a}. Due to the merely perceiving and transferring and not further interpreting character, \textit{Crowdsourcing} is on the lowest level of participation levels. A more specific form of \textit{Crowdsourcing} is \textit{Crowdsensing} which refers to the process of measuring and collecting data by a large mass of contributors that involves using mobile devices and/or sensors to collect information about the environment. This is also known as \acrfull{mcs} \autocite{guoParticipatorySensingMobile2014, liuSurveyMobileCrowdsensing2018}.
\acrshort*{mcs} is part of a widespread transition in the way data is gathered and managed, with a shift away from conventional methods towards incorporating mobile devices, web platforms, and apps \autocite{capponiSurveyMobileCrowdsensing2019, sanllorentecapdevilaSuccessFactorsCitizen2020}. This transition is being driven by the development and proliferation of information technology infrastructure, which includes the collection, sharing, storage, cleaning and analysis of data \autocite{fraislCitizenScienceEnvironmental2022}. These components of the information technology infrastructure can be grouped into a four-layer architecture which is described in detail in the paper by \autocite{capponiSurveyMobileCrowdsensing2019}.
The first and top layer is the \textit{application layer} concerned about high-level user, task- and overall design and organizational aspects with some examples being user recruitment's, scheduling and contribution management. The \textit{data layer} as the second layer refers to storage, processing and analysis of the received data and is followed by the \textit{communication layer} which refers to methodological and technological aspects of the reporting characteristics. These include cellular, internet or other networks and their means of transmission. The bottom layer, the centrepiece of this architecture, is the \textit{sensing layer} which includes all tools, technologies and equipment involved in the data acquisition process \autocite{capponiSurveyMobileCrowdsensing2019}. Measurements can be of different types, intentional or unintentional, at the occurrence of an event or continuous, and are based on human observation, instrumental measurements or a combination of both \autocite{zhengCrowdsourcingMethodsData2018}. In this architecture hierarchy, data flows generally from the lowest to the highest layer \autocite{aceves-buenoCitizenScienceApproach2015,capponiSurveyMobileCrowdsensing2019,zhengCrowdsourcingMethodsData2018}.
Besides generally applicable challenges of \acrlong{cbm} such as data quality and participant motivation, main challenges of \acrshort*{mcs} are seen in the socio-technical, privacy and security realms referring to hard- and software availability, reliability and usability as well as balancing access rights, anonymisation and encoding with data trustworthiness \autocite{aceves-buenoCitizenScienceApproach2015,alfonsoMOBILEPHONEAPPLICATIONS2012,capponiSurveyMobileCrowdsensing2019,liuSurveyMobileCrowdsensing2018, minkmanCitizenScienceWater2015, noureenCrowdsensingSocioTechnicalChallenges2017a}. Nonetheless, \acrshort*{mcs} also provides many opportunities and solutions to designers, operators and participants alike. Among those are the relatively good and easy scalability and increase of monitoring network density, low barriers for participation and two-way communication options as well as high potential for automatization and interoperability with other applications and frameworks \autocite{alfonsoMOBILEPHONEAPPLICATIONS2012,minkmanCitizenScienceWater2015,sanllorentecapdevilaSuccessFactorsCitizen2020,weeserCitizenSciencePioneers2018a}. In the following, practical examples of \acrshort*{cbm} and \acrshort*{mcs} or a combination of both are presented, highlighting the wide-ranging application possibilities. %together with their advantages and disadvantages.

\subsection*{Practical Examples of CBM and MCS}\label{subsec:practical_examples}

The potential applications for \acrshort*{mcs}, embedded in \acrshort*{cbm} or as a stand-alone project, are, as for all Citizen Science, wide-ranging and diverse. Besides the thematic diversity, the socio-technical implementation, size and complexity can differ substantially from project to project. Established networks like the \acrfull*{cocorahs} founded in 1998 USA with over 25.000 observers facilitate the collection of daily weather observations and the sharing of written impact impressions via an online platform \autocite{cocorahsCoCoRaHSCommunityCollaborative2023,lackstromBackyardHydroclimatologyCitizen2022}. The Audubon's Christmas Bird Count (CBC) even goes back to the December of 1900 and in its 120th anniversary year over 81.000 observers counted more than 30 million individual birds \autocite{lebaron122ndChristmasBird2022}. Another major project in the realm of \textit{Crowdsourcing} and \acrshort*{mcs} is the 2004 founded OpenStreetMap Foundation. Started as a reaction to the failed release of geographic information in the United Kingdom, OSM as a collaborative community effort quickly became one of the most important sources of geographic information world wide \autocite{bennettOpenStreetMap2010, openstreetmapcontributorsOpenStreetMapBasemap2020}. Additional contemporary developments include the concept of \acrshort*{mcs} in citizen participation, Smart Cities, resource management, transport and behaviour evaluation and many more \autocite{dipasDIPASOrgDIPAS2023,europeancommissionCitizencentredApproachSmart2021, wangSurveyApplicationKey2022}. 
Other projects with a thematic focus on health, water and early warning are considered in more detail in the remaining part of this section. Health, as \acrfull*{cbs} is successfully implemented as \acrshort*{cbm} with NYSS as \acrshort*{mcs} in Somalia (FIXME: see chapter \ref*{TODO:}), and water and early warning projects, as they are thematically related to this work. Projects concerning VGI will not be discussed in depth in this context, as mapping in this project will most definitely be carried out by professional and trained personnel. 

\subsubsection*{Community-based Surveillance}\label{subsubsec:cbs}

Conventional surveillance systems for monitoring health of animals, humans and the environment rely on information of medical professionals, health facility records, and laboratory examinations to detect abnormalities that could signify potential outbreaks and newly emerging pathogens \autocite{mcneilLandscapeParticipatorySurveillance2022a}. However, these data are not sufficiently accessible in all regions of the world  to allow adequate responses \autocite{mcneilLandscapeParticipatorySurveillance2022a,nikolayEvaluatingHospitalBasedSurveillance2017}. The strong developments and increasing availability of mobile technologies, the recognition of the value of local knowledge in health management, and recently reinforced by the COVID 19 pandemic, have led to an an increasingly widespread use of \acrshort*{cbs} \autocite{kullenbergWhatCitizenScience2016,mcneilLandscapeParticipatorySurveillance2022a}. The \autocite{technicalcontributorstothejune2018whomeetingDefinitionCommunitybasedSurveillance2019} defined \acrshort*{cbs} as "the systematic detection and reporting of events of public health significance within a community by community members". With the growing importance of the \textit{One Health} approach, these "events of public health significance" span across the domains of human, animal and ecosystem health \autocite{cdcOneHealthBasics2022}.
\autocite{mcneilLandscapeParticipatorySurveillance2022a} identified 60 different ongoing surveillance systems across five continents. These systems were covering the three domains either stand-alone or in combination, on different spatial scales and with different technical characteristics. However, all projects have used some kind of digital technology, with websites and smartphones as the most common vehicles. Furthermore, a high percentage of the surveyed projects have noted the usefulness of the \acrshort*{cbs} approach as it "improved community knowledge and understanding" (78\%) and "earlier detection" (67\%). This finding is supported by various other studies \autocite{byrneCommunitycentredApproachGlobal2020,jarrettEvaluationPopulationMobility2020,mcgowanCommunitybasedSurveillanceInfectious2022,metugeHumanitarianLedCommunitybased2021,ratnayakeEarlyDetectionCholera2020,ratnayakePeoplecentredSurveillanceNarrative2020,technicalcontributorstothejune2018whomeetingDefinitionCommunitybasedSurveillance2019}.
The \acrshort*{cbs} approach has proven to be a more advantageous complement to the conventional system, especially if certain conditions are taken into account. \autocite{gueninParticipatoryEpidemiologicalOne2022} highlights the importance of congruent definitions and their adaptation to the different actors and roles as well as the adaptation of (two-way) communication channels. Preceding suitability assessments, simple design and reasonable incorporation of technology, effective community engagement, reliable and close surveillance through supervisors of local volunteers especially in the beginning as well as evaluation and feedback opportunities have been highlighted as key drivers for success. These drivers were grouped by \autocite{mcgowanCommunitybasedSurveillanceInfectious2022} in relation to (1) surveillance workers, (2) the community, (3) case detection and reporting, and (4) integration. Most of these factors and more have already been mentioned in the \acrshort*{cbm} context. They were linked to having a decisive influence on the quality of embeddedness in existing systems, acceptance, trust and ultimately its implementation in decision-making and response. In addition to these key success factors, main challenges remain in ethical and privacy considerations, availability of resources and fast response capacities in case of an event as well as community expectation management. Furthermore, \autocite{boetzelaerEvaluationCommunityBased2020} findings indicate, that the additional benefits of \acrshort*{cbs} in already stable settings are limited as the approach is resource intensive. Nevertheless, the increasing application of \acrshort*{cbs} in low-resource or conflict-affected areas, where the full range of benefits were brought to bear. These benefits include CBSs' early warning capabilities and showed promising capacities to address current gaps in health related information and response management, especially in regard to spatial coverage and lower response times \autocite{metugeHumanitarianLedCommunitybased2021, ratnayakePeoplecentredSurveillanceNarrative2020}. \autocite{metugeHumanitarianLedCommunitybased2021} has additionally been able to fruitfully adapt \acrshort*{cbs} for related issues such as displacement and malnutrition and the SRCS is currently using CBS together with the MCS platform NYSS from the Norwegian Red Cross (NRC) in Somalia.\newline

%maybe add figure from here https://www.cbsrc.org/what-is-nyss

% NYSS
NYSS is an open-source implementation following the \acrshort*{mcs} concept and is primarily developed by the \acrshort*{nrc} \autocite{nrcNyssToolDeveloped2022}. The platform allows for high degrees of automatization in regard to data collection, storage, validation and analysis as well as feedback and notification possibilities.\newline
In regard to law, privacy and data security, NYSS servers are located in Ireland and are therefore under European Union data protection law. Besides these law requirements, NYSS has conducted a \acrfull{dpia} in 2020 \autocite{quinnNyssDATAPROTECTION2020} which has generally attested to good standards and made some recommendations for further improvement. Additionally, \autocite{quinnNyssDATAPROTECTION2020} highlighted that the "DPIA is an ongoing process" (p.57) which needs to be conducted regularly which goes in line with the general recommendations for \acrshort{cs} evaluation practices. The \acrshort{nrc} further conducted a recent evaluation of \acrshort{cbs} and NYSS, but the report was not yet published at the time of writing.\newline
While NYSS is developed and operated by the \acrshort{nrc}, the data and most of the operations processing of personal user data is owned, overseen and controlled by the respective National Society. Though, no personal data is collected, stored and processed in NYSS \autocite{nrcNYSSCommunitybasedSurveillance2021}.\newline
The aim of developing NYSS was to provide a simple data collection tool for early warning, rapid reporting and fast response and not for larger data collection endeavours for e.g. the collection of forecast related ground truth data. Therefore, the current \acrshort{cbs} collection and transmission functions via simple SMS and pre-defined codes. Thus, the collection is limited to these codes and their respective meaning. Nonetheless, due to this restriction to simple coded SMS, a normal phone and mobile network are sufficient for data collection. A smartphone and internet connection is thus not necessary. The codes have a specific order and are separated by '#'. In a single report, the code in the context of \acrshort{cbs} consists of 

    \[health risk/event # sex # age\]

where the health risk is represented by one number, sex is either male (1) or female (2) and age is categorized into 0-4 years (1) or 5 years and older (2). Aggregated reports are used in case of higher numbers and represent "a summary of several cases" \autocite[35]{nrcNYSSCommunitybasedSurveillance2021}. Here, the order is decisive for the kind of information and the number represents the actual number of cases. The correctness of the code is automatically checked by the system and a feedback message is sent, also giving advice on how to react in regard to the specific disease, sex and age. The subsequent processing and potential escalation of the report can be seen in figure \ref*{TODO:}.

\missingfigure{https://www.cbsrc.org/what-is-nyss bottom image}

This representation also shows the close involvement in regional processes for response purposes and the implementation of evaluation and supervision processes in the overall structure. A dashboard with map and table views, displaying data collectors and messages facilitates further supervision as well as the fast and simple escalation of warnings to health officials and other organizations through an integration of other organisations \autocite{nrcNYSSCommunitybasedSurveillance2021,nrcWhatNyss2023}. This high level of automation and good integration into existing organisational structures and actor networks enables rapid responses, often in less than 24 hours \autocite{jungCommunityBasedSurveillance2022}.\newline
Technically, NYSS is primarily coded in C# and JavaScript based on a Microsoft Azure storage solution. The SMS are received and via a physical SMS gateway and asynchronously processed by an internal bus communication system. The receiving functions are structurally separated from the reporting functions and connected via internal API requests. The parsing and validation takes place in the internal ReportAPI and the feedback SMS is sent via a data collector forwarding the message to an email-to-sms service which sends the information back to the volunteer \autocite{nrcNyssToolDeveloped2023,nrcNYSSCommunitybasedSurveillance2021,nrcWhatNyss2023}. The source code, together with the documentation, is open source and available on GitHub (\href{https://github.com/nyss-platform-norcross/nyss}{NYSS} and its \href{https://github.com/nyss-platform-norcross/nyss/tree/master/Infrastructure}{infrastructure and architecture} documentation.)

\subsubsection*{Community-based Water Monitoring and Management}\label{subsubsec:cbwm}

\acrfull*{cbwm} is an application example of \acrshort*{cbm} which gained mayor public interest particularly in North America, Europe, Australia and Southeast Asia \autocite{kirschkeCitizenScienceProjects2022, koehlerCitizenParticipationCollaborative2008, livinglakescanadaElevatingCommunityBased2018}. \acrshort*{cbwm} practices range from small monitoring projects to integrated partnerships or councils for the management of watersheds \autocite{westonCommunityBasedWaterMonitoring2015}. Just as for CBM and CBS, participant engagement, data quality control and management, sustainable funding and embedding in existing structures are key to successful integration and implementation of such projects \autocite{allenCommunityBasedWaterMonitoring2018,livinglakescanadaCommunityBasedWaterMonitoring2018,westonCommunityBasedWaterMonitoring2015}.
An overview of primarily water and weather related citizen science projects can be seen in table \ref*{TODO:}. Striking is the already mentioned globally unequal distribution of the projects with a strong emphasis on North American Countries. Furthermore, their focus is mostly on river, lake, groundwater and precipitation levels or focusses on their respective water quality. The technical solutions are mostly not freely available and not open source (FIXME: true? -> and extend). 

\missingfigure{overview table about water related CS projects with CBS/MCS aspects}
% table? table could be worth it.
% project name; topic; region/location; organization; participation level; tools, website/paper

Further noticeable are the technical requirements, which almost always require some sort of smartphone, dedicated measurement equipment or internet access. Only Weeser et al.'s approach is based on simple text messages but were limited in content to a station ID and the indicated stream water level. Here, signs explaining the monitoring and transmission process with pictures and instructions in Swahili and English were placed next to a water level indicator, encouraging passers-by to contribute \autocite{weeserCitizenSciencePioneers2018a}. \autocite[1597]{weeserCitizenSciencePioneers2018a} noted, that method of "transmitting the observations using simple cell phones and text messages turned out to be stable and reliable without major technical problems" in the context of their work in low-income rural areas in Kenya. The problem of occasionally insufficient network coverage was overcome by participants waiting until they reached a network before transmitting, making network availability not a limiting factor in this study. \autocite{wilson-jonesUsingMobilePhones2012} established and evaluated an Android mobile based system to support rural water quality monitoring in South Africa by simplifying connection between managers and operators of municipal test facilities. While all municipalities expressed the system as beneficial exemplifying the usefulness of fast, easy and low resource-intensive communication possibilities in such a context, this project does not necessarily fall within the sphere of \acrshort*{cs}, as the target group here was professional staff. Drawing on their literature review of water quality studies under climate change, \autocite[147]{huangManagementDrinkingWater2020} recommend the application of a "hybrid modality in which community management is the mainstay with supplement from external support" also considering differences in local realities and stakeholder opinions and needs.

One approach to embed \acrshort*{cwbm} into local traditional community water management practices is proposed by \autocite{dayCommunitybasedWaterResources2009}. \autocite{dayCommunitybasedWaterResources2009} argues, that overarching concepts like the \acrfull*{iwrm} remain to large and complex to be manageable and implementable on local levels and additionally often fail to adequately include local stakeholders. Building on the decentralized and locally better opperationalisable version of \acrshort*{iwrm} called 'light IWRM' \autocite{butterworthFindingPracticalApproaches2010,moriartyIntegratedWaterResources2004} and its practical component of Water safety plans (WSP) \autocite{bartramWaterSafetyPlan2009}, \autocite{dayCommunitybasedWaterResources2009} created a community-based water resource management framework (see figure \ref*{TODO:}). 

% framework for the first couple of side goals
\missingfigure{TODO: “Figure 2. Community-based water resource management” ([Day, 2009, p. 59](zotero://select/groups/4773535/items/YWSNQ8A2)) ([pdf](zotero://open-pdf/groups/4773535/items/ETPCI5RI?page=14&annotation=BAMSY255))}

This framework provides the foundation for monitoring by encompassing the specifics of arid regions also with regard to possible drought phases, community needs, risks and water resource assets. Furthermore, the community is seen primarily as a partner rather than a beneficiary, also taking into account internal communal heterogeneity and inequalities making it a good conceptual basis for this works water source monitoring design approach. The usefulness and practical applicability of this framework is presented by \autocite{oxfamIntroductionCommunityBasedWater2009}, as they make this framework the basis of their community-based water resource management implementation guide for field programmes in dryland areas.
Further work for guiding principles in the sphere of \acrshort*{cbwm} are numerous and interested readers are referred to \autocite{westonCommunityBasedWaterMonitoring2015}.

\subsubsection*{Other community-based concepts and initiatives}\label{subsec:cbc} % ???????????? or different name? -> ? Community-Based Disaster Risk Management
% done. Keep it short. Not as important. weeeellll that did not really work out though.

Potential capabilities and areas of application to apply the concept of \acrshort*{cbm} and \acrshort*{mcs} are wide-ranging and numerous. Besides health- and water related domains, Community-based Disaster Risk Reduction (CBDRR), Disaster Risk Management (CBDRM) and Early Warning Systems (CBEWS) / information dissemination are rising fields of application. While health and water-related projects can be part of the broader CBDRR or CBDRM approach, depending on their focus, many projects about CBDRR, CBDRM and CBEWS focus on natural disasters such as droughts, fires, typhoons, (flash) floods, and landslides \autocite{machereraReviewStudiesCommunity2016,manaloBellBottleTechnology2013,pinedaRedefiningCommunityBased2015,smithCommunitybasedEarlyWarning2017,tarchianiCommunityImpactBased2020,trogrlicIndigenousKnowledgeEarly2018,vhumbunuCountingDayZero2021}. Based on \autocite{unisdrUNISDRTerminologyDisaster2009}, \autocite[198]{vhumbunuCountingDayZero2021} defines CBDRM as "the involvement of potentially affected communities in disaster risk management at the local level by building their capacities to assess their vulnerability to natural disasters and develop strategies necessary to mitigate the impact of these disasters" and further states, that "at the core of these concepts is the involvement of communities in making decisions and implementing disaster risk management strategies, actions, and initiatives".
Examples for participatory Disaster Management Software are large and multi-purpose platforms like Ushahidi, Sahana Eden and Kobo \autocite{koboorganizationKoboToolbox,sahanafoundationSahanaEDEN2016,ushahidiCrowdsourcingSolutionsEmpower}. A smaller but dedicated approach to bridge indigenous knowledge and modern science by disseminating early drought information and warnings is the framework ITIKI (Information Technology and Indigenous Knowledge with Intelligence) \autocite{akanbiDevelopmentRuleBasedDrought2018,masindeEffectiveDroughtEarly2014a,masindeImplementationRoadmapDownscaling2013,masindeDownscalingAfricaDrought2018,masindeFrameworkPredictingDroughts2010a,masindeITIKIBridgeAfrican2012,masindeITIKIMobileBased2019,nyetanyaneIntegrationIndigenousKnowledge2020,thothelaSurveyIntelligentAgroclimate2021a}. This system integrates scientific and indigenous drought forecasts by combining local and expert knowledge, technical components like wireless sensors, mobile phones and artificial intelligence analysis capacities to provide micro-level forecasts to local farmers and communities. Positive effects of local drought forecast dissemination could also be confirmed by \autocite{anderssonLocalEarlyWarning2020}'s (FIXME:) study while also mentioning, that local capacities or pre-conditions often limited a positive respond to the early warning.
(FIXME: check if this is correct)\autocite{gladfelterPoliticsParticipationCommunitybased2018,inayathEARLYWARNINGSYSTEM2018,trogrlicIndigenousKnowledgeEarly2018} highlight the importance to tailor the information to the needs, capacities and social structures of communities on the ground to enable their successful implementation. Accounting for community heterogeneity is also emphasized by \autocite{gladfelterPoliticsParticipationCommunitybased2018} as people may be incapable to respond to early warnings due to a lack or resources or knowledge. In addition, \autocite[21]{inayathEARLYWARNINGSYSTEM2018} advocates that early warning messages should be "simple, timely, and encourage early action" to enable an appropriate response in the first place.
Another problem in implementing participatory early warning systems is the gap between classical top-down approaches and community-based bottom-up initiatives. Successfully bridging the gap between these two approaches by directly coordinating available technical capacities through a participatory approach is possible according to \autocite{tarchianiCommunityImpactBased2020}. This is supported by \autocite{henriksenParticipatoryEarlyWarning2018} findings, that bottom-up approaches in contrast to classical concepts better facilitate the integration of local stakeholders in processes of decision-making and risk management.Generally, \autocite{marcheziniReviewStudiesParticipatory2018} literature review indicates a shortage of research in regard to citizen science and CBEWS and \autocite{baudoinEarlyWarningSystems2014} additionally notes the need to significantly improve the design and application of early warning systems. \autocite{baudoinEarlyWarningSystems2014} advocates for an integrated cross-scale approach ensuring the involvement of the at-risk population at all stages of the management process. Further arguing for "early warning systems that are both technically systematic and people-centred" \autocite[15]{baudoinEarlyWarningSystems2014}.\newline
The CBS approach together with the \acrshort{mcs} NYSS application has thus shown that \acrshort{cbm} and \acrshort{mcs} can be successfully applied to the local context and adapted to other topics as shown by the \acrshort{cbwm} and CBDRR approaches. More on the regional implementation in sections \ref*{subsec:case_eap} and \ref*{subsec:stage5_appl}.

% further classified literature reviews in the realm of geophysics: Zheng et al. 2018 p.717 Table 2 (method classification) + 3 (management) + 4 (quality assurance) + 5 (processing) + 6 (data privacy)

%----------------------------------------------------------------------------------------
%	SECTION 8 Case Study Area (+ application of the rest)
%----------------------------------------------------------------------------------------


\section{Case Study Area (+ application of the rest)}\label{sec:case_area} % maybe put this in the beginning of this all?

Northern Somalia, also known as Somaliland, is a region located in the Horn of Africa. Officially referred to as the Republic of Somaliland, it is a self-declared independent, de facto sovereign state, but it is not recognised internationally and is still considered part of Somalia. Somaliland is bordered by the Gulf of Aden to the north, Somalia to the east, the Federal Republic of Ethiopia to the south and west, and the Republic of Djibouti to the northwest. The claimed region encompasses around 177,000 km$^2$ and has an estimated population size between 4.2 to 5.5 million people, depending on the source \autocite{petrucciLandscapeLandformsNorthern2022,republicofsomaliaCountryProfile20212021,somaliredcrescentsocietyFeasibilityStudyPotential2022}. Administratively, Somaliland according to international standards, Somaliland is divided into 5 regions, from east to west and north to south, Awdal, Woqooyi Galbeed, Todgheer, Sanaag and Sool with the capital being Hargeisa in Woqooyi Galbeed (see figure \ref*{TODO: create map of Somaliland with regions}) \autocite{republicofsomaliaCountryProfile20212021}. Somaliland's own constitution divides the country into 6 regions where Woqooyi Galbeed is further divided into Maroodijeex (Hargeisa region) and Sahil \autocite{republicofsomalilandRegionsDistrictsSelfmanagement2019}.

\missingfigure{Somaliland figure with all the fancy stuff plllllls}

This chapter will give a brief overview about the geography, economy and social conditions. It will place the above concepts in the context of past and present local conditions and elaborate on current work on early warning concepts and projects. 


\subsection{Geography \& Climate}

The geography of this region is marked by its arid and semi-arid conditions, with a diverse range of physical and environmental features that define its landscape. Topographically, Somaliland can be divided into three main zones: the coastal plain Guban, the mountain range Oogo and the plateau Hawd \autocite{republicofsomaliaCountryProfile20212021}. The Guban (Somali for 'the burnt') area is a very hot and arid region averaging less than 100\,mm rainfall per year with potential evapotranspiration exceeding rainfall by thirty times \autocite{salemTerritorialDiagnosticReport2016}. Furthermore, it is not unusual to have no rain at all for 2-3 consecutive years. The Oogo mountain ranges receive up to 500-600\,mm of rainfall annually with equal evapotranspiration potential, and annual mean temperatures of 20-24\,°C, with peaks rarely exceeding 35\,°C. Temperature conditions on the Hawd plateaus are comparable, but precipitation can be lower and the potential evapotranspiration is at a factor of about 1.5 \autocite{abdulkadirAssessmentDroughtRecurrence2017,salemTerritorialDiagnosticReport2016}. 
Somaliland's climate is typically arid to semi-arid and experiences four distinct seasons. The primary rainy season, known as Gu', takes place from April to June and contributes to about 50-60\,\% of the annual precipitation. The secondary rainy season, called Dayr, lasts from August to November and accounts for approximately 20-30\,\% of the total rainfall. The remaining two seasons are Jiilaal and Xagaa, which occur from December to March and July to August, respectively, and are characterized by dry conditions \autocite{abdulkadirAssessmentDroughtRecurrence2017,republicofsomaliaCountryProfile20212021}.
A detailed description of the geological features of Somaliland, together with many pictorial impressions can be found in \autocite{petrucciLandscapeLandformsNorthern2022}. The soil types in Somaliland are closely linked to its geomorphology and are typically marked by poor structure, high permeability, low capacity to retain moisture, and insufficient internal drainage \autocite{salemTerritorialDiagnosticReport2016}. The naturally sparse vegetation, tree cutting and overgrazing also lead to accelerated soil erosion \autocite{salemTerritorialDiagnosticReport2016}. Nomadic and transhumance pastoralism activities influence around 90\,\%, and agro-pastoralism about 2\,\% of the land with often adverse environmental effects \autocite{salemTerritorialDiagnosticReport2016}. Besides poor soils, high levels of erosion and a challenging climate, little water resources stress the local fauna, flora and human population. 

\subsection{Water Sources}\label{subsec:water_sources}

Often insufficient knowledge about hydrogeological conditions and access depths of more than 100\,m caused a very limited number of boreholes, approximately 300 in Somaliland \autocite{faoswalimHydrogeologicalSurveyAssessment2012, petrucciLandscapeLandformsNorthern2022, salemTerritorialDiagnosticReport2016}. As there are no permanent rivers in Somaliland, the use of surface water is primarily based on water retention structures storing part of the water supply beyond the rainy season \autocite{petrucciLandscapeLandformsNorthern2022}. Wide and open structures called \textit{balleys} can store large volumes of water, but do not last as long as \textit{berkads}. % FIXME: is the borehole number correct? --> at least strong disagreement between sources!
Traditional berkads are commonly 3 to 4 meters deep, 7 to 9 meters wide and 10 to 13 meters in length. Build materials are commonly stones and clay and some are covered with organic materials such as sticks and bushes. Berkads are generally constructed in clusters and usually built on a slope to collect water during the rainy season, but are sometimes filled by man-made canals with or without impurity collection facilities \autocite{walkerChangingPastoralismEthiopian1998}. These missing mechanisms during the filling process can result in contamination of the water with organic matter, animal or human faeces etc. \autocite{mercycorpsIMPROVEDBERKADDESIGNS2017}. The same lack of separation of animals and humans can also lead to contamination when water is extracted. Improved designs exist and more sophisticated versions nowadays use concrete, are properly roofed to counteract evaporation and have adequate inflow and extraction mechanisms to prevent contamination \autocite{mercycorpsIMPROVEDBERKADDESIGNS2017, petrucciLandscapeLandformsNorthern2022}. Following \autocite{mercycorpsIMPROVEDBERKADDESIGNS2017} calculations, an improved berkad needs to have a volume of about 1000 to 1200\,cubic meters to  withstand a 3 month dry period with a monthly extraction of 288\,m$^3$. This amount would serve 240 persons (20l/day/person), 150 camels (12l/day/camel) and approximately 2000 (1.5l/day/animal) sheep and goats. Currently valid total number of Berkads for Somaliland do not exist but \autocite{walkerChangingPastoralismEthiopian1998} estimated about 12.000 berkads clustered in 126 groups in the ethiopian district in Gashaamo, which borders Somaliland in the south. \autocite{birchWeUsedSing2008} notes 7000 berkads for the Hawd region, although with an unknown number of non-operational berkads. The sheer number and reliance of pastoralists and communities on berkads mentioned by \autocite{walkerChangingPastoralismEthiopian1998} and \autocite{birchSomalilandSomaliRegion2008} illustrate their importance. Besides boreholes and berkads, shallow wells, springs and dams are types of water sources. Available datasets about all water sources but especially berkads, concerning e.g. their location, functionality, status of ownership and other factors are limited, mostly outdated and unknown in quality \autocite{FAOSWALIMSomalia}. % is that correct? --> sie widersprechen sich vor allem alle (?) -> analyse muss ich mal machen.

% Current stuff about water sources (?) -> or in the results (?) but I would need to analyze that stuff thouuuuuugh. maybe if I have time in the end?
% https://spatial.faoswalim.org/layers/geonode:Strategic_Water_Sources_20225#/

% IF TIME: analyse SWALIM sources and create map. --> way more and more functioning but old and unreliable https://swims.faoswalim.org/livemap/view
% differences in 2018, 2020, 2022

% https://climseries.faoswalim.org/station/
% only one groundwater station

% mindmap of locals: -> difficult to compare as it does not cover the same region.
% “water points” ([“Changing Pastoralism in Region 5”, p. 8](zotero://select/groups/4773535/items/FXJGUTLD)) ([pdf](zotero://open-pdf/groups/4773535/items/BIAA5M57?page=8&annotation=J48Y7AG5))

\subsection{Political, social and economic circumstances} % + history?

After being ruled by the Ottoman Empire and subsequent British colonisation, Somaliland gained independence on 26th, June 1960. A few days later Somaliland voluntarily merged with Italian Somalia to form the Somali Republic. From 1969 until 1991, Somali Republic was controlled by a military junta, led by Siyad Barre which from a supremacy of the southern part cruelly and partly arbitrarily suppressed the northern one, Somaliland. Arrests, mining of water points and executions culminated in the genocide of thousands of members of the largest clan, the Isaaq tribe \autocite{peiferStoppingMassKillings2009,republicofsomaliaCountryProfile20212021}. Since the collapse of the Siad Barre regime in 1991, Somaliland has developed into one of the most politically stable democracies in the Horn of Africa, but challenged in recent times due to the postponement of elections \autocite{bbcSomalilandProfile2022, fortiPocketStabilityUnderstanding2011}. Though, internel conflicts and border disputes with Puntland in the east continue until today \autocite{filhoDEMOCRACYAFRICAOUTSTANDING2021}. Nowadays, Somaliland is a presidential republic, combining its traditional clan culture with modern democratic elements and structures of the House of Representatives and Elders \autocite{salemTerritorialDiagnosticReport2016}.
Somaliland has a GDP of approx. 1.5-2\$ billion, mostly based on remittances from Somalilanders working abroad and main export being livestock, per capita income is only in the hundreds of dollars \autocite{klobucistaSomalilandHornAfrica2018, republicofsomaliaCountryProfile20212021, worldbankNewWorldBank2014}. Low literacy rates (~48\% for adults above 15), a ~35\% secondary school education completion rate and high unemployment rates further complicate the situation \autocite{republicofsomaliaCountryProfile20212021,worldbankNewWorldBank2014}. Due to its reliance on pastoralism and livestock for mayor parts of its economy and food security, Somaliland is prone to natural disasters \autocite{usaidcenterforresilienceEconomicsResilienceDrought2018}.


%-----------------------------------
%	SUBSECTION 8.4
%-----------------------------------

\subsection{Hazards and risks}

Drought, flash floods, land degradation and conflict all pose risks to Somaliland's environment and society, with droughts posing the greatest threat in recent centuries \autocite{abdulkadirAssessmentDroughtRecurrence2017}. Several historical and current analyses and predictions indicate, that these phenomena will not get less but possibly intensify and become more frequent driven by large phenomena like the El Niño-Southern Oscillation and rising Sea Surface Temperatures (SST) \autocite{abdulkadirAssessmentDroughtRecurrence2017,aliMitigatingNaturalDisasters2017a, balintMonitoringDroughtCombined2013, erianGARSpecialReport2021, FAOSWALIMSomalia, museiSPEIbasedSpatialTemporal2021, nationaldroughtcommitteeSOMALILANDDROUGHTRAPID2022,trisosAfrica2022}. Population growth, deforestation and desertification, groundwater depletion and land grabbing further stresses the situation \autocite{aliMitigatingNaturalDisasters2017a}. While a rough tendency can be derived from such predictions, \autocite[10]{abdulkadirAssessmentDroughtRecurrence2017} findings indicate, that the forecast quality of global climate model simulations "show varying results and therefore remain uncertain for Somaliland". 
Geographically, the eastern regions Sanaag, Sool and Todgheer are historically the most severely impacted ones \autocite{abdulkadirAssessmentDroughtRecurrence2017, FAOSWALIMSomalia}. In the period since 1960, Somaliland experienced 17 major droughts with the most intense and widespread droughts in 1973-1974, 1984, 1991, 2010/2011, 2016/2017 and 2021 until today \autocite{abdulkadirAssessmentDroughtRecurrence2017, credEMDATInternationalDisasters2023}. The worst drought in 2010-2012 led to a famine, where more than 200.000 people died and over 2.6 million people were affected all over Somalia \autocite{srcsDRMStrategicPlan2021}.
Currently, the almost complete failures of five successive rainfall seasons, rising food prices and severe water shortages are adding up to another stressful situation putting 810.000 people in need of emergency assistance \autocite{nationaldroughtcommitteeSOMALILANDDROUGHTRAPID2022}. This number is projected to rise substantially if the current drought conditions persist \autocite{swansonNearlyMillionPeople2022}. Shallow wells and most Berkeds have dried up, leaving boreholes and expensive water trucking as the last options for water supply \autocite{nationaldroughtcommitteeSOMALILANDDROUGHTRAPID2022}.
Cascading droughts can have cascading impacts as affected people are forced into bad feedback-loops to respond to the immediate crisis, reducing their coping capacity and thus further increasing their vulnerability to future events \autocite{usaidEconomicsResilienceDrought2018}. \autocite{usaidEconomicsResilienceDrought2018} hypothesised, that these post-shock impacts can better be mitigated by early interventions than by late response. Although, \autocite{usaidEconomicsResilienceDrought2018} states, that there is very little data to support this statement and that it is primarily based upon logical deduction and not field data. Nonetheless, this assumption is also supported by \autocite{aliMitigatingNaturalDisasters2017a}, \autocite{abdulkadirAssessmentDroughtRecurrence2017} as well as by the growing Forecast based Financing practitioners \autocite{gualazziniEWEAEarlyWarning2021, harrowsmithFutureForecastImpact2020}

\subsection{EAP + forecast, trigger and so on}\label{subsec:case_eap}

The 2011 famine in Somalia was projected 11 month in advance. Despite this early warning, the international community failed to react adequately and in time to prevent the worst \autocite{elisabethstephensFORECASTBASEDACTION2015, hillbrunerWhenEarlyWarning2012}. Subsequent evaluations point to two main areas of concern. On the one hand, there was a lack of timely funding, but on the other hand, the concept of preventive action had not yet permeated the humanitarian community and response activities were still seen as the standard \autocite{elisabethstephensFORECASTBASEDACTION2015}. This failure, as well as the successive improvements in forecasting and the growing scientific interest and knowledge about the positive impact of early warning and anticipation measures, laid the foundation for the current development of the EAP for Somalia. As the project is still in progress, detailed information is not yet possible to present in all areas and the presented information is also subject to constant changes and future developments. Nevertheless, critical points for this work can be derived and the need for further developments can be elaborated.

The interest to develop an \acrshort*{eap} for a slow-onset hazard such as drought only recently started to become more popular within the RCRC as the focus laid on fast-onset disasters thus far \autocite{rcrcFORECASTBASEDFINANCINGEARLY2020}. \autocite{rcrcFORECASTBASEDFINANCINGEARLY2020} presented the first adaptation of the general manual of the \acrshort{ifrc} (see \autocite{ifrcFbFPractitionersManual2023b}), merging experiences of pilot projects to adjusted guidelines for the development of FbF and early actions in the context of drought. Currently, at least seven National Societies (Kenya, Uganda, Ethiopia, Zimbabwe, Somalia, Lesotho and Niger) are planning, developing or have recently completed a drought \acrshort*{eap} \autocite{lesothoredcrosssocietyEARLYACTIONPROTOCOL2022,nigerredcrosssocietyNigerDroughtEarly2021,rcrcFORECASTBASEDFINANCINGEARLY2020}.
The \acrfull*{srcs} has completed their preliminary \textit{Feasibility Study on Potential Use of Forecast-based Financing (FbF)} in June 2022. A pilot study shall be conducted to test practical implementation feasibility in Somaliland and potentially Puntland with emphasis on, from highest to lowest priority: droughts, health, (flash) floods, cyclones, locusts, and conflicts. Besides the detailed description and justification for each type of disaster, the assessment also confirmed the well positioning of the \acrshort*{srcs} to undertake such a FbF program and to embed it into the general \acrlong*{drm}.
The implementation of a FbF program cannot be done by a National Society alone. Besides the \acrshort*{srcs} numerous other stakeholders will take part in providing information, resources or knowledge as well as acting upon aforementioned. The landscape of actors is wide and includes many local, regional, national and international governmental and non-governmental groups, initiatives, centers and organisations. To name but a few: The Ministries of Agriculture (MoA), of Water Resources (MoWR), of Health Development (MoHD) and of Humanitarian Affairs and Disaster Management (HADMA) and others include Somaliland's state actors. (TODO:)\acrfull{brcis} and \acrfull{cdrmc} compromise local and regional NGO networks and committees. The UN (\acrshort{fao}, \acrshort{ocha}, \acrshort{undrr}, \acrshort{wfp}, \acrshort{who}, World Bank, \acrshort{wmo}, \acrshort{grc}, \acrshort{nrc} and \acrshort{ifrc} are a selection of international actors engaged in Somalia. Added to this are a number of other think tanks, climate centres and forecasting providers, making the integration of the respective actors an important but also intricate affair, especially in the light of the multi-faceted nature of droughts\autocite{rcrcFORECASTBASEDFINANCINGEARLY2020,somaliredcrescentsocietyFeasibilityStudyPotential2022}.

Forecasts are also provided by various organisations and scales. The \acrshort*{fewsnet} releases famine warnings and reports for the entire african continent on a regular basis \autocite{fewsnetFamineEarlyWarning2023}. Regional forecasts are provided by the Climate Predictions and Applications Centre (ICPAC) based on global models for the Greater Horn of Africa region \autocite{icpacDeliveringClimateServices2023}. More small-scale prognoses are released from FAO's \acrshort*{swalim} and \acrshort*{fsnau} programs which monitor different drought indicators based on relatively few weather stations (100 manual and 10 automatic in all of Somalia) and remotely gathered and modelled climate information \autocite{faoswalimSWALIMWeatherMonitoring2014,somaliredcrescentsocietyFeasibilityStudyPotential2022}. There are two other local seasonal forecasts issued by government agencies and disseminated by the responsible agency, \acrshort*{nadfor} to stakeholders at all levels for natural hazard warnings \autocite{somaliredcrescentsocietyFeasibilityStudyPotential2022}. Besides SRCS's own disease \acrshort*{cbs} informing actions for health related issues, data of local circumstances influence forecasts only scarcely and infrequently.

Up to this point, it has not yet been decided which prediction and reaction trigger should be chosen for the SRCSs' \acrshort*{eap} but it will inevitably be based on scarce coverage and primarily large scale data, as it is the case for the \acrshortpl*{eap} in Niger and Lesotho \autocite{lesothoredcrosssocietyEARLYACTIONPROTOCOL2022,nigerredcrosssocietyNigerDroughtEarly2021}.

The trigger methodology will be a staggered trigger, following current recommendations of the \autocite{rcrcFORECASTBASEDFINANCINGEARLY2020} but its definition remains a challenge due to the currently very tense situation and the medium-term changes in weather and climate over the last 10 years. Under these conditions, it is quite difficult to determine a \textit{normal} period against which \textit{drought events} can be measured and will ultimately depend on the chosen forecast. Conceivable triggers could be the predicted failure of one or more consecutive rainy seasons or a specific classification warning for food or water insecurity but will also depend on selected actions. \autocite[19]{gettliffeOCHAAnticipatoryAction2021} found, that triggers need to be linked to their respective intervention, or otherwise will "led to significant challenges".
Identified actions by the feasibility study for the \acrshort*{eap} for drought interventions are water storage rehabilitation, de-stocking, early or alternative short growth crop planting, cash distributions, women and children shelters as well as water trucking \autocite{somaliredcrescentsocietyFeasibilityStudyPotential2022}. The Ministry of Livestock and Fisheries Development notes, that de-stocking will hardly be feasible due to little trust in forecasts by livestock owners as well as no internationally approved abattoir which limits the amount to local market capacities \autocite{somaliredcrescentsocietyFeasibilityStudyPotential2022}. \autocite{gualazziniEWEAEarlyWarning2021} propose water voucher as viable alternative to water trucking in regions where a functional market of private water vendors already exists. Besides \acrshortpl{aa}, adequate policies for water management, price regulations, and allocation mechanisms are seen as potential opportunities to mitigate further drought impacts \autocite{gualazziniEWEAEarlyWarning2021,wangPropagationDroughtMeteorological2016}.
Besides the mentioned forecasts of natural phenomena, SRCS has successfully set up a \acrshort*{cbs} project developing and utilizing the platform NYSS to monitor and react to disease outbreaks on community level since 2018 \autocite{jungCommunityBasedSurveillance2022}. 

Besides \acrshort*{srcs} and \acrshort*{ifrc}, \acrshort*{ocha} and \acrshort*{brcis} also developed anticipatory action plans for Somalia in recent years. \acrshort*{ocha} followed with their pilot study in 2020 conventional frameworks in regard to forecasts and triggers, using large scale indices with a combined trigger of pre-identified thresholds \autocite{gettliffeOCHAAnticipatoryAction2021,ochaANTICIPATORYACTIONPLAN2020}. Chosen actions comprise all major fields of food security, WASH, education, health and risk communication, often with lead times of multiple weeks to months. In their evaluation, \autocite{gettliffeOCHAAnticipatoryAction2021} synthesized many lessons learned in all areas. Highlighting the buy-in of all stakeholders, early expectation setting, the importance for parallel development of \acrshortpl{aa} together with explicit, linked and robust trigger mechanisms. Cash transfers were "identified across several clusters as the preferred action" where local markets and the operational context allow \autocites[21]{gettliffeOCHAAnticipatoryAction2021,ochaANTICIPATORYACTIONPLAN2020}.

\acrshort*{brcis} created their own \acrfull{crtrms} to integrate local information. The \acrshort*{crtrms} is based on key informants from a selection of a small group of 2-3 communities which represent a larger population of 10-12 communities. These information are then triangulated with regional, national and international secondary information sources to ultimately propose relevant anticipatory measures. The survey together with the triangulation should allow triggering within 12 days after data collection but commonly averages on 25 days in practice. Besides the relatively long duration, key informants are well aware, that their given information may influence the amount of humanitarian assistance in the area, highlighting the importance of trustbuilding and data triangulation\autocite{gualazziniEWEAEarlyWarning2021}. Indicators and thresholds are categorized into \textit{normal}, \textit{alert}, and \textit{alarm} allowing for \textit{red-flagging} of areas based on either one very strong impact or on a pre-defined amount of cumulative impacts in multiple areas. For example, one indicator is the condition of primary water sources in communities. These are assessed at the end of rainy season and categorized based on their water level into normal \textit{(more than half-full [75\%] or full)}, alert \textit{(half-full [50\%])} or alert \textit{(less than half-full [25\%] or empty)} which allows for a seasonal prediction and corresponding flagging.

%----------------------------------------------------------------------------------------
%	SECTION 9 Conclusion literature
%----------------------------------------------------------------------------------------

% alrighty bitches. lets fking go!

% maybe.. just maybe take this one for the discussion and replace it with a short summary.. I think, a summary might be the better idea at this point anyways


\section{Summary Literature}
% integrate NYSS

This chapter outlined the overall theoretical background of the case studies context by starting with wide ranging and complex concepts such as Water Security, Water Scarcity, Drought and their respective indicators and indices subsequently narrowing them down to the actual case study area and the problem at hand.\newline
Relatively new approaches to mitigate, instead of focus on post-disaster response was described in the concept of \acrlong{fbf} and respective sub-parts. The \acrshort{fbf} approach is based on impact forecasts which predict what the weather will do, instead of conventional forecasts that predict what the weather will be. Based on this knowledge, protocols can be developed which specify the exact threshold to trigger corresponding \acrlongpl{aa} to counteract impact of the disaster. To facilitate this, the knowledge needs to be highly local and relevant to the specific action. \acrlong{cs}, together with its sub-divisions of \acrlong{cbm} and \acrlong{mcs} was introduced and practical examples with \acrlong{cbs} and \acrlong{cbwm} further exemplified its area of application. In this context, several other \acrshort{cs} projects were identified and their characteristics elaborated.\newline
In the last section, the previously presented concepts were broken down to the concrete case study area. The case study area was described in detail in terms of its physical, social, political and economic conditions and the development of the EAP development overlying this project was delineated. % The literature and project analysis suggested the need to adjust and potentially develop a new framework to account for the case study specifics, as no suitable framework could be identified.  




% The concepts of Water Security, Water Scarcity and Drought are wide ranging, complex in nature and various definitions exist for each of these concepts. Water Security links an extensive network of interrelated trans- and inter-sectoral systems together and can be seen as umbrella term for the extensive web around water availability and its many components. Water Scarcity, as it has been defined in this work, has a physical and economic aspect which refer to the availability of water resources and the various capital conditions for its extraction, respectively \autocite{faoCopingWaterScarcity2012}. Therefore, the absolute lack of water in the current situation, water shortage, always has a human (long-term) dimension, particularly on the demand side. Drought is most often considered in four stages, namely meteorological, agricultural, hydrogeological and socioeconomic drought with each having their specific sets of impacts, indicators and indices. Drought, as natural hazard, is differentiated to aridity by its short but severe nature. Conceptual definitions give an idea and set boundaries to the concept of drought while operational definitions focus on its onset, duration, severity and spatial coverage.
% Measuring and predicting droughts is complex and often multiple individual or sets of indicators and indices are combined to get a full picture of the situation. Indices are themselves often composites of multiple indicators and mathematical functions. The intricate and interconnected nature of droughts leads to uncertainties in forecasts making it more difficult to define impact thresholds for anticipatory measures. Impact is generally expressed by a combination of hazard severity, the exposure of assets and their vulnerability. The latter often summarized in the term risk.
% Forecast based Financing is a relatively newly emerged phenomena in the realm of humanitarian aid that promotes anticipatory action before the impact. This is based on a hazard and risk assessment to identify thresholds on which to trigger pre-defined actions that reduce the impact, documented and defined in an Early Action Protocol. For successful implementation, triggers and actions should be developed and directly linked. This is often not feasible as local information data is either missing completely or is outdated.
% Citizen Science is the integration of citizens in scientific or public endeavours projects and can promote various benefits to all engaged stakeholders if implemented and operated correctly. Citizen Science, similar to the above concepts is wide-ranging and complex. Under this umbrella, \acrlong*{cbs} together with \acrlong*{mcs} provide practically realizable frameworks and guidelines for the successful application of Citizen Science. \acrlong*{cbs} and \acrlong{cbwm} demonstrate the feasibility and transferability of these concepts.
% Somaliland lies within the Great Horn of Africa and can geographically and climatologically be seen as a generally arid and water scarce region with poor soils, scarce vegetation and water resources. It has historically been troubled by droughts, as well as internal and external conflicts, which regularly exacerbate the already tense situation. Somaliland is one of the poorest regions in the world but managed to develop a relatively stable democracy for the last 30 years. Currently many local, national and international organization work on mitigating and responding to a further tense situation with famine expected in mid-2023. 

% Certain limitations and gaps could be identified in the above conducted literature review. The concepts of Water Security, Water Scarcity and Drought need to be clearly defined and broken down to the specific region and application. Unfortunately, the long history of conflicts and insecurities severely limited scientific research in this region in particular, which generally led to little scholarly information on the region.
% While numerous international forecasts exist and local assessments start to emerge, timely, highly local and up-to-date data is not available for most areas of interest. Thus, the direct link of trigger and anticipatory action is often not given, making the implementation less effective, efficient, targeted and it takes more time. 
% The concept of FbF is now generally well established, but the drought use case is new and not yet well researched, which severely limits the amount of guidelines and frameworks available for this particular application. Thus, each new project and study has, at least in part, an exploratory character.
% Citizen Science projects in regard to water are geographically primarily focused on North American and European countries, relating most of the scientific findings to the respective context. Furthermore, these water related projects mainly focus on river, lake or groundwater level or water quality monitoring and not on direct community water source investigations to facilitate early actions. The review of data availability and reliability of current datasets revealed a clear need for more up-to-date and complete information on water source locations and characteristics, especially with regard to the highly important Berkad water type. 

% Embedded in the above broad concepts and reasoned by the tense context, scarce data situation and current EAP development, this work will partly address found gaps in literature by building on the successful methodology of CBS for issues of health and the thematic insights in water related subjects from other projects presented above. This work will thus find new ways of combining the proven CBS approach with MCS on the topic of direct community water source monitoring for the creation of a highly local and up to date water level threshold to trigger directly related anticipatory actions in a resource scarce environment.
% The community-based monitoring and management concept has, to the best of my knowledge, not been carried out once nor scientifically investigated in regard to directly monitoring and managing of rural community water sources, especially not in resource scarce environments like Somaliland.

% adjust
% "To the best of our knowledge, no previous studies have utilized [insert approach or method] to investigate [insert research question]. This is an important gap in the literature, as [explain why this is important]. Therefore, this study aims to fill this gap by applying [insert approach or method] to [insert research question]."
% % Chapter Template

\chapter{Methodology} % Main chapter title

\label{chap3:methodology} % Change X to a consecutive number; for referencing this chapter elsewhere, use \ref{ChapterX}

%----------------------------------------------------------------------------------------
%	SECTION 1
%----------------------------------------------------------------------------------------

% Intro
Based on the philosophical ideas of interpretism and post-positivism, this chapter will present the methodological framework this work utilised. Embedded in an inductive design type of an exploratory, iterative case study, a mixed-method approach with data and document analysis as well as expert interviews was adopted. In case of the interviews, non-probability sampling together with the snowball approach was applied. The transcribed interviews were subsequently coded facilitating an open thematic coding strategy. The final design was guided by the 6-stage-design for \acrlong*{cs} projects in ecological science by \autocite{fraislCitizenScienceEnvironmental2022} and further deepened by a more social concept, the Seven-layer Model of Collaboration by \autocite{briggsSevenLayerModelCollaboration}. Moreover, multiple other guidelines for the creation of a \acrshort*{cs} program were consolidated and their recommendations taken into account.\newline
In the following, each of the above mentioned concepts is presented, reasoned and embedded in the overall context of the research. The aim and context of this research, the attempt to design a roadmap for a \acrshort*{cs} project for participatory and remote water source mapping and monitoring in a resource scarce environment in Somaliland, faced several challenges and constraints. These constraints are also highlighted along with other important methodological strengths and limitations.

\section*{Research Design} % is that the correct heading?

% Philosophy (interpretism and post-positivism) -> Research Philosophy
Positivism, historically emerged as a combination of rationalism and empiricism by french philosopher Auguste Comte, highlights the objectivity of knowledge and emphasizes the fundamental need for verification through observations. Positivism was, and often still is, mainly linked to quantitative research methods such as experiments and surveys, highlighting the independent and objective nature of scientific research. In contrast, interpretism (sometimes also called anti-positivism), often equated with qualitative methods such as participant observations and unstructured interviews, argues that objectivity is largely impossible and all knowledge is subjective in nature. Both of these more extreme approaches were increasingly criticised at the end of the 20th century. Emerging concepts like post-positivism, for example, acknowledges biases and discusses the idea that while truth cannot be objectively proven, false claims can be rejected \autocite{pelzResearchMethodsSocial,trochimResearchMethodsKnowledge2001}.\linebreak[1]
This work, on the one hand, relied on a deductive, quantitative analyses and theory testing of conducted programs, projects and literature to synthesise best practices and guidelines. On the other hand, it also applied interpretive, inductive methods like interviews to gain new expert knowledge about the context and realities of the case study to build new theory. Therefore, a combination of both approaches will be reflected in a mixed-method approach which benefits of both, quantitative and qualitative data.

% Research Type
The research type that "allows for in-depth, multi-faceted explorations of complex issues in their real-life settings" \autocite{croweCaseStudyApproach2011} is the \textit{case research} or often just called \texttit{case study}. This definition goes back to \autocite{yinCaseStudyResearch1984} and highlights the core strengths of this approach. The research type of the case study is particularly well suited for exploratory, rather than for descriptive or explanatory research, closely examining circumstances within a specific geographical area and context \autocite{zainalCaseStudyResearch2007}. Here, various quantitative and qualitative forms from both historical and real time data can be investigated and examined directly in its given context \autocite{fitzgeraldCaseStudiesResearch1999}. This allows for a detailed, multi-perspective and specific investigation of that particular topic of interest, which might not be given when examining the parts individually \autocite{pelzResearchMethodsSocial, zainalCaseStudyResearch2007}. Yet, case studies also comes with a number of trade-offs.\linebreak[1]
More extreme critics see the case study method only as a loose 'story', which in the worse case is even connected to the scientist himself \autocite{fitzgeraldCaseStudiesResearch1999}. This criticism refers to the lack of rigour and "very little basis for scientific generalisation" which can lead to low external validity \autocites{yinCaseStudyResearch1984}[5]{zainalCaseStudyResearch2007}. The internal validity often remains weak due to no or poor experimental control complicating causal relationship testing and the multifaceted nature of case studies make it dependent on the researcher and prone to bias and some kind of subjectivity. Besides the internal and external validity, constructed validity and reliability need to be accounted for. Constructed validity refers to "the extent to which a study investigates what it claims to investigate" \autocite[3]{gibbertWhatPassesRigorous2008} so that "the researcher can correctly evaluate the studied concepts" \autocite[277]{ferreiraHowImproveValidity2020}. It can be addressed by establishing a clear chain of evidence and triangulation of perspectives and sources \autocite{gibbertWhatPassesRigorous2008}. Repeatability by other scientists using the same methodology to arrive at the same insights is termed 'reliability'. Repeatability refers therefore to the "absence of random error" and can be enhanced by clear procedures and good documentation \autocite[5]{gibbertWhatPassesRigorous2008}.\linebreak[1]
Furthermore, case studies are frequently criticized for being excessively long, challenging to execute, and requiring significant documentation efforts \autocite{yinCaseStudyResearch1984}. Alternative research types such as experimental research, desk or field surveys and survey research also all have their inherent advantages and limitations. For example, experimental research often deductively examines cause-effect relationships in an isolated context, making it strong in internal validity but low in external validity as the often artificial isolation does not reflect the real world \autocite{pelzResearchMethodsSocial}. Survey research, as another example, has a variety of advantages, such as measuring unobservable data (e.g. peoples preferences, beliefs and values), it is easily scalable and can be carried out independently of time and space. Nonetheless, it is also subject to many biases such as sampling bias, recall bias and non-response bias \autocite{pelzResearchMethodsSocial}.
% justification case study
Since all research types have their advantages and disadvantages, it is primarily a weighing of these strengths and weaknesses that determines the choice of method. The strength of the case study make this research type ideal for a holistic exploration and understanding of novel and under-researched areas within a "complex and dynamic context where it is difficult to isolate variables or where there are multiple, influencing variables" \autocites[2]{fitzgeraldCaseStudiesResearch1999}{zainalCaseStudyResearch2007}. Therefore, a case study is an excellent method not only to test theories but also to develop new theories and frameworks, as it is the aim of this (pilot) study \autocite{pelzResearchMethodsSocial, zainalCaseStudyResearch2007}. \linebreak[1]
Along with the exploratory technique, an iterative technique is adopted, referring to the "visiting and revisiting [of] the data and connecting them with emerging insights, progressively leading to refined focus and understandings" \autocite[77]{srivastavaPracticalIterativeFramework2009}. Thereby, each interview or questionnaire was directly transcribed, coded, analysed and merged with the insights gained up to that point. Along with the conducted literature review verify previous findings and iteratively expand the insights.Together with the theoretical background information and project analyses, newly acquired knowledge could be iteratively integrated, triangulated and used as a basis for further research and interviews. This made it possible to check, deepen and refine the knowledge gained piece by piece.
% lit analysis
The background analysis started with the review of the already established database of the related project and was subsequently extended. Broad concepts were used as a basis to lay a thematically large, yet steadily more specific foundation. For the in-depth analysis of previous \acrshort*{cs} projects and further insights into the case study itself, grey and peer-reviewed literature was consulted. Grey literature, as academic literature on Somalia was found to be generally scarce. The search was based on different combinations of keywords and their synonyms which could be derived from the underlying concepts and their respective specifications. Furthermore, in the course of the work, further literature and project suggestions were received from the project team and from the interviewees. For the selection of CS projects, core areas of interest were formulated and derived from the thematic focus of the fundamental concepts. Therefore, the thematic focus was on community-based participatory environmental or risk monitoring, but always with a focus on water related issues. The geographical location, size or technical facilities had no influence. Subsequently, the projects were tabulated and jointly evaluated manually, since the absolute number of projects finally selected was manageable at 20 without additional software.\linebreak[1]
The analysis of existing data sets of water source point and feature information was considered, but was discarded at a very early stage. The very limited reliability, completeness and actuality of the available data sets had already been reviewed and stated by the project team before the start of this work. This lack of data was one reason for this work, and a short analysis via QGIS was able to confirm these statements and thus, due to relevance and time constraints, the focus shifted to the design approach.\linebreak[0]
In addition to the thematic focus on water-related issues, the practical implementation on the ground was investigated by analysing the already established \acrshort*{cbs} program of the \acrshort*{srcs}. This was done primarily by interviewing the responsible managers.
% interviews
The selection of interviewees was based on a strategy of targeted expert sampling, i.e. a non-probability method that focused on reaching key informants and conducting expert interviews. This technique was employed as the expertise and experience of the individuals was crucial rather than focusing on broad, generalisable statements \autocite{pelzResearchMethodsSocial}. This method was further extended by adopting a snowball sampling approach which helped to identify further stakeholders and potential candidates.\linebreak[1]
The target persons were primarily people who know the local context and/or are potential stakeholders in a possible implementation of the design in question. The first interview came about through existing contacts of the project in which this work is embedded, and the interviewee was the project leader of the FbF approach in the \acrshort{srcs}. In the further course, the CBS project manager on the Norwegian Red Cross side and the CBS manager on the Somali side were also interviewed. Between these two interviews, there was a second interview with the project manager of the \acrshort{srcs}' \acrshort{fbf} team. More interviews with representatives of the Ministry of Water Resources, \acrshort{nadfor}, FAOSWALIM, \acrshort{brcis} and the technicians responsible for the \acrshort{nyss} platform were envisaged but could not be conducted. The interviews with the project leader on the \acrshort{srcs} side unfortunately had to be replaced by written questionnaires, as he could not free up time for an interview and only the higher flexibility of the online questionnaire allowed him to contribute at all. For the conversion of the interview guidelines into questionnaires, the recommendations of \autocite{harknessCCSGQuestionnaireDesign2016} were followed. The interviews and questionnaires themselves were semi-structured and mostly consisted of open ended questions to allow for the interviewee to give a free response as opposed to predefined answer options (see appendix TODO: \ref*{ABCD}). Open-ended questions can facilitate more detailed answers, new insights and overall allow for an unlimited response in terms of scope and focus while they also complicate relevant information abstraction and following analysis \autocite{pelzResearchMethodsSocial}.\linebreak[1]
The intelligent verbatim transcription of the interviews was facilitated by the newly developed neural net called Whisper \autocite{openaiIntroducingWhisper2022,openaiWhisper2023} and subsequently checked and corrected in MaxQDA 2022. For the further analysis, the recommendations of \autocite{radikerFocusedAnalysisQualitative2020} were followed. The coding strategy followed an inductive, open thematic manual coding approach. Codes were not strictly predefined, to be able to appropriately incorporate newly gained expertise, but only broadly categorized into the main themes of interest. More dedicated coding approaches based on e.g. the grounded theory or the hermeneutic analysis were not applied, as the given information was the focus of interest and not e.g. the identification of subjective constructs and underlying meaning \autocite{pelzResearchMethodsSocial}. Nonetheless, based on the criticism on positivism, provided information was not taken as unbiased and objective but interpreted in the context of its perspective.
% Limitations
% muss hier noch was hin? oder eher in die Diskussion? wenige Interviews sind ja kein Methodenfehler.. und Biases, validity etc. sind oben angesprochen.

% dann: fraisl et all + 7 layer model + guidelines (especially Minkman Goals CS)

% presentation of 6-stage-cycle fraisl et al. & 7-layer model

\section{Design Frameworks}

The design of the roadmap for participatory community water source monitoring is particularly guided by two mutually complementary design frameworks. The six iterative stages for the design and implementation of a Citizen Science project in environmental and ecological sciences by \autocite{fraislCitizenScienceEnvironmental2022} will lay the conceptual foundation for this work. It covers the entire life cycle of a Citizen Science project in its six phases in an iterative way, starting with problem assessment and finishing with evaluation procedures. Stage three, the "designing the project" stage, will further be enhanced by the \acrfull{slmc} by \autocite{briggsSevenLayerModelCollaboration} as the guidance of the \acrfull{ssf} is relatively scarce for practical application at this stage. The SLMC can help to reduce cognitive overload and give further impulses from a social point of view through its conceptual division into 7 levels in the design of cooperation projects. Furthermore, several other guidelines will briefly be presented as they can provide further valuable insights for the design of a \acrshort*{cs} project.

\subsection{6-Stage-Framework}

In their work, \autocite{fraislCitizenScienceEnvironmental2022} pull information from all kinds of \acrshort{cs} programmes, projects and scientific guidance. While their thematic orientation is on projects in the field of ecology and environmental sciences, the underlying principles which they describe are successfully applied in a wide variety of other thematically differently oriented projects \autocite{fraislCitizenScienceEnvironmental2022}. The developed \acrfull{ssf} concentrates on the participation level (1) Crowdsourcing and (2) Distributed Intelligence. On these levels, citizens are primarily contributors and partly asked to interpret the sensed information. This was outlined in chapter \ref*{sec:cs}. All six stages are interconnected and should all be considered throughout the project to incorporate new information, feedback and lessons learned \autocite{fraislCitizenScienceEnvironmental2022}. An overview of the \textit{citizen science project life cycle} can be seen in figure \ref{TODO: figure stages fraisl}

\missingfigure{ssf design - possibly do it myself as the figure is not as nice here (too much other stuff)}

In stage 1 the overall need and problems are identified and their boundaries defined. This includes the gathering of potential solutions, limitations and the formulation of research questions. Stage 2 closely examines the potential application of \acrshort*{cs} in the identified boundaries. The focus is on the fruitfulness of the involvement of participants to reach the formulated objectives and answer the research questions. This may be related to many project specifications, e.g. temporal and spatial scale, required expertise and intended target groups. As the second major consideration for the reasonable integration of \acrshort*{cs} participants, \autocite[2]{fraislCitizenScienceEnvironmental2022} note, that the project need to benefit the participants by "addressing their needs or fostering new skill and expertise". After problems, needs and applicability are addressed, the objectives and aims of the project need to be defined in detail together with the prospective participants in stage 3. In addition to the main objectives, secondary objectives such as awareness and knowledge building as well as its transfer could also be pursued. Just as concerns of data storage and analysis, privacy and ethics, selection of methods and training strategies as well as communication means and instruments are also parts of this stage. Furthermore, the tasks of the participants need to be defined in detail, also including any benefits and safety considerations. Stage 4 is concerned with the building of the community by identifying participants motivations, education levels and other demographic information as well as issues of acknowledgments, feedback and sustaining participation. Planning of data management in terms of collection, storage, assuring quality, analysis and privacy and security are highlighted in stage 5. Although evaluation is the main theme of stage 6, it is seen more as an ongoing effort that is recommended throughout the project to allow for feedback and improvement at each stage.
This work has focussed primarily on the stages one to three as stage 4, community building is not necessary to that extent since the \acrshort*{srcs} already has a vivid network of active and motivated volunteers. The fifth stage, was partly considered, but could not be mainly worked on due to time constraints and unresolved issues in the design phase and as this project did not leave the design phase, a major evaluation in phase 6 was obsolete.

\subsection{Seven-Layer Model of Collaboration}

The design pattern of the \acrshort*{slmc} was integrated into the above \acrshort*{ssf} in stage 3 \textit{designing the project} to better handle and structure the high complexity of the roadmap design. The \acrshort*{slmc} was specifically designed to reduce cognitive (over-)load for the design of a complex, interrelated project in a social-technical context. It does so, by separating concerns at design time into seven layers and  corresponding methods and techniques. These, as presented by \autocite{briggsSevenLayerModelCollaboration}, are primarily aimed at the collaboration of groups, but the overall pattern can be preserved when applied to designs in other contexts \autocite{diggelenGroundedDesignDesign2009}. Therefore, the following explanations of the individual layers, their methods and techniques are adapted to this work, while maintaining the general pattern developed by \autocite{briggsSevenLayerModelCollaboration}. The seven, slightyl adjusted layers are, Goals, Products, Activities, Methods, Techniques, Tools and Scripts (see figure \ref*{TODO: slmc image}). 

\missingfigure{slmc all layers figure -> beware of the Methods adjustement}

The layers are ordered hierarchically with Goals being the top-most layer. Changes made in one layer, may need to be accounted for in the lower, but not necessarily in the upper, layers. The \textit{Goal-layer} incorporates all overarching goals and objectives of the project. The \textit{Products-Layer} sum up all tangible or intangible components or outcomes that are necessary to achieve the formulated targets in the \textit{Goal-Layer}. The required activities that yield these products, are grouped in the \textit{Activities-Layer}. These activities formulate what needs to be done to reach the goals and can have sub-products and sub-activities of their own. The fourth level is the most different from the original, as it does not deal with the procedures of cooperation but with the applied methods during the activities. The \textit{Techniques-Layer} specifies the involved techniques and practices and the \textit{Tools-layer} summarises all relevant artifacts or apparatus used. The final procedures are described in detail and defined in the bottommost layer, the \textit{Script-Layer}. Further concerns, interactions and justifications between and for each layer are extensively described by \autocite{briggsSevenLayerModelCollaboration} and while there is a lot of value in these remarks, the thematic adaptations and focus of this work make further exploration in this context obsolete. Nonetheless, interested readers are invited for further independent exploration. 

\subsection{Citizen Guidelines and Recommendations}

Practical recommendations, frameworks and guidelines for \acrlong*{cs} projects have become numerous and thematically wide-ranging. Networks and programmes of researchers and practitioners covering all or parts of the stages proposed by the \acrshort*{ssf} are spatially widespread at the global level and include a variety of regional, national or global application levels. Some examples of such networks and programmes are government programmes such as the US-run \href{https://www.citizenscience.gov/}{citizenscience.gov} website or the EU platform \href{https://eu-citizen.science/}{eu-citizen.science}, support platforms such as \href{https://citsci.org/}{CitSci} and the \href{https://citizenscience.ch/en/}{Citizen Science Center Zurich} as well as regionally focused associations like \href{http://cienciaparticipativa.net/la-ricap/}{La Red Iberoamericana de Ciencia Participativa (RICAP)}, \href{https://citizenscience.asia/}{CitizenScience.Asia} and the \href{https://www.usiu.ac.ke/citsci-africa-association/}{Citizen Science Africa Association}.\newline
Besides these knowledge-hubs, a vast variety of different scientific and and grey literature exists. \autocite{fraislCitizenScienceEnvironmental2022} and \autocite{westonCommunityBasedWaterMonitoring2015} have each listed and summarized many of recommendations and \autocite{garciaFindingWhatYou2021} even created a guide to citizen science guidelines. In this work, the guiding principles and recommendations of \autocite{minkmanCitizenScienceWater2015} were further considered and integrated. These guidelines were developed in cooperation with water management authorities in the Netherlands and concentrate on advice that can be implemented in practice. Furthermore, \autocite{minkmanCitizenScienceWater2015} derived a set of six potential goals which could be addressed through \acrlong*{cs} in water management namely (a) awareness raising, (b) public education, (c) policy development, (d) method improvements, (e) knowledge building, and (f) management improvements. Each \acrshort*{cs} project can address multiple of these goals and with varying emphasis. This work adopted these six goals as the starting point for further analysis and they formed the first level of the SLMC in stage 3 of the \acrshort*{ssf}. The remaining guidelines developed by \autocite{minkmanCitizenScienceWater2015} and above mentioned projects, programmes and associations were incorporated into the design and structuring of the questions for the interviews.
For the second stage of the \acrshort*{ssf}, determining the applicability of the citizen science approach, \autocite{fraislCitizenScienceEnvironmental2022}s own set of recommendations together with the IFRC recommendations for a \acrshort*{cbs} project assessment were applied \autocite{goodermoteConductingAssessmentCommunitybased2020}.

% possibly include something like
% and a collection of those guidelines can be seen in Table XYZ
% BRCiS learning, Minkman guidelines and so on.. + ESCA 10 guidelines and characteristics
% concluding summery
The research design and the applied frameworks and supporting guidelines were presented in this chapter. Case study research was identified as the most appropriate research type for the objective of this thesis and a mixed methods approach of a deductive literature analysis and an inductive interview strategy was defined. The advantages, disadvantages and main limitations of these approaches were outlined and will be further discussed in the following chapters. The guiding principles for the design of the roadmap were found in \acrlong*{ssf} and \acrlong*{slmc} and their integration was detailed. Further work from related projects, networks and associations was presented and their integration into the overall methodology of this work was outlined. Following these decisions for the design and application of the methodology, the results of this work are presented in the next chapter.



% Guidelines:
% - Consider the design elements to make informed decisions
%     - Design tips:
%         - Comprehend what citizen science is and decide whether it is suitable as a means
%             -> formulate clear goals of the project ([Minkman, 2015, p. 177](zotero://select/groups/4773535/items/ZKLE6CPT)) ([pdf](zotero://open-pdf/groups/4773535/items/QMAPCSZG?page=177\&annotation=DERHMEYN))
%             -> formulate sub-goals and related products (e.g. list of things that should be known, ...)
%             -> determine which activities are most suitable to deliver the products
%         - Collaborate with partners
%             -> Identify partners (research institutes, official agencies, organizations, interest groups)
%             -> be aware of their interests and goals + who has an interest in a non-functioning system?
%         - Beware of financial motivations
%             -> infrastructure, development, coordination, training, equipment, etc. are non-negligible costs
%         - Key success-factor: Match patterns of collaboration with organizational capacity (regarding coordination and support)
%             -> determine patterns of collaboration: governance and level of involvement (-> contributory, collaborative, or co-created)
%         - Match techniques with organizational capacity (regarding data processing)
%             -> MCS (Mobile Crowd Sensing)
%                 - Pros:
%                     * highly mobile and scalable
%                     * low-cost
%                     * automatic time stamp and GPS possible
%                     * citizens could interfere when necessary
%                 - Cons:
%                     * devices are not specifically developed for the sensing task
%                     * the target audience may not be familiar with smartphones
%                     * difficult to ensure data trustworthiness
%                     * high risk of privacy invasion
%             -> An MCS application should be useful rather than easy-to-use
%                 	-> TAM (Technology Acceptance Model): ease of use, usefulness, behavioural intention to use a product
%         - Organize a pilot or have a trial period
%         - align organization interest/motivation with citizen interest/motivation
%             -> motivation overview p.167 + create ownership
%         - make sure data us used and provide feedback
%         - constantly recruit new participants
 
% - is citizen science suitable?
% - collaborate with strategic partners
% - care about the associated costs (development, infrastructure, coordination, training, etc.)
% - techniques and citizen science method should match organizational capacities
% - take (social) scientific knowledge into account
% - start with a pilot project
% - keep the capacity of the target audience, users and authorities, in mind. balance data trustworthiness and privacy is the main challenge

%----------------------------------------------------------------------------------------
%	SECTION 1
%----------------------------------------------------------------------------------------
% ESCA 10 Principles of Citizen Science (!767WRLDX!) as important as Minkman -> though minkman -> goals

% I actually already got most of them included
% “Summary of Recommendations for effective citizen science and community-based water monitoring and management” ([Weston and Conrad, 2015, p. 4](zotero://select/groups/4773535/items/49HXDHSH)) ([pdf](zotero://open-pdf/groups/4773535/items/CCHM5SNH?page=4&annotation=LBR9MKTV))

% core paper CBWM
% “Community-based water resources management” ([Day, 2009, p. 47](zotero://select/groups/4773535/items/YWSNQ8A2)) ([pdf](zotero://open-pdf/groups/4773535/items/ETPCI5RI?page=2&annotation=RQLJMKL7))
% +
% tons of recommendations
% “Community-Based Water Monitoring in Nova Scotia: Solutions for Sustainable Watershed Management” ([Weston and Conrad, 2015, p. 1](zotero://select/groups/4773535/items/49HXDHSH)) ([pdf](zotero://open-pdf/groups/4773535/items/CCHM5SNH?page=1&annotation=94TW58QF))

% https://brcis.shinyapps.io/EWEA_dashboard/

% 1. “Community leaders can play a greater role in humanitarian response.” 
% 2. “2. Risk monitoring has proved highly effective and essential for future programmes.” 
% 3. “3. Indicators should be capable of capturing the compound effects of multiple shocks.”
% “4. Vulnerable communities need improved access to climate and weather information” 
% “5. Local-level EWEA committees strengthen local capacities to prepare for and respond to shocks.” 
% “6. Flexible and shock-responsive funding mechanisms successfully improved food security.” 
% “7. Communities highly valued and engaged with the BRCiS integrated approach” ([Gualazzini, 2021, p. 20](zotero://select/groups/4773535/items/BWDYDL8T)) ([pdf](zotero://open-pdf/groups/4773535/items/8U5XVU5K?page=20&annotation=UTZLTZB3))

% “BRCiS early warning and early action vision (2022–2026)” ([Gualazzini, 2021, p. 20](zotero://select/groups/4773535/items/BWDYDL8T)) ([pdf](zotero://open-pdf/groups/4773535/items/8U5XVU5K?page=20&annotation=QVKJRXK6))



% Chapter Template

\chapter{Results} % Main chapter title

\label{chapter4}


% "The first step is to craft a brief introduction to the chapter. This intro is vital as it provides some context for your findings. In your introduction, you should begin by reiterating your problem statement and research questions and highlight the purpose of your research. Make sure that you spell this out for the reader so that the rest of your chapter is well contextualised.
% The next step is to briefly outline the structure of your results chapter. In other words, explain what’s included in the chapter and what the reader can expect. In the results chapter, you want to tell a story that is coherent, flows logically, and is easy to follow, so make sure that you plan your structure out well and convey that structure (at a high level), so that your reader is well oriented." 
%----------------------------------------------------------------------------------------
%	SECTION 1
%----------------------------------------------------------------------------------------
%using simple transmission protocols such as SMS

The first aim of this thesis was to answer a deductive hypothesis, namely whether CBS and MCS are potentially viable approaches to answer the primary inductive research question of this thesis. The main aim was to create a roadmap for the development of a community-based participatory monitoring approach in the context of developing an EAP in a resource-limited setting. While the first aim could already be partially addressed in the preceding chapters (see \ref*{subsec:practical_examples} and \ref*{subsec:water_sources}), the hypothesis is further supported by the results presented below, particularly in Stages 1 and 2. %adjust. main aim roadmap -> applicability in the second half of the results
% this is the intro to the second half, partly at least - need to rewrite afterwards
The stages follow the SSF (see chapter \ref*{subsec:ssf}) and are presented in the coming sections. Stages 1 and 2 are concerned with the context, problem definition and feasibility assessment. A short conclusion is attached to each of these stages, justifying further progress to the next phase. Stage 3 \textit{Designing the Project} incorporates the \acrshort*{slmc} and is the central element of this chapter by presenting the roadmap in detail. Stage 4 to 6 contain brief descriptions of the SRCS's community building and volunteer recruitment, data management practices with an emphasis on NYSS and the evaluation procedures which are currently in place, respectively. The results are based on the conducted interviews, the project analysis and the literature analysis. The full interview and questionnaire transcripts are presented in Appendix \ref*{TODO:} and the protocol to the data analysis in Appendix \ref*{TODO:}. The chapter is concluded with a summary of the key findings.

\section{Design Roadmap}
% today:::
% 1. Methodology angleichen (afterwards)
% 2. this chapter --> pull all guidelines together + double check my own stuff in stage 3 SLMC
% If I get all of this done by today, it's really good.
% Saturday:
% write the second half of the results
% Sunday:
% reread methodology, adjust introduction and write result summary possibly prepare discussion

% discussion:
% read some guidelines mate. just read some guidelines. Bed time. can't think anymore.
% last big chapter! Woop Woop!!
% discuss context e.g. water security, water scarcity and drought
% discuss FbF and CS
% 




Design of the roadmap --> SSF, SLMC, Guidelines + projects - literature review and interviews

While the first hypothesis could be verified (if applicable) the deductive/inductive roadmap design is the focus of this part -

second half: application of this roadmap with a focus on Stages 1-3, whereas in stage 3, the focus lies on the identification of applicable and fruitful water level trigger development focussed on the water source type berkad.  

% it is common in case studies to integrate the results with the discussion as the context is very important and changes during the study often influence the results during the work. This work separates the chapters in principle but some brief explanations are given within the result chapter in order to be able to follow along with its content and decisions on the development of the framework. (?needed?)

% this grouping was based on multiple findings.
In the first part
--> roadmap design, 

1. Project requirements -> could be grouped into four groups -> Foundation, Innovations, Knowledge and Management 
% -> Innovation -> it is believed, that every context is special and that some sort of adjustments and innovations must be made every time - varying extend but nonetheless (not many papers talked about this though)
% foundation: transition of CBS --> no CBS without thorough embeddedness into the context with a good and well laid out basis - policy development is somewhat in between innovation and foundation. but e.g. the framework of Day 2009 lies a good foundation for this -> though, not that new and mana IWRM concepts exist as well. Though, as already discussed, often way too complicated to be applicable in practice --> Day found to be extensive but also limited enough to be realisable.
% knowledge, gathers everything that needs to be known for the entire project. Is answered across all stages 
% management -> what the goal is: initial and regular information gathering --> good to know what the knowledge is needed for
2. adjustments/expansions of the six stages


\subsection{Project requirements}



These categories can be grouped under the umbrella term "project requirements". Project requirements refer to the essential conditions, features, or capabilities that a project must meet or possess to achieve its objectives successfully.

"What the project must be based on" relates to the foundational elements or principles that the project must be built upon.

"What needs to be invented" refers to the innovative or novel solutions that may be required to address specific challenges or achieve project objectives.

"What needs to be known" includes the knowledge or information that is necessary to understand the problem, develop solutions, and implement the project successfully.

"What needs to be done" encompasses the specific tasks, activities, or processes that are necessary to complete the project and achieve the desired outcomes.


\subsubsection{The Overview: Knowledge Building}
%%%%%%%%%%%%%%%%%%%%%%%%%%%%%%%%%%%%%%%%%%%%%%%%%%%%%%%%%%%%%%%%%%%%%%%%%%%%%%%%%%%%%%%%%%%%%%%%%%%
%%%%%%%%%%%%%%%%%%%%%%%%%%%%%%%%%%%%% !!! SUBSECTION 1 !!! %%%%%%%%%%%%%%%%%%%%%%%%%%%%%%%%%%%%%%%%
%%%%%%%%%%%%%%%%%%%%%%%%%%%%%%%%%%%%%%%%%%%%%%%%%%%%%%%%%%%%%%%%%%%%%%%%%%%%%%%%%%%%%%%%%%%%%%%%%%%


\subsubsection{The Innovations: Policy Development \& Method Improvements}%%%%%%%%%%%%%%%%%%%%%%%%%%%%%%%%%%%%%%%%%%%%%%%%%%%%%%%%%%%%%%%%%%%%%%%%%%%%%%%%%%%%%%%%%%%%%%%%%%%
%%%%%%%%%%%%%%%%%%%%%%%%%%%%%%%%%%%%% !!! SUBSECTION 1 !!! %%%%%%%%%%%%%%%%%%%%%%%%%%%%%%%%%%%%%%%%
%%%%%%%%%%%%%%%%%%%%%%%%%%%%%%%%%%%%%%%%%%%%%%%%%%%%%%%%%%%%%%%%%%%%%%%%%%%%%%%%%%%%%%%%%%%%%%%%%%%


\subsubsection{The Foundation: Awareness Raising \& Public Education}%%%%%%%%%%%%%%%%%%%%%%%%%%%%%%%%%%%%%%%%%%%%%%%%%%%%%%%%%%%%%%%%%%%%%%%%%%%%%%%%%%%%%%%%%%%%%%%%%%%
%%%%%%%%%%%%%%%%%%%%%%%%%%%%%%%%%%%%% !!! SUBSECTION 1 !!! %%%%%%%%%%%%%%%%%%%%%%%%%%%%%%%%%%%%%%%%
%%%%%%%%%%%%%%%%%%%%%%%%%%%%%%%%%%%%%%%%%%%%%%%%%%%%%%%%%%%%%%%%%%%%%%%%%%%%%%%%%%%%%%%%%%%%%%%%%%%


\subsubsection{The Monitoring: Management Improvements}%%%%%%%%%%%%%%%%%%%%%%%%%%%%%%%%%%%%%%%%%%%%%%%%%%%%%%%%%%%%%%%%%%%%%%%%%%%%%%%%%%%%%%%%%%%%%%%%%%%
%%%%%%%%%%%%%%%%%%%%%%%%%%%%%%%%%%%%% !!! SUBSECTION 1 !!! %%%%%%%%%%%%%%%%%%%%%%%%%%%%%%%%%%%%%%%%
%%%%%%%%%%%%%%%%%%%%%%%%%%%%%%%%%%%%%%%%%%%%%%%%%%%%%%%%%%%%%%%%%%%%%%%%%%%%%%%%%%%%%%%%%%%%%%%%%%%




\subsection{Stage 1: Context and Problem identification}
%%%%%%%%%%%%%%%%%%%%%%%%%%%%%%%%%%%%%%%%%%%%%%%%%%%%%%%%%%%%%%%%%%%%%%%%%%%%%%%%%%%%%%%%%%%%%%%%%%%
%%%%%%%%%%%%%%%%%%%%%%%%%%%%%%%%%%%%%%%%%%%%%%%%%%%%%%%%%%%%%%%%%%%%%%%%%%%%%%%%%%%%%%%%%%%%%%%%%%%
%%%%%%%%%%%%%%%%%%%%%%%%%%%%%%%%%%%%%%% !!! SECTION 1 !!! %%%%%%%%%%%%%%%%%%%%%%%%%%%%%%%%%%%%%%%%%
%%%%%%%%%%%%%%%%%%%%%%%%%%%%%%%%%%%%%%%%%%%%%%%%%%%%%%%%%%%%%%%%%%%%%%%%%%%%%%%%%%%%%%%%%%%%%%%%%%%
%%%%%%%%%%%%%%%%%%%%%%%%%%%%%%%%%%%%%%%%%%%%%%%%%%%%%%%%%%%%%%%%%%%%%%%%%%%%%%%%%%%%%%%%%%%%%%%%%%%
%%%%%%%%%%%%%%%%%%%%%%%%%%%%%%%%%%%%%%%%%%%%%%%%%%%%%%%%%%%%%%%%%%%%%%%%%%%%%%%%%%%%%%%%%%%%%%%%%%%
This first stage is the exploration phase of the overall project \autocite{citizenscience.govBasicStepsYour}. It is aimed at identifying prevailing conditions in all areas that may be covered or touched by the project and even if it does not go into too much detail, the identification efforts must be thorough and as complete as possible. Oversights in this stage can have serious consequences in later stages. To enable this identification, project boundaries must first be defined by the overall objective and the problems to be solved, which also take into account challenges, positive and negative constraints as well as resource requirements. In addition, potential key stakeholders should be involved from the beginning and comparable projects need to be carefully identified to avoid duplication. \autocite{citizenscience.govBasicStepsYour,fraislCitizenScienceEnvironmental2022,minkmanCitizenScienceWater2015}. Based on this information, possible solutions can be derived and hypotheses or research questions formulated (\autocite{fraislCitizenScienceEnvironmental2022, silvertownNewDawnCitizen2009}). Additionally, evaluation practices and sustainability considerations should be integrated into the project as early as possible although they are only defined in detail at a later stage \autocite{fraislCitizenScienceEnvironmental2022}.


\subsection{Stage 2: Assess the feasibility of the Citizen Science approach.}
%%%%%%%%%%%%%%%%%%%%%%%%%%%%%%%%%%%%%%%%%%%%%%%%%%%%%%%%%%%%%%%%%%%%%%%%%%%%%%%%%%%%%%%%%%%%%%%%%%%
%%%%%%%%%%%%%%%%%%%%%%%%%%%%%%%%%%%%%%%%%%%%%%%%%%%%%%%%%%%%%%%%%%%%%%%%%%%%%%%%%%%%%%%%%%%%%%%%%%%
%%%%%%%%%%%%%%%%%%%%%%%%%%%%%%%%%%%%%%% !!! SECTION 2 !!! %%%%%%%%%%%%%%%%%%%%%%%%%%%%%%%%%%%%%%%%%
%%%%%%%%%%%%%%%%%%%%%%%%%%%%%%%%%%%%%%%%%%%%%%%%%%%%%%%%%%%%%%%%%%%%%%%%%%%%%%%%%%%%%%%%%%%%%%%%%%%
%%%%%%%%%%%%%%%%%%%%%%%%%%%%%%%%%%%%%%%%%%%%%%%%%%%%%%%%%%%%%%%%%%%%%%%%%%%%%%%%%%%%%%%%%%%%%%%%%%%
%%%%%%%%%%%%%%%%%%%%%%%%%%%%%%%%%%%%%%%%%%%%%%%%%%%%%%%%%%%%%%%%%%%%%%%%%%%%%%%%%%%%%%%%%%%%%%%%%%%







\subsection{Stage 3: Designing the Project}%%%%%%%%%%%%%%%%%%%%%%%%%%%%%%%%%%%%%%%%%%%%%%%%%%%%%%%%%%%%%%%%%%%%%%%%%%%%%%%%%%%%%%%%%%%%%%%%%%%
%%%%%%%%%%%%%%%%%%%%%%%%%%%%%%%%%%%%%%%%%%%%%%%%%%%%%%%%%%%%%%%%%%%%%%%%%%%%%%%%%%%%%%%%%%%%%%%%%%%
%%%%%%%%%%%%%%%%%%%%%%%%%%%%%%%%%%%%%%% !!! SECTION 3 !!! %%%%%%%%%%%%%%%%%%%%%%%%%%%%%%%%%%%%%%%%%
%%%%%%%%%%%%%%%%%%%%%%%%%%%%%%%%%%%%%%%%%%%%%%%%%%%%%%%%%%%%%%%%%%%%%%%%%%%%%%%%%%%%%%%%%%%%%%%%%%%
%%%%%%%%%%%%%%%%%%%%%%%%%%%%%%%%%%%%%%%%%%%%%%%%%%%%%%%%%%%%%%%%%%%%%%%%%%%%%%%%%%%%%%%%%%%%%%%%%%%
%%%%%%%%%%%%%%%%%%%%%%%%%%%%%%%%%%%%%%%%%%%%%%%%%%%%%%%%%%%%%%%%%%%%%%%%%%%%%%%%%%%%%%%%%%%%%%%%%%%


\subsection{Stage 4: Community Building}
%%%%%%%%%%%%%%%%%%%%%%%%%%%%%%%%%%%%%%%%%%%%%%%%%%%%%%%%%%%%%%%%%%%%%%%%%%%%%%%%%%%%%%%%%%%%%%%%%%%
%%%%%%%%%%%%%%%%%%%%%%%%%%%%%%%%%%%%%%%%%%%%%%%%%%%%%%%%%%%%%%%%%%%%%%%%%%%%%%%%%%%%%%%%%%%%%%%%%%%
%%%%%%%%%%%%%%%%%%%%%%%%%%%%%%%%%%%%%%% !!! SECTION 4 !!! %%%%%%%%%%%%%%%%%%%%%%%%%%%%%%%%%%%%%%%%%
%%%%%%%%%%%%%%%%%%%%%%%%%%%%%%%%%%%%%%%%%%%%%%%%%%%%%%%%%%%%%%%%%%%%%%%%%%%%%%%%%%%%%%%%%%%%%%%%%%%
%%%%%%%%%%%%%%%%%%%%%%%%%%%%%%%%%%%%%%%%%%%%%%%%%%%%%%%%%%%%%%%%%%%%%%%%%%%%%%%%%%%%%%%%%%%%%%%%%%%
%%%%%%%%%%%%%%%%%%%%%%%%%%%%%%%%%%%%%%%%%%%%%%%%%%%%%%%%%%%%%%%%%%%%%%%%%%%%%%%%%%%%%%%%%%%%%%%%%%%



\subsection{Stage 5: Data Management}
%%%%%%%%%%%%%%%%%%%%%%%%%%%%%%%%%%%%%%%%%%%%%%%%%%%%%%%%%%%%%%%%%%%%%%%%%%%%%%%%%%%%%%%%%%%%%%%%%%%
%%%%%%%%%%%%%%%%%%%%%%%%%%%%%%%%%%%%%%%%%%%%%%%%%%%%%%%%%%%%%%%%%%%%%%%%%%%%%%%%%%%%%%%%%%%%%%%%%%%
%%%%%%%%%%%%%%%%%%%%%%%%%%%%%%%%%%%%%%% !!! SECTION 5 !!! %%%%%%%%%%%%%%%%%%%%%%%%%%%%%%%%%%%%%%%%%
%%%%%%%%%%%%%%%%%%%%%%%%%%%%%%%%%%%%%%%%%%%%%%%%%%%%%%%%%%%%%%%%%%%%%%%%%%%%%%%%%%%%%%%%%%%%%%%%%%%
%%%%%%%%%%%%%%%%%%%%%%%%%%%%%%%%%%%%%%%%%%%%%%%%%%%%%%%%%%%%%%%%%%%%%%%%%%%%%%%%%%%%%%%%%%%%%%%%%%%
%%%%%%%%%%%%%%%%%%%%%%%%%%%%%%%%%%%%%%%%%%%%%%%%%%%%%%%%%%%%%%%%%%%%%%%%%%%%%%%%%%%%%%%%%%%%%%%%%%%



\subsection{Stage 6: Evaluation}
%%%%%%%%%%%%%%%%%%%%%%%%%%%%%%%%%%%%%%%%%%%%%%%%%%%%%%%%%%%%%%%%%%%%%%%%%%%%%%%%%%%%%%%%%%%%%%%%%%%
%%%%%%%%%%%%%%%%%%%%%%%%%%%%%%%%%%%%%%%%%%%%%%%%%%%%%%%%%%%%%%%%%%%%%%%%%%%%%%%%%%%%%%%%%%%%%%%%%%%
%%%%%%%%%%%%%%%%%%%%%%%%%%%%%%%%%%%%%%% !!! SECTION 6 !!! %%%%%%%%%%%%%%%%%%%%%%%%%%%%%%%%%%%%%%%%%
%%%%%%%%%%%%%%%%%%%%%%%%%%%%%%%%%%%%%%%%%%%%%%%%%%%%%%%%%%%%%%%%%%%%%%%%%%%%%%%%%%%%%%%%%%%%%%%%%%%
%%%%%%%%%%%%%%%%%%%%%%%%%%%%%%%%%%%%%%%%%%%%%%%%%%%%%%%%%%%%%%%%%%%%%%%%%%%%%%%%%%%%%%%%%%%%%%%%%%%
%%%%%%%%%%%%%%%%%%%%%%%%%%%%%%%%%%%%%%%%%%%%%%%%%%%%%%%%%%%%%%%%%%%%%%%%%%%%%%%%%%%%%%%%%%%%%%%%%%%


\subsection{Main Section 1}















%begin of second half
\section{Stage 1: Context and Problem identification}
%%%%%%%%%%%%%%%%%%%%%%%%%%%%%%%%%%%%%%%%%%%%%%%%%%%%%%%%%%%%%%%%%%%%%%%%%%%%%%%%%%%%%%%%%%%%%%%%%%%
%%%%%%%%%%%%%%%%%%%%%%%%%%%%%%%%%%%%%%%%%%%%%%%%%%%%%%%%%%%%%%%%%%%%%%%%%%%%%%%%%%%%%%%%%%%%%%%%%%%
%%%%%%%%%%%%%%%%%%%%%%%%%%%%%%%%%%%%%%% !!! SECTION 1 !!! %%%%%%%%%%%%%%%%%%%%%%%%%%%%%%%%%%%%%%%%%
%%%%%%%%%%%%%%%%%%%%%%%%%%%%%%%%%%%%%%%%%%%%%%%%%%%%%%%%%%%%%%%%%%%%%%%%%%%%%%%%%%%%%%%%%%%%%%%%%%%
%%%%%%%%%%%%%%%%%%%%%%%%%%%%%%%%%%%%%%%%%%%%%%%%%%%%%%%%%%%%%%%%%%%%%%%%%%%%%%%%%%%%%%%%%%%%%%%%%%%
%%%%%%%%%%%%%%%%%%%%%%%%%%%%%%%%%%%%%%%%%%%%%%%%%%%%%%%%%%%%%%%%%%%%%%%%%%%%%%%%%%%%%%%%%%%%%%%%%%%

The brief water source data analysis along with given context, resource restriction, stakeholders and comparable projects as well as problem and goal statements of the interviewees are covered in this first stage. It also builds on the preceding case study area section \ref{sec:case_area} in chapter \ref{chapter2}.\newline
% current situation
The current drought and water scarcity situation in Somaliland has now lasted five years and has greatly impacted the water sector in Somaliland in terms of quantity and quality (I1). I1 describes the current crisis as \textit{"huge and response activities are being overwhelmed by the need"} which will lead to \textit{"commercialization and overpricing of fresh water"} further exacerbating the situation. This was underlined by I3 who describes the current water situation of the rural population as \textit{"whatever source of water they can find is what they have"}. I3 further mentions, that the people sometimes \textit{won't have enough water to wash their hands"} or for other necessary things and that \textit{"they [then] don't think of what kind of water they can get, whether it's bad or something like that"} but only focus on having at least something to drink. Increased water shortage because of bad quality was also reported in the literature (see section \ref*{subsec:water_quality}). Water can potentially be contaminated at all stages of the water collection process, from the initial abstraction of water, through transport and storage, to the use of the water (I3). Water quality is difficult to assess on site, as the the colour is not a good indicator and other parameters can only be determined technically which required equipment and training (I1, I3). Furthermore, I3 confirms the reports in literature that contamination of water in berkads, through their shared use with animals, can happen even before water withdrawal. This depends very much on the construction method and how it is used. The rehabilitation as well as training how to adequately use a berkad are already activities of the \acrshort*{srcs} (I3) and besides the water quality, the quantity also depends on the kind of construction and amount of withdrawal. According to I3, supply period can range from one month to half a year, so information on these parameters is crucial for estimating the potential duration of supply (I3).\newline
Water monitoring and management is not a problem in urban areas as there is an agency responsible for water supply but the problem is primarily in rural and nomad areas, where 70\,\% of the people live, according to I3. \autocite{republicofsomaliaCountryProfile20212021}, on the other hand, estimates an urban and semi-urban, sedentary population rate of 53\,\%.\newline
The selection of beneficiaries for response activities of the \acrshort*{srcs} are currently conducted on the basis of a preceding joint priority setting with the government. This prioritisation is based on \textit{"assumed vulnerability per community based on Number of \acrfull{idp} camps in the area, number of women headed families, predicted IPC classifications etc."} (I1.2). \acrfullpl{aa} have not been implemented due to the "already prevailing crisis" where the "needs are [already] dire and the current \acrshort*{srcs}s focus is on response mechanisms to address the already visible impacts of drought" (I1.2). Additionally, "there has been any actions yet due to the fact that there is no water monitoring and trigger mechanism in place" (I1). The monitoring was itself hampered by the fact that \textit{"Berkads location data is currently missing"} (I1). This statement compares well with the experience of the current project team working on the EAP implementation and the assessment of available data sets (A. Schulze-Eckel \& A. Schauss, personal communication, November 2022).\newline
Table \ref*{TODO} shows all available data sets about water sources in Somaliland, provided by \acrshort*{swalim}. % and OSM (?)

\missingfigure{table swalim water sources} % see joplin
% maybe add OSM data as well.. quite some work though as not one single tag can be used.. could be a good comparison though. Only one data provider is a bit mehh.. and no OSM is hard to explain and I got the python scripts already.. need to refine the methodology though. Should be worth it. + add settlements to it

The spatial distribution of the datasets across Somalia is relatively balanced, with focal points in the regions with many or larger settlements. Based on the SWALIM settlement data set, there are currently 2123 settlements in Somaliland. These settlement data are mostly from the years 2002 and 2006. The total number of water sources varies widely between and over time of the other data sets. The timeliness of the data also has a wide range, from relatively few pieces of data from 2019 in the 2020 dataset to data from the 1980s in the same dataset is much represented. The 2022 dataset misses information about timeliness altogether and the other datasets all have many blank entries as well. Furthermore, many water sources are labelled as 'abandoned' or 'non-functional', e.g. in the 2022 dataset those are 147 out of 685.\newline
The data sets are fed by many sources and actors e.g. FAOSWALIM or other UN organisation, \acrshort*{mowr}, \acrshort*{nadfor} and other NGOs which constructed some water sources in some communities (I1). These actors, along with the community and their elders, local government representatives, SRCS and their volunteers and private berkad owners are also the potential stakeholders of the mapping and monitoring of berkads. Here, I1 notes, that besides the \acrshort*{srcs}, the \acrshort*{mowr}, \acrshort*{nadfor} and the constructing NGOs are the most important stakeholders. The \acrshort*{mowr} and NGOs have the technical expertise in construction, rehabilitation and monitoring and \acrshort*{nadfor} has a comparable community level programme for monitoring \textit{"livestock body condition, market prices as well as weather variables"} (I1). 
Other comparable programmes exist from \acrshort*{ocha}, \acrshort*{brcis} and the \acrshort*{cbs} programme run by the Ministry of Health and the \acrshort*{nrc}. While these projects may broadly be comparable, none of the Interviewees know of a project that conducts similar things to this works approach (I1, I2, I3). I2 also suggests that the projects are close enough to each other to pass on experiences and recommendations, e.g. from the MoH to the MoRW, in order to overcome initial scepticism and reluctance.\newline
Challenges, limitations and requirements are mentioned in areas of privately owned berkads, community expectation handling and the dissolution of misconceptions as well as potentially already overstretched SRCS staff and volunteers (I1). I1 mentioned, that private owners of berkads may prevent the volunteer form gaining access to their berkad which would result in less information but also in tension in the community. Giving information from the community to someone else may also generally require some explanation (I2). Furthermore, some \textit{"information on past details per particular geographical areas"}(I1.2) can be difficult to access, as \textit{"Somalis are highly mobile communities"} (I1.2). The monitoring could furthermore develop \textit{"hugh expectations from the communities as there is the ongoing drought. Whenever there is monitoring of resources communities believe this should be followed up by instant aid"} (I1). 
Addressing some of the challenges mentioned above, the \textit{"community elders should be engaged before the start of the mapping and monitoring as they will help dispel misconceptions about the project"} (I1) and the \textit{"ministry of water resources should be in the loop during the entire project duration"} (I1). Nonetheless, the \textit{"community and SRCS goals match as both focus on closing the knowledge [gap] currently existing"} (I1) in regard to the number, status of ownership, location and capacity of the berkads per community, district and regional level. This will \textit{"inform decision makers on the priority areas to focus on"} (I1). Therefore, I1 expects that this information from the site triangulated with weather forecasts can help to form robust triggers to take appropriate and informed \acrlongpl*{aa} before critical water levels are reached.


\section{Stage 2: Determining if Citizen Science is the right Approach}
%%%%%%%%%%%%%%%%%%%%%%%%%%%%%%%%%%%%%%%%%%%%%%%%%%%%%%%%%%%%%%%%%%%%%%%%%%%%%%%%%%%%%%%%%%%%%%%%%%%
%%%%%%%%%%%%%%%%%%%%%%%%%%%%%%%%%%%%%%%%%%%%%%%%%%%%%%%%%%%%%%%%%%%%%%%%%%%%%%%%%%%%%%%%%%%%%%%%%%%
%%%%%%%%%%%%%%%%%%%%%%%%%%%%%%%%%%%%%%% !!! SECTION 2 !!! %%%%%%%%%%%%%%%%%%%%%%%%%%%%%%%%%%%%%%%%%
%%%%%%%%%%%%%%%%%%%%%%%%%%%%%%%%%%%%%%%%%%%%%%%%%%%%%%%%%%%%%%%%%%%%%%%%%%%%%%%%%%%%%%%%%%%%%%%%%%%
%%%%%%%%%%%%%%%%%%%%%%%%%%%%%%%%%%%%%%%%%%%%%%%%%%%%%%%%%%%%%%%%%%%%%%%%%%%%%%%%%%%%%%%%%%%%%%%%%%%
%%%%%%%%%%%%%%%%%%%%%%%%%%%%%%%%%%%%%%%%%%%%%%%%%%%%%%%%%%%%%%%%%%%%%%%%%%%%%%%%%%%%%%%%%%%%%%%%%%%

In this stage, the practical capacities, and applicability and suitability of the \acrshort*{cbm} and \acrshort*{mcs} approach for community-based water monitoring were examined in this context. The SRCS has 249 paid employees, of which 30 work in the risk management and \acrlong*{aa} domain and an additional 1500 volunteers are "evenly spread" across the country (I1). I3 emphasises the \textit{"good relations and good reputation"} that \acrshort*{srcs} has with the communities, making them \textit{"one of the most trusted organizations in the country"} which helps to do programs at community level. The 'feasibility study on Potential Use of \acrshort*{fbf} for SRCS' \autocite[44]{somaliredcrescentsocietyFeasibilityStudyPotential2022} recognizes a \textit{strong national organization} with a \textit{strong volunteer base at community level} that \textit{provides monitoring and hazard warning capacities}. Furthermore, \textit{highly skilled and experienced management staff at coordination and Branch levels} is stated. Nonetheless, \textit{minimal domestic resource mobilization} and a \textit{lack of meteorological, geo-spatial analysis, data management and IT staff} has also been detected.
% NYSS
This lack of resources and digital capacities was addressed in the \acrshort*{cbs} project by the \acrshort*{nrc} and their NYSS platform. Generally, \acrshort{cbs} is \textit{nothing new in itself and often used in health contexts} (I2). \acrshort*{cbs} in Somaliland started in Burao in 2018 with 75 community volunteers, as cholera had broken out in the same region in 2017 (I3). After the pilot was successful, \acrshort*{cbs} has since been extended to all regions but \textit{\acrshort*{srcs} only focusses on hot spot areas where they expect outbreaks to happen} (I3). This development took place over the course of a year with much feedback from the SRCS and is now \textit{"very effective and very supportive"} (I3). The \acrfull*{moh} was and is \textit{constantly in the loop to decide together what, how, when and who} and could gain good experiences with NYSS over the years (I2). By now, NYSS is well embedded in the local conditions and \textit{"mobile teams [...] can be deployed immediately within hours so they can do the response"} (I3) in collaboration with other partners such as the \textit{"government, MoH, WHO, and other sister RCRC organisations} (I3).\newline
% reasoning why this approach works
I2 mentions, that the monitoring of water sources \textit{would fit well thematically, because it [low and poor water quality] is a health risk.} and that although it would require some adjustments and considerations, it would organically expand technical expertise and functionality as it is not \textit{radically new}. Besides NYSS, being methodologically similar but different in thematic orientation, several other projects could be identified in literature, that are oriented towards the same topic but differ in their implementation and operation procedures (compare section \ref*{subsubsec:cbwm}). It can be deduced from this that \acrshort{cbm} and \acrshort{mcs} can in principle be applied to the thematic issue. Furthermore, \autocite{fraislCitizenScienceEnvironmental2022} themselves describe approaches that focus on the "in situ monitoring" of water resources and at the same time benefit the respective participant as adequate for the use of a \acrlong{cs} approach. \acrshort{srcs} does not pay their volunteers but provides training and covers travel expenses as incentives. Moreover, volunteers are generally well regarded and are selected by the community itself. Intrinsic motivation is therefore present (I2, I3).
\autocite{fraislCitizenScienceEnvironmental2022} prerequisites are thus fulfilled, and the key objectives of the RCRC \acrshort{cbs} preliminary assessment can be answered positively. The \acrshort{srcs} is an adequate partner with sufficient capacities and experiences to implement the mapping and monitoring approach and can thus be further developed in stage 3. 


\section{Stage 3: Designing the Project}
TODO:TODO:TODO:TODO:TODO:TODO:TODO:TODO:TODO:TODO:TODO:TODO:TODO:TODO:TODO:TODO:TODO:TODO:TODO:TODO:TODO:
%%%%%%%%%%%%%%%%%%%%%%%%%%%%%%%%%%%%%%%%%%%%%%%%%%%%%%%%%%%%%%%%%%%%%%%%%%%%%%%%%%%%%%%%%%%%%%%%%%%
%%%%%%%%%%%%%%%%%%%%%%%%%%%%%%%%%%%%%%%%%%%%%%%%%%%%%%%%%%%%%%%%%%%%%%%%%%%%%%%%%%%%%%%%%%%%%%%%%%%
%%%%%%%%%%%%%%%%%%%%%%%%%%%%%%%%%%%%%%% !!! SECTION 3 !!! %%%%%%%%%%%%%%%%%%%%%%%%%%%%%%%%%%%%%%%%%
%%%%%%%%%%%%%%%%%%%%%%%%%%%%%%%%%%%%%%%%%%%%%%%%%%%%%%%%%%%%%%%%%%%%%%%%%%%%%%%%%%%%%%%%%%%%%%%%%%%
%%%%%%%%%%%%%%%%%%%%%%%%%%%%%%%%%%%%%%%%%%%%%%%%%%%%%%%%%%%%%%%%%%%%%%%%%%%%%%%%%%%%%%%%%%%%%%%%%%%
%%%%%%%%%%%%%%%%%%%%%%%%%%%%%%%%%%%%%%%%%%%%%%%%%%%%%%%%%%%%%%%%%%%%%%%%%%%%%%%%%%%%%%%%%%%%%%%%%%%


% rework all of this. won't be the last piece.. but it will be way better.
The design of the roadmap for implementing the research aim is presented in detail in this section. This stage, as described in chapter \ref*{sec:design_framework} integrates the \acrshort{slmc} to better structure the design process. The following division of the sections is a grouping of the defined (sub-)goals of \autocite{minkmanCitizenScienceWater2015} and represents the first level of the SLMC (see Fig. \ref{TODO:}). This categorisation of the goals was supported by statements from I1 and arranged chronologically. The order of the goals in this chapter represents this timeline. The foundation of the design is represented in the first two goals \textit{Awareness Raising \& Public Education}. The sub-goals \textit{Policy Development \& Method Improvements} describe technical, social and political prerequisites, necessities and recommendations which will facilitate the central goals of this project \textit{Knowledge Building} and \textit{Management Improvements}. 

\missingfigure{TODO: overview of the goals.--> grouped as in the book + subchapter titles}

Each of these sections is divided into two parts. In the first part, the (sub-)goals are further broken down in detail in regard to required products, activities and their respective sub-products and sub-activities. This constitutes the general design roadmap and is displayed by means of a tree diagram. In the second part, insights are presented that could be determined in the course of this work for the respective activities or products.


% how stuff is displayed --> e.g. activities in violet and circles, products in squares and orange and so on.. maybe adjust line for sub-stuff (?)

%%%%%%%%%%%%%%%%%%%%%%%%%%%%%%%%%%% INTERVIEW I1 RICHARD %%%%%%%%%%%%%%%%%%%%%%%%%%%%%%%%%%%%%%%%%%

- WANTED RESULTS
•	What results must be achieved in order to reach the goals? Please list them.
o	i) Location data of the berkads (coordinates),
o	ii)Volunteer orientation on water resources monitoring
o	iii) determining the ownership status of each berkad
o	iv) community sensitization to dispel misconceptions about the mapping and water monitoring exercise
o	v) water levels monitoring
o	vi) triggering action based on water levels
o	vii) Determining the water level trigger

- most critical ones:
o	Location data as this will enable determine the serving capacity of each berkad i.e the total number of communities dependant on each berkad. One important result is also community sensitisation to dispel misconceptions within the communtiies. The community will need to understand why the SRCS will be monitoring water bodies. Triggering for action is also a key result as the end goal will be to counter water shortages so as to mitigate water shortages

- chronological order:







%%%%%%%%%%%%%%%%%%%%%%%%%%%%%%%%%%% INTERVIEW I1.2 RICHARD %%%%%%%%%%%%%%%%%%%%%%%%%%%%%%%%%%%%%%%%






%%%%%%%%%%%%%%%%%%%%%%%%%%%%%%%%%%% INTERVIEW I2 JUNG %%%%%%%%%%%%%%%%%%%%%%%%%%%%%%%%%%%%%%%%%%%%%







%%%%%%%%%%%%%%%%%%%%%%%%%%%%%%%%%%% INTERVIEW I3 BELEDI %%%%%%%%%%%%%%%%%%%%%%%%%%%%%%%%%%%%%%%%%%%







Intro

define objectives in detail
learn from other projects --> already done in stage 1
stuff about data storage, quality, analysis, privacy and security 
sampling bias (not that important)
training strategies
communication plan
define participant tasks and benefits in detail 

goals were ordered in this order --> I1: chronological order and focus due to focus on EAP and AA and constraints of the work itself

--> to order design phase --> as mentioned, SLMC
--> roadmap -> focus was the identification of goals, sub-goals and objectives and respective products and activities and for e and f a bit more. 


% goals a-d party necessary, party added benefit/bonus --> embeddedness is a fundamental requirement for CBS projects.

Each subsection has an overview diagram, in which the goal, products and activities are displayed in a structured tree-diagram. This diagram, together with potential implications and results, is described in detail in the respective section. Sections to the goals a-d are kept concise, with sections (e) and (f) being more extensive.

% order has meaning as well bottom to top.


each with tree diagram % mal sehen wie man das genau gestalten kann..
and then top-down description of the roadmap
+ results from interviews and literature

+ some integration of interview or literature analysis results 

the main objective is the mapping and monitoring of berkeds BUT: side goals may exist and benefit the project an the participants as well % already mentioned
thus --> the following sub-chapters will each be concerned with one of these six goals. As the main objective of this work falls into the scope of the goals (e) knowledge building (mapping) and (f) management improvements (monitoring) these were therefore also the focus of the work. The first 4 objectives are therefore somewhat shorter and not dealt with in as much detail. % already mentioned in the methodology, be concise.

presented for each of the minkman goals --> goal (+objective)
(a) awareness raising, (b) public education, (c) policy development, (d) method improvements, (e) knowledge building, and (f) management improvements








% pull the two three goals together to one? each is rather small and hard to separate anayway..
% in the way of: Build a foundation for mapping and monitoring.
% --> Determine and establish preceding procedures including awareness raising, public education, policy development and method improvements

TODO:TODO:TODO:TODO:TODO:TODO:TODO:TODO:TODO:TODO:TODO:TODO:TODO:TODO:TODO:TODO:TODO:TODO:TODO:TODO:TODO:
\subsection{The Foundation: Awareness Raising \& Public Education}

%%%%%%%%%%%%%%%%%%%%%%%%%%%%%%%%%%% INTERVIEW I1 RICHARD %%%%%%%%%%%%%%%%%%%%%%%%%%%%%%%%%%%%%%%%%%
RESULTS
o	i) Volunteer orientation on water resources monitoring
o	ii) community sensitization to dispel misconceptions about the mapping and water monitoring exercise

o	The project might create huge expectations from the communities as there is the ongoing drought. Whenever there is monitoring of resources communities believe this should be followed up by instant aid. Private berkad owners might not be willing to contribute to the project. They might bar Volunteers from accessing their berkads thus creating tension between community volunteers and berkad owners.

o	ii) Community elders should be engaged before the start of the mapping and monitoring as they will help dispel misconceptions about the project
o	iii) the ministry of water resources should be in the loop during the entire project duration

o	From the community/volunteer perspective the main goal would be for them to know the existing water resources within their vicinity, as well as the capacity of these water bodies. The main goal being to ascertain whether these water bodies are able to withstand the demand during drought periods.

o	The community and SRCS goals match as both focus on closing the knowledge currently existing regarding berkads numbers per district, community and regional level




%%%%%%%%%%%%%%%%%%%%%%%%%%%%%%%%%%% INTERVIEW I1.2 RICHARD %%%%%%%%%%%%%%%%%%%%%%%%%%%%%%%%%%%%%%%%

Awareness raising and information dissemination should be more on informing the communities on how to improve water quality at local level e,g boiling before drinking. Involving private berked owners is also feasible however their involvement could be limited as they are more concerned about their business models i.e selling of the water and preserving their berkeds than being part of the overall response/Anticipatory action mechanism. Nevertheless there is the potential to work closely with the private berked owners. This can be done through rehabilitation of their privately owned berkeds in return for their involvement in response and anticipatory action activities related to addressing water shortages.

-	Utilise the community based SRCS volunteers to engage communities and sensitise the communoties on the riole the SRCS plays. Also establishing a robust feedback and Complaints mechanism that ensures communities can easily relay their feedback.


%%%%%%%%%%%%%%%%%%%%%%%%%%%%%%%%%%% INTERVIEW I2 JUNG %%%%%%%%%%%%%%%%%%%%%%%%%%%%%%%%%%%%%%%%%%%%%
(4, 5) It must be certain, that something happens after a report is send or at least communicated in which situations they have to handle it themselves. This should be communicated, discussed and decided beforehand. Otherwise, it could fall back negatively on the SRCS and the Volunteer (33).
(24) SRCS will meet the elders before the start and will explain what, who and when things happen. They continue this throughout in roughly monthly or quarterly intervals.
(35) Maybe communicate a timeframe how long it will take until water arrives so they can plan for themselves.

Stakeholder > Communities
-	(88) Reasons for reporting must be explained in detail before the start of it.
-	(90) They will have expectations of this project.


%%%%%%%%%%%%%%%%%%%%%%%%%%%%%%%%%%% INTERVIEW I3 BELEDI %%%%%%%%%%%%%%%%%%%%%%%%%%%%%%%%%%%%%%%%%%%






\subsubsection{Integrated Water Management (Goal and Groundwork for mapping/monitoring)}


one of the goals of public education and policy development

Integrating communities into Water managemend practices is needed!

ground work -> public education --> inform community about water shortages (?) -> create "agreement on water priorization during times of acute drought" + "a common shared agreement to prevent water resource contamination, as well as mitigating over-abstraction." (Day, 2009: 52)

BUT:
“Communities may not always represent a homogeneous, consenting group” ([Day, 2009, p. 52](zotero://select/groups/4773535/items/YWSNQ8A2)) ([pdf](zotero://open-pdf/groups/4773535/items/ETPCI5RI?page=7&annotation=NMA3CWDS))
--> different interests, resources, knowledge, access rights, own hierarchy + status level

how about private water sources?
“Strikingly, the traditional influence of sheikhs in water management does not, however, extend to supervision of private wells or boreholes operated by farmers or individual landowners. Sh” ([Day, 2009, p. 55](zotero://select/groups/4773535/items/YWSNQ8A2)) ([pdf](zotero://open-pdf/groups/4773535/items/ETPCI5RI?page=10&annotation=FXZW8RBU))

AA -> manage water and prioritize effectively (e.g. start with it now)
“This is significant because in drought-prone environments little at tempt is made to inform communities about their available ground water resources and there is minimal emphasis or preparation for monitoring groundwater fluctuations, prioritizing water usage dur ing periods of hardship or assisting communities to develop basic contingency plans with relief agencies or local authorities acting as a back-stop to provide support during periods of acute” ([Day, 2009, p. 51](zotero://select/groups/4773535/items/YWSNQ8A2)) ([pdf](zotero://open-pdf/groups/4773535/items/ETPCI5RI?page=6&annotation=6C4BAHSE))

“hardship. When describing community water projects, 'sustainability' is often referred to. In reality the immediate challenge is to introduce sound steward ship of water resources to assist communities to resist and recover from drought or low and variable rainfall” ([Day, 2009, p. 51](zotero://select/groups/4773535/items/YWSNQ8A2)) ([pdf](zotero://open-pdf/groups/4773535/items/ETPCI5RI?page=6&annotation=DA9BJCDN))

“r of distinct advantages of engaging in commu nity-based water resource mana” ([Day, 2009, p. 51](zotero://select/groups/4773535/items/YWSNQ8A2)) ([pdf](zotero://open-pdf/groups/4773535/items/ETPCI5RI?page=6&annotation=85TSGZK9))

FIGURE of the framework: “Figure 2. Community-based water resource manageme” ([Day, 2009, p. 59](zotero://select/groups/4773535/items/YWSNQ8A2)) ([pdf](zotero://open-pdf/groups/4773535/items/ETPCI5RI?page=14&annotation=BAMSY255))
%% --> this framework as 'base'/foundation/Early Action/AA?

“Local water users often possess detailed indigenous knowledge related to water resources, water needs and historical change that has occurred related to water use. 
Water users recognize that water is a fundamental component of their subsistence-based livelihoods, which helps to weave rela tionships between water users. 
Communities are able to monitor agreed water usage on a daily basis, as part of their daily activities. 
Communities often have historical mechanisms for conflict and dispute resolution related to water resource management, which may require continued support and assistance to evolve and adapt to global challenges. 
Effective water management requires community participation; this principle is well understood in development li” ([Day, 2009, p. 52](zotero://select/groups/4773535/items/YWSNQ8A2)) ([pdf](zotero://open-pdf/groups/4773535/items/ETPCI5RI?page=7&annotation=4RVGKM5R))

“However, responsible planning for drought mitigation at community level is often omitted.” ([Day, 2009, p. 47](zotero://select/groups/4773535/items/YWSNQ8A2)) ([pdf](zotero://open-pdf/groups/4773535/items/ETPCI5RI?page=2&annotation=S4ACQRPL))

“Communities frequently remain excluded from any basic capacity building, centred on water resource management, as part of a localized Integrated Water Resources Management (IWRM) programme.” ([Day, 2009, p. 47](zotero://select/groups/4773535/items/YWSNQ8A2)) ([pdf](zotero://open-pdf/groups/4773535/items/ETPCI5RI?page=2&annotation=WBL45NKR))

“Community-based water resources management” ([Day, 2009, p. 47](zotero://select/groups/4773535/items/YWSNQ8A2)) ([pdf](zotero://open-pdf/groups/4773535/items/ETPCI5RI?page=2&annotation=RQLJMKL7))



% to facilitate this and further implement this roadmap --> policy (social frameworks) and methods (pracitcal techniques and tools) need to be developed.



\subsection{The Innovations: Policy Development \& Method Improvements}
TODO:TODO:TODO:TODO:TODO:TODO:TODO:TODO:TODO:TODO:TODO:TODO:TODO:TODO:TODO:TODO:TODO:TODO:TODO:TODO:TODO:

--> here: methods are developed for the goals e and f --> not done yet, therefore no methods in those goals. Here not methods and so on, as it was not the focus of this work (?)


water level measuring
-> different options
-> folding rule or rule with categorizations (3 or 5) --> categorization is kinda key (is it?)
-> or local knowledge (how long will it last) (outlook) b

method: generally: coded SMS

%%%%%%%%%%%%%%%%%%%%%%%%%%%%%%%%%%% INTERVIEW I1 RICHARD %%%%%%%%%%%%%%%%%%%%%%%%%%%%%%%%%%%%%%%%%%
RESULTS
o	v) Determining the water level trigger
specifically important:
o	Determining the water levels to trigger action.



%%%%%%%%%%%%%%%%%%%%%%%%%%%%%%%%%%% INTERVIEW I1.2 RICHARD %%%%%%%%%%%%%%%%%%%%%%%%%%%%%%%%%%%%%%%%






%%%%%%%%%%%%%%%%%%%%%%%%%%%%%%%%%%% INTERVIEW I2 JUNG %%%%%%%%%%%%%%%%%%%%%%%%%%%%%%%%%%%%%%%%%%%%%

(26) Weekly updates about the water level from the volunteer should work.
(28, 29) It should be communicated, that reports will be checked by the supervisor in order to prevent false reports in hope of more water. If this happens frequently, a solution must be conceptualized.





%%%%%%%%%%%%%%%%%%%%%%%%%%%%%%%%%%% INTERVIEW I3 BELEDI %%%%%%%%%%%%%%%%%%%%%%%%%%%%%%%%%%%%%%%%%%%






TODO:TODO:TODO:TODO:TODO:TODO:TODO:TODO:TODO:TODO:TODO:TODO:TODO:TODO:TODO:TODO:TODO:TODO:TODO:TODO:TODO:
\subsection{The Assemblage: Knowledge Building}
% important: name the mapping campaign! planned as SRCS staff -> educated personal, no rookies, no volunteers.


with preceding procedures including awareness raising, public education, policy development and method improvements determined and accomplished now proceed to the actual mapping task of Berkeds.
%%%%%%%%%%%%%%%%%%%%%%%%%%%%%%%%%%% INTERVIEW I1 RICHARD %%%%%%%%%%%%%%%%%%%%%%%%%%%%%%%%%%%%%%%%%%
RESULTS
o	iii) Location data of the berkads (coordinates)
o	iv) determining the ownership status of each berkad





%%%%%%%%%%%%%%%%%%%%%%%%%%%%%%%%%%% INTERVIEW I1.2 RICHARD %%%%%%%%%%%%%%%%%%%%%%%%%%%%%%%%%%%%%%%%
Regarding Anticipatory actions, there has been any actions yet due to the fact that there is no water monitoring and trigger mechanism in place % though there is of course some response

The SRCS in consultation with the government select target communities based on a pre existing selection /vulnerability criteria based on either number of IDP camps etc

BERKADS - important indicators besides •	the location,•	ownership,•	total number of people or communities dependant on the berkad,•	its water storage capacity and•	functioning

Other information might included the year it was built, the last time it was rehabilitated etc. However this kind of information might be missing as you need people with community/institutional memory to provide such kind of information. Somalis are highly mobile communities and it will be difficult to get information on past details per particular geographical area.


% local knowledge could be well utilized
% Does the community have an idea of how long the water of their water sources will last? How good is this prediction usually?
-	Yes they have an idea. These kinds of predictions are good as communities usually have their own control measures to ensure equitable distribution of water e,g how many containers per family etc. The berkeds are usually locked to ensure there is controlled acess to the water stored.




%%%%%%%%%%%%%%%%%%%%%%%%%%%%%%%%%%% INTERVIEW I2 JUNG %%%%%%%%%%%%%%%%%%%%%%%%%%%%%%%%%%%%%%%%%%%%%


(16) fusion with other datasets can be challenging and is a lot of work




%%%%%%%%%%%%%%%%%%%%%%%%%%%%%%%%%%% INTERVIEW I3 BELEDI %%%%%%%%%%%%%%%%%%%%%%%%%%%%%%%%%%%%%%%%%%%






TODO:TODO:TODO:TODO:TODO:TODO:TODO:TODO:TODO:TODO:TODO:TODO:TODO:TODO:TODO:TODO:TODO:TODO:TODO:TODO:TODO:
\subsection{The Monitoring: Management Improvements}
%one example how to execute the entire SLMC tree (see Miro)
%%%%%%%%%%%%%%%%%%%%%%%%%%%%%%%%%%% INTERVIEW I1 RICHARD %%%%%%%%%%%%%%%%%%%%%%%%%%%%%%%%%%%%%%%%%%
RESULTS:
o	vi) water levels monitoring
o	vii) triggering action based on water levels


%%%%%%%%%%%%%%%%%%%%%%%%%%%%%%%%%%% INTERVIEW I1.2 RICHARD %%%%%%%%%%%%%%%%%%%%%%%%%%%%%%%%%%%%%%%%
berked rehabilitation (SRCS), water trucking (other agencies), distribution of water purification tablets (SRCS), multi purpose cash, awareness campaigns related to hygiene promotion (SRCS). Regarding Anticipatory actions, there has been any actions yet due to the fact that there is no water monitoring and trigger mechanism in place.

Assistance/response is based on the initial prioritization of target areas that SRCS conducts. The prioritization is based on assumed vulnerability per community based on Number of IDP camps in the area, number of women headed families, predicted IPC classifications etc.
% limitations
The SRCS in consultation with the government select target communities based on a pre existing selection /vulnerability criteria based on either number of IDP camps etc


Activities such as berked rehabilitation are done in consultations with the communities and SRCS branches who flag/identify berkeds in need of repairing. Repairing may consists of re roofing/ roofing, and brickwork to strengthen the structure. The berkeds are meant to capture run off water in case of rainfall incidences. In cases where there hasnt been rains for a prolonged time then water trucks are deployed to deliver water to the communities. Cash has been an imoortattn modality to adress the water shortages. In the current prevailing drought, water and food insecurity crisis, water is now being sold by private players. So the cash has come in handy to t least enable the communities to buy fresh water for drinking. Water sources such as dug wells are often contaminated as livestock i,e camels, goats also drink water from those same water bodies as well.

Water levels in berkeds could be a good indicator, however it cannot be a stand alone indicator. This has to be combined by meteorological forecasts and local knowledge as well.
% triangulation with other sources. -> trust and quality as always

-proposed AA:
-	Awareness raising and information dissemination should be more on informing the communities on how to improve water quality at local level e,g boiling before drinking. Involving private berked owners is also feasible however their involvement could be limited as they are more concerned about their business models i.e selling of the water and preserving their berkeds than being part of the overall response/Anticipatory action mechanism. Nevertheless there is the potential to work closely with the private berked owners. This can be done through rehabilitation of their privately owned berkeds in return for their involvement in response and anticipatory action activities related to addressing water shortages.

yes, i) timely distribution of cash to enable communities to buy and stock fresh water
ii) timely distribution of water purification tablets
iii) timely rehabilitation of other water sources such as boreholes

- TRIGGER
These water levels are ideal i.e 
o	Empty (no water at all) 
o	Critical (1 day of water supply remaining), 
o	Low (3 days of water supply remaining), 
o	Middle (5 days of water supply remianing) 
o	High (full capacity)

-	Low category should trigger AA (he possibly meant water trucking and/or cash)

- MONITORING INTERVAL
•	Water level (daily monitoring)
•	Berked condition (annually)
•	Number of people accessing the water form the berked (weekly/monthly)

- WATER QUALITY
Water quality is difficult to monitor at community level as it is a technical activity. Unless if the SRCS through the branch staff are equipped with water testing equipment as well as training them on the water parameters to be tested.
+ not aware of any feasible water quality tests on the ground.. (should be further investigated though..)


% No Answer to Water Trucking

%%%%%%%%%%%%%%%%%%%%%%%%%%%%%%%%%%% INTERVIEW I2 JUNG %%%%%%%%%%%%%%%%%%%%%%%%%%%%%%%%%%%%%%%%%%%%%
(25) Might be beneficial to bring the MoH in contact with the MoWR in order to transfer their experiences with CBS to the new approach.
(37) One to three codes for regular monitoring should be alright but not more, as more codes make it more complicated and will narrow down the choice which Volunteer to take. 

(30, 31) Sending photos would also be possible, though smartphones and internet are often not available. Julia Jung is not supporting the distribution of smartphones for ‘several reasons’

(35) Maybe communicate a timeframe how long it will take until water arrives so they can plan for themselves.

(9, 49) in special occasion up to 7 numbers
(10) small notes that explain the codes and their order
(27) Other or more information besides the codes could be clarified via phone and inserted by the supervisor manually.

(26) Weekly updates about the water level from the volunteer should work.
(28, 29) It should be communicated, that reports will be checked by the supervisor in order to prevent false reports in hope of more water. If this happens frequently, a solution must be conceptualized.

%%%%%%%%%%%%%%%%%%%%%%%%%%%%%%%%%%% INTERVIEW I3 BELEDI %%%%%%%%%%%%%%%%%%%%%%%%%%%%%%%%%%%%%%%%%%%
-(10) SRCS only focusses on hot spot areas where they expect outbreaks to happen. They do not cover all of the country.
-(11, 14) Response happens together with other partners such as the government, MoH, WHO, and other sister RCRC organisations.
-(13) Reports are verified by the regional officer.
-(15) “SRCS has mobile teams who can be deployed immediately within hours so they can do the response.”

-(70) Water monitoring in urban areas is not a problem. There is an agency that is responsible for water supply. “So we don't have any issue with that.”
-(71) Contrary, “when it comes to the rural areas, and nomad areas, is the way we have the problem”
-(72, 62) water can get contaminated, “For example, when you are taking from the source, and again, when you are traveling with the water, and again, when you are storing the water in your home, or when you are using even the water”
-(73) “water may be clear in colour, but when we are using it, it is contaminated. So we cannot decide by the colour”
-(26) “Whatever source of water they can find is what they have.”
-(43) SRCS has a good connection to the local communities and thus they get direct feedback about water shortages.

% WATER TRUCKING
-(74) before putting water in the berked, they initially clean it.
-(74) water trucking “can take a long distance to the, for example, main cities” and is thus costly and requires lots of time.
-(75) but still, “When it comes to the water trucking, it depends” à no universal solution will work
-(76, 77) Thus far, information come from SRCS assessments, community themselves, government or FSNAU (Food Security and Nutrition Analysis Unit) (https://fsnau.org/)
-(78) Prioritisation is based on the government + “all these efforts together they decide where these resources” are going including SRCS. -> Most vulnerable in the area are prioritised and that mostly depends.
-See (81) FbF
-(41) “Sometimes there was water trucking and, you know, the SRCS or the other organizations, even the commercial or trade people, they were supporting to the communities who are in need”

% FbF
-(7) Volunteers provide Oral Rehydration Salts and aquatabs.
-(18) If the water level of berkeds becomes less or scarce, the volunteers provide hygiene and health promotion activities.
-(79) Sometimes the people/community/some households buy water trucking themselves by bringing their money together. They do this in the initial phase.
% !!
-(80) “people they are depending on their livestock” but “if there is a drought the livestock become weak or die” and the people cannot afford to buy water (trucking). “[...] this is the time they talk to the other NGOs or the government and say we need support [...]”
-(81) Water trucking is not only financed by the government and NGOs but also by normal people (“good willers” who then try to gather money and buy water. They “distribute according to the need” and look at the magnitude of the problem and where it exists - refer this water to the to the community”.
-(19, 20, 21) In case of water shortage, volunteers tell people how to prevent waterborne diseases by providing hygiene and health promotion as well as water purification and water boiling actions.




%%%%%%%%%%%%%%%%%%%%%%%%%%%%%%%%%%%%%%%%%%%%%%%%%%%%%%%%%%%%%%%%%%%%%%%%%%%%%%%%%%%%%%%%%%%%%%%%%%%
%%%%%%%%%%%%%%%%%%%%%%%%%%%%%%%%%%%%%%%%%%%%%%%%%%%%%%%%%%%%%%%%%%%%%%%%%%%%%%%%%%%%%%%%%%%%%%%%%%%
%%%%%%%%%%%%%%%%%%%%%%%%%%%%%%%%%%%%%%% !!! SECTION 4-6 !!! %%%%%%%%%%%%%%%%%%%%%%%%%%%%%%%%%%%%%%%
%%%%%%%%%%%%%%%%%%%%%%%%%%%%%%%%%%%%%%%%%%%%%%%%%%%%%%%%%%%%%%%%%%%%%%%%%%%%%%%%%%%%%%%%%%%%%%%%%%%
%%%%%%%%%%%%%%%%%%%%%%%%%%%%%%%%%%%%%%%%%%%%%%%%%%%%%%%%%%%%%%%%%%%%%%%%%%%%%%%%%%%%%%%%%%%%%%%%%%%
%%%%%%%%%%%%%%%%%%%%%%%%%%%%%%%%%%%%%%%%%%%%%%%%%%%%%%%%%%%%%%%%%%%%%%%%%%%%%%%%%%%%%%%%%%%%%%%%%%%
\section{Stages 4 to 6}
These sections were not the main focus of this work as feasibility assessment and subsequent design of a roadmap was the primary focus. Furthermore, for stage 4 and 5, groundwork is completed otherwise and therefore a theoretical and practical foundation is already well established. The volunteer recruitment process and community building procedure of the \acrshort*{srcs} is outlined in Stage 4 and potential data management tools are described in Stage 5. Locally implemented procedures and strategies for Stage 6 Evaluation are briefly presented thereafter. A short summary of the results concludes this chapter.

% skip this and make stage 4-6 sections of their own. Does not fit to the rest of the structure otherwise.. and it's not necessary to go in to too much detail



\section{Stage 4: Community Building} % volunteers + SRCS + training - etc.
%%%%%%%%%%%%%%%%%%%%%%%%%%%%%%%%%%% INTERVIEW I1 RICHARD %%%%%%%%%%%%%%%%%%%%%%%%%%%%%%%%%%%%%%%%%%


paid employees does the SRCS have in total?
o	249
Anticipatory Actions?
o	30
•	How many volunteers does the SRCS have?
o	1500
Volunteers spread across the country? 
-> Volunteers are evenly spread across the country.
However some regiosn have inactive volunteers due to less activities there whilst some regions have active volunteers due to the amount of project work being undertaken there



%%%%%%%%%%%%%%%%%%%%%%%%%%%%%%%%%%% INTERVIEW I1.2 RICHARD %%%%%%%%%%%%%%%%%%%%%%%%%%%%%%%%%%%%%%%%






%%%%%%%%%%%%%%%%%%%%%%%%%%%%%%%%%%% INTERVIEW I2 JUNG %%%%%%%%%%%%%%%%%%%%%%%%%%%%%%%%%%%%%%%%%%%%%

-	(66) Meanwhile, refreshers are not conducted monthly anymore as the volunteers know their business by now. 
-	(112, 119) Volunteers are from within the community and chosen by the elders. They do not get incentives from the SRCS. Their incentives are helping their community and the travels for the trainings.
-	Preliminary trainings, supervision and regular refreshers are important and useful.
-	(120) Volunteers are mostly women as they stay in the community and do not travel as much as men.

Stakeholder > Communities
-	(88) Reasons for reporting must be explained in detail before the start of it.
-	(90) They will have expectations of this project.

(44) No money is given either to the MoH nor to Volunteers.

(53) Supervisor can validate reports e.g. vie phone.
(54) Great success factors are the training, supervision and regular refreshers by the supervisors.


%%%%%%%%%%%%%%%%%%%%%%%%%%%%%%%%%%% INTERVIEW I3 BELEDI %%%%%%%%%%%%%%%%%%%%%%%%%%%%%%%%%%%%%%%%%%%
-(5, 6, 7) Oral Rehydration Points are in the community and run by volunteers of the community who get trained and then support the community by education and promotion of e.g. WASH activities and going to the actual treatment centers.
-(19, 20, 21) In case of water shortage, volunteers tell people how to prevent waterborne diseases by providing hygiene and health promotion as well as water purification and water boiling actions.
-(23) They are trained “not to wait until the people become fall sick, but in a professional mechanism”
-(24) Volunteers “do awareness raising, hygiene and health promotion sessions by doing, for example, group sessions by visiting house to house, visiting to meeting and all these things.”
-(22) SRCS distributes aqua tablets per month “to the volunteers so that they can manage at community level if there is a case.”

-(19, 20, 21) In case of water shortage, volunteers tell people how to prevent waterborne diseases by providing hygiene and health promotion as well as water purification and water boiling actions.
-(23) They are trained “not to wait until the people become fall sick, but in a professional mechanism”
-(24) Volunteers “do awareness raising, hygiene and health promotion sessions by doing, for example, group sessions by visiting house to house, visiting to meeting and all these things.”
-(22) SRCS distributes aqua tablets per month “to the volunteers so that they can manage at community level if there is a case.”
-(43) SRCS has good relations with the communities.
-(44) “community leaders are the one who tells the SRCS or other partners or the government that there is a water shortage”
-(45) “Somalis they support a lot each other when it comes to the disasters or something like that.“ – “everybody in the community whether in the urban or rural areas is participating to support each other”
-(46)”SRCS has a good reputation and image at community levels” “it is one of the most trusted organization in the country. So it is one of the most trusted organization in the country. So there is a strong relation at community level. So that it helps us also to do this program as community level.”
-(48) Response happens together with the MoH.
-(49) SRCS spends a lot of time on community bond building.
-(50) “SRCS, what we do is to provide any necessary support at the community level.”
Volunteers
-„when we are recruiting, I can't say, I cannot say recruiting. When we are you know going to get volunteers of that community, we go to the community.“
-(52) The person must be willing to be a volunteer as the SRCS is not paying them.
-(52, 53) Volunteers have a good reputation in the community and are selected by the community or the committees in that community based on their criteria’s.
-(54, 55, 56) After the selection, SRCS is doing a small assessment about e.g. reading and writing skills and then provide training to them on the basis of the CBS program (health promotions, coding, etc.).
-(57) After the training, the volunteers are send back to their communities and start working there.
-(59) In the community the community leaders, community committees and also the community health committees are important. “They are the one who are supporting” the SRCS on site. There is also a good collaboration between these groups, the volunteer and the SRCS.
-Volunteers teach mothers, and communities about health, how to prevent water contamination, etc.



\section{Stage 5: Data Management}


\subsection{CBS and NYSS} (NYSS qustionnaire was already developed, but never used (see Appendix XYZ))
-> no data collection tool (!!) 
% I need to include some technical details and stuff somewhere.. or at least name the alternatives to NYSS.. could do that in the result part as Julia Jung talked about that stuff as well

%%%%%%%%%%%%%%%%%%%%%%%%%%%%%%%%%%% INTERVIEW I1 RICHARD %%%%%%%%%%%%%%%%%%%%%%%%%%%%%%%%%%%%%%%%%%





%%%%%%%%%%%%%%%%%%%%%%%%%%%%%%%%%%% INTERVIEW I1.2 RICHARD %%%%%%%%%%%%%%%%%%%%%%%%%%%%%%%%%%%%%%%%






%%%%%%%%%%%%%%%%%%%%%%%%%%%%%%%%%%% INTERVIEW I2 JUNG %%%%%%%%%%%%%%%%%%%%%%%%%%%%%%%%%%%%%%%%%%%%%



GOAL OF NYSS:
(41) The goal of NYSS was the provision of a simple data collection tool
(44, 46) The goal was not the collection of data nor forecasting, the goal was to create a platform for Early Warning.
(76) goal: investigation and response from the MoH.

% I2 good experiences with the method are available
CBS
(1) Nothing new in itself, often used in health contexts
(2) Can be a problem to send information about community members – possibly also about water sources? 
(5) CBS does not generally have a negative connotation in Somalia, but could be (?) -> talk to MoH to translate experiences to MoWR
(104) MoH has good experiences with NYSS in the health sector. These experiences should be translated and widened on other topics.
(36) Monitoring water sources would work well with the CBS system and fit well with the overall theme of health risks.


-	(71) MoH in the loop right from the beginning to find out what else is going on, can often be combined with an interview for the assessment.


-	(75, 95) Acceptance and understanding of the MoH for the introduction of NYSS is often an issue. Also because a lot of organizations want to integrate their own tools and instruments – may be just too many.
-	(97) MoH discussed a lot and still discusses a lot.
-	(98) Trust in the MoH is not high. That was one result of the evaluation.
-	(102) MoH is part of the ToT (Training of Trainers)
(103) Julia Jung does not know of other organizations, which would conduct similar things. Except MSF is active in Somalia/Somaliland but focussed on direct health responses. Further information should be provided by the Ministry or latest by the community itself.
(23) before CBS, there is always an assessment

Stakeholder > Ministry of Health
-	(71) MoH in the loop right from the beginning to find out what else is going on, can often be combined with an interview for the assessment.

-	(25, 99) Might be beneficial to bring the MoH in contact with the MoWR in order to transfer their experiences with CBS to the new approach.
-	(102) MoH is part of the ToT (Training of Trainers)

(2) Can be a problem to send information about community members – possibly also about water sources? 
(36) Monitoring water sources would work well with the CBS system and fit well with the overall theme of health risks.

(28, 29) It should be communicated, that reports will be checked by the supervisor in order to prevent false reports in hope of more water. If this happens frequently, a solution must be conceptualized.

(30, 31) Sending photos would also be possible, though smartphones and internet are often not available. Julia Jung is not supporting the distribution of smartphones for ‘several reasons’

-	(3, 12, 20) Excel, SMS and manual insert into excel works as well. NYSS biggest advantage: lots of automatization
-	(40) Kobo would possibly be the best alternative if it doesn’t work with NYSS.

-	(105) WHO is interested in NYSS.
-	(93) IFRC used and uses Kobo.
-	(94) IFRC decides about the development of NYSS.

(8) difference to most other tools: possible with a basic phone. (19) which is necessary because CBS is often used in regions, where no Smartphones are accessible. More complex input requires also more educated volunteers.
(12) The good side on NYSS is, that everything is automated
(43) NYSS itself is relatively new. March 2020 was the first time it was used in its current form.

(16) fusion with other datasets can be challenging and is a lot of work
(87) integration of this into NYSS is work and it needs to be discussed who does it and who pays for it.

-	(49) Automatic integration with other data, e.g. from the Ministry is laborious and can be complicated. Though it’s doable.

-	(58) The platform is not made for individual, personal data.
-	(64) data collection in itself is not possible via NYSS.
-	(56) Server from NYSS is in Ireland for data protection reasons and its easier to maintain for the developers.
-	(57) The server location can be an issue for the Ministry as they do not have control over the data. The ownership of the data lies with the National Society. 



%%%%%%%%%%%%%%%%%%%%%%%%%%%%%%%%%%% INTERVIEW I3 BELEDI %%%%%%%%%%%%%%%%%%%%%%%%%%%%%%%%%%%%%%%%%%%

(1) CBS started in 2018 in Burao with 75 community volunteers because of the 2017 cholera outbreak in the same region
-(5, 6, 7) Oral Rehydration Points are in the community and run by volunteers of the community who get trained and then support the community by education and promotion of e.g. WASH activities and going to the actual treatment centers.
-(11, 14) Response happens together with other partners such as the government, MoH, WHO, and other sister RCRC organisations.
-(25) The idea to implement CBS came from the will to identify the cases in the community early enough to respond at community level and stop an outbreak immediately.”
-(33) NYSS was developed over the course of a year with a lot of feedback from the SRCS and is now “very effective and very supportive.”
-(34) NYSS is highly automatized and alerts are given when thresholds are reached based on geographical location.
-(36) “So any mobile you can use it.”
-(37) “No need to have a smartphone but you are using SMS.”
-(38) must be a network in that area and again SMS Eagle.”
-(39) Before the automatization, they downloaded the data from the NYSS platform and analyzed it via Excel.
-(58) Feedback on wrong reports is given by the regional supervisor very timely and they support the volunteers to send the report in the right format.






\subsection{Stage 6: Evaluation}
-> community meetings with the elders every month or so
-> feedback messages NYSS and so on
-> fraisl: ongoing effort + metrics for measuring success --> implemented in NYSS -> reliance, completeness and quality of the contributions and the activity of the volunteers


%%%%%%%%%%%%%%%%%%%%%%%%%%%%%%%%%%% INTERVIEW I1 RICHARD %%%%%%%%%%%%%%%%%%%%%%%%%%%%%%%%%%%%%%%%%%





%%%%%%%%%%%%%%%%%%%%%%%%%%%%%%%%%%% INTERVIEW I1.2 RICHARD %%%%%%%%%%%%%%%%%%%%%%%%%%%%%%%%%%%%%%%%

-	Utilise the community based SRCS volunteers to engage communities and sensitise the communoties on the riole the SRCS plays. Also establishing a robust feedback and Complaints mechanism that ensures communities can easily relay their feedback.




%%%%%%%%%%%%%%%%%%%%%%%%%%%%%%%%%%% INTERVIEW I2 JUNG %%%%%%%%%%%%%%%%%%%%%%%%%%%%%%%%%%%%%%%%%%%%%

"Usually, volunteers are chosen by the elders and the main criterium is not them being the smartest."
(28, 29) It should be communicated, that reports will be checked by the supervisor in order to prevent false reports in hope of more water. If this happens frequently, a solution must be conceptualized.

(32) Should work with prior training, supervision and feedback.

(34) Additional communication via phone is useful and necessary especially for details and instant feedback e.g. to communicate lack in response and its reasons.

-	(109) SRCS invests a lot into communication and feedback with the communities.
-	(113) SRCS are no rookies. They know how to communicate – big part of their culture.
(27) Other or more information besides the codes could be clarified via phone and inserted by the supervisor manually.

-	The option for feedback messages comes from discussions with the SRCS
-	(60) An evaluation has been conducted but not yet made public.
-	(66) SRCS knows their business. They are no rookies.

(53) Supervisor can validate reports e.g. vie phone.

-	(109) SRCS invests a lot into communication and feedback with the communities.
-	(113) SRCS are no rookies. They know how to communicate – big part of their culture.



%%%%%%%%%%%%%%%%%%%%%%%%%%%%%%%%%%% INTERVIEW I3 BELEDI %%%%%%%%%%%%%%%%%%%%%%%%%%%%%%%%%%%%%%%%%%%
-(13) Reports are verified by the regional officer.
-(58) Feedback on wrong reports is given by the regional supervisor very timely and they support the volunteers to send the report in the right format.
-(43) SRCS has a good connection to the local communities and thus they get direct feedback about water shortages.
-(43) SRCS has good relations with the communities.
-(44) “community leaders are the one who tells the SRCS or other partners or the government that there is a water shortage”



\section{Summary Results + key lessons learned (?)}
%%%%%%%%%%%%%%%%%%%%%%%%%%%%%%%%%%% INTERVIEW I1 RICHARD %%%%%%%%%%%%%%%%%%%%%%%%%%%%%%%%%%%%%%%%%%





%%%%%%%%%%%%%%%%%%%%%%%%%%%%%%%%%%% INTERVIEW I1.2 RICHARD %%%%%%%%%%%%%%%%%%%%%%%%%%%%%%%%%%%%%%%%






%%%%%%%%%%%%%%%%%%%%%%%%%%%%%%%%%%% INTERVIEW I2 JUNG %%%%%%%%%%%%%%%%%%%%%%%%%%%%%%%%%%%%%%%%%%%%%







%%%%%%%%%%%%%%%%%%%%%%%%%%%%%%%%%%% INTERVIEW I3 BELEDI %%%%%%%%%%%%%%%%%%%%%%%%%%%%%%%%%%%%%%%%%%%













%%%%%%%%%%%%%%%%%%%%%%%%%%%%%%%%%%%%%%%%%%%%%%%%%%%%
summary
--> + review to the deductive hypothesis --> could the lit review and the interviews answer this? here or discussion?




% The concluding summary is very important because it summarises your key findings and lays the foundation for the discussion chapter. Keep in mind that some readers may skip directly to this section (from the introduction section), so make sure that it can be read and understood well in isolation.
% In this section, you need to remind the reader of the key findings. That is, the results that directly relate to your research questions and that you will build upon in your discussion chapter. Remember, your reader has digested a lot of information in this chapter, so you need to use this section to remind them of the most important takeaways.



























\section{Main Section 1}

% put this in the result section.
intro to Somalia EAP: https://docs.google.com/document/d/1xUEXm8RxVHTO468KqXSAoBX-cpkPwiff/edit
https://heigit.atlassian.net/wiki/spaces/FIS/pages/1704096/Indices

current EAP stage in somalia
--> Somalia so far. But: still under development.

“Hazards Exposure and Vulnerability” ([Somali Red Crescent Society, 2022, p. 13](zotero://select/groups/4773535/items/FZ6BJHJA)) ([pdf](zotero://open-pdf/groups/4773535/items/RJKNZZZ2?page=17&annotation=IRH526LN))

“Feasibility Study on Potential Use of Forecast-based Financing (FbF) for SRCS Final Report” ([Somali Red Crescent Society, 2022, pp. -3](zotero://select/groups/4773535/items/FZ6BJHJA)) ([pdf](zotero://open-pdf/groups/4773535/items/RJKNZZZ2?page=1&annotation=KHCH33GX)





% Quality criteria for Early Action Protocols
https://heigit.atlassian.net/wiki/download/attachments/1704186/FbA-EAP-criteria-May-2022.docx?version=1&modificationDate=1677660171372&cacheVersion=1&api=v2
(summery available -> confluence)


%-----------------------------------
%	SUBSECTION 2
%-----------------------------------







%----------------------------------------------------------------------------------------
%	SECTION 2
%----------------------------------------------------------------------------------------

\section{}


%----------------------------------------------------------------------------------------
%	SECTION 1
%----------------------------------------------------------------------------------------

\section{Case study protocol}

%----------------------------------------------------------------------------------------
%	SECTION 1
%----------------------------------------------------------------------------------------

\section{Main Section 1}

%----------------------------------------------------------------------------------------
%	SECTION 1
%----------------------------------------------------------------------------------------

\section{Main Section 1}

%----------------------------------------------------------------------------------------
%	SECTION 1
%----------------------------------------------------------------------------------------



"5.3 Types of water resources monitoring
As it has been indicated above, water resources monitoring provides information on the state and trends of quantitative and qualitative characteristics of the monitored object, level and distribution of anthropogenic loads, state of ecosystems and a degree of possibility of satisfaction of various needs in water, municipal and economic.

There are three main types of water resources monitoring, used in the water resources management system:
%% main points:
local monitoring, performed for solution of specific local problems on a limited part of the water body or the territory;
global (background) monitoring, performed at man-impact free sites, or on sites with low level of anthropogenic influence. Such monitoring is performed for acquisition of information on steady natural characteristics of environmental components. The background monitoring of water bodies is used for evaluation and/or prognostication of shifts in their state caused by economic activities;
comprehensive (regime) monitoring performed at the water body observation network for determination of the actual state of the water body, for decision making on efficient use, protection, and restoration of water resources;
critical or alarm monitoring, performed at sites of high risk for immediate warning about unfavourable situations caused primarily by human activities.
In the water management system there is also a special type of monitoring for wastewater discharges to the water body."https://echo2.epfl.ch/VICAIRE/mod_4/chapt_5/main.htm

"As a rule, the following parameters are always monitored:

Water Temperature.
Transparency or Turbidity.
pH.
Conductivity.
Dissolved oxygen (DO).
Total phosphorus.
Total nitrogen.
Nitrogen, Ammonia.
Nitrogen, Nitrate.
Soluble Reactive Phosphorus.
Faecal coliform bacteria." https://echo2.epfl.ch/VICAIRE/mod_4/chapt_5/main.htm
%% --> not gonna happen but still interesting

https://www.oxfamwash.org/water/cbwrm/Oxfam%20CBWRM%20Companion,%202009.pdf
%----------------------------------------------------------------------------------------
%	SECTION 1
%----------------------------------------------------------------------------------------

\section{Main Section 1}

%----------------------------------------------------------------------------------------
%	SECTION 1
%----------------------------------------------------------------------------------------

\section{NOTES}

discussion: the SLMC was good until general activities - then: it has to become quite detailed --> not feasible in the scope of this work. but a good framework to continue the work with.
-> size of Berkad is detrimental! --> Richard water levels are for a week. Improved ones are for a couple of months --> water level is only one variable! needs to be correlated with the extraction and storage size

- BRCiS: "Cash transfers were "identified across several clusters as the preferred action" where local markets and the operational context allow" (TB) --> but only there and market prices are very high when drought sets in
--> Beledi highlights help throughout the society -> but water prices rise very high.. 

- better comparison between the comparable projects of NADFOR, MoWR and OCHA was not possible due to lack of raw data --> no interviews

% I1
-results for the implementation of AAs:
o	i) Determining the water level to trigger action
o	ii) water levels monitoring
o	iii) triggering action based on water levels

% I2
(104) MoH has good experiences with NYSS in the health sector. These experiences should be translated and widened on other topics.

% zu stage 2
The funding of a pilot study as a decisive factor could not be clarified in this framework, but does not affect the further conception. --> though there is an ongoing proposal which would build on top of this work (if granted).

% APPEDNIX
% include questions, questionnaires, transcripts and codes
The first interview came about through existing contacts of the project in which this work is embedded, and the interviewee was the project leader of the FbF approach in the \acrshort{srcs} (I1). In the further course, the CBS project manager on the Norwegian Red Cross side (I2) and the CBS manager on the Somali side (I3) were also interviewed. Between these two interviews, there was a second interview with the project manager of the \acrshort{srcs}' \acrshort{fbf} team (I1.2).


% about SLMC
-> sub-sub-activities or products could also be labelled differently.. not all levels were clear and distinct

--> Methods, techniques, tools and scripts need to be created/developed -> therefore sub-goal instead of following the SLMC --> believed to: better overview, prerequisites -> important



% while multiple other frameworks and guidelines already exist, their numrous number is also explained/reflected by the need, that the frameworks need to be focussed on a specific topic, region and environment in order to give meaningful advice and not only generic information that is too coarse to be of great use. Therefore, -> new development and adjustments --> fraisl was e.g. focused on environmental and ecological stuff, CBS was too heavy on the social side e.g. patient privacy and case handling is way less important in this case



% recommended to combine the BRCiS appraoch with I1 categorization. one more extensive long-term assessment at the end of the rain season together with the other rarely changing indicators (e.g. number of animals and people) but also combined with an triangulation with stakeholder and thorough analysis and interpretation of data --> integration into decision-making processes. --> seasonal warning and identification of vulnerability + short-term facilitation of immediate action with the short-term water level warnings

% regular reporting with event based reporting as water in Berkads can e.g. dry up quicker than predicted or turn bad or what ever else..
% % Chapter Template

\chapter{Discussion} % Main chapter title

\label{Chapter5} % Change X to a consecutive number; for referencing this chapter elsewhere, use \ref{ChapterX}

%----------------------------------------------------------------------------------------
%	SECTION 1
%----------------------------------------------------------------------------------------

\section{Main Section 1}
as: preparatory study (??)
%-----------------------------------
%	SUBSECTION 1
%-----------------------------------
"in your discussion, include a paragraph on strengths and weaknesses that you have discovered as a result of carrying out the research, say what you would do differently, and what would lend itself to further research."


\subsection{Discussion indices and forecasts}

localization of forecasts and so on.. why it is difficult to use SPEI, SPI and so on and others.. --> more data to operationalize is very important, even though it should not be the task of the SRCS --> high costs but a lot of trust in the communities and CBS is also working great.
drought indicator (what exists so far? -> pre study)

% drought monitoring tool for somalia
https://cdi.faoswalim.org/index/rfe-maps/2022

% Drought Assessment and Monitoring using some drought indicators in the semi-arid Puntland state of Somalia
https://d1wqtxts1xzle7.cloudfront.net/61927859/DroughtAssessmentandMonitoringUsingSomeDroughtIndicatorsintheSemi-AridPuntlandStateofSomalia_FEB_2019NovemberIssue11A20200129-7600-1cwi1.pdf?1580293969=&response-content-disposition=inline%3B+filename%3DDROUGHT_ASSESSMENT_AND_MONITORING_USING.pdf&Expires=1678353893&Signature=U3ArqyQZMaz2Pbm6iwGv-0qMl0meKe6028igIkC5qjjTYaQ4OtNvKRiAKBzGNVWYCOBNxMmOtgriFg7Mavekgov3BRt38tG-e7tD86OYU4qS0zQfshEmQxBD9tM~CqqO45XBASpGpuQFl5Atks6ADcjQef03Cds7bRoTCrwyz-rOJvHo3nyHk~lbfupqWrgYzZ6oE~YTmUC6iQ7xNtKf-XNcjW4z6R7Rtvu1NXvoX~YWLFRrMa50O3kTUISwEEsD-Rw7QiVNPF~VBOa2~PepbkueXRRGuCRolvb1q95DQVqHXQkaS6jG1UUvJ-vrmqH0wAX-zmo-r31yuGQLbkQ5UQ__&Key-Pair-Id=APKAJLOHF5GGSLRBV4ZA

but: 
“Scale is critical in assessing water security [31]. National level assessments make it difficult to take action at operationalization level.” ([Mishra et al., 2021, p. 8](zotero://select/groups/4773535/items/MD2Z2HTF)) ([pdf](zotero://open-pdf/groups/4773535/items/366Z36U7?page=8&annotation=6IAXCXUB))

and: “Creating and using indicators for water security has to be directed towards some management control or assessment action.” ([Mishra et al., 2021, p. 8](zotero://select/groups/4773535/items/MD2Z2HTF)) ([pdf](zotero://open-pdf/groups/4773535/items/366Z36U7?page=8&annotation=P72LT9Y8))


% maybeeeeee and maybe not.. tend not to include it
“2 Local knowledge in drought monitoring: an introduction to the literature review” ([Giordano et al., 2013, p. 526](zotero://select/groups/4773535/items/B7LM5ZR4)) ([pdf](zotero://open-pdf/groups/4773535/items/7I66DBIK?page=4&annotation=Z33M5FLQ))

“There has been little effort to align the spatiotemporal granularity of socioeconomic assessments with the granularity of weather or climate monitoring.” ([Enenkel et al., 2020, p. 1161](zotero://select/groups/4773535/items/RX575C79)) ([pdf](zotero://open-pdf/groups/4773535/items/XD499UNK?page=1&annotation=QBTLFCXM))
“we highlight the need to collect and analyze environmental and socioeconomic data together and discuss novel strategies for coordinated data collection via mobile technologies from a drought risk management perspective.” ([Enenkel et al., 2020, p. 1161](zotero://select/groups/4773535/items/RX575C79)) ([pdf](zotero://open-pdf/groups/4773535/items/XD499UNK?page=1&annotation=9BUBHWNB))

“but questions related to coping capacities, migration, poverty, water supply, access to food and markets, or political conflict remain unanswered or are even decoupled from routine drought risk assessments” ([Enenkel et al., 2020, p. 1162](zotero://select/groups/4773535/items/RX575C79)) ([pdf](zotero://open-pdf/groups/4773535/items/XD499UNK?page=2&annotation=HE48ZWFA))

“The handbook of drought indicators (Svoboda et al. 2016) lists more than 50 drought indices. Not a single one of these indices connects climate anomalies to socioeconomic vulnerabilities,” ([Enenkel et al., 2020, p. 1163](zotero://select/groups/4773535/items/RX575C79)) ([pdf](zotero://open-pdf/groups/4773535/items/XD499UNK?page=3&annotation=EDKZFHJX))
--> there were more indicators in the following year, but those were limited due to availibiltiy of socioeconomic data


% List of drought indicators
https://heigit.atlassian.net/wiki/spaces/FIS/pages/1704096/Indices


% e.g. SPEI has difficulties because of the scarce hydro-meteorological monitoring network
%SPEI-based spatial and temporal evaluation of drought in Somalia
http://www.repository.embuni.ac.ke/bitstream/handle/embuni/3705/SPEI-based%20spatial%20and%20temporal%20evaluation%20of%20drought%20in%20Somalia.pdf?sequence=1&isAllowed=y
% notes from the team
evaluate the spatio-temporal variations of drought occurrences in Somalia for the period between 1980 and 2015 as quantified by Standardized Precipitation Evapotranspiration Index (SPEI)
The temporal variations in drought showed decreasing trends in severity and increasing trends in drought duration as the SPEI timescales increas
However, the study was limited by the low security levels in Somalia that makes
the country largely inaccessible. This has complicated possibilities for
ground-truthing available information. Besides, the 1991 civil war in
Somalia led to the collapse of hydro-meteorological monitoring network
(Muchiri, 2007). The current stations established by Food and Agricul-
ture Organization Somalia Water and Land Information Management
(FAO SWALIM) from 2002 do not have uniform spatial distribution
throughout the country and therefore cannot be useful in computing
reference SPE

% this level of impact can then.. 
one can access the impact of a certain weather phenomenon or climate development on the situation on site. These information can then facilitate exact and precise assessments of the conditions and help to react accordingly (policy, management, etc. ) 
% possibly too much of a discussion.. 


% later on
However, it requires complex and various information and is thus often difficult to implement \autocite{liuWaterScarcityAssessments2017}.

... in the light of
While all of these more dedicated indicators may present great value, relatively simple Indices such as the SPEI is not possible for Somalia due to the lack of a good enough weather monitoring network. 

% this level of impact can then.. 
one can access the impact of a certain weather phenomenon or climate development on the situation on site. These information can then facilitate exact and precise assessments of the conditions and help to react accordingly (policy, management, etc. ) 
% possibly too much of a discussion.. 


... in the light of
While all of these more dedicated indicators may present great value, relatively simple Indices such as the SPEI is not possible for Somalia due to the lack of a good enough weather monitoring network. 



research vs. reality

other perspectives than process oriented -> 7-layer model self-critisism (resource, behavioral perspectives, + value and communication network design approaches)

“It is important to note that local indicators cannot be collected specifically for the FbF system by RCRC national societies.” ([RCRC, 2020, p. 30](zotero://select/groups/4773535/items/UESIQTRJ)) ([pdf](zotero://open-pdf/groups/4773535/items/P5JPVZ97?page=30&annotation=LRDPV2M7))

“Indeed, collecting data on local indicators would require from the national society a team of enumerators that work continually to collect and process that information in all places where the program could possibly trigger (e.g. collect food price information for every village market). This would have extensive cost implications and likely over-burden the national society staff and volunteers.” ([RCRC, 2020, p. 30](zotero://select/groups/4773535/items/UESIQTRJ)) ([pdf](zotero://open-pdf/groups/4773535/items/P5JPVZ97?page=30&annotation=2YIIK6ZY))

“As such, the inclusion of local indicators into an FbA trigger must involve assessing what indicators are relevant for the impacts the program is trying to anticipate and identify which of those indicators are already collected (e.g. the ministry of agriculture's food price bulletin) and are available at the time they would be needed to inform a possible trigger.” ([RCRC, 2020, p. 30](zotero://select/groups/4773535/items/UESIQTRJ)) ([pdf](zotero://open-pdf/groups/4773535/items/P5JPVZ97?page=30&annotation=7X3RFGVB))

“Thinking outside the box in terms of both hydro-meteorological and socio-economic indicators could be particularly useful” ([RCRC, 2020, p. 31](zotero://select/groups/4773535/items/UESIQTRJ)) ([pdf](zotero://open-pdf/groups/4773535/items/P5JPVZ97?page=31&annotation=GNZJ3FR5))


%-----------------------------------
%	SUBSECTION 2
%-----------------------------------


\subsection{FbF, EAP and actions}


,“Climate information presented as early warnings are only as valuable as the actions that are taken in response to the information, even if the information is a perfect warning of future events.” ([Mariani et al., 2015, p. 8](zotero://select/groups/4773535/items/8THVVJVK)) ([pdf](zotero://open-pdf/groups/4773535/items/GYUFNK32?page=8&annotation=Y9DG5FSE))



%During the development of an \acrshort{eap} more and complex challenges need to be addressed such as the lack of coordination and integration between sectors, limited capacities on the ground to decide and react accordingly due to data and resource shortages as well as the reality that they often work in areas where they have to operate in a complex network of actors and conflicting interests.

% --> language is important. --> staggered trigger -> cannot call everything an action! There is a difference between small and large repsonsen etc.. especially in the perception of people
“We also found that the phrases we had been using to trigger action, Be Aware, Be Prepared and Take Action, were not driving the actions we had hoped. We used Take Action exclusively for high impact warnings, but we need people to be taking action for low and medium impact warnings too, if the impacts are to be mitigated. We have replaced these phrases with detailed impact information and linked our warnings to advice and guidance from our partner organisations.” ([Harrowsmith et al., 2020, p. 62](zotero://select/groups/4773535/items/QJ397Y54)) ([pdf](zotero://open-pdf/groups/4773535/items/2GS362N5?page=62&annotation=KZK3TSK3))


%%%%%%%%%%%%%%%%%%%%%%%%%%%%%%%%%%%%%%%%%%%%%%%%%
%%%%%%%%%%%% trigger:
“Given the different layers of complexity with drought, different types of triggers may be required beyond what is often used in EAP development. For instance, unconventional triggers for FbA for drought could include metrics such as staple food prices, percentages of crop failure, and other elements of food security early warning systems.” ([RCRC, 2020, p. 30](zotero://select/groups/4773535/items/UESIQTRJ)) ([pdf](zotero://open-pdf/groups/4773535/items/P5JPVZ97?page=30&annotation=JZV26DPP))
% --> even the RCRC is still looking for good triggers -> maybe water levels are a good way -> reasoning for this study


\subsection{Citizen Science -> CBS and Tools}
This works follows the “general public, typically as part of a collaborative project with professional scientists” ([Kirschke et al., 2022, p. 2](zotero://select/groups/4773535/items/GPC3LDT5)) ([pdf](zotero://open-pdf/groups/4773535/items/AI7HRQYC?page=2&annotation=BTE2JFKV)) which is basically “community-based monitoring, which also explicitly focuses on the public ’s involvement in monitoring processes” ([Kirschke et al., 2022, p. 2](zotero://select/groups/4773535/items/GPC3LDT5)) ([pdf](zotero://open-pdf/groups/4773535/items/AI7HRQYC?page=2&annotation=K4URWWLV))

“Local citizens are seen as essential participants in collaborative environmental management because they can provide vital information about the area’s natural and sociopolitical systems as well as support for measures to address non–point source pollution” ([Koehler and Koontz, 2008, p. 143](zotero://select/groups/4773535/items/2PKW9CYX)) ([pdf](zotero://open-pdf/groups/4773535/items/YYG4JK3R?page=1&annotation=MV6JAERB))


“Furthermore it is an interesting communication tool in the light of science communication. Correspondingly water managers should be interested in participatory monitoring in the light of integrated water management.” ([Minkman, 2015, p. 199](zotero://select/groups/4773535/items/ZKLE6CPT)) ([pdf](zotero://open-pdf/groups/4773535/items/QMAPCSZG?page=199&annotation=GHI9KSDA))

% in comparison to sensor networks:
“It is suggested, based on literature, that water authorities might prefer MCS to wireless sensor networks, for its mobility, lower costs and scalability.” ([Minkman, 2015, p. 11](zotero://select/groups/4773535/items/ZKLE6CPT)) ([pdf](zotero://open-pdf/groups/4773535/items/QMAPCSZG?page=11&annotation=RM5PX55M))

%%%%%%%%%%%%%%%%%%%%%%%%%%%%% water scarcity and rural water sources
“Community cultures, economies, and environments differ across countries and regions. These differences should be considered when designing hybrid management strategies, so that all actors can be appropriately enabled and the mechanism which is most effective for the given community can be identified.” ([Huang et al., 2020, p. 147](zotero://select/groups/4773535/items/9CSBLJNJ)) ([pdf](zotero://open-pdf/groups/4773535/items/G5BEZQ7C?page=12&annotation=WV5DXV5I))

“On this basis, it is essential to expand research area to study the various threats from climate variability to rural drinking water safety, and then to develop corresponding measures to address those threats to water security.” ([Huang et al., 2020, p. 147](zotero://select/groups/4773535/items/9CSBLJNJ)) ([pdf](zotero://open-pdf/groups/4773535/items/G5BEZQ7C?page=12&annotation=HCQHR7YR))

-> can change the way the water sector is funded: “With these data readily available, performance-related contracts that incentivize sustainable service delivery over short-term infrastructure investment can become the norm.” ([Thomson, 2021, p. 11](zotero://select/groups/4773535/items/UQLXVVYI)) ([pdf](zotero://open-pdf/groups/4773535/items/K9XBXPQD?page=11&annotation=JP9R4Y89))


“Our results indicate that using the phones to transmit more than just water quality data will likely improve the effectiveness and sustainability of this type of intervention.” ([Kumpel et al., 2015, p. 10846](zotero://select/groups/4773535/items/GPM4C7RJ)) ([pdf](zotero://open-pdf/groups/4773535/items/7VXVKEXK?page=1&annotation=4DJIADX2))

\subsection{Own results and how the relate to all of this shizzle}


\subsection{water source/access monitoring etc.}

%core paper:
“Community-based water resources management” ([Day, 2009, p. 47](zotero://select/groups/4773535/items/YWSNQ8A2)) ([pdf](zotero://open-pdf/groups/4773535/items/ETPCI5RI?page=2&annotation=RQLJMKL7))


“Global Monitoring of Water Supply and Sanitation: History, Methods and Future Challenges” ([Bartram et al., 2014, p. 8137](zotero://select/groups/4773535/items/6AWUJTW5)) ([pdf](zotero://open-pdf/groups/4773535/items/BFNSQGWS?page=1&annotation=ZWSBJVDM))

generally: one should look out for eurocentric views and perceptions / attitudes
Afrotopia

%%%%%%%%%%%%%%% CBS important papers: 2 systematic reviews
“Community-based surveillance of infectious diseases: a systematic review of drivers of success” ([McGowan et al., 2022, p. 1](zotero://select/groups/4773535/items/P74WDM6C)) ([pdf](zotero://open-pdf/groups/4773535/items/NP79EIIE?page=1&annotation=XJRZ7YIR))
“The Landscape of Participatory Surveillance Systems Across the One Health Spectrum: Systematic Review” ([McNeil et al., 2022, p. 1](zotero://select/groups/4773535/items/LVLYX8N5)) ([pdf](zotero://open-pdf/groups/4773535/items/4YG35TC6?page=1&annotation=IMCGBJLB))

-> "map closely to principles of participatory community engagement" (McGowan 2022 p.1)
+ “Other factors included: strong supervision and training, a strong sense of responsibility for community health, effective engagement of community informants, close proximity of surveillance workers to communities, the use of simple and adaptable case definitions, quality assurance, effective use of technology, and the use of data for real-time decision-making.” ([McGowan et al., 2022, p. 1](zotero://select/groups/4773535/items/P74WDM6C)) ([pdf](zotero://open-pdf/groups/4773535/items/NP79EIIE?page=1&annotation=FLMKXLLM))
-> tech

% compare CS with sensor networks
“Moving beyond the Technology: A Socio-technical Roadmap for Low-Cost Water Sensor Network Applications” ([Mao et al., 2020, p. 9145](zotero://select/groups/4773535/items/VEJUB8D9)) ([pdf](zotero://open-pdf/groups/4773535/items/KMCIFXB8?page=1&annotation=2A7QA3V3))




\subsection{Subsection 2}

application design 
“Citizen Science experiments have demonstrated that at least three issues must be addressed: commitment of people participating in data collection (were social component is key), the validation of data collected and sent by citizens and privacy concerns about the collection of personal information.” ([Alfonso and Jonoski, 2012, p. 6](zotero://select/groups/4773535/items/4W4Q8E6B)) ([pdf](zotero://open-pdf/groups/4773535/items/PU944FGI?page=6\&annotation=LZXCJDFB))

“Households’ perceptions of their drinking water quality were mostly influenced by the water’s visual appearance, and these perceptions in general motivated their use of HWT.” ([Daniel et al., 2020, p. 1](zotero://select/groups/4773535/items/ZA5MKHPC)) ([pdf](zotero://open-pdf/groups/4773535/items/CS4A7XIN?page=1&annotation=KHJHJFPM))

“The findings confirm other studies on technology acceptance, as the mobile crowd sensing technology should be useful rather than free of effort.” ([Minkman, 2015, p. 12](zotero://select/groups/4773535/items/ZKLE6CPT)) ([pdf](zotero://open-pdf/groups/4773535/items/QMAPCSZG?page=12&annotation=IQEBUPYG))



“For external experts, they may also benefit from community support to inform scientific processes, such as collecting data that spans across a large geographic region and having an enhanced understanding of community interests.” ([Huang et al., 2020, p. 144](zotero://select/groups/4773535/items/9CSBLJNJ)) ([pdf](zotero://open-pdf/groups/4773535/items/G5BEZQ7C?page=9&annotation=QPIKDKB9))
“Management of Drinking Water Source in Rural Communities under Climate Change” ([Huang et al., 2020, p. 136](zotero://select/groups/4773535/items/9CSBLJNJ)) ([pdf](zotero://open-pdf/groups/4773535/items/G5BEZQ7C?page=1&annotation=92FTSBXQ))
“For local communities, their needs of safe drinking water could be met and their abilities to manage and maintain water supply could be enhanced.” ([Huang et al., 2020, p. 144](zotero://select/groups/4773535/items/9CSBLJNJ)) ([pdf](zotero://open-pdf/groups/4773535/items/G5BEZQ7C?page=9&annotation=AL2NB5DK))
“Furthermore, this model could help increase scientific awareness among community members and engage the community with the environment.” ([Huang et al., 2020, p. 144](zotero://select/groups/4773535/items/9CSBLJNJ)) ([pdf](zotero://open-pdf/groups/4773535/items/G5BEZQ7C?page=9&annotation=FYU4BIKP))
“During the management of rural drinking water sources, a hybrid modality in which community management is the mainstay with supplement from external support from other organizations is highly recommended.” ([Huang et al., 2020, p. 147](zotero://select/groups/4773535/items/9CSBLJNJ)) ([pdf](zotero://open-pdf/groups/4773535/items/G5BEZQ7C?page=12&annotation=SQ6P8UBN)) % may also be a good transition to CBWM
% --> water related stuff --> subject is generally well researched and tried out in other circumstances and contexts




% --> further comparison though possibly also good in the discussion section (?)
“When Are Mobile Phones Useful for Water Quality Data Collection? An Analysis of Data Flows and ICT Applications among Regulated Monitoring Institutions in Sub-Saharan Africa” ([Kumpel et al., 2015, p. 10846](zotero://select/groups/4773535/items/GPM4C7RJ)) ([pdf](zotero://open-pdf/groups/4773535/items/7VXVKEXK?page=1&annotation=M5J4FFSH))


%----------------------------------------------------------------------------------------
%	SECTION 2
%----------------------------------------------------------------------------------------

\section{Main Section 2}

There is a need for research exploring if marginalized perspectives are excluded in crowdsourcing and self-reporting approaches, the recall bias and measurement error in self-reporting (Bell et al. 2019), and if incentives, memory triggers, or other mechanisms can be used to address these issues.


local knowledge
gender inequalities

etc. not really mentioned but have huge potential to improve all of this. Big question is HOW!
%----------------------------------------------------------------------------------------
%	SECTION 2 Future outlook
%----------------------------------------------------------------------------------------

“5.1.3. Challenges and Future Directions” ([Zheng et al., 2018, p. 721](zotero://select/groups/4773535/items/LJU68CG4)) ([pdf](zotero://open-pdf/groups/4773535/items/U8QNZLI6?page=24&annotation=D54SWR46))

“Connecting top-down weather and climate data with bottom-up socioeconomic data via machine learning” ([Enenkel et al., 2020, p. 1166](zotero://select/groups/4773535/items/RX575C79)) ([pdf](zotero://open-pdf/groups/4773535/items/XD499UNK?page=6&annotation=BCGJNKZB))

% thus nice in theory, but not useful in practice
% in regard to drought and water scarcity
% and while there is an extensive body of literature about these topics, the minor details are not of great interest to this work but the general conclusion, that physical, large scale drought or water scarcity indicators do not capture the required level of detail and impact that is needed to operationally act upon. Also, while the complexity of these concepts is due to the level of complexity of the surveyed phenomenon, its application and comparison is hindered. Thus a method to assess local impact, that builds and incorporates these concepts in a practically applicable manner is needed to adequately address this detrimental topic. 

% self surveying is generally not recommended:
“Indeed, collecting data on local indicators would require from the national society a team of enumerators that work continually to collect and process that information in all places where the program could possibly trigger (e.g. collect food price information for every village market). This would have extensive cost implications and likely over-burden the national society staff and volunteers.” ([RCRC, 2020, p. 30](zotero://select/groups/4773535/items/UESIQTRJ)) ([pdf](zotero://open-pdf/groups/4773535/items/P5JPVZ97?page=30&annotation=2YIIK6ZY))

“As such, the inclusion of local indicators into an FbA trigger must involve assessing what indicators are relevant for the impacts the program is trying to anticipate and identify which of those indicators are already collected (e.g. the ministry of agriculture's food price bulletin) and are available at the time they would be needed to inform a possible trigger.” ([RCRC, 2020, p. 30](zotero://select/groups/4773535/items/UESIQTRJ)) ([pdf](zotero://open-pdf/groups/4773535/items/P5JPVZ97?page=30&annotation=7X3RFGVB))


% --> thus - monthly review of users is necessary
“In a region where migration is one of the main coping mechanisms for drought, a targeted survey focusing on the early detection of migration movements would help mobilize the timely allocation of resources by humanitarian decision-makers or even the mitigation of drought impacts.” ([Enenkel et al., 2020, p. 1167](zotero://select/groups/4773535/items/RX575C79)) ([pdf](zotero://open-pdf/groups/4773535/items/XD499UNK?page=7&annotation=N9FRDA9C))



% keeping data up to date is crucial in ensuring correct vulnerability and exposure data 
“Vulnerability and exposure changes over time, particularly after an extreme weather or climate event. Datasets must be kept up to date to ensure the impact-based forecast or warning using this data is reliable. Recognise that many official governmental data sources, such as a national census or demographic and health surveys, are updated infrequently – every five or ten years.” ([Harrowsmith et al., 2020, p. 28](zotero://select/groups/4773535/items/QJ397Y54)) ([pdf](zotero://open-pdf/groups/4773535/items/2GS362N5?page=28&annotation=5XVAQCTY))





% VULNERABILITY
%Understanding vulnerability and resilience in Somalia
https://www.ncbi.nlm.nih.gov/pmc/articles/PMC7768599/pdf/JAMBA-12-856.pdf

“Without scope to accommodate dynamic vulnerabilities, actions cannot be effectively targeted or may prove ineffective.” ([Boult et al., 2022, p. 4](zotero://select/groups/4773535/items/B2AQSTYL)) ([pdf](zotero://open-pdf/groups/4773535/items/W9TFLH43?page=4&annotation=YYURM2E3))




\subsection{Limitations}
“Citizen science has several limitations, including the wide range of required skills outside the research subject, sustaining engagement, biases related to data collection and analysis, sensor calibration issues and varying data privacy regulations around the world, among others.” ([Fraisl et al., 2022, p. 13](zotero://select/groups/4773535/items/FBJD7SWS)) ([pdf](zotero://open-pdf/groups/4773535/items/7WBDKYDY?page=13&annotation=WNQZRHRF))

% % Chapter Template

\chapter{Conclusion} % Main chapter title

\label{ChapterX} % Change X to a consecutive number; for referencing this chapter elsewhere, use \ref{ChapterX}

%----------------------------------------------------------------------------------------
%	SECTION 1
%----------------------------------------------------------------------------------------

\section{Conclusion}

%-----------------------------------
%	SUBSECTION 1
%-----------------------------------
\subsection{Subsection 1}


%-----------------------------------
%	SUBSECTION 2
%-----------------------------------

\subsection{Subsection 2}

%----------------------------------------------------------------------------------------
%	SECTION 2
%----------------------------------------------------------------------------------------

\section{Future Outlook}




% % Chapter Template

\chapter{Discussion and Justification} % Main chapter title

\label{chapter5} % Change X to a consecutive number; for referencing this chapter elsewhere, use \ref{ChapterX}

this is a citation. \autocite{fraislCitizenScienceEnvironmental2022} 
% % Chapter Template

\chapter{Theoretical Background} % Main chapter title

\label{ChapterX} % Change X to a consecutive number; for referencing this chapter elsewhere, use \ref{ChapterX}

%----------------------------------------------------------------------------------------
%	SECTION 1
%----------------------------------------------------------------------------------------

TODO:

rework methods -> implement prc and so on
data analysis protocol (see joplin) or add to methods (?)

% naja check. Da steht zumindest was.. yikes.
discussion chapter
% ! Einleitung discussion section überarbeiten. Das klingt kacke.

% check
introduction - check

conclusion

outlook

abstract (at least a first draft)

% !!!!!
Eidesstattliche Erklärung!!
% !!!!!


% check
Zitationen fixen! noch vor den figures -> das Zeug gibt nur noch mehr Probleme

Zitation fixen -> nur letztes Wort von Nachnamen genannt

Zitationen im Text anpassen mit oder ohne Klammern etc... shizzle ehh

figures

tables!! -> vor allem der project analysis table ist wichtig! + data analysis table

map of Somaliland!!

Karte für case study area - ggf. auch für wasser resources (jedoch schwierig welchen Datensatz..)

% Eigene Result Abbildungen nochmals überprüfen und Einbinden

% alles überlesen -> nicht zu doll eskalieren beim Neuschreiben..

Appendix:
transcripts
results as png in large
questionnaires and interview guidelines (also for NYSS)

fix abbreviations (könnte sich auch zusammen mit den Zitationen erledigen.)

nette Email ans Prüfungsamt, Zipf und Lautenbach verfassen, DEN ANHANG NICHT VERGESSEN und abschicken.

Think about the title.. yoooo

Done.


wenn irgendmöglich.. so ein wenig tags für OSM raussuchen. Da kommt eh nix weiter bei rum.. #biasisreal








On the one hand, citizens can help to fill data gaps of categorized measurements such as simple assessments of dry-to-wet conditions which correspond to the above mentioned technical drought indicators \autocite{lackstromBackyardHydroclimatologyCitizen2022}. On the other hand, citizens can contribute their local knowledge which can potentially draw on years of experience and encompass a wide range of locally important aspects \autocite{butteFrameworkWaterSecurity2022,koehlerCitizenParticipationCollaborative2008,njambi-szlapkaIntegratingCommunityVoices}.








\section{How do I change the colors of links?}

The color of links can be changed to your liking using:

{\small\verb!\hypersetup{urlcolor=red}!}, or

{\small\verb!\hypersetup{citecolor=green}!}, or

{\small\verb!\hypersetup{allcolor=blue}!}.

\noindent If you want to completely hide the links, you can use:

{\small\verb!\hypersetup{allcolors=.}!}, or even better: 

{\small\verb!\hypersetup{hidelinks}!}.

\noindent If you want to have obvious links in the PDF but not the printed text, use:

{\small\verb!\hypersetup{colorlinks=false}!}.





\begin{table}
    \caption[Drought Types]{Different Types of Drought. Source: Own representation, based on \autocite{rcrcFORECASTBASEDFINANCINGEARLY2020,wilhiteUnderstandingDroughtPhenomenon1985}}
    \begin{adjustbox}{center,max width=\linewidth}
        \def\arraystretch{1.5}
        \begin{tabular}{m{2cm}m{5cm}}
            \toprule
            \bf Type of Drought & \bf Characteristics  \\ 
            \midrule
            Meteorological      & Duration and degree of dryness in comparison to the normal average, mostly based on precipitation and temperature  \\ 
            Agricultural        & Impacts on agriculture measured by low soil-moisture, evapotranspiration rate and soil water deficits  \\ 
            Hydrological        & Impact on surface and subsurface water by level of depletion  \\ 
            Socioeconomic       & Negative implications for and impacts on society when demand exceeds supply  \\ 
            \bottomrule
        \end{tabular}
    \end{adjustbox}
    \label{tab:th_drought_types}
\end{table}





% research objectives
1. To conduct a comprehensive review of existing literature and guidelines related to the design and implementation of \acrlong{cs} programmes, and try to align and apply these principles to the research aim and overall case study context.

2. To assess the feasibility of the \acrlong{cs} approach in the given context by identifying potential challenges and opportunities for successful implementation, and to propose recommendations for addressing these challenges.

3. if feasible, develop a replicable and adaptable framework for community-based participatory water source mapping and monitoring in the context of \acrlong{fbf}, based on the principles and recommendations identified in objectives 1 and 2.

4. To apply the adapted and developed frameworks in order to create a roadmap for the implementation of the proposed project, including specific products, actions, and stakeholders involved.

% research question
1. What specific guidelines and best practices exist for the design and implementation of community-based participatory water source mapping and monitoring programmes in resource-limited and water-scarce settings, and how can they be applied to the thematic direction of \acrlong{fbf}? % none -> wider angle -> FbF

2. Based on the identified frameworks and principles, which specific combinations of frameworks and guidelines are best suited for developing a replicable and adaptable community-based participatory water source mapping and monitoring design in resource-limited and water-scarce settings, while also ensuring feasibility for \acrlong{fbf}?

3. In the specific context of this case study, how can the developed framework be applied to create a tailored roadmap for the implementation of a community-based participatory water source mapping and monitoring project, including specific products, activities, and stakeholders involved in the project?



% ideas to the second questions

Based on a determined set of the identified frameworks and guidelines, how can a replicable and adaptable community-based participatory mapping and monitoring approach for water sources in resource-limited and water-scarce settings be designed for FbF application?

how can a replicable and adaptable community-based participatory water source mapping and monitoring design in a resource-limited and water-scarce setting for the application of FbF be conceptualised?


What are the key frameworks and recommendations identified in objectives 1 and 2 for designing and implementing a community-based participatory water source mapping and monitoring program in the context of FbF, and how can these be adapted to develop a replicable and adaptable framework for future implementation?



In order to meet this challenge, the \acrlong{rcrc} Movement together with the \acrlong*{rcrccc} started the \acrfull*{fbf} programme in 2007 to facilitate \acrlongpl{aa} instead of post-disaster reactions \autocite{ifrcForecastbasedFinancingNew2019}. Together with their local partners, the \acrfull*{ifrc} is working on establishing so called Early Action Protocols (EAPs) to ensure better organization and coordination of Anticipatory Actions in the face of an incoming hazard. These actions are based on a predefined interplay of forecast, trigger and financing mechanisms to ensure rapid, scientific based responses.\newline

Somaliland, being no exception to the above mentioned climatic trend, is characterized by droughts with far reaching impacts on ecological, economic, and social aspects \autocite{abdulkadirAssessmentDroughtRecurrence2017}. Defined by a semi-arid, four-season climate with two extensive dry seasons and an economic backbone of pastoralism and rain-fed agriculture, water accessibility is of key importance in Somaliland \autocite{abdulkadirAssessmentDroughtRecurrence2017,petrucciLandscapeLandformsNorthern2022,republicofsomalilandSomalilandCountryProfile2021}.

% impact of drought on the community
% --> impact forecasts
% --> mapping & monitoring of water source type Berkad

https://www.unwater.org/our-work/integrated-monitoring-initiative-sdg-6




“Countries in which less than 50\% of the population uses improved drinking water sources are all located in sub-Saharan Africa and Oceania 91-100\% 76-90\% 50-75\% <50\% insufficient data or not applicable Proportion of the population using improved drinking water sources in 2015 ■ 91–100\% ■ 76–90\% ■ 50–75\% ■ <50\% ■ INSUFFICIENT DATA OR NOT APPLICABLE” ([World Health Organization, 2016, p. 15](zotero://select/groups/4773535/items/KVAKZ9ZT)) ([pdf](zotero://open-pdf/groups/4773535/items/4STYK52H?page=14\&annotation=FBURDS4T))


In the Horn of Africa, much of the population still has no access to improved drinking water sources or in the case of Somalia,  


Nonetheless, direct contributions and communication from and with volunteers or community members remain a challenge in the joint management of hazards and risks. The tasks are numerous and need to take into consideration different aspects, ranging from cultural differences to different background knowledge and technical capabilities and capacities.

“Intervening early to respond to spikes in need – i.e. before negative coping strategies are employed - can deliver significant gains and should be prioritized.” ([USAID, 2018, p. 6](zotero://select/groups/4773535/items/LGRWAU43)) ([pdf](zotero://open-pdf/groups/4773535/items/MBXSCVWR?page=6&annotation=C47BGB9V))




Besides the further development of more fine grained technical solutions, the integration of local citizens is another way forward. Engaging local citizens and communities and giving them an active voice in defining and co-producing \acrshortpl{aa} and knowledge can be of multiple benefit to communities and enrich the data generated \autocite{somaliredcrescentsocietyFeasibilityStudyPotential2022, njambi-szlapkaIntegratingCommunityVoices}. 



On the one hand, citizens can help to fill data gaps of categorized measurements such as simple assessments of dry-to-wet conditions which correspond to the above mentioned technical drought indicators \autocite{lackstromBackyardHydroclimatologyCitizen2022}. On the other hand, citizens can contribute their local knowledge which can potentially draw on years of experience and encompass a wide range of locally important aspects \autocite{butteFrameworkWaterSecurity2022,koehlerCitizenParticipationCollaborative2008,njambi-szlapkaIntegratingCommunityVoices}. The \acrshort{ifrc} states, that the "community engagement and accountability (CEA) is essential […] to build acceptance and trust” for effective and sustainable outcomes \autocite{ifrcCommunityEngagementAccountability}.\newline



In the last two decades, \acrlong{cs} has become a vibrant area of scientific interest covering various aspects in many different contexts \autocite{kirschkeCitizenScienceProjects2022,kullenbergWhatCitizenScience2016}. Relatively recent developments in \acrlong{cbm} and \acrlong{mcs} now make it possible for a large number of citizens to contribute to scientific, social and environmental endeavours with just a simple phone \autocite{butteFrameworkWaterSecurity2022}. This engagement of the general public can have multiple benefits for a wide variety of aspects. Scientific processes of e.g. linking climate variability to local water security can be informed, the public's education and awareness about specific topics can be raised, and decision-making and overall management can be enhanced, if the project is embedded in these procedures \autocite{huangManagementDrinkingWater2020,kirschkeCitizenScienceProjects2022,minkmanCitizenScienceWater2015}.


Furthermore, \acrshort{cs} projects have demonstrated their ability to gather data and fill gaps particularly in formerly data sparse regions in an effective and cost-efficient manner \autocite{butteFrameworkWaterSecurity2022,lackstromBackyardHydroclimatologyCitizen2022,weeserCitizenSciencePioneers2018a}. However, currently \acrshort{cs} projects and studies are primarily located in North America, Europe and Australia \autocite{kirschkeCitizenScienceProjects2022, koehlerCitizenParticipationCollaborative2008, livinglakescanadaElevatingCommunityBased2018}. In the field of environment and water monitoring, these projects are mainly concerned about  \autocite{kirschkeCitizenScienceProjects2022}. Social.Water, CoCoRaHS and \autocite{speirSolutionsCurrentChallenges2022}'s study are examples of those environmental data collection and drought monitoring implementations focussing on monitoring river, lake, groundwater and precipitation levels. However, these approaches all require internet access and more technical equipment, making them unfeasible for low-income conditions \autocite{fienenSocialWaterCrowdsourcing2012a,lackstromBackyardHydroclimatologyCitizen2022,lowryGrowingPainsCrowdsourced2019}.\newline


% !
% !
% !
% !
% !
% !!!!!!!!!!!!!!!!!!!!!!!!!!!!!!!!!!!!!!!!!!!!!!!!!!!!!!!!!!!!!!!!!!!!!!!!!!!!!!!!!!!!!!!!!!!!!!

https://libguides.usc.edu/writingguide/discussion

"The stages of the roadmap outlined above give a good overview of what needs to be done and in what order. However, the process-oriented structure makes it difficult not to overlook important information, as there is no more detailed and grouped listing of this. This catalogue attempts to close this gap. " yeah well.. right mate but how good is it really?

more difficult to transfer to other contexts with concise and concrete requirements but: tried to keep them relative general by also not being to far off - future will show how well that worked.

Systematically explain the meaning of your case study findings and why you believe they are important. 



1. Most effectively demonstrates your ability as a researcher to think critically about an issue, to develop creative solutions to problems based upon a logical synthesis of the findings, and to formulate a deeper, more profound understanding of the research problem under investigation;
2. Presents the underlying meaning of your research, notes possible implications in other areas of study, and explores possible improvements that can be made in order to further develop the concerns of your research;
3. Highlights the importance of your study and how it can contribute to understanding the research problem within the field of study;
4. Presents how the findings from your study revealed and helped fill gaps in the literature that had not been previously exposed or adequately described; and,
5. Engages the reader in thinking critically about issues based on an evidence-based interpretation of findings; it is not governed strictly by objective reporting of information.


\textbf{The content of the discussion section of your paper most often includes:}

Explanation of results: Comment on whether or not the results were expected for each set of findings; go into greater depth to explain findings that were unexpected or especially profound. If appropriate, note any unusual or unanticipated patterns or trends that emerged from your results and explain their meaning in relation to the research problem.
References to previous research: Either compare your results with the findings from other studies or use the studies to support a claim. This can include re-visiting key sources already cited in your literature review section, or, save them to cite later in the discussion section if they are more important to compare with your results instead of being a part of the general literature review of prior research used to provide context and background information. Note that you can make this decision to highlight specific studies after you have begun writing the discussion section.
Deduction: A claim for how the results can be applied more generally. For example, describing lessons learned, proposing recommendations that can help improve a situation, or highlighting best practices.
Hypothesis: A more general claim or possible conclusion arising from the results [which may be proved or disproved in subsequent research]. This can be framed as new research questions that emerged as a consequence of your analysis.


\textbf{Keep the following sequential points in mind as you organize and write the discussion section of your paper:}

Think of your discussion as an inverted pyramid. Organize the discussion from the general to the specific, linking your findings to the literature, then to theory, then to practice [if appropriate].

Use the same key terms, narrative style, and verb tense [present] that you used when describing the research problem in your introduction.

Begin by briefly re-stating the research problem you were investigating and answer all of the research questions underpinning the problem that you posed in the introduction.

Describe the patterns, principles, and relationships shown by each major findings and place them in proper perspective. The sequence of this information is important; first state the answer, then the relevant results, then cite the work of others. If appropriate, refer the reader to a figure or table to help enhance the interpretation of the data [either within the text or as an appendix].

Regardless of where it's mentioned, a good discussion section includes analysis of any unexpected findings. This part of the discussion should begin with a description of the unanticipated finding, followed by a brief interpretation as to why you believe it appeared and, if necessary, its possible significance in relation to the overall study. If more than one unexpected finding emerged during the study, describe each of them in the order they appeared as you gathered or analyzed the data. As noted, the exception to discussing findings in the same order you described them in the results section would be to begin by highlighting the implications of a particularly unexpected or significant finding that emerged from the study, followed by a discussion of the remaining findings.

Before concluding the discussion, identify potential limitations and weaknesses if you do not plan to do so in the conclusion of the paper. Comment on their relative importance in relation to your overall interpretation of the results and, if necessary, note how they may affect the validity of your findings. Avoid using an apologetic tone; however, be honest and self-critical [e.g., in retrospect, had you included a particular question in a survey instrument, additional data could have been revealed].

The discussion section should end with a concise summary of the principal implications of the findings regardless of their significance. Give a brief explanation about why you believe the findings and conclusions of your study are important and how they support broader knowledge or understanding of the research problem. This can be followed by any recommendations for further research. However, do not offer recommendations which could have been easily addressed within the study. This would demonstrate to the reader that you have inadequately examined and interpreted the data.



\textbf{Problems to Avoid}
Do not waste time restating your results. 
Do not introduce new results in the discussion section.

but focus on the interpretation of those results and their significance in relation to the research problem, not the data itself.




"How do the findings of this study contribute to the existing body of knowledge on community-based participatory mapping and monitoring in the context of forecast-based financing?
In what ways could the framework developed in this study be adapted for use in other contexts, and what challenges might arise in doing so?
What ethical considerations must be taken into account when implementing a community-based participatory approach to water source mapping and monitoring, and how were these addressed in this study?
How did the use of a mixed-methods approach impact the validity and reliability of the study's findings?
To what extent did the involvement of local community members in the research process influence the outcomes of the study?
In what ways could the framework developed in this study be further refined or improved upon in future research?
What implications do the findings of this study have for policy-makers and practitioners in the fields of water resource management and disaster risk reduction?
How might the methods and approach used in this study be adapted for other types of community-based initiatives or projects?
What challenges did the research team encounter during the development and implementation of the framework, and how were these addressed?
How might the findings of this study contribute to broader discussions around the role of community participation in the management of natural resources and disaster risk reduction efforts?" ChatGPT









% !!!!!!!!!!!!!!!!!!!!!!!!!!!!!!!!!!!!!!!!!!!!!!!!!!!!!!!!!!!!!!!!!!!!!!!!!!!!!!!!!!!!!!!!!!!!!!
% literature "discussion"

Besides addressing the first objective, to \textit{conduct a comprehensive review of existing literature and guidelines related to the design and implementation of \acrlong{cs} programmes, and identify relevant work in regard to the research aim and overall case study context} the literature and \acrshort{cs} project analysis could also create a sound foundation for the following study (see section \ref{subsec:stage1_appl}). % The exploration of the conceptual and practical context allowed the identification and specification of relevant frameworks, aspects and gaps in literature for the subsequent research objectives.\newline
Breaking down the broad concepts of Water Security, Water Scarcity and Drought along with their indicators and indices to the local context highlighted that only relatively rough forecasts are available for Somaliland (see section \ref{subsec:indicators}). Currently, climate, weather and hazard forecasts for Somaliland are either based on international indices like SPI or on a scarce network of local weather gauging stations (see section \ref*{subsec:case_eap}). Besides their coarseness, these indices predict the climate or weather itself and not its impacts, making them unsuitable for \acrlong{fbf} (see section \ref*{subsec:eap}). For successful implementation of \acrshort{fbf}, triggers and actions should be developed and directly linked (see section \ref{subsec:trigger} and \ref{subsec:case_eap}). This is often not feasible as local information about water sources is either missing completely, is incomplete or outdated (see section \ref{subsec:stage1_appl}). This highlighted the need for new local impact indicators for the creation of which the \acrshort{cs} approach was consulted. Several \acrshort{cbm}, \acrshort{mcs}, \acrshort{cbs}, \acrshort{cbwm} and other risk related \acrshort{cs} frameworks and respective guidelines could be identified but none of them exactly matched the intended application (see section \ref{sec:cs}). While "there is no one-size-fits-all approach" \autocite[2]{fraislCitizenScienceEnvironmental2022}, the existing frameworks either focussed on different thematics, different contexts, had different participation levels, different goals or a combination of the above (see sections \ref{subsubsec:cbwm}, \ref{subsubsec:cbs} and \ref{subsec:cbc}). This is consistent with \autocite{butteFrameworkWaterSecurity2022}'s and \autocite{carrionCROWDSOURCINGWATERQUALITY2020}'s findings that existing frameworks guiding the development of water security data collection projects are often very specific and limited to certain factors, in many cases also not taking socio-economic factors into account. At the same time, frameworks like the on from \autocite{butteFrameworkWaterSecurity2022,eu-citizen.scienceEUCitizenScience,citizenscience.govBasicStepsYour} and others were too broad, to be more than general guidelines. Therefore, no applicable framework existed for the implementation of a community-based participatory mapping and monitoring of water sources approach in a water-scarce and resource-limited setting. Especially not, with the focus on providing feasible information for triggering \acrshortpl{aa} in the context of an \acrshort{eap} and in collaboration with a \acrshort{rcrc} National Society.\newline

Other networks like \acrlong{brcis} and the local branch of \acrshort{ocha} implemented their own early action approaches in Somaliland. However, on the one hand with different goals, and on the other hand with different methods (see sections \ref{subsec:stage1_appl} and \ref{subsec:case_eap}). While \acrshort{brcis} collects and interpolates qualitative local information, \acrshort{ocha} bases their early actions on the before mentioned large scale indicators. These approaches are either too slow or to coarse to address the aim of this research, but the concrete experiences from projects in the case study area are valuable to adapt and relate other information to the given context. The transfer of knowledge from other regions, projects and topics is necessary, as scientific literature about the case study area of Somaliland is generally scarce. In addition to these case study related domains, there are further gaps in knowledge when in comes to the application of the \acrshort{fbf} approach on the slow-onset hazard of drought. Generally, the concept of \acrshort{fbf} is now well established in regard to fast-onset disasters, but the drought use case is relatively new (2020) and not yet well researched, which severely limits the amount of guidelines and frameworks available for this particular application (see section \ref{subsec:eap}). Thus, each new project or study focussing on this hazard in the context of \acrshort{fbf} has, at least in part, an exploratory character.\newline
As any new project or study addressing this hazard within these concepts is thus 'automatically' exploratory in nature and no other suitable framework could be identified, the literature and project review suggested the need to develop a new framework to address the specifics of the case study (see section \ref{subsec:cbm}). However, before the new conceptualisation, the general feasibility had to be assessed first, leading to the second objective of this work.


% !!!!!!!!!!!!!!!!!!!!!!!!!!!!!!!!!!!!!!!!!!!!!!!!!!!!!!!!!!!!!!!!!!!!!!!!!!!!!!!!!!!!!!!!!!!!!!
% SSDR and PRC "discussion"
Having established the feasibility of the \acrshort{cs} approach, the third objective to \textit{develop a replicable and adaptable framework for community-based participatory water source mapping and monitoring in the context of \acrlong{fbf}, based on the principles and recommendations identified in objectives 1 and 2} could be pursued. There are now a high number and wide variety of guidelines, \acrshort{cs} associations, initiatives and projects to choose from, that the question of the necessity to add just another one to the list suggests itself. The literature analysis suggested that, above all, the high variety is explained and reflected in the high diversity of the \acrshort{cs} approach of e.g. methodology, temporal and spatial scale, goals, context, level of participation and overall goal (see section \ref{sec:cs}). \autocite{fraislCitizenScienceEnvironmental2022, westonCommunityBasedWaterMonitoring2015} and \autocite{zhengCrowdsourcingMethodsData2018} all summarise a wide variety of these guidelines and \autocite{garciaFindingWhatYou2021} even created a \textit{Guide to Citizen Science Guidelines}. Nonetheless, none of these met the needs of this work, which prompted the development of a new framework. That goes along with the recommendations, of \autocite{garciaFindingWhatYou2021}, that the development and thus transfer of experience in guidelines is the currently the best practice in the field when new, previously unrealised combinations of the thematic diversity listed above are approached and realised \autocite{garciaFindingWhatYou2021}. This highlights the need, that frameworks need to be focussed on a specific topic, region and environment in order to give meaningful advice and not only generic information that is too coarse to be of great use.\newline

The decision to build on \autocite{fraislCitizenScienceEnvironmental2022}'s \acrlong{ssf} was primarily driven by its timeliness, comprehensiveness and focus on environmental issues as it was clear, that a more social and local component can be integrated from the \acrshort{srcs}'s experiences with \acrshort{cbs}. The usefulness of the interpolation of these two approaches was particularly evident in the consideration of personal data. While observing natural phenomena at the level of data collection did not raise too many privacy concerns for \autocite{fraislCitizenScienceEnvironmental2022}, this was almost the opposite for CBS \autocite{ifrcCommunityBasedSurveillanceGuiding2017}. Applying these contrasting perspectives to the issue of water sources was thus able to address both the physical and social components well by considering trade-offs between the two 'extremes'. This claim was further supported over the course of this work, when the iterative integration of other guidelines from several divergent foci into the existing framework could be implemented smoothly and only minor revisions had to be made. \autocite{mcgowanCommunitybasedSurveillanceInfectious2022} also found that the success factors of \acrshort{cbs} are closely linked to the general principles of participatory community engagement and could therefore be transferred to other participatory surveillance preparedness activities. 

As the individual stages are outlined and described in detail in section \ref{sec:design_roadmap} and all identified relatable guidance is included into the design roadmap, no further discussion of the individual stages will be undertaken. % geht das so durch (?) neee



Nonetheless, this processual \acrshort{ssf} has some short-comings which will mostly be addressed in the discussion of its actual application (section \ref{TODO:}) and following limitation section \ref{TODO:} of this entire work.\newline


However, a couple of shortcomings became apparent right at the beginning of the application in the third phase. It was increasingly difficult to keep an overview of the actual project requirements and their interdependencies in terms of subject matter and temporal constraints (see section \ref{subsubsec:knowledge}). Furthermore, \acrshort{cbs}, \acrshort{cbwm} and other approaches have strongly emphasised the importance of embedding the project into prevailing social and decision-making conditions and procedures, which was under-represented in the \acrshort{ssf} (see section \ref{subsubsec:groundwork}). The Interviewees also highlighted, the time (over a year) and resource requirements, which they needed for the development and adaptation of methods and techniques to start with the \acrshort{cbs} project in Somaliland (I2, I3). This goes along with \autocite{garciaFindingWhatYou2021}'s findings, that some adjustments and tailoring always need to be done when implementing a new project (see section \ref{subsubsec:innovations}). Together with the emerging need to structure smaller developments and create an overview of decision dependencies, a fourth area of management became apparent that needed to be addressed (see section \ref{subsubsec:management}).



These reasons provided the ground for the new development of the \acrlong{prc} to expand the \acrshort{ssf} from the third phase onwards. Since the first and second stages are primarily exploratory in nature, it is believed that the \acrshort{prc} should not be integrated in these stages, as it could limit the exploration to the given categories. Should something be overlooked, completions are still possible at the beginning of the third phase.\newline

The conceptualisation and structure of the \acrshort{prc} with the \acrlong{slmc}, guided by the derived goals by \autocite{minkmanCitizenScienceWater2015} further supports the structuring and elaboration of the dependency. Emphasis is given to the top three layers, the \textit{Goal-, Products-, and Activities-Layer}., firstly due to time and information constraints and secondly as practical applicable\textit{ methods, techniques, tools and scripts} need to be highly adjusted to the local context (see section \ref{subsec:slmc}). Both are main limitations of this work and will further be discussed in the last section of this chapter. Nonetheless, the limitation of this work may not be true for subsequent work, which can then benefit from the deeper and more detailed structural possibilities of the \acrshort{slmc}. Despite the fact that the thematic focus of the \acrshort{slmc} is not on \acrshort{cs}, the overall design pattern could be adopted well. The successful preservation of design patterns to other fields is also supported by \autocite{diggelenGroundedDesignDesign2009}'s findings. Thus, although the work has a sound methodological basis, it is primarily based on the perspective of a process- and requirements-oriented understanding and reasoning of the design phase. Other perspectives such as resource, behavioural network or stakeholder networks, cultural norms and values, as well as the communication network perspective may play a role in certain aspects, but are of secondary importance in this work. Engaging these other perspectives more in depth could yield further important insights by encouraging a more holistic view of the design.



% project requirements
However, the process-oriented structure makes it difficult not to overlook important information and outline their inter-dependencies, as there is no more detailed, structured and grouped listing of this. 



%  feasibility

Since the feasibility had to be determined before this work could move on to address the other research objectives and questions, the second objective to \textit{assess the feasibility of the \acrlong{cs} approach in the given context by identifying potential challenges and opportunities for successful implementation, and to propose recommendations for addressing these challenges} was an interim result of the work. 

Based on the developed framework in section \ref{subsec:stage2_design} the feasibility was already assessed in sections \ref{subsec:stage1_appl} and \ref{subsec:stage2_appl}. 

This assessment combined and applied general, international guidance from many projects and studies with local experiences with the \acrshort{cbs} program. It is believed that, even though no dedicated pilot study could be conducted, this combination and interpolation of experiences can reasonably suggest the feasibility of the \acrshort{cs} concept for this application. However, this claim can ultimately only be verified or falsified by a pilot study on site. Furthermore, several challenges such as the embedding into local decision-making and processes, actual tailoring to local conditions and clarifying financial capacities could not further be investigated due to the limited amount of interviews with local stakeholders and ongoing developments of the superordinate project.\newline


Due to the already conducted discussion in section \ref{subsec:stage2_appl} and challenges that cannot be investigated further in this context, the remaining part of this section focusses more on how, why and in what order this assessment was realised as it is believed that this holds more value to the reader than iterating over the discussion again.\newline



Since, to the best of my knowledge, no work has been conducted with the combination of methods, goals and context of this work, there was no concrete existing guidance to assess the feasibility of this approach to achieve the research's aim in the first place. The lack of suitable frameworks for this project made it thus necessary to work on the development of the framework and its application step by step and not only chronologically, at least to some extent. This was facilitated by the iterative working approach, which made it possible to first sketch out possible solutions and then deepen them when the conditions were met accordingly. This was also the case in addressing the second objective and the early conduction of the feasibility assessment is also recommended by multiple other guidelines \autocite{citizenscience.govBasicStepsYour,garciaFindingWhatYou2021,ifrcCommunityBasedSurveillanceGuiding2017,ifrcFbFPractitionersManual2023b,minkmanCitizenScienceWater2015}.\newline


% kann weg
The \acrlong{ssf} and \acrlong{slmc} were adopted at an early stage of the work to have a general direction for the development. To conduct the assessment, the third research objective had to be somewhat anticipated in order to provide an initial framework for the structured feasibility assessment. This framework, now conceptually integrated in the second stage of the design roadmap (see section \ref{subsec:stage2_design}) was in the beginning primarily a combination of the \acrshort{ssf}'s second stage and the feasibility assessment of the \acrshort{cbs} of the \acrshort{ifrc}. The final feasibility assessment took place on the current basis, which was further underpinned with some additional guidelines, best practices and knowledge of the interviewees over the course of multiple iterations.\newline


When designing a framework for or directly assessing the feasibility of \acrshort{cs}, it becomes clear that \textit{feasibility} depends on a variety of factors, but also that there are no clear rules that must be followed. Each \acrshort{cs} project is somewhat special and the flexible concept also allows for several adaptations (see section \ref{sec:cs}). Therefore, the feasibility is not assessed by a specific set of rules, but rather how well it relates to general principles and factors of success. This makes sense in the way, that what specifically works in e.g. \autocite{minkmanCitizenScienceWater2015}'s approach in the Netherlands may not be feasible in Somaliland, e.g. the use of smartphone sensors as the rural population in Somaliland has few smartphones and internet coverage is poor. Assessing challenges and opportunities is thus a highly specific and local task and depends on many factors.\newline
Nonetheless, the \acrshort{ecsa} along with many other associations and studies developed \acrshort{cs} principles and characteristics that support the successful design, implementation and operation of a \acrshort{cs} project. Furthermore, a \acrshort{cbs} project was already successfully implemented and in operation for several years within the context and the \acrshort{srcs} but focused on a different topic. This, again highlights the thorough analysis of local comparable projects, mentioned in stage 1, section \ref{subsec:stage1_design}. The actual feasibility assessment therefore focussed primarily on the differences between the \acrshort{cbs} and the potential water source mapping and monitoring project.\newline


% %? %? alternative designs
% % % I guess better leave this out and put it in the discussion section (?) 
% % “Using Remote Sensing to Map and Monitor Water Resources in Arid and Semiarid Regions” ([Klemas and Pieterse, 2015, p. 33](zotero://select/groups/4773535/items/BVN6IXG5)) ([pdf](zotero://open-pdf/groups/4773535/items/UPSYZXDK?page=1&annotation=4DPZD4BZ))
% % % Establishing an operational waterhole monitoring system using satellite data and hydrologic modelling: Application in the pastoral regions of East Africa
% % https://earlywarning.usgs.gov/docs/Senay-et-al-Pastoralism-Research-\Policy-and-Practice-2013.pdf
% % “Global Monitoring of Water Supply and Sanitation: History, Methods and Future Challenges” ([Bartram et al., 2014, p. 8137](zotero://select/groups/4773535/items/6AWUJTW5)) ([pdf](zotero://open-pdf/groups/4773535/items/BFNSQGWS?page=1&annotation=ZWSBJVDM))
% % “Remote monitoring of rural water systems: A pathway to improved performance and sustainability?” ([Thomson, 2021, p. 1](zotero://select/groups/4773535/items/UQLXVVYI)) ([pdf](zotero://open-pdf/groups/4773535/items/K9XBXPQD?page=1&annotation=B59B5U68))

% % but: internet access and so on.. "We specifically focus on water-related low-cost sensor networks"
% “Moving beyond the Technology: A Socio-technical Roadmap for Low-Cost Water Sensor Network Applications” ([Mao et al., 2020, p. 9145](zotero://select/groups/4773535/items/VEJUB8D9)) ([pdf](zotero://open-pdf/groups/4773535/items/KMCIFXB8?page=1&annotation=2A7QA3V3))

% % more on pump failure stuff but hey..
% “Remote monitoring of rural water systems: A pathway to improved performance and sustainability?” ([Thomson, 2021, p. 1](zotero://select/groups/4773535/items/UQLXVVYI)) ([pdf](zotero://open-pdf/groups/4773535/items/K9XBXPQD?page=1&annotation=B59B5U68))

% % dynamic adjustments of thresholds
% “Towards drought impact-based forecasting in a multi-hazard context” ([Boult et al., 2022, p. 1](zotero://select/groups/4773535/items/B2AQSTYL)) ([pdf](zotero://open-pdf/groups/4773535/items/W9TFLH43?page=1&annotation=GL47JLV7))

% % ITIKI -> benefits of giving information
% “ITIKI: bridge between African indigenous knowledge and modern science of drought prediction” ([Masinde and Bagula, 2012, p. 274](zotero://select/groups/4773535/items/EW9XSSZP)) ([pdf](zotero://open-pdf/groups/4773535/items/3WQ4S9PE?page=1&annotation=5BKZELQW))

% % “Our results indicate that using the phones to transmit more than just water quality data will likely improve the effectiveness and sustainability of this type of intervention.” ([Kumpel et al., 2015, p. 10846](zotero://select/groups/4773535/items/GPM4C7RJ)) ([pdf](zotero://open-pdf/groups/4773535/items/7VXVKEXK?page=1&annotation=4DJIADX2)) 
% \include{Chapters/example_chapter} 

\clearpage

%----------------------------------------------------------------------------------------
%	BIBLIOGRAPHY
%----------------------------------------------------------------------------------------

\printbibliography[heading=bibintoc]

%----------------------------------------------------------------------------------------


%----------------------------------------------------------------------------------------
%	THESIS CONTENT - APPENDICES
%----------------------------------------------------------------------------------------

\appendix % Cue to tell LaTeX that the following "chapters" are Appendices

% Include the appendices of the thesis as separate files from the Appendices folder
% Uncomment the lines as you write the Appendices

%% Appendix A

\chapter{Appendix A: GitHub Repository} % Main appendix title

\label{AppendixA} % For referencing this appendix elsewhere, use \ref{AppendixA}

Please see the git repository at github.com/BoSott/masterthesis for all resources used in this thesis. Here, in addition to the following appendices, the developed figures are provided together with the questionnaires, interview transcripts and data used.

%% Appendix B

\chapter{Appendix B: Questionnaires and Interview Guidelines} % Main appendix title

\label{AppendixB} % For referencing this appendix elsewhere, use \ref{AppendixB}

The transcripts of the interviews are listed together with the questions chronologically in the order in which the interviews were conducted.

\section{Questionnaire \& Answers I1}

Interviewer: Bosse Sottmann\newline
Medium: Google Forms\newline
Interviewee: GRC FbF Manager of the SRCS \newline
Date: 31.01.2023

\subsection*{Questionnaire SRCS: Defining Goals and Outcomes}

\textbf{Introduction}\newline
The goal of this project is the design of a practically applicable volunteer sensing-based water source monitoring approach primarily for the water source type of Berkads.
Sub-goals are based on the learnings how water access can be measured specifically for this water source type, what information needs to be known about the source initially, continuously and if the incorporation of local knowledge is useful and possible -- and if yes, how and which specific information are helpful in the context of Anticipatory Actions.\newline
The work will be based on a variety of different sources of information. In addition to this questionnaire and subsequent discussions, also with other stakeholders, best practices and knowledge will be gathered from the literature. Therefore, your input to this questionnaire is critical in multiple ways. Your and the SRCSs opinions, experiences and needs will be the foundation for all of the following work - ensuring that the resulting design meets your requirements and that it aligns to the constraints of this context. Based on the defined goals and following results you will mention here, best practices and knowledge from other contexts can be transferred and applied to this project.\newline
Methodologically, this project coarsely follows the 7-layer-Model of Collaboration. We start to define goals, sub-goals and the actual results we need to accomplish in order to reach these goals. Further down the road, we will build on top of this by defining sets of actions to accomplish the results, thinking about patterns of collaboration internally and with other stakeholders and further defining specific techniques, tools and scripts. Thus, creating a design, that is adapted to the specific context and at the same time applies current best practices from around the world. Nonetheless, a disclaimer should be made, that this is a project in the context of a Master's thesis, thus a fully-fledged design ready to get launched is out of scope of this work. Yet, it can lay a good foundation for the following work.

\subsection*{General questions}
Please give some brief information about yourself and about your organization.

What is your role in the SRCS?\newline
\textit{Offering technical support to the ongoing Forecast based Financing project}

How long have you worked in your position?\newline
\textit{1 year}

What are your predominant tasks?\newline
\textit{Offering technical support to the ongoing Forecast based Financing project as well coordinating the partnership between SRCS and the German Red Cross}

How many paid employees does the SRCS have in total?\newline
\textit{249}

How many of those work in the sector of risk management and/or Anticipatory Actions?\newline
\textit{30}

How many volunteers does the SRCS have?\newline
\textit{1500}

How are the Volunteers spread across the country? Are there regions where there are essentially no or fewer volunteers?\newline
\textit{The Volunteers are evenly spread across the country. However some regions have inactive volunteers due to less activities there whilst some regions have active volunteers due to the amount of project work being undertaken there
}

\subsection*{Goals of the project}

This section is about the main and sub-goals of this project. A main goal could be the mapping of accessible water sources, whereas a sub-goal of this could be e.g. the training and education of volunteers for this task. Please think creatively, without the restriction of limiting resources and please also think about related fields that can be (indirectly) affected by this.\newline
What are the main goals for the project of mapping and monitoring water sources from the perspective of the SRCS? What do you ultimately want to achieve?\newline
\textit{Location is key! Berkads location data is currently missing and this has resulted in the SRCS not being able to quantify the number of existing berkads per region. The main goal of the project will be ascertain the location of berkads and capturing key info such as berkad ownership status (some berkads are privately owned thus not everyone can access water from them. Important info to be also captured include the total number of people or communities dependant on the berkad as well as the storage capacity of the Berkad. Monitoring of water sources would enable seamless prioritization of regions to deliver water (water trucking). Ultimately, in terms of Anticipatory Actions water sources monitoring would enable triggering action before critical water levels}

Can you think of sub-goals that would go along with each main goal?

Can you think of additional goals from the perspective of the community or volunteer potentially involved in the project?\newline
\textit{From the community/volunteer perspective the main goal would be for them to know the existing water resources within their vicinity, as well as the capacity of these water bodies. The main goal being to ascertain whether these water bodies are able to withstand the demand during drought periods.}

Which of these goals could match well and which might be competing?\newline
\textit{The community and SRCS goals match as both focus on closing the knowledge currently existing regarding berkads numbers per district, community and regional level}

\subsection*{Wanted results}
In order to fulfil the defined goals, it is important to further specify actual outcomes by the project. These may deal with issues of quality, effectiveness, efficiency, and other product related characteristics. Following the example from above, this could be a collection of data about water sources in the project area meeting certain pre-defined quality standards. Another result for the sub-goal of trained volunteers might be a collection of appropriate training materials.
What results must be achieved in order to reach the goals? Please list them.\newline
\textit{i) Location data of the berkads (coordinates),}\newline
\textit{ii)Volunteer orientation on water resources monitoring}\newline
\textit{iii) determining the ownership status of each berkad}\newline
\textit{iv) community sensitization to dispel misconceptions about the mapping and water monitoring exercise}\newline
\textit{v) water levels monitoring}\newline
\textit{vi) triggering action based on water levels}\newline
\textit{vii) Determining the water level trigger}

Which of those results do you consider to be the most critical? Please list them and if possible, explain why.\newline
\textit{Location data as this will enable determine the serving capacity of each berkad i.e the total number of communities dependant on each berkad. One important result is also community sensitisation to dispel misconceptions within the communtiies. The community will need to understand why the SRCS will be monitoring water bodies. Triggering for action is also a key result as the end goal will be to counter water shortages so as to mitigate water shortages.}

Based on your experience - in which chronological order do the results have to be processed? Please order them accordingly and add some explanation if possible.\newline
\textit{i) Volunteer orientation on water resources monitoring}\newline
\textit{ii) community sensitization to dispel misconceptions about the mapping and water monitoring exercise}\newline
\textit{iii) Location data of the berkads (coordinates)}\newline
\textit{iv) determining the ownership status of each berkad}\newline
\textit{v) Determining the water level trigger}\newline
\textit{vi) water levels monitoring}\newline
\textit{vii) triggering action based on water levels}

Which of the results specifically required in the design phase do you think are the most critical?\newline
\textit{Determining the water levels to trigger action.}

Which results would specifically be required to address Anticipatory Actions?\newline
\textit{i) Determining the water level to trigger action}\newline
\textit{ii) water levels monitoring}\newline
\textit{iii) triggering action based on water levels}

\subsection*{Stakeholders}
One of the key success factors for a successful design and implementation of a crowdsensing project mentioned in the literature is the early inclusion of all involved stakeholders. This section is about them.
What stakeholders are involved in the mapping and monitoring of water? Who is actively involved, passively affected or just indirectly concerned (please add this information respectively)?\newline
\textit{i)SRCS Volunteers Actively involved}\newline
\textit{ii) Communities Actively involved}\newline
\textit{iii) Berkad owners Actively Involved}\newline
\textit{iv) Community elders Actively involved}\newline
\textit{v) Other NGOs Indirectly concerned}\newline
\textit{vi) The Government Ministry of Water resources}

Which stakeholders, other organizations, communities, groups, or individuals could additionally contribute to the project in terms of knowledge, resources, or other kinds of qualities? Think creatively, 'around the corner' and gladly draw from your experience in other projects.\newline
\textit{The Ministry of Water resources , NADFOR, Other NGOs because they have also constructed some berkads in some communities, FAOSWALIM, the Local government political leadership i.e the Regional governor etc,}

Of all these stakeholders, who are the most important ones - and why? What should we know about them that might be critical for the success of this project?\newline
\textit{The Ministry of Water resources because i believe have a database on existing drilled boreholes in Somaliland (although the lack berkad data), The ministry of water also has the technical expertise in water resources monitoring. NADFOR because they have a similar ongoing programme on community level monitoring of livestock body condition, market prices as well as weather variables. Other NGOs because they have also constructed some berkads in some communities.}

\subsection*{Resource availability and positive/negative constraints}
This section is concerned with the context and environment of the project. To be applicable to the actual context, the design requires negative and positive constraints. In contrast to goals, constraints define stricter limits that need to be respected in order to improve the chances of success. For example, a constraint can define what is not possible (negative constraint), as well as what essential functions need to be met (positive constraint) and what would simply be an added value.

Please list all negative constraints you can think of regarding this project. These can be the classic areas of human resources, knowledge and financial capacities as well as softer social requirements and constraints.\newline
\textit{The project might create huge expectations from the communities as there is the ongoing drought. Whenever there is monitoring of resources communities believe this should be followed up by instant aid. Private berkad owners might not be willing to contribute to the project. They might bar Volunteers from accessing their berkads thus creating tension between community volunteers and berkad owners.}

Now, please list all positive constraints you can think of regarding this project.

Are there any other requirements and/or restrictions that we need to consider when designing this project right from the beginning? These could include, for example, content, format, time management or cultural specifics.\newline
\textit{i)There is an ongoing drought and thus the SRCS staff and volunteers might be over stretched in drought response activities.}\newline
\textit{ii) Community elders should be engaged before the start of the mapping and monitoring as they will help dispel misconceptions about the project}\newline
\textit{iii) the ministry of water resources should be in the loop during the entire project duration}

\subsection*{Final remarks}
Here you can write anything that you find important but was not addressed above. You can give feedback to the questions, raise concerns about some other issues or highlight certain aspects or perspectives.\newline
Final remarks \newline
\textit{All the best in your studies!}


\section{Questions \& Transcript I2}
Interviewer: Bosse Sottmann\newline
Medium: Zoom\newline
Interviewee (I): NRC CBS Somalia Manager \newline
Date: 02.02.2023

Interviewer: Mal gucken ob ich das alles beantworten kann. Ansonsten weiß ich vielleicht jemanden, der es k{\"o}nnte.

I: Ich kann jetzt ja mal vorstellen was ich so im Kopf habe, was so die gr{\"o}ßeren Fragen sind. Und zwar einmal - was auch vielleicht auch Ihr Background ist. Wie das NRC auch involviert war {\"u}berhaupt in den ganzen Prozess, wie es da so rum ging. Wie das angefangen hat, also die Zusammenarbeit mit dem SRCS. Was damals die Probleme, als auch das Goal war, also nicht die Probleme in der Umsetzung sondern was war eigentlich die Fragestellung, was war eigentlich das Ziel dessen. Warum genau Crowdsensing, warum haben Sie genau die Methode gew{\"a}hlt und keine andere? Und welche Alternativen gab es vielleicht auch? Das w{\"a}re so eine gr{\"o}ßere Frage. Eimal vielleicht einen {\"u}berblick zu geben w{\"a}re f{\"u}r mich sehr hilfreich {\"u}ber die Designphase - also wie haben Sie das so organisiert auch mit dem SRCS und welche Stakeholder waren dabei? Wie wurde mit den Volunteers umgegangen, wie wurde das gehandhabt? Wie kam die Communities mit ins Spiel? Wie haben Sie das zum beispiel auch mit den Elders in der Community kommuniziert? Wie lief das so? Ich sage nur, das sind Themen, die mich gerade interessieren. Ich weiß, in einer halben Stunde kriegen wir die nicht alle durch. Wie ginge es dann vielleicht auch {\"u}ber in die Implementation und in die Operational Phase von diesem gesamten Projekt. Gab es bisher schon eine Evaluation? Wie l{\"a}uft das so nebenbei? Und wie k{\"o}nnte das vielleicht denn in Zukunft aussehen? Und entweder so am Ende oder eben auch w{\"a}hrenddessen - wie k{\"o}nnte man das jetzt gut auf das kommende Projekt umlegen? - nat{\"u}rlich potentiell kommendes Projekt. Das ist nat{\"u}rlich nicht in Stein gemeißelt. Aber was k{\"o}nnte man da als Key Take Aways mitnehmen? Wo man jetzt von Anfang an achtet, dann  haben wir schon viel gewonnen

Interviewer: Ok. Also mal gucken was ich behalten habe. Also vielleicht noch kurz : Ich habe beim Norwegischen Roten Kreuz erst vor zweieinhalb Jahren angefangen. Tats{\"a}chlich - und war in der Anfangsphase von der Entwicklung der Plattform um die es ja, glaube ich ja, vorrangig geht, gar nicht dabei. Ich habe ein bischen Brackground Informationen wo es her kommt oder ich weiß warum sie entwickelt wurde. Wie ungef{\"a}hr sie entwickelt wurde, aber so die spezifischen Sachen weiß ich nicht, ob ich so viel dazu beitragen kann. Also die NYSS Plattform so wie sie jetzt ist, die entwickelt sich permanent weiter. Als ich angefangen habe war sie viel oder um einiges weniger komplex. Und eines der Ziele dieser Plattform war es auch ein einfaches Data collection tool bereitzustellen. F{\"u}r die National Societies. Es sollte nie, die Idee war nie es weiter zu entwickeln, weiter zu entwickeln, weiter zu entwickeln. Die Idee war, dass wie es bisher gemacht wurde, und IFRC hat bis vor Kurzem Kobo benutzt zum Beispiel und es gab diverse Formen vor wie NYSS jetzt ist. Die waren {\"a}hnlich aber, genau, waren in einem fr{\"u}heren Stadium und auf Basis dessen wo man gesehen hat was fehlt. Und zum Beispiel mit SRCS I3 sehr viel zusammen gesessen wurde und diskutiert wurde und was w{\"a}re jetzt n{\"o}tig. Also zum Beispiel diese Feedbackmessages kommt, dieses Feature kommt aus Diskussionen mit SRCS. Also es gab, jetzt weiß ich gar nicht wie viel w{\"a}hrend der ersten Pr{\"a}sentation gesagt habe dazu - also da gab es ja 2017 ein riesigen Choleraausbruch in Somaliland - da ist das quasi geboren zumindest f{\"u}r Somaliland CBS zu machen. Ich habe vorher bei {\"a}rzte ohne Grenzen gearbeitet. Da, genau, ist das jetzt ja nichts Neues. Das machen viele schon. Es entwickelt sich halt immer, immer weiter und jetzt ist es viel popul{\"a}rer auch durch Covid deswegen hat das Ganze auch viel mehr, ja Popularit{\"a}t bekommen. Also die Plattform, wie sie jetzt so ist, ist 2019 als das war, ne, als ich angefangen habe ging das erst los, kurz vorher. 2020 das erste Mal im M{\"a}rz {\"u}berhaupt so in der Form benutzt worden. Und hinsichtlich der Entwicklung wurde zusammen diskutiert mit IFRC, den National Societies speziell SRCS, welche Bed{\"u}rfnisse sie haben, was macht Sinn, was nicht. Und dann gab es so Codeons [Hackathons], in unterschiedlichen Orten. Einer war im Senegal, einer war in Norwegen, einer war in Budapest. Wo quasi Volunteers, also Softwareentwickler diese Plattform mitentwickelt haben. Jetzt ist es so, dass wir zwei Softwareentwickler in Oslo haben, die speziell, ja, daran arbeiten. Es gab letztens noch Mal eine {\"a}hnliche Veranstaltung, aber so ein großes Projekt jetzt nicht mehr - genau. Und die Idee, von dieser Plattform, ist nicht, das habe ich glaube ich beim letzten Mal schon gesagt. Das Ziel dieser Plattform ist Early Warning. Nicht 'wir sammeln jetzt Daten zu allem m{\"o}glichen'. Also wir haben immer wieder Diskussionen, wo wir andere Community health Aktivit{\"a}ten haben und die w{\"u}rden gerne NYSS benutzen um Daten zu sammeln. Zum Beispiel SGBV, die hatten gefragt, ob sie nicht mit NYSS die Healthcare Worker informieren k{\"o}nnten ob da nicht in der Community Opfer von SGBV [Sexual and Gender-Based Violence] oder [...] von SGBV war. Aber das machen wir nicht. Also ne wir geben ja zum Beispiel wir - Warum es auch nicht f{\"u}r andere case management Sachen benutzt werden kann weil keine Individuellen Daten zum Beispiel in dieser Plattform sein sollen. Daf{\"u}r ist sie einfach nicht gemacht. Die Communities, also hinsichtlich Erfahrung mit den Communities. Also gut ist nicht nur in Somaliland der Fall, aber ich weiß, dass es da am Anfang auch Problem war, dass jemand von der Community 'ne Report {\"u}ber die, {\"u}ber jemanden {\"u}ber ein Community member schickt. SRCS investiert sehr sehr viel in Kommunikation, in Feedback vice versa mit der Community. Die treffen die regelm{\"a}ßig, h{\"o}ren sich an was die, wie die das alles so finden, welche Bedarfe sie haben. Wir haben auch eine große Evaluierung gemacht f{\"u}r CBS, nicht NYSS, wir haben auch eine Evaluierung zu NYSS an sich gemacht, die wird hoffentlich bald ver{\"o}ffentlicht. Wo wir auch geguckt haben welche anderen Instrumente, oder welche andere Reporting Tools gibt es von anderen organisationen. Wo sind wir besser oder anders. Was sind die Vorteile und Nachteile weil die Idee - also das Problem, was ich oft habe ist - ich geh in ein Land und wir wollen CBS machen und dann ist da aber schon ein anderes, 'ne {\"a}hnliche Software oder reporting Tool. Und dann muss ich wissen, welche sind das? Wie funkionieren die? Macht es Sinn trotzdem noch advocacy f{\"u}r NYSS zu tun. Genau, also wir haben beides getan. CBS evaluiert f{\"u}r SRCS und auch die NYSS Plattform mehr global. Aber vielleicht noch eine Geschichte, die immer wieder ein Problem ist, wenn wir die Plattform einf{\"u}hren wollen ist die Akzeptanz und das Verst{\"a}ndnis von MoH, also Ministry of Health. Weil die finden das oft nicht gut, dass wir, weil es kommen ja alle m{\"o}glichen Organisationen st{\"a}ndig mit ihren eigenen Instrumenten, Methoden und was weiß ich und das finden die oft nicht so gut plus (ahhh), ich weiß auch nicht warum, also jetzt machen wir es nicht mehr so, h{\"a}ngen es nicht mehr so hoch auf, aber der Server von NYSS ist ja nicht im Land, ne? Der ist in Irland f{\"u}r Data Protection Reasons und weil es auch einfacher ist f{\"u}r die Softwareentwickler, ja sich darum zu k{\"u}mmern. Aber das gef{\"a}llt dem Ministries of Health nicht. Gerade in L{\"a}ndern, die ein bischen [lach], o.k. L{\"a}nder, die paranoid sind kann man so nicht sagen, aber es gibt L{\"a}nder, die Panik haben, dass Dinge publik werden, die nicht publik werden sollen. Und wenn man, es gibt in vielen L{\"a}ndern - zum Beispiel erkl{\"a}rt man ungern einen Choleraausbruch, richtig? Die wollen also die Daten in ihren eigenen H{\"a}nden haben. Das niemand wie unser HQ in Oslo da Zugriff drauf hat UND SIE NICHT. Also die haben ja Zugriff auf die Plattform, aber nur zu einem Teil. Die Daten, die Ownership von den Daten ist aber mit der National Society und das finden die oft nicth so gut plus das N{\"a}chste, nicht Problem, und zum Teil haben wir das jetzt auch gel{\"o}st. Man hat Daten von den Gesundheitseinrichtungen, so Disease Sruveillance und dann haben wir die Community Based Surveillance [CBS], die unterschiedlich sind. Aber die Idee ist, sie zusammen zu bringen und da haben wir jetzt zumindest, ich weiß nicht ob dir [...] ein Begriff ist, aber das ist ja ein Datacollection Tool, was in vielen gerade afrikanischen L{\"a}ndern beim Ministry of Health genutzt wird. Um auf der gesundheitseinrichtungsebene Daten von Patienten zu sammeln um auch Trends in Erkrankungen und sowas  zu analysieren. Und Case management kann man damit glaube ich auch und jetzt vor Kruzem, ja haben wir es hingekriegt die Daten von der NYSS Plattform dann automatisch dann in dieses DHIS2 [?] District Health Information System reinzuschieben. genau, ja, also das als kurze Zusammenfassung. Jetzt weiß ich nicht wo ich vielleicht noch ein wenig mehr erz{\"a}hlen sollte. 

Interviewer: Ich glaube, dass es auch noch ein paar Fragen gibt, die wert sind, gestellt zu werden, aber ein Punkt w{\"a}re f{\"u}r mich noch, der Stakeholder Ministry of Health, kann jetzt bei Ihnen mit rein, das w{\"a}re ja das Pendant bei uns, das Water Ministry, also das Ministry for Water Resources. Gab es denn sonst in der Community noch oder von den Volunteers oder anderen Stakeholder, mit denen Sie teilweise auch Ziele hatten, die zusammen lagen oder auch Ziele, die sich vielleicht auch ein bisschen entgegengesetzt haben? Also gab es da auch Widerst{\"a}nde oder gab es da noch andere Personen oder Bereiche, Rollen?

I: Von der Community?

Interviewer: Von der Community als auch von subnationaler Ebene oder von regionaler Ebene oder eben auch wirklich vom lokalen Volunteer? 

I: Ja also Minister of Health definitiv, die haben sehr viele Diskussionen gef{\"u}hrt, die f{\"u}hren die auch jetzt noch. Gut in anderen L{\"a}ndern zum Beispiel haben wir Minister of Agriculture noch mit dabei, weil da auch Erkrankungen von Tieren zum Teil mit berichten. Mit den Communities war es am Anfang, ich glaube nicht unbedingt nur mit der Plattform an sich verbunden, sondern halt mit diesem Reporting, dass sie das am Anfang nicht verstanden haben, wo geht das hin, warum informiert ihr Minister of Health {\"u}ber jemanden der hier krank ist. Aber es gab dann halt viele Sessions, mit denen wo man erkl{\"a}rt hat warum, weshalb, wieso. Das Vertrauen in Minister of Health ist nicht so groß, das kam bei der Evaluation auch raus, aber die merken das was passiert durch SRCS, wenn mit CBS an sich. So f{\"a}hrt also wenn das Programm durchs Deutsche Rote Kreuz gestartet wird, also definitiv werden da ein paar Veranstaltungen mit den Community Leaders stattfinden um zu erkl{\"a}ren was wird gemacht, wer macht was, was passiert wenn wir da, und das halt kontinuierlich also wie gesagt, die gehen da einmal im Monat oder einmal im Quartal und sitzen mit den Community Leader zusammen aber f{\"u}r die Details zu dem Thema ist tats{\"a}chlich I3 der Richtige, weil der ist von Anfang an dabei, der ist permanent im Projekt draußen und ist da am besten auch wahrscheinlich kann der da auch gut beraten wer am besten, mit wem am besten f{\"u}r das Thema zu sprechen ist. 

Interviewer: Ja, I1 hatte das schon einiges angesprochen, aber das ist mit Sicherheit nochmal sehr sehr gut, die stehen bestimmt auch mit einem anderen in Kontakt 

I: Ja, also I3 die, genau, er hat mir erz{\"a}hlt, dass er mit I3 gesprochen hatte und dann I3 das an mich weitergeleitet hatte, obwohl er also gerade f{\"u}r solche Fragen definitiv der richtige Ansprechpartner ist.

Interviewer: Und was gibt es denn noch f{\"u}r andere Methoden, die noch genutzt werden neben diesem Crowdsensing und dem Volunteer-Sensing in anderen Regionen? 

I: Kobo, oder einfach in Kobo haben sie zum Beispiel in Uganda genutzt oder in Bukina Faso haben wir jetzt einfach Excel-Sheets also die Volunteers statt von diesen Plattformen eine SMS zu schicken, schicken die gleiche SMS an ihren Supervisor und der tr{\"a}gt das dann in Excel ein, weil das Ministry of Health wollte nicht, dass wir nicht, dass wir NYSS benutzen. Und dann waren wir es halt ein bisschen umst{\"a}ndlicher, aber am Ende eigentlich das Gleiche Aber dieses, die wollten halt keinen, die haben eine Organisation, die ein Community-Health-Programm einf{\"u}hrt, ich glaube die Zahlen halt ans Ministry of Health. Das tun wir nicht. Und dann haben sie sich f{\"u}r dieses Programm entschieden, auch wenn es nicht das Gleiche tut, was nichts macht, aber das verstehen die Leute halt nicht immer. Genau, also wahrscheinlich auch die Arbeit mit den Ministries, vielleicht kann man da, aber da ist vielleicht I3, auch der richtige Ansprechpartner m{\"o}glicherweise macht es auch Sinn, da jemanden vom Minister of Health mit hinzunehmen oder dass man sich austauscht mit dem vom Minister of Health, um die Benefits auch zu zeigen damit man da nicht wieder von vorne anf{\"a}ngt und die die gleiche Skepsis haben, sondern wenn die sehen, das gleiche Tool ist schon vom Minister of Health benutzt und erfolgreich benutzt die wollen das landesweit einf{\"u}hren und wir sind kurz davor, ist es nat{\"u}rlich leichter auch m{\"o}glicherweise die anderen Departments der Regierung zu {\"u}berzeugen 

Interviewer: Ja, noch mal kurz ein St{\"u}ck zur{\"u}ck, [...], wir hatten ja auch schon {\"u}ber das gesprochen, noch mal ein bisschen tickentechnischer, aber doch noch im Management bleiben, dass man das von der Analogenwelt in das Digitale reinbekommt und daf{\"u}r, dass man dann nat{\"u}rlich auch Kategorien braucht, zusammen mit diesem [...]  Data Collection Platform, wie ist da so die Stimmung oder der Gedanke bez{\"u}glich eben auch der Erhebung und dem Monitoring von Water Resources weil wenn ich jetzt sage, Water Availability hat nat{\"u}rlich jetzt okay, wie viel Wasser ist da, aber es gibt nat{\"u}rlich auf der anderen Seite auch welche Qualit{\"a}t hat es wie viel kommt nach, wer hat {\"u}berhaupt Access dazu, es gibt einige Berkads oder sogar mehrere, die dann nat{\"u}rlich auch privat gehandhabt werden und wo nicht jeder Access zu hat. Da m{\"u}sste man tats{\"a}chlich eine ganze Menge Daten durchaus erheben und auch durchaus so gestalten, dass sie flexibel ver{\"a}nderbar sind, weil wenn jetzt zum Beispiel ein Dorf oder ein Hirte kommt mit 200 Tieren, dann ist es nat{\"u}rlich eine hohe Wasserabgabe. Dann w{\"a}re es doch auch nat{\"u}rlich zum Sinne des Forecastings aber auch wieder Datenerhebung, insofern, wie passt das auch dazu? 

I: Also es ist kein Forecasting Tool, es ist ein Early Warning Tool, deswegen habe ich beim letzten Mal auch darauf gepocht und ich hatte es so verstanden, dass es eher Early Warning, Early Action benutzt wird. Die Wahrscheinlichkeit, dass wir das Rote Kreuz sich darauf einl{\"a}sst, das weiterzuentwickeln, zu dem Zweck, sehe ich nicht, das kann ich jetzt schon sagen. Wenn es wirklich eine einfache Geschichte ist, die man einfach machen kann, wo es einfach nur geht, eine SMS zu schicken, wo es darum geht, hier ist kein Wasser mehr, ist das eine andere Geschichte, als wenn es eine große Data Collection, also die werden kein, das weiß ich jetzt schon, das wird nichts. Deswegen hatte ich beim letzten Mal noch gefragt, was wird wirklich wozu und was ist an Daten n{\"o}tig? Das w{\"a}re okay, wenn es dar{\"u}ber hinausgeht. Das glaube ich nicht. 

Interviewer: Genau, deswegen frage ich nach, das Ziel ist ja an sich, einen Trigger zu definieren, dass der dem dortigen somalischen Ruten Kreuz dann die Ermittlung gibt, wir haben langsam kein Wasser mehr und wir brauchen jetzt demn{\"a}chst eine L{\"o}sung. Das ist ja an sich das Ziel, das heißt, wenn wir wissen, okay, das kriegen wir nicht in die Plattform rein, dann kann man ja auch den anderen Schritt machen, okay, dann muss der Volunteer besser geschult werden, weil dann muss der Volunteer das selber {\"u}berblicken k{\"o}nnen. Dann k{\"o}nnen wir das halt nicht in der Plattform errechnen, sondern dann brauchen wir, das w{\"a}re auch die n{\"a}chste Frage, wie kann man auf dieses lokale Wissen vertrauen, wie kann man das gut einbinden vielleicht auch, wir wissen ja auch. 

I: Okay, nochmal zur{\"u}ck zu den vorherigen, also was der Volunteer schicken w{\"u}rde, w{\"a}re ja regelm{\"a}ßige w{\"o}chentliche Updates, ist es voll, halb voll oder leer. Es gibt noch die M{\"o}glichkeit, dass dann jemand, der die Plattform h{\"a}ndelt, und das sind dann ja nicht die Volunteers, wenn man anruft, also was passiert bei dem, was wir machen, ist ja immer noch der Supervisor, der das validiert. Die rufen da an, sind es wirklich die Symptome, die die gerade berichtet haben. Und erst dann gibt der Klick, okay, das ist wirklich ein True-Alert. Und dann hat der Supervisor noch die M{\"o}glichkeit, wenn wir zum Beispiel, die Idee dann ist ja, es geht ans Ministery of Health und dann soll der Supervisor und Volunteer gucken, dass da Investigation Response vom, m{\"o}glichst vom Minister of Health passiert. Und dann k{\"o}nnen die in so einem Event-Blog noch Notizen machen zu diesem Alert. Also, was man tun, was man {\"u}berlegen k{\"o}nnte, wenn jetzt noch irgendwelche extra Informationen zu dieser Information, okay, es ist leer oder voll, wenn es noch irgendwelche anderen Informationen n{\"o}tig w{\"a}ren, k{\"o}nnte der Supervisor am Telefon das checken und immernoch noch eintragen, richtig? Also, ich kann das auch vielleicht noch mal zeigen in der Demonstration.

Interviewer: Ich habe mir die Demonstration angeschaut. Ich habe mich da ein bisschen mehr angelesen. 

I: Okay, gut, also da gibt es diesen Event-Blog, der w{\"a}re eine M{\"o}glichkeit, wenn man noch mehr Informationen zu diesem Alert haben m{\"o}chte. Hinsicht, ob man den Daten trauen kann, also, das kommt jetzt so ein bisschen drauf an, weil die Themen sind ja schon ein bisschen anders. In unserem Fall berichten sie ja von jemandem, der krank ist mit bestimmten Symptomen oder mehrere, die das gleiche haben oder Tiere. In dem Fall w{\"a}re da jetzt kein, okay, der Benefit w{\"a}re, da passiert was, richtig? Die Ministry of Health kommt, macht Vaccination Campaign, Chlorination Activities, bla, bla. In dem Fall jetzt mit dem Wasser, gut, wenn die Volunteers, also, was ein großer success Faktor f{\"u}r dieses Projekt, CBS-Projekt mit SRCS ist, ist Supervision, das Refresher Training, Supervision, dass die Supervisor geben, glaube ich, mindestens einmal im Monat, zumindest in der Vergangenheit. Jetzt werden die, die brauchen das nicht mehr monatlich, weil wir haben jetzt Evaluation gemacht, wir sehen es in den Daten, die wissen, was sie tun. Die kennen ihr Business. Das wird in dem Projekt am Anfang mehr intensiv sein, dass die Supervisor da h{\"a}ufiger hinm{\"u}ssen, checken m{\"u}ssen, haben sie das jetzt wirklich richtig eingesch{\"a}tzt. Und wenn die nach einer Weile sehen, dass die oft falsch, falsche Reports senden, also immer sagt, es ist leer, es ist leer, in der Hoffnung, sie kriegen vielleicht mehr Wasser in dieser Community, dann muss man sich halt was einfallen lassen. Also das ist jetzt was, was mich spontan einf{\"a}llt, was sein k{\"o}nnte, einfach die Hoffnung, es ist zwar voll, aber ich sage jetzt trotzdem, es ist nur halb voll, weil dann kommt jemand und gibt uns mehr Wasser, gerade mit den Droughts der vergangenen Zeit, das k{\"o}nnte sein, aber dann muss man einfach mit den Volunteers arbeiten und denen quasi auch klar machen, was passiert wann und wir checken das. Was man auch machen kann und das ist weniger sensitiv als bei uns, Fotos schicken, wenn die Smartphones haben. Das ist nicht immer der Fall, aber in manchen Projekten werden Fotos von Symptomen, zum Beispiel von der Haut oder so geschickt. In dem Fall jetzt mit diesen Wasserspeichern k{\"o}nnte man einfach, wenn man denn Smartphones verteilen m{\"o}chte, ich bin kein Fan davon, aus unterschiedlichen Gr{\"u}nden, aber wenn das Teil der Geschichte ist, dann kann man die auch fragen, ob sie nicht ein Foto schickt, zusammen mit dem Report, um das nochmal zu validieren, wenn man nicht da vorbeifahren will. Aber in den Trainings zeigt man denen halt, okay, ab wann ist es voll, ab wann ist es halb voll, wie soll es aussehen und dann kann man erst so ein bisschen durch supervision und feedback kontrollieren. Und wenn der Volunteer immer, also irgendwann merken sie auch, das macht keinen Sinn, der kommt keine und bringt mir extra Wasser. Dann hat es so erledigt, das will ich jetzt mal. Aber gut, man weiß nie, was am Ende, wirklich da eine Community ist passiert. 

Interviewer: Ja, wie ist das denn, wenn man da ist, wenn man jetzt da reingeht, haben die dann sofort auch irgendwie die Expectations, dass sie auch Hilfe bekommen? 

I: Ja.

Interviewer: Weil es kann ja auch sein, okay, in dieses Risiko mit Wasser geht ja auch noch mehr rein, also vielleicht kann die Community sich auch selber helfen oder man sagt, okay, die haben halt nur so viel, wie ist da also die Erfahrung, vielleicht bei dem Haupt, von dem Projekt, vielleicht auch von anderen Projekten, wie kann man damit umgehen, von Anfang an im Design-Prozess? 

I: Okay, jetzt war es ein bisschen weg. Okay, aber ich glaube, ich habe es verstanden. Also, was wir, wenn wir mit Ministery of Health zusammenarbeiten oder {\"u}berhaupt uns daf{\"u}r entscheiden, ob wir CBS machen oder nicht, es muss sicher sein, da ist, es passiert was, nachdem dieser Report geschickt wurde. Und das erste, was passiert, ist, dass die Volunteers, die sind ja geschult in First Aid und Health Promotion, die k{\"o}nnen schon mal ein bisschen was tun. Also die Community, es ist nicht so, ach, ich bin krank und nichts passiert. Also die Volunteers tun was. Wir haben die Mobile Clinics von SRCS, die manchmal die sind die antworten, aber das funktioniert inzwischen ganz gut aber mit logistischer Hilfe oder finanzieller Hilfe kommt auch Minister of Healthy. Genau, also was klar sein muss ist, es geht nicht nur darum, einen Report zu senden und da ist ein Bedarf und nichts passiert. Also was klar sein muss von Anfang an, ist, was passiert wann, wann erwarten wir und das muss man mit den Communities auch klar machen, wann, in welcher Situation erwarten wir von euch, dass ihr euch selbst helft. Aber das wird ja sicherlich vorher definiert, wann m{\"u}ssen wir wirklich mit einem Truck kommen und diesen Speicher auff{\"u}llen oder wann sollten die noch warten oder was weiß ich. Also das auf jeden Fall mit denen besprechen und klar machen, sonst stehen die da und am Ende sind es die Volunteers, die die Probleme haben. Und es f{\"a}llt negativ auf SRCS zur{\"u}ck.

Interviewer: Aber das Netzwerk ist groß genug auch von den Supervisoren, dass es nicht komplett automatisiert sein muss. Man kann schon davon ausgehen, da sagen wir, sie melden jetzt es ist halb voll, dann halb leer, dass der Supervisor da anrufen kann und sagen kann, okay, unsere Kapazit{\"a}ten sind an der Grenze.

I: Okay, es bricht zu sehr. Ich habe einen Teil jetzt gar nicht verstanden. Also irgendwas mit Netzwerk von Supervisoren habe ich noch was verstanden. Ist das? Ja, jetzt ist besser. 

Interviewer: Okay, wir versuchen mal n{\"a}her ran. Ja. Ist es m{\"o}glich, dar{\"u}ber zu gehen, dass man sagt, okay, nachher, wenn man sagt, es ist nur noch halb voll, der Supervisor sagt dann, okay, uns fehlen die Kapazit{\"a}ten, euch was zu liefern gerade. Anderen Communities geht es noch schlimmer. Ihr m{\"u}sst besser rationieren. Wie ist die Erfahrung da? Ist das kleinteilig genug, dass das nicht automatisiert sein muss, diese Feedback-Messages?

I: Also ich glaube, das kann man am Telefon besprechen. Ich w{\"u}rde nur vorher den Community sagen, dass es Situationen geben kann aufgrund von A, B, C, D, dass wir eben nicht in der Lage sein werden, das sofort zu f{\"u}llen, dass es l{\"a}nger dauern kann. Und wir euch dann bitten zu rationalisieren. Dann kann man mit denen besprechen, wo sie glauben, dass man dann vielleicht rationalisieren k{\"o}nnte. Vielleicht kann man denen auch versuchen, ein Timeframe zu geben, wie lange man br{\"a}uchte, damit die das selber einsch{\"a}tzen k{\"o}nnen. Aber, ja. Genau. Also falls das der Fall sein sollte, dass der Fall eintritt, das muss man vorher besprechen und das ank{\"u}ndigen, dass es passieren kann, weil dann verliert man zumindest nicht das Vertrauen. Wenn man jetzt schon antizipieren kann, da ist das Risiko, und du weißt es, Contingency Plan, at Community Level, dann zusammen mit denen werden sich vielleicht auch meckern und beklagen, aber dann weiß man zumindest, was passieren kann. Dann ist der Schaden wahrscheinlich geringer f{\"u}r das Image von allen Beteiligten. Ja. 

Interviewer: Okay, wir sind jetzt schon {\"u}ber die halbe Stunde. 

I: Ja, ich weiß nicht, was ich noch sagen kann.

Interviewer: Vielleicht einmal noch eine qualitative Eingliederung, dessen wir dar{\"u}ber auch gesprochen haben. Was waren so die Key Points, die Key Lessons learned, aber auch die Probleme, wo man sagt, da k{\"o}nnten wir jetzt drauf achten. Was sehen Sie als kritischste Punkte an? Oder wo sagen Sie, okay, da sollte man jetzt von Anfang an sehr genau drauf achten? Und welche Herausforderungen sind gekommen? Was k{\"o}nnte der Fokus sein, was sollte im Fokus stehen durch die Erfahrung?

I:  Um so ein Programm einzuf{\"u}hren, nicht NYSS an sich, aber CBS an sich ja?

Interviewer: CBS und genau an sich, aber jetzt vielleicht auch in speziellerer Bezug auf das Water Early Action, Anticipatory Action, Water Monitoring. Nat{\"u}rlich herauskommt aus dem eingef{\"u}hrten CBS. 

I: Also, so ein paar Keys f{\"u}r die Implementierung oder Planung von CBS. Also, keine Ahnung, ich werde es jetzt einfach rein und manche sind vielleicht schon auf dem Schirm und andere nicht. Also, wir machen ja immer, bevor wir CBS tun, ein Assessment richtig? Im Land auf Nationellebene, auf Projektebene, um zu gucken, gibt es da schon was. Weil ich habe das ja schon zweimal gehabt im S{\"u}d-Sudan, in Nigeria, eigentlich gibt es da schon CBS. Dann muss man gucken, welche gap es dann g{\"a}be und wie kann die National Society das f{\"u}llen. Das ist eins. Und dann treffen wir Minister of Health und erkl{\"a}ren, manchmal, also oft treffe ich die vorher schon, und erkl{\"a}re, was wir tun, was wir dieses Assessment tun und man muss sowieso Interviews mit denen f{\"u}hren. Insofern kann man das gut kombinieren, um schon herauszufinden, was eigentlich, also in dem Wasserprogramm w{\"a}re es dann irgendwie diese andere Department. Und dann spiegeln wir die Ergebnisse zur{\"u}ck an Ministery of Health und dann zusammen entscheiden wir halt, was machen wir, so machen wir es, wie machen wir es. Und dann versuchen wir nat{\"u}rlich auch unsere Vorstellungen damit reinfließen zu lassen, wie zum Beispiel die Plattform zu benutzen. Oder wenn wir zum Beispiel sehen, dass bestimmte Health Risks die Ministery of Health m{\"o}chte, wir aber denken, das macht keinen Sinn oder es sind zu viele. Wir machen immer, wir entwickeln die Strategie, das Protokoll f{\"u}r CBS immer zusammen mit Ministery of Health. Die sind mit in den TOTs (Training Of Trainers), dann Volunteers sind von der Community, die Community Leaders, suchen die aus. Hier muss man uns in unserem Fall, und ich glaube, SRCS ist generell eine Strategie, keine Incentives. Das ist tats{\"a}chlich auch nur zum Beispiel nur in Somaliland mehr oder weniger der Fall, dass die ohne Incentives arbeiten, deren Incentive ist quasi Trainings. Da kriegen wir dann immer f{\"u}r den Transport. Genau, also Community muss die aussuchen und I3 kann da auch noch mehr Inputs geben, weil am Anfang oder oft, gerade wenn Incentives gezahlt werden, dann nat{\"u}rlich die Tochter der Sohn oder wie auch immer da ausgesucht wird. Die gehen aber wirklicherweise in drei Monate studieren und dann muss man einen neuen finden. So, also oft sind es Frauen, die Volunteers sind, weil die eben in der Community bleiben und nicht wie die M{\"a}nner, die mal hier mal da sind. Genau, dann, genau, regular supervision, evaluations, talking with communities, Ministery of Health. Wir haben regelm{\"a}ßige Meetings, wir sind in den Meetings mit dem Minister of Health. Ja, ich glaube, das andere habe ich dann auch schon, also nicht, dass man mit den Communities am Anfang, reden muss, wenn man das Programm aufsetzt, aber ich glaube, SRCS, machen das ja, die sind ja keine Anf{\"a}nger. Die machen das auf jeden Fall, also die wissen das einfach. Ich glaube, es ist einfach ein Teil der Kultur das zu tun, bevor man irgendwo reingeht mit irgendwas. 

Interviewer: Gibt es das mit dem internationalen Kontext oder aus dem nationalen Kontext noch andere NGOs oder andere Player, die da irgendwie ihre Finger mit rein mit drin haben wollen oder doch noch ein bisschen querschießen?

I: Querschießen nicht, aber zum Beispiel, also versuchen wir mit der WHO also mit der Weltgesundheitsorganisation, die sind ja in allen L{\"a}ndern, mit denen auch zusammenzuarbeiten, die sind interessiert daran, was wir tun. Es gibt CDC, Center for Disease Control, die gibt es in, aber die geh{\"o}ren meistens zum Ministery of Health. Das Problem mit anderen Organisationen, okay, querschießen tun die nicht, aber also das Problem, was wir zum Beispiel im S{\"u}dsudan hatten, da wo wir hin wollten, da war schon eine andere Organisation, die Community Health gemacht hat und CBS, aber nur zum Teil. Die Communities haben sich aber zum Beispiel beschwert, dass die das eigentlich gar nicht tun oder nicht genug und die Qualit{\"a}t nicht gut ist und die wollten, dass das Rote Kreuz mehr macht. Aber am Ende kann man da, und das Bekloppte war auch das Ministery of Health einfach keine Ahnung hatte, wer was gemacht hat, weil die h{\"a}tten von Anfang an sagen m{\"u}ssen, nee, da haben wir schon jemand. Wir haben ein großes Assessment gemacht und am Ende habe ich mit diesen Organisationen zusammensetzt und dann erz{\"a}hlen die uns, dass die da quasi diesen Plan haben oder bereits angefangen haben, dass {\"a}hnliche Sachen zu implementieren und selbst wenn die Qualit{\"a}t schlecht ist, kann man da nicht einfach das gleiche machen. Das glaube ich aber jetzt nicht, in Somaliland, dass da so viele andere sind, die das gleiche machen. CBS definitiv nicht, da ist niemand anders. Ich weiß nicht, ob MSF da irgendwo ist, aber ja, in Somalia, glaube ich nicht. Ah ne, die sind auch in Somalia. Aber genau, da kann man ja, und normalerweise sollte das das Ministry wissen, wer da was macht und wenn nicht die, dann sp{\"a}testens die Communitys. Genau, also man muss einfach gucken, wenn man das Assessment macht, sind da andere irgendwo, es kann ja, muss ja nicht unbedingt in den Projektlocations sein, woanders, von denen man auch lernen kann. Also es ist eher diese Competition-Geschichte, als, und da muss man halt viele Organisationen, die haben halt das Geld f{\"u}r diese Region bekommen, dann m{\"u}ssen die da hin, ne, und dann wird es aber einfach, ja, ein bisschen bl{\"o}d. Aber generell so, gegen CBS, alle wollen CBS machen. Alle wissen, dass es gut ist. 

I: Ja, also ich finde es ziemlich erstaunlich, weil ich habe jetzt in der wissenschaftlichen Literatur sehr wenig gefunden.

I: Es ist leider nicht viel ver{\"o}ffentlicht und ich versuche seit zwei Jahren, einen Artikel zu schreiben. {\"u}ber CBS an sich.

Interviewer: Also CBS, ja, also es gab jetzt von 2021, meine ich, noch ein Paper, die jetzt gerade auch f{\"u}r Broad sehr daf{\"u}r advokiert haben, mehrere Sachen mal aufzunehmen, was aber so ein bisschen mehr in Data Collection eingeht. Dass man halt sagt, okay, wir regieren jetzt nicht mehr dar{\"u}ber mit Satellitendaten und großen nationalen, {\"u}bernationalen Datens{\"a}tzen, die wir erheben, sondern wir wollen halt vor allem auch ein Impact Assessment machen von Drought. Haben durchaus ein, zwei sehr aufwendige Questionnaires entwickelt, nicht unbedingt komplizierte, aber durchaus aufwendig, die dann auch in diese longitudinal studies mit reinpassen von WHO. Also da gibt es mehr oder minder auch eher einen Call f{\"u}r diese Paper. 

I: Ja, genau, also von Conlfict and Health habe ich schon seit zwei Jahren Call for Papers, um was {\"u}ber CBS zu ver{\"o}ffentlichen. Ich habe diverse Papers. Ich weiß nicht, ob das genau, das ist, also ist mehr health related. Ich kann die, kann die schicken, wo ein bisschen, ja, zumindest was publiziert wurde {\"u}ber CBS, wie es implementiert ist mit den Outcomes. Aber genau, wir haben es. Ich brauche nur Zeit, um ganz genau zu sagen, was wir hier im Lande sind. Ja, das kann ich noch schicken. Genau, noch mal, was halt oft der Fall ist, aber ich glaube, in Somaliland ist das nicht das Problem. Es gibt L{\"a}nder, wo die Regierung einfach sagt, weil oft das Label CBS Surveillance hat eine negative Implikation. Zum Beispiel in Pakistan k{\"o}nnen wir CBS so nicht verwenden. Wir sagen halt, wir berichten von Kranken aus der Community. Weil Surveillance, da gehen die Alarmglocken an, dass man da irgendwo ausspioniert oder so. Also das, aber dadurch, dass Somaliland jetzt zumindest im Gesundheitsministerium inzwischen einfach viele Jahre Erfahrung und gute Erfahrung hat nit CBS, sollte man die nutzen. Ja, um das auszuweiten auf andere Themen. 

Interviewer: Okay, dann h{\"a}tte ich da auch nur noch eine Frage oder vielleicht eine Bitte. Also nach einer Einsch{\"a}tzung von der generellen Impression {\"u}ber das gesamte Projekt, jetzt vielleicht CBS als auch dieses, diesen Ausblick, sag ich mal, auf ein m{\"o}gliches Wasserquellen-Monitoring und vielleicht auch noch eine Frage, gibt es von Ihrer Seite irgendwie W{\"u}nsche oder so, dass die gerne irgendwie so aus der Erfahrung mit reinfließen sollten?

I: Also wir machen ja alles jetzt basierend auf Erfahrung der letzten Jahre und Projekte. Und die fließen dann immer direkt und die sind jetzt auch an die, das was ich gerade erz{\"a}hlt habe, eingeflossen. Also warum ich glaube und versuche, dass n{\"a}chste Woche, wenn ich mit denen spreche, da vielleicht doch zu veranknern kann, es sei denn, genau, die Bedarfe sind gr{\"o}ßer als das, was ich jetzt verstanden habe. Also diese Wassersourcegeschichten, deshalb es w{\"u}rde halt thematisch gut passend, weil es ein Health Risk ist, um auch Tiergesundheit, Menschgesundheit negativ zu beeinflussen und Outbreaks zu, wie sagt man, also quasi die Basis zu bereiten f{\"u}r Outbreaks, ob es jetzt beim Tier oder beim Mensch ist. Insofern w{\"u}rde das thematisch eigentlich, und wir haben ja diesen Unusual Events, wo eigentlich genau oder {\"a}hnliche Sachen ja bereits schon ber{\"u}cksichtigt sind. Das Spezifische hier w{\"a}re quasi, dass sie von einem bestimmten Wasserpunkt oder watersourcepunkt kommen und dann vielleicht noch Unterkategorien hat, aber ansonsten ist es, der ist jetzt nicht neu neu. Insofern finde ich, w{\"u}rde es gut passen, aber man weiß immer nicht, was ich vielleicht {\"u}bersehen habe. Das werde ich jetzt n{\"a}chste Woche herausfinden, aber bislang haben sie mir noch nicht gesagt, dass es total abwegig ist. 

Interviewer: Genau, von meiner Seite ja auch, ich versuche ja gerade diesen Designprozess, dieses Projekt zu designen, und da geht es ja gerade jetzt, vor allem jetzt momentan in der Phase, deswegen hatte ich auch nachgefragt, ob wir uns jetzt schon unterhalten k{\"o}nnen, ganz klar um diese Constraints. Also wenn ich weiß, okay, alles was m{\"o}glich ist, ist... Wir haben Berkad  1, 2, 3, und ich kann sagen, voll, halb leer oder leer, oder vielleicht 5 Phasen, und das ist das, was das System leisten kann, und alles andere m{\"u}sste {\"u}ber das Telefon mit dem Supervisor geregelt werden, dann ist das ja vollkommen in Ordnung. Deswegen frage ich jetzt genau, also ich bin pers{\"o}nlich gar nicht festgefahren, und ich glaube auch hier, jetzt sonst keiner bei uns, dass es jetzt genau so oder so aussehen muss, sondern es soll funktionieren und es soll in den Prozess reinpassen, deswegen ist bei uns, denke ich, eine hohe Offenheit da, und gerade jetzt eben herauszufinden, in was f{\"u}r einem Kontext, in was f{\"u}r einem Rahmen arbeiten wir, was k{\"o}nnen wir machen und wie k{\"o}nnen wir es machen. 

I: Ja, vielleicht noch die Idee, warum wir zum Beispiel diese Codo-Sache haben, oder auch, dass das alles mit einem normalen Telefon ist, was der Unterschied zu vielen anderen Tools ist. Ich glaube, es gibt kaum ein anderes Surveillance-Tool, was mit einem normalen Basic Phone m{\"o}glich ist. Die meisten brauchen ein Smartphone, und das ist in vielen Gegenden, wo wir arbeiten, nicht m{\"o}glich. Erstens kann ich nicht st{\"a}ndig Smartphones verteilen, oft haben wir auch gar kein Netzwerk und so weiter und die Volunteers mit denen wir arbeiten, das Kriterium f{\"u}r die ist nicht gut gebildet zu sein und oft bei den Tools die es gibt wo mehr Daten mehr Informationen n{\"o}tig ist f{\"u}r Case management, zum Beispiel wo die Volunteers dann selber eintragen m{\"u}ssen, dann brauchst du jemand der Englisch spricht oder zumindest schreiben kann und das ist halt auch nicht der Fall. Und durch das Telefonat, wenn man mehr Informationen braucht zu einem bestimmten Report, das kann man einfach so wie bei uns auch der Supervisor einfach erledigen, der Zugang zu zur Plattform hat. Weil dies m{\"o}glicherweise gar nicht, nicht eingeben k{\"o}nnen... 

Interviewer: Ich hatte bei NYSS bei den Codes gesehen, dass man schon sagen kann m{\"a}nnlich, weiblich, unter 5, {\"u}ber 5, sind so zwei bis drei Sachen, also durchaus zum Beispiel was Wasserqualit{\"a}t angeht, je weniger Wasser drin ist desto mehr Schadstoffkonzentration habe ich ja auch durchaus h{\"a}ufig. Kann man also das vielleicht auch noch mitnehmen pro Berkad, dass man sagt pro Berkad, kann man auch noch sagen, der ist voll und die Wasserqualit{\"a}t sieht gut aus, und er ist auch accessible f{\"u}r uns? 

I: Also wir werden es noch nicht gezeigt, es ist auch m{\"o}glich mit NYSS, wenn wir zum Beispiel Outbreaks haben, kann man nicht mehr zu jedem der krank ist einen Report schicken, das macht keinen Sinn, weil wir wissen, da ist ein Outbreak, jetzt geht es mehr darum zu wissen, wie viele pro Tag hast du gefunden und so weiter. Also es gibt auch die M{\"o}glichkeit mit noch mehr Codes mehr Informationen zu vermitteln. Also zum Beispiel haben wir, wenn Cholera-Ausbruch ist, werden bestimmte Volunteers, die m{\"o}glicherweise, also die werden ausgew{\"a}hlt aufgrund wahrscheinlich auch ihres kognitiven, kognitiven Capacity, ja, die dann von diesen Oral Rehydration Coins einmal pro Tag eine l{\"a}nger Code schicken. Der beinhaltet dann, okay, wie viele waren heute da, wie viele waren weiblich, wie viele m{\"a}nnlich, wie viele unter f{\"u}nf, wie viele {\"u}ber f{\"u}nf, wie viele sind da gestorben, wie viele sind von einem anderen Village gekommen, also ich glaube es sind am Ende bis zu sieben Zahlen. Wenn man den kleine Tools, also was wir am Anfang machen, die ins Somaliland, die Volunteers, die wissen das inzwischen, aber die haben auch so kleine Zettelchen, wo die Codes quasi, kann ich auch noch schicken, Codes quasi erkl{\"a}rt sind, was die bedeuten und wie sie sich, also sie kriegen ja Trainings, aber das kann man ja nicht alles behalten, ich auch nicht, ich muss st{\"a}ndig gucken, welcher Code ist jetzt was und dann kann man denen das geben, wenn die auch mehr, also was man {\"u}berlegen k{\"o}nnte in dem Fall ist, genau der erste Code ist vielleicht, die Nummer des Water Source, dann, ob sie voll ist oder nicht, dann kann man sagen, eins, zwei, drei und dann Hasch, keine Ahnung, welche andere Kategorien m{\"o}glich sind, dass man da noch zwei, drei, w{\"u}rde ich zu weit gehen, andere Codes, weil das schr{\"a}nkt dann wieder ein, wen man als Volunteer nehmen kann und dann kann man das auch schreiben, was was ist. 

Interviewer: Genau, also zwei bis drei andere Codes, das war auch so das, was ich mir vorgestellt hatte und was dann ja auch schon viel helfen k{\"o}nnte, weil auch wenn die Wasserqualit{\"a}t, vielleicht viel Wasser da ist, aber sie sagen, die Wasserqualit{\"a}t ist schlecht, auch dann k{\"o}nnte man ja schon eine Early Action draus machen, dass sie sagen, okay, wir bringen irgendwie etwas um, also zum Beispiel Chlor, um eben einen Ausbruch von Krankheit, wegen schlechter Wasserqualit{\"a}t, in dem Fall schon. 

I: Und dem wird dann erkl{\"a}rt, wann das Wasser schlecht ist, wie sie das einsch{\"a}tzen k{\"o}nnen. 

Interviewer: Genau, das ist jetzt gerade ein Gedanke, der l{\"a}uft parallel, aber genau das m{\"u}sste nat{\"u}rlich alles mitlaufen und vielleicht kann man dann auch so ein kleines Zettelchen oder dann, wie muss es riechen, wie muss es schmecken, wie ist es, wenn man sagt, okay, jetzt wird es kritisch. Ich bin kein Experte im Wassermanagement. Ich glaube, das kann man mal... 

I: Ich habe nicht verstanden, dass die Tiere da auch dran trinken oder nicht? 

Interviewer: So wie ich das verstanden habe, gibt es davon ganz, ganz viele unterschiedliche M{\"o}glichkeiten. Also manche von diesen Berkads sind einfach nur L{\"o}cher im Boden, andere sind von NGOs gebaut mit betoniert, andere sind dann auch weiter, dass sie sagen, okay, sie haben sogar noch ein Blechdach dr{\"u}ber, die dann sogar noch weitergehen und sagen, okay, da verdunstet das dann und alles, was verdunstet, l{\"a}uft ab in einen extra Trichter, in so ein extra Gef{\"a}ß, was dann schon dadurch dann eigentlich mehr oder minder sauber ist, weil es erst mal verdunstet und dann abl{\"a}uft. Also es gibt wohl sehr, sehr viele und auch das ist so ein bisschen noch eine Frage, ist der Berkad {\"u}berhaupt funktional? Also das w{\"a}re auch so ein bisschen eine Early Action, wenn wirklich eine Rainy Season kommt, welcher Berkad ist {\"u}berhaupt in der Lage, Wasser aufzunehmen und welcher braucht erst noch Reparaturen? Aber das w{\"a}re nochmal ein bisschen was anderes, das l{\"a}uft so ein bisschen nebenher. Das Thema wird doch im Detail sehr komplex und es wird schon nochmal [...], aber ich glaube, das ist ja bei den meisten so. Das ist nat{\"u}rlich jetzt nichts mehr Neues. 

I: Okay, nur noch ein Kommentar, falls das aus irgendwelchen Gr{\"u}nden nicht sein sollte, wie ich es vorher schon gesagt habe, es gibt ja diverse andere M{\"o}glichkeiten, die gehen. Also das Gute an dies ist all das automatische, richtig? Feedback messages, notifications zum Ministry, notifications zum Supervisor, automatische Maps, Graphs and so forth. Das aber wenn jetzt aus irgendeinem Grund das nicht m{\"o}glich ist, dann kann man immer noch mit normaler SMS, die der Supervisor dann nachverfolgt und den Eintrag in Excel macht, dann seine eigenen automatischen, zum Beispiel das Team in Burkina Faso, die National Society, die haben super Typen, der da ganz tolle automatische Graphs in Excel kreiert, die alle {\"a}hnlich sind wie NYSS, nur dass es eben manuell eingetragen werden muss. 

Interviewer: Ja, sonst ist NYSS aber ja auch Open Source, wenn ich richtig gesehen habe.

I: Ja, genau. Aber also wenn jetzt zum Beispiel, jetzt was ich meine ist, wenn jetzt zum Beispiel IFRC zum Beispiel sagt, aber wir wollen {\"u}ber unusual events nicht hinausgehen f{\"u}r solche Sachen, dann macht es keinen Sinn. Also wenn wir jetzt nicht sagen, weil in dem Fall w{\"u}rde das tats{\"a}chlich bedeuten, wir brauchen neuen Code f{\"u}r diese Geschichte und dann all die anderen Codes, also Arbeit ist da schon und die Frage w{\"a}re dann auch, okay wer zahlt die Arbeitsstunden, macht das Norwegische Rotkreuz bla bla bla, wenn da Interesse ist oder muss es vom Deutschen Rotkreuz getragen werden, die extra Stunden, die f{\"u}r diese Weiterentwicklung n{\"o}tig sind. Genau, wenn das nicht stattfindet, dann macht es m{\"o}glicherweise einfach keinen Sinn, weil das, was die Idee ist, dann einfach damit nicht m{\"o}glich ist, wenn man nur sagen kann, ich habe ein unusual event, ist die Frage, ob man nicht besser, einfach eine SMS schickt zum Supervisor mit mehr Information. Aber das k{\"o}nnen wir gucken, vielleicht weiß ich  Ende n{\"a}chster Woche mehr. 

Interviewer: Gut, dann dr{\"u}cke ich da mal die Daumen. 

I: Ja, genau, ich auch. Gut. Dann kann I1 ja den Link mit I3 herstellen, der ist gerade in Nairobi, um f{\"u}r Sanlisa f{\"u}r Oslo zu k{\"a}mpfen. Genau, dann ist er in Oslo deswegen, aber jetzt w{\"a}hrend er in Nairobi ist, hat er vielleicht auch Zeiten online. 

Interviewer: Gut, vielen Dank. 

I: Alles gerne. Viel Erfolg. 

Interviewer: Danke sch{\"o}n. Falls noch irgendwie Gedanken kommen oder so, oder auch die Sachen gerne {\"u}ber E-Mail. 

I: Die Artikel schicke ich noch. 

Interviewer: Vielen Dank. Und vielleicht h{\"a}ufiger von anderen Tools geredet neben Kobo, falls da noch mal kurz eine ganz kurze informelle Liste, falls da irgendetwas... 

I: Zu Kobo? 

Interviewer: Ne, nicht unbedingt zu Kobo, aber falls es noch mal andere Ideen gibt oder noch mal wie auch immer. 

I: Ja, okay, alles klar. Ja, aber wahrscheinlich w{\"a}re Kobo dann schon eher die bessere L{\"o}sung, falls es mit NYSS nicht klappt. 

Interviewer: Okay, vielen Dank. 

I: Okay, viel Erfolg. 

Interviewer: Danke sch{\"o}n. 

I: Tsch{\"u}ss. 

Interviewer: Tsch{\"u}ss. 

I: Ciao, ciao. 


\section{Questionnaire  \& Answers I1.2}
Interviewer: Bosse Sottmann\newline
Medium: Google Forms\newline
Interviewee: GRC FbF Manager of the SRCS \newline
Date: 28.02.2023

\subsection*{Introduction}
Hello and As-salamu alaykum,\newline
thank you for taking time to give some insights to your experiences!\newline
My name is Bosse Sottmann, and I am currently studying at the Heidelberg University and am enrolled in the Master's programme in Geography. In the context of this programme, I am currently working on my thesis in the HeiGIT team that is involved in the development of the Early Action Protocol.\newline
The overall goal of this project is the development of a mapping and monitoring approach on community level primarily for the water source type of Berkads to ultimately enable triggering action before critical water levels. Sub-goals are based on the learnings which water levels trigger which actions, what information needs to be known about the source initially, continuously and what other information would be helpful in the context of Anticipatory Actions.\newline
The work will be based on a variety of different sources of information. In addition to this questionnaire and others, best practices and knowledge will be gathered from the literature. Therefore, your input to this questionnaire is critical in learning more from the local perspective in order to not only transfer experiences and learnings to the new design but make this applicable to the local circumstances as best as possible. Your and the SRCSs opinions, experiences and needs will be the foundation of the work – ensuring that the resulting design meets your requirements.\newline
The questions are structured in multiple stages and each question will be open-ended. Thus, feel free to add more information where ever you want or think necessary. Additionally, at the end of the questionnaire, further remarks can be made.\newline
Nonetheless, a disclaimer should be made, that this is a project in the context of a Master's thesis, thus a fully-fledged design ready to get launched is out of scope of this work. Yet, it can lay a good foundation for the following work.\newline
All answers will be confidential and you can skip any questions you are not comfortable with.\newline
Thank you very much for your time and energy!


\subsection*{General Questions}
Please give a short introduction to yourself, your role, experiences and work.
\textit{I1. Im the GRC Forecast based Financing delegate based in Hargeisa, Somaliland. I am supporting the SRCS to undertake the Forecast based Financing project. The project aims to develop Anticipatory Actions that will counter forecastable hazards such as drought, flooding, cyclones and epidemics.}

\subsection*{Local context}
Please tell me about the current conditions on site to gain a better understanding of the local circumstances by walking me through the process of how anticipatory actions/response in regard to water availability/shortage currently work on community level.\newline
\textit{Due to the recurrent droughts, the water sector in Somalia/Somaliland has been greatly impacted. This relates to water shortages (quantities) then also reduced water quality. The main response activities that have been adopted to address the current water crisis are berked rehabilitation (SRCS), water trucking (other agencies), distribution of water purification tablets (SRCS), multi purpose cash, awareness campaigns related to hygiene promotion (SRCS). Regarding Anticipatory actions, there has been any actions yet due to the fact that there is no water monitoring and trigger mechanism in place. Assistance/response is based on the initial prioritization of target areas that SRCS conducts. The prioritization is based on assumed vulnerability per community based on Number of IDP camps in the area, number of women headed families, predicted IPC classifications etc. Activities such as berked rehabilitation are done in consultations with the communities and SRCS branches who flag/identify berkeds in need of repairing. Repairing may consists of re roofing/ roofing, and brickwork to strengthen the structure. The berkeds are meant to capture run off water in case of rainfall incidences. In cases where there hasnt been rains for a prolonged time then water trucks are deployed to deliver water to the communities. Cash has been an imoortattn modality to adress the water shortages. In the current prevailing drought, water and food insecurity crisis, water is now being sold by private players. So the cash has come in handy to t least enable the communities to buy fresh water for drinking. Water sources such as dug wells are often contaminated as livestock i,e camels, goats also drink water from those same water bodies as well.}

\subsection*{Anticipatory Actions}
The current recommendations for the development of triggers for drought focussed Anticipatory Actions are that these should be staggered and closely related to the development of the overall situation and local impacts.

Based on your experience, are water levels in berkads good indicators for drought impacts on the community? Could the indicator be enhanced by local knowledge?\newline
\textit{Water levels in berkeds could be a good indicator, however it cannot be a stand alone indicator. This has to be combined by meteorological forecasts and local knowledge as well.}

The ultimate anticipatory action would be the trucking of water to those who need it most. Going further, one could think about other anticipatory actions that could be triggered beforehand such as awareness raising, information dissemination and involving private berkad owners. What do you think about these proposed Anticipatory Actions? Which potentials and challenges do you see? \newline
\textit{Awareness raising and information dissemination should be more on informing the communities on how to improve water quality at local level e,g boiling before drinking. Involving private berked owners is also feasible however their involvement could be limited as they are more concerned about their business models i.e selling of the water and preserving their berkeds than being part of the overall response/Anticipatory action mechanism. Nevertheless there is the potential to work closely with the private berked owners. This can be done through rehabilitation of their privately owned berkeds in return for their involvement in response and anticipatory action activities related to addressing water shortages.}

Do you possibly have other, more specific or different ideas for Anticipatory Actions? \newline
\textit{yes, i) timely distribution of cash to enable communities to buy and stock fresh water}\newline
\textit{ii) timely distribution of water purification tablets}\newline
\textit{iii) timely rehabilitation of other water sources such as boreholes}

\subsection*{Monitoring}
To facilitate monitoring, the water levels need to be categorized.  

Which water level categorisation do you think as useful? (e.g. how many categories?) How detailed does it need to be in order to be useful and how coarse does it need to be to remain monitorable?  
\textit{These water levels are ideal i.e }\newline
Empty (no water at all) \newline
\textit{Critical (1 day of water supply remaining),}\newline
\textit{Low (3 days of water supply remaining),} \newline
\textit{Middle (5 days of water supply remianing)}\newline
\textit{High (full capacity)}

Which water level category should trigger which Anticipatory Action?  
\textit{Low category}\newline

Which parameters should be monitored weekly, monthly or even only annually?  
\textit{Water level (daily monitoring)}\newline
\textit{Berked condition (annually)}\newline
\textit{Number of people accessing the water form the berked (weekly/monthly)}


\subsection*{Water Quality}
Can you think of ways in which water quality could be included in the monitoring process? \newline
\textit{Water quality is difficult to monitor at community level as it is a technical activity. Unless if the SRCS through the branch staff are equipped with water testing equipment as well as training them on the water parameters to be tested.}

Do you know of any solutions that have proven effective for local water quality monitoring by volunteers in your given circumstances?  
\textit{I'm not aware of any.}

\subsection*{Resource limitations}
How do you currently decide which community to help when resources are scarce?\newline
\textit{The SRCS in consultation with the government select target communities based on a pre existing selection /vulnerability criteria based on either number of IDP camps etc}

What are your experiences? - What would be good ways to deal with potential loss of trust and frustrations in the moment and possibly beforehand?  \newline
\textit{Utilise the community based SRCS volunteers to engage communities and sensitise the communoties on the riole the SRCS plays. Also establishing a robust feedback and Complaints mechanism that ensures communities can easily relay their feedback.}

What other challenges do you see in regard to resource limitations?  \newline
\textit{The current crisis is huge and response activities are being overwhelmed by the need. This will lead to commercialization and overpricing of fresh water.}

\subsection*{Berkads}

In the beginning of the mapping and monitoring of Berkads, their location and related key information shall be captured. Determined key information are so far:
the location,\newline
ownership,\newline
total number of people or communities dependant on the berkad,\newline
its water storage capacity and\newline
functioning\newline
Which other social, technical or context related indicators, parameters or features would you add to this list of important information about Berkads in regard to Anticipatory Actions? Which challenges might arise in the capturing or monitoring of these information on site?\newline  
\textit{Other information might included the year it was built, the last time it was rehabilitated etc. However this kind of information might be missing as you need people with community/institutional memory to provide such kind of information. Somalis are highly mobile communities and it will be difficult to get information on past details per particular geographical area.}

Does the community have an idea of how long the water of their water sources will last? How good is this prediction usually?
\textit{Yes they have an idea. These kinds of predictions are good as communities usually have their own control measures to ensure equitable distribution of water e,g how many containers per family etc. The berkeds are usually locked to ensure there is controlled acess to the water stored.}

\subsection*{Water Trucking}

How does the trucking of water currently work? On what information do you act?

Which roles (e.g. funder, executer, manager, etc.) exist and who usually fills those?

What resources (human resources, finances, water, etc.) and how many/much of these are required for one action of water trucking?

How does the availability of water trucking is spread across Somaliland? Are there certain water points that are used for that?

How long is the average response time from getting the information to the filling?

\subsection*{Final remarks}
Would you like to share additional experiences, lessons learnt or other key points?\newline
\textit{It is proving to be a challenge to plan for anticipatory actions in an already prevailing crisis in Somalia/Somaliland. Already the needs are dire and the current SRCSs focus is on response mechanims to adres the already visible impacts of drought.}

Is there anything else you would like to add or any final thoughts you would like to share before we conclude the questionnaire?
\textit{water monitoring is vital as it will inform decision makers on the priority areas to focus on. Community level water monitoring plus weather forecast information will help form robust Anticipatory Action systems as well as informed response mechanism.}


\section{Questions\& Transcript I3}
Interviewer: Bosse Sottmann\newline
Medium: Zoom\newline
Interviewee,I: SRCS CBS Manager \newline
Date: 04.03.2023

All right, maybe I just shortly introduce myself. I don't know how much you know or how much people told you. I'm Bosse. I'm a master's student at the University of Heidelberg and I'm currently writing my math thesis exactly about this topic. So I talked to you. You possibly know Melanie? 

I: Yeah, my name is A. F. H. I3 is the last name. 

Interviewer:  All right. I'm sorry. 

I: No problem. I'm called I3. Dr. I3 always. So no problem for that. I will write down in the chat my full name so that we can... 

Interviewer:   I'm sorry, Mr. I3. All right. Oh, I'm sorry for that. That is on my head. 

I: No problem. It's okay. Always the people who are here call me I3 only. 

Interviewer:  Okay, how do I pronounce that? 

I: Dr. I3. 

Interviewer:  Dr. I3. All right. Cool. So Mr. I3. Yeah, so I'm writing my math thesis right now. And in the context of this, I'm trying to figure out how we can best set up a monitoring and mapping approach for water sources and for berkeds in the region of Somaliland. And I already got some great answers from I1. And I would be super grateful if you could give me some more information. Twofold. Once on the project of NYSS and CBS. As I talked to I2 and she recommended me to talk to you because I'm not a big fan of the project. But I talked to I2 and she recommended me to talk to you because she said, you know, all the things which happened on the ground and you're the expert in that. And if you still have a bit more time, I'd be grateful to talk about some of the monitoring things I still have open questions to. If you're okay with that.

I: Okay. So can I start or you have a question that you would like to ask me question by question and then my answer for that. 

Interviewer:  I think it would be great just to. Of course, you can give a general introduction if you want, if you like. That would be great, actually. Okay. 

I: Thank you very much. As already mentioned, my name is Dr. Abdifatah Hussein I3. So I would like to give you some information about community based surveillance and also NYSS platform. So when it comes to the community based surveillance, we started 2018 in Burao. The capital city of the Todgheer region, we piloted 75 community volunteers. And we piloted, we look at how community based surveillance is applicable in Somaliland. And what brought to our attention to establish community based surveillance at Buroa. This in Buroa, I mean in Todgheer region, there was a cholera outbreak 2017, which badly affected the communities in Todgheer region, and also other regions, but mainly badly affected in Buroa city, where about 700,000 people live in that area. And it came without saying or, you know, the cases was unpredictable and then it was escalating in the community and then they spread out all the community. So the problem came, you know, the people, for example, the Ministry of Health and the other people, they recognized that there is an outbreak and the outbreak at this peak. And then at that stage, SRCS or Somaliland Red Cross Society is, you know, at that time, giving warning and signals to the Ministry of Health and saying there is a cholera or acute water diarrhea, which is starting in these areas. And for many reasons, the Ministry was saying still the cases we have seen is still, you know, the normal cases we are getting from the communities or something like that. So the problem, you know, it reached this peak. And then at that time, SRCS and its sister organizations or BNS, they established, you know, SRCS, they sent the request that they can come to Somaliland to support so that, you know, the cholera can be managed, you know, because when it comes to the capacity of the government and also, and, you know, the magnitude of the disease became, you know, something which is not the government can not manage. In that case, we requested other national societies to come to Somaliland to support. So initially, Canadian Red Cross, they responded and then within 48 hours, they sent an ERU mission, Emergency Response Unit, so that they came here and they were well equipped with their vehicles and other medical logistics and staff and also the equipment which can be managed in the cholera outbreak. So they were having also what's called a tent for, in tent for cholera management, cholera treatment centers. We established that one and we were managing there for that time. And then again, we established what's called Oral Rehydration Points. So Oral Rehydration Points, we hired community volunteers, people who were provided training from the unit and then they went to the community, they are supporting the community because some people, when they have a diarrhea, they are not going to come to the health facility so that they can get the needed support because they were, you know, a bit reluctant to see other people that they have a diarrhea or, you know, have a stigma or something. So they were not happy to do that. But the Oral Rehydration Point, they supported us at the community level. So they were going house to house so that they can give health and health promotion activities in the same way. They were providing ORS, SYNC and also and other like aquatabs so that they can provide the water and something like that. In that case, this supported a lot. And, you know, the cases who are coming to the health facility or the cholera treatment center, they use it because they were getting support at the community level. In that idea, we said, as the SCRS, one of the lessons learned is that, you know, the cholera came to our country without saying. And we think about in the way that we can identify the cases in the community early enough so that we can identify and respond at community level. So we can stop, you know, an outbreak immediately when it has started or to be noticed early enough. So that's the idea. We came up with community-based surveillance. And as I already mentioned, we piloted and in the Todgheer region, we recruited 75 community volunteers and then the pilot became successful. Then at that time, we were focusing on three districts, Aynabo District, Oodweyne District, and Burao District. And then we scaled out to the other districts, like the Buhoodle District, which is in the Todgheer region. And then we again scaled up to the other regions and then we moved to Todgheer region and again Sool region. And last year [2022], we moved to Senaag region while the two rest areas and regions we scaled this year. So almost I can say now community-based surveillance reached all six regions in Somaliland. We are only focusing on the hotspot area where there is an epidemic, a prone disease areas or where we expect the outbreak to happen. We are not covering all the, for example, all the area, all the country. But we are covering the hotspot areas where we think that outbreak may start or happen. And there is a lot of outbreaks which the community volunteers identified and we have done investigation with the collaboration with the Ministry of Health. And then, you know, we still have that outbreak there. I can give you an example about that. For example, the first case of COVID-19 was from one of our community volunteers in the community. And then the Ministry of Health, they have done the investigation. They took the samples, they sent to Nairobi and then the case became positive. That's one thing. Okay. And the other thing I can mention is that they reported this kind of [fever] at community level. And then, you know, that case was solved at community level. We shifted mobile teams we have so that they can manage that cases. In the same way, there was a success, but I can say an outbreak of measles in the country. And then, for example, one community called a community volunteer in that community. He sent two cases of suspected measles. And one day after, he sent three others. And the next day, after one day, he sent three other cases. And so immediately, our community volunteers, I mean, our CBS officers at the regional level verified that it is much in the community case definition. If you may share, let me close the door. Okay, so that the cases, the Ministry of Health, SRCS team and again WHO together, they went into that community, they took samples and then they sent to the national lab for further investigation and the cases, two of the five cases became positive and then we have done mass immunization against the measles. And then at that time, I remember 5,300 children between age to nine months to nine years was immunized. So I think I can say is when it comes to the community based surveillance is one of the things that can easily detect early enough at community level, the health risk in the community and SRCS has mobile teams who can be deployed immediately within hours so that they can do the response. And also I would like to thank to the Norwegian Red Cross who are supporting this program to run since 2018 and then whenever there is an outbreak, we immediately request support and they profile the nearly support. So I think that is the general view of the CBS when it comes to the NYSS. We started and you know, it composes of, there is what I can say, instruments which you need when you are using NYSS. For example, you must have a mobile. So any mobile you can use it. So there is no need to have a smartphone but you are using SMS. So call an SMS. If you are using SMS, you will use at any mobile but the other thing is also there must be a network in that area and again SMS Eagle. SMS Eagle is a device which captures information which sends the community volunteers with a local member and then when it captures that information again, it uploads to the NYSS platform. So initially the NYSS platform, we are supporting the developers, the need at the community level. So whatever we need, we were discussing with them and then they were updating in that way. So it took around one year to build on but initially the NYSS platform, it captured the information but again we were downloading that information in the NYSS platform and then we manually analyzing through Excel format. But finally we reached a stage that the system or a NYSS platform can automatically analyze itself and again it can give you an alert if the [hella space] key reaches the threshold according to the geographical location that reports are coming from so that you can also early, so that you can get a message through your mobile or your email or something like that. So you can also flow up immediately when you get this alert and mobile all these things. So I think that's general view about the NYSS and also the CBS, how we started and all these things. Thank you and again if you have any further questions, please do not hesitate.

Interviewer:   I do. Thank you so much first of all for this very good coverage and introduction. I learned a lot. How do you, what do you think about NYSS? Would you do it again and why did you decide to do this kind of crowd sensing or working with your volunteers? How did that work specifically? Do you know, so how do you recruit volunteers? How do you train them? How do you get into contact with them? Are they chosen? Could you tell me a bit more about those? 

I: Yeah of course. Thank you very much for your question. For example when we are recruiting, I can't say, I cannot say recruiting. When we are you know going to get volunteers of that community, we go to the community. 

Interviewer:  Are you still there? 

I: conscious or are you just promoting the idea of community? The person can read and write. Again, that the person willing to be to be a community volunteers. And again, that, you know, has a reputation of the community, good reputation in the community, willing to work on a voluntary basis because as I said, SRCS is not paying to our volunteers. So the person must be willing to be a volunteer. And again, the community, the community or the committees in that community themselves they were selecting that according to the criteria for that people. When they select and then we do assessment, small assessment, for example, how they can read or write or something like that. And then after that, then we, we let them to, to come together and then we provide training according to the community basis surveillance. When it comes to the [NECBHFA], when it comes to the ECV or ECVHRA and also ECV and also how to report health risks and in terms of coding and all these things. And then we provide that training. We send that, send back to their communities and then they start working with the community. There is a regional supervisor. So looking at the cases coming to the NYSS platform, if there is an error reporting, they immediately communicate to the volunteers and then they support to send them in the right format. So that is the way it works. So when it comes to the selection, the community they select, and then we are working the community leaders, the community committees and also community health committees. They are the one who are supporting when they are working with their communities. If there is a cases in their community, they are the one who is communicating to them. And some of the time the community volunteers, they go to the community by visiting, by doing a house to house visit and then see if there is a cases in the community. So their friends, community themselves, community leaders, they are the one who give us this information. And then we tell them if there is a health risk in that community to SRCS, its role is to come to the community to support them, to control that coming up disease in the community. So in that way, there is a good collaboration. The other thing I can mention is that, SRCS has a good reputation and image at community levels. So it is one of the most trusted organization in the country. So there is a strong relation at community level. So that it helps us also to do this program as community level. 

Interviewer:  Well, thank you. Great that you have so much trust in communities. Coming away from the NYSS and looking at the new project or water monitoring, could you possibly explain me a bit more about the local context at the moment? How is water managed and how does the community live on water and how do they do that? How do they manage their water currently? 

I: Okay, thank you very much. And there is a recent years, or I can mention for the last five years, there was a recurrent drought which is happening in the country that is badly affected the communities in Somaliland. So always there is a water shortage, but in terms of, you know, and affording waterborne diseases, for example. We train community volunteers how to deal with that. For example, WASH component, or hygiene and health promotion activities they do at community level. So what they do is that they teach mothers, they teach the community how to prevent water contaminations and how to prevent water to be contaminated. Because there is a scarcity of water sometimes, one of the important thing is hand washing, what they do. And the people sometimes they will say to you, we don't have enough water. So how we can wash our hands? Because they will say, we don't have any enough water. But again, they give that information to the community so that they can be safe about waterborne diseases. So mainly what they do is to, for example, purify water treatment, like using apple tabs so that they can use, for example, one [tear cam], one tablet for that. And again, another method they use is water poly. So they boil water. If there is no, what is called an apple tab or something like that. But again, they teach how water can be contaminated. For example, when you are taking from the source, and again, when you are traveling with the water, and again, when you are storing the water in your home, or when you are using even the water, the process of contamination, they explain for that so that they can afford. Because if the water was clear when they get from the water source, again, during taking that water to home, in that period, the water can be even contaminated. So they do, and then, for example, when it comes to the berkeds, they use buckets. And berkeds, for example, animal and people, together they use the water in the berked. But what they do is that, to avoid contamination of water in the berked, they make trenches that water flows, and then animal can drink water outside the berked. And again, when they are using this water, they also cover this, and they cover the berkeds and all these things. So they do method, which they try their best to keep the water clean and to be safe when the people are using. Thank you.

Interviewer:   Thank you very much for the answer. You talked about berkeds. So the goal, which also I1 proposed of this project, was early warning for water shortages. So if there is no, for example, no water anymore, we need to respond locally. What do you, could you think of early warnings or anticipatory actions before the last minute, before it's empty? So for example, the berked is half full. We try to, I don't know, raise awareness about it, or we can say we can't really do something or anything else, like, or we did distribute water purification tablets or so. Can you possibly think of anticipatory actions that you could relate to, like the water level in the berked and to which water level would you relate that? 

I: Okay, thank you very much. And, you know, mostly it depends, because of when water shortage has already mentioned, there was a recurrent drought happening in Somaliland. And, you know, there is a, this recurrent drought is, you know, badly affected the communities. And then whatever source of water they can find is what they have. And sometimes they think about what they can drink for themselves instead of having other necessary things. And for example, when the, when there is a, you know, water shortage, it's just this, then they don't think of what kind of water they can get, whether it's bad or something like that. Sometimes there was water trucking and, you know, the SRCS or the other organizations, even the commercial or trade people, they were supporting to the communities who are in need, because this is, and then sometimes, when that water trucking comes, they put the water again to the berkeds. So initially they clean the berked, and again they put that water, but it is kind of water trucking so that they can take a long distance to the, for example, main cities, they travel, and then they put the water into the berked, it's again to fill it so that the community again uses it. This is costly and also has, you know, this, yeah, this time. 

Interviewer:  How does water trucking works? So who's doing the water trucking and how do you know to which community currently you need to go? 

I: Okay, thank you very much. When it comes to the water trucking, it depends. For example, if there is, we do assessment and the community themselves, they talk themselves and then they say there is a lot of, there is a water shortage in their community, so they do, what is called, press release, or they do, yeah, it's called a, yeah. We do press release and then we say there is a water shortage in our community and also as there is FSAU which also sometimes reports the problems existing in the community and SRCS also has a good relation with the community and again that contact we have the community they communicated directly to the SRCS complaining about that there is a water shortage in the community and also the government they have a unit which you know works with the if there is an emergency something like that and supports the community so all these efforts together they decide where these resources include for example and SRCS they organize the resources available and then they go to the community communities who are you know most vulnerable and in that area so mostly it depends and the information coming from the build and then that information is analyzed when it's analyzed we look at where is the most priority area to build is going to be immediately. 

Interviewer:  Thank you very much. Can you think of what else would you like to monitor in regard to water besides the water level? Do you have anything else you would like to or which way you think it is useful to monitor via for example a volunteer who is sending a coded SMS? 

Interviewer:  Yeah actually when it comes to the water related diseases for example diarrhoea they were monitoring about that and they were closely watching that and then they were reporting any case for me this is the community case definition they are the one who's who's reporting on that and then we are also looking at the threshold if the case is reached the threshold immediately we provide in that area they also devil they offer to do hygiene and health promotion activities and also we notify the surrounding communities in that community to also to be notified to know that there is a increase in cases in that community. What they do is to you know tell and you know the community that there is a increase in cases of diarrhoea in the community due to the water related problems and in that case they were providing that information. Again they were reporting to SRCS and SRCS is again they go to the community they visit what they do is they if there is a lot of cases received in that community we shift a mobile team who can do case management at community level and that and you know the case management they do at community level and if there is needed another support it is the time that SRCS immediately you know and with the collaboration and with the Ministry of Health to contain the outbreak in that area so always it depends the scenario. 

Interviewer:  Thank you for your answer and the insights. How do you currently or for example a scenario the volunteer is sending that the berked is empty and that they don't have water or only very few water but all resources are taken and trucking is not possible. How could one prepare for that situation so that for example the SRCS does not lose trust and so how would you communicate and synthesize the community about the possibility that response might not always be possible?

I:  Okay and normally community volunteers they are reporting health risks in the community, health related diseases in the community mainly those who can make outbreaks like good water diarrhea like measles, COVID-19 recently and and then the community so that the main you know and health risk is there looking at that community level so that's one thing. The other thing when it comes to the water for example as you mentioned if the water level of berkeds became you know less or scarce and then what they do is that they provide hygiene and health promotion activities but those who are community leaders are the one who tells the SRCS or other partners or the government that there is a water shortage but sometimes they buy themselves they collect the money between them then they buy water and then they use water trucking for themselves and then for example two or three families they go together they bring their money together and then they buy one water tank to take you know water to their community so that's what initial phase when they do but when you know for example the people they are depending on their livestock and then if there is a drought the livestock become weak or die and then when they see that they are not afford to buy this water or trucking that water this is the time they talk to the other NGOs or the government and say we need support when it comes to me and this only not only and when it comes to the water trucking not only for the government and not only for the NGOs and even the normal people they participate there is a you know people who are good willers and then try to get money from the people who are you know those who have something and then they gather that money and then they buy that money for water and then they distribute according to the need and how they are looking at where the magnitude of the they are looking at where the magnitude of the problem exists and then they refer this water to the to the community so in that case it has different levels so one they can manage themselves and the other level that they can request the NGOs and all these things and the last stage where everybody in the community whether in the urban or rural areas is participating to support each other and one of the good things i can mention is that you know the Somalis they support a lot each other when it comes to the disasters or something like that. 

Interviewer:  Great, thank you for your answer. Good to hear. I think one last question in regard to water quality. Do you know of a way locally how people can access or assess their water quality? So for example, the volunteer can see that the water quality in the berked is not good and now can ask for water purification tablets. Do you know of a way to monitor that locally?

I:  Thank you very much for your question. When it comes to the volunteers, for example, water may be clear in colour, but when we are using it, it is contaminated. So we cannot decide by the colour. Actually, what they do is that is prevention, to do early prevention instead of waiting when the colour of the water changes or there is a remnant in the water or something like that can be clearly seen. What they do is that early enough when there is a water shortage, they tell the people if there is a water shortage, there is a lot of related waterborne diseases, mainly diarrhea is one of it. So that they provide hygiene and health promotion. How, you know, and SRCS, we distribute amount of aqua tablets per month or SRCS sink to the volunteers so that they can manage at community level if there is a case. So when it comes to the aqua tablets, they provide throughout the year so that the household level can be used because the end user is the household when water comes to their community, whether it's trucking or where it's berkeds or where it's surface water whatsoever. So, you know, early, what we train is for the community, volunteers is not to wait until the people become fall sick, but in a professional mechanism. So they do all these things. They do awareness raising, hygiene and health promotion sessions by doing, for example, group sessions by visiting house to house, visiting to meeting and all these things. And we're talking about that we thought that at that time is related, you know, what's going on in the community.

Interviewer:   Thank you for your answer. I only have two more questions, if you're okay with that. 

I: I'm okay, no worries. 

Interviewer:  Looking at this water monitoring approach. What are your thoughts about that? What would you wish for? How you can do you have any any more, you would like to add to that? 

I: When it comes to the water monitoring. Okay. You know, in our context, water monitoring, when it comes to the urban areas, yes, water, there is an agency who is responsible for water supply. They do what's called a den, a chlorination of the water, and then they have the one who's responsible. So we don't have any issue with that. But when it comes to the rural areas, and nomad ares, is the way we have the problem. And, for example, mainly 70, around 70\% of our community, they live in the rural areas. And then there is the, there, you know, there, where there's a lot of, you know, water trucking and then berkeds use and all these things happening. So in that case, as a SRCS, what we do is to provide any necessary support at the community level. When it comes to the monitoring of one of one of that. For example, when we do rehabilitation of berkeds, we also train the community themselves is the proper use of the berkeds, and then the safety of the water. And again, how you know to monitor that the water is contaminated. For example, one of the things that contaminated water is, is the, you know, when they are using the, the water themselves because they were when they are using or taking water from the berked is that that is the time that they can contaminate. So we provide a training, the community, and also we give them was called the ownership of the of the berkeds if the SRCS rehabilitated or build a berked for for them. It is not for SRCS for the community so the community think take the responsibility to monitor and, and, you know, and have the ownership of the berkeds and all these things.

Interviewer:   Thank you. I mean, one, one additional question, and because how big our berkeds, usually, how long does water last. 

I: You know, it depends. The community who are using it. Sometimes it can last within three months, sometimes it can last one month. Sometimes it can last for half a year. 

Interviewer:  Okay. Yeah, well thank you. Yeah, I was just wondering if it's just like a week or a month. So that gives me some more information. For the last question, would you like to add anything else would you like me to know something that I should not forget, or which, on what should I focus. What do you think, do you have anything more you would like to add?

I: Okay, thank you very much. What I would like to add is that, for example, when it comes to the NYSS and NYSS platform. We have the system or the NYSS platform is very effective and very supportive. And we can identify immediately if there is a health risk in the community. Also, we have sometimes a challenge about the SMSeagle. Because the SMSeagle is a device that, for example, captured all these messages coming from the community and also uploaded to the NYSS platform, which is cloud based. In that case, sometimes there was a problem about the function of the SMSeagle. So, several times we have encountered that it failed. And then we have got a gap to get these reports. So that gap sometimes can carry it not to identify early enough if there is a, you know, health risk coming from the community. So that's one of the issue I would like to highlight. The other thing is okay. Thank you very much. Really appreciate it.

Interviewer:   Thank you so much for your time, Mr. Balidi. It helped me a great, great lot to understand. It's not always easy to understand things from so far away, but I try to do my best and I'm very thankful for your help. 

I: Okay. Thank you very much. And anytime you have a question, I will be available for you to support your thesis.

Interviewer:  Thank you. Oh, it's just not only for my thesis. I want to do good work so you can continue helping people and do your job. I would like to support that. That's all. 

I: Okay. Thank you. 

Interviewer:  Thank you very much for your time. 

I: Okay. Bye. Bye. 

Interviewer:  Bye.


\end{document}  
