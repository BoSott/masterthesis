% Appendix B

\chapter{Appendix B: Questionnaires and Interview Guidelines} % Main appendix title

\label{AppendixB} % For referencing this appendix elsewhere, use \ref{AppendixB}

The transcripts of the interviews are listed together with the questions chronologically in the order in which the interviews were conducted.

\section{Questionnaire \& Answers I1}

Interviewer: Bosse Sottmann\newline
Medium: Google Forms\newline
Interviewee: GRC FbF Manager of the SRCS \newline
Date: 31.01.2023

\subsection*{Questionnaire SRCS: Defining Goals and Outcomes}

\textbf{Introduction}\newline
The goal of this project is the design of a practically applicable volunteer sensing-based water source monitoring approach primarily for the water source type of Berkads.
Sub-goals are based on the learnings how water access can be measured specifically for this water source type, what information needs to be known about the source initially, continuously and if the incorporation of local knowledge is useful and possible -- and if yes, how and which specific information are helpful in the context of Anticipatory Actions.\newline
The work will be based on a variety of different sources of information. In addition to this questionnaire and subsequent discussions, also with other stakeholders, best practices and knowledge will be gathered from the literature. Therefore, your input to this questionnaire is critical in multiple ways. Your and the SRCSs opinions, experiences and needs will be the foundation for all of the following work - ensuring that the resulting design meets your requirements and that it aligns to the constraints of this context. Based on the defined goals and following results you will mention here, best practices and knowledge from other contexts can be transferred and applied to this project.\newline
Methodologically, this project coarsely follows the 7-layer-Model of Collaboration. We start to define goals, sub-goals and the actual results we need to accomplish in order to reach these goals. Further down the road, we will build on top of this by defining sets of actions to accomplish the results, thinking about patterns of collaboration internally and with other stakeholders and further defining specific techniques, tools and scripts. Thus, creating a design, that is adapted to the specific context and at the same time applies current best practices from around the world. Nonetheless, a disclaimer should be made, that this is a project in the context of a Master's thesis, thus a fully-fledged design ready to get launched is out of scope of this work. Yet, it can lay a good foundation for the following work.

\subsection*{General questions}
Please give some brief information about yourself and about your organization.

What is your role in the SRCS?\newline
\textit{Offering technical support to the ongoing Forecast based Financing project}

How long have you worked in your position?\newline
\textit{1 year}

What are your predominant tasks?\newline
\textit{Offering technical support to the ongoing Forecast based Financing project as well coordinating the partnership between SRCS and the German Red Cross}

How many paid employees does the SRCS have in total?\newline
\textit{249}

How many of those work in the sector of risk management and/or Anticipatory Actions?\newline
\textit{30}

How many volunteers does the SRCS have?\newline
\textit{1500}

How are the Volunteers spread across the country? Are there regions where there are essentially no or fewer volunteers?\newline
\textit{The Volunteers are evenly spread across the country. However some regions have inactive volunteers due to less activities there whilst some regions have active volunteers due to the amount of project work being undertaken there
}

\subsection*{Goals of the project}

This section is about the main and sub-goals of this project. A main goal could be the mapping of accessible water sources, whereas a sub-goal of this could be e.g. the training and education of volunteers for this task. Please think creatively, without the restriction of limiting resources and please also think about related fields that can be (indirectly) affected by this.\newline
What are the main goals for the project of mapping and monitoring water sources from the perspective of the SRCS? What do you ultimately want to achieve?\newline
\textit{Location is key! Berkads location data is currently missing and this has resulted in the SRCS not being able to quantify the number of existing berkads per region. The main goal of the project will be ascertain the location of berkads and capturing key info such as berkad ownership status (some berkads are privately owned thus not everyone can access water from them. Important info to be also captured include the total number of people or communities dependant on the berkad as well as the storage capacity of the Berkad. Monitoring of water sources would enable seamless prioritization of regions to deliver water (water trucking). Ultimately, in terms of Anticipatory Actions water sources monitoring would enable triggering action before critical water levels}

Can you think of sub-goals that would go along with each main goal?

Can you think of additional goals from the perspective of the community or volunteer potentially involved in the project?\newline
\textit{From the community/volunteer perspective the main goal would be for them to know the existing water resources within their vicinity, as well as the capacity of these water bodies. The main goal being to ascertain whether these water bodies are able to withstand the demand during drought periods.}

Which of these goals could match well and which might be competing?\newline
\textit{The community and SRCS goals match as both focus on closing the knowledge currently existing regarding berkads numbers per district, community and regional level}

\subsection*{Wanted results}
In order to fulfil the defined goals, it is important to further specify actual outcomes by the project. These may deal with issues of quality, effectiveness, efficiency, and other product related characteristics. Following the example from above, this could be a collection of data about water sources in the project area meeting certain pre-defined quality standards. Another result for the sub-goal of trained volunteers might be a collection of appropriate training materials.
What results must be achieved in order to reach the goals? Please list them.\newline
\textit{i) Location data of the berkads (coordinates),}\newline
\textit{ii)Volunteer orientation on water resources monitoring}\newline
\textit{iii) determining the ownership status of each berkad}\newline
\textit{iv) community sensitization to dispel misconceptions about the mapping and water monitoring exercise}\newline
\textit{v) water levels monitoring}\newline
\textit{vi) triggering action based on water levels}\newline
\textit{vii) Determining the water level trigger}

Which of those results do you consider to be the most critical? Please list them and if possible, explain why.\newline
\textit{Location data as this will enable determine the serving capacity of each berkad i.e the total number of communities dependant on each berkad. One important result is also community sensitisation to dispel misconceptions within the communtiies. The community will need to understand why the SRCS will be monitoring water bodies. Triggering for action is also a key result as the end goal will be to counter water shortages so as to mitigate water shortages.}

Based on your experience - in which chronological order do the results have to be processed? Please order them accordingly and add some explanation if possible.\newline
\textit{i) Volunteer orientation on water resources monitoring}\newline
\textit{ii) community sensitization to dispel misconceptions about the mapping and water monitoring exercise}\newline
\textit{iii) Location data of the berkads (coordinates)}\newline
\textit{iv) determining the ownership status of each berkad}\newline
\textit{v) Determining the water level trigger}\newline
\textit{vi) water levels monitoring}\newline
\textit{vii) triggering action based on water levels}

Which of the results specifically required in the design phase do you think are the most critical?\newline
\textit{Determining the water levels to trigger action.}

Which results would specifically be required to address Anticipatory Actions?\newline
\textit{i) Determining the water level to trigger action}\newline
\textit{ii) water levels monitoring}\newline
\textit{iii) triggering action based on water levels}

\subsection*{Stakeholders}
One of the key success factors for a successful design and implementation of a crowdsensing project mentioned in the literature is the early inclusion of all involved stakeholders. This section is about them.
What stakeholders are involved in the mapping and monitoring of water? Who is actively involved, passively affected or just indirectly concerned (please add this information respectively)?\newline
\textit{i)SRCS Volunteers Actively involved}\newline
\textit{ii) Communities Actively involved}\newline
\textit{iii) Berkad owners Actively Involved}\newline
\textit{iv) Community elders Actively involved}\newline
\textit{v) Other NGOs Indirectly concerned}\newline
\textit{vi) The Government Ministry of Water resources}

Which stakeholders, other organizations, communities, groups, or individuals could additionally contribute to the project in terms of knowledge, resources, or other kinds of qualities? Think creatively, 'around the corner' and gladly draw from your experience in other projects.\newline
\textit{The Ministry of Water resources , NADFOR, Other NGOs because they have also constructed some berkads in some communities, FAOSWALIM, the Local government political leadership i.e the Regional governor etc,}

Of all these stakeholders, who are the most important ones - and why? What should we know about them that might be critical for the success of this project?\newline
\textit{The Ministry of Water resources because i believe have a database on existing drilled boreholes in Somaliland (although the lack berkad data), The ministry of water also has the technical expertise in water resources monitoring. NADFOR because they have a similar ongoing programme on community level monitoring of livestock body condition, market prices as well as weather variables. Other NGOs because they have also constructed some berkads in some communities.}

\subsection*{Resource availability and positive/negative constraints}
This section is concerned with the context and environment of the project. To be applicable to the actual context, the design requires negative and positive constraints. In contrast to goals, constraints define stricter limits that need to be respected in order to improve the chances of success. For example, a constraint can define what is not possible (negative constraint), as well as what essential functions need to be met (positive constraint) and what would simply be an added value.

Please list all negative constraints you can think of regarding this project. These can be the classic areas of human resources, knowledge and financial capacities as well as softer social requirements and constraints.\newline
\textit{The project might create huge expectations from the communities as there is the ongoing drought. Whenever there is monitoring of resources communities believe this should be followed up by instant aid. Private berkad owners might not be willing to contribute to the project. They might bar Volunteers from accessing their berkads thus creating tension between community volunteers and berkad owners.}

Now, please list all positive constraints you can think of regarding this project.

Are there any other requirements and/or restrictions that we need to consider when designing this project right from the beginning? These could include, for example, content, format, time management or cultural specifics.\newline
\textit{i)There is an ongoing drought and thus the SRCS staff and volunteers might be over stretched in drought response activities.}\newline
\textit{ii) Community elders should be engaged before the start of the mapping and monitoring as they will help dispel misconceptions about the project}\newline
\textit{iii) the ministry of water resources should be in the loop during the entire project duration}

\subsection*{Final remarks}
Here you can write anything that you find important but was not addressed above. You can give feedback to the questions, raise concerns about some other issues or highlight certain aspects or perspectives.\newline
Final remarks \newline
\textit{All the best in your studies!}


\section{Questions \& Transcript I2}
Interviewer: Bosse Sottmann\newline
Medium: Zoom\newline
Interviewee (I): NRC CBS Somalia Manager \newline
Date: 02.02.2023

Interviewer: Mal gucken ob ich das alles beantworten kann. Ansonsten weiß ich vielleicht jemanden, der es k{\"o}nnte.

I: Ich kann jetzt ja mal vorstellen was ich so im Kopf habe, was so die gr{\"o}ßeren Fragen sind. Und zwar einmal - was auch vielleicht auch Ihr Background ist. Wie das NRC auch involviert war {\"u}berhaupt in den ganzen Prozess, wie es da so rum ging. Wie das angefangen hat, also die Zusammenarbeit mit dem SRCS. Was damals die Probleme, als auch das Goal war, also nicht die Probleme in der Umsetzung sondern was war eigentlich die Fragestellung, was war eigentlich das Ziel dessen. Warum genau Crowdsensing, warum haben Sie genau die Methode gew{\"a}hlt und keine andere? Und welche Alternativen gab es vielleicht auch? Das w{\"a}re so eine gr{\"o}ßere Frage. Eimal vielleicht einen {\"u}berblick zu geben w{\"a}re f{\"u}r mich sehr hilfreich {\"u}ber die Designphase - also wie haben Sie das so organisiert auch mit dem SRCS und welche Stakeholder waren dabei? Wie wurde mit den Volunteers umgegangen, wie wurde das gehandhabt? Wie kam die Communities mit ins Spiel? Wie haben Sie das zum beispiel auch mit den Elders in der Community kommuniziert? Wie lief das so? Ich sage nur, das sind Themen, die mich gerade interessieren. Ich weiß, in einer halben Stunde kriegen wir die nicht alle durch. Wie ginge es dann vielleicht auch {\"u}ber in die Implementation und in die Operational Phase von diesem gesamten Projekt. Gab es bisher schon eine Evaluation? Wie l{\"a}uft das so nebenbei? Und wie k{\"o}nnte das vielleicht denn in Zukunft aussehen? Und entweder so am Ende oder eben auch w{\"a}hrenddessen - wie k{\"o}nnte man das jetzt gut auf das kommende Projekt umlegen? - nat{\"u}rlich potentiell kommendes Projekt. Das ist nat{\"u}rlich nicht in Stein gemeißelt. Aber was k{\"o}nnte man da als Key Take Aways mitnehmen? Wo man jetzt von Anfang an achtet, dann  haben wir schon viel gewonnen

Interviewer: Ok. Also mal gucken was ich behalten habe. Also vielleicht noch kurz : Ich habe beim Norwegischen Roten Kreuz erst vor zweieinhalb Jahren angefangen. Tats{\"a}chlich - und war in der Anfangsphase von der Entwicklung der Plattform um die es ja, glaube ich ja, vorrangig geht, gar nicht dabei. Ich habe ein bischen Brackground Informationen wo es her kommt oder ich weiß warum sie entwickelt wurde. Wie ungef{\"a}hr sie entwickelt wurde, aber so die spezifischen Sachen weiß ich nicht, ob ich so viel dazu beitragen kann. Also die NYSS Plattform so wie sie jetzt ist, die entwickelt sich permanent weiter. Als ich angefangen habe war sie viel oder um einiges weniger komplex. Und eines der Ziele dieser Plattform war es auch ein einfaches Data collection tool bereitzustellen. F{\"u}r die National Societies. Es sollte nie, die Idee war nie es weiter zu entwickeln, weiter zu entwickeln, weiter zu entwickeln. Die Idee war, dass wie es bisher gemacht wurde, und IFRC hat bis vor Kurzem Kobo benutzt zum Beispiel und es gab diverse Formen vor wie NYSS jetzt ist. Die waren {\"a}hnlich aber, genau, waren in einem fr{\"u}heren Stadium und auf Basis dessen wo man gesehen hat was fehlt. Und zum Beispiel mit SRCS I3 sehr viel zusammen gesessen wurde und diskutiert wurde und was w{\"a}re jetzt n{\"o}tig. Also zum Beispiel diese Feedbackmessages kommt, dieses Feature kommt aus Diskussionen mit SRCS. Also es gab, jetzt weiß ich gar nicht wie viel w{\"a}hrend der ersten Pr{\"a}sentation gesagt habe dazu - also da gab es ja 2017 ein riesigen Choleraausbruch in Somaliland - da ist das quasi geboren zumindest f{\"u}r Somaliland CBS zu machen. Ich habe vorher bei {\"a}rzte ohne Grenzen gearbeitet. Da, genau, ist das jetzt ja nichts Neues. Das machen viele schon. Es entwickelt sich halt immer, immer weiter und jetzt ist es viel popul{\"a}rer auch durch Covid deswegen hat das Ganze auch viel mehr, ja Popularit{\"a}t bekommen. Also die Plattform, wie sie jetzt so ist, ist 2019 als das war, ne, als ich angefangen habe ging das erst los, kurz vorher. 2020 das erste Mal im M{\"a}rz {\"u}berhaupt so in der Form benutzt worden. Und hinsichtlich der Entwicklung wurde zusammen diskutiert mit IFRC, den National Societies speziell SRCS, welche Bed{\"u}rfnisse sie haben, was macht Sinn, was nicht. Und dann gab es so Codeons [Hackathons], in unterschiedlichen Orten. Einer war im Senegal, einer war in Norwegen, einer war in Budapest. Wo quasi Volunteers, also Softwareentwickler diese Plattform mitentwickelt haben. Jetzt ist es so, dass wir zwei Softwareentwickler in Oslo haben, die speziell, ja, daran arbeiten. Es gab letztens noch Mal eine {\"a}hnliche Veranstaltung, aber so ein großes Projekt jetzt nicht mehr - genau. Und die Idee, von dieser Plattform, ist nicht, das habe ich glaube ich beim letzten Mal schon gesagt. Das Ziel dieser Plattform ist Early Warning. Nicht 'wir sammeln jetzt Daten zu allem m{\"o}glichen'. Also wir haben immer wieder Diskussionen, wo wir andere Community health Aktivit{\"a}ten haben und die w{\"u}rden gerne NYSS benutzen um Daten zu sammeln. Zum Beispiel SGBV, die hatten gefragt, ob sie nicht mit NYSS die Healthcare Worker informieren k{\"o}nnten ob da nicht in der Community Opfer von SGBV [Sexual and Gender-Based Violence] oder [...] von SGBV war. Aber das machen wir nicht. Also ne wir geben ja zum Beispiel wir - Warum es auch nicht f{\"u}r andere case management Sachen benutzt werden kann weil keine Individuellen Daten zum Beispiel in dieser Plattform sein sollen. Daf{\"u}r ist sie einfach nicht gemacht. Die Communities, also hinsichtlich Erfahrung mit den Communities. Also gut ist nicht nur in Somaliland der Fall, aber ich weiß, dass es da am Anfang auch Problem war, dass jemand von der Community 'ne Report {\"u}ber die, {\"u}ber jemanden {\"u}ber ein Community member schickt. SRCS investiert sehr sehr viel in Kommunikation, in Feedback vice versa mit der Community. Die treffen die regelm{\"a}ßig, h{\"o}ren sich an was die, wie die das alles so finden, welche Bedarfe sie haben. Wir haben auch eine große Evaluierung gemacht f{\"u}r CBS, nicht NYSS, wir haben auch eine Evaluierung zu NYSS an sich gemacht, die wird hoffentlich bald ver{\"o}ffentlicht. Wo wir auch geguckt haben welche anderen Instrumente, oder welche andere Reporting Tools gibt es von anderen organisationen. Wo sind wir besser oder anders. Was sind die Vorteile und Nachteile weil die Idee - also das Problem, was ich oft habe ist - ich geh in ein Land und wir wollen CBS machen und dann ist da aber schon ein anderes, 'ne {\"a}hnliche Software oder reporting Tool. Und dann muss ich wissen, welche sind das? Wie funkionieren die? Macht es Sinn trotzdem noch advocacy f{\"u}r NYSS zu tun. Genau, also wir haben beides getan. CBS evaluiert f{\"u}r SRCS und auch die NYSS Plattform mehr global. Aber vielleicht noch eine Geschichte, die immer wieder ein Problem ist, wenn wir die Plattform einf{\"u}hren wollen ist die Akzeptanz und das Verst{\"a}ndnis von MoH, also Ministry of Health. Weil die finden das oft nicht gut, dass wir, weil es kommen ja alle m{\"o}glichen Organisationen st{\"a}ndig mit ihren eigenen Instrumenten, Methoden und was weiß ich und das finden die oft nicht so gut plus (ahhh), ich weiß auch nicht warum, also jetzt machen wir es nicht mehr so, h{\"a}ngen es nicht mehr so hoch auf, aber der Server von NYSS ist ja nicht im Land, ne? Der ist in Irland f{\"u}r Data Protection Reasons und weil es auch einfacher ist f{\"u}r die Softwareentwickler, ja sich darum zu k{\"u}mmern. Aber das gef{\"a}llt dem Ministries of Health nicht. Gerade in L{\"a}ndern, die ein bischen [lach], o.k. L{\"a}nder, die paranoid sind kann man so nicht sagen, aber es gibt L{\"a}nder, die Panik haben, dass Dinge publik werden, die nicht publik werden sollen. Und wenn man, es gibt in vielen L{\"a}ndern - zum Beispiel erkl{\"a}rt man ungern einen Choleraausbruch, richtig? Die wollen also die Daten in ihren eigenen H{\"a}nden haben. Das niemand wie unser HQ in Oslo da Zugriff drauf hat UND SIE NICHT. Also die haben ja Zugriff auf die Plattform, aber nur zu einem Teil. Die Daten, die Ownership von den Daten ist aber mit der National Society und das finden die oft nicth so gut plus das N{\"a}chste, nicht Problem, und zum Teil haben wir das jetzt auch gel{\"o}st. Man hat Daten von den Gesundheitseinrichtungen, so Disease Sruveillance und dann haben wir die Community Based Surveillance [CBS], die unterschiedlich sind. Aber die Idee ist, sie zusammen zu bringen und da haben wir jetzt zumindest, ich weiß nicht ob dir [...] ein Begriff ist, aber das ist ja ein Datacollection Tool, was in vielen gerade afrikanischen L{\"a}ndern beim Ministry of Health genutzt wird. Um auf der gesundheitseinrichtungsebene Daten von Patienten zu sammeln um auch Trends in Erkrankungen und sowas  zu analysieren. Und Case management kann man damit glaube ich auch und jetzt vor Kruzem, ja haben wir es hingekriegt die Daten von der NYSS Plattform dann automatisch dann in dieses DHIS2 [?] District Health Information System reinzuschieben. genau, ja, also das als kurze Zusammenfassung. Jetzt weiß ich nicht wo ich vielleicht noch ein wenig mehr erz{\"a}hlen sollte. 

Interviewer: Ich glaube, dass es auch noch ein paar Fragen gibt, die wert sind, gestellt zu werden, aber ein Punkt w{\"a}re f{\"u}r mich noch, der Stakeholder Ministry of Health, kann jetzt bei Ihnen mit rein, das w{\"a}re ja das Pendant bei uns, das Water Ministry, also das Ministry for Water Resources. Gab es denn sonst in der Community noch oder von den Volunteers oder anderen Stakeholder, mit denen Sie teilweise auch Ziele hatten, die zusammen lagen oder auch Ziele, die sich vielleicht auch ein bisschen entgegengesetzt haben? Also gab es da auch Widerst{\"a}nde oder gab es da noch andere Personen oder Bereiche, Rollen?

I: Von der Community?

Interviewer: Von der Community als auch von subnationaler Ebene oder von regionaler Ebene oder eben auch wirklich vom lokalen Volunteer? 

I: Ja also Minister of Health definitiv, die haben sehr viele Diskussionen gef{\"u}hrt, die f{\"u}hren die auch jetzt noch. Gut in anderen L{\"a}ndern zum Beispiel haben wir Minister of Agriculture noch mit dabei, weil da auch Erkrankungen von Tieren zum Teil mit berichten. Mit den Communities war es am Anfang, ich glaube nicht unbedingt nur mit der Plattform an sich verbunden, sondern halt mit diesem Reporting, dass sie das am Anfang nicht verstanden haben, wo geht das hin, warum informiert ihr Minister of Health {\"u}ber jemanden der hier krank ist. Aber es gab dann halt viele Sessions, mit denen wo man erkl{\"a}rt hat warum, weshalb, wieso. Das Vertrauen in Minister of Health ist nicht so groß, das kam bei der Evaluation auch raus, aber die merken das was passiert durch SRCS, wenn mit CBS an sich. So f{\"a}hrt also wenn das Programm durchs Deutsche Rote Kreuz gestartet wird, also definitiv werden da ein paar Veranstaltungen mit den Community Leaders stattfinden um zu erkl{\"a}ren was wird gemacht, wer macht was, was passiert wenn wir da, und das halt kontinuierlich also wie gesagt, die gehen da einmal im Monat oder einmal im Quartal und sitzen mit den Community Leader zusammen aber f{\"u}r die Details zu dem Thema ist tats{\"a}chlich I3 der Richtige, weil der ist von Anfang an dabei, der ist permanent im Projekt draußen und ist da am besten auch wahrscheinlich kann der da auch gut beraten wer am besten, mit wem am besten f{\"u}r das Thema zu sprechen ist. 

Interviewer: Ja, I1 hatte das schon einiges angesprochen, aber das ist mit Sicherheit nochmal sehr sehr gut, die stehen bestimmt auch mit einem anderen in Kontakt 

I: Ja, also I3 die, genau, er hat mir erz{\"a}hlt, dass er mit I3 gesprochen hatte und dann I3 das an mich weitergeleitet hatte, obwohl er also gerade f{\"u}r solche Fragen definitiv der richtige Ansprechpartner ist.

Interviewer: Und was gibt es denn noch f{\"u}r andere Methoden, die noch genutzt werden neben diesem Crowdsensing und dem Volunteer-Sensing in anderen Regionen? 

I: Kobo, oder einfach in Kobo haben sie zum Beispiel in Uganda genutzt oder in Bukina Faso haben wir jetzt einfach Excel-Sheets also die Volunteers statt von diesen Plattformen eine SMS zu schicken, schicken die gleiche SMS an ihren Supervisor und der tr{\"a}gt das dann in Excel ein, weil das Ministry of Health wollte nicht, dass wir nicht, dass wir NYSS benutzen. Und dann waren wir es halt ein bisschen umst{\"a}ndlicher, aber am Ende eigentlich das Gleiche Aber dieses, die wollten halt keinen, die haben eine Organisation, die ein Community-Health-Programm einf{\"u}hrt, ich glaube die Zahlen halt ans Ministry of Health. Das tun wir nicht. Und dann haben sie sich f{\"u}r dieses Programm entschieden, auch wenn es nicht das Gleiche tut, was nichts macht, aber das verstehen die Leute halt nicht immer. Genau, also wahrscheinlich auch die Arbeit mit den Ministries, vielleicht kann man da, aber da ist vielleicht I3, auch der richtige Ansprechpartner m{\"o}glicherweise macht es auch Sinn, da jemanden vom Minister of Health mit hinzunehmen oder dass man sich austauscht mit dem vom Minister of Health, um die Benefits auch zu zeigen damit man da nicht wieder von vorne anf{\"a}ngt und die die gleiche Skepsis haben, sondern wenn die sehen, das gleiche Tool ist schon vom Minister of Health benutzt und erfolgreich benutzt die wollen das landesweit einf{\"u}hren und wir sind kurz davor, ist es nat{\"u}rlich leichter auch m{\"o}glicherweise die anderen Departments der Regierung zu {\"u}berzeugen 

Interviewer: Ja, noch mal kurz ein St{\"u}ck zur{\"u}ck, [...], wir hatten ja auch schon {\"u}ber das gesprochen, noch mal ein bisschen tickentechnischer, aber doch noch im Management bleiben, dass man das von der Analogenwelt in das Digitale reinbekommt und daf{\"u}r, dass man dann nat{\"u}rlich auch Kategorien braucht, zusammen mit diesem [...]  Data Collection Platform, wie ist da so die Stimmung oder der Gedanke bez{\"u}glich eben auch der Erhebung und dem Monitoring von Water Resources weil wenn ich jetzt sage, Water Availability hat nat{\"u}rlich jetzt okay, wie viel Wasser ist da, aber es gibt nat{\"u}rlich auf der anderen Seite auch welche Qualit{\"a}t hat es wie viel kommt nach, wer hat {\"u}berhaupt Access dazu, es gibt einige Berkads oder sogar mehrere, die dann nat{\"u}rlich auch privat gehandhabt werden und wo nicht jeder Access zu hat. Da m{\"u}sste man tats{\"a}chlich eine ganze Menge Daten durchaus erheben und auch durchaus so gestalten, dass sie flexibel ver{\"a}nderbar sind, weil wenn jetzt zum Beispiel ein Dorf oder ein Hirte kommt mit 200 Tieren, dann ist es nat{\"u}rlich eine hohe Wasserabgabe. Dann w{\"a}re es doch auch nat{\"u}rlich zum Sinne des Forecastings aber auch wieder Datenerhebung, insofern, wie passt das auch dazu? 

I: Also es ist kein Forecasting Tool, es ist ein Early Warning Tool, deswegen habe ich beim letzten Mal auch darauf gepocht und ich hatte es so verstanden, dass es eher Early Warning, Early Action benutzt wird. Die Wahrscheinlichkeit, dass wir das Rote Kreuz sich darauf einl{\"a}sst, das weiterzuentwickeln, zu dem Zweck, sehe ich nicht, das kann ich jetzt schon sagen. Wenn es wirklich eine einfache Geschichte ist, die man einfach machen kann, wo es einfach nur geht, eine SMS zu schicken, wo es darum geht, hier ist kein Wasser mehr, ist das eine andere Geschichte, als wenn es eine große Data Collection, also die werden kein, das weiß ich jetzt schon, das wird nichts. Deswegen hatte ich beim letzten Mal noch gefragt, was wird wirklich wozu und was ist an Daten n{\"o}tig? Das w{\"a}re okay, wenn es dar{\"u}ber hinausgeht. Das glaube ich nicht. 

Interviewer: Genau, deswegen frage ich nach, das Ziel ist ja an sich, einen Trigger zu definieren, dass der dem dortigen somalischen Ruten Kreuz dann die Ermittlung gibt, wir haben langsam kein Wasser mehr und wir brauchen jetzt demn{\"a}chst eine L{\"o}sung. Das ist ja an sich das Ziel, das heißt, wenn wir wissen, okay, das kriegen wir nicht in die Plattform rein, dann kann man ja auch den anderen Schritt machen, okay, dann muss der Volunteer besser geschult werden, weil dann muss der Volunteer das selber {\"u}berblicken k{\"o}nnen. Dann k{\"o}nnen wir das halt nicht in der Plattform errechnen, sondern dann brauchen wir, das w{\"a}re auch die n{\"a}chste Frage, wie kann man auf dieses lokale Wissen vertrauen, wie kann man das gut einbinden vielleicht auch, wir wissen ja auch. 

I: Okay, nochmal zur{\"u}ck zu den vorherigen, also was der Volunteer schicken w{\"u}rde, w{\"a}re ja regelm{\"a}ßige w{\"o}chentliche Updates, ist es voll, halb voll oder leer. Es gibt noch die M{\"o}glichkeit, dass dann jemand, der die Plattform h{\"a}ndelt, und das sind dann ja nicht die Volunteers, wenn man anruft, also was passiert bei dem, was wir machen, ist ja immer noch der Supervisor, der das validiert. Die rufen da an, sind es wirklich die Symptome, die die gerade berichtet haben. Und erst dann gibt der Klick, okay, das ist wirklich ein True-Alert. Und dann hat der Supervisor noch die M{\"o}glichkeit, wenn wir zum Beispiel, die Idee dann ist ja, es geht ans Ministery of Health und dann soll der Supervisor und Volunteer gucken, dass da Investigation Response vom, m{\"o}glichst vom Minister of Health passiert. Und dann k{\"o}nnen die in so einem Event-Blog noch Notizen machen zu diesem Alert. Also, was man tun, was man {\"u}berlegen k{\"o}nnte, wenn jetzt noch irgendwelche extra Informationen zu dieser Information, okay, es ist leer oder voll, wenn es noch irgendwelche anderen Informationen n{\"o}tig w{\"a}ren, k{\"o}nnte der Supervisor am Telefon das checken und immernoch noch eintragen, richtig? Also, ich kann das auch vielleicht noch mal zeigen in der Demonstration.

Interviewer: Ich habe mir die Demonstration angeschaut. Ich habe mich da ein bisschen mehr angelesen. 

I: Okay, gut, also da gibt es diesen Event-Blog, der w{\"a}re eine M{\"o}glichkeit, wenn man noch mehr Informationen zu diesem Alert haben m{\"o}chte. Hinsicht, ob man den Daten trauen kann, also, das kommt jetzt so ein bisschen drauf an, weil die Themen sind ja schon ein bisschen anders. In unserem Fall berichten sie ja von jemandem, der krank ist mit bestimmten Symptomen oder mehrere, die das gleiche haben oder Tiere. In dem Fall w{\"a}re da jetzt kein, okay, der Benefit w{\"a}re, da passiert was, richtig? Die Ministry of Health kommt, macht Vaccination Campaign, Chlorination Activities, bla, bla. In dem Fall jetzt mit dem Wasser, gut, wenn die Volunteers, also, was ein großer success Faktor f{\"u}r dieses Projekt, CBS-Projekt mit SRCS ist, ist Supervision, das Refresher Training, Supervision, dass die Supervisor geben, glaube ich, mindestens einmal im Monat, zumindest in der Vergangenheit. Jetzt werden die, die brauchen das nicht mehr monatlich, weil wir haben jetzt Evaluation gemacht, wir sehen es in den Daten, die wissen, was sie tun. Die kennen ihr Business. Das wird in dem Projekt am Anfang mehr intensiv sein, dass die Supervisor da h{\"a}ufiger hinm{\"u}ssen, checken m{\"u}ssen, haben sie das jetzt wirklich richtig eingesch{\"a}tzt. Und wenn die nach einer Weile sehen, dass die oft falsch, falsche Reports senden, also immer sagt, es ist leer, es ist leer, in der Hoffnung, sie kriegen vielleicht mehr Wasser in dieser Community, dann muss man sich halt was einfallen lassen. Also das ist jetzt was, was mich spontan einf{\"a}llt, was sein k{\"o}nnte, einfach die Hoffnung, es ist zwar voll, aber ich sage jetzt trotzdem, es ist nur halb voll, weil dann kommt jemand und gibt uns mehr Wasser, gerade mit den Droughts der vergangenen Zeit, das k{\"o}nnte sein, aber dann muss man einfach mit den Volunteers arbeiten und denen quasi auch klar machen, was passiert wann und wir checken das. Was man auch machen kann und das ist weniger sensitiv als bei uns, Fotos schicken, wenn die Smartphones haben. Das ist nicht immer der Fall, aber in manchen Projekten werden Fotos von Symptomen, zum Beispiel von der Haut oder so geschickt. In dem Fall jetzt mit diesen Wasserspeichern k{\"o}nnte man einfach, wenn man denn Smartphones verteilen m{\"o}chte, ich bin kein Fan davon, aus unterschiedlichen Gr{\"u}nden, aber wenn das Teil der Geschichte ist, dann kann man die auch fragen, ob sie nicht ein Foto schickt, zusammen mit dem Report, um das nochmal zu validieren, wenn man nicht da vorbeifahren will. Aber in den Trainings zeigt man denen halt, okay, ab wann ist es voll, ab wann ist es halb voll, wie soll es aussehen und dann kann man erst so ein bisschen durch supervision und feedback kontrollieren. Und wenn der Volunteer immer, also irgendwann merken sie auch, das macht keinen Sinn, der kommt keine und bringt mir extra Wasser. Dann hat es so erledigt, das will ich jetzt mal. Aber gut, man weiß nie, was am Ende, wirklich da eine Community ist passiert. 

Interviewer: Ja, wie ist das denn, wenn man da ist, wenn man jetzt da reingeht, haben die dann sofort auch irgendwie die Expectations, dass sie auch Hilfe bekommen? 

I: Ja.

Interviewer: Weil es kann ja auch sein, okay, in dieses Risiko mit Wasser geht ja auch noch mehr rein, also vielleicht kann die Community sich auch selber helfen oder man sagt, okay, die haben halt nur so viel, wie ist da also die Erfahrung, vielleicht bei dem Haupt, von dem Projekt, vielleicht auch von anderen Projekten, wie kann man damit umgehen, von Anfang an im Design-Prozess? 

I: Okay, jetzt war es ein bisschen weg. Okay, aber ich glaube, ich habe es verstanden. Also, was wir, wenn wir mit Ministery of Health zusammenarbeiten oder {\"u}berhaupt uns daf{\"u}r entscheiden, ob wir CBS machen oder nicht, es muss sicher sein, da ist, es passiert was, nachdem dieser Report geschickt wurde. Und das erste, was passiert, ist, dass die Volunteers, die sind ja geschult in First Aid und Health Promotion, die k{\"o}nnen schon mal ein bisschen was tun. Also die Community, es ist nicht so, ach, ich bin krank und nichts passiert. Also die Volunteers tun was. Wir haben die Mobile Clinics von SRCS, die manchmal die sind die antworten, aber das funktioniert inzwischen ganz gut aber mit logistischer Hilfe oder finanzieller Hilfe kommt auch Minister of Healthy. Genau, also was klar sein muss ist, es geht nicht nur darum, einen Report zu senden und da ist ein Bedarf und nichts passiert. Also was klar sein muss von Anfang an, ist, was passiert wann, wann erwarten wir und das muss man mit den Communities auch klar machen, wann, in welcher Situation erwarten wir von euch, dass ihr euch selbst helft. Aber das wird ja sicherlich vorher definiert, wann m{\"u}ssen wir wirklich mit einem Truck kommen und diesen Speicher auff{\"u}llen oder wann sollten die noch warten oder was weiß ich. Also das auf jeden Fall mit denen besprechen und klar machen, sonst stehen die da und am Ende sind es die Volunteers, die die Probleme haben. Und es f{\"a}llt negativ auf SRCS zur{\"u}ck.

Interviewer: Aber das Netzwerk ist groß genug auch von den Supervisoren, dass es nicht komplett automatisiert sein muss. Man kann schon davon ausgehen, da sagen wir, sie melden jetzt es ist halb voll, dann halb leer, dass der Supervisor da anrufen kann und sagen kann, okay, unsere Kapazit{\"a}ten sind an der Grenze.

I: Okay, es bricht zu sehr. Ich habe einen Teil jetzt gar nicht verstanden. Also irgendwas mit Netzwerk von Supervisoren habe ich noch was verstanden. Ist das? Ja, jetzt ist besser. 

Interviewer: Okay, wir versuchen mal n{\"a}her ran. Ja. Ist es m{\"o}glich, dar{\"u}ber zu gehen, dass man sagt, okay, nachher, wenn man sagt, es ist nur noch halb voll, der Supervisor sagt dann, okay, uns fehlen die Kapazit{\"a}ten, euch was zu liefern gerade. Anderen Communities geht es noch schlimmer. Ihr m{\"u}sst besser rationieren. Wie ist die Erfahrung da? Ist das kleinteilig genug, dass das nicht automatisiert sein muss, diese Feedback-Messages?

I: Also ich glaube, das kann man am Telefon besprechen. Ich w{\"u}rde nur vorher den Community sagen, dass es Situationen geben kann aufgrund von A, B, C, D, dass wir eben nicht in der Lage sein werden, das sofort zu f{\"u}llen, dass es l{\"a}nger dauern kann. Und wir euch dann bitten zu rationalisieren. Dann kann man mit denen besprechen, wo sie glauben, dass man dann vielleicht rationalisieren k{\"o}nnte. Vielleicht kann man denen auch versuchen, ein Timeframe zu geben, wie lange man br{\"a}uchte, damit die das selber einsch{\"a}tzen k{\"o}nnen. Aber, ja. Genau. Also falls das der Fall sein sollte, dass der Fall eintritt, das muss man vorher besprechen und das ank{\"u}ndigen, dass es passieren kann, weil dann verliert man zumindest nicht das Vertrauen. Wenn man jetzt schon antizipieren kann, da ist das Risiko, und du weißt es, Contingency Plan, at Community Level, dann zusammen mit denen werden sich vielleicht auch meckern und beklagen, aber dann weiß man zumindest, was passieren kann. Dann ist der Schaden wahrscheinlich geringer f{\"u}r das Image von allen Beteiligten. Ja. 

Interviewer: Okay, wir sind jetzt schon {\"u}ber die halbe Stunde. 

I: Ja, ich weiß nicht, was ich noch sagen kann.

Interviewer: Vielleicht einmal noch eine qualitative Eingliederung, dessen wir dar{\"u}ber auch gesprochen haben. Was waren so die Key Points, die Key Lessons learned, aber auch die Probleme, wo man sagt, da k{\"o}nnten wir jetzt drauf achten. Was sehen Sie als kritischste Punkte an? Oder wo sagen Sie, okay, da sollte man jetzt von Anfang an sehr genau drauf achten? Und welche Herausforderungen sind gekommen? Was k{\"o}nnte der Fokus sein, was sollte im Fokus stehen durch die Erfahrung?

I:  Um so ein Programm einzuf{\"u}hren, nicht NYSS an sich, aber CBS an sich ja?

Interviewer: CBS und genau an sich, aber jetzt vielleicht auch in speziellerer Bezug auf das Water Early Action, Anticipatory Action, Water Monitoring. Nat{\"u}rlich herauskommt aus dem eingef{\"u}hrten CBS. 

I: Also, so ein paar Keys f{\"u}r die Implementierung oder Planung von CBS. Also, keine Ahnung, ich werde es jetzt einfach rein und manche sind vielleicht schon auf dem Schirm und andere nicht. Also, wir machen ja immer, bevor wir CBS tun, ein Assessment richtig? Im Land auf Nationellebene, auf Projektebene, um zu gucken, gibt es da schon was. Weil ich habe das ja schon zweimal gehabt im S{\"u}d-Sudan, in Nigeria, eigentlich gibt es da schon CBS. Dann muss man gucken, welche gap es dann g{\"a}be und wie kann die National Society das f{\"u}llen. Das ist eins. Und dann treffen wir Minister of Health und erkl{\"a}ren, manchmal, also oft treffe ich die vorher schon, und erkl{\"a}re, was wir tun, was wir dieses Assessment tun und man muss sowieso Interviews mit denen f{\"u}hren. Insofern kann man das gut kombinieren, um schon herauszufinden, was eigentlich, also in dem Wasserprogramm w{\"a}re es dann irgendwie diese andere Department. Und dann spiegeln wir die Ergebnisse zur{\"u}ck an Ministery of Health und dann zusammen entscheiden wir halt, was machen wir, so machen wir es, wie machen wir es. Und dann versuchen wir nat{\"u}rlich auch unsere Vorstellungen damit reinfließen zu lassen, wie zum Beispiel die Plattform zu benutzen. Oder wenn wir zum Beispiel sehen, dass bestimmte Health Risks die Ministery of Health m{\"o}chte, wir aber denken, das macht keinen Sinn oder es sind zu viele. Wir machen immer, wir entwickeln die Strategie, das Protokoll f{\"u}r CBS immer zusammen mit Ministery of Health. Die sind mit in den TOTs (Training Of Trainers), dann Volunteers sind von der Community, die Community Leaders, suchen die aus. Hier muss man uns in unserem Fall, und ich glaube, SRCS ist generell eine Strategie, keine Incentives. Das ist tats{\"a}chlich auch nur zum Beispiel nur in Somaliland mehr oder weniger der Fall, dass die ohne Incentives arbeiten, deren Incentive ist quasi Trainings. Da kriegen wir dann immer f{\"u}r den Transport. Genau, also Community muss die aussuchen und I3 kann da auch noch mehr Inputs geben, weil am Anfang oder oft, gerade wenn Incentives gezahlt werden, dann nat{\"u}rlich die Tochter der Sohn oder wie auch immer da ausgesucht wird. Die gehen aber wirklicherweise in drei Monate studieren und dann muss man einen neuen finden. So, also oft sind es Frauen, die Volunteers sind, weil die eben in der Community bleiben und nicht wie die M{\"a}nner, die mal hier mal da sind. Genau, dann, genau, regular supervision, evaluations, talking with communities, Ministery of Health. Wir haben regelm{\"a}ßige Meetings, wir sind in den Meetings mit dem Minister of Health. Ja, ich glaube, das andere habe ich dann auch schon, also nicht, dass man mit den Communities am Anfang, reden muss, wenn man das Programm aufsetzt, aber ich glaube, SRCS, machen das ja, die sind ja keine Anf{\"a}nger. Die machen das auf jeden Fall, also die wissen das einfach. Ich glaube, es ist einfach ein Teil der Kultur das zu tun, bevor man irgendwo reingeht mit irgendwas. 

Interviewer: Gibt es das mit dem internationalen Kontext oder aus dem nationalen Kontext noch andere NGOs oder andere Player, die da irgendwie ihre Finger mit rein mit drin haben wollen oder doch noch ein bisschen querschießen?

I: Querschießen nicht, aber zum Beispiel, also versuchen wir mit der WHO also mit der Weltgesundheitsorganisation, die sind ja in allen L{\"a}ndern, mit denen auch zusammenzuarbeiten, die sind interessiert daran, was wir tun. Es gibt CDC, Center for Disease Control, die gibt es in, aber die geh{\"o}ren meistens zum Ministery of Health. Das Problem mit anderen Organisationen, okay, querschießen tun die nicht, aber also das Problem, was wir zum Beispiel im S{\"u}dsudan hatten, da wo wir hin wollten, da war schon eine andere Organisation, die Community Health gemacht hat und CBS, aber nur zum Teil. Die Communities haben sich aber zum Beispiel beschwert, dass die das eigentlich gar nicht tun oder nicht genug und die Qualit{\"a}t nicht gut ist und die wollten, dass das Rote Kreuz mehr macht. Aber am Ende kann man da, und das Bekloppte war auch das Ministery of Health einfach keine Ahnung hatte, wer was gemacht hat, weil die h{\"a}tten von Anfang an sagen m{\"u}ssen, nee, da haben wir schon jemand. Wir haben ein großes Assessment gemacht und am Ende habe ich mit diesen Organisationen zusammensetzt und dann erz{\"a}hlen die uns, dass die da quasi diesen Plan haben oder bereits angefangen haben, dass {\"a}hnliche Sachen zu implementieren und selbst wenn die Qualit{\"a}t schlecht ist, kann man da nicht einfach das gleiche machen. Das glaube ich aber jetzt nicht, in Somaliland, dass da so viele andere sind, die das gleiche machen. CBS definitiv nicht, da ist niemand anders. Ich weiß nicht, ob MSF da irgendwo ist, aber ja, in Somalia, glaube ich nicht. Ah ne, die sind auch in Somalia. Aber genau, da kann man ja, und normalerweise sollte das das Ministry wissen, wer da was macht und wenn nicht die, dann sp{\"a}testens die Communitys. Genau, also man muss einfach gucken, wenn man das Assessment macht, sind da andere irgendwo, es kann ja, muss ja nicht unbedingt in den Projektlocations sein, woanders, von denen man auch lernen kann. Also es ist eher diese Competition-Geschichte, als, und da muss man halt viele Organisationen, die haben halt das Geld f{\"u}r diese Region bekommen, dann m{\"u}ssen die da hin, ne, und dann wird es aber einfach, ja, ein bisschen bl{\"o}d. Aber generell so, gegen CBS, alle wollen CBS machen. Alle wissen, dass es gut ist. 

I: Ja, also ich finde es ziemlich erstaunlich, weil ich habe jetzt in der wissenschaftlichen Literatur sehr wenig gefunden.

I: Es ist leider nicht viel ver{\"o}ffentlicht und ich versuche seit zwei Jahren, einen Artikel zu schreiben. {\"u}ber CBS an sich.

Interviewer: Also CBS, ja, also es gab jetzt von 2021, meine ich, noch ein Paper, die jetzt gerade auch f{\"u}r Broad sehr daf{\"u}r advokiert haben, mehrere Sachen mal aufzunehmen, was aber so ein bisschen mehr in Data Collection eingeht. Dass man halt sagt, okay, wir regieren jetzt nicht mehr dar{\"u}ber mit Satellitendaten und großen nationalen, {\"u}bernationalen Datens{\"a}tzen, die wir erheben, sondern wir wollen halt vor allem auch ein Impact Assessment machen von Drought. Haben durchaus ein, zwei sehr aufwendige Questionnaires entwickelt, nicht unbedingt komplizierte, aber durchaus aufwendig, die dann auch in diese longitudinal studies mit reinpassen von WHO. Also da gibt es mehr oder minder auch eher einen Call f{\"u}r diese Paper. 

I: Ja, genau, also von Conlfict and Health habe ich schon seit zwei Jahren Call for Papers, um was {\"u}ber CBS zu ver{\"o}ffentlichen. Ich habe diverse Papers. Ich weiß nicht, ob das genau, das ist, also ist mehr health related. Ich kann die, kann die schicken, wo ein bisschen, ja, zumindest was publiziert wurde {\"u}ber CBS, wie es implementiert ist mit den Outcomes. Aber genau, wir haben es. Ich brauche nur Zeit, um ganz genau zu sagen, was wir hier im Lande sind. Ja, das kann ich noch schicken. Genau, noch mal, was halt oft der Fall ist, aber ich glaube, in Somaliland ist das nicht das Problem. Es gibt L{\"a}nder, wo die Regierung einfach sagt, weil oft das Label CBS Surveillance hat eine negative Implikation. Zum Beispiel in Pakistan k{\"o}nnen wir CBS so nicht verwenden. Wir sagen halt, wir berichten von Kranken aus der Community. Weil Surveillance, da gehen die Alarmglocken an, dass man da irgendwo ausspioniert oder so. Also das, aber dadurch, dass Somaliland jetzt zumindest im Gesundheitsministerium inzwischen einfach viele Jahre Erfahrung und gute Erfahrung hat nit CBS, sollte man die nutzen. Ja, um das auszuweiten auf andere Themen. 

Interviewer: Okay, dann h{\"a}tte ich da auch nur noch eine Frage oder vielleicht eine Bitte. Also nach einer Einsch{\"a}tzung von der generellen Impression {\"u}ber das gesamte Projekt, jetzt vielleicht CBS als auch dieses, diesen Ausblick, sag ich mal, auf ein m{\"o}gliches Wasserquellen-Monitoring und vielleicht auch noch eine Frage, gibt es von Ihrer Seite irgendwie W{\"u}nsche oder so, dass die gerne irgendwie so aus der Erfahrung mit reinfließen sollten?

I: Also wir machen ja alles jetzt basierend auf Erfahrung der letzten Jahre und Projekte. Und die fließen dann immer direkt und die sind jetzt auch an die, das was ich gerade erz{\"a}hlt habe, eingeflossen. Also warum ich glaube und versuche, dass n{\"a}chste Woche, wenn ich mit denen spreche, da vielleicht doch zu veranknern kann, es sei denn, genau, die Bedarfe sind gr{\"o}ßer als das, was ich jetzt verstanden habe. Also diese Wassersourcegeschichten, deshalb es w{\"u}rde halt thematisch gut passend, weil es ein Health Risk ist, um auch Tiergesundheit, Menschgesundheit negativ zu beeinflussen und Outbreaks zu, wie sagt man, also quasi die Basis zu bereiten f{\"u}r Outbreaks, ob es jetzt beim Tier oder beim Mensch ist. Insofern w{\"u}rde das thematisch eigentlich, und wir haben ja diesen Unusual Events, wo eigentlich genau oder {\"a}hnliche Sachen ja bereits schon ber{\"u}cksichtigt sind. Das Spezifische hier w{\"a}re quasi, dass sie von einem bestimmten Wasserpunkt oder watersourcepunkt kommen und dann vielleicht noch Unterkategorien hat, aber ansonsten ist es, der ist jetzt nicht neu neu. Insofern finde ich, w{\"u}rde es gut passen, aber man weiß immer nicht, was ich vielleicht {\"u}bersehen habe. Das werde ich jetzt n{\"a}chste Woche herausfinden, aber bislang haben sie mir noch nicht gesagt, dass es total abwegig ist. 

Interviewer: Genau, von meiner Seite ja auch, ich versuche ja gerade diesen Designprozess, dieses Projekt zu designen, und da geht es ja gerade jetzt, vor allem jetzt momentan in der Phase, deswegen hatte ich auch nachgefragt, ob wir uns jetzt schon unterhalten k{\"o}nnen, ganz klar um diese Constraints. Also wenn ich weiß, okay, alles was m{\"o}glich ist, ist... Wir haben Berkad  1, 2, 3, und ich kann sagen, voll, halb leer oder leer, oder vielleicht 5 Phasen, und das ist das, was das System leisten kann, und alles andere m{\"u}sste {\"u}ber das Telefon mit dem Supervisor geregelt werden, dann ist das ja vollkommen in Ordnung. Deswegen frage ich jetzt genau, also ich bin pers{\"o}nlich gar nicht festgefahren, und ich glaube auch hier, jetzt sonst keiner bei uns, dass es jetzt genau so oder so aussehen muss, sondern es soll funktionieren und es soll in den Prozess reinpassen, deswegen ist bei uns, denke ich, eine hohe Offenheit da, und gerade jetzt eben herauszufinden, in was f{\"u}r einem Kontext, in was f{\"u}r einem Rahmen arbeiten wir, was k{\"o}nnen wir machen und wie k{\"o}nnen wir es machen. 

I: Ja, vielleicht noch die Idee, warum wir zum Beispiel diese Codo-Sache haben, oder auch, dass das alles mit einem normalen Telefon ist, was der Unterschied zu vielen anderen Tools ist. Ich glaube, es gibt kaum ein anderes Surveillance-Tool, was mit einem normalen Basic Phone m{\"o}glich ist. Die meisten brauchen ein Smartphone, und das ist in vielen Gegenden, wo wir arbeiten, nicht m{\"o}glich. Erstens kann ich nicht st{\"a}ndig Smartphones verteilen, oft haben wir auch gar kein Netzwerk und so weiter und die Volunteers mit denen wir arbeiten, das Kriterium f{\"u}r die ist nicht gut gebildet zu sein und oft bei den Tools die es gibt wo mehr Daten mehr Informationen n{\"o}tig ist f{\"u}r Case management, zum Beispiel wo die Volunteers dann selber eintragen m{\"u}ssen, dann brauchst du jemand der Englisch spricht oder zumindest schreiben kann und das ist halt auch nicht der Fall. Und durch das Telefonat, wenn man mehr Informationen braucht zu einem bestimmten Report, das kann man einfach so wie bei uns auch der Supervisor einfach erledigen, der Zugang zu zur Plattform hat. Weil dies m{\"o}glicherweise gar nicht, nicht eingeben k{\"o}nnen... 

Interviewer: Ich hatte bei NYSS bei den Codes gesehen, dass man schon sagen kann m{\"a}nnlich, weiblich, unter 5, {\"u}ber 5, sind so zwei bis drei Sachen, also durchaus zum Beispiel was Wasserqualit{\"a}t angeht, je weniger Wasser drin ist desto mehr Schadstoffkonzentration habe ich ja auch durchaus h{\"a}ufig. Kann man also das vielleicht auch noch mitnehmen pro Berkad, dass man sagt pro Berkad, kann man auch noch sagen, der ist voll und die Wasserqualit{\"a}t sieht gut aus, und er ist auch accessible f{\"u}r uns? 

I: Also wir werden es noch nicht gezeigt, es ist auch m{\"o}glich mit NYSS, wenn wir zum Beispiel Outbreaks haben, kann man nicht mehr zu jedem der krank ist einen Report schicken, das macht keinen Sinn, weil wir wissen, da ist ein Outbreak, jetzt geht es mehr darum zu wissen, wie viele pro Tag hast du gefunden und so weiter. Also es gibt auch die M{\"o}glichkeit mit noch mehr Codes mehr Informationen zu vermitteln. Also zum Beispiel haben wir, wenn Cholera-Ausbruch ist, werden bestimmte Volunteers, die m{\"o}glicherweise, also die werden ausgew{\"a}hlt aufgrund wahrscheinlich auch ihres kognitiven, kognitiven Capacity, ja, die dann von diesen Oral Rehydration Coins einmal pro Tag eine l{\"a}nger Code schicken. Der beinhaltet dann, okay, wie viele waren heute da, wie viele waren weiblich, wie viele m{\"a}nnlich, wie viele unter f{\"u}nf, wie viele {\"u}ber f{\"u}nf, wie viele sind da gestorben, wie viele sind von einem anderen Village gekommen, also ich glaube es sind am Ende bis zu sieben Zahlen. Wenn man den kleine Tools, also was wir am Anfang machen, die ins Somaliland, die Volunteers, die wissen das inzwischen, aber die haben auch so kleine Zettelchen, wo die Codes quasi, kann ich auch noch schicken, Codes quasi erkl{\"a}rt sind, was die bedeuten und wie sie sich, also sie kriegen ja Trainings, aber das kann man ja nicht alles behalten, ich auch nicht, ich muss st{\"a}ndig gucken, welcher Code ist jetzt was und dann kann man denen das geben, wenn die auch mehr, also was man {\"u}berlegen k{\"o}nnte in dem Fall ist, genau der erste Code ist vielleicht, die Nummer des Water Source, dann, ob sie voll ist oder nicht, dann kann man sagen, eins, zwei, drei und dann Hasch, keine Ahnung, welche andere Kategorien m{\"o}glich sind, dass man da noch zwei, drei, w{\"u}rde ich zu weit gehen, andere Codes, weil das schr{\"a}nkt dann wieder ein, wen man als Volunteer nehmen kann und dann kann man das auch schreiben, was was ist. 

Interviewer: Genau, also zwei bis drei andere Codes, das war auch so das, was ich mir vorgestellt hatte und was dann ja auch schon viel helfen k{\"o}nnte, weil auch wenn die Wasserqualit{\"a}t, vielleicht viel Wasser da ist, aber sie sagen, die Wasserqualit{\"a}t ist schlecht, auch dann k{\"o}nnte man ja schon eine Early Action draus machen, dass sie sagen, okay, wir bringen irgendwie etwas um, also zum Beispiel Chlor, um eben einen Ausbruch von Krankheit, wegen schlechter Wasserqualit{\"a}t, in dem Fall schon. 

I: Und dem wird dann erkl{\"a}rt, wann das Wasser schlecht ist, wie sie das einsch{\"a}tzen k{\"o}nnen. 

Interviewer: Genau, das ist jetzt gerade ein Gedanke, der l{\"a}uft parallel, aber genau das m{\"u}sste nat{\"u}rlich alles mitlaufen und vielleicht kann man dann auch so ein kleines Zettelchen oder dann, wie muss es riechen, wie muss es schmecken, wie ist es, wenn man sagt, okay, jetzt wird es kritisch. Ich bin kein Experte im Wassermanagement. Ich glaube, das kann man mal... 

I: Ich habe nicht verstanden, dass die Tiere da auch dran trinken oder nicht? 

Interviewer: So wie ich das verstanden habe, gibt es davon ganz, ganz viele unterschiedliche M{\"o}glichkeiten. Also manche von diesen Berkads sind einfach nur L{\"o}cher im Boden, andere sind von NGOs gebaut mit betoniert, andere sind dann auch weiter, dass sie sagen, okay, sie haben sogar noch ein Blechdach dr{\"u}ber, die dann sogar noch weitergehen und sagen, okay, da verdunstet das dann und alles, was verdunstet, l{\"a}uft ab in einen extra Trichter, in so ein extra Gef{\"a}ß, was dann schon dadurch dann eigentlich mehr oder minder sauber ist, weil es erst mal verdunstet und dann abl{\"a}uft. Also es gibt wohl sehr, sehr viele und auch das ist so ein bisschen noch eine Frage, ist der Berkad {\"u}berhaupt funktional? Also das w{\"a}re auch so ein bisschen eine Early Action, wenn wirklich eine Rainy Season kommt, welcher Berkad ist {\"u}berhaupt in der Lage, Wasser aufzunehmen und welcher braucht erst noch Reparaturen? Aber das w{\"a}re nochmal ein bisschen was anderes, das l{\"a}uft so ein bisschen nebenher. Das Thema wird doch im Detail sehr komplex und es wird schon nochmal [...], aber ich glaube, das ist ja bei den meisten so. Das ist nat{\"u}rlich jetzt nichts mehr Neues. 

I: Okay, nur noch ein Kommentar, falls das aus irgendwelchen Gr{\"u}nden nicht sein sollte, wie ich es vorher schon gesagt habe, es gibt ja diverse andere M{\"o}glichkeiten, die gehen. Also das Gute an dies ist all das automatische, richtig? Feedback messages, notifications zum Ministry, notifications zum Supervisor, automatische Maps, Graphs and so forth. Das aber wenn jetzt aus irgendeinem Grund das nicht m{\"o}glich ist, dann kann man immer noch mit normaler SMS, die der Supervisor dann nachverfolgt und den Eintrag in Excel macht, dann seine eigenen automatischen, zum Beispiel das Team in Burkina Faso, die National Society, die haben super Typen, der da ganz tolle automatische Graphs in Excel kreiert, die alle {\"a}hnlich sind wie NYSS, nur dass es eben manuell eingetragen werden muss. 

Interviewer: Ja, sonst ist NYSS aber ja auch Open Source, wenn ich richtig gesehen habe.

I: Ja, genau. Aber also wenn jetzt zum Beispiel, jetzt was ich meine ist, wenn jetzt zum Beispiel IFRC zum Beispiel sagt, aber wir wollen {\"u}ber unusual events nicht hinausgehen f{\"u}r solche Sachen, dann macht es keinen Sinn. Also wenn wir jetzt nicht sagen, weil in dem Fall w{\"u}rde das tats{\"a}chlich bedeuten, wir brauchen neuen Code f{\"u}r diese Geschichte und dann all die anderen Codes, also Arbeit ist da schon und die Frage w{\"a}re dann auch, okay wer zahlt die Arbeitsstunden, macht das Norwegische Rotkreuz bla bla bla, wenn da Interesse ist oder muss es vom Deutschen Rotkreuz getragen werden, die extra Stunden, die f{\"u}r diese Weiterentwicklung n{\"o}tig sind. Genau, wenn das nicht stattfindet, dann macht es m{\"o}glicherweise einfach keinen Sinn, weil das, was die Idee ist, dann einfach damit nicht m{\"o}glich ist, wenn man nur sagen kann, ich habe ein unusual event, ist die Frage, ob man nicht besser, einfach eine SMS schickt zum Supervisor mit mehr Information. Aber das k{\"o}nnen wir gucken, vielleicht weiß ich  Ende n{\"a}chster Woche mehr. 

Interviewer: Gut, dann dr{\"u}cke ich da mal die Daumen. 

I: Ja, genau, ich auch. Gut. Dann kann I1 ja den Link mit I3 herstellen, der ist gerade in Nairobi, um f{\"u}r Sanlisa f{\"u}r Oslo zu k{\"a}mpfen. Genau, dann ist er in Oslo deswegen, aber jetzt w{\"a}hrend er in Nairobi ist, hat er vielleicht auch Zeiten online. 

Interviewer: Gut, vielen Dank. 

I: Alles gerne. Viel Erfolg. 

Interviewer: Danke sch{\"o}n. Falls noch irgendwie Gedanken kommen oder so, oder auch die Sachen gerne {\"u}ber E-Mail. 

I: Die Artikel schicke ich noch. 

Interviewer: Vielen Dank. Und vielleicht h{\"a}ufiger von anderen Tools geredet neben Kobo, falls da noch mal kurz eine ganz kurze informelle Liste, falls da irgendetwas... 

I: Zu Kobo? 

Interviewer: Ne, nicht unbedingt zu Kobo, aber falls es noch mal andere Ideen gibt oder noch mal wie auch immer. 

I: Ja, okay, alles klar. Ja, aber wahrscheinlich w{\"a}re Kobo dann schon eher die bessere L{\"o}sung, falls es mit NYSS nicht klappt. 

Interviewer: Okay, vielen Dank. 

I: Okay, viel Erfolg. 

Interviewer: Danke sch{\"o}n. 

I: Tsch{\"u}ss. 

Interviewer: Tsch{\"u}ss. 

I: Ciao, ciao. 


\section{Questionnaire  \& Answers I1.2}
Interviewer: Bosse Sottmann\newline
Medium: Google Forms\newline
Interviewee: GRC FbF Manager of the SRCS \newline
Date: 28.02.2023

\subsection*{Introduction}
Hello and As-salamu alaykum,\newline
thank you for taking time to give some insights to your experiences!\newline
My name is Bosse Sottmann, and I am currently studying at the Heidelberg University and am enrolled in the Master's programme in Geography. In the context of this programme, I am currently working on my thesis in the HeiGIT team that is involved in the development of the Early Action Protocol.\newline
The overall goal of this project is the development of a mapping and monitoring approach on community level primarily for the water source type of Berkads to ultimately enable triggering action before critical water levels. Sub-goals are based on the learnings which water levels trigger which actions, what information needs to be known about the source initially, continuously and what other information would be helpful in the context of Anticipatory Actions.\newline
The work will be based on a variety of different sources of information. In addition to this questionnaire and others, best practices and knowledge will be gathered from the literature. Therefore, your input to this questionnaire is critical in learning more from the local perspective in order to not only transfer experiences and learnings to the new design but make this applicable to the local circumstances as best as possible. Your and the SRCSs opinions, experiences and needs will be the foundation of the work – ensuring that the resulting design meets your requirements.\newline
The questions are structured in multiple stages and each question will be open-ended. Thus, feel free to add more information where ever you want or think necessary. Additionally, at the end of the questionnaire, further remarks can be made.\newline
Nonetheless, a disclaimer should be made, that this is a project in the context of a Master's thesis, thus a fully-fledged design ready to get launched is out of scope of this work. Yet, it can lay a good foundation for the following work.\newline
All answers will be confidential and you can skip any questions you are not comfortable with.\newline
Thank you very much for your time and energy!


\subsection*{General Questions}
Please give a short introduction to yourself, your role, experiences and work.
\textit{I1. Im the GRC Forecast based Financing delegate based in Hargeisa, Somaliland. I am supporting the SRCS to undertake the Forecast based Financing project. The project aims to develop Anticipatory Actions that will counter forecastable hazards such as drought, flooding, cyclones and epidemics.}

\subsection*{Local context}
Please tell me about the current conditions on site to gain a better understanding of the local circumstances by walking me through the process of how anticipatory actions/response in regard to water availability/shortage currently work on community level.\newline
\textit{Due to the recurrent droughts, the water sector in Somalia/Somaliland has been greatly impacted. This relates to water shortages (quantities) then also reduced water quality. The main response activities that have been adopted to address the current water crisis are berked rehabilitation (SRCS), water trucking (other agencies), distribution of water purification tablets (SRCS), multi purpose cash, awareness campaigns related to hygiene promotion (SRCS). Regarding Anticipatory actions, there has been any actions yet due to the fact that there is no water monitoring and trigger mechanism in place. Assistance/response is based on the initial prioritization of target areas that SRCS conducts. The prioritization is based on assumed vulnerability per community based on Number of IDP camps in the area, number of women headed families, predicted IPC classifications etc. Activities such as berked rehabilitation are done in consultations with the communities and SRCS branches who flag/identify berkeds in need of repairing. Repairing may consists of re roofing/ roofing, and brickwork to strengthen the structure. The berkeds are meant to capture run off water in case of rainfall incidences. In cases where there hasnt been rains for a prolonged time then water trucks are deployed to deliver water to the communities. Cash has been an imoortattn modality to adress the water shortages. In the current prevailing drought, water and food insecurity crisis, water is now being sold by private players. So the cash has come in handy to t least enable the communities to buy fresh water for drinking. Water sources such as dug wells are often contaminated as livestock i,e camels, goats also drink water from those same water bodies as well.}

\subsection*{Anticipatory Actions}
The current recommendations for the development of triggers for drought focussed Anticipatory Actions are that these should be staggered and closely related to the development of the overall situation and local impacts.

Based on your experience, are water levels in berkads good indicators for drought impacts on the community? Could the indicator be enhanced by local knowledge?\newline
\textit{Water levels in berkeds could be a good indicator, however it cannot be a stand alone indicator. This has to be combined by meteorological forecasts and local knowledge as well.}

The ultimate anticipatory action would be the trucking of water to those who need it most. Going further, one could think about other anticipatory actions that could be triggered beforehand such as awareness raising, information dissemination and involving private berkad owners. What do you think about these proposed Anticipatory Actions? Which potentials and challenges do you see? \newline
\textit{Awareness raising and information dissemination should be more on informing the communities on how to improve water quality at local level e,g boiling before drinking. Involving private berked owners is also feasible however their involvement could be limited as they are more concerned about their business models i.e selling of the water and preserving their berkeds than being part of the overall response/Anticipatory action mechanism. Nevertheless there is the potential to work closely with the private berked owners. This can be done through rehabilitation of their privately owned berkeds in return for their involvement in response and anticipatory action activities related to addressing water shortages.}

Do you possibly have other, more specific or different ideas for Anticipatory Actions? \newline
\textit{yes, i) timely distribution of cash to enable communities to buy and stock fresh water}\newline
\textit{ii) timely distribution of water purification tablets}\newline
\textit{iii) timely rehabilitation of other water sources such as boreholes}

\subsection*{Monitoring}
To facilitate monitoring, the water levels need to be categorized.  

Which water level categorisation do you think as useful? (e.g. how many categories?) How detailed does it need to be in order to be useful and how coarse does it need to be to remain monitorable?  
\textit{These water levels are ideal i.e }\newline
Empty (no water at all) \newline
\textit{Critical (1 day of water supply remaining),}\newline
\textit{Low (3 days of water supply remaining),} \newline
\textit{Middle (5 days of water supply remianing)}\newline
\textit{High (full capacity)}

Which water level category should trigger which Anticipatory Action?  
\textit{Low category}\newline

Which parameters should be monitored weekly, monthly or even only annually?  
\textit{Water level (daily monitoring)}\newline
\textit{Berked condition (annually)}\newline
\textit{Number of people accessing the water form the berked (weekly/monthly)}


\subsection*{Water Quality}
Can you think of ways in which water quality could be included in the monitoring process? \newline
\textit{Water quality is difficult to monitor at community level as it is a technical activity. Unless if the SRCS through the branch staff are equipped with water testing equipment as well as training them on the water parameters to be tested.}

Do you know of any solutions that have proven effective for local water quality monitoring by volunteers in your given circumstances?  
\textit{I'm not aware of any.}

\subsection*{Resource limitations}
How do you currently decide which community to help when resources are scarce?\newline
\textit{The SRCS in consultation with the government select target communities based on a pre existing selection /vulnerability criteria based on either number of IDP camps etc}

What are your experiences? - What would be good ways to deal with potential loss of trust and frustrations in the moment and possibly beforehand?  \newline
\textit{Utilise the community based SRCS volunteers to engage communities and sensitise the communoties on the riole the SRCS plays. Also establishing a robust feedback and Complaints mechanism that ensures communities can easily relay their feedback.}

What other challenges do you see in regard to resource limitations?  \newline
\textit{The current crisis is huge and response activities are being overwhelmed by the need. This will lead to commercialization and overpricing of fresh water.}

\subsection*{Berkads}

In the beginning of the mapping and monitoring of Berkads, their location and related key information shall be captured. Determined key information are so far:
the location,\newline
ownership,\newline
total number of people or communities dependant on the berkad,\newline
its water storage capacity and\newline
functioning\newline
Which other social, technical or context related indicators, parameters or features would you add to this list of important information about Berkads in regard to Anticipatory Actions? Which challenges might arise in the capturing or monitoring of these information on site?\newline  
\textit{Other information might included the year it was built, the last time it was rehabilitated etc. However this kind of information might be missing as you need people with community/institutional memory to provide such kind of information. Somalis are highly mobile communities and it will be difficult to get information on past details per particular geographical area.}

Does the community have an idea of how long the water of their water sources will last? How good is this prediction usually?
\textit{Yes they have an idea. These kinds of predictions are good as communities usually have their own control measures to ensure equitable distribution of water e,g how many containers per family etc. The berkeds are usually locked to ensure there is controlled acess to the water stored.}

\subsection*{Water Trucking}

How does the trucking of water currently work? On what information do you act?

Which roles (e.g. funder, executer, manager, etc.) exist and who usually fills those?

What resources (human resources, finances, water, etc.) and how many/much of these are required for one action of water trucking?

How does the availability of water trucking is spread across Somaliland? Are there certain water points that are used for that?

How long is the average response time from getting the information to the filling?

\subsection*{Final remarks}
Would you like to share additional experiences, lessons learnt or other key points?\newline
\textit{It is proving to be a challenge to plan for anticipatory actions in an already prevailing crisis in Somalia/Somaliland. Already the needs are dire and the current SRCSs focus is on response mechanims to adres the already visible impacts of drought.}

Is there anything else you would like to add or any final thoughts you would like to share before we conclude the questionnaire?
\textit{water monitoring is vital as it will inform decision makers on the priority areas to focus on. Community level water monitoring plus weather forecast information will help form robust Anticipatory Action systems as well as informed response mechanism.}


\section{Questions\& Transcript I3}
Interviewer: Bosse Sottmann\newline
Medium: Zoom\newline
Interviewee,I: SRCS CBS Manager \newline
Date: 04.03.2023

All right, maybe I just shortly introduce myself. I don't know how much you know or how much people told you. I'm Bosse. I'm a master's student at the University of Heidelberg and I'm currently writing my math thesis exactly about this topic. So I talked to you. You possibly know Melanie? 

I: Yeah, my name is A. F. H. I3 is the last name. 

Interviewer:  All right. I'm sorry. 

I: No problem. I'm called I3. Dr. I3 always. So no problem for that. I will write down in the chat my full name so that we can... 

Interviewer:   I'm sorry, Mr. I3. All right. Oh, I'm sorry for that. That is on my head. 

I: No problem. It's okay. Always the people who are here call me I3 only. 

Interviewer:  Okay, how do I pronounce that? 

I: Dr. I3. 

Interviewer:  Dr. I3. All right. Cool. So Mr. I3. Yeah, so I'm writing my math thesis right now. And in the context of this, I'm trying to figure out how we can best set up a monitoring and mapping approach for water sources and for berkeds in the region of Somaliland. And I already got some great answers from I1. And I would be super grateful if you could give me some more information. Twofold. Once on the project of NYSS and CBS. As I talked to I2 and she recommended me to talk to you because I'm not a big fan of the project. But I talked to I2 and she recommended me to talk to you because she said, you know, all the things which happened on the ground and you're the expert in that. And if you still have a bit more time, I'd be grateful to talk about some of the monitoring things I still have open questions to. If you're okay with that.

I: Okay. So can I start or you have a question that you would like to ask me question by question and then my answer for that. 

Interviewer:  I think it would be great just to. Of course, you can give a general introduction if you want, if you like. That would be great, actually. Okay. 

I: Thank you very much. As already mentioned, my name is Dr. Abdifatah Hussein I3. So I would like to give you some information about community based surveillance and also NYSS platform. So when it comes to the community based surveillance, we started 2018 in Burao. The capital city of the Todgheer region, we piloted 75 community volunteers. And we piloted, we look at how community based surveillance is applicable in Somaliland. And what brought to our attention to establish community based surveillance at Buroa. This in Buroa, I mean in Todgheer region, there was a cholera outbreak 2017, which badly affected the communities in Todgheer region, and also other regions, but mainly badly affected in Buroa city, where about 700,000 people live in that area. And it came without saying or, you know, the cases was unpredictable and then it was escalating in the community and then they spread out all the community. So the problem came, you know, the people, for example, the Ministry of Health and the other people, they recognized that there is an outbreak and the outbreak at this peak. And then at that stage, SRCS or Somaliland Red Cross Society is, you know, at that time, giving warning and signals to the Ministry of Health and saying there is a cholera or acute water diarrhea, which is starting in these areas. And for many reasons, the Ministry was saying still the cases we have seen is still, you know, the normal cases we are getting from the communities or something like that. So the problem, you know, it reached this peak. And then at that time, SRCS and its sister organizations or BNS, they established, you know, SRCS, they sent the request that they can come to Somaliland to support so that, you know, the cholera can be managed, you know, because when it comes to the capacity of the government and also, and, you know, the magnitude of the disease became, you know, something which is not the government can not manage. In that case, we requested other national societies to come to Somaliland to support. So initially, Canadian Red Cross, they responded and then within 48 hours, they sent an ERU mission, Emergency Response Unit, so that they came here and they were well equipped with their vehicles and other medical logistics and staff and also the equipment which can be managed in the cholera outbreak. So they were having also what's called a tent for, in tent for cholera management, cholera treatment centers. We established that one and we were managing there for that time. And then again, we established what's called Oral Rehydration Points. So Oral Rehydration Points, we hired community volunteers, people who were provided training from the unit and then they went to the community, they are supporting the community because some people, when they have a diarrhea, they are not going to come to the health facility so that they can get the needed support because they were, you know, a bit reluctant to see other people that they have a diarrhea or, you know, have a stigma or something. So they were not happy to do that. But the Oral Rehydration Point, they supported us at the community level. So they were going house to house so that they can give health and health promotion activities in the same way. They were providing ORS, SYNC and also and other like aquatabs so that they can provide the water and something like that. In that case, this supported a lot. And, you know, the cases who are coming to the health facility or the cholera treatment center, they use it because they were getting support at the community level. In that idea, we said, as the SCRS, one of the lessons learned is that, you know, the cholera came to our country without saying. And we think about in the way that we can identify the cases in the community early enough so that we can identify and respond at community level. So we can stop, you know, an outbreak immediately when it has started or to be noticed early enough. So that's the idea. We came up with community-based surveillance. And as I already mentioned, we piloted and in the Todgheer region, we recruited 75 community volunteers and then the pilot became successful. Then at that time, we were focusing on three districts, Aynabo District, Oodweyne District, and Burao District. And then we scaled out to the other districts, like the Buhoodle District, which is in the Todgheer region. And then we again scaled up to the other regions and then we moved to Todgheer region and again Sool region. And last year [2022], we moved to Senaag region while the two rest areas and regions we scaled this year. So almost I can say now community-based surveillance reached all six regions in Somaliland. We are only focusing on the hotspot area where there is an epidemic, a prone disease areas or where we expect the outbreak to happen. We are not covering all the, for example, all the area, all the country. But we are covering the hotspot areas where we think that outbreak may start or happen. And there is a lot of outbreaks which the community volunteers identified and we have done investigation with the collaboration with the Ministry of Health. And then, you know, we still have that outbreak there. I can give you an example about that. For example, the first case of COVID-19 was from one of our community volunteers in the community. And then the Ministry of Health, they have done the investigation. They took the samples, they sent to Nairobi and then the case became positive. That's one thing. Okay. And the other thing I can mention is that they reported this kind of [fever] at community level. And then, you know, that case was solved at community level. We shifted mobile teams we have so that they can manage that cases. In the same way, there was a success, but I can say an outbreak of measles in the country. And then, for example, one community called a community volunteer in that community. He sent two cases of suspected measles. And one day after, he sent three others. And the next day, after one day, he sent three other cases. And so immediately, our community volunteers, I mean, our CBS officers at the regional level verified that it is much in the community case definition. If you may share, let me close the door. Okay, so that the cases, the Ministry of Health, SRCS team and again WHO together, they went into that community, they took samples and then they sent to the national lab for further investigation and the cases, two of the five cases became positive and then we have done mass immunization against the measles. And then at that time, I remember 5,300 children between age to nine months to nine years was immunized. So I think I can say is when it comes to the community based surveillance is one of the things that can easily detect early enough at community level, the health risk in the community and SRCS has mobile teams who can be deployed immediately within hours so that they can do the response. And also I would like to thank to the Norwegian Red Cross who are supporting this program to run since 2018 and then whenever there is an outbreak, we immediately request support and they profile the nearly support. So I think that is the general view of the CBS when it comes to the NYSS. We started and you know, it composes of, there is what I can say, instruments which you need when you are using NYSS. For example, you must have a mobile. So any mobile you can use it. So there is no need to have a smartphone but you are using SMS. So call an SMS. If you are using SMS, you will use at any mobile but the other thing is also there must be a network in that area and again SMS Eagle. SMS Eagle is a device which captures information which sends the community volunteers with a local member and then when it captures that information again, it uploads to the NYSS platform. So initially the NYSS platform, we are supporting the developers, the need at the community level. So whatever we need, we were discussing with them and then they were updating in that way. So it took around one year to build on but initially the NYSS platform, it captured the information but again we were downloading that information in the NYSS platform and then we manually analyzing through Excel format. But finally we reached a stage that the system or a NYSS platform can automatically analyze itself and again it can give you an alert if the [hella space] key reaches the threshold according to the geographical location that reports are coming from so that you can also early, so that you can get a message through your mobile or your email or something like that. So you can also flow up immediately when you get this alert and mobile all these things. So I think that's general view about the NYSS and also the CBS, how we started and all these things. Thank you and again if you have any further questions, please do not hesitate.

Interviewer:   I do. Thank you so much first of all for this very good coverage and introduction. I learned a lot. How do you, what do you think about NYSS? Would you do it again and why did you decide to do this kind of crowd sensing or working with your volunteers? How did that work specifically? Do you know, so how do you recruit volunteers? How do you train them? How do you get into contact with them? Are they chosen? Could you tell me a bit more about those? 

I: Yeah of course. Thank you very much for your question. For example when we are recruiting, I can't say, I cannot say recruiting. When we are you know going to get volunteers of that community, we go to the community. 

Interviewer:  Are you still there? 

I: conscious or are you just promoting the idea of community? The person can read and write. Again, that the person willing to be to be a community volunteers. And again, that, you know, has a reputation of the community, good reputation in the community, willing to work on a voluntary basis because as I said, SRCS is not paying to our volunteers. So the person must be willing to be a volunteer. And again, the community, the community or the committees in that community themselves they were selecting that according to the criteria for that people. When they select and then we do assessment, small assessment, for example, how they can read or write or something like that. And then after that, then we, we let them to, to come together and then we provide training according to the community basis surveillance. When it comes to the [NECBHFA], when it comes to the ECV or ECVHRA and also ECV and also how to report health risks and in terms of coding and all these things. And then we provide that training. We send that, send back to their communities and then they start working with the community. There is a regional supervisor. So looking at the cases coming to the NYSS platform, if there is an error reporting, they immediately communicate to the volunteers and then they support to send them in the right format. So that is the way it works. So when it comes to the selection, the community they select, and then we are working the community leaders, the community committees and also community health committees. They are the one who are supporting when they are working with their communities. If there is a cases in their community, they are the one who is communicating to them. And some of the time the community volunteers, they go to the community by visiting, by doing a house to house visit and then see if there is a cases in the community. So their friends, community themselves, community leaders, they are the one who give us this information. And then we tell them if there is a health risk in that community to SRCS, its role is to come to the community to support them, to control that coming up disease in the community. So in that way, there is a good collaboration. The other thing I can mention is that, SRCS has a good reputation and image at community levels. So it is one of the most trusted organization in the country. So there is a strong relation at community level. So that it helps us also to do this program as community level. 

Interviewer:  Well, thank you. Great that you have so much trust in communities. Coming away from the NYSS and looking at the new project or water monitoring, could you possibly explain me a bit more about the local context at the moment? How is water managed and how does the community live on water and how do they do that? How do they manage their water currently? 

I: Okay, thank you very much. And there is a recent years, or I can mention for the last five years, there was a recurrent drought which is happening in the country that is badly affected the communities in Somaliland. So always there is a water shortage, but in terms of, you know, and affording waterborne diseases, for example. We train community volunteers how to deal with that. For example, WASH component, or hygiene and health promotion activities they do at community level. So what they do is that they teach mothers, they teach the community how to prevent water contaminations and how to prevent water to be contaminated. Because there is a scarcity of water sometimes, one of the important thing is hand washing, what they do. And the people sometimes they will say to you, we don't have enough water. So how we can wash our hands? Because they will say, we don't have any enough water. But again, they give that information to the community so that they can be safe about waterborne diseases. So mainly what they do is to, for example, purify water treatment, like using apple tabs so that they can use, for example, one [tear cam], one tablet for that. And again, another method they use is water poly. So they boil water. If there is no, what is called an apple tab or something like that. But again, they teach how water can be contaminated. For example, when you are taking from the source, and again, when you are traveling with the water, and again, when you are storing the water in your home, or when you are using even the water, the process of contamination, they explain for that so that they can afford. Because if the water was clear when they get from the water source, again, during taking that water to home, in that period, the water can be even contaminated. So they do, and then, for example, when it comes to the berkeds, they use buckets. And berkeds, for example, animal and people, together they use the water in the berked. But what they do is that, to avoid contamination of water in the berked, they make trenches that water flows, and then animal can drink water outside the berked. And again, when they are using this water, they also cover this, and they cover the berkeds and all these things. So they do method, which they try their best to keep the water clean and to be safe when the people are using. Thank you.

Interviewer:   Thank you very much for the answer. You talked about berkeds. So the goal, which also I1 proposed of this project, was early warning for water shortages. So if there is no, for example, no water anymore, we need to respond locally. What do you, could you think of early warnings or anticipatory actions before the last minute, before it's empty? So for example, the berked is half full. We try to, I don't know, raise awareness about it, or we can say we can't really do something or anything else, like, or we did distribute water purification tablets or so. Can you possibly think of anticipatory actions that you could relate to, like the water level in the berked and to which water level would you relate that? 

I: Okay, thank you very much. And, you know, mostly it depends, because of when water shortage has already mentioned, there was a recurrent drought happening in Somaliland. And, you know, there is a, this recurrent drought is, you know, badly affected the communities. And then whatever source of water they can find is what they have. And sometimes they think about what they can drink for themselves instead of having other necessary things. And for example, when the, when there is a, you know, water shortage, it's just this, then they don't think of what kind of water they can get, whether it's bad or something like that. Sometimes there was water trucking and, you know, the SRCS or the other organizations, even the commercial or trade people, they were supporting to the communities who are in need, because this is, and then sometimes, when that water trucking comes, they put the water again to the berkeds. So initially they clean the berked, and again they put that water, but it is kind of water trucking so that they can take a long distance to the, for example, main cities, they travel, and then they put the water into the berked, it's again to fill it so that the community again uses it. This is costly and also has, you know, this, yeah, this time. 

Interviewer:  How does water trucking works? So who's doing the water trucking and how do you know to which community currently you need to go? 

I: Okay, thank you very much. When it comes to the water trucking, it depends. For example, if there is, we do assessment and the community themselves, they talk themselves and then they say there is a lot of, there is a water shortage in their community, so they do, what is called, press release, or they do, yeah, it's called a, yeah. We do press release and then we say there is a water shortage in our community and also as there is FSAU which also sometimes reports the problems existing in the community and SRCS also has a good relation with the community and again that contact we have the community they communicated directly to the SRCS complaining about that there is a water shortage in the community and also the government they have a unit which you know works with the if there is an emergency something like that and supports the community so all these efforts together they decide where these resources include for example and SRCS they organize the resources available and then they go to the community communities who are you know most vulnerable and in that area so mostly it depends and the information coming from the build and then that information is analyzed when it's analyzed we look at where is the most priority area to build is going to be immediately. 

Interviewer:  Thank you very much. Can you think of what else would you like to monitor in regard to water besides the water level? Do you have anything else you would like to or which way you think it is useful to monitor via for example a volunteer who is sending a coded SMS? 

Interviewer:  Yeah actually when it comes to the water related diseases for example diarrhoea they were monitoring about that and they were closely watching that and then they were reporting any case for me this is the community case definition they are the one who's who's reporting on that and then we are also looking at the threshold if the case is reached the threshold immediately we provide in that area they also devil they offer to do hygiene and health promotion activities and also we notify the surrounding communities in that community to also to be notified to know that there is a increase in cases in that community. What they do is to you know tell and you know the community that there is a increase in cases of diarrhoea in the community due to the water related problems and in that case they were providing that information. Again they were reporting to SRCS and SRCS is again they go to the community they visit what they do is they if there is a lot of cases received in that community we shift a mobile team who can do case management at community level and that and you know the case management they do at community level and if there is needed another support it is the time that SRCS immediately you know and with the collaboration and with the Ministry of Health to contain the outbreak in that area so always it depends the scenario. 

Interviewer:  Thank you for your answer and the insights. How do you currently or for example a scenario the volunteer is sending that the berked is empty and that they don't have water or only very few water but all resources are taken and trucking is not possible. How could one prepare for that situation so that for example the SRCS does not lose trust and so how would you communicate and synthesize the community about the possibility that response might not always be possible?

I:  Okay and normally community volunteers they are reporting health risks in the community, health related diseases in the community mainly those who can make outbreaks like good water diarrhea like measles, COVID-19 recently and and then the community so that the main you know and health risk is there looking at that community level so that's one thing. The other thing when it comes to the water for example as you mentioned if the water level of berkeds became you know less or scarce and then what they do is that they provide hygiene and health promotion activities but those who are community leaders are the one who tells the SRCS or other partners or the government that there is a water shortage but sometimes they buy themselves they collect the money between them then they buy water and then they use water trucking for themselves and then for example two or three families they go together they bring their money together and then they buy one water tank to take you know water to their community so that's what initial phase when they do but when you know for example the people they are depending on their livestock and then if there is a drought the livestock become weak or die and then when they see that they are not afford to buy this water or trucking that water this is the time they talk to the other NGOs or the government and say we need support when it comes to me and this only not only and when it comes to the water trucking not only for the government and not only for the NGOs and even the normal people they participate there is a you know people who are good willers and then try to get money from the people who are you know those who have something and then they gather that money and then they buy that money for water and then they distribute according to the need and how they are looking at where the magnitude of the they are looking at where the magnitude of the problem exists and then they refer this water to the to the community so in that case it has different levels so one they can manage themselves and the other level that they can request the NGOs and all these things and the last stage where everybody in the community whether in the urban or rural areas is participating to support each other and one of the good things i can mention is that you know the Somalis they support a lot each other when it comes to the disasters or something like that. 

Interviewer:  Great, thank you for your answer. Good to hear. I think one last question in regard to water quality. Do you know of a way locally how people can access or assess their water quality? So for example, the volunteer can see that the water quality in the berked is not good and now can ask for water purification tablets. Do you know of a way to monitor that locally?

I:  Thank you very much for your question. When it comes to the volunteers, for example, water may be clear in colour, but when we are using it, it is contaminated. So we cannot decide by the colour. Actually, what they do is that is prevention, to do early prevention instead of waiting when the colour of the water changes or there is a remnant in the water or something like that can be clearly seen. What they do is that early enough when there is a water shortage, they tell the people if there is a water shortage, there is a lot of related waterborne diseases, mainly diarrhea is one of it. So that they provide hygiene and health promotion. How, you know, and SRCS, we distribute amount of aqua tablets per month or SRCS sink to the volunteers so that they can manage at community level if there is a case. So when it comes to the aqua tablets, they provide throughout the year so that the household level can be used because the end user is the household when water comes to their community, whether it's trucking or where it's berkeds or where it's surface water whatsoever. So, you know, early, what we train is for the community, volunteers is not to wait until the people become fall sick, but in a professional mechanism. So they do all these things. They do awareness raising, hygiene and health promotion sessions by doing, for example, group sessions by visiting house to house, visiting to meeting and all these things. And we're talking about that we thought that at that time is related, you know, what's going on in the community.

Interviewer:   Thank you for your answer. I only have two more questions, if you're okay with that. 

I: I'm okay, no worries. 

Interviewer:  Looking at this water monitoring approach. What are your thoughts about that? What would you wish for? How you can do you have any any more, you would like to add to that? 

I: When it comes to the water monitoring. Okay. You know, in our context, water monitoring, when it comes to the urban areas, yes, water, there is an agency who is responsible for water supply. They do what's called a den, a chlorination of the water, and then they have the one who's responsible. So we don't have any issue with that. But when it comes to the rural areas, and nomad ares, is the way we have the problem. And, for example, mainly 70, around 70\% of our community, they live in the rural areas. And then there is the, there, you know, there, where there's a lot of, you know, water trucking and then berkeds use and all these things happening. So in that case, as a SRCS, what we do is to provide any necessary support at the community level. When it comes to the monitoring of one of one of that. For example, when we do rehabilitation of berkeds, we also train the community themselves is the proper use of the berkeds, and then the safety of the water. And again, how you know to monitor that the water is contaminated. For example, one of the things that contaminated water is, is the, you know, when they are using the, the water themselves because they were when they are using or taking water from the berked is that that is the time that they can contaminate. So we provide a training, the community, and also we give them was called the ownership of the of the berkeds if the SRCS rehabilitated or build a berked for for them. It is not for SRCS for the community so the community think take the responsibility to monitor and, and, you know, and have the ownership of the berkeds and all these things.

Interviewer:   Thank you. I mean, one, one additional question, and because how big our berkeds, usually, how long does water last. 

I: You know, it depends. The community who are using it. Sometimes it can last within three months, sometimes it can last one month. Sometimes it can last for half a year. 

Interviewer:  Okay. Yeah, well thank you. Yeah, I was just wondering if it's just like a week or a month. So that gives me some more information. For the last question, would you like to add anything else would you like me to know something that I should not forget, or which, on what should I focus. What do you think, do you have anything more you would like to add?

I: Okay, thank you very much. What I would like to add is that, for example, when it comes to the NYSS and NYSS platform. We have the system or the NYSS platform is very effective and very supportive. And we can identify immediately if there is a health risk in the community. Also, we have sometimes a challenge about the SMSeagle. Because the SMSeagle is a device that, for example, captured all these messages coming from the community and also uploaded to the NYSS platform, which is cloud based. In that case, sometimes there was a problem about the function of the SMSeagle. So, several times we have encountered that it failed. And then we have got a gap to get these reports. So that gap sometimes can carry it not to identify early enough if there is a, you know, health risk coming from the community. So that's one of the issue I would like to highlight. The other thing is okay. Thank you very much. Really appreciate it.

Interviewer:   Thank you so much for your time, Mr. Balidi. It helped me a great, great lot to understand. It's not always easy to understand things from so far away, but I try to do my best and I'm very thankful for your help. 

I: Okay. Thank you very much. And anytime you have a question, I will be available for you to support your thesis.

Interviewer:  Thank you. Oh, it's just not only for my thesis. I want to do good work so you can continue helping people and do your job. I would like to support that. That's all. 

I: Okay. Thank you. 

Interviewer:  Thank you very much for your time. 

I: Okay. Bye. Bye. 

Interviewer:  Bye.
