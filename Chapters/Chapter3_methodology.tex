% Chapter Template

\chapter{Methodology} % Main chapter title

\label{chap3:methodology} % C

% Intro
Based on the philosophical ideas of interpretism and post-positivism, this chapter will present the methodological framework this work utilised. Embedded in an inductive design type of an exploratory, iterative case study, a mixed-method approach with data and document analysis as well as expert interviews was adopted. In case of the interviews, non-probability sampling together with the snowball approach was applied. The transcribed interviews were subsequently coded facilitating an open thematic coding strategy. The final design was guided by the 6-stage-design for \acrlong*{cs} projects in ecological science by \autocite{fraislCitizenScienceEnvironmental2022} and further deepened by a more social concept, the Seven-layer Model of Collaboration by \autocite{briggsSevenLayerModelCollaboration}. Moreover, multiple other guidelines for the creation of a \acrshort*{cs} program were consolidated and their recommendations taken into account.\newline
In the following, each of the above mentioned concepts is presented, reasoned and embedded in the overall context of the research. The aim and context of this research, the attempt to design a roadmap for a \acrshort*{cs} project for participatory and remote water source mapping and monitoring in a resource scarce environment in Somaliland, faced several challenges and constraints. These constraints are also highlighted along with other important methodological strengths and limitations.

\section{Research Design} % is that the correct heading?

% Philosophy (interpretism and post-positivism) -> Research Philosophy
Positivism, historically emerged as a combination of rationalism and empiricism by french philosopher Auguste Comte, highlights the objectivity of knowledge and emphasizes the fundamental need for verification through observations. Positivism was, and often still is, mainly linked to quantitative research methods such as experiments and surveys, highlighting the independent and objective nature of scientific research. In contrast, interpretism (sometimes also called anti-positivism), often equated with qualitative methods such as participant observations and unstructured interviews, argues that objectivity is largely impossible and all knowledge is subjective in nature. Both of these more extreme approaches were increasingly criticised at the end of the 20th century. Emerging concepts like post-positivism, for example, acknowledges biases and discusses the idea that while truth cannot be objectively proven, false claims can be rejected \autocite{pelzResearchMethodsSocial,trochimResearchMethodsKnowledge2001}.\linebreak[1]
This work, on the one hand, relied on a deductive, quantitative analyses and theory testing of conducted programs, projects and literature to synthesise best practices and guidelines. On the other hand, it also applied interpretive, inductive methods like interviews to gain new expert knowledge about the context and realities of the case study to build new theory. Therefore, a combination of both approaches will be reflected in a mixed-method approach which benefits of both, quantitative and qualitative data.

% Research Type
The research type that "allows for in-depth, multi-faceted explorations of complex issues in their real-life settings" \autocite{croweCaseStudyApproach2011} is the \textit{case research} or often just called \texttit{case study}. This definition goes back to \autocite{yinCaseStudyResearch1984} and highlights the core strengths of this approach. The research type of the case study is particularly well suited for exploratory, rather than for descriptive or explanatory research, closely examining circumstances within a specific geographical area and context \autocite{zainalCaseStudyResearch2007}. Here, various quantitative and qualitative forms from both historical and real time data can be investigated and examined directly in its given context \autocite{fitzgeraldCaseStudiesResearch1999}. This allows for a detailed, multi-perspective and specific investigation of that particular topic of interest, which might not be given when examining the parts individually \autocite{pelzResearchMethodsSocial, zainalCaseStudyResearch2007}. Yet, case studies also comes with a number of trade-offs.\linebreak[1]
More extreme critics see the case study method only as a loose 'story', which in the worse case is even connected to the scientist himself \autocite{fitzgeraldCaseStudiesResearch1999}. This criticism refers to the lack of rigour and "very little basis for scientific generalisation" which can lead to low external validity \autocites{yinCaseStudyResearch1984}[5]{zainalCaseStudyResearch2007}. The internal validity often remains weak due to no or poor experimental control complicating causal relationship testing and the multifaceted nature of case studies make it dependent on the researcher and prone to bias and some kind of subjectivity. Besides the internal and external validity, constructed validity and reliability need to be accounted for. Constructed validity refers to "the extent to which a study investigates what it claims to investigate" \autocite[3]{gibbertWhatPassesRigorous2008} so that "the researcher can correctly evaluate the studied concepts" \autocite[277]{ferreiraHowImproveValidity2020}. It can be addressed by establishing a clear chain of evidence and triangulation of perspectives and sources \autocite{gibbertWhatPassesRigorous2008}. Repeatability by other scientists using the same methodology to arrive at the same insights is termed 'reliability'. Repeatability refers therefore to the "absence of random error" and can be enhanced by clear procedures and good documentation \autocite[5]{gibbertWhatPassesRigorous2008}.\linebreak[1]
Furthermore, case studies are frequently criticized for being excessively long, challenging to execute, and requiring significant documentation efforts \autocite{yinCaseStudyResearch1984}. Alternative research types such as experimental research, desk or field surveys and survey research also all have their inherent advantages and limitations. For example, experimental research often deductively examines cause-effect relationships in an isolated context, making it strong in internal validity but low in external validity as the often artificial isolation does not reflect the real world \autocite{pelzResearchMethodsSocial}. Survey research, as another example, has a variety of advantages, such as measuring unobservable data (e.g. peoples preferences, beliefs and values), it is easily scalable and can be carried out independently of time and space. Nonetheless, it is also subject to many biases such as sampling bias, recall bias and non-response bias \autocite{pelzResearchMethodsSocial}.
% justification case study
Since all research types have their advantages and disadvantages, it is primarily a weighing of these strengths and weaknesses that determines the choice of method. The strength of the case study make this research type ideal for a holistic exploration and understanding of novel and under-researched areas within a "complex and dynamic context where it is difficult to isolate variables or where there are multiple, influencing variables" \autocites[2]{fitzgeraldCaseStudiesResearch1999}{zainalCaseStudyResearch2007}. Therefore, a case study is an excellent method not only to test theories but also to develop new theories and frameworks, as it is the aim of this (pilot) study \autocite{pelzResearchMethodsSocial, zainalCaseStudyResearch2007}. \linebreak[1]
Along with the exploratory technique, an iterative technique is adopted, referring to the "visiting and revisiting [of] the data and connecting them with emerging insights, progressively leading to refined focus and understandings" \autocite[77]{srivastavaPracticalIterativeFramework2009}. Thereby, each interview or questionnaire was directly transcribed, coded, analysed and merged with the insights gained up to that point. Along with the conducted literature review verify previous findings and iteratively expand the insights.Together with the theoretical background information and project analyses, newly acquired knowledge could be iteratively integrated, triangulated and used as a basis for further research and interviews. This made it possible to check, deepen and refine the knowledge gained piece by piece.
% lit analysis
The background analysis started with the review of the already established database of the related project and was subsequently extended. Broad concepts were used as a basis to lay a thematically large, yet steadily more specific foundation. For the in-depth analysis of previous \acrshort*{cs} projects and further insights into the case study itself, grey and peer-reviewed literature was consulted. Grey literature, as academic literature on Somalia was found to be generally scarce. The search was based on different combinations of keywords and their synonyms which could be derived from the underlying concepts and their respective specifications. Furthermore, in the course of the work, further literature and project suggestions were received from the project team and from the interviewees. For the selection of CS projects, core areas of interest were formulated and derived from the thematic focus of the fundamental concepts. Therefore, the thematic focus was on community-based participatory environmental or risk monitoring, but always with a focus on water related issues. The geographical location, size or technical facilities had no influence. Subsequently, the projects were tabulated and jointly evaluated manually, since the absolute number of projects finally selected was manageable at 20 without additional software.\linebreak[1]
The analysis of existing data sets of water source point and feature information was considered, but was discarded at a very early stage. The very limited reliability, completeness and actuality of the available data sets had already been reviewed and stated by the project team before the start of this work. The lack of data was the main reason for this work to start with, and a short analysis via QGIS was able to confirm these statements and thus, due to relevance and time constraints, the focus shifted to the design approach.\linebreak[0]
In addition to the thematic focus on water-related issues, the practical implementation on the ground was investigated by analysing the already established \acrshort*{cbs} program of the \acrshort*{srcs}. This was done primarily by interviewing the responsible managers.
% interviews
The selection of interviewees was based on a strategy of targeted expert sampling, i.e. a non-probability method that focused on reaching key informants and conducting expert interviews. This technique was employed as the expertise and experience of the individuals was crucial rather than focusing on broad, generalisable statements \autocite{pelzResearchMethodsSocial}. This method was further extended by adopting a snowball sampling approach which helped to identify further stakeholders and potential candidates.\linebreak[1]
The target persons were primarily people who know the local context and/or are potential stakeholders in a possible implementation of the design in question. The first interview came about through existing contacts of the project in which this work is embedded, and the interviewee was the project leader of the FbF approach in the \acrshort{srcs} (I1). In the further course, the CBS project manager on the Norwegian Red Cross side (I2) and the CBS manager on the Somali side (I3) were also interviewed. Between these two interviews, there was a second interview with the project manager of the \acrshort{srcs}' \acrshort{fbf} team (I1.2). More interviews with representatives of the Ministry of Water Resources, \acrshort{nadfor}, FAOSWALIM, \acrshort{brcis} and the technicians responsible for the \acrshort{nyss} platform were envisaged (TODO: see appendix XYZ for the questions) but could not be conducted. The interviews with the project leader on the \acrshort{srcs} side unfortunately had to be replaced by written questionnaires, as he could not free up time for an interview and only the higher flexibility of the online questionnaire allowed him to contribute at all. For the conversion of the interview guidelines into questionnaires, the recommendations of \autocite{harknessCCSGQuestionnaireDesign2016} were followed. The interviews and questionnaires themselves were semi-structured and mostly consisted of open ended questions to allow for the interviewee to give a free response as opposed to predefined answer options (see appendix TODO: \ref*{ABCD}). Open-ended questions can facilitate more detailed answers, new insights and overall allow for an unlimited response in terms of scope and focus while they also complicate relevant information abstraction and following analysis \autocite{pelzResearchMethodsSocial}.\linebreak[1]
The intelligent verbatim transcription of the interviews was facilitated by the newly developed neural net called Whisper \autocite{openaiIntroducingWhisper2022,openaiWhisper2023} and subsequently checked and corrected in MaxQDA 2022. For the further analysis, the recommendations of \autocite{radikerFocusedAnalysisQualitative2020} were followed. The coding strategy followed an inductive, open thematic manual coding approach. Codes were not strictly predefined, to be able to appropriately incorporate newly gained expertise, but only broadly categorized into the main themes of interest. More dedicated coding approaches based on e.g. the grounded theory or the hermeneutic analysis were not applied, as the given information was the focus of interest and not e.g. the identification of subjective constructs and underlying meaning \autocite{pelzResearchMethodsSocial}. Nonetheless, based on the criticism on positivism, provided information was not taken as unbiased and objective but interpreted in the context of its perspective.
% Limitations
% muss hier noch was hin? oder eher in die Diskussion? wenige Interviews sind ja kein Methodenfehler.. und Biases, validity etc. sind oben angesprochen.


\section{Design Frameworks}\label{sec:design_framework}

The design of the roadmap for participatory community water source monitoring is particularly guided by two mutually complementary design frameworks. The six iterative stages for the design and implementation of a Citizen Science project in environmental and ecological sciences by \autocite{fraislCitizenScienceEnvironmental2022} will lay the conceptual foundation for this work. It covers the entire life cycle of a Citizen Science project in its six phases in an iterative way, starting with problem assessment and finishing with evaluation procedures. Stage three, the "designing the project" stage, will further be enhanced by the \acrfull{slmc} by \autocite{briggsSevenLayerModelCollaboration} as the guidance of the \acrfull{ssf} is relatively scarce for practical application at this stage. The SLMC can help to reduce cognitive overload and give further impulses from a social point of view through its conceptual division into 7 levels in the design of cooperation projects. Furthermore, several other guidelines will briefly be presented as they can provide further valuable insights for the design of a \acrshort*{cs} project.

\subsection{6-Stage-Framework}\label{subsec:ssf}

In their work, \autocite{fraislCitizenScienceEnvironmental2022} pull information from all kinds of \acrshort*{cs} programmes, projects and scientific guidance. While their thematic orientation is on projects in the field of ecology and environmental sciences, the underlying principles which they describe are successfully applied in a wide variety of other thematically differently oriented projects \autocite{fraislCitizenScienceEnvironmental2022}. The developed \acrfull{ssf} concentrates on the participation level (1) Crowdsourcing and (2) Distributed Intelligence. On these levels, citizens are primarily contributors and partly asked to interpret the sensed information. This was outlined in chapter \ref*{sec:cs}. All six stages are interconnected and should all be considered throughout the project to incorporate new information, feedback and lessons learned \autocite{fraislCitizenScienceEnvironmental2022}. An overview of the \textit{citizen science project life cycle} can be seen in figure \ref{TODO: figure stages fraisl}

\missingfigure{ssf design - possibly do it myself as the figure is not as nice here (too much other stuff)}

In stage 1 the overall need and problems are identified and their boundaries defined. This includes the gathering of potential solutions, limitations and the formulation of research questions. Stage 2 closely examines the potential application of \acrshort*{cs} in the identified boundaries. The focus is on the fruitfulness of the involvement of participants to reach the formulated objectives and answer the research questions. This may be related to many project specifications, e.g. temporal and spatial scale, required expertise and intended target groups. As the second major consideration for the reasonable integration of \acrshort*{cs} participants, \autocite[2]{fraislCitizenScienceEnvironmental2022} note, that the project need to benefit the participants by "addressing their needs or fostering new skill and expertise". After problems, needs and applicability are addressed, the objectives and aims of the project need to be defined in detail together with the prospective participants in stage 3. In addition to the main objectives, secondary objectives such as awareness and knowledge building as well as its transfer could also be pursued. Just as concerns of data storage and analysis, privacy and ethics, selection of methods and training strategies as well as communication means and instruments are also parts of this stage. Furthermore, the tasks of the participants need to be defined in detail, also including any benefits and safety considerations. Stage 4 is concerned with the building of the community by identifying participants motivations, education levels and other demographic information as well as issues of acknowledgments, feedback and sustaining participation. Planning of data management in terms of collection, storage, assuring quality, analysis and privacy and security are highlighted in stage 5. Although evaluation is the main theme of stage 6, it is seen more as an ongoing effort that is recommended throughout the project to allow for feedback and improvement at each stage.
This work has focussed primarily on the stages one to three as stage 4, community building is not necessary to that extent since the \acrshort*{srcs} already has a vivid network of active and motivated volunteers. The fifth stage, was partly considered, but could not be mainly worked on due to time constraints and unresolved issues in the design phase and as this project did not leave the design phase, a major evaluation in phase 6 was obsolete. % though I did consider those.. see results --> rewrite this section to account for the changes

\subsection{Seven-Layer Model of Collaboration}

The design pattern of the \acrshort*{slmc} was integrated into the above \acrshort*{ssf} in stage 3 \textit{designing the project} to better handle and structure the high complexity of the roadmap design. The \acrshort*{slmc} was specifically designed to reduce cognitive (over-)load for the design of a complex, interrelated project in a social-technical context. It does so, by separating concerns at design time into seven layers and  corresponding methods and techniques. These, as presented by \autocite{briggsSevenLayerModelCollaboration}, are primarily aimed at the collaboration of groups, but the overall pattern can be preserved when applied to designs in other contexts \autocite{diggelenGroundedDesignDesign2009}. Therefore, the following explanations of the individual layers, their methods and techniques are adapted to this work, while maintaining the general pattern developed by \autocite{briggsSevenLayerModelCollaboration}. The seven, slightyl adjusted layers are, Goals, Products, Activities, Methods, Techniques, Tools and Scripts (see figure \ref*{TODO: slmc image}). 

\missingfigure{slmc all layers figure -> beware of the Methods adjustement}

The layers are ordered hierarchically with Goals being the top-most layer. Changes made in one layer, may need to be accounted for in the lower, but not necessarily in the upper, layers. The \textit{Goal-layer} incorporates all overarching goals and objectives of the project. The \textit{Products-Layer} sum up all tangible or intangible components or outcomes that are necessary to achieve the formulated targets in the \textit{Goal-Layer}. The required activities that yield these products, are grouped in the \textit{Activities-Layer}. These activities formulate what needs to be done to reach the goals and can have sub-products and sub-activities of their own. The fourth level is the most different from the original, as it does not deal with the procedures of cooperation but with the applied methods during the activities. The \textit{Techniques-Layer} specifies the involved techniques and practices and the \textit{Tools-layer} summarises all relevant artifacts or apparatus used. The final procedures are described in detail and defined in the bottommost layer, the \textit{Script-Layer}. Further concerns, interactions and justifications between and for each layer are extensively described by \autocite{briggsSevenLayerModelCollaboration} and while there is a lot of value in these remarks, the thematic adaptations and focus of this work make further exploration in this context obsolete. Nonetheless, interested readers are invited for further independent exploration. In this work, emphasis is given to the top three layers, the \textit{Goal-, Products-, and Activities-Layer}.%, firstly due to time and information constraints and secondly as practical applicable methods need to be highly adjusted to the local context. This will further be discussed and outlined in subsequent chapters.

\subsection{Citizen Guidelines and Recommendations}

Practical recommendations, frameworks and guidelines for \acrlong*{cs} projects have become numerous and thematically wide-ranging. Networks and programmes of researchers and practitioners covering all or parts of the stages proposed by the \acrshort*{ssf} are spatially widespread at the global level and include a variety of regional, national or global application levels. Some examples of such networks and programmes are government programmes such as the US-run \href{https://www.citizenscience.gov/}{citizenscience.gov} website or the EU platform \href{https://eu-citizen.science/}{eu-citizen.science}, support platforms such as \href{https://citsci.org/}{CitSci} and the \href{https://citizenscience.ch/en/}{Citizen Science Center Zurich} as well as regionally focused associations like \href{http://cienciaparticipativa.net/la-ricap/}{La Red Iberoamericana de Ciencia Participativa (RICAP)}, \href{https://citizenscience.asia/}{CitizenScience.Asia} and the \href{https://www.usiu.ac.ke/citsci-africa-association/}{Citizen Science Africa Association}.\newline
Besides these knowledge-hubs, a vast variety of different scientific and and grey literature exists. \autocite{fraislCitizenScienceEnvironmental2022} and \autocite{westonCommunityBasedWaterMonitoring2015} have each listed and summarized many of recommendations and \autocite{garciaFindingWhatYou2021} even created a guide to citizen science guidelines. In this work, the guiding principles and recommendations of \autocite{minkmanCitizenScienceWater2015} were further considered and integrated. These guidelines were developed in cooperation with water management authorities in the Netherlands and concentrate on advice that can be implemented in practice. Furthermore, \autocite{minkmanCitizenScienceWater2015} derived a set of six potential goals which could be addressed through \acrlong*{cs} in water management namely (a) awareness raising, (b) public education, (c) policy development, (d) method improvements, (e) knowledge building, and (f) management improvements. The boundaries between the individual goals can be fluid and are not set in stone. Each \acrshort*{cs} project can address multiple of these goals and with varying emphasis. Although the overall objective of this work was already defined, all these six goals were chosen as a starting point for further analysis in order not to overlook potentially useful secondary goals. These 6 goals formed the first level of the SLMC in stage 3 of the \acrshort*{ssf}. However, the focus was on goals (e) and (f), as the main aims of this work fell into their areas. The remaining guidelines developed by \autocite{minkmanCitizenScienceWater2015} and above mentioned projects, programmes and associations were incorporated into the design and structuring of the questions for the interviews.
For the second stage of the \acrshort*{ssf}, determining the applicability of the citizen science approach, \autocite{fraislCitizenScienceEnvironmental2022}s own set of recommendations together with the IFRC recommendations for a \acrshort*{cbs} project assessment were applied \autocite{goodermoteConductingAssessmentCommunitybased2020}.





% possibly include something like
% and a collection of those guidelines can be seen in Table XYZ
% BRCiS learning, Minkman guidelines and so on.. + ESCA 10 guidelines and characteristics + all guidelines listed by weston 2015



% methods
the roadmap is oriented on Fraisl' et al.'s work as their work already summarized multiple guidelines. It adopts the grouping of the overall design procedure in six stages but adjusts these in correspondence to other guidelines. Most emphasis is given to the BRCiS and especially to work of the IFRC in regard to CBS and FbF manuals and recommendations. Furthermore, the \acrshort{ssf} procedural categorization is extended by a thematic grouping of the project requirements. The development and further elaboration of this grouping is guided by the \acrshort{slmc}.



% concluding summery
The research design and the applied frameworks and supporting guidelines were presented in this chapter. Case study research was identified as the most appropriate research type for the objective of this thesis and a mixed methods approach of a deductive literature analysis and an inductive interview strategy was defined. The advantages, disadvantages and main limitations of these approaches were outlined and will be further discussed in the following chapters. The guiding principles for the design of the roadmap were found in \acrlong*{ssf} and \acrlong*{slmc} and their integration was detailed. Further work from related projects, networks and associations was presented and their integration into the overall methodology of this work was outlined. Following these decisions for the design and application of the methodology, the results of this work are presented in the next chapter.




















% Guidelines:
% - Consider the design elements to make informed decisions
%     - Design tips:
%         - Comprehend what citizen science is and decide whether it is suitable as a means
%             -> formulate clear goals of the project ([Minkman, 2015, p. 177](zotero://select/groups/4773535/items/ZKLE6CPT)) ([pdf](zotero://open-pdf/groups/4773535/items/QMAPCSZG?page=177\&annotation=DERHMEYN))
%             -> formulate sub-goals and related products (e.g. list of things that should be known, ...)
%             -> determine which activities are most suitable to deliver the products
%         - Collaborate with partners
%             -> Identify partners (research institutes, official agencies, organizations, interest groups)
%             -> be aware of their interests and goals + who has an interest in a non-functioning system?
%         - Beware of financial motivations
%             -> infrastructure, development, coordination, training, equipment, etc. are non-negligible costs
%         - Key success-factor: Match patterns of collaboration with organizational capacity (regarding coordination and support)
%             -> determine patterns of collaboration: governance and level of involvement (-> contributory, collaborative, or co-created)
%         - Match techniques with organizational capacity (regarding data processing)
%             -> MCS (Mobile Crowd Sensing)
%                 - Pros:
%                     * highly mobile and scalable
%                     * low-cost
%                     * automatic time stamp and GPS possible
%                     * citizens could interfere when necessary
%                 - Cons:
%                     * devices are not specifically developed for the sensing task
%                     * the target audience may not be familiar with smartphones
%                     * difficult to ensure data trustworthiness
%                     * high risk of privacy invasion
%             -> An MCS application should be useful rather than easy-to-use
%                 	-> TAM (Technology Acceptance Model): ease of use, usefulness, behavioural intention to use a product
%         - Organize a pilot or have a trial period
%         - align organization interest/motivation with citizen interest/motivation
%             -> motivation overview p.167 + create ownership
%         - make sure data us used and provide feedback
%         - constantly recruit new participants
 
% - is citizen science suitable?
% - collaborate with strategic partners
% - care about the associated costs (development, infrastructure, coordination, training, etc.)
% - techniques and citizen science method should match organizational capacities
% - take (social) scientific knowledge into account
% - start with a pilot project
% - keep the capacity of the target audience, users and authorities, in mind. balance data trustworthiness and privacy is the main challenge

%----------------------------------------------------------------------------------------
%	SECTION 1
%----------------------------------------------------------------------------------------
% ESCA 10 Principles of Citizen Science (!767WRLDX!) as important as Minkman -> though minkman -> goals

% I actually already got most of them included
% “Summary of Recommendations for effective citizen science and community-based water monitoring and management” ([Weston and Conrad, 2015, p. 4](zotero://select/groups/4773535/items/49HXDHSH)) ([pdf](zotero://open-pdf/groups/4773535/items/CCHM5SNH?page=4&annotation=LBR9MKTV))

% core paper CBWM
% “Community-based water resources management” ([Day, 2009, p. 47](zotero://select/groups/4773535/items/YWSNQ8A2)) ([pdf](zotero://open-pdf/groups/4773535/items/ETPCI5RI?page=2&annotation=RQLJMKL7))
% +
% tons of recommendations
% “Community-Based Water Monitoring in Nova Scotia: Solutions for Sustainable Watershed Management” ([Weston and Conrad, 2015, p. 1](zotero://select/groups/4773535/items/49HXDHSH)) ([pdf](zotero://open-pdf/groups/4773535/items/CCHM5SNH?page=1&annotation=94TW58QF))

% https://brcis.shinyapps.io/EWEA_dashboard/

% 1. “Community leaders can play a greater role in humanitarian response.” 
% 2. “2. Risk monitoring has proved highly effective and essential for future programmes.” 
% 3. “3. Indicators should be capable of capturing the compound effects of multiple shocks.”
% “4. Vulnerable communities need improved access to climate and weather information” 
% “5. Local-level EWEA committees strengthen local capacities to prepare for and respond to shocks.” 
% “6. Flexible and shock-responsive funding mechanisms successfully improved food security.” 
% “7. Communities highly valued and engaged with the BRCiS integrated approach” ([Gualazzini, 2021, p. 20](zotero://select/groups/4773535/items/BWDYDL8T)) ([pdf](zotero://open-pdf/groups/4773535/items/8U5XVU5K?page=20&annotation=UTZLTZB3))

% “BRCiS early warning and early action vision (2022–2026)” ([Gualazzini, 2021, p. 20](zotero://select/groups/4773535/items/BWDYDL8T)) ([pdf](zotero://open-pdf/groups/4773535/items/8U5XVU5K?page=20&annotation=QVKJRXK6))


