% Chapter Template

\chapter{Chapter Title Here} % Main chapter title

\label{ChapterX} % Change X to a consecutive number; for referencing this chapter elsewhere, use \ref{ChapterX}

%----------------------------------------------------------------------------------------
%	SECTION 1
%----------------------------------------------------------------------------------------

\section{Main Section 1}

--> interpretive case research design with mixed-method approach > data and expert interview analysis using grounded theory with open, axial and possibly selective coding strategies

"We employed a mixed methods case study methodology that combined semi-structured interviews, technological and environmental surveys, and observations. "https://www.mdpi.com/2079-9276/9/6/77


based on grounded theory (?) -> inductive methodology/reasoning -> construction of hypothesis through the collection and analysis of data


--> "These ideas/concepts are said to "emerge" from the data. The researchers tag those ideas/concepts with codes that succinctly summarize the ideas/concepts. As more data are collected and re-reviewed, codes can be grouped into higher-level concepts and then into categories."https://en.wikipedia.org/wiki/Grounded_theory

--> Coding is an important part of it --> place incident into category and possibly order it hierarchically based on category or properties(dimension)

Sampling Method:
non-probability purposively sampling - deliberate selection of participants based on 'key informant technique' going forward in a snowball approach

understand WHAT I measure!
-> nominal, ordinal, interval, ratio

used instruments(?)


distinction between research design and method:
design is logical, while method is logistical
--> design is the plan, method is how to realize It

factors:
- ethics
- validity of data and reliability
- time
- 




mixed-method, (approach)
case study design (type of study) with 
exploratory/Explanatory (what, how and why) interests (study subtype) ("Note that explanatory research does not seek to provide conclusive answers, but to give an avenue to researchers to plumb the depths of the subject."https://research.com/research/types-of-research-design)

data analysis and semi-structured iterative interviews (research methods)


generally inductive procedure -> hypotheses are formulated afterwards and not upfront

data gathering technics:
- semi-structured interviews (grounded theory??)
- quantitative analysis of existing data sources
- analysis of former or existing projects/studies in the field
--> triangulate research instruments to provide different views of the case -> avoid the problem of observer bias (?)

Construct validity means selecting the most appropriate measurement tool for the concepts
being studied. Does your tool really measure what you want to assess?
Internal validity is another term for using different methodological tools to “triangulate” the
data. What other methods can you use to check for the same phenomenon?
External validity refers to how well the data can be applied beyond the circumstances of the
case to more general situations. Can you apply your data across the industry and to others?
Reliability means the extent to which the results can be repeated in ways that yield the same
results. That is, the extent to which the results are accurate and stable. Are you confident that
your study can be repeated by others and the results will be the same?

data collection preparation
•	 Databases designed to codify data
•	 Protocols for interviews and surveys
•	 Pilot studies that capture the concepts and data needed
•	 Formats for narrative reporting
•	 Field notes
•	 Procedures for tape or video recordings

Anticipate key events and problems
•	 Have a plan for unexpected changes, delayed appointments, lack of office space,
unavailability of staff, etc.
•	 Be open to contrary findings and unexpected events or interview responses
•	 Approach people who feel threatened or unsure about the case study in a delicate
manner
•	 Be prepared to revise the research design.

•	 Evidence must be collected systematically. While the case study methodology is
very flexible, it must be clear how the data from various sources contribute to the
overall aims of the study.
•	 You should not collect data randomly. There needs to be a purpose for collecting
certain data. Refine your research question/statement if necessary.
•	 Data must be stored in formats that can be referenced so that the patterns of
information are clear.
•	 Researchers should be able to see causal factors associated with the information
collected (How is X related to Y?)
•	 If changes need to be made to the data collection procedure, these changes need to
be recorded and documented.
•	 You need to be able to record anecdotes, comments and illustrations/examples
easily—these might turn out to be vital pieces of qualitative information.
•	 Notes should be kept recording the thoughts you have about the evolving case
study (e.g., in a case diary)

Analysis:

•	 One can quickly get lost in a web of complex data unless you have a clear research
objective. You should aim to seek data that answers this objective from as many
different sources as possible. Don’t try to establish thirty different things: establish
one thing well with data from different areas that confirm it.
•	 Good researchers cross-check the facts and discrepancies their data. They also
tabulate information so that it can be checked easily.
•	 Focus interviews may be needed to re-confirm existing data.
•	 Flow charts or other displays, and tabulating frequency of events are a good way of
recording and analysing information.
•	 Quantitative data can be used to corroborate and support the qualitative data
obtained and vice-versa.
•	 Multiple investigators can assist in seeing the patterns in the data. This can ensure
confidence in the data. If there are discrepancies in how the data is viewed then the
researchers need to investigate again.
•	 You should also investigate across case studies (from one company to another).
Don’t just settle on one example from one case.

%-----------------------------------
%	SUBSECTION 1
%-----------------------------------
\subsection{Subsection 1}
e.g.
"First, rural water supply in sub-Saharan Africa is critically contextualized as a ‘wicked problem’. Second, specific challenges to rural water supply in Tanzania are quantitatively assessed using expert interviews. Analysis of these coupled with academic and practitioner-oriented literature demonstrates the need to"https://practicalactionpublishing.com/article/2452/internet-of-things-innovation-in-rural-water-supply-in-sub-saharan-africa-a-critical-assessment-of-emerging-ict

%-----------------------------------
%	SUBSECTION 2
%-----------------------------------

\subsection{Subsection 2}
paper chapter title
“2 Local knowledge in drought monitoring: an introduction to the literature review” ([Giordano et al., 2013, p. 526](zotero://select/groups/4773535/items/B7LM5ZR4)) ([pdf](zotero://open-pdf/groups/4773535/items/7I66DBIK?page=4&annotation=Z33M5FLQ))
%----------------------------------------------------------------------------------------
%	SECTION 2
%----------------------------------------------------------------------------------------

\section{Main Section 2}

Selection of stakeholders:
“both information users—i.e., local decision makers—and information producers—i.e., members of the local community that can provide knowledge and information about the drought’s impacts at a local level—should be involved in the design process.” ([Giordano et al., 2013, p. 529](zotero://select/groups/4773535/items/B7LM5ZR4)) ([pdf](zotero://open-pdf/groups/4773535/items/7I66DBIK?page=7\&annotation=MNSWFWEE))

%----------------------------------------------------------------------------------------
%	SECTION 456
%----------------------------------------------------------------------------------------
difference scientific vs. stakeholders perception of drought impact
“The comparison between stakeholders’ perception of drought impacts and scientific knowledge allowed us to draw some preliminary conclusions concerning both the drought impacts at local level and the coherence between local and scientific knowledge on drought impacts.” ([Giordano et al., 2013, p. 539](zotero://select/groups/4773535/items/B7LM5ZR4)) ([pdf](zotero://open-pdf/groups/4773535/items/7I66DBIK?page=17\&annotation=QFD3UE4C))
“the correlation degree between perception indicators and scientific indicators is high only when considering a direct impact of drought, for example the reduction of productivity for non-irrigated crops.” ([Giordano et al., 2013, p. 540](zotero://select/groups/4773535/items/B7LM5ZR4)) ([pdf](zotero://open-pdf/groups/4773535/items/7I66DBIK?page=18&annotation=6KRIS4DA))
“Firstly, stakeholders tend to oversimplify the cause-effect chain” ([Giordano et al., 2013, p. 540](zotero://select/groups/4773535/items/B7LM5ZR4)) ([pdf](zotero://open-pdf/groups/4773535/items/7I66DBIK?page=18&annotation=IBV34JQL))
“Stakeholders seem to focus exclusively on the portion of the system they perceived.” ([Giordano et al., 2013, p. 541](zotero://select/groups/4773535/items/B7LM5ZR4)) ([pdf](zotero://open-pdf/groups/4773535/items/7I66DBIK?page=19&annotation=73P565M3))
“On the one hand, this allows them to have a clear picture of the evolutionary trends of the different variables that make up that portion of the system” ([Giordano et al., 2013, p. 541](zotero://select/groups/4773535/items/B7LM5ZR4)) ([pdf](zotero://open-pdf/groups/4773535/items/7I66DBIK?page=19&annotation=ZJSACLXJ))
“On the other hand, due to the limitations of their viewpoint, stakeholders tend to neglect the existence of multiple causes for certain observed phenomena.” ([Giordano et al., 2013, p. 541](zotero://select/groups/4773535/items/B7LM5ZR4)) ([pdf](zotero://open-pdf/groups/4773535/items/7I66DBIK?page=19&annotation=8CUNSCXS))
“Secondly, the data collected shows that the stakeholders tend to aggregate the variables describing the drought impacts, for example the impacts of crop productivity vary dramatically according to the crop under consideration, whereas stakeholders perceived a general reduction in productivity.” ([Giordano et al., 2013, p. 541](zotero://select/groups/4773535/items/B7LM5ZR4)) ([pdf](zotero://open-pdf/groups/4773535/items/7I66DBIK?page=19&annotation=TT7JRW3U))


“This work also demonstrates that there is a lack of synchronism between the stakeholders’ perception and the drought onset.” ([Giordano et al., 2013, p. 541](zotero://select/groups/4773535/items/B7LM5ZR4)) ([pdf](zotero://open-pdf/groups/4773535/items/7I66DBIK?page=19&annotation=H3XRYH3Q))

“In fact, stakeholders become aware of the drought only when they recognize the impacts on their perceived environment.” ([Giordano et al., 2013, p. 541](zotero://select/groups/4773535/items/B7LM5ZR4)) ([pdf](zotero://open-pdf/groups/4773535/items/7I66DBIK?page=19&annotation=869MME8Q))
“Often, those impacts are recognizable only when the drought is at its peak” ([Giordano et al., 2013, p. 541](zotero://select/groups/4773535/items/B7LM5ZR4)) ([pdf](zotero://open-pdf/groups/4773535/items/7I66DBIK?page=19&annotation=XTTUYS8M))

“Therefore, stakeholders’ indicators cannot be used to detect the drought onset or to assess the duration of the phenomenon” ([Giordano et al., 2013, p. 541](zotero://select/groups/4773535/items/B7LM5ZR4)) ([pdf](zotero://open-pdf/groups/4773535/items/7I66DBIK?page=19&annotation=C2QMHW76))

“They are more suitable to support the evaluation of the severity of the drought, and particularly of the indirect impacts.” ([Giordano et al., 2013, p. 541](zotero://select/groups/4773535/items/B7LM5ZR4)) ([pdf](zotero://open-pdf/groups/4773535/items/7I66DBIK?page=19&annotation=K8Z77CBR))

“The results of our experience show that drought perception indicators could be useful for defining thresholds of severity rather than for evaluating drought evolutionary trends. These thresholds could support decision makers in the development of effective drought risk management strategies.” ([Giordano et al., 2013, p. 542](zotero://select/groups/4773535/items/B7LM5ZR4)) ([pdf](zotero://open-pdf/groups/4773535/items/7I66DBIK?page=20&annotation=X47HGTNQ))
%----------------------------------------------------------------------------------------
%	SECTION 2 - coding + analysis
%----------------------------------------------------------------------------------------
A cooperative research process (Gummesson, 2002) was followed. It ranged from the verification of interview transcription, empirical material interpretation, and discussion of the final framework. By doing this, participants were provided a chance to verify whether the transcription/analysis was accurate. Interactions with research participants play an important role in idea generation and concept testing. This process also allows informants to provide feedback and suggestions to further improve and strengthen the findings of the study.

“the transcribed interview texts were coded and concepts were developed. These concepts were then combined to develop categories. These categories and results of interviews’ interpretation were triangulated with meeting observation field notes and documents.” ([Rashid et al., 2019, p. 8](zotero://select/groups/4773535/items/23TRE6V7)) ([pdf](zotero://open-pdf/groups/4773535/items/9EE7QIYZ?page=8&annotation=HJPAQSE4))

%----------------------------------------------------------------------------------------
%	SECTION 2
%----------------------------------------------------------------------------------------


%----------------------------------------------------------------------------------------
%	SECTION 2
%----------------------------------------------------------------------------------------
