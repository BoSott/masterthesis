% Chapter Template

\chapter{Chapter Title Here} % Main chapter title

\label{ChapterX} % Change X to a consecutive number; for referencing this chapter elsewhere, use \ref{ChapterX}

%----------------------------------------------------------------------------------------
%	SECTION 1
%----------------------------------------------------------------------------------------

\section{Research Design}

“Case Study Method: A Step-by-Step Guide for Business Researchers” ([Rashid et al., 2019, p. 1](zotero://select/groups/4773535/items/23TRE6V7)) ([pdf](zotero://open-pdf/groups/4773535/items/9EE7QIYZ?page=1&annotation=6PQGIXRY))

“Case Study Method for Design Research: A Justification” ([Teegavarapu et al., 2008, p. 0](zotero://select/groups/4773535/items/KZ4TKKH4)) ([pdf](zotero://open-pdf/groups/4773535/items/6UWHB8WI?page=1&annotation=DELVNW3E))
%-----------------------------------
%	SUBSECTION 1
%-----------------------------------
\subsection{Subsection 1}
The case study methodology
methodology rationale 
sampling strategy and sampling Criteria
data collection and analysis
coding and the analytic strategy
reliability and validity
data validation protocols
chapter conclusion
https://supervisorbullying.com/case-study-methodology-chapter/?amp=1

Case study methodology
https://bmcmedresmethodol.biomedcentral.com/articles/10.1186/1471-2288-11-100#:~:text=A%20case%20study%20is%20a,particularly%20in%20the%20social%20sciences.

network analysis + semi-structured interviews -> tailor the choice of variables to regional/national level/conditions “identifying key stakeholders, decision-makers, communities at risk, data collection agencies” ([Enenkel et al., 2020, p. 1166](zotero://select/groups/4773535/items/RX575C79)) ([pdf](zotero://open-pdf/groups/4773535/items/XD499UNK?page=6&annotation=USUTKZ27))



--> interpretive case research design with mixed-method approach > data and expert interview analysis using grounded theory with open, axial and possibly selective coding strategies

"We employed a mixed methods case study methodology that combined semi-structured interviews, technological and environmental surveys, and observations. "https://www.mdpi.com/2079-9276/9/6/77


based on grounded theory (?) -> inductive methodology/reasoning -> construction of hypothesis through the collection and analysis of data


--> "These ideas/concepts are said to "emerge" from the data. The researchers tag those ideas/concepts with codes that succinctly summarize the ideas/concepts. As more data are collected and re-reviewed, codes can be grouped into higher-level concepts and then into categories."https://en.wikipedia.org/wiki/Grounded_theory

--> Coding is an important part of it --> place incident into category and possibly order it hierarchically based on category or properties(dimension)

Sampling Method:
non-probability purposively sampling - deliberate selection of participants based on 'key informant technique' going forward in a snowball approach

understand WHAT I measure!
-> nominal, ordinal, interval, ratio

used instruments(?)


distinction between research design and method:
design is logical, while method is logistical
--> design is the plan, method is how to realize It

factors:
- ethics
- validity of data and reliability
- time
- 

mixed-method, (approach)
case study design (type of study) with 
exploratory/Explanatory (what, how and why) interests (study subtype) ("Note that explanatory research does not seek to provide conclusive answers, but to give an avenue to researchers to plumb the depths of the subject."https://research.com/research/types-of-research-design)

data analysis and semi-structured iterative interviews (research methods)


generally inductive procedure -> hypotheses are formulated afterwards and not upfront

data gathering technics:
- semi-structured interviews (grounded theory??)
- quantitative analysis of existing data sources
- analysis of former or existing projects/studies in the field
--> triangulate research instruments to provide different views of the case -> avoid the problem of observer bias (?)

"Advantages of Exploratory Research

    Lower costs of conducting the study
    Flexibility and adaptability to change
    Exploratory research is effective in laying the groundwork that will lead to future studies.
    Exploratory studies can potentially save time by determining at the earlier stages the types of research that are worth pursuing

 
Disadvantages of Exploratory Research

    Inclusive nature of research findings
    Exploratory studies generate qualitative information and interpretation of such type of information is subject to bias
    These types of studies usually make use of a modest number of samples that may not adequately represent the target population. Accordingly, findings of exploratory research cannot be generalized to a wider population.
    Findings of such type of studies are not usually useful in decision making in a practical level."
    https://research-methodology.net/research-methodology/research-design/exploratory-research/

%-----------------------------------
%	SUBSECTION 2
%-----------------------------------

\subsection{validity, reliability, etc.}

Construct validity means selecting the most appropriate measurement tool for the concepts
being studied. Does your tool really measure what you want to assess?
Internal validity is another term for using different methodological tools to “triangulate” the
data. What other methods can you use to check for the same phenomenon?
External validity refers to how well the data can be applied beyond the circumstances of the
case to more general situations. Can you apply your data across the industry and to others?
Reliability means the extent to which the results can be repeated in ways that yield the same
results. That is, the extent to which the results are accurate and stable. Are you confident that
your study can be repeated by others and the results will be the same?


%----------------------------------------------------------------------------------------
%	SECTION 2
%----------------------------------------------------------------------------------------

\section{Interviews & Questionnaires}
"In the research methodology chapter, you would describe the methods and procedures used to conduct the expert interviews, including the sampling strategy, data collection techniques, and data analysis methods. You would also explain the rationale for using expert interviews as a research method, and discuss any limitations or potential biases associated with this method." ChatGPT


% transcription (whisper) coding + analysis
A cooperative research process (Gummesson, 2002) was followed. It ranged from the verification of interview transcription, empirical material interpretation, and discussion of the final framework. By doing this, participants were provided a chance to verify whether the transcription/analysis was accurate. Interactions with research participants play an important role in idea generation and concept testing. This process also allows informants to provide feedback and suggestions to further improve and strengthen the findings of the study.

“the transcribed interview texts were coded and concepts were developed. These concepts were then combined to develop categories. These categories and results of interviews’ interpretation were triangulated with meeting observation field notes and documents.” ([Rashid et al., 2019, p. 8](zotero://select/groups/4773535/items/23TRE6V7)) ([pdf](zotero://open-pdf/groups/4773535/items/9EE7QIYZ?page=8&annotation=HJPAQSE4))


"Transcribing interviews
Before you get started with transcription, decide whether to conduct verbatim transcription or intelligent verbatim transcription.

If pauses, laughter, or filler words like “umm” or “like” affect your analysis and research conclusions, conduct verbatim transcription and include them.
If not, you can conduct intelligent verbatim transcription, which excludes fillers, fixes any grammatical issues, and is usually easier to analyze." https://www.scribbr.com/methodology/semi-structured-interview/


"Due to the open-ended nature of many semi-structured interviews, you will most likely be conducting thematic analysis, rather than content analysis.
You closely examine your data to identify common topics, ideas, or patterns. This can help you draw preliminary conclusions about your participants’ views, knowledge or experiences.
After you have been through your responses a few times, you can collect the data into groups identified by their “code.” These codes give you a condensed overview of the main points and patterns identified by your data.
Next, it’s time to organize these codes into themes. Themes are generally broader than codes, and you’ll often combine a few codes under one theme. After identifying your themes, make sure that these themes appropriately represent patterns in responses."https://www.scribbr.com/methodology/semi-structured-interview/

% questionnaire. Following is more about large questionnaires instead of small number like my more interview style questionnaires
"Scarce existing research capacity
International best practice survey guidelines indicate that effective recruitment, selection and training of survey interviewers is essential in order to ensure good surve data quality [16]. Having the right field staff on board can minimize interviewer effects (measurement errors for which interviewers are responsible), while controlling costs by optimizing interviewer efficiency [16]. However, since no project of this nature had previously been conducted in Manguzi, and skills-building opportunities in the area were scarce, we knew it would not be possible to recruit local field team members who already had the appropriate research skills and experience for our project. At the same time, we wanted to work with a field team consisting predominantly of local staff, for various reasons: we knew their knowledge of the local culture and general context would add significant value to the project [28, 29]; we knew it was important to the community that we train and hire young adults from Manguzi; and this was consistent with the ethos and commitment of all project partners to strengthening local research capacity as a means of advancing health and local development [15, 28, 30–32]. As indicated above, unemployment in the area was high and work options limited, so this study represented a unique opportunity for a selected group of individuals to gain field research training and experience that could help increase their future prospects of finding work. However, selecting staff without previous relevant experience – and in some cases no previous formal work experience at all – was a challenge. Moreover, we realised that we ran the risk of some individuals not managing to learn the required skills in the available time before the launch of the rural fieldwork." https://health-policy-systems.biomedcentral.com/articles/10.1186/1478-4505-11-14
"Our experience reinforced how a lack of relevant local skills can be an obstacle in remote areas with little research history, and how this may require creative solutions. As indicated, all project partner organisations placed significant importance on local capacity building as key to quality improvement and individual personal development [28, 30, 31, 33] and project funding was specifically allocated to these activities. "

"Consulting with the community and obtaining the consent of key representatives to conduct research is essential for various ethical and practical reasons [34, 35]. "

"No matter the extent of scientific rigour, consultation and planning, unexpected obstacles will inevitably emerge during field research, and these will require adaptive strategies to be overcome. In particular, conducting survey research in remote rural areas of the developing world may pose unique challenges that require appropriate context-relevant responses. Unfortunately, text books and the current body of scientific papers do little to equip researchers for the experience that awaits them. Obtaining trust and buy-in from key gatekeepers, overcoming logistics difficulties, effectively developing local skills and managing staff relocation are some fieldwork aspects that may be particularly trying when working in deep rural settings."

" In particular, we realised the importance of a strong local partnership, based on mutual respect, clear agreed goals of collaboration and shared development interests [26, 27], but also the willingness and honesty to question decisions and discuss alternative approaches."

problems and challenges are inevitable but addressing them in a timely and effective manner seems to be of high importance. (https://health-policy-systems.biomedcentral.com/articles/10.1186/1478-4505-11-14)

Guidelines https://ccsg.isr.umich.edu/chapters/questionnaire-design/

"This chapter borrows terminology from translation studies, which define ‘source language’ as the language translated out of and ‘target language’ as the language translated into. In like fashion, the chapter distinguishes between ‘source questionnaires’ and ‘target questionnaires.’ Source questionnaires are questionnaires used as a blueprint to produce other questionnaires, usually on the basis of translation into other languages (see Translation: Overview); target questionnaires are versions produced from the source questionnaire, usually on the basis of translation or translation and adaptation (see Adaptation). Target questionnaires enable researchers to study populations who could not be studied using the source questionnaire."

" Research by Liu, Suzer-Gurtekin, Keusch & Lee (2019) finds evidence of acquiescence response style (ARS) (the tendency to choose ‘agree’ or ‘yes’ responses) in cross-cultural surveys. Black and Hispanic respondents show more ARS compared to White respondents, while one out of three statistical models provide evidence that Black respondents display more extreme response style (ERS) (the tendency to choose the two endpoints of response scales more frequently than other categories) compared to their White counterparts."

fact categories
https://www.maxqda-press.com/wp-content/uploads/sites/4/978-3-948768072.pdf
%----------------------------------------------------------------------------------------
%	SECTION 2
%----------------------------------------------------------------------------------------

\section{Tool analysis}

%----------------------------------------------------------------------------------------
%	SECTION 3
%----------------------------------------------------------------------------------------

\section{data / document analysis}

codes added with the analysis of Beledis interview
%----------------------------------------------------------------------------------------
%	SECTION 4
%----------------------------------------------------------------------------------------

\section{Frameworks}

% e.g. since it can be characterized as 'wicked problem', thorough frameworks are required to guide the development and design phase
"First, rural water supply in sub-Saharan Africa is critically contextualized as a ‘wicked problem’. Second, specific challenges to rural water supply in Tanzania are quantitatively assessed using expert interviews. Analysis of these coupled with academic and practitioner-oriented literature demonstrates the need to"https://practicalactionpublishing.com/article/2452/internet-of-things-innovation-in-rural-water-supply-in-sub-saharan-africa-a-critical-assessment-of-emerging-ict

The combination of the  process focused Seven-Layer of Collaboration Model together with the 6 stage design and implementation cycle proposed by \autocite[postnote]{fraislCitizenScienceEnvironmental2022} (“Citizen science in environmental and ecological sciences” ([Fraisl et al., 2022, p. 1](zotero://select/groups/4773535/items/FBJD7SWS)) ([pdf](zotero://open-pdf/groups/4773535/items/7WBDKYDY?page=1&annotation=5HMYC85E))) embeds this work in a larger methodological framework. Besides this foundation, the developed guidelines for crowd sensing projects by \autocite[postnote]{minkman} \Autocite*[test]{minkmanCitizenScienceWater2015} will accompany this work.


% starting point
The 7-layer-model provides a detailed roadmap for the design of collaboration systems that allows for the separation of concerns and thus minimizes cognitive overload. While it is focused on the design phase, it is not conclusive nor specifically designed for the development of a citizen science project. Therefore, it is combined with the design and implementation 5-stage-cycle specifically addressing Crowdsourcing projects. Furthermore, practical guidelines from other projects and a literature review are also considered.

The design and implementation cycle was specifically developed as a toolkit that "provides five basic process steps for planning, designing and carrying out a crowdsourcing or citizen science project" (https://www.citizenscience.gov/toolkit/#). This toolkit gives a great overall framework for the development of a citizen science project, but lacks specifications in the design stage. In this work is applied twofold. First, it embeds the seven-layer model in a crowdsensing project development framework precisely designed for this purpose and secondly it provides a good overview about all stages, facilitating a comprehensive analysis of other citizen science projects. 


The 7-Layer Model of Collaboration can add additional (more detailed) guidance specifically in the design phase. Thus, the 7-Layer Model is the primary, in detail guide, while the 5-stages-cycle embeds this into the larger context of the full life cycle of such a development.
Besides these methodological frameworks, tangible guidelines based on other projects experiences are also integrated into the design phase as focal points to which additional attention is paid in the process.

Nonetheless there are areas were both systems overlap. The first stage of the 5-stage-cycle e.g. overlaps with the first 'goal' layer of the 7-layer-model and the fifth stage aligns with the 5th,6th, and 7th layer. While these overlaps may exist, they do not contradict each other but just highlight the importance of those aspects also in later stages of the citizen science project.


what about dimensions: economic, political, social, cognitive, physical, and technical dimensions
--> all play into this project but to different degrees. Focus here is on the network, process and technical dimension on how to implement such a citizen science project in a resource scarce environment  

%% not sure.. needs some more thinking.. fliegt raus. Ist zu viel und nicht nötig.
While this gives the work a well founded methodological base, it is primarily based on the perspective of a process focussed understanding/thinking of the design phase. Other perspectives like the ones of resource, behavioral network / stakeholder, value network and culture, as well as the communication network perspective may come into play in certain aspects but are of secondary nature in this work.

--> these perspectives accompany specifically the design process in the 7-layer-model - encouraging a more holistic view of the design.

%-----------------------------------
%	SUBSECTION 4.1
%-----------------------------------
\subsection{6-Stages-Cycle}



%-----------------------------------
%	SUBSECTION 4.2
%-----------------------------------

\subsection{7 layer model of collaboration}


“Reflection on the Seven-Layer Model of Collaboration” ([Minkman, 2015, p. 196](zotero://select/groups/4773535/items/ZKLE6CPT)) ([pdf](zotero://open-pdf/groups/4773535/items/QMAPCSZG?page=196&annotation=77VLLID3))



“A Seven-Layer Model of Collaboration: Separation of Concerns for Designers of Collaboration Systems” ([Briggs et al., pp. -1](zotero://select/groups/4773535/items/SRDHCWQ7)) ([pdf](zotero://open-pdf/groups/4773535/items/QTN3UENG?page=1&annotation=S2YZBBKM))

“Integrating the research into the Seven-Layer Model of Collaboration 12.1” ([Minkman, 2015, p. 165](zotero://select/groups/4773535/items/ZKLE6CPT)) ([pdf](zotero://open-pdf/groups/4773535/items/QMAPCSZG?page=165&annotation=SUXRFEMP))

%-----------------------------------
%	SUBSECTION 4.3
%-----------------------------------
\subsection{Minkman 2015}\label{subsec:guidelines}
use this as beginning?
“Chapter 13 Synthesis – product” ([Minkman, 2015, p. 173](zotero://select/groups/4773535/items/ZKLE6CPT)) ([pdf](zotero://open-pdf/groups/4773535/items/QMAPCSZG?page=173&annotation=5FECLF3E))

Guidelines:
- Consider the design elements to make informed decisions
    - Design tips:
        - Comprehend what citizen science is and decide whether it is suitable as a means
            -> formulate clear goals of the project ([Minkman, 2015, p. 177](zotero://select/groups/4773535/items/ZKLE6CPT)) ([pdf](zotero://open-pdf/groups/4773535/items/QMAPCSZG?page=177\&annotation=DERHMEYN))
            -> formulate sub-goals and related products (e.g. list of things that should be known, ...)
            -> determine which activities are most suitable to deliver the products
        - Collaborate with partners
            -> Identify partners (research institutes, official agencies, organizations, interest groups)
            -> be aware of their interests and goals + who has an interest in a non-functioning system?
        - Beware of financial motivations
            -> infrastructure, development, coordination, training, equipment, etc. are non-negligible costs
        - Key success-factor: Match patterns of collaboration with organizational capacity (regarding coordination and support)
            -> determine patterns of collaboration: governance and level of involvement (-> contributory, collaborative, or co-created)
        - Match techniques with organizational capacity (regarding data processing)
            -> MCS (Mobile Crowd Sensing)
                - Pros:
                    * highly mobile and scalable
                    * low-cost
                    * automatic time stamp and GPS possible
                    * citizens could interfere when necessary
                - Cons:
                    * devices are not specifically developed for the sensing task
                    * the target audience may not be familiar with smartphones
                    * difficult to ensure data trustworthiness
                    * high risk of privacy invasion
            -> An MCS application should be useful rather than easy-to-use
                	-> TAM (Technology Acceptance Model): ease of use, usefulness, behavioural intention to use a product
        - Organize a pilot or have a trial period
        - align organization interest/motivation with citizen interest/motivation
            -> motivation overview p.167 + create ownership
        - make sure data us used and provide feedback
        - constantly recruit new participants
 

- is citizen science suitable?
- collaborate with strategic partners
- care about the associated costs (development, infrastructure, coordination, training, etc.)
- techniques and citizen science method should match organizational capacities
- take (social) scientific knowledge into account
- start with a pilot project
- keep the capacity of the target audience, users and authorities, in mind. balance data trustworthiness and privacy is the main challenge
- 

“To set up a successful citizen science project, alignment of goals between water authorities and citizens is important and this thesis’s findings can provide initial understanding of citizen motivations.” ([Minkman, 2015, p. 11](zotero://select/groups/4773535/items/ZKLE6CPT)) ([pdf](zotero://open-pdf/groups/4773535/items/QMAPCSZG?page=11&annotation=F2MJEUJY))

other MSc developed guidelines for water authorities

“Guidelines for water authorities” ([Minkman, 2015, p. 12](zotero://select/groups/4773535/items/ZKLE6CPT)) ([pdf](zotero://open-pdf/groups/4773535/items/QMAPCSZG?page=12&annotation=F6Q2WRGX))

--> taking these guidelines and apply them?????

“Testing the guidelines in a practical set up would be an excellent start to further develop the knowledge on how to implement citizen science in water quality monitoring.” ([Minkman, 2015, p. 13](zotero://select/groups/4773535/items/ZKLE6CPT)) ([pdf](zotero://open-pdf/groups/4773535/items/QMAPCSZG?page=13&annotation=MCHKYZAE))
\section{Main Section 2}


%----------------------------------------------------------------------------------------
%	SECTION 1
%----------------------------------------------------------------------------------------
ESCA 10 Principles of Citizen Science (!767WRLDX!) as important as Minkman -> though minkman -> goals

https://www.citizenscience.gov/toolkit/howto/step2/#

“Citizen Science as an Approach for Overcoming Insufficient Monitoring and Inadequate Stakeholder Buy-in in Adaptive Management: Criteria and Evidence” ([Aceves-Bueno et al., 2015, p. 493](zotero://select/groups/4773535/items/YK2MKLA9)) ([pdf](zotero://open-pdf/groups/4773535/items/WGGHNGZB?page=1&annotation=6KGDS66K))

“To set up a successful citizen science project, alignment of goals between water authorities and citizens is important and this thesis’s findings can provide initial understanding of citizen motivations.” ([Minkman, 2015, p. 11](zotero://select/groups/4773535/items/ZKLE6CPT)) ([pdf](zotero://open-pdf/groups/4773535/items/QMAPCSZG?page=11&annotation=F2MJEUJY))




\section{Main Section 1}