% Chapter Template

\chapter{Structure} % Main chapter title

\label{Structure} % Change X to a consecutive number; for referencing this chapter elsewhere, use \ref{ChapterX}
%----------------------------------------------------------------------------------------

% Define some commands to keep the formatting separated from the content 
\newcommand{\keyword}[1]{\textbf{\#1}}
\newcommand{\tabhead}[1]{\textbf{\#1}}
\newcommand{\code}[1]{\texttt{\#1}}
\newcommand{\file}[1]{\texttt{\bfseries\#1}}
\newcommand{\option}[1]{\texttt{\itshape\#1}}

%----------------------------------------------------------------------------------------
%	SECTION 1
%----------------------------------------------------------------------------------------


I. Acknowledgements

II. Summery / Abstract (1p)

III. Table of Contents

IV. Table of figures and tables

V. Abbreviations

\subsection*{1. Introduction (5-10 pages)}
\begin{itemize}
\item Background and motivation for the research
\item Problem statement and research questions
\item Scope and limitations of the study
\item Overview of the thesis structure
\end{itemize}

\subsection*{2. Literature review (20-30 pages)}
\begin{itemize}
\item Review of relevant literature and research in the field of geoinformatics
\item Theoretical framework and concepts related to the research questions
\item Identification of gaps and limitations in the existing literature
\end{itemize}

\subsection*{3. Methodology (15-25 pages)}
\begin{itemize}
\item Description of the research design and methods used in the study (new method, new data, new application)
\item Data collection procedures and sources of data
\item Data analysis techniques and tools used
\item Ethical considerations and limitations of the study
\end{itemize}

\subsection*{4. Results (20-30 pages)}
\begin{itemize}
\item Presentation and interpretation of the research findings
\item Analysis of the data and their implications for the research questions
\item Use of maps, charts, and other visual aids to illustrate the results
\end{itemize}

\subsection*{5. Discussion (15-25 pages)}
\begin{itemize}
\item Evaluation of the research findings in light of the literature review
\item Comparison of the results with previous research and theories
\item Interpretation of the implications and limitations of the study
\item Suggestions for future research in the field of geoinformatics
\end{itemize}

\subsection*{6. Conclusion (5-10 pages)}
\begin{itemize}
\item Summary of the research findings and their implications
\item Contributions to the field of geoinformatics
\item Limitations and recommendations for further research
\item Conclusion and final thoughts on the research
\end{itemize}

\subsection*{7. References (varies)}
\begin{itemize}
\item List of all sources cited in the thesis
\end{itemize}

\subsection*{8. Appendices (varies)}
\begin{itemize}
\item Additional data, maps, and other supplementary materials used in the research.
\end{itemize}


\subsection*{9. Statement of independent work}
\begin{itemize}
\item I hereby confirm that this thesis was written independently by myself without the use of any sources beyond those cited, and all passages and ideas taken from other sources are cited accordingly.
\end{itemize}
