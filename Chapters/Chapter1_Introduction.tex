% Chapter Template

\chapter{Introduction} % Main chapter title

\label{Chapter1} % Change X to a consecutive number; for referencing this chapter elsewhere, use \ref{ChapterX}


%----------------------------------------------------------------------------------------
%	SECTION 0 Proposal
%----------------------------------------------------------------------------------------

\section{Proposal}
\# Overview
\#\# Motivation: Importance of Drought + local users

With the projection of rising frequency and intensity of droughts throughout vast parts of the African continent, measures for prediction, monitoring and evidence based anticipatory actions and management become ever more important \autocite{abdulkadirAssessmentDroughtRecurrence2017,adelekanAfricaClimateChange2022,vereintenationenSpecialReportDrought2021}.

In order to meet this challenge, the Red Cross Red Crescent Movement together with the Red Cross Red Crescent Climate Center started the Forecast-based Financing (FbF) programme in 2007 to facilitate Anticipatory Actions instead of post-disaster reactions. Together with their local partners, the International Federation of Red Cross Red Crescent Societies (IFRC) is working on establishing so called Early Action Protocols (EAPs) to ensure better organization and coordination of Anticipatory Actions in the face of an incoming hazard. These actions are based on a predefined interplay of forecast, trigger and financing mechanisms to ensure rapid, scientific based responses.

Somaliland, being no exception to the above mentioned climatic trend, is characterized by droughts with far reaching impacts on ecological, economic, and social aspects. Defined by a semi-arid, four-season climate with two extensive dry seasons and an economic backbone of pastoralism and rain-fed agriculture, water accessibility is of key importance in Somaliland \autocite{abdulkadirAssessmentDroughtRecurrence2017,petrucciLandscapeLandformsNorthern2022,republicofsomalilandSomalilandCountryProfile2021}.
In addition, the final report of the FbF feasibility study identifies five other hazards besides droughts, namely flood, cyclone, disease, locusts and conflict. Of all these hazards, drought was ranked as the greatest threat due to its increasing frequency, severity and wide-ranging consequences.\newline

Existing drought forecasts and trigger indicators are mostly based on physical indicators and cover especially macro- and international levels, e.g. EDDI, SPI, SPEI \todo{look them up + which forecast is defined by the SRCS?}. Fine grained up-to-date forecasts which not only include physical but also social circumstances and knowledge on local levels are often scarce or even non-existent \todo[fancyline]{sources}. "However, assessments focused only on physical variables and processes fail to capture why drought matters [...]."\autocite[3]{lackstromBackyardHydroclimatologyCitizen2022}, how it impacts communities and which mitigation strategies are locally taken \todo{source?}.\newline

Besides the further development of more fine grained technical solutions the integration of local knowledge is another way forward. Engaging local people and communities and giving them an active voice in defining and co-producing anticipatory actions and knowledge can be of multiple benefit to communities and enrich the data generated \autocite{somaliredcrescentsocietyFeasibilityStudyPotential2022, njambi-szlapkaIntegratingCommunityVoices}\todo{source}. On the one hand, citizens can help to fill data gaps of categorized measurements such as simple assessments of dry-to-wet conditions which correspond to the above mentioned technical drought indicators \autocite{lackstromBackyardHydroclimatologyCitizen2022}. On the other hand, citizens can contribute their local knowledge which can potentially draw on years of experience, encompass a wide range of aspects and give them a feeling of co-owning which in itself can already bring many advantages \autocite{njambi-szlapkaIntegratingCommunityVoices}\todo{source}. The IFRC states, that the "community engagement and accountability (CEA) is essential […] to build acceptance and trust” \autocite{ifrcCommunityEngagementAccountability}(IFRC n.d.) for effective and sustainable outcomes. 
Nonetheless, direct contributions and communication from and with volunteers or community members remain a challenge in the joint management of hazards and risks. The tasks are numerous and need to take into consideration different aspects, ranging from cultural differences to different background knowledge and technical capabilities and capacities.

Qualitative interviews can capture a wide range of local contribution, knowledge and information but are very labour intensive, often limited in time and hence not feasible on larger scale for monitoring purposes. Providing a technical solution that enables mutual contact between communities and the Somalia Red Crescent Society (SRCS) in a simple and accessible way would make it possible to collect this most helpful information. By ensuring that sovereignty over the collected data rests with the community, the decision-making power and benefits remain with those concerned. Therefore, in the development and application of an Early Action Protocol (EAP) in the context of Forecast based Action (FbA) in Somalia, the technical aspect in particular holds great potential to benefit numerous communities over the long term.

While there is a comparatively good basis with regard to the general availability of data on social, economic and natural information and conditions, as well as their spatial and qualitative characteristics, the timeliness of this information varies greatly. Especially in the context of Anticipatory Action, time is of the essence and a swift action based on up-to-date data is crucial. Gathering and incorporating local knowledge through manual surveys by a central organisation can be of great value, as shown above, but is time-consuming and slow.

The benefits of overcoming the high labour costs and the temporal handicap by providing a simple and accessible tool for the SRCS’s large network of volunteers can already be seen in other comparable contexts using the Community Based Surveillance (CBS) programme, its successor NYSS, the Ushahidi platform or tools like Social.Water, ITIKI and CoCoRaHS \autocite{fienenSocialWaterCrowdsourcing2012a} \todo{sources}. The CBS is a pioneering technical approach to disease outbreak surveillance in Somalia and was developed as part of the SRCS Health Strategy 2019-2023. In this process, “volunteers report health risks through their respective locations and send to the SRCS data platform […]” \autocite[57]{somaliredcrescentsocietyFeasibilityStudyPotential2022} through coded SMS. So far, 315 volunteers from 115 villages have specifically been trained. In at least two cases, an outbreak could be contained thanks to early detection (SRCS 2021, 2022). On top of this already proven application, further functions could be added as needed.
Social.Water, ITIKI and CoCoRaHS are environmental data collection and drought monitoring implementations. These approaches are based on Crowdsensing approaches and partly employ more sophisticated technical solutions via internet connections, wireless sensor networks or photographs and have been implemented and tested in recent studies or ongoing monitoring programmes. While suggesting that the method of Crowdsensing for data collection and monitoring is fit for purpose, these approaches are not feasible to this context due to internet connection and technical equipment requirements, lack of categorization or their focus on just environmental data acquisition without taking sufficient account of the impact on the community impact and social realities.

The aim of this study is therefore to incorporate local knowledge on the availability of water sources into the monitoring of drought impacts in an effort to support triggering and Anticipatory Actions under the Early Action Protocol. For this purpose, already proven methods are combined. First, semi-structured (expert-) interviews will be conducted to generate in-depth local knowledge. Based on these findings on local (pre-)conditions, needs and limitations, a monitoring tool based on a volunteersensing approach will then be conceptualized and subsequently discussed with local decision-makers. A prototype development based on the Social.Water tool is a further possibility. In addition, the question of the equality of local to scientific knowledge will be raised and further influences of such a contribution on social conditions will be investigated \todo{too much?}.
This work is thus based on the question \textit{"In what way can a collaborative Volunteer Sensing Approach be conceptualized to facilitate community water source monitoring in Somaliland in order to enhance early drought triggers and following Anticipatory Actions?"} and explores following hypotheses:
1. Local knowledge can enhance early drought impact triggers in the context of Anticipatory Actions in Somaliland.
2. Water source availability, accessibility and quality are feasible ecological and social indicators for monitoring early drought impacts and guide Anticipatory Actions in Somaliland.
3. The combination of qualitative interviews and Volunteered Crowdsensing is a promising approach to qualitatively and quantitatively monitor early drought impacts on communities in Somaliland.

%----------------------------------------------------------------------------------------
%	SECTION 1
%----------------------------------------------------------------------------------------

\section{topic and context / background motivation}
background, timely, importance
% Background and motivation for the research / Topic and Context
% topic and context (academic debate, practical problem)

% Begin by introducing your dissertation topic and giving any necessary background information. It’s important to contextualize your research and generate interest. Aim to show why your topic is timely or important. You may want to mention a relevant news item, academic debate, or practical problem.

climate change

“Climate variability, climate change and global trends in drought hazard” ([“Special report on drought 2021”, 2021, p. 29](zotero://select/groups/4773535/items/RAAM9PVS)) ([pdf](zotero://open-pdf/groups/4773535/items/7AK5QVBL?page=31&annotation=7NYZ8XDU))
“Figure 1.3. Change in meteorological drought frequency (events/decade) from recent past (1981–2010) to 2100 for four projected warming levels of global surface air temperature (left) and number of drought events with stronger severity than ever recorded in the recent past (1981–2010) (right)” ([“Special report on drought 2021”, 2021, p. 32](zotero://select/groups/4773535/items/RAAM9PVS)) ([pdf](zotero://open-pdf/groups/4773535/items/7AK5QVBL?page=34&annotation=D7VGP69I))

% IPCC
“9.5.5 East Africa” ([Adelekan et al., 2022, p. 1327](zotero://select/groups/4773535/items/8HKQ6EZ3)) ([pdf](zotero://open-pdf/groups/4773535/items/9H76NWQ2?page=43&annotation=U8RXJHIH))

“Climate change has increased heat waves (high confidence) and drought (medium confidence) on land, and doubled the probability of marine heatwaves around most of Africa (high confidence). Multi-year droughts have become more frequent in west Africa, and the 2015–2017 Cape Town drought was three times more likely2 due to human-caused climate change. {9.5.3–7, 9.5.10} Increases in drought frequency and duration are projected over large parts of southern Africa above 1.5°C global warming (high confidence), with decreased precipitation in North Africa at 2°C global warming (high confidence), and above 3°C global warming, meteorological drought frequency will increase, and duration will double from approximately 2 months to 4 months in parts of North Africa, the western Sahel and southern Africa (medium confidence). {9.5.2, 9.5.3, 9.5.6.}” ([Adelekan et al., 2022, p. 1290](zotero://select/groups/4773535/items/8HKQ6EZ3)) ([pdf](zotero://open-pdf/groups/4773535/items/9H76NWQ2?page=6&annotation=DTHH5TGR))



impact of drought on the community

“Somaliland is characterized by drought” ([Abdulkadir, 2017, p. 104172233225941010000](zotero://select/groups/4773535/items/G2RYLAC4)) ([pdf](zotero://open-pdf/groups/4773535/items/5PPIPUKD?page=1&annotation=PTQSFYGD))

“known to have the most far-reaching impacts of all natural disasters” ([Abdulkadir, 2017, p. 104172233225941010000](zotero://select/groups/4773535/items/G2RYLAC4)) ([pdf](zotero://open-pdf/groups/4773535/items/5PPIPUKD?page=1&annotation=C6R6KIS4))

drought & water scarcity


from global to regional (horn of Africa) to local (Somalia - Somaliland)


historic events -> from the past to the present (response) to the future (anticipatory)



Early Actions / Anticipatory Actions / FbF



based on forecasts -> Forecasts



mapping & monitoring of water source type Berkad


“Water scarcity, as a supply/demand-driven and natural and/or human-made phenomenon, is one of the greatest challenges of the twenty-first century” ([pdf](zotero://open-pdf/groups/4773535/items/JM82W3ZF?page=13&annotation=WZB8I8FY))

%----------------------------------------------------------------------------------------
%	SECTION 2
%----------------------------------------------------------------------------------------
\section{Focus and Scope}
narrow, pin point
% After a brief introduction to your general area of interest, narrow your focus and define the scope of your research.

% You can narrow this down in many ways, such as by:

% Geographical area
% Time period
% Demographics or communities
% Themes or aspects of the topic
transition: this will help to tackle worsening situation


ground truthing
local impact forecasts
local knowledge (while it could integrate local knowledge more, this work will only touch this topic)
Volunteersensing
mapping and monitoring with Crowdsensing
EAP & AAs & trigger







%----------------------------------------------------------------------------------------
%	SECTION 3
%----------------------------------------------------------------------------------------
\section{Relevance and Importance}
motivation, relation, insights, reasoning
% It’s essential to share your motivation for doing this research, as well as how it relates to existing work on your topic. Further, you should also mention what new insights you expect it will contribute.

% Start by giving a brief overview of the current state of research. You should definitely cite the most relevant literature, but remember that you will conduct a more in-depth survey of relevant sources in the literature review section, so there’s no need to go too in-depth in the introduction.

% Depending on your field, the importance of your research might focus on its practical application (e.g., in policy or management) or on advancing scholarly understanding of the topic (e.g., by developing theories or adding new empirical data). In many cases, it will do both.

% Ultimately, your introduction should explain how your thesis or dissertation:

% Helps solve a practical or theoretical problem
% Addresses a gap in the literature
% Builds on existing research
% Proposes a new understanding of your topic
transition: this will help to tackle worsening situation

whatever source of water they can find is what they have. (2023-03-04_Beledi, Pos. 20)


“This shift is reflected globally through policy initiatives of the United Nations such as Agenda 21 or the Aarhus Convention which emphasize that “the serious environmental, social, and economic challenges faced by societies worldwide cannot be addressed by public authorities alone” (UNECE, 2008).” ([Weston and Conrad, 2015, p. 1](zotero://select/groups/4773535/items/49HXDHSH)) ([pdf](zotero://open-pdf/groups/4773535/items/CCHM5SNH?page=1&annotation=DEWMV6FY))

“Additionally, several of the existing tools were found to be very specific and limited physical and hydrological aspects, failing to incorporate important factors such as socio-economic and political characteristics.” ([Butte et al., 2022, p. 17](zotero://select/groups/4773535/items/QB97YZ2M)) ([pdf](zotero://open-pdf/groups/4773535/items/Q936I2JN?page=17&annotation=KWSSFHK2))

more and more severe droughts in Somalia

but Forecasts primarily global scale
thus far, AAs are often limited by data availability
reparation and water trucking as mayor actions/response -> detailed information necessary

+ embeddedness of citizen projects -> more benefits (see goals)



https://www.unwater.org/our-work/integrated-monitoring-initiative-sdg-6

"Stronger accountability: Data can communicate that work is being done and progress is happening. Data can enable greater transparency, which reduces inefficiency and corruption.
Attracting commitment and investments: Data can quantify problems and make it easier to communicate needs for political commitment and public and private investments.
Evidence-based decision-making: Data can inform policy- and decision-makers of where to focus efforts and which solutions are most effective, to ensure the greatest possible gains with existing resources.
Leaving no one behind: Disaggregated data can help identify specific groups or areas with unmet needs and higher levels of risk, to which interventions can be targeted."
https://www.unwater.org/our-work/integrated-monitoring-initiative-sdg-6/background

“Experts in trigger methodology have indicated a more appropriate strategy may be to build on tools that currently exist at the government level such as national drought monitoring systems. As such, the ideal is an iterative process with the ground level along with a technology push that creates new ways to analyse drought and drought risk.” ([RCRC, 2020, p. 28](zotero://select/groups/4773535/items/UESIQTRJ)) ([pdf](zotero://open-pdf/groups/4773535/items/P5JPVZ97?page=28&annotation=977VS8FC))

% DRM Strategic Plan General and Specific Objectives
“2.1.1 Specific Objective 1 Vulnerable communities’ resilience at SRCS target areas strengthened through anticipatory actions, response, recovery, and disaster risk reduction, and they can anticipate and effectively respond to and recover from disasters and crisis by 2026.” ([“SRCS DRM Strategic Plan_final 9thNovember 2021-converted.pdf”, p. 15](zotero://select/groups/4773535/items/LFCBRZLD)) ([pdf](zotero://open-pdf/groups/4773535/items/6IL6K72G?page=15&annotation=ZFICKZRA))

% --> even the RCRC is still looking for good triggers -> maybe water levels are a good way -> reasoning for this study (see background identical text)

“Countries in which less than 50\% of the population uses improved drinking water sources are all located in sub-Saharan Africa and Oceania 91-100\% 76-90\% 50-75\% <50\% insufficient data or not applicable Proportion of the population using improved drinking water sources in 2015 ■ 91–100\% ■ 76–90\% ■ 50–75\% ■ <50\% ■ INSUFFICIENT DATA OR NOT APPLICABLE” ([World Health Organization, 2016, p. 15](zotero://select/groups/4773535/items/KVAKZ9ZT)) ([pdf](zotero://open-pdf/groups/4773535/items/4STYK52H?page=14\&annotation=FBURDS4T))

“The methods by which the Joint Monitoring Programme (JMP) of WHO and UNICEF” ([Bartram et al., 2014, p. 8137](zotero://select/groups/4773535/items/6AWUJTW5)) ([pdf](zotero://open-pdf/groups/4773535/items/BFNSQGWS?page=1&annotation=UL4Q2I4V))
“substantial limitations: current methods do not address water quality, equity of access, or extra-household services.” ([Bartram et al., 2014, p. 8137](zotero://select/groups/4773535/items/6AWUJTW5)) ([pdf](zotero://open-pdf/groups/4773535/items/BFNSQGWS?page=1&annotation=TIPCEXGG))

current challenges for utilisation of forecasting systems: scarse coverage of weather stations and poor utilisation by the farmers often due to bad dissemination channels  (too coarse, too unreliable)






“Spotlight on Somalia: Can we learn from failure? In Somalia in 2011, a famine was declared that, along with the complexity of the conflict situation, was responsible for thousands of deaths. At its peak, almost 4 in every 10 children in Southern Somalia were acutely malnourished, and 4 million people were estimated to be without basic food. The horror of this tragedy has since haunted the international community, who received 11 months of early warnings before a famine was declared. Beginning with La Nina forecasts almost one year in advance, FEWS-NET and others provided briefing notes and warning information to humanitarian actors in the region. Several months later, these alerts explained that rainy seasons had already failed, and that major impacts were extremely likely (Hillbruner and Moloney 2012). There has been a great deal of analysis of this event, in which several conclusions have come to light. One is that funding needs to be more readily available based on forecasted information (Lautze et al. 2012). The below graph (Hillbruner and Moloney 2012) demonstrates how large-scale funding was mobilized in the aftermath of the famine declaration, and was ultimately available after the most vulnerable had died. Secondly, the humanitarian community needs to clearly take responsibility for acting in advance of a disaster, even in complex cases like the Somali context. At the moment, such organizations are not held accountable for failure to act on early warning, as disaster response is considered business-asusual. Shouldering the responsibility to act in this critical moment between a warning and a disaster could avoid such impacts in the future (Lautze et al. 2012).” ([pdf](zotero://open-pdf/groups/4773535/items/WKFPUZCW?page=7&annotation=LIY2ABLI))




%----------------------------------------------------------------------------------------
%	SECTION 4
%----------------------------------------------------------------------------------------
\section{Questions and objectivese}
expectations, central aim, framework
% Perhaps the most important part of your introduction is your questions and objectives, as it sets up the expectations for the rest of your thesis or dissertation. How you formulate your research questions and research objectives will depend on your discipline, topic, and focus, but you should always clearly state the central aim of your research.

% If your research aims to test hypotheses, you can formulate them here. Your introduction is also a good place for a conceptual framework that suggests relationships between variables.


research questions
development of 'roadmap'
combination of existing products/concept to new tool in new environment




%----------------------------------------------------------------------------------------
%	SECTION 5
%----------------------------------------------------------------------------------------
\section{Overview of the structure}
1-2 sentences outline, brief summery of each chapter
% To help guide your reader, end your introduction with an outline of the structure of the thesis or dissertation to follow. Share a brief summary of each chapter, clearly showing how each contributes to your central aims. However, be careful to keep this overview concise: 1-2 sentences should be enough.

transition: the way forward...

more or less standard shizzle


%----------------------------------------------------------------------------------------
%	SECTION 6
%----------------------------------------------------------------------------------------
\section{Definitions of terminology}

drought
local knowledge


%----------------------------------------------------------------------------------------
%	Notes:
%----------------------------------------------------------------------------------------


Checklist: Introduction 0 / 7
I have introduced my research topic in an engaging way.

I have provided necessary context to help the reader understand my topic.

I have clearly specified the focus of my research.

I have shown the relevance and importance of the dissertation topic.

I have clearly stated the problem or question that my research addresses.

I have outlined the specific objectives of the research.

I have provided an overview of the dissertation’s structure.