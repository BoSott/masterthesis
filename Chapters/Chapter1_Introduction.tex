% Chapter Template

\chapter{Introduction} % Main chapter title

\label{Chapter1} 
% The aim of this study is therefore to incorporate local knowledge on the availability of water sources into the monitoring of drought impacts in an effort to support triggering and Anticipatory Actions under the Early Action Protocol. For this purpose, already proven methods are combined. First, semi-structured (expert-) interviews will be conducted to generate in-depth local knowledge. Based on these findings on local (pre-)conditions, needs and limitations, a monitoring tool based on a volunteersensing approach will then be conceptualized and subsequently discussed with local decision-makers. 


Water is a crucial element for sustaining life, and access to it is a fundamental necessity for every society and human being. Nonetheless, water security is increasingly becoming a pressing issue affecting the lives of billions of people across the world \autocite{caretta2022water}. In order to meet this challenge, the United Nations have already recognized the importance of water security in 2015 and have made clean water and sanitation the sixth Sustainable Development Goal \autocite{unGoalEnsureAvailability2016}. Yet, in many regions water scarcity as a long-term water supply and demand imbalance is expected to further deteriorate leading to the recent announcement of the \acrlong{wmo} that \textit{"water scarcity [...] is one of the greatest challenges of the twenty-first century"} \autocite[7]{idmpDroughtWaterScarcity2022}.\newline
% droughts
Droughts, known for their widespread and potentially extreme impacts, can further exacerbate a strained water scarcity situation \autocite{idmpDroughtWaterScarcity2022}. With the projection of rising frequency and intensity of droughts throughout vast parts of the African continent, measures for prediction, monitoring and evidence based \acrfullpl{aa} and management become ever more important \autocite{abdulkadirAssessmentDroughtRecurrence2017,trisosAfrica2022,vereintenationenSpecialReportDrought2021}.\newline
% case study
As Somaliland is characterised by a semi-arid four-season climate with two extended dry seasons and an economic backbone of pastoralism and rain-fed agriculture, access to water is crucial \autocite{abdulkadirAssessmentDroughtRecurrence2017,petrucciLandscapeLandformsNorthern2022,republicofsomalilandSomalilandCountryProfile2021}. Furthermore, Somaliland is no exception to the above mentioned climatic trend and with more than 17 major droughts in the last 60 years, a recent famine in 2010-2012, and an increasingly devastating situation again since 2018, Somalia is severely affected by droughts \autocite{abdulkadirAssessmentDroughtRecurrence2017,credEMDATInternationalDisasters2023}. The final report of the \acrfull{fbf} feasibility study identified five other hazards besides drought, namely floods, cyclones, diseases, locusts and conflict. But of all these hazards, drought was ranked as the greatest threat due to its increasing frequency, severity and far-reaching consequences \autocite{scrsFeasibilityStudyPotential2022}.\newline

% Call for action
The presently but also generally very challenging water resource situation in Somaliland requires comprehensive, effective and efficient solutions. The \acrfull{srcs}, \acrfull{grc} and \acrfull{heigit} are currently working together to develop strategies to prepare for these disasters. However, it has become apparent that many drinking water related activities are severely limited in their feasibility as information on most water sources is either incomplete, outdated or non-existent. Therefore, the \acrshort{srcs} calls for new and innovative approaches to close this information and management gap.

% fbf
The \acrshort{fbf} programme was generally started in 2007 by the \acrlong{rcrc} Movement together with the \acrlong{rcrccc} to facilitate \acrshortpl{aa} instead of post-disaster reactions \autocite{ifrcForecastbasedFinancingNew2019}. Together with their local partners, the \acrfull{ifrc} is working on establishing so called \acrfullpl{eap} to ensure better organization and coordination of \acrshortpl{aa} in the face of an incoming disaster. These \acrshortpl{aa} are based on a predefined interplay of evidence based forecasts, triggers, actions and financing mechanisms to ensure rapid reactions.\newline
% forecasts
Triggering \acrshortpl{aa} is generally linked to certain forecast thresholds. Once these thresholds are reached, the \acrshortpl{aa} are carried out. The forecasts are generally based on drought indicators and indices, which mostly include only physical indicators and are primarily macro and international in scope \autocite{svobodaHandbookDroughtIndicators2016}. Fine grained up-to-date forecasts which also include social realities and knowledge on local levels are rare or even non-existent \autocite{enenkelWhyPredictClimate2020,masindeFrameworkPredictingDroughts2010a}. "However, assessments focused only on physical variables and processes fail to capture why drought matters [...]."\autocite[3]{lackstromBackyardHydroclimatologyCitizen2022} and how it directly impacts communities \autocite{boultDroughtImpactbasedForecasting2022,enenkelWhyPredictClimate2020}.\newline
% citizens
One way for more on site information is the involvement of local citizens. Engaging local citizens and communities into monitoring activities and decision-making can be of multiple benefit to a wide variety of aspects \autocite{scrsFeasibilityStudyPotential2022, njambi-szlapkaIntegratingCommunityVoices}. Scientific processes of e.g. linking climate variability to local water security can be informed, the public's education and awareness about specific topics can be raised, and decision-making and overall management can be enhanced \autocite{huangManagementDrinkingWater2020,kirschkeCitizenScienceProjects2022,minkmanCitizenScienceWater2015}. The \acrshort{ifrc} states, that the "community engagement and accountability (CEA) is essential [\dots] to build acceptance and trust” for effective and sustainable outcomes" \autocite{ifrcCommunityEngagementAccountability}.\newline

In the last two decades, \acrfull{cs} has become a vibrant area of scientific interest covering various aspects in many different contexts \autocite{kirschkeCitizenScienceProjects2022,kullenbergWhatCitizenScience2016}. Relatively recent developments in \acrlong{cbm} and \acrlong{mcs} now make it possible for a large number of citizens to contribute to scientific, social and environmental endeavours with just a simple phone \autocite{butteFrameworkWaterSecurity2022}. By applying these approaches, \acrshort{cs} projects have demonstrated their ability to gather and fill data gaps particularly in formerly data sparse regions in an effective and cost-efficient manner \autocite{butteFrameworkWaterSecurity2022,lackstromBackyardHydroclimatologyCitizen2022,weeserCitizenSciencePioneers2018a}. However, environmentally and ecologically oriented projects and studies are currently unevenly distributed around the world, with a focus on North America, Europe and Australia, making it difficult to provide guidance for other contexts \autocite{kirschkeCitizenScienceProjects2022, koehlerCitizenParticipationCollaborative2008, livinglakescanadaElevatingCommunityBased2018}. Furthermore, most of these projects require internet access and dedicated technical equipment, which makes their methods unfeasible for low-income contexts \autocite{fienenSocialWaterCrowdsourcing2012a,lackstromBackyardHydroclimatologyCitizen2022,lowryGrowingPainsCrowdsourced2019}.\newline
Most developed frameworks and guidelines therefore primarily represent experiences and applications from potentially very different contexts and only roughly comparable thematic foci. However, broadly comparable projects show the general applicability of participatory monitoring approaches also in the Global South (BRCiS, OCHA, CBS, weeser 2018). In particular, the \acrlong{cbs} project established by the \acrshort{srcs} and others to facilitate preventive health related activities in Somaliland indicates the local implementability \autocite{ifrcCommunityBasedSurveillanceGuiding2017,scrsFeasibilityStudyPotential2022}.\newline

%! HEEEEEEEEEEEEEEEEEEEEEEEEEEEEEEEEEEEEEEEEEEEEEEEEEEEEEEEEEEEEEEEEEEEELP HEEEEEEEEEEEEEEEEEEEEEEEEEEEEEEEEEEEEEELP MEEEEEEEEEEEEEEEEEEEEEEEEEE!!!!!!!!!!!!!
Following the need for better data to facilitate \acrshortpl{aa} in the tense situation of Somaliland, this work attempts to develop an innovative community-based participatory water source mapping and monitoring framework by combining the concepts of FbF and CS. By bringing together the developed methods and gained experiences from thematically comparable projects and from local implementations, this framework adapts and builds on current knowledge to recombine the concepts to the conditions in Somaliland. In addition, the developed framework is subsequently applied by interviewing local experts to create a roadmap for practical implementation that will ultimately address the existing data gaps and enable respective \acrshort{aa} for tackling water shortages proactively. In order to provide a basis for the practical implementation of this goal, two research questions need to be addressed and answered:

%! OOOOOOOOOOOOOOOOOOOOOOOOOOOOOOODEER
Building on the concepts of FbF and CS, developed methods, and experiences gained from thematically comparable projects and from local implementations, this work attempts to \textit{adapt and apply an approach for community-based participatory mapping and monitoring of water sources in the water-scarce and resource-limited setting of Somaliland in collaboration with the \acrshort{srcs} to facilitate respective \acrlongpl{aa} in the context of \acrlong{fbf}, with the ultimate goal of improving water management and availability to address water shortages}. In order to provide a basis for the practical implementation of this goal, two research questions need to be addressed and answered:

% this work attempts to combine the methods and experiences from thematically comparable projects and local implementations to form a community-based participatory water source mapping and monitoring approach that is fit for purpose in the context of Somaliland. In addition, the developed framework will be applied by interviewing local experts to create a roadmap for practical implementation that will ultimately address the existing data gaps and enable the relevant \acrshort{aa}.
% gap by \textit{adapting and applying an approach for community-based participatory mapping and monitoring of water sources in a water-scarce and resource-limited setting in collaboration with a national non-governmental organization to facilitate respective \acrlongpl{aa} in the context of \acrlong{fbf}, with the goal of improving water management and availability to address water shortages}. In order to provide a basis for the practical implementation of this goal, two research questions need to be addressed and answered:

\begin{questions}
    \item How can a replicable and adaptable framework for community-based participatory water source mapping and monitoring in regard to the research aim and the overall case study context of Somaliland be developed?
    \item In the specific context of Somaliland, how can the developed framework be applied to create a tailored roadmap for the implementation of a community-based participatory water source mapping and monitoring project for triggering Anticipatory Actions to address water shortages?
\end{questions}

% some stuff about evaluations (?)
\noindent For the purpose of answering these research questions, this thesis is structured as follows: Chapter \ref{chapter2} introduces fundamental concepts such as Water Security, Water Scarcity and Drought, and provides an overview of the concepts and sub-concepts of \acrshort{fbf} and \acrshort{cs}. Furthermore, the case study area of Somaliland is presented in this chapter, covering its climate, geography, socio-economic situation and current \acrlong{drr} developments. Chapter 3 provides an overview of the methods used in this thesis and briefly introduces the two main underlying frameworks and other related guidelines that guide the new design. The following Chapter \ref{chapter4}, presents the newly developed community-based water source monitoring \acrfull{ssdr} and \acrfull{prc} framework. The findings of the interviews and the application of this framework to the context of Somaliland are subsequently presented. The following discussion of the developed design framework, the findings of the application, as well as the elaboration on the general limitations of this work are dealt with in Chapter 5. The conclusion in Chapter \ref{chapter6} summarises the main findings, reflects on their contribution to science and practice and provides an outlook for further work.
