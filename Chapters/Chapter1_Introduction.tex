% Chapter Template

\chapter{Introduction} % Main chapter title

\label{Chapter1} % Change X to a consecutive number; for referencing this chapter elsewhere, use \ref{ChapterX}

%  possibly just one research question with some sub-questions
%----------------------------------------------------------------------------------------
%	SECTION 0 Proposal
%----------------------------------------------------------------------------------------
% The introduction should clearly establish the focus and purpose of the literature review.
% highlight research gap + emphasize the timeliness
% Locate your own research within the context of existing literature [very important!]. 
% reiterate central problem
% focus and purpose of literature review
% brief summery scholarly context
% highlight research gap
% emphasize timeliness
% isn't this all too much?

% Overview: 7 Common Introduction Mistakes
% 1. Not providing sufficient context for the study
% -> sufficient contextual foundation (literature and real world)
% -> what, where, who, when
% -> what the fk is going on! -> foundation for research justification
% write for the intelligent layman (intellectual curious but not an expert)
% start from the bottom

% 2. Not presenting a strong justification for the research topic
% justify research aims (not only: it hasn't been done before -> originality is important but not the only thing)
% -> discuss novelty of the project
% -> also the practical and theoretical importance (more research in the global south)
% (a) What are you researching (and how is this novel/original)?
% (b) Why is it important (will add value to the field)?
% (c) Who is going to benefit from the research or who will struggle without it?

% 3. Having a research topic that’s too broad
% -> not too broad
% -> tightly define the research \textbf{aim}

% 4. Having poorly defined research aims, objectives and research questions
% -> aim, objectives and questions

% 5. Having misaligned research aims, objectives and research questions

% 6. Not having well-defined and/or justified scope
% narrow scope and specify boundaries
% where, when, who --> focused, manageable, replicable
% -> justify scope
% -> mention how much research has already been undertaken in the research area of interest
% -> say one it hasn't been done (could have been done, but as I know from I2, publications are rare)

% 7. Not providing a clear structural outline of the document
% -> detailed and clear outline of the thesis, -> document structure -> orient the reader -> where to find stuff
% does not need to be lengthy -> a line or two for each chapter

% research aims 

% This study sets out to assess and apply community-based participatory water source mapping and monitoring frameworks, designs and applications in a resource restricted environment in cooperation with a national non-government organisation.

% The aim of this study is to explore, potentially adapt and apply community-based participatory mapping and monitoring guidelines and frameworks for water sources in a water-scarce and resource-limited setting in collaboration with a national non-governmental organisation.

% research aim
% The study sets out to design an approach for community-based participatory mapping and monitoring of water sources in a water-scarce and resource-limited setting in collaboration with a national non-governmental organisation to facilitate respective \acrlongpl{aa} in the context of \acrlong{fbf}.

% The aim of this study is to design and test an approach for community-based participatory mapping and monitoring of water sources in a water-scarce and resource-limited setting in collaboration with a national non-governmental organization to facilitate respective \acrlongpl{aa} in the context of \acrlong{fbf}, with the goal of improving water management and accessibility in underserved communities.

% % research objectives
% 1. To understand the context and explore principles, guidelines and best practices for the application of a \acrlong{cs} programme.
% 2. To assess the feasibility of the \acrlong{cs} approach in the given context.
% 3. To potentially adjust and expand those frameworks and recommendations to the research aim and overall case study context.
% 4. To apply the framework(s) in order to create a roadmap for the design approach in regard to the prevailing conditions. 

% % research questions
% 1. What guidelines and best practices exist for the design of a \acrlong{cs} project in the thematic direction of the research aim.
% 2. Which combination of frameworks best suits this approach and how can they further be adjusted to the research aim?
% 3. How can a \acrlong{cs} project be conceptualised for the application of a community-based participatory water source monitoring programme with special attention to its feasibility for \acrlongpl{aa} in a resource restricted environment?
% 3.1. What information needs to be known about the source initially and continuously?

% you are an experienced grad coach with prolonged experience in designing and writing Master Theses. Give your detailed feedback to the following chain of research aim, objectives and questions and outline potential improvements by providing several reformulations with explanations why those are superior.

% as: preparatory study (??)

% Ensuring water security is considered as one of the major key challenges of the twenty-first century .

%----------------------------------------------------------------------------------------
%	SECTION 1 topic and context / background motivation
%----------------------------------------------------------------------------------------
% background, timely, importance
% Background and motivation for the research / Topic and Context
% topic and context (academic debate, practical problem)

% Begin by introducing your dissertation topic and giving any necessary background information. It’s important to contextualize your research and generate interest. Aim to show why your topic is timely or important. You may want to mention a relevant news item, academic debate, or practical problem.
Water is a crucial element for sustaining life, and access to it is a fundamental necessity for every society and human being. Nonetheless, water security is increasingly becoming a pressing issue affecting the lives of billions of people across the world \autocite{caretta2022water}. In order to meet this challenge, the United Nations have already recognized the importance of water security in 2015 and have made clean water and sanitation the sixth Sustainable Development Goal \autocite{unGoalEnsureAvailability2016}. Yet, in many regions water scarcity as a long-term water supply/demand imbalance is expected to further deteriorate leading to the recent announcement of the \acrlong*{wmo} that \textit{"water scarcity [...] is one of the greatest challenges of the twenty-first century"} \autocite[7]{idmpDroughtWaterScarcity2022}.\newline
% droughts
Droughts, known for their widespread and potentially extreme impacts, can further exacerbate a strained water scarcity situation \autocite{idmpDroughtWaterScarcity2022}. With the projection of rising frequency and intensity of droughts throughout vast parts of the African continent, measures for prediction, monitoring and evidence based anticipatory actions and management become ever more important \autocite{abdulkadirAssessmentDroughtRecurrence2017,trisosAfrica2022,vereintenationenSpecialReportDrought2021}.\newline
% case study
As Somaliland is characterised by a semi-arid four-season climate with two extended dry seasons and an economic backbone of pastoralism and rain-fed agriculture, access to water is crucial. \autocite{abdulkadirAssessmentDroughtRecurrence2017,petrucciLandscapeLandformsNorthern2022,republicofsomalilandSomalilandCountryProfile2021}. Furthermore, Somaliland is no exception to the above mentioned climatic trend and with more than 17 major droughts in the last 60 years, a recent famine in 2010-2012, and an increasingly devastating situation again since 2018, Somalia is severely affected by droughts \autocite{abdulkadirAssessmentDroughtRecurrence2017,credEMDATInternationalDisasters2023}. The final report of the \acrshort{fbf} feasibility study identified five other hazards besides drought, namely floods, cyclones, diseases, locusts and conflict. But of all these hazards, drought was ranked as the greatest threat due to its increasing frequency, severity and far-reaching consequences \autocite{scrsFeasibilityStudyPotential2022}.\newline
% fbf
The \acrfull{fbf} programme was started in 2007 by the \acrlong{rcrc} Movement together with the \acrlong*{rcrccc} to facilitate \acrlongpl{aa} instead of post-disaster reactions \autocite{ifrcForecastbasedFinancingNew2019}. Together with their local partners, the \acrfull*{ifrc} is working on establishing so called Early Action Protocols (EAPs) to ensure better organization and coordination of \acrshortpl{aa} in the face of an incoming disaster. These \acrlongpl{aa} are based on a predefined interplay of evidence based forecasts, triggers, actions and financing mechanisms to ensure rapid reactions.\newline
% forecasts
Triggering \acrshortpl{aa} is generally linked to certain forecast thresholds. Once these thresholds are reached, the \acrshortpl{aa} are carried out. Forecasts are  most often based on existing drought indicators and indices such as EDDI, SPI, SPEI mostly integrate physical indicators and refer primarily to the macro and international level and are thus relatively coarse \autocite{svobodaHandbookDroughtIndicators2016}. Fine grained up-to-date forecasts which not only include physical but also social circumstances and knowledge on local levels are rare or even non-existent \autocite{enenkelWhyPredictClimate2020,masindeFrameworkPredictingDroughts2010a}. "However, assessments focused only on physical variables and processes fail to capture why drought matters [...]."\autocite[3]{lackstromBackyardHydroclimatologyCitizen2022} and how it directly impacts communities \autocite{boultDroughtImpactbasedForecasting2022,enenkelWhyPredictClimate2020}.\newline
% “There has been little effort to align the spatiotemporal granularity of socioeconomic assessments with the granularity of weather or climate monitoring.” ([Enenkel et al., 2020, p. 1161](zotero://select/groups/4773535/items/RX575C79)) ([pdf](zotero://open-pdf/groups/4773535/items/XD499UNK?page=1&annotation=QBTLFCXM))
%----------------------------------------------------------------------------------------
%	SECTION 2 Focus and Scope
%----------------------------------------------------------------------------------------
% After a brief introduction to your general area of interest, narrow your focus and define the scope of your research.
% You can narrow this down in many ways, such as by:
% Geographical area
% Time period
% Demographics or communities
% Themes or aspects of the topic
% transition: this will help to tackle worsening situation
% ground truthing
% local impact forecasts
% local knowledge (while it could integrate local knowledge more, this work will only touch this topic)
% Volunteersensing
% mapping and monitoring with Crowdsensing
% EAP & AAs & trigger
% citizens
Besides the further development of more fine grained technical solutions, the integration of local citizens is another way forward. Engaging local citizens and communities and giving them an active voice in defining and co-producing \acrshortpl{aa} and knowledge can be of multiple benefit to a wide variety of aspects and enrich the data generated \autocite{scrsFeasibilityStudyPotential2022, njambi-szlapkaIntegratingCommunityVoices}. Scientific processes of e.g. linking climate variability to local water security can be informed, the public's education and awareness about specific topics can be raised, and decision-making and overall management can be enhanced by local knowledge, if the project is embedded in these procedures \autocite{huangManagementDrinkingWater2020,kirschkeCitizenScienceProjects2022,minkmanCitizenScienceWater2015}. The \acrshort{ifrc} states, that the "community engagement and accountability (CEA) is essential […] to build acceptance and trust” for effective and sustainable outcomes \autocite{ifrcCommunityEngagementAccountability}.\newline
In the last two decades, \acrlong{cs} has become a vibrant area of scientific interest covering various aspects in many different contexts \autocite{kirschkeCitizenScienceProjects2022,kullenbergWhatCitizenScience2016}. Relatively recent developments in \acrlong{cbm} and \acrlong{mcs} now make it possible for a large number of citizens to contribute to scientific, social and environmental endeavours with just a simple phone \autocite{butteFrameworkWaterSecurity2022}. By applying these approaches, \acrshort{cs} projects have demonstrated their ability to gather and fill data gaps particularly in formerly data sparse regions in an effective and cost-efficient manner \autocite{butteFrameworkWaterSecurity2022,lackstromBackyardHydroclimatologyCitizen2022,weeserCitizenSciencePioneers2018a}. However, currently \acrshort{cs} projects and studies are primarily located in North America, Europe and Australia \autocite{kirschkeCitizenScienceProjects2022, koehlerCitizenParticipationCollaborative2008, livinglakescanadaElevatingCommunityBased2018}. Social.Water, CoCoRaHS and \autocite{speirSolutionsCurrentChallenges2022}'s study are examples of those environmental data collection and drought monitoring implementations focussing on monitoring river, lake, groundwater and precipitation levels. Yet, all these approaches require internet access and more technical equipment, making them unfeasible for low-income conditions \autocite{fienenSocialWaterCrowdsourcing2012a,lackstromBackyardHydroclimatologyCitizen2022,lowryGrowingPainsCrowdsourced2019}.\newline
Most developed frameworks and guidelines therefore primarily represent experiences and applications from above mentioned regions and thematic foci. However, broadly comparable projects show the general applicability of participatory monitoring approaches also in the Global South (BRCiS, OCHA, CBS, weeser 2018). In particular, the \acrlong{cbs} project established by the \acrshort{srcs} and others to survey diseases in order to prevent outbreaks through \acrlongpl{aa} in Somaliland, show the local implementability \autocite{ifrcCommunityBasedSurveillanceGuiding2017,scrsFeasibilityStudyPotential2022}.\newline

% possibly include that this work is under the umbrella of a current \acrlong{eap} development and also requested by the \acrlong{srcs}

The presently very challenging water resource situation in Somaliland requires comprehensive, effective and efficient solutions. The current \acrlong{eap} developments under the umbrella of \acrshort{fbf} could be a potential candidate, but preventive activities are currently severely limited by the poor data situation. Therefore, the local National Society, the \acrfull{srcs} calls for new and innovative approaches to close this information and management gap. Building on the methods developed and experience gained from thematically comparable projects and local implementations, this work attempts to reduce this gap by \textit{adapting and applying an approach for community-based participatory mapping and monitoring of water sources in a water-scarce and resource-limited setting in collaboration with a national non-governmental organization to facilitate respective \acrlongpl{aa} in the context of \acrlong{fbf}, with the goal of improving water management and availability to address water shortages}. In order to provide a basis for the practical implementation of this goal, two research questions need to be addressed and answered:

\begin{questions}
    \item How can a replicable and adaptable framework for community-based participatory water source mapping and monitoring in regard to the research aim and the overall case study context be developed?
    \item In the specific context of this case study, how can the developed framework be applied to create a tailored roadmap for the implementation of a community-based participatory water source mapping and monitoring project for triggering Anticipatory Actions to address water shortages in Somaliland?
\end{questions}

% some stuff about evaluations (?)
\noindent For the purpose of answering these research questions, this thesis is structured as follows: Chapter \ref{chapter2} introduces fundamental concepts and provides relevant background information on the case study area. Chapter \ref{chapter3} covers the methodology for developing and applying the framework and Chapter \ref{chapter4} presents the resulting findings. The subsequent discussion of the methods, results and limitations in Chapter \ref{chapter5} is structured in the same way, also with regard to the sequence of important topics. The conclusion in Chapter \ref{chapter6} summarises the main findings, reflects on the questions and results in terms of their contribution to science and practice and provides an outlook for further work.

% These two chapters start with the methodological framework development in the first part and continue with the application-oriented topic in the second part



% It starts with the discussion of fundamental concepts of water security, drought, and water scarcity, followed by an elaboration on \acrlong{fbf} including forecast, trigger and actions specifics. The \acrlong{cs} concept together its sub-concepts \acrlong{cbm}, \acrlong{mcs} and practical examples is subsequently presented. The chapter concludes with a description of the case study area.
 
% define and subsequently address the methodological and conceptual gap in literature for the design, implementation and operation of a methodological Citizen Science based, FbF policy oriented and drought focussed framework. And secondly, a first application and evaluation of the newly developed framework in the context of the FbF project in Somaliland. 

% In answering these research questions, multiple sub-goals can also be discussed.. 


% in the light of above challenges -> 

% taking low data availability into account --> the combination of the water related topic, with the proven mechanisms of cbs can help to address information gap in the operation of fbf by providing timely, local and accurate information about local circumstances to trigger directly corresponding \acrshortpl{aa}.

% % in regard to Somalia and BRCiS rationale to perform more monitoring
% “Prior to 2019, BRCiS consortium members did not have a single primary data collection mechanism for making informed decisions and bridging the gap between the national-level early warning systems and community information needs. Although there was substantive commonality between the members’ individual systems, key challenges remained.” ([Gualazzini, 2021, p. 4](zotero://select/groups/4773535/items/BWDYDL8T)) ([pdf](zotero://open-pdf/groups/4773535/items/8U5XVU5K?page=4&annotation=P963PKKK))
% “Data inaccuracy, access limitations, a lack of thresholds and inadequate resources for data collection resulted in non-standardised information, preventing comparison across areas and weakening strategic decision-making” ([Gualazzini, 2021, p. 4](zotero://select/groups/4773535/items/BWDYDL8T)) ([pdf](zotero://open-pdf/groups/4773535/items/8U5XVU5K?page=4&annotation=L5SELDEJ))
% “Even when receiving timely information on the ground, Members often had to wait for validation from national sources such as the Food Security and Nutrition Analysis Unit (FSNAU) to make decisions, and the resources for anticipatory action were very limited.” ([Gualazzini, 2021, p. 4](zotero://select/groups/4773535/items/BWDYDL8T)) ([pdf](zotero://open-pdf/groups/4773535/items/8U5XVU5K?page=4&annotation=CZST8S2L))

% % takes long (25 days) is super resource intensive (key informant interviews -> 3 communities, 4 people each, lots of work.) + trust + triangulation + barely any automatization and rather coarse -> no detailed risk and vulnerability assessments (though this point may not be too valid.. -> the people know the area)
% “Real-Time Risk Monitoring (RTRM)” ([Gualazzini, 2021, p. 4](zotero://select/groups/4773535/items/BWDYDL8T)) ([pdf](zotero://open-pdf/groups/4773535/items/8U5XVU5K?page=4&annotation=HZFMU84X))} % not too sure if I wanna include this or where.. maybe only in the discussion part? -> in comparison to Crowdsensing/MCS this is slow and relatively coarse. -> OCHA 2022 learning as well

% but: 
% “Scale is critical in assessing water security [31]. National level assessments make it difficult to take action at operationalization level.” ([Mishra et al., 2021, p. 8](zotero://select/groups/4773535/items/MD2Z2HTF)) ([pdf](zotero://open-pdf/groups/4773535/items/366Z36U7?page=8&annotation=6IAXCXUB))

% “The methods by which the Joint Monitoring Programme (JMP) of WHO and UNICEF” ([Bartram et al., 2014, p. 8137](zotero://select/groups/4773535/items/6AWUJTW5)) ([pdf](zotero://open-pdf/groups/4773535/items/BFNSQGWS?page=1&annotation=UL4Q2I4V))
% “substantial limitations: current methods do not address water quality, equity of access, or extra-household services.” ([Bartram et al., 2014, p. 8137](zotero://select/groups/4773535/items/6AWUJTW5)) ([pdf](zotero://open-pdf/groups/4773535/items/BFNSQGWS?page=1&annotation=TIPCEXGG))

% and: “Creating and using indicators for water security has to be directed towards some management control or assessment action.” ([Mishra et al., 2021, p. 8](zotero://select/groups/4773535/items/MD2Z2HTF)) ([pdf](zotero://open-pdf/groups/4773535/items/366Z36U7?page=8&annotation=P72LT9Y8))

% % DRM Strategic Plan General and Specific Objectives
% “2.1.1 Specific Objective 1 Vulnerable communities’ resilience at SRCS target areas strengthened through anticipatory actions, response, recovery, and disaster risk reduction, and they can anticipate and effectively respond to and recover from disasters and crisis by 2026.” ([“SRCS DRM Strategic Plan_final 9thNovember 2021-converted.pdf”, p. 15](zotero://select/groups/4773535/items/LFCBRZLD)) ([pdf](zotero://open-pdf/groups/4773535/items/6IL6K72G?page=15&annotation=ZFICKZRA))
% % --> even the RCRC is still looking for good triggers -> maybe water levels are a good way -> reasoning for this study (see background identical text)
% “Thinking outside the box in terms of both hydro-meteorological and socio-economic indicators could be particularly useful” ([RCRC, 2020, p. 31](zotero://select/groups/4773535/items/UESIQTRJ)) ([pdf](zotero://open-pdf/groups/4773535/items/P5JPVZ97?page=31&annotation=GNZJ3FR5))


% % now its better (?) but still rather slow compared to MCS

% current challenges for utilisation of forecasting systems: scarse coverage of weather stations and poor utilisation by the farmers often due to bad dissemination channels  (too coarse, too unreliable)

% more and more severe droughts in Somalia

% but Forecasts primarily global scale
% thus far, AAs are often limited by data availability
% reparation and water trucking as mayor actions/response -> detailed information necessary



% request of the SRCS -> practically wanted



% --> research aim yoo

% research questions

% structure





% \autocite[20]{gualazziniEWEAEarlyWarning2021} highlights the value of \acrshort{cbm} triangulated data as it ensures the consideration of local perspective in "high-level aggregated data and early action decision-making processes" and decentralises disaster management. 





% % specify, thought that this would be about the methodology
% Qualitative interviews can capture a wide range of local contribution, knowledge and information but are very labour intensive, often limited in time and hence not feasible on larger scale for monitoring purposes.



% benefits
% Providing a technical solution that enables mutual contact between communities and the Somalia Red Crescent Society (SRCS) in a simple and accessible way would make it possible to collect this most helpful information. By ensuring that sovereignty over the collected data rests with the community, the decision-making power and benefits remain with those concerned. Therefore, in the development and application of an Early Action Protocol (EAP) in the context of Forecast based Action (FbA) in Somalia, the technical aspect in particular holds great potential to benefit numerous communities over the long term.

% While there is a comparatively good basis with regard to the general availability of data on social, economic and natural information and conditions, as well as their spatial and qualitative characteristics, the timeliness of this information varies greatly. Especially in the context of Anticipatory Action, time is of the essence and a swift action based on up-to-date data is crucial. Gathering and incorporating local knowledge through manual surveys by a central organisation can be of great value, as shown above, but is time-consuming and slow.






%----------------------------------------------------------------------------------------
%	SECTION 3 Relevance and Importance --> justification
%----------------------------------------------------------------------------------------
% motivation, relation, insights, reasoning
% It’s essential to share your motivation for doing this research, as well as how it relates to existing work on your topic. Further, you should also mention what new insights you expect it will contribute.

% Start by giving a brief overview of the current state of research. You should definitely cite the most relevant literature, but remember that you will conduct a more in-depth survey of relevant sources in the literature review section, so there’s no need to go too in-depth in the introduction.

% Depending on your field, the importance of your research might focus on its practical application (e.g., in policy or management) or on advancing scholarly understanding of the topic (e.g., by developing theories or adding new empirical data). In many cases, it will do both.

% Ultimately, your introduction should explain how your thesis or dissertation:

% Helps solve a practical or theoretical problem
% Addresses a gap in the literature
% Builds on existing research
% Proposes a new understanding of your topic
% transition: this will help to tackle worsening situation

% state of the art (!!!!!!!!!!)
%technical feasibility: important here? --> make this the state of the research


% current limitations




% thus nice in theory, but not useful in practice
% in regard to drought and water scarcity
% and while there is an extensive body of literature about these topics, the minor details are not of great interest to this work but the general conclusion, that physical, large scale drought or water scarcity indicators do not capture the required level of detail and impact that is needed to operationally act upon. Also, while the complexity of these concepts is due to the level of complexity of the surveyed phenomenon, its application and comparison is hindered. Thus a method to assess local impact, that builds and incorporates these concepts in a practically applicable manner is needed to adequately address this detrimental topic. 


% --> mapping -> refers to the entire initial phase - not only the geographical location but to the gathering of all key features of a water source. 

%----------------------------------------------------------------------------------------
%	SECTION 4 Questions and objectives
%----------------------------------------------------------------------------------------
% Perhaps the most important part of your introduction is your questions and objectives, as it sets up the expectations for the rest of your thesis or dissertation. How you formulate your research questions and research objectives will depend on your discipline, topic, and focus, but you should always clearly state the central aim of your research.

% If your research aims to test hypotheses, you can formulate them here. Your introduction is also a good place for a conceptual framework that suggests relationships between variables.

% “Firstly, there is a need for coordinated institutional responses in addressing matters related to water scarcity.” ([Leal Filho et al., 2022, p. 11](zotero://select/groups/4773535/items/CBFHWKLP)) ([pdf](zotero://open-pdf/groups/4773535/items/CG4XVTAI?page=11&annotation=PVHEGLHC))

% “Secondly, it is important to correlate water availability – including groundwater – with an efficient water use to maximise and safeguard sustainable access for all users.” ([Leal Filho et al., 2022, p. 11](zotero://select/groups/4773535/items/CBFHWKLP)) ([pdf](zotero://open-pdf/groups/4773535/items/CG4XVTAI?page=11&annotation=RVPH3I3J))

% “One of main challenges for adaptation for the coming decade is to extend planned adaptation at the local level and better integrate projected risk of climate change and variability into local autonomous responses.” ([Leal Filho et al., 2022, p. 11](zotero://select/groups/4773535/items/CBFHWKLP)) ([pdf](zotero://open-pdf/groups/4773535/items/CG4XVTAI?page=11&annotation=K4DUGJBX))



% overcome limitations / incorporating recommendations: e.g. increased support/engagement of poeple who actually use the reports (e.g. SRCS officials) 
% fill geographic and information gap ->

% its about drought forecasting and early trigger but at the same time highly local and practical information where and which water sources are good and functioning and which are not. -> highly practical information. Some data exist but (mostly) outdated.
% about getting local knowledge from SRCS Volunteers and their community as well as returning information about the bigger picture
% in order to enhance the quality of data for managing severe droughts in Somaliland. (one short paragraph -> motivation)
% provide number of weather stations in the area






%----------------------------------------------------------------------------------------
%	SECTION 5 Overview of the structure
%----------------------------------------------------------------------------------------
% 1-2 sentences outline, brief summery of each chapter
% To help guide your reader, end your introduction with an outline of the structure of the thesis or dissertation to follow. Share a brief summary of each chapter, clearly showing how each contributes to your central aims. However, be careful to keep this overview concise: 1-2 sentences should be enough.


% fuuuuuuuuuuse
% The aim of this study is therefore to incorporate local knowledge on the availability of water sources into the monitoring of drought impacts in an effort to support triggering and Anticipatory Actions under the Early Action Protocol. For this purpose, already proven methods are combined. First, semi-structured (expert-) interviews will be conducted to generate in-depth local knowledge. Based on these findings on local (pre-)conditions, needs and limitations, a monitoring tool based on a volunteersensing approach will then be conceptualized and subsequently discussed with local decision-makers. A prototype development based on the Social.Water tool is a further possibility. In addition, the question of the equality of local to scientific knowledge will be raised and further influences of such a contribution on social conditions will be investigated \todo{too much?}.




% mention, that the classical structure is not for this work..
% transition: the way forward...
% more or less standard shizzle

% somewhat in the result chapter (-> the how? -> it currently can't)
% possibly merge results with discussion: based on this: https://libguides.usc.edu/writingguide/assignments/casestudy and these studies/guidelines/Principles



% Case Studies. Writing@CSU. Colorado State University; Gerring, John. Case Study Research: Principles and Practices. New York: Cambridge University Press, 2007; Merriam, Sharan B. Qualitative Research and Case Study Applications in Education. Rev. ed. San Francisco, CA: Jossey-Bass, 1998; Miller, Lisa L. “The Use of Case Studies in Law and Social Science Research.” Annual Review of Law and Social Science 14 (2018): TBD; Mills, Albert J., Gabrielle Durepos, and Eiden Wiebe, editors. Encyclopedia of Case Study Research. Thousand Oaks, CA: SAGE Publications, 2010; Putney, LeAnn Grogan. "Case Study." In Encyclopedia of Research Design, Neil J. Salkind, editor. (Thousand Oaks, CA: SAGE Publications, 2010), pp. 116-120; Simons, Helen. Case Study Research in Practice. London: SAGE Publications, 2009; Kratochwill, Thomas R. and Joel R. Levin, editors. Single-Case Research Design and Analysis: New Development for Psychology and Education. Hilldsale, NJ: Lawrence Erlbaum Associates, 1992; Swanborn, Peter G. Case Study Research: What, Why and How? London : SAGE, 2010; Yin, Robert K. Case Study Research: Design and Methods. 6th edition. Los Angeles, CA, SAGE Publications, 2014; Walo, Maree, Adrian Bull, and Helen Breen. “Achieving Economic Benefits at Local Events: A Case Study of a Local Sports Event.” Festival Management and Event Tourism 4 (1996): 95-106.


%----------------------------------------------------------------------------------------
%	Notes:
%----------------------------------------------------------------------------------------


% Checklist: Introduction 0 / 7
% I have introduced my research topic in an engaging way.

% I have provided necessary context to help the reader understand my topic.

% I have clearly specified the focus of my research.

% I have shown the relevance and importance of the dissertation topic.

% I have clearly stated the problem or question that my research addresses.

% I have outlined the specific objectives of the research.

% I have provided an overview of the dissertation’s structure.