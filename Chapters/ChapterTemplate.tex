% Chapter Template

\chapter{Theoretical Background} % Main chapter title

\label{ChapterX} % Change X to a consecutive number; for referencing this chapter elsewhere, use \ref{ChapterX}

%----------------------------------------------------------------------------------------
%	SECTION 1
%----------------------------------------------------------------------------------------


%-----------------------------------
%	SUBSECTION 1
%-----------------------------------
The inclusion of local knowledge in the system of Early Warning and Anticipatory Action can result in many benefits as already mentioned in the end of chapter \ref*{subsec:indicators}. Adapting knowledge and policies to local conditions and people as well as learning from them, strengthening autonomous responses and involving local stakeholders in all stages of the processes are just some of the potential ways to improve implementations \autocite{giordanoIntegrationLocalScientific2013a,idmpDroughtWaterScarcity2022,lackstromBackyardHydroclimatologyCitizen2022,lealfilhoRoleIndigenousKnowledge2022,lealfilhoUnderstandingResponsesClimaterelated2022}. One way to include local knowledge is through Citizen Science, very broadly defined as "public participation in scientific research and knowledge production" \autocite{fraislCitizenScienceEnvironmental2022} .
Historically, the first citizen science project was possibly the Christmas Bird Count run by the National Audubon Society in the USA every year since 1900 \autocite{linkHierarchicalModelRegional2006,silvertownNewDawnCitizen2009}. Since around 2000, the number of publications in regard to Citizen Science has risen substantially and has established itself as a vibrant area of scientific interest \autocite{kirschkeCitizenScienceProjects2022}. As more and new thematic fields joined this area of interest, numerous approaches have been made to define Citizen Science more precisely \autocite{haklayWhatCitizenScience2021}. Over 30 definitions were selected by \autocite{haklayWhatCitizenScience2021} to explore their ambiguity and extend the best practice principles and characteristics of citizen science established by the European Citizen Science Association (ESCA) \autocite{escaTenPrinciplesCitizen2015,escaECSACharacteristicsCitizen2020}. Different political, scientific or societal lenses along with a variety of focal points such as (1) biology, conservation and ecology, (2) geographic data and (3) social sciences and health related issues have all contributed to the concept of Citizen Science \autocite{haklayWhatCitizenScience2021,kirschkeCitizenScienceProjects2022,kullenbergWhatCitizenScience2016}.
The first, natural research and conservation, is the orientation most frequently related to Citizen Science with overlapping concepts to community-based, volunteer and participatory monitoring. It has common interests with the second category of Volunteered Geographic Information (VGI) in topics such as crowdsourcing and data quality whereas the the third category mostly resolves around public engagement with intersections to CS in public participation \autocite{kullenbergWhatCitizenScience2016}. In order to highlight the core of Citizen Science alongside the different disciplinary orientations of the research, different frameworks, guidelines and levels of participation have been designed.\autocite{kirschkeCitizenScienceProjects2022} created a three cluster framework of design principles around \textit{citizen} and \textit{institutional} characteristics, together with their \textit{forms of interaction}. Within these categories \autocite{kirschkeCitizenScienceProjects2022} highlight various qualities and skills such as age, social status, motivation, knowledge and education of the contributing citizens, financial and human resources on the institutional side and the method and density of communication and feedback practices as important parts of interactions. Guidelines and principles further specify, expand and structure these broad topics to make them practically applicable in various contexts \autocite{citizenscience.govBasicStepsYour,escaTenPrinciplesCitizen2015,escaECSACharacteristicsCitizen2020,EUCitizenScience2023,fraislCitizenScienceEnvironmental2022,garciaFindingWhatYou2021,minkmanCitizenScienceWater2015,pocockStrategicFrameworkSupport,skarlatidouWhatVolunteersWant2019}. Citizen science projects can also be differentiated according to how engagement with participants is designed. This is refered to as the \textit{levels of participation} and is commonly structured into four levels. Increasing in participation intensity, \autocite{buckinghamshumGlobalParticipatoryPlatform2012} categorize them into (1) Crowdsourcing, (2) Distributed Intelligence, (3) Participation Science and (4) Extreme Citizen Science. Following this categorization, participants can be (1) 'sensors', (2) 'interpreters', (3) enganged in problem definition and data collection or even (4) part of the analysis. 
Depending on the level of participation and thematic orientation, Citizen Science is related to concepts of classic monitoring practices (1), transdisciplicary reaserch emphasizing engagement of the public along the entire process (2 & 3) and participation involving "groups that are or perceive themselves as being affected by the decision" (3 & 4) \autocites{buckinghamshumGlobalParticipatoryPlatform2012}{conradReviewCitizenScience2011}{minkmanCitizenScienceWater2015}[1]{rennParticipatoryProcessesDesigning2006}. 
Current challenges and limitations in Citizen Science projects are the complex demands in the conceptualization and design process with a wide range of required skills and resources, recruiting participants and sustaining their motivation, data quality and accurracy considerations, biases in collection and analysis as well as privacy regulations \autocite{fraislCitizenScienceEnvironmental2022}. Furthermore, research and CS projects are currently unevenly distributed on a global scale with an overrepresentation of North American countries resulting in less experiences and guidelines for other areas and contexts \autocite{kirschkeCitizenScienceProjects2022}. Nonetheless, numerous studies suggest promising developments and application possibilities addressing all of the above mentioned challenges in design, participants and data related issues \autocite{buckinghamshumGlobalParticipatoryPlatform2012,buddeParticipatorySensingParticipatory2017,escaECSACharacteristicsCitizen2020,fraislCitizenScienceEnvironmental2022,lowryGrowingPainsCrowdsourced2019,pocockStrategicFrameworkSupport,ruttenHowGetKeep2017,weeserCitizenSciencePioneers2018a}. 