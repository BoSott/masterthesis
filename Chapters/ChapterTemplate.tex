% Chapter Template

\chapter{Theoretical Background} % Main chapter title

\label{ChapterX} % Change X to a consecutive number; for referencing this chapter elsewhere, use \ref{ChapterX}

%----------------------------------------------------------------------------------------
%	SECTION 1
%----------------------------------------------------------------------------------------


%-----------------------------------
%	SUBSECTION 1
%-----------------------------------
\section{some other title Drought \& Water security} % irgendwas in Richtung grundlegende Konzepte oder so.. was weiß ich ^^

Water security is a theoretical construct that has emerged in the 21st century to frame the overall water objectives and goals to guide local to global water management and policy development \autocite{sadoffWaterSecurity2020a}. It "links together the web of food, energy, climate, economic growth, and human security challenges that the world economy faces over the next two decades" \autocite[5]{wefBubbleCloseBursting2009}. In more detail, it is about "the availability of an acceptable quantity and quality of water for health, livelihoods, ecosystems and production, coupled with an acceptable level of water-related risks to people, environments and economies."\autocite{greySinkSwimWater2007}.
Water security integrates therefore economic, social and environmental dimensions into an interconnected and complex system of human and natural relations by addressing risks of too much, too little or poor quality water \autocite{vanbeekWaterSecurityPutting2014, mishraWaterSecurityChanging2021}. Due to the focus of this work, emphasis is placed on factors that decrease water security due to too little water availability. Besides other factors, natural disasters such as droughts, and water scarcity are the main drivers for insufficient quantities of water \autocite{caretta2022water}. Water quality and access are briefly addressed in addition to provide a more comprehensive  understanding of water security for the following chapters.

%-----------------------------------
%	SUBSECTION 2.1
%-----------------------------------

\subsection{About Drought}\label{subsec:about_drought}

\todo{TODO: change NDMC sources to knew ones}

Drought as highly complex and severe climate-related multi-hazard has far reaching, cascading and interconnected consequences affecting natural ecosystems, societies and economies \autocite{vereintenationenSpecialReportDrought2021}. Historically, droughts are a recurring feature that can occur in all climates. They can geographically extend over small areas to entire sub-continents and are slow onset events that can persist for a few weeks to several years. These high spatial and temporal variabilities make drought not only challenging to define but due to its slow onset, droughts are often only recognized when they are well advanced \autocite{idmpDroughtWaterScarcity2022,vereintenationenSpecialReportDrought2021}. While some drought conditions over large areas can be associated to some low-frequency changes in atmospheric conditions such as the El Niño, accurate cause identification can be rather challenging on smaller scales and requires many different parameters \autocite{botaiAnalysisDroughtProgression2019, vereintenationenSpecialReportDrought2021}.

%In order to approach this complexity, drought is most often defined from four different perspectives, focussing on different manifestations and stages. These definitions are outlined in the coming sub-chapter \ref*{subsec:drought_definitions}, followed by a section addressing the necessary indicators  currently employed in practice for these definitions.
%Generally, droughts are commonly characterized by deviations or the complete failure of climate and weather systems that drive the hydrological cycle compared to normal conditions\autocite{botaiAnalysisDroughtProgression2019,idmpDroughtWaterScarcity2022,vanloonDroughtHumanmodifiedWorld2016,vereintenationenSpecialReportDrought2021}. A more in depth definition can be found in the sub-chapter \ref*{subsec:drought_definitions}.

%“Many studies characterize drought via three main aspects: (i) intensity, (ii) duration and (iii) spatial coverage (He et al.,2018;Adede et al.,2019;Dai,2011).” ([Balti et al., 2020, p. 3](zotero://select/groups/4773535/items/V9YH9QFQ)) ([pdf](zotero://open-pdf/groups/4773535/items/JC4XTVWE?page=3&annotation=WPTV8SEV))
%instead of the first sentence: This complex concept of drought commonly encompassing .... dimensions of interrelated causes ... 

The concept of drought commonly encompasses multiple temporal, spatial and thematic dimensions. This complex conglomeration of interrelated causes and effects makes definition of \textit{droughts} a fairly multi-layered undertaking \autocite{balintMonitoringDroughtCombined2013}. Several well-known definitions (in this category) are for example from the \autocite{theamericanheritagedictionaryoftheenglishlanguageDrought2022} defining drought as "a long period of abnormally low rainfall, especially one that adversely affects growing or living conditions". \autocite[2]{palmerMeteorologicalDrought1965} defines drought as "a prolonged and abnormal moisture deficiency." or \autocite{vanloonDroughtHumanmodifiedWorld2016} defines droughts simply as "an exceptional lack of water compared to normal conditions". Other drought definitions emphasize its natural and/or human origin, its special characteristics, impact and temporal duration or even understand "drought as a system of causality where the link between causes and effects is random in nature {balintMonitoringDroughtCombined2013, baltiReviewDroughtMonitoring2020, idmpDroughtWaterScarcity2022,loonDroughtAnthropocene2016, wangPropagationDroughtMeteorological2016, wilhiteUnderstandingDroughtPhenomenon1985}. Already in the 1980s, \autocite{wilhiteUnderstandingDroughtPhenomenon1985} found more than 150 published definitions of drought. Besides the categorization into a conceptual or operational category , \autocite{wilhiteUnderstandingDroughtPhenomenon1985} proposed a clustering of these definitions into four types, namely meteorological drought, agricultural drought, hydrological drought and socio-economic drought. This classification is still widespread today \autocite{balintMonitoringDroughtCombined2013, baltiReviewDroughtMonitoring2020,idmpDroughtWaterScarcity2022,vereintenationenSpecialReportDrought2021}.

The conceptual category refers to a general formulation of an idea of drought to understand its concept and identify its boundaries and is often formulated in relative terms \autocite{wilhiteUnderstandingDroughtPhenomenon1985}. Definitions in the operational category try to define how drought functions in terms of its onset, duration, severity and spatial coverage also covering how this can be measured via indices \autocite{balintMonitoringDroughtCombined2013, nationaldroughtmitigationcenterWhatDrought, wilhiteUnderstandingDroughtPhenomenon1985}. With these definitions, the current situation is usually compared to a historical average, which is usually based on a 30-year period, which presupposes the development and continuous measurement of indicators and indices that can be used. \autocite{vereintenationenSpecialReportDrought2021,wilhiteUnderstandingDroughtPhenomenon1985}.

The four types of drought are commonly conceptually defined and brought into practice by operational specifications. They can be understood as different, but complementary stages of the same process and are generally cascading in reason and time but can overlap and are difficult to completely unravel. Figure \todo{TODO:, see https://drought.unl.edu/Education/DroughtIn-depth/TypesofDrought.aspx} shows an overview about the different types, their succession and cascading elements and table \todo{TODO: see RCRC 2020 p.11} displays the four types at a glance.

\missingfigure{This is just a test.}

The \textit{meteorological drought} is usually characterized by the duration and the degree of dryness in comparison to the normal average amount and try to conceptually understand how weather patterns can impact water availability. These definitions are specific for a regions atmospheric conditions, e.g. regions with a year-round precipitations regime such as tropical rainforest need different definitions and thresholds than e.g. climates characterized by seasonal rainfall patterns \autocite{nationaldroughtmitigationcenterTypesDrought}. Operational categorization mostly involves using precipitation, moisture, temperature and wind indicators to determine the onset, severity, and duration of the drought.

\textit{Agricultural drought} definitions establish a connection between different features of meteorological drought with their impacts on agriculture. Soil-moisture, differences between actual and potential evapotranspiration and soil water deficits are some of the operationalized indicators for monitoring this type of drought \autocite{baltiReviewDroughtMonitoring2020,nationaldroughtmitigationcenterTypesDrought,wilhiteUnderstandingDroughtPhenomenon1985}.

The type of \textit{hydrological drought} is associated with the impact of meteorological drought on surface or subsurface water resources such as rivers, lakes, and groundwater. Hydrological drought occurs when these indicators drop below normal levels \autocite{palmerMeteorologicalDrought1965}. The fastest responding indicator of this type of drought is most often the variability of streamflow. The water levels of lakes and groundwater usually lag behind the occurrence of the meteorological or agricultural drought which is why the hydrological drought is often out of phase with the previously mentioned types. The hydrological drought is commonly defined on the basis of watershed or river basin scale \autocite{baltiReviewDroughtMonitoring2020,nationaldroughtmitigationcenterTypesDrought,wilhiteUnderstandingDroughtPhenomenon1985}.

The \textit{socioeconomic drought} differs from the aforementioned types as it can also incorporate features of these types of drought to associate them with the demand and supply of some social or economic good. It therefore relates the impact of all other types of droughts on human population and its various sectors of society such as food security, health, and the economy. It is therefore sometimes also interchangeably used with drought impacts. Operational categorization involves using socioeconomic indicators such as unemployment rates and food prices to assess the severity and duration of the drought \autocite{nationaldroughtmitigationcenterTypesDrought,wilhiteUnderstandingDroughtPhenomenon1985}.

\missingfigure{This is just a test.}

The shown economic, social and environmental impacts of drought in figure \todo{TODO:} depend on the severity of, and the risk to drought. These three concepts of impact, severity and risk are interrelated concepts used to assess and understand the effects of drought on various sectors. 

\todo{TODO: insert definitions of severity, risk and impact}
% definitions of severity, risk and impact --> short! two sentences! not more -> severity is not that important, impact comes later again and risk as well so let's get going.


Thereby, in alignment with the definition of \autocite{vanloonDroughtHumanmodifiedWorld2016} it is the exceptional severity of the water shortage that distinguishes drought from aridity, a ordinarily recurrent or fully dry climate, and from water scarcity as a long-term "supply/demand and natural and/or human-made phenomenon" \autocites[7]{idmpDroughtWaterScarcity2022}{vereintenationenSpecialReportDrought2021, vanClimatologicalRiskDroughts2017}. Water scarcity is described in more detail in the following chapter \ref*{subsec:water_scarcity}.

%----------------------------------------------------------------------------------------
%	SUBSECTION 2.2 Water Scarcity
%----------------------------------------------------------------------------------------


\subsection{Water Scarcity}\label{subsec:water_scarcity}
% human induced water shortage component
% what about water security? “Water Security: A Complex Concept” ([Butte et al., 2022, p. 1](zotero://select/groups/4773535/items/QB97YZ2M)) ([pdf](zotero://open-pdf/groups/4773535/items/Q936I2JN?page=1&annotation=XD3AGTA6))
% and insecurity? “Progress in household water insecurity metrics: a crossdisciplinary approach” ([Jepson et al., 2017, p. 1](zotero://select/groups/4773535/items/HWX5JRS4)) ([pdf](zotero://open-pdf/groups/4773535/items/NHEUUZI9?page=1&annotation=SE7MN8X2))

Water scarcity, as for drought or water security, is defined in many different ways. The sixth IPCC Assessment Report defines water scarcity broadly as "a mismatch between the demand for fresh water and its availability, quantified in physical terms" \autocite[560]{caretta2022water}. Here, social and economic components are outsourced to the broader concept of water security and insecurity, focussing primarily on physical water scarcity \autocite{caretta2022water}. In contrast, the Food and Agricultural Organization of the United Nations defines water scarcity as "a gap between available supply and expressed demand of freshwater in a specified domain, under prevailing institutional arrangements (including both resource ‘pricing’ and retail charging arrangements) and infrastructural conditions" \autocite[5]{faoCopingWaterScarcity2012} further summarizing that water security is "an excess of water demand over available supply" \autocite[6]{faoCopingWaterScarcity2012}. Thus, highlighting the human dimension of this interactive and relative concept of physical and economic water scarcity. Hereby, physical water scarcity refers to a situation in which there is not enough water available in quantitative terms to meet demand whereas economic water scarcity occurs when inadequate infrastructure, institutional or financial capital obstructs access to water resources "even though water in nature is available to meet human demands" \autocites{idmpDroughtWaterScarcity2022}[11]{moldenWaterFoodWater2007}.
Water scarcity and drought are in a complex interrelationship with each other. A short overview about the key differences between water scarcity and drought are given in table \todo{TODO:“Table 1. Characteristics and impacts of water scarcity and drought” ([IDMP, 2022, p. 3](zotero://select/groups/4773535/items/LNSL8VD2)) ([pdf](zotero://open-pdf/groups/4773535/items/JM82W3ZF?page=9&annotation=QNC4A3FG))}. 


\missingfigure{differences water scarcity to drought}
% Distinctions between water scarcity and drought:

% also in Joplin

%“Table 1. Characteristics and impacts of water scarcity and drought Water scarcity Drought Length Long-term to permanent Temporary (weeks to multiyear) Driving forces Demand–supply imbalance, human-driven, and/or natural (overexploitation, pollution). Climate change can impact both supply and demand Natural climate variability which can be modified/amplified by climate change Potential impacts Restricted water availability, environmental degradation, desertification, exacerbated inequalities in access to water resources, potential competition Water shortages, competition, environmental degradation Measures Long-term IWRM to bring supply and demand back into sustainable balance Integrated drought management, including: (1) monitoring and early warning; (2quality) vulnerability and impact assessment; and (3) risk mitigation, preparedness and response Source: adapted from Hohenwallner et al. (2011) DROUGHT AND WATER SCARCITY – DEFINITIONS AND CHARACTERISTICS” ([pdf](zotero://open-pdf/groups/4773535/items/JM82W3ZF?page=9&annotation=E3EQRILA))

Furthermore, potential mutual reinforcements, climate change, increased water use and poor water management can make it sometimes difficult to clearly separate these concepts \autocite{idmpDroughtWaterScarcity2022,lealfilhoUnderstandingResponsesClimaterelated2022,liuWaterScarcityAssessments2017,rcrcFORECASTBASEDFINANCINGEARLY2020}. Nonetheless, following the definition of \autocite{faoCopingWaterScarcity2012} the concept of water scarcity always gives water shortage a human dimension in particular on the demand side. The quality of policies, planning and management on the demand side can be seen as critical to the overall severity of the impact of water scarcity \autocite{idmpDroughtWaterScarcity2022,faoCopingWaterScarcity2012,vereintenationenSpecialReportDrought2021}. The supply side can be influenced by human activities, but it is not a mandatory prerequisite. \autocite{idmpDroughtWaterScarcity2022}. 

Besides the already mentioned water scarcity on the basis of physical quantity and economical factors, water scarcity can also be caused by water of unacceptable quality and lack of access to water services \autocite{faoCopingWaterScarcity2012}. Acknowledging water quality induced water scarcity as an additional factor is only a relatively new development in literature \autocite{liuThreedimensionalWaterScarcity2020} but together with inadequate access highlights further challenges in ensuring water security \autocite{caretta2022water, mishraWaterSecurityChanging2021}.

\subsection{Water Quality \& Access}
% + evtl. human health related water borne diseases and CBS

As could be seen in the previous chapter, besides the quantitative availability of water, its accessibility and quality are crucial. Inadequate water quality can be related to numerous health and environmental issues and can further limit the availability of water for given uses \autocite{rcrcFORECASTBASEDFINANCINGEARLY2020, faoCopingWaterScarcity2012}. Unlike the previous concepts, water quality has mostly fixed indicators by which the condition can be determined but historically, and still today, water quality assessment is primarily carried out in laboratories with preceding water sampling activities. This procedure not only makes the determination of water quality a laborious and costly process, but also places high demands on equipment and personnel, so that it is not sufficient for large-scale rural assessments. \autocite{tariqOpenSourceWater2021,worldmeteorologicalorganizationPlanningWaterqualityMonitoring2013}. While simpler methods for in situ water quality monitoring exist, they are either insufficient or often still need to much investment and knowledge to conduct for widespread and frequent monitoring \autocite{worldmeteorologicalorganizationPlanningWaterqualityMonitoring2013}. Nonetheless, new solutions are being developed to simplify and scale affordable water quality assessments to rural areas e.g. \autocite{ighaloComprehensiveReviewWater2020,tariqOpenSourceWater2021}. While the direct assessment of water quality might be challenging, poor water quality can be linked to other factors. Environmental awareness, poor sanitation and hygiene conditions of people in rural areas were for example considered as major causes for contamination of water at the source \autocite{zamxakaMicrobiologicalPhysicochemicalAssessment2004}.

The definition of water access is again a rather challenging undertaking. The \autocite[254]{worldbankWorldDevelopmentReport1997} defined water access in rural areas by "access implies members of the household do not have to spend a disproportionate part of the day fetching water." While both time and distance still play a crucial role in literature when investigating water access \autocite{cassiviDrinkingWaterAccessibility2019,cassiviEvaluatingSelfreportedMeasures2021,emenikeAccessingSafeDrinking2017}, the term also gained a social component \autocite{emenikeAccessingSafeDrinking2017,mitlinUnaffordableUndrinkable}. \autocite{obeng-odoomAccessWater2012} adds four additional factors namely, affordability, quality, equitable distribution to the definition of water access to fully understand if users have access to water in daily live. \autocite{unitednations/developmentprogrammeDeepeningDemocracyFragmented2002} links these parameters to the access to an improved water source which should provide safe drinking water.
The access to improved water sources is therefore generally considered as crucial in the reaching of water security \autocite{cdcAssessingAccessWater2022}. Proactive measures to drought and water scarcity can not only potentially minimize or even neutralize impacts and are considerably more cost-efficient, early warning and anticipatory actions for drought and water scarcity impacts become ever more important \autocite{faoandun-waterProgressLevelWater2021,idmpDroughtWaterScarcity2022,worldbankHighDryClimate2016}.

%----------------------------------------------------------------------------------------
%	SECTION 4 FbF, EAP, AA & Early Warning
%----------------------------------------------------------------------------------------

\subsection{indicators with risk/vulnerability, indices and impact} % yeah well... choose another title.

Indicators and Indices are often used to translate complex matters into easier to explain numbers and scales that can be measured, tracked and reasonably compared \autocite{blauveltSystematizingEnvironmentalIndicators2014,williamsUsingIndicatorsExplain2017}. This can range from capturing simple measurements to complex and detailed issues that can not only depict ecological conditions but its interactions with societies \autocite{blauveltSystematizingEnvironmentalIndicators2014,mishraWaterSecurityChanging2021}. Indicators and Indices can thus establish a clear and common understanding of a concept or parts of it in a quantifiable and more objective way.
Here, an indicator is understood as a measurable parameter that provides information on the state or trend of an issue or problem. It can be a physical, chemical, biological, or socio-economic variable, such as temperature, soil moisture or streamflow and can be measured locally or remotely. An Index is a composite measure that aggregates multiple indicators into a single value or score \autocite{unitednationsuniversityTooManyIndicators2017,williamsUsingIndicatorsExplain2017, svobodaHandbookDroughtIndicators2016}. Indices are commonly developed on regional or national level to account for the specific circumstances \autocite{unitednationsuniversityTooManyIndicators2017}. This case specification, together with different measurement and aggregation methods, partial inconsistency of definitions and differently focussed objectives on qualitative, quantitative, risk or impact scenarios can constrain their practical application and intercomparability \autocite{svobodaHandbookDroughtIndicators2016,unitednationsuniversityTooManyIndicators2017}. 
Since there is no one definition of drought, water scarcity or security, there is no one best solution to the choice between the many indicators and indices for either of those.



% all indicators are proxies -> good information -> include that somewhere
%“There exist many physical indicators of a drought that are monitored by scientists and governments to track the development of drought impacts. The complex and insidious nature of drought means that all these indicators are proxies to understand the impacts that dry conditions are having on an area. Our main suggestion here would be to examine the World Meteorological Organisation Handbook of Drought Indicators and Indices (which are classified in a traffic-light method of ease of use) identifies which indices could be available and appropriate for the context.” ([RCRC, 2020, p. 15](zotero://select/groups/4773535/items/UESIQTRJ)) ([pdf](zotero://open-pdf/groups/4773535/items/P5JPVZ97?page=15&annotation=XR5BI5Y7))


% possibly shorten this.. naming all these indices might be a little overkill, though naming none is also not feasible.
Precipitation, evapotranspiration, soil moisture, lake and groundwater levels, streamflow and vegetation water stress are among the most prominent drought indicators \autocite{europeandroughtobservatoryDroughtIndicators2017}. In order to adequately account for the different drought stages different drought indices, that aggregate these and other indicators, are applied. Among the most prominent meteorological drought indics are the Standardized Precipitation Index \textit{SPI} together with its extension the Standardized Precipitation-Evapotranspiration Index \textit{SPEI} \autocite{europeandroughtobservatoryDroughtIndicators2017,ncarStandardizedPrecipitationEvapotranspiration,ncarStandardizedPrecipitationIndex}. Agricultural drought indices like the Soil Moisture Anomaly \textit{SMA} or the Anomaly of Vegetation Condition \textit{FAPAR Anomaly} are based on soil moisture indicators and absorbed radiation fractions, respectively. By quantifying water flow volumina, the Low Flow Index \textit{LFI} belongs to the hydrological drought indices \autocite{europeandroughtobservatoryDroughtIndicators2017, svobodaHandbookDroughtIndicators2016}. In addition to these and other types of indices, such as Combined Drought Indices, the \textit{Handbook for Drought Indicators and Indices} lists over 50 drought indicators and indices. For further and more in-depth information, please refer to the interactive website of the \acrfull{IDMP} launched by the \acrfull{wmo} and \acrfull*{gwp} \autocite{idmpIndicatorsIndicesIntegrated2021}. 

All of these drought indices give a good impression about the physical side of climate anomalies, but none of the above mentioned indices link those climate anomalies to socioeconomic vulnerabilities \autocite{enenkelWhyPredictClimate2020}. \autocite{mishraWaterSecurityChanging2021} argue, that the framing of water security challenges extends beyond singular indicators. \autocite{lackstromBackyardHydroclimatologyCitizen2022} argue further, that assessments that only consider physical factors overlook the broader impact of drought on social, economic, and ecological systems.
The simple but widely used Falkenmark Indicator (Falkenmark et al. 1989) incorporates human factors by calculating a ratio between the given amount of water and the number of people living within that domain. By further categorizing this ratio to a level of water scarcity, the Falkenmark Indicator indicates the supply sides effects of water scarcity but variabilities, demand and socioeconomic factors are not represented. More dedicated indices like the \acrfull*{iwmi} Indicator and the \acrfull*{wpi} as well as other indices measuring water security give a more extensive representation of the overall situation \autocite{arreguin-cortesMunicipalLevelWater2019,liuWaterScarcityAssessments2017}. The \acrshort{wpi} for example represents the weighted average of five pre-standardized components namely, water availability, access, capacity, use and environment \autocite{sullivanWaterPovertyIndex2003}.

Determining the right set of indicators and indices for a given region to e.g. assess hazard severity depends on the objective and available data and is often a balancing act between many factors and circumstances \autocite{svobodaHandbookDroughtIndicators2016}. Besides the pure description of what certain natural or social circumstances \textit{are}, there is an growing interest to understand what these conditions will \textit{do} \autocite{boultDroughtImpactbasedForecasting2022, lackstromBackyardHydroclimatologyCitizen2022}.

The effects of these conditions on the ground are most often called the \textit{impact} of a certain weather phenomenon or climate development such as a drought hazard. Impacts can be direct or indirect and a generally difficult to quantify economically \autocite{vereintenationenSpecialReportDrought2021}. The level of impact is commonly determined based on the severity of the hazard, the exposure of the investigated elements and their respective vulnerabilities \autocite{harrowsmithFutureForecastImpact2020,svobodaHandbookDroughtIndicators2016,vereintenationenSpecialReportDrought2021}.
This concept is generally expressed by the risk equation

        \[Risk = f(Hazard, Exposure, Vulnerability)\]

    where

        \[Vulnerability = f(Level of Coping Capacity, Level of Adaptive Capacity)\]

\autocite{boultDroughtImpactbasedForecasting2022,harrowsmithFutureForecastImpact2020,vereintenationenSpecialReportDrought2021}. Drought hazard can be evaluated and described by the above mentioned indicators and indices with difficulties lying in the contextualization and setting of the threshold levels to separate between fluctuations within the normal range and extreme events. Exposure is commonly defined as social, economic, cultural or natural assets, services or resources in places that could be adversely affected by a hazard \autocite{ipccClimateChange20142014}. Exposed elements can be more ore less vulnerable to the hazard. Vulnerability conditions are determined by the sensitivity or susceptibility of a system, community or individual to physical, social, economic or environmental factors or processes \autocite{ipccClimateChange20142014}. These conditions are often further described as the level of coping and adaptive capacities. Coping capacities refer to available skills and resources of systems, organizations or individuals to address, manage and overcome unfavourable circumstances \autocite{ipccGlossaryTerms2012}. In the same manner, adaptive capacities relate to preparation, reduction and moderation of those impacts.

The establishement of a functional relationship between the hazard, exposure and vulnerability to its impact can be rather difficult for numerous reasons and is further discussed by \autocite{boultDroughtImpactbasedForecasting2022} for interested readers. Moreover, all these factors change over time, so that the quality of the calculations depends strongly on the timeliness of the data basis \autocite{harrowsmithFutureForecastImpact2020}. 

Relatively recent approaches argue for numerous benefits and reasons for greater inclusion of local knowledge and community integration in these approaches \autocite{balehegnIndigenousWeatherClimate2019,dubeFrameworkIntegrationTraditional2016,ebhuomaFrameworkIntegratingScientific2020,giordanoIntegrationLocalScientific2013a,greyIntegratingLocalIndigenous2020,hermansExploringIntegrationLocal2022a,mercerCultureDisasterRisk2012,mutasaKnowledgeApartheidDisaster2015,nyetanyaneIntegrationIndigenousKnowledge2020,nyongValueIndigenousKnowledge2007}. Another emerging area in scientific interest is the gender inequality of drought impacts \autocite{acharyaWhenRiverTalks2019,fanningDroughtDisplacementLivelihoods2018,hiwasakiLocalIndigenousKnowledge2015,mustafaGenderingFloodEarly2015,sachsRoutledgeHandbookGender2020,saniGenderOtherVulnerabilities2022}. Although these topics are of great interest, they fall largely outside the scope of this particular work.

An understanding of the severity of droughts and their current impacts enables targeted responses, as well as to allow for the development of future predictions based on current conditions. In this context, recent efforts have increasingly emphasized proactive and forward-looking measures in disaster relief initiatives. The forthcoming chapter will explore this relatively recent shift in approach and its implications for improving drought management strategies.

\section{FbF, EAP, AA & Early Warning + trigger}
% one possible solution to prevent impact
% FbF
% IFRC & RCRC
% EAPs
% Early Warning/Actions &  Anticipatory Actions
% triggered by forecast

Traditionally, disaster management efforts have primarily focused on long-term preparedness or post-disaster response, thus only providing assistance and relief to affected communities after a disaster has occurred (policy overview, hyogo framework [UNISDR], coughland et al 2015). The lack of standardized procedures for forecast-based actions led to disaster warnings often going unheard \autocite{kolenImpactsStormXynthia2013}. In the context of increasing frequency and severity of natural disasters, coupled with the impacts of climate change, the need for a more proactive approach that can reduce the impact of disasters on vulnerable communities became apparent \autocite{coughlandeperezForecastbasedFinancingApproach2015,trisosAfrica2022}. Nonetheless, financial resources were for the time being strongly directed towards post-disaster response and incentives to invest in new and complex scientific developments including relatively high uncertainties were limited \autocite{coughlandeperezActionbasedFloodForecasting2016}. This changed with the development and successful integration of several new forecast-based financing systems that utilized the opportunity gap between a forecast and the disaster to successfully reduced corresponding impact. Based on this, to "substantially increase the availability of and access to multi-hazard early warning systems and disaster risk information and assessments to people by 2030" became one of seven global targets of the Sendai for Disaster Risk Reduction 2015-2030 \autocites{coughlandeperezActionbasedFloodForecasting2016}[12]{undrrSendaiFrameworkDisaster}. Today, large institutions have now specialized sections for the financing of Early Actions such as the Climate Risk and Early Warning Systems Initiative \textit{(CREWS)} and the Global Risk Financing Facility \textit{(GRiF)} to support and backup Early Actions \textit{(EAs)} \autocite{crewsClimateRiskEarly,GlobalRiskFinancing}. Forecast-based Financing \textit{(FbF)} has thus emerged as a promising approach to disaster management that enables proactive, timely, and cost-effective responses to disasters \autocite{coughlandeperezForecastbasedFinancingApproach2015} (TODO: add “FORECAST-BASED FINANCING An innovative approach” ([pdf](zotero://open-pdf/groups/4773535/items/3C2CE7BS?page=1&annotation=UKWEKCTA))).

The IFRC together with the Red Cross Climate Center \textit{(RCCC)} and German Red Cross \textit{(GRC)} have developed and improved the FbF programme to fund EAs since 2007 \autocite{ifrcForecastbasedFinancingNew2019}. 

\missingfigure[options]{“Figure 1 - FbF Diagram” ([RCRC, 2020, p. 3](zotero://select/groups/4773535/items/UESIQTRJ)) ([pdf](zotero://open-pdf/groups/4773535/items/P5JPVZ97?page=3&annotation=W3UC7H26))}

Following \autocite{coughlandeperezForecastbasedFinancingApproach2015, coughlandeperezActionbasedFloodForecasting2016} the structure of FbF can be distilled down to:
    “When forecast states that an agreed-upon probability threshold is exceeded for a hazard of a designated magnitude, then an action with an associated cost must be taken that has a desired effect and is carried out by a designated organisation.” \autocite[2]{coughlandeperezActionbasedFloodForecasting2016}.
Thus, the FbF approach involves three main components (1) triggering (2) pre-defined EAs and securing a (3) financing mechanism in advance (compare \ref{TODO: figure fbf}) (TODO: “Forecast-based Financing A new era for the humanitarian system” ([pdf](zotero://open-pdf/groups/4773535/items/KQZXSWVN?page=1&annotation=3BW2ZYST))). These components are summarized in an Early Action Protocol \textit{(EAP)} (TODO: cite policy overview “These three components are summarized in Early Action Protocols (EAPs).” ([“Forecast-based financing: A policy overview”, p. 2](zotero://select/groups/4773535/items/35XBEGJ7)) ([pdf](zotero://open-pdf/groups/4773535/items/8YZAQB5L?page=2&annotation=58UQZK6T))). 

\missingfigure{“FBF has three” ([pdf](zotero://open-pdf/groups/4773535/items/K89MIG2V?page=3&annotation=PF946AET))}