% Chapter Template

\chapter{Theoretical Background} % Main chapter title

\label{ChapterX} % Change X to a consecutive number; for referencing this chapter elsewhere, use \ref{ChapterX}

%----------------------------------------------------------------------------------------
%	SECTION 1
%----------------------------------------------------------------------------------------

TODO:

rework methods -> implement prc and so on
data analysis protocol (see joplin) or add to methods (?)

discussion chapter

introduction

conclusion

outlook

abstract (at least a first draft)

Zitationen fixen! noch vor den figures -> das Zeug gibt nur noch mehr Probleme

figures

tables!! -> vor allem der project analysis table ist wichtig! + data analysis table

Karte für case study area - ggf. auch für wasser resources (jedoch schwierig welchen Datensatz..)

Eigene Result Abbildungen nochmals überprüfen und Einbinden

alles überlesen -> nicht zu doll eskalieren beim Neuschreiben..

Appendix:
transcripts
results as png in large
questionnaires and interview guidelines (also for NYSS)

fix abbreviations (könnte sich auch zusammen mit den Zitationen erledigen.)

nette Email ans Prüfungsamt, Zipf und Lautenbach verfassen, DEN ANHANG NICHT VERGESSEN und abschicken.

Think about the title.. yoooo

Done.


wenn irgendmöglich.. so ein wenig tags für OSM raussuchen. Da kommt eh nix weiter bei rum.. #biasisreal





























% research objectives
1. To conduct a comprehensive review of existing literature and guidelines related to the design and implementation of \acrlong{cs} programmes, and try to align and apply these principles to the research aim and overall case study context.

2. To assess the feasibility of the \acrlong{cs} approach in the given context by identifying potential challenges and opportunities for successful implementation, and to propose recommendations for addressing these challenges.

3. if feasible, develop a replicable and adaptable framework for community-based participatory water source mapping and monitoring in the context of \acrlong{fbf}, based on the principles and recommendations identified in objectives 1 and 2.

4. To apply the adapted and developed frameworks in order to create a roadmap for the implementation of the proposed project, including specific products, actions, and stakeholders involved.

% research question
1. What specific guidelines and best practices exist for the design and implementation of community-based participatory water source mapping and monitoring programmes in resource-limited and water-scarce settings, and how can they be applied to the thematic direction of \acrlong{fbf}? % none -> wider angle -> FbF

2. Based on the identified frameworks and principles, which specific combinations of frameworks and guidelines are best suited for developing a replicable and adaptable community-based participatory water source mapping and monitoring design in resource-limited and water-scarce settings, while also ensuring feasibility for \acrlong{fbf}?

3. In the specific context of this case study, how can the developed framework be applied to create a tailored roadmap for the implementation of a community-based participatory water source mapping and monitoring project, including specific products, activities, and stakeholders involved in the project?



% ideas to the second questions

Based on a determined set of the identified frameworks and guidelines, how can a replicable and adaptable community-based participatory mapping and monitoring approach for water sources in resource-limited and water-scarce settings be designed for FbF application?

how can a replicable and adaptable community-based participatory water source mapping and monitoring design in a resource-limited and water-scarce setting for the application of FbF be conceptualised?


What are the key frameworks and recommendations identified in objectives 1 and 2 for designing and implementing a community-based participatory water source mapping and monitoring program in the context of FbF, and how can these be adapted to develop a replicable and adaptable framework for future implementation?

