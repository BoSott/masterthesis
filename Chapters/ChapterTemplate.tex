% Chapter Template

\chapter{Theoretical Background} % Main chapter title

\label{ChapterX} % Change X to a consecutive number; for referencing this chapter elsewhere, use \ref{ChapterX}

%----------------------------------------------------------------------------------------
%	SECTION 1
%----------------------------------------------------------------------------------------

TODO:

rework methods -> implement prc and so on
data analysis protocol (see joplin) or add to methods (?)

% naja check. Da steht zumindest was..
discussion chapter

% check
introduction - check

conclusion

outlook

abstract (at least a first draft)

% check
Zitationen fixen! noch vor den figures -> das Zeug gibt nur noch mehr Probleme

Zitation fixen -> nur letztes Wort von Nachnamen genannt

Zitationen im Text anpassen mit oder ohne Klammern etc... shizzle ehh

figures

tables!! -> vor allem der project analysis table ist wichtig! + data analysis table

map of Somaliland!!

Karte für case study area - ggf. auch für wasser resources (jedoch schwierig welchen Datensatz..)

Eigene Result Abbildungen nochmals überprüfen und Einbinden

alles überlesen -> nicht zu doll eskalieren beim Neuschreiben..

Appendix:
transcripts
results as png in large
questionnaires and interview guidelines (also for NYSS)

fix abbreviations (könnte sich auch zusammen mit den Zitationen erledigen.)

nette Email ans Prüfungsamt, Zipf und Lautenbach verfassen, DEN ANHANG NICHT VERGESSEN und abschicken.

Think about the title.. yoooo

Done.


wenn irgendmöglich.. so ein wenig tags für OSM raussuchen. Da kommt eh nix weiter bei rum.. #biasisreal








On the one hand, citizens can help to fill data gaps of categorized measurements such as simple assessments of dry-to-wet conditions which correspond to the above mentioned technical drought indicators \autocite{lackstromBackyardHydroclimatologyCitizen2022}. On the other hand, citizens can contribute their local knowledge which can potentially draw on years of experience and encompass a wide range of locally important aspects \autocite{butteFrameworkWaterSecurity2022,koehlerCitizenParticipationCollaborative2008,njambi-szlapkaIntegratingCommunityVoices}.




















% research objectives
1. To conduct a comprehensive review of existing literature and guidelines related to the design and implementation of \acrlong{cs} programmes, and try to align and apply these principles to the research aim and overall case study context.

2. To assess the feasibility of the \acrlong{cs} approach in the given context by identifying potential challenges and opportunities for successful implementation, and to propose recommendations for addressing these challenges.

3. if feasible, develop a replicable and adaptable framework for community-based participatory water source mapping and monitoring in the context of \acrlong{fbf}, based on the principles and recommendations identified in objectives 1 and 2.

4. To apply the adapted and developed frameworks in order to create a roadmap for the implementation of the proposed project, including specific products, actions, and stakeholders involved.

% research question
1. What specific guidelines and best practices exist for the design and implementation of community-based participatory water source mapping and monitoring programmes in resource-limited and water-scarce settings, and how can they be applied to the thematic direction of \acrlong{fbf}? % none -> wider angle -> FbF

2. Based on the identified frameworks and principles, which specific combinations of frameworks and guidelines are best suited for developing a replicable and adaptable community-based participatory water source mapping and monitoring design in resource-limited and water-scarce settings, while also ensuring feasibility for \acrlong{fbf}?

3. In the specific context of this case study, how can the developed framework be applied to create a tailored roadmap for the implementation of a community-based participatory water source mapping and monitoring project, including specific products, activities, and stakeholders involved in the project?



% ideas to the second questions

Based on a determined set of the identified frameworks and guidelines, how can a replicable and adaptable community-based participatory mapping and monitoring approach for water sources in resource-limited and water-scarce settings be designed for FbF application?

how can a replicable and adaptable community-based participatory water source mapping and monitoring design in a resource-limited and water-scarce setting for the application of FbF be conceptualised?


What are the key frameworks and recommendations identified in objectives 1 and 2 for designing and implementing a community-based participatory water source mapping and monitoring program in the context of FbF, and how can these be adapted to develop a replicable and adaptable framework for future implementation?



In order to meet this challenge, the \acrlong{rcrc} Movement together with the \acrlong*{rcrccc} started the \acrfull*{fbf} programme in 2007 to facilitate \acrlongpl{aa} instead of post-disaster reactions \autocite{ifrcForecastbasedFinancingNew2019}. Together with their local partners, the \acrfull*{ifrc} is working on establishing so called Early Action Protocols (EAPs) to ensure better organization and coordination of Anticipatory Actions in the face of an incoming hazard. These actions are based on a predefined interplay of forecast, trigger and financing mechanisms to ensure rapid, scientific based responses.\newline

Somaliland, being no exception to the above mentioned climatic trend, is characterized by droughts with far reaching impacts on ecological, economic, and social aspects \autocite{abdulkadirAssessmentDroughtRecurrence2017}. Defined by a semi-arid, four-season climate with two extensive dry seasons and an economic backbone of pastoralism and rain-fed agriculture, water accessibility is of key importance in Somaliland \autocite{abdulkadirAssessmentDroughtRecurrence2017,petrucciLandscapeLandformsNorthern2022,republicofsomalilandSomalilandCountryProfile2021}.

% impact of drought on the community
% --> impact forecasts
% --> mapping & monitoring of water source type Berkad

https://www.unwater.org/our-work/integrated-monitoring-initiative-sdg-6




“Countries in which less than 50\% of the population uses improved drinking water sources are all located in sub-Saharan Africa and Oceania 91-100\% 76-90\% 50-75\% <50\% insufficient data or not applicable Proportion of the population using improved drinking water sources in 2015 ■ 91–100\% ■ 76–90\% ■ 50–75\% ■ <50\% ■ INSUFFICIENT DATA OR NOT APPLICABLE” ([World Health Organization, 2016, p. 15](zotero://select/groups/4773535/items/KVAKZ9ZT)) ([pdf](zotero://open-pdf/groups/4773535/items/4STYK52H?page=14\&annotation=FBURDS4T))


In the Horn of Africa, much of the population still has no access to improved drinking water sources or in the case of Somalia,  


Nonetheless, direct contributions and communication from and with volunteers or community members remain a challenge in the joint management of hazards and risks. The tasks are numerous and need to take into consideration different aspects, ranging from cultural differences to different background knowledge and technical capabilities and capacities.

“Intervening early to respond to spikes in need – i.e. before negative coping strategies are employed - can deliver significant gains and should be prioritized.” ([USAID, 2018, p. 6](zotero://select/groups/4773535/items/LGRWAU43)) ([pdf](zotero://open-pdf/groups/4773535/items/MBXSCVWR?page=6&annotation=C47BGB9V))




Besides the further development of more fine grained technical solutions, the integration of local citizens is another way forward. Engaging local citizens and communities and giving them an active voice in defining and co-producing \acrshortpl{aa} and knowledge can be of multiple benefit to communities and enrich the data generated \autocite{somaliredcrescentsocietyFeasibilityStudyPotential2022, njambi-szlapkaIntegratingCommunityVoices}. 



On the one hand, citizens can help to fill data gaps of categorized measurements such as simple assessments of dry-to-wet conditions which correspond to the above mentioned technical drought indicators \autocite{lackstromBackyardHydroclimatologyCitizen2022}. On the other hand, citizens can contribute their local knowledge which can potentially draw on years of experience and encompass a wide range of locally important aspects \autocite{butteFrameworkWaterSecurity2022,koehlerCitizenParticipationCollaborative2008,njambi-szlapkaIntegratingCommunityVoices}. The \acrshort{ifrc} states, that the "community engagement and accountability (CEA) is essential […] to build acceptance and trust” for effective and sustainable outcomes \autocite{ifrcCommunityEngagementAccountability}.\newline



In the last two decades, \acrlong{cs} has become a vibrant area of scientific interest covering various aspects in many different contexts \autocite{kirschkeCitizenScienceProjects2022,kullenbergWhatCitizenScience2016}. Relatively recent developments in \acrlong{cbm} and \acrlong{mcs} now make it possible for a large number of citizens to contribute to scientific, social and environmental endeavours with just a simple phone \autocite{butteFrameworkWaterSecurity2022}. This engagement of the general public can have multiple benefits for a wide variety of aspects. Scientific processes of e.g. linking climate variability to local water security can be informed, the public's education and awareness about specific topics can be raised, and decision-making and overall management can be enhanced, if the project is embedded in these procedures \autocite{huangManagementDrinkingWater2020,kirschkeCitizenScienceProjects2022,minkmanCitizenScienceWater2015}.


Furthermore, \acrshort{cs} projects have demonstrated their ability to gather data and fill gaps particularly in formerly data sparse regions in an effective and cost-efficient manner \autocite{butteFrameworkWaterSecurity2022,lackstromBackyardHydroclimatologyCitizen2022,weeserCitizenSciencePioneers2018a}. However, currently \acrshort{cs} projects and studies are primarily located in North America, Europe and Australia \autocite{kirschkeCitizenScienceProjects2022, koehlerCitizenParticipationCollaborative2008, livinglakescanadaElevatingCommunityBased2018}. In the field of environment and water monitoring, these projects are mainly concerned about  \autocite{kirschkeCitizenScienceProjects2022}. Social.Water, CoCoRaHS and \autocite{speirSolutionsCurrentChallenges2022}'s study are examples of those environmental data collection and drought monitoring implementations focussing on monitoring river, lake, groundwater and precipitation levels. However, these approaches all require internet access and more technical equipment, making them unfeasible for low-income conditions \autocite{fienenSocialWaterCrowdsourcing2012a,lackstromBackyardHydroclimatologyCitizen2022,lowryGrowingPainsCrowdsourced2019}.\newline











