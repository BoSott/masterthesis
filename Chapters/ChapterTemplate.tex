% Chapter Template

\chapter{Theoretical Background} % Main chapter title

\label{ChapterX} % Change X to a consecutive number; for referencing this chapter elsewhere, use \ref{ChapterX}

%----------------------------------------------------------------------------------------
%	SECTION 1
%----------------------------------------------------------------------------------------

TODO:

rework methods -> implement prc and so on
data analysis protocol (see joplin) or add to methods (?)

% naja check. Da steht zumindest was..
discussion chapter

% check
introduction - check

conclusion

outlook

abstract (at least a first draft)

% check
Zitationen fixen! noch vor den figures -> das Zeug gibt nur noch mehr Probleme

Zitation fixen -> nur letztes Wort von Nachnamen genannt

Zitationen im Text anpassen mit oder ohne Klammern etc... shizzle ehh

figures

tables!! -> vor allem der project analysis table ist wichtig! + data analysis table

map of Somaliland!!

Karte für case study area - ggf. auch für wasser resources (jedoch schwierig welchen Datensatz..)

Eigene Result Abbildungen nochmals überprüfen und Einbinden

alles überlesen -> nicht zu doll eskalieren beim Neuschreiben..

Appendix:
transcripts
results as png in large
questionnaires and interview guidelines (also for NYSS)

fix abbreviations (könnte sich auch zusammen mit den Zitationen erledigen.)

nette Email ans Prüfungsamt, Zipf und Lautenbach verfassen, DEN ANHANG NICHT VERGESSEN und abschicken.

Think about the title.. yoooo

Done.


wenn irgendmöglich.. so ein wenig tags für OSM raussuchen. Da kommt eh nix weiter bei rum.. #biasisreal








On the one hand, citizens can help to fill data gaps of categorized measurements such as simple assessments of dry-to-wet conditions which correspond to the above mentioned technical drought indicators \autocite{lackstromBackyardHydroclimatologyCitizen2022}. On the other hand, citizens can contribute their local knowledge which can potentially draw on years of experience and encompass a wide range of locally important aspects \autocite{butteFrameworkWaterSecurity2022,koehlerCitizenParticipationCollaborative2008,njambi-szlapkaIntegratingCommunityVoices}.




















% research objectives
1. To conduct a comprehensive review of existing literature and guidelines related to the design and implementation of \acrlong{cs} programmes, and try to align and apply these principles to the research aim and overall case study context.

2. To assess the feasibility of the \acrlong{cs} approach in the given context by identifying potential challenges and opportunities for successful implementation, and to propose recommendations for addressing these challenges.

3. if feasible, develop a replicable and adaptable framework for community-based participatory water source mapping and monitoring in the context of \acrlong{fbf}, based on the principles and recommendations identified in objectives 1 and 2.

4. To apply the adapted and developed frameworks in order to create a roadmap for the implementation of the proposed project, including specific products, actions, and stakeholders involved.

% research question
1. What specific guidelines and best practices exist for the design and implementation of community-based participatory water source mapping and monitoring programmes in resource-limited and water-scarce settings, and how can they be applied to the thematic direction of \acrlong{fbf}? % none -> wider angle -> FbF

2. Based on the identified frameworks and principles, which specific combinations of frameworks and guidelines are best suited for developing a replicable and adaptable community-based participatory water source mapping and monitoring design in resource-limited and water-scarce settings, while also ensuring feasibility for \acrlong{fbf}?

3. In the specific context of this case study, how can the developed framework be applied to create a tailored roadmap for the implementation of a community-based participatory water source mapping and monitoring project, including specific products, activities, and stakeholders involved in the project?



% ideas to the second questions

Based on a determined set of the identified frameworks and guidelines, how can a replicable and adaptable community-based participatory mapping and monitoring approach for water sources in resource-limited and water-scarce settings be designed for FbF application?

how can a replicable and adaptable community-based participatory water source mapping and monitoring design in a resource-limited and water-scarce setting for the application of FbF be conceptualised?


What are the key frameworks and recommendations identified in objectives 1 and 2 for designing and implementing a community-based participatory water source mapping and monitoring program in the context of FbF, and how can these be adapted to develop a replicable and adaptable framework for future implementation?



In order to meet this challenge, the \acrlong{rcrc} Movement together with the \acrlong*{rcrccc} started the \acrfull*{fbf} programme in 2007 to facilitate \acrlongpl{aa} instead of post-disaster reactions \autocite{ifrcForecastbasedFinancingNew2019}. Together with their local partners, the \acrfull*{ifrc} is working on establishing so called Early Action Protocols (EAPs) to ensure better organization and coordination of Anticipatory Actions in the face of an incoming hazard. These actions are based on a predefined interplay of forecast, trigger and financing mechanisms to ensure rapid, scientific based responses.\newline

Somaliland, being no exception to the above mentioned climatic trend, is characterized by droughts with far reaching impacts on ecological, economic, and social aspects \autocite{abdulkadirAssessmentDroughtRecurrence2017}. Defined by a semi-arid, four-season climate with two extensive dry seasons and an economic backbone of pastoralism and rain-fed agriculture, water accessibility is of key importance in Somaliland \autocite{abdulkadirAssessmentDroughtRecurrence2017,petrucciLandscapeLandformsNorthern2022,republicofsomalilandSomalilandCountryProfile2021}.

% impact of drought on the community
% --> impact forecasts
% --> mapping & monitoring of water source type Berkad

https://www.unwater.org/our-work/integrated-monitoring-initiative-sdg-6




“Countries in which less than 50\% of the population uses improved drinking water sources are all located in sub-Saharan Africa and Oceania 91-100\% 76-90\% 50-75\% <50\% insufficient data or not applicable Proportion of the population using improved drinking water sources in 2015 ■ 91–100\% ■ 76–90\% ■ 50–75\% ■ <50\% ■ INSUFFICIENT DATA OR NOT APPLICABLE” ([World Health Organization, 2016, p. 15](zotero://select/groups/4773535/items/KVAKZ9ZT)) ([pdf](zotero://open-pdf/groups/4773535/items/4STYK52H?page=14\&annotation=FBURDS4T))


In the Horn of Africa, much of the population still has no access to improved drinking water sources or in the case of Somalia,  


Nonetheless, direct contributions and communication from and with volunteers or community members remain a challenge in the joint management of hazards and risks. The tasks are numerous and need to take into consideration different aspects, ranging from cultural differences to different background knowledge and technical capabilities and capacities.

“Intervening early to respond to spikes in need – i.e. before negative coping strategies are employed - can deliver significant gains and should be prioritized.” ([USAID, 2018, p. 6](zotero://select/groups/4773535/items/LGRWAU43)) ([pdf](zotero://open-pdf/groups/4773535/items/MBXSCVWR?page=6&annotation=C47BGB9V))




Besides the further development of more fine grained technical solutions, the integration of local citizens is another way forward. Engaging local citizens and communities and giving them an active voice in defining and co-producing \acrshortpl{aa} and knowledge can be of multiple benefit to communities and enrich the data generated \autocite{somaliredcrescentsocietyFeasibilityStudyPotential2022, njambi-szlapkaIntegratingCommunityVoices}. 



On the one hand, citizens can help to fill data gaps of categorized measurements such as simple assessments of dry-to-wet conditions which correspond to the above mentioned technical drought indicators \autocite{lackstromBackyardHydroclimatologyCitizen2022}. On the other hand, citizens can contribute their local knowledge which can potentially draw on years of experience and encompass a wide range of locally important aspects \autocite{butteFrameworkWaterSecurity2022,koehlerCitizenParticipationCollaborative2008,njambi-szlapkaIntegratingCommunityVoices}. The \acrshort{ifrc} states, that the "community engagement and accountability (CEA) is essential […] to build acceptance and trust” for effective and sustainable outcomes \autocite{ifrcCommunityEngagementAccountability}.\newline



In the last two decades, \acrlong{cs} has become a vibrant area of scientific interest covering various aspects in many different contexts \autocite{kirschkeCitizenScienceProjects2022,kullenbergWhatCitizenScience2016}. Relatively recent developments in \acrlong{cbm} and \acrlong{mcs} now make it possible for a large number of citizens to contribute to scientific, social and environmental endeavours with just a simple phone \autocite{butteFrameworkWaterSecurity2022}. This engagement of the general public can have multiple benefits for a wide variety of aspects. Scientific processes of e.g. linking climate variability to local water security can be informed, the public's education and awareness about specific topics can be raised, and decision-making and overall management can be enhanced, if the project is embedded in these procedures \autocite{huangManagementDrinkingWater2020,kirschkeCitizenScienceProjects2022,minkmanCitizenScienceWater2015}.


Furthermore, \acrshort{cs} projects have demonstrated their ability to gather data and fill gaps particularly in formerly data sparse regions in an effective and cost-efficient manner \autocite{butteFrameworkWaterSecurity2022,lackstromBackyardHydroclimatologyCitizen2022,weeserCitizenSciencePioneers2018a}. However, currently \acrshort{cs} projects and studies are primarily located in North America, Europe and Australia \autocite{kirschkeCitizenScienceProjects2022, koehlerCitizenParticipationCollaborative2008, livinglakescanadaElevatingCommunityBased2018}. In the field of environment and water monitoring, these projects are mainly concerned about  \autocite{kirschkeCitizenScienceProjects2022}. Social.Water, CoCoRaHS and \autocite{speirSolutionsCurrentChallenges2022}'s study are examples of those environmental data collection and drought monitoring implementations focussing on monitoring river, lake, groundwater and precipitation levels. However, these approaches all require internet access and more technical equipment, making them unfeasible for low-income conditions \autocite{fienenSocialWaterCrowdsourcing2012a,lackstromBackyardHydroclimatologyCitizen2022,lowryGrowingPainsCrowdsourced2019}.\newline


% !
% !
% !
% !
% !
% !!!!!!!!!!!!!!!!!!!!!!!!!!!!!!!!!!!!!!!!!!!!!!!!!!!!!!!!!!!!!!!!!!!!!!!!!!!!!!!!!!!!!!!!!!!!!!

https://libguides.usc.edu/writingguide/discussion

"The stages of the roadmap outlined above give a good overview of what needs to be done and in what order. However, the process-oriented structure makes it difficult not to overlook important information, as there is no more detailed and grouped listing of this. This catalogue attempts to close this gap. " yeah well.. right mate but how good is it really?

more difficult to transfer to other contexts with concise and concrete requirements but: tried to keep them relative general by also not being to far off - future will show how well that worked.

Systematically explain the meaning of your case study findings and why you believe they are important. 



1. Most effectively demonstrates your ability as a researcher to think critically about an issue, to develop creative solutions to problems based upon a logical synthesis of the findings, and to formulate a deeper, more profound understanding of the research problem under investigation;
2. Presents the underlying meaning of your research, notes possible implications in other areas of study, and explores possible improvements that can be made in order to further develop the concerns of your research;
3. Highlights the importance of your study and how it can contribute to understanding the research problem within the field of study;
4. Presents how the findings from your study revealed and helped fill gaps in the literature that had not been previously exposed or adequately described; and,
5. Engages the reader in thinking critically about issues based on an evidence-based interpretation of findings; it is not governed strictly by objective reporting of information.


\textbf{The content of the discussion section of your paper most often includes:}

Explanation of results: Comment on whether or not the results were expected for each set of findings; go into greater depth to explain findings that were unexpected or especially profound. If appropriate, note any unusual or unanticipated patterns or trends that emerged from your results and explain their meaning in relation to the research problem.
References to previous research: Either compare your results with the findings from other studies or use the studies to support a claim. This can include re-visiting key sources already cited in your literature review section, or, save them to cite later in the discussion section if they are more important to compare with your results instead of being a part of the general literature review of prior research used to provide context and background information. Note that you can make this decision to highlight specific studies after you have begun writing the discussion section.
Deduction: A claim for how the results can be applied more generally. For example, describing lessons learned, proposing recommendations that can help improve a situation, or highlighting best practices.
Hypothesis: A more general claim or possible conclusion arising from the results [which may be proved or disproved in subsequent research]. This can be framed as new research questions that emerged as a consequence of your analysis.


\textbf{Keep the following sequential points in mind as you organize and write the discussion section of your paper:}

Think of your discussion as an inverted pyramid. Organize the discussion from the general to the specific, linking your findings to the literature, then to theory, then to practice [if appropriate].

Use the same key terms, narrative style, and verb tense [present] that you used when describing the research problem in your introduction.

Begin by briefly re-stating the research problem you were investigating and answer all of the research questions underpinning the problem that you posed in the introduction.

Describe the patterns, principles, and relationships shown by each major findings and place them in proper perspective. The sequence of this information is important; first state the answer, then the relevant results, then cite the work of others. If appropriate, refer the reader to a figure or table to help enhance the interpretation of the data [either within the text or as an appendix].

Regardless of where it's mentioned, a good discussion section includes analysis of any unexpected findings. This part of the discussion should begin with a description of the unanticipated finding, followed by a brief interpretation as to why you believe it appeared and, if necessary, its possible significance in relation to the overall study. If more than one unexpected finding emerged during the study, describe each of them in the order they appeared as you gathered or analyzed the data. As noted, the exception to discussing findings in the same order you described them in the results section would be to begin by highlighting the implications of a particularly unexpected or significant finding that emerged from the study, followed by a discussion of the remaining findings.

Before concluding the discussion, identify potential limitations and weaknesses if you do not plan to do so in the conclusion of the paper. Comment on their relative importance in relation to your overall interpretation of the results and, if necessary, note how they may affect the validity of your findings. Avoid using an apologetic tone; however, be honest and self-critical [e.g., in retrospect, had you included a particular question in a survey instrument, additional data could have been revealed].

The discussion section should end with a concise summary of the principal implications of the findings regardless of their significance. Give a brief explanation about why you believe the findings and conclusions of your study are important and how they support broader knowledge or understanding of the research problem. This can be followed by any recommendations for further research. However, do not offer recommendations which could have been easily addressed within the study. This would demonstrate to the reader that you have inadequately examined and interpreted the data.



\textbf{Problems to Avoid}
Do not waste time restating your results. 
Do not introduce new results in the discussion section.

but focus on the interpretation of those results and their significance in relation to the research problem, not the data itself.




"How do the findings of this study contribute to the existing body of knowledge on community-based participatory mapping and monitoring in the context of forecast-based financing?
In what ways could the framework developed in this study be adapted for use in other contexts, and what challenges might arise in doing so?
What ethical considerations must be taken into account when implementing a community-based participatory approach to water source mapping and monitoring, and how were these addressed in this study?
How did the use of a mixed-methods approach impact the validity and reliability of the study's findings?
To what extent did the involvement of local community members in the research process influence the outcomes of the study?
In what ways could the framework developed in this study be further refined or improved upon in future research?
What implications do the findings of this study have for policy-makers and practitioners in the fields of water resource management and disaster risk reduction?
How might the methods and approach used in this study be adapted for other types of community-based initiatives or projects?
What challenges did the research team encounter during the development and implementation of the framework, and how were these addressed?
How might the findings of this study contribute to broader discussions around the role of community participation in the management of natural resources and disaster risk reduction efforts?" ChatGPT









% !!!!!!!!!!!!!!!!!!!!!!!!!!!!!!!!!!!!!!!!!!!!!!!!!!!!!!!!!!!!!!!!!!!!!!!!!!!!!!!!!!!!!!!!!!!!!!
% literature "discussion"

Besides addressing the first objective, to \textit{conduct a comprehensive review of existing literature and guidelines related to the design and implementation of \acrlong{cs} programmes, and identify relevant work in regard to the research aim and overall case study context} the literature and \acrshort{cs} project analysis could also create a sound foundation for the following study (see section \ref{subsec:stage1_appl}). % The exploration of the conceptual and practical context allowed the identification and specification of relevant frameworks, aspects and gaps in literature for the subsequent research objectives.\newline
Breaking down the broad concepts of Water Security, Water Scarcity and Drought along with their indicators and indices to the local context highlighted that only relatively rough forecasts are available for Somaliland (see section \ref{subsec:indicators}). Currently, climate, weather and hazard forecasts for Somaliland are either based on international indices like SPI or on a scarce network of local weather gauging stations (see section \ref*{subsec:case_eap}). Besides their coarseness, these indices predict the climate or weather itself and not its impacts, making them unsuitable for \acrlong{fbf} (see section \ref*{subsec:eap}). For successful implementation of \acrshort{fbf}, triggers and actions should be developed and directly linked (see section \ref{subsec:trigger} and \ref{subsec:case_eap}). This is often not feasible as local information about water sources is either missing completely, is incomplete or outdated (see section \ref{subsec:stage1_appl}). This highlighted the need for new local impact indicators for the creation of which the \acrshort{cs} approach was consulted. Several \acrshort{cbm}, \acrshort{mcs}, \acrshort{cbs}, \acrshort{cbwm} and other risk related \acrshort{cs} frameworks and respective guidelines could be identified but none of them exactly matched the intended application (see section \ref{sec:cs}). While "there is no one-size-fits-all approach" \autocite[2]{fraislCitizenScienceEnvironmental2022}, the existing frameworks either focussed on different thematics, different contexts, had different participation levels, different goals or a combination of the above (see sections \ref{subsubsec:cbwm}, \ref{subsubsec:cbs} and \ref{subsec:cbc}). This is consistent with \autocite{butteFrameworkWaterSecurity2022}'s and \autocite{carrionCROWDSOURCINGWATERQUALITY2020}'s findings that existing frameworks guiding the development of water security data collection projects are often very specific and limited to certain factors, in many cases also not taking socio-economic factors into account. At the same time, frameworks like the on from \autocite{butteFrameworkWaterSecurity2022,eu-citizen.scienceEUCitizenScience,citizenscience.govBasicStepsYour} and others were too broad, to be more than general guidelines. Therefore, no applicable framework existed for the implementation of a community-based participatory mapping and monitoring of water sources approach in a water-scarce and resource-limited setting. Especially not, with the focus on providing feasible information for triggering \acrshortpl{aa} in the context of an \acrshort{eap} and in collaboration with a \acrshort{rcrc} National Society.\newline

Other networks like \acrlong{brcis} and the local branch of \acrshort{ocha} implemented their own early action approaches in Somaliland. However, on the one hand with different goals, and on the other hand with different methods (see sections \ref{subsec:stage1_appl} and \ref{subsec:case_eap}). While \acrshort{brcis} collects and interpolates qualitative local information, \acrshort{ocha} bases their early actions on the before mentioned large scale indicators. These approaches are either too slow or to coarse to address the aim of this research, but the concrete experiences from projects in the case study area are valuable to adapt and relate other information to the given context. The transfer of knowledge from other regions, projects and topics is necessary, as scientific literature about the case study area of Somaliland is generally scarce. In addition to these case study related domains, there are further gaps in knowledge when in comes to the application of the \acrshort{fbf} approach on the slow-onset hazard of drought. Generally, the concept of \acrshort{fbf} is now well established in regard to fast-onset disasters, but the drought use case is relatively new (2020) and not yet well researched, which severely limits the amount of guidelines and frameworks available for this particular application (see section \ref{subsec:eap}). Thus, each new project or study focussing on this hazard in the context of \acrshort{fbf} has, at least in part, an exploratory character.\newline
As any new project or study addressing this hazard within these concepts is thus 'automatically' exploratory in nature and no other suitable framework could be identified, the literature and project review suggested the need to develop a new framework to address the specifics of the case study (see section \ref{subsec:cbm}). However, before the new conceptualisation, the general feasibility had to be assessed first, leading to the second objective of this work.


% !!!!!!!!!!!!!!!!!!!!!!!!!!!!!!!!!!!!!!!!!!!!!!!!!!!!!!!!!!!!!!!!!!!!!!!!!!!!!!!!!!!!!!!!!!!!!!
% SSDR and PRC "discussion"
Having established the feasibility of the \acrshort{cs} approach, the third objective to \textit{develop a replicable and adaptable framework for community-based participatory water source mapping and monitoring in the context of \acrlong{fbf}, based on the principles and recommendations identified in objectives 1 and 2} could be pursued. There are now a high number and wide variety of guidelines, \acrshort{cs} associations, initiatives and projects to choose from, that the question of the necessity to add just another one to the list suggests itself. The literature analysis suggested that, above all, the high variety is explained and reflected in the high diversity of the \acrshort{cs} approach of e.g. methodology, temporal and spatial scale, goals, context, level of participation and overall goal (see section \ref{sec:cs}). \autocite{fraislCitizenScienceEnvironmental2022, westonCommunityBasedWaterMonitoring2015} and \autocite{zhengCrowdsourcingMethodsData2018} all summarise a wide variety of these guidelines and \autocite{garciaFindingWhatYou2021} even created a \textit{Guide to Citizen Science Guidelines}. Nonetheless, none of these met the needs of this work, which prompted the development of a new framework. That goes along with the recommendations, of \autocite{garciaFindingWhatYou2021}, that the development and thus transfer of experience in guidelines is the currently the best practice in the field when new, previously unrealised combinations of the thematic diversity listed above are approached and realised \autocite{garciaFindingWhatYou2021}. This highlights the need, that frameworks need to be focussed on a specific topic, region and environment in order to give meaningful advice and not only generic information that is too coarse to be of great use.\newline

The decision to build on \autocite{fraislCitizenScienceEnvironmental2022}'s \acrlong{ssf} was primarily driven by its timeliness, comprehensiveness and focus on environmental issues as it was clear, that a more social and local component can be integrated from the \acrshort{srcs}'s experiences with \acrshort{cbs}. The usefulness of the interpolation of these two approaches was particularly evident in the consideration of personal data. While observing natural phenomena at the level of data collection did not raise too many privacy concerns for \autocite{fraislCitizenScienceEnvironmental2022}, this was almost the opposite for CBS \autocite{ifrcCommunityBasedSurveillanceGuiding2017}. Applying these contrasting perspectives to the issue of water sources was thus able to address both the physical and social components well by considering trade-offs between the two 'extremes'. This claim was further supported over the course of this work, when the iterative integration of other guidelines from several divergent foci into the existing framework could be implemented smoothly and only minor revisions had to be made. \autocite{mcgowanCommunitybasedSurveillanceInfectious2022} also found that the success factors of \acrshort{cbs} are closely linked to the general principles of participatory community engagement and could therefore be transferred to other participatory surveillance preparedness activities. 

As the individual stages are outlined and described in detail in section \ref{sec:design_roadmap} and all identified relatable guidance is included into the design roadmap, no further discussion of the individual stages will be undertaken. % geht das so durch (?) neee



Nonetheless, this processual \acrshort{ssf} has some short-comings which will mostly be addressed in the discussion of its actual application (section \ref{TODO:}) and following limitation section \ref{TODO:} of this entire work.\newline


However, a couple of shortcomings became apparent right at the beginning of the application in the third phase. It was increasingly difficult to keep an overview of the actual project requirements and their interdependencies in terms of subject matter and temporal constraints (see section \ref{subsubsec:knowledge}). Furthermore, \acrshort{cbs}, \acrshort{cbwm} and other approaches have strongly emphasised the importance of embedding the project into prevailing social and decision-making conditions and procedures, which was under-represented in the \acrshort{ssf} (see section \ref{subsubsec:groundwork}). The Interviewees also highlighted, the time (over a year) and resource requirements, which they needed for the development and adaptation of methods and techniques to start with the \acrshort{cbs} project in Somaliland (I2, I3). This goes along with \autocite{garciaFindingWhatYou2021}'s findings, that some adjustments and tailoring always need to be done when implementing a new project (see section \ref{subsubsec:innovations}). Together with the emerging need to structure smaller developments and create an overview of decision dependencies, a fourth area of management became apparent that needed to be addressed (see section \ref{subsubsec:management}).



These reasons provided the ground for the new development of the \acrlong{prc} to expand the \acrshort{ssf} from the third phase onwards. Since the first and second stages are primarily exploratory in nature, it is believed that the \acrshort{prc} should not be integrated in these stages, as it could limit the exploration to the given categories. Should something be overlooked, completions are still possible at the beginning of the third phase.\newline

The conceptualisation and structure of the \acrshort{prc} with the \acrlong{slmc}, guided by the derived goals by \autocite{minkmanCitizenScienceWater2015} further supports the structuring and elaboration of the dependency. Emphasis is given to the top three layers, the \textit{Goal-, Products-, and Activities-Layer}., firstly due to time and information constraints and secondly as practical applicable\textit{ methods, techniques, tools and scripts} need to be highly adjusted to the local context (see section \ref{subsec:slmc}). Both are main limitations of this work and will further be discussed in the last section of this chapter. Nonetheless, the limitation of this work may not be true for subsequent work, which can then benefit from the deeper and more detailed structural possibilities of the \acrshort{slmc}. Despite the fact that the thematic focus of the \acrshort{slmc} is not on \acrshort{cs}, the overall design pattern could be adopted well. The successful preservation of design patterns to other fields is also supported by \autocite{diggelenGroundedDesignDesign2009}'s findings. Thus, although the work has a sound methodological basis, it is primarily based on the perspective of a process- and requirements-oriented understanding and reasoning of the design phase. Other perspectives such as resource, behavioural network or stakeholder networks, cultural norms and values, as well as the communication network perspective may play a role in certain aspects, but are of secondary importance in this work. Engaging these other perspectives more in depth could yield further important insights by encouraging a more holistic view of the design.



% project requirements
However, the process-oriented structure makes it difficult not to overlook important information and outline their inter-dependencies, as there is no more detailed, structured and grouped listing of this. 