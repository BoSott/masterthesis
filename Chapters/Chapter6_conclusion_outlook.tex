\chapter{Conclusion \& Outlook}

\label{chapter6}
\gls{formula}
\gls{aa}
\glspl{aa}
This chapter concludes the study by summarising the main research findings in relation to the research questions and aim. It highlights the value of its contributions and suggests possibilities for future research.\newline
This study has investigated the intersection of \acrfull{fbf} policies and techniques, \acrfull{cs} approaches and methods, and water management structures and procedures in Somaliland. This investigation was driven by the aim to \textit{first develop a new and transferable framework for community-based participatory mapping and monitoring of water sources for water-scarce and resource-limited settings to facilitate relevant Anticipatory Actions in the context of FbF and then apply this framework to create an implementation roadmap for the SRCS, ultimately aiming to improve water management and information availability to better address water shortages in Somaliland proactively}.\newline
Guided by two research questions on what such a framework could look like and how it could subsequently be applied in Somaliland, a mixed-methods approach combining literature analysis and expert consultations was applied to develop a tailor-made framework and create an implementation roadmap. The results indicate that integrating the concepts of \acrshort{fbf} and \acrshort{cs} for monitoring water source levels in resource-scarce settings to ultimately trigger \acrlongpl{aa} into one framework is theoretically possible and may add a local socio-economic perspective to the otherwise physically focused insights. In the case of Somaliland, the practical feasibility of this integrated framework can also be reasonably assumed based on the literature and especially the expert interviews.\newline

This work further diversified the literature on \acrlong{cs} projects by contributing a case study in regions other than North America and Europe. The development of the adaptable and replicable \acrlong{ssdr} and \acrlong{prc} framework may allow other work with similar aims and conditions to have a closer start of reference for designing their own project. Specifically in this context, the thesis laid a starting point for the implementation of a practical pilot study by the \acrlong{srcs}. This may lead to better data on water sources, which in turn could contribute to the implementation of \acrlongpl{aa} to ultimately better address water shortages.\newline

This study was primarily constrained by a modest number of interviewees, no opportunity for on site work and general time constraints. Therefore, the work remained at the conceptual stage and could not be evaluated against a practical application. Other evaluation options, such as direct comparison with other similar projects, were not feasible due to the novelty of the project and the consequent lack of similar ones. In addition, no concrete technical approaches and possibilities for data triangulation could be formulated as decisions at management level had not yet been finalised.\newline

Future research can directly continue where this work left of by implementing an on site pilot study, continue to dig deeper into one of the many questions that have arisen or focus on overcoming the current limitations.
A pilot study could potentially address most of the primary constraints of this work and continue to adapt, implement and evaluate it locally. There are several questions worth asking in such a case study. Apparent areas of interest are the investigation of the water level measurement method, corresponding codes, and the assessment of triggers and Anticipatory Actions. In terms of community engagement, exploring ways of integrating \acrlong{iwrm} with prevailing local practices on an equal footing, asking what the involvement of community elders might look like, and addressing issues of community heterogeneity and gender inequalities may all be potentially fruitful enquiries. Also of great interest is what benefits a two-way communication with the participants could further yield, particularly in terms of receiving and integrating local and indigenous knowledge and providing weather and climate predictions and warnings. Furthermore, investigations and evaluations in various other fields will be required when further exploring a practical implementation.\newline
The limitations of low external and internal validity and the question of whether the framework can be applied to other contexts may be addressed by further case studies in similar contexts and by including other methods such as upscaled surveys. The further investigation of the link between the water level proxy, vulnerability and impact may add further value to the argument of constructed validity. In addition, inter-project comparisons as well as comparisons with other methods, e.g. (remote-)sensor networks could be investigated. Besides these application-related questions, it would be interesting to examine more closely the recognised positivity bias and its effects in \acrshort{cs} guidelines and frameworks.\newline

% In conclusion, the situation of water availability in Somaliland could be identified as a highly complex and challenging environment and the combination of \acrlong{fbf} and \acrlong{cs} concepts could be identified as potentially fruitful for the monitoring of water sources. A further, especially practical, investigation into the issue and linking it to preventive measures may contain great value. In addition, proof-of-concept, -value and -use may be demonstrated by going the last research mile.
