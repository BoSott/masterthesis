% Chapter Template

\chapter{Conclusion} % Main chapter title

\label{ChapterX} % Change X to a consecutive number; for referencing this chapter elsewhere, use \ref{ChapterX}

%----------------------------------------------------------------------------------------
%	SECTION 1
%----------------------------------------------------------------------------------------
\section{Conclusion}
% TODO: integrate research questions!!
%? Overall, there are now a high number and wide variety of guidelines, \acrshort{cs} associations, initiatives and projects to choose from, that the question of the necessity to add just another one to the list suggests itself. The literature analysis suggested that, above all, the high variety is explained and reflected in the high diversity of the \acrshort{cs} approach of e.g. methodology, temporal and spatial scale, goals, context, level of participation and overall goal (see section \ref{sec:cs}). The many encountered open questions in this work further strengthen the argument, that the demand side of guidelines is still higher than what the supply side currently delivers. \autocite{fraislCitizenScienceEnvironmental2022, westonCommunityBasedWaterMonitoring2015} and \autocite{zhengCrowdsourcingMethodsData2018} all summarise a wide variety of these guidelines and \autocite{garciaFindingWhatYou2021} even created a \textit{Guide to Citizen Science Guidelines}. Nonetheless, none of these met the needs of this work, which prompted the development of a new framework. That goes along with the recommendations, of \autocite{garciaFindingWhatYou2021}, that the development and thus transfer of experience in guidelines is the currently the best practice in the field when new, previously unrealised combinations of the thematic diversity listed above are approached and realised \autocite{garciaFindingWhatYou2021}. This highlights the need, that frameworks need to be focussed on a specific topic, region and environment in order to give meaningful advice and not only generic information that is too coarse to be of great use.

% synthesise all major points -> take away message
% -> what is found, what is valuable, how to apply it and further research
% KEY AREAS
% Summarise the key findings of the study
% Explicitly answer the research question(s) and address the research aims
% Inform the reader of the study’s main contributions
% Discuss any limitations or weaknesses of the study (major stuff)
% Present recommendations for future research
% step 1: Craft a brief introduction section
This chapter concludes the study by summarising the main research findings in relation to the research questions and aim and discussing their value and contribution. It also illustrates the limitations of the study and suggests possibilities for future research.\newline
% Step 2: Discuss the overall findings in relation to the research aims
% -> broader findings + research aims
% _> reminding your reader: “This study aimed to…” and “the results indicate that…”
% -> no bold claims
This study has investigated the intersection of \acrlong{fbf} policies and techniques, \acrlong{cs} approaches and methods, and water management structures and procedures in Somaliland. This was attempted by \textit{adapting and applying an approach for community-based participatory mapping and monitoring of water sources in a water-scarce and resource-limited setting in collaboration with a national non-governmental organization to facilitate respective \acrlongpl{aa} in the context of \acrlong{fbf}, with the goal of improving water management and availability to address water shortages}. Utilizing a mixed-methods approach that combined literature analysis and expert consultation, a tailored framework could be developed and an implementation roadmap created. The results show that integrating the concepts of \acrlong{fbf} and \acrlong{cs} for monitoring and \acrlongpl{aa} based on water source levels in resource-scarce settings into one framework is theoretically possible. In the case of Somaliland, the practical feasibility of this integrated framework can also be assumed to be feasible based on the results.\newline % TODO:maybe an extra paragraph about a closer look at the results
% Step 3: Discuss how your study contributes to the field
% -> "achieved in your study, highlighting why this is important and valuable, and how it can be used or applied"
% -> theory and praxis
% -> research outputs
% -> reflect on gaps -> and how those were addressed
% -> relation to relevant theories (contradict/confirm?)
% -> findings applied in the real world
Fundamentally, this work further diversified the literature on \acrlong{cs} projects by contributing a case study in regions other than North America and Europe. The development of the adaptable and replicable framework in this context may allow other work with similar aims and conditions to have a closer start of reference for designing their own project. Specifically in this context, the thesis laid a starting point for the implementation of a practical pilot study by the \acrlong{srcs}. Therefore, perhaps contributing to better data regarding water sources and ultimately leading to the implementation of \acrlongpl{aa}.\newline
% --> balance between firm but humble - did I made it?
% Step 4: Reflect on the limitations of your study
% ->  critically reflect on the limitations 
% -> do not repeat the discussion section
% -> e.g. generalisability, sample size, collection/analysis techniques, lack of whatever.., time constraints
% -> don't undermine the research
% -> understand limitations, justify them with the given constraints and let them know how to improve those
This study was primarily constrained by a modest number of interviewees, no opportunity for on site work and general time constraints. Therefore, the work remained at the conceptual stage and could not be evaluated against a practical application. Other evaluation options, such as comparison with other projects, were not feasible due to the novelty of the project and the consequent lack of similar ones. In addition, no concrete technical approaches and possibilities for data triangulation could be formulated because decisions at management level had not yet been finalised.\newline
% Step 5: Make recommendations for future research
% -> based on limitations, interesting or surprising stuff, how one could build on this study
%----------------------------------------------------------------------------------------
% Step 5: make recommendations for implementation and future research
%----------------------------------------------------------------------------------------
Future research can directly continue where this work left of by implementing an on site pilot study, continue to dig deeper into one of the many questions that have arisen or focus on overcoming the current limitations.
A pilot study could potentially address all primary constraints of this work and continue to adapt, implement and evaluate it locally. However, before a case study may be conducted, there are several other questions worth asking. In regard to local embeddedness in the community, the heterogeneity, gender inequalities and the involvement of the elders are exciting fields. Also, what lies at the intersection of two-way communication and integration in terms of local and indigenous knowledge, ways and benefits of the distribution of weather and climate forecast information and warnings are also highly interesting areas. The investigation and evaluation of solutions for awareness raising, public education, volunteer engagements, policy development, decision-making, method improvements and data quality may be further paths to explore.\newline
The limitations of low external and internal validity and the question of whether the framework can be applied to other contexts be addressed by further case studies in similar contexts or the inclusion of other methods such as upscaled surveys. The further investigation of the link between the water level proxy, vulnerability and impact may add further value to argument of constructed validity. In addition, inter-project comparisons as well as comparisons with other methods, e.g. (remote-) sensor networks could be investigated.\newline
% Step 6: Wrap up with a closing summary
In conclusion, the situation of water in Somaliland could be identified as a highly complex and challenging environment. The combination of \acrlong{fbf} and \acrlong{cs} could be identified as potentially fruitful for monitoring water source levels. A further, especially practical, investigation into the issue and linking it to preventive measures may contain great value. In addition, proof-of-concept, -value and -use may be demonstrated by going the last research mile. 

% APPEDNIX
% include questions, questionnaires, transcripts and codes