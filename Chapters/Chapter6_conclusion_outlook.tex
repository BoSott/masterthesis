% Chapter Template

\chapter{Conclusion} % Main chapter title

\label{ChapterX} % Change X to a consecutive number; for referencing this chapter elsewhere, use \ref{ChapterX}

%----------------------------------------------------------------------------------------
%	SECTION 1
%----------------------------------------------------------------------------------------

\section{Conclusion}

%-----------------------------------
%	SUBSECTION 1
%-----------------------------------
\subsection{Subsection 1}


%-----------------------------------
%	SUBSECTION 2
%-----------------------------------

\subsection{Limitations}

% won't of course change everything
% also: impact is based on humans, governance, social patterns and behaviour. Who would have guessed?!

“In humanitarian practice, the term "drought" is often used to refer to some socio-meteorological combination where water shortages produce stress on human and livelihood systems. Droughts are a function of the fragility of human systems, and they become disasters where systems cannot cope with deviations from the hydro-meteorological norm. It has been argued that droughts are particularly devastating when livelihood choices are strongly determined by the climate (e.g. the decision to grow certain crops, or traditional seasonal migration patterns) - for instance, if in a given year, the weather patterns are different than normal, those livelihoods are especially vulnerable to these changes. It has also been argued that droughts pose specific challenges to income generating activities marked by low productivity that are not able to take advantage of ‘good years’ in order to provide a buffer during ‘bad years’.” ([RCRC, 2020, p. 12](zotero://select/groups/4773535/items/UESIQTRJ)) ([pdf](zotero://open-pdf/groups/4773535/items/P5JPVZ97?page=12&annotation=C622YWIR))

%----------------------------------------------------------------------------------------
%	SECTION 2
%----------------------------------------------------------------------------------------

\section{Future Outlook}

local knowledge, gender, be on the ground and talk to people, pre-study, citizen science, crowdsourcing, crowdsensing, participant motivation, data accuracy, data quality,
% da fällt mir schon noch viel ein..

test phase -> pre-study in a small local area which is not too badly affected currently 
