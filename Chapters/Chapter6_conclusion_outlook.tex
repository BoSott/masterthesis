% Chapter Template

\chapter{Conclusion} % Main chapter title

\label{ChapterX} % Change X to a consecutive number; for referencing this chapter elsewhere, use \ref{ChapterX}

%----------------------------------------------------------------------------------------
%	SECTION 1
%----------------------------------------------------------------------------------------
Finally, we will conclude with some reflections on the wider implications of our findings and the potential for our approach to contribute to sustainable development and social justice in water-scarce and resource-limited settings.

\section{Conclusion}

Both these examples further underline the importance of overall resource management, 


more general and high-level -> report the main research outcomes

synthesise all major points -> take away message
-> what is found, what is valuable, how to apply it and further research

% KEY AREAS
Summarise the key findings of the study
Explicitly answer the research question(s) and address the research aims
Inform the reader of the study’s main contributions
Discuss any limitations or weaknesses of the study (major stuff)
Present recommendations for future research



% step 1: Craft a brief introduction section
what they can expect to find in the chapter, and in what order

e.g. "This chapter will conclude the study by summarising the key research findings in relation to the research aims and questions and discussing the value and contribution thereof. It will also review the limitations of the study and propose opportunities for future research."

% Step 2: Discuss the overall findings in relation to the research aims
-> broader findings + research aims
_> reminding your reader: “This study aimed to…” and “the results indicate that…”
-> no bold claims

% Step 3: Discuss how your study contributes to the field
-> "achieved in your study, highlighting why this is important and valuable, and how it can be used or applied"
-> theory and praxis
-> research outputs
-> reflect on gaps -> and how those were addressed
-> relation to relevant theories (contradict/confirm?)
-> findings applied in the real world

--> balance between firm but humble

% Step 4: Reflect on the limitations of your study
->  critically reflect on the limitations 
-> do not repeat the discussion section
-> e.g. generalisability, sample size, collection/analysis techniques, lack of whatever.., time constraints

-> don't undermine the research
-> understand limitations, justify them with the given constraints and let them know how to improve those

% Step 5: Make recommendations for future research
-> based on limitations, interesting or surprising stuff, how one could build on this study


% Step 6: Wrap up with a closing summary
-> brevity is key here -> only include key takeaways (one paragraph)

%-----------------------------------
%	SUBSECTION 1
%-----------------------------------
\subsection{Subsection 1}

benefits in contrast to satellite stuff

% I2
(104) MoH has good experiences with NYSS in the health sector. These experiences should be translated and widened on other topics.



% I1
-results for the implementation of AAs:
o	i) Determining the water level to trigger action
o	ii) water levels monitoring
o	iii) triggering action based on water levels


The combination of both mechanisms (short and seasonal) might be the best choice for the creation of a staggered trigger.


The trigger will follow the overall trigger methodology of the \acrshort{eap} and 
trigger

method: generally: coded SMS



the project itself, the development of the framework and also the applicatoin of it was an iterative process which got substantially better with each iteration - more iterations, possibly even better but sometimes good is also great :o


% recommended to combine the BRCiS appraoch with I1 categorization. one more extensive long-term assessment at the end of the rain season together with the other rarely changing indicators (e.g. number of animals and people) but also combined with an triangulation with stakeholder and thorough analysis and interpretation of data --> integration into decision-making processes. --> seasonal warning and identification of vulnerability + short-term facilitation of immediate action with the short-term water level warnings



% I1.2
-	Water levels in berkeds could be a good indicator, however it cannot be a stand alone indicator. This has to be combined by meteorological forecasts and local knowledge as well.
- Regarding Anticipatory actions, there has been any actions yet due to the fact that there is no water monitoring and trigger mechanism in place.

% regular reporting with event based reporting as water in Berkads can e.g. dry up quicker than predicted or turn bad or what ever else..


with preceding procedures including awareness raising, public education, policy development and method improvements determined and accomplished now proceed to the actual mapping task of Berkeds.


" In particular, we realised the importance of a strong local partnership, based on mutual respect, clear agreed goals of collaboration and shared development interests [26, 27], but also the willingness and honesty to question decisions and discuss alternative approaches."

The development of a guidance for this project may seem overkill as so many guidelines already exist but none was made for such as topic in such a context --> summarisation was necessary. --> generalisation is potentially possible when the case study is conducted and maybe more case-studies are integrated (multi-case-study design elevates external validity) and a manual could be the goal... 



% self surveying is generally not recommended:
“Indeed, collecting data on local indicators would require from the national society a team of enumerators that work continually to collect and process that information in all places where the program could possibly trigger (e.g. collect food price information for every village market). This would have extensive cost implications and likely over-burden the national society staff and volunteers.” ([RCRC, 2020, p. 30](zotero://select/groups/4773535/items/UESIQTRJ)) ([pdf](zotero://open-pdf/groups/4773535/items/P5JPVZ97?page=30&annotation=2YIIK6ZY))

“As such, the inclusion of local indicators into an FbA trigger must involve assessing what indicators are relevant for the impacts the program is trying to anticipate and identify which of those indicators are already collected (e.g. the ministry of agriculture's food price bulletin) and are available at the time they would be needed to inform a possible trigger.” ([RCRC, 2020, p. 30](zotero://select/groups/4773535/items/UESIQTRJ)) ([pdf](zotero://open-pdf/groups/4773535/items/P5JPVZ97?page=30&annotation=7X3RFGVB))


\section{Future Outlook}

“5.1.3. Challenges and Future Directions” ([Zheng et al., 2018, p. 721](zotero://select/groups/4773535/items/LJU68CG4)) ([pdf](zotero://open-pdf/groups/4773535/items/U8QNZLI6?page=24&annotation=D54SWR46))



local knowledge, gender, be on the ground and talk to people, pre-study, citizen science, crowdsourcing, crowdsensing, participant motivation, data accuracy, data quality,
% da fällt mir schon noch viel ein..

test phase -> pre-study in a small local area which is not too badly affected currently 

maybe formulation of hypothesis -> inductive stuff ends in this --> therefore, its falsification should be tested in future research.


is feasible? --> yes

basic framework/roadmap --> yes

practical pilot case study --> to be done.


further information about the water source - e.g. accessibility could be surveyed and potential integrations via codes need to be delineated


methodology worked, frameworks were good, time, interviews and resources were the constraints

% future outlook CBM


Beyond data collection, \autocite{gualazziniEWEAEarlyWarning2021} highlights the possibility to adopt a two-way communication capability to also distribute climate and weather forecasts to participants. Thus "communities have direct access to short-term weather forecasts and climate risk alerts to mitigate risk and protect livelihoods, while also providing information from the site" \autocite[20]{gualazziniEWEAEarlyWarning2021}. % ITIKI may also contribute to this statement


further open questions in all directions --> also look at my own tree-diagrams

There is a need for research exploring if marginalized perspectives are excluded in crowdsourcing and self-reporting approaches, the recall bias and measurement error in self-reporting (Bell et al. 2019), and if incentives, memory triggers, or other mechanisms can be used to address these issues.



local knowledge
gender inequalities

etc. not really mentioned but have huge potential to improve all of this. Big question is HOW!

% zu stage 2
The funding of a pilot study as a decisive factor could not be clarified in this framework, but does not affect the further conception. --> though there is an ongoing proposal which would build on top of this work (if granted).



% compare CS with sensor networks
“Moving beyond the Technology: A Socio-technical Roadmap for Low-Cost Water Sensor Network Applications” ([Mao et al., 2020, p. 9145](zotero://select/groups/4773535/items/VEJUB8D9)) ([pdf](zotero://open-pdf/groups/4773535/items/KMCIFXB8?page=1&annotation=2A7QA3V3))



% APPEDNIX
% include questions, questionnaires, transcripts and codes
The first interview came about through existing contacts of the project in which this work is embedded, and the interviewee was the project leader of the FbF approach in the \acrshort{srcs} (I1). In the further course, the CBS project manager on the Norwegian Red Cross side (I2) and the CBS manager on the Somali side (I3) were also interviewed. Between these two interviews, there was a second interview with the project manager of the \acrshort{srcs}' \acrshort{fbf} team (I1.2).




% das wäre doch mal ein feiner Outlook für future research.
How can the proposed approach for community-based participatory mapping and monitoring of water sources in a water-scarce and resource-limited setting in collaboration with a national non-governmental organization be effectively tested and evaluated to determine its effectiveness in improving water management and accessibility in underserved communities?


%----------------------------------------------------------------------------------------
% Step 5: make recommendations for implementation and future research
%----------------------------------------------------------------------------------------

% 1. The practical application of your findings -> usage in the real real world - fly little bird
% -> contributions, applicability where, how and why? - stay realistic mate.
% 2. Suggestions for future research (opportunities for future research) -> build on it, improve by overcoming limitations (e.g. surveys)
% anything that I did not consider? also mention how this could be done.

% Step 6: provide a concluding summary
recap of key findings -> directly addressing research question
-> make sure this section flows well -> + strong connection to opening section

Tips 'n Tricks:

- keep it consistent with your introduction -> same tense and same key terms
- spell out findings and interpretations -> no assumptions about the reader
- avoid absolute terms (rather: suggest, indicate) -> be humble in my language
- well-structured and consistent headings





single-case design (could be extended through future studies)
Exploratory research is furthermore an effective tool to lay groundwork for future studies.

“proof-of-concept research to demonstrate the functional feasibility of a solution; proof-of-value research to investigate whether a solution can create value across a variety of conditions; and proof-of-use research” ([Nunamaker et al., 2015, p. 0](zotero://select/groups/4773535/items/Y8BXIHDQ)) ([pdf](zotero://open-pdf/groups/4773535/items/C8VM8EY8?page=1&annotation=3BBX7DN9))


“Current weather forecasts are still alien to African farmers, most of whom live in rural areas and struggle with illiteracy and poor communications infrastructure” ([Masinde and Bagula, 2012, p. 274](zotero://select/groups/4773535/items/EW9XSSZP)) ([pdf](zotero://open-pdf/groups/4773535/items/3WQ4S9PE?page=1&annotation=HPZC9Z65))


% maybeeeeee and maybe not.. tend not to include it
“2 Local knowledge in drought monitoring: an introduction to the literature review” ([Giordano et al., 2013, p. 526](zotero://select/groups/4773535/items/B7LM5ZR4)) ([pdf](zotero://open-pdf/groups/4773535/items/7I66DBIK?page=4&annotation=Z33M5FLQ))

% difficulty also emphasized by this (eher introduction?)
% later on
However, it requires complex and various information and is thus often difficult to implement \autocite{liuWaterScarcityAssessments2017}.


“Other questions raised include; how to combine relevant science, technology and local knowledge” ([pdf](zotero://open-pdf/groups/4773535/items/SH3998CR?page=15&annotation=4DI7BR57))

% --> language is important. --> staggered trigger -> cannot call everything an action! There is a difference between small and large responses etc.. especially in the perception of people
“We also found that the phrases we had been using to trigger action, Be Aware, Be Prepared and Take Action, were not driving the actions we had hoped. We used Take Action exclusively for high impact warnings, but we need people to be taking action for low and medium impact warnings too, if the impacts are to be mitigated. We have replaced these phrases with detailed impact information and linked our warnings to advice and guidance from our partner organisations.” ([Harrowsmith et al., 2020, p. 62](zotero://select/groups/4773535/items/QJ397Y54)) ([pdf](zotero://open-pdf/groups/4773535/items/2GS362N5?page=62&annotation=KZK3TSK3))

% multiple-case studies --> the more the better --> this is just the beginning of a pilot case
- exploratory studies for discovering relevant constructs in areas where theory building is at the formative stages
understanding complex processes involving multiple participants and interacting sequences of events in one setting
-> single if constraints -> mmultiple case studies, better generalisable

other research type such as experimental or survey research were not fit for purpose - but: survey research within a pilot study with the communities might be feasible to get some more juicy quantity (but: biases)