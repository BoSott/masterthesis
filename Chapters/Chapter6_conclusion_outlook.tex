% Chapter Template

\chapter{Conclusion} % Main chapter title

\label{ChapterX} % Change X to a consecutive number; for referencing this chapter elsewhere, use \ref{ChapterX}

%----------------------------------------------------------------------------------------
%	SECTION 1
%----------------------------------------------------------------------------------------

\section{Conclusion}

%-----------------------------------
%	SUBSECTION 1
%-----------------------------------
\subsection{Subsection 1}



" In particular, we realised the importance of a strong local partnership, based on mutual respect, clear agreed goals of collaboration and shared development interests [26, 27], but also the willingness and honesty to question decisions and discuss alternative approaches."


%-----------------------------------
%	SUBSECTION 2
%-----------------------------------

\subsection{Limitations}

% from the methodology 
Furthermore, case studies are frequently criticized for being excessively long, challenging to execute, and requiring significant documentation efforts \autocite{yinCaseStudyResearch1984}.
-->
it's true. Case studies are difficult and there is a lot to learn.
e.g. case study protocol --> nice to have but how to handle short term changes that srew everything? such as non responding interviewees and so on.. 

% won't of course change everything
% also: impact is based on humans, governance, social patterns and behaviour. Who would have guessed?!

“In humanitarian practice, the term "drought" is often used to refer to some socio-meteorological combination where water shortages produce stress on human and livelihood systems. Droughts are a function of the fragility of human systems, and they become disasters where systems cannot cope with deviations from the hydro-meteorological norm. It has been argued that droughts are particularly devastating when livelihood choices are strongly determined by the climate (e.g. the decision to grow certain crops, or traditional seasonal migration patterns) - for instance, if in a given year, the weather patterns are different than normal, those livelihoods are especially vulnerable to these changes. It has also been argued that droughts pose specific challenges to income generating activities marked by low productivity that are not able to take advantage of ‘good years’ in order to provide a buffer during ‘bad years’.” ([RCRC, 2020, p. 12](zotero://select/groups/4773535/items/UESIQTRJ)) ([pdf](zotero://open-pdf/groups/4773535/items/P5JPVZ97?page=12&annotation=C622YWIR))


% funding
Funding is not considered in this work, as it is generally out of scope of a Master Thesis to account for that. This is the task of the project management and should be covered by some grant.

%----------------------------------------------------------------------------------------
%	SECTION 2
%----------------------------------------------------------------------------------------

\section{Future Outlook}

local knowledge, gender, be on the ground and talk to people, pre-study, citizen science, crowdsourcing, crowdsensing, participant motivation, data accuracy, data quality,
% da fällt mir schon noch viel ein..

test phase -> pre-study in a small local area which is not too badly affected currently 


maybe formulation of hypothesis -> inductive stuff ends in this --> therefore, its falsification should be tested in future research.


is feasible? --> yes

basic framework/roadmap --> yes

practical pilot case study --> to be done.


further information about the water source - e.g. accessibility could be surveyed and potential integrations via codes need to be delineated


methodology worked, frameworks were good, time, interviews and resources were the constraints

% future outlook CBM


Beyond data collection, \autocite{gualazziniEWEAEarlyWarning2021} highlights the possibility to adopt a two-way communication capability to also distribute climate and weather forecasts to participants. Thus "communities have direct access to short-term weather forecasts and climate risk alerts to mitigate risk and protect livelihoods, while also providing information from the site" \autocite[20]{gualazziniEWEAEarlyWarning2021}. % ITIKI may also contribute to this statement


further open questions in all directions --> also look at my own tree-diagrams



% zu stage 2
The funding of a pilot study as a decisive factor could not be clarified in this framework, but does not affect the further conception. --> though there is an ongoing proposal which would build on top of this work (if granted).






% APPEDNIX
% include questions, questionnaires, transcripts and codes
The first interview came about through existing contacts of the project in which this work is embedded, and the interviewee was the project leader of the FbF approach in the \acrshort{srcs} (I1). In the further course, the CBS project manager on the Norwegian Red Cross side (I2) and the CBS manager on the Somali side (I3) were also interviewed. Between these two interviews, there was a second interview with the project manager of the \acrshort{srcs}' \acrshort{fbf} team (I1.2).




% das wäre doch mal ein feiner Outlook für future research.
How can the proposed approach for community-based participatory mapping and monitoring of water sources in a water-scarce and resource-limited setting in collaboration with a national non-governmental organization be effectively tested and evaluated to determine its effectiveness in improving water management and accessibility in underserved communities?