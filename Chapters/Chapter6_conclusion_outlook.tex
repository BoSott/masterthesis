% Chapter Template

\chapter{Conclusion} % Main chapter title

\label{ChapterX} % Change X to a consecutive number; for referencing this chapter elsewhere, use \ref{ChapterX}

%----------------------------------------------------------------------------------------
%	SECTION 1
%----------------------------------------------------------------------------------------

\section{Conclusion}

%-----------------------------------
%	SUBSECTION 1
%-----------------------------------
\subsection{Subsection 1}



" In particular, we realised the importance of a strong local partnership, based on mutual respect, clear agreed goals of collaboration and shared development interests [26, 27], but also the willingness and honesty to question decisions and discuss alternative approaches."


%-----------------------------------
%	SUBSECTION 2
%-----------------------------------

\subsection{Limitations}

% from the methodology 
Furthermore, case studies are frequently criticized for being excessively long, challenging to execute, and requiring significant documentation efforts \autocite{yinCaseStudyResearch1984}.
-->
it's true. Case studies are difficult and there is a lot to learn.
e.g. case study protocol --> nice to have but how to handle short term changes that srew everything? such as non responding interviewees and so on.. 

% won't of course change everything
% also: impact is based on humans, governance, social patterns and behaviour. Who would have guessed?!

“In humanitarian practice, the term "drought" is often used to refer to some socio-meteorological combination where water shortages produce stress on human and livelihood systems. Droughts are a function of the fragility of human systems, and they become disasters where systems cannot cope with deviations from the hydro-meteorological norm. It has been argued that droughts are particularly devastating when livelihood choices are strongly determined by the climate (e.g. the decision to grow certain crops, or traditional seasonal migration patterns) - for instance, if in a given year, the weather patterns are different than normal, those livelihoods are especially vulnerable to these changes. It has also been argued that droughts pose specific challenges to income generating activities marked by low productivity that are not able to take advantage of ‘good years’ in order to provide a buffer during ‘bad years’.” ([RCRC, 2020, p. 12](zotero://select/groups/4773535/items/UESIQTRJ)) ([pdf](zotero://open-pdf/groups/4773535/items/P5JPVZ97?page=12&annotation=C622YWIR))

%----------------------------------------------------------------------------------------
%	SECTION 2
%----------------------------------------------------------------------------------------

\section{Future Outlook}

local knowledge, gender, be on the ground and talk to people, pre-study, citizen science, crowdsourcing, crowdsensing, participant motivation, data accuracy, data quality,
% da fällt mir schon noch viel ein..

test phase -> pre-study in a small local area which is not too badly affected currently 


maybe formulation of hypothesis -> inductive stuff ends in this --> therefore, its falsification should be tested in future research.


is feasible? --> yes

basic framework/roadmap --> yes

practical pilot case study --> to be done.


further information about the water source - e.g. accessibility could be surveyed and potential integrations via codes need to be delineated


methodology worked, frameworks were good, time, interviews and resources were the constraints