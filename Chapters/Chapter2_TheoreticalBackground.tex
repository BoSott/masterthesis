% Chapter Template

\chapter{Theoretical Background} % Main chapter title

\label{Chapter2} % Change X to a consecutive number; for referencing this chapter elsewhere, use \ref{ChapterX}

%----------------------------------------------------------------------------------------
%	SECTION 1
%----------------------------------------------------------------------------------------

\section{Introduction}
The introduction should clearly establish the focus and purpose of the literature review.
highlight research gap + emphasize the timeliness
Locate your own research within the context of existing literature [very important!]. 
% reiterate central problem
% focus and purpose of literature review
% brief summery scholarly context
% highlight research gap
% emphasize timeliness


\subsection{reasoning}:
https://www.unwater.org/our-work/integrated-monitoring-initiative-sdg-6

"Stronger accountability: Data can communicate that work is being done and progress is happening. Data can enable greater transparency, which reduces inefficiency and corruption.
Attracting commitment and investments: Data can quantify problems and make it easier to communicate needs for political commitment and public and private investments.
Evidence-based decision-making: Data can inform policy- and decision-makers of where to focus efforts and which solutions are most effective, to ensure the greatest possible gains with existing resources.
Leaving no one behind: Disaggregated data can help identify specific groups or areas with unmet needs and higher levels of risk, to which interventions can be targeted."
https://www.unwater.org/our-work/integrated-monitoring-initiative-sdg-6/background

“Experts in trigger methodology have indicated a more appropriate strategy may be to build on tools that currently exist at the government level such as national drought monitoring systems. As such, the ideal is an iterative process with the ground level along with a technology push that creates new ways to analyse drought and drought risk.” ([RCRC, 2020, p. 28](zotero://select/groups/4773535/items/UESIQTRJ)) ([pdf](zotero://open-pdf/groups/4773535/items/P5JPVZ97?page=28&annotation=977VS8FC))

% DRM Strategic Plan General and Specific Objectives
“2.1.1 Specific Objective 1 Vulnerable communities’ resilience at SRCS target areas strengthened through anticipatory actions, response, recovery, and disaster risk reduction, and they can anticipate and effectively respond to and recover from disasters and crisis by 2026.” ([“SRCS DRM Strategic Plan_final 9thNovember 2021-converted.pdf”, p. 15](zotero://select/groups/4773535/items/LFCBRZLD)) ([pdf](zotero://open-pdf/groups/4773535/items/6IL6K72G?page=15&annotation=ZFICKZRA))

% --> even the RCRC is still looking for good triggers -> maybe water levels are a good way -> reasoning for this study (see background identical text)

“Countries in which less than 50\% of the population uses improved drinking water sources are all located in sub-Saharan Africa and Oceania 91-100\% 76-90\% 50-75\% <50\% insufficient data or not applicable Proportion of the population using improved drinking water sources in 2015 ■ 91–100\% ■ 76–90\% ■ 50–75\% ■ <50\% ■ INSUFFICIENT DATA OR NOT APPLICABLE” ([World Health Organization, 2016, p. 15](zotero://select/groups/4773535/items/KVAKZ9ZT)) ([pdf](zotero://open-pdf/groups/4773535/items/4STYK52H?page=14\&annotation=FBURDS4T))

“The methods by which the Joint Monitoring Programme (JMP) of WHO and UNICEF” ([Bartram et al., 2014, p. 8137](zotero://select/groups/4773535/items/6AWUJTW5)) ([pdf](zotero://open-pdf/groups/4773535/items/BFNSQGWS?page=1&annotation=UL4Q2I4V))
“substantial limitations: current methods do not address water quality, equity of access, or extra-household services.” ([Bartram et al., 2014, p. 8137](zotero://select/groups/4773535/items/6AWUJTW5)) ([pdf](zotero://open-pdf/groups/4773535/items/BFNSQGWS?page=1&annotation=TIPCEXGG))

request of the SRCS -> practically wanted
understanding the full scope and knowing which water sources are at what level and quality can help with management decisions and trigger certain events very locally
current challenges (?) what do I want to address? (number 1,2,3)
outline of the thesis/project

overcome limitations / incorporating recommendations: e.g. increased support/engagement of poeple who actually use the reports (e.g. SRCS officials) 
fill geographic and information gap ->

its about drought forecasting and early trigger but at the same time highly local and practical information where and which water sources are good and functioning and which are not. -> highly practical information. Some data exist but (mostly) outdated.
about getting local knowledge from SRCS Volunteers and their community as well as returning information about the bigger picture
in order to enhance the quality of data for managing severe droughts in Somaliland. (one short paragraph -> motivation)
provide number of weather stations in the area

--> problem: conclusion of existing sources, tools and forecasts - only macro/international level? or are there meso/micro forecasts available?
better understanding the forecasting and its implications on the ground are crucial. --> local information. tons of Volunteers but even more water sources. Continue with Crowdsensing? Implications?

--> highlight the pro of this work
"The highly localized information provided by observers can fill drought monitoring gaps by ground-truthing quantitative indicators and offering information in places where other monitoring tools may not exist. Overall, the research team found that strategic investments in time and funding can help fill in geographic and temporal gaps in drought monitoring information through volunteer observations."
https://www.drought.gov/news/research-confirms-role-citizen-science-contributions-drought-detection-and-monitoring and https://doi.org/10.1175/BAMS-D-21-0157.1

“Citizen science programmes are promising cost-efficient methods to monitor environmental resources, which make them especially suitable for low-income countries to overcome their sparse data resolution.” ([Weeser et al., 2018, p. 1598](zotero://select/groups/4773535/items/SFA2MLHC)) ([pdf](zotero://open-pdf/groups/4773535/items/GP79FHFC?page=9&annotation=4E9JCTQ5))
“Since today's citizen science studies are mostly located in high-income countries, we are enthusiastic to motivate the scientific community to conduct citizen science studies in low-income countries.” ([Weeser et al., 2018, p. 1598](zotero://select/groups/4773535/items/SFA2MLHC)) ([pdf](zotero://open-pdf/groups/4773535/items/GP79FHFC?page=9&annotation=TYD7Q2ZD))

% also in case study
“The number of affected people will be 1,200,420 persons across all the six main regions in Somaliland. The top priority needs of the people affected to date are mainly water (70{\%), Food (21\%) and Health (9\%).” ([National Drought Committee, 2022, p. 3](zotero://select/groups/4773535/items/7XJRE6LM)) ([pdf](zotero://open-pdf/groups/4773535/items/2F59E4UZ?page=3&annotation=8JZVBSM6))

%-----------------------------------
%	SUBSECTION 1.1
%-----------------------------------
\subsection{relevance}
relevance
Somalia is komplett am krepieren because of a multi-year long drought - ...\% of damage/conflicts etc. is based on droughts. severe shit! --> first case study introduction (geographically, socially, etc.) 
scope: of the EAP/FbF project

%-----------------------------------
%	SUBSECTION 1.2
%-----------------------------------
\subsection{Subsection 2}

%----------------------------------------------------------------------------------------
%	SECTION 2 Drought & drought Index
%----------------------------------------------------------------------------------------


\section{Drought & drought Index}


%-----------------------------------
%	SUBSECTION 2.1
%-----------------------------------
\subsection{Drought definitions}

“I. Definitions of Droughts” ([RCRC, 2020, p. 10](zotero://select/groups/4773535/items/UESIQTRJ)) ([pdf](zotero://open-pdf/groups/4773535/items/P5JPVZ97?page=10&annotation=4EVXYNL7))
%--> meteorological, hydrological, agricultural, socio-economic

what drought is (definition)
drought monitoring but not only physical indicators but socials as well (water source accessibility) “However, assessments focused only on physical variables and processes fail to capture why drought matters, in other words, how social, economic, and ecological systems are affected (i.e., impacts) (Redmond 2002; Van Loon et al. 2016; Wilhite and Glantz 1985).” ([Lackstrom et al., 2022, p. 3](zotero://select/groups/4773535/items/YI366LQY)) ([pdf](zotero://open-pdf/groups/4773535/items/3JTQ72UN?page=3\&annotation=72WYIE7B))
current monitoring approaches e.g. (“the extent to which volunteers’ assessments of dry-to-wet conditions correspond to objective drought indicators (EDDI, SPI, SPEI) typically employed for monitoring drought” ([Lackstrom et al., 2022, p. 2](zotero://select/groups/4773535/items/YI366LQY)) ([pdf](zotero://open-pdf/groups/4773535/items/3JTQ72UN?page=2\&annotation=DQ4ZNIRS)))
“quantitative indicators (namely the SPEI, SPI, and EDDI).” ([Lackstrom et al., 2022, p. 26](zotero://select/groups/4773535/items/YI366LQY)) ([pdf](zotero://open-pdf/groups/4773535/items/3JTQ72UN?page=26\&annotation=G4KIJYUR))

“Monitoring Drought with the Combined Drought Index in Kenya” ([Balint et al., 2013, p. 1](zotero://select/groups/4773535/items/C6T6K9AP)) ([pdf](zotero://open-pdf/groups/4773535/items/ZT9SQQ2M?page=1\&annotation=DMSJ4J63))
introduction is good.

drought definitions, drought types (meteorological, agricultural, hydrological and estimation)

“Droughts are also compounding hazards: their impacts grow overtime and can overlap with other hazards.” ([RCRC, 2020, p. 22](zotero://select/groups/4773535/items/UESIQTRJ)) ([pdf](zotero://open-pdf/groups/4773535/items/P5JPVZ97?page=22&annotation=XQFR49HQ))

“2 Local knowledge in drought monitoring: an introduction to the literature review” ([Giordano et al., 2013, p. 526](zotero://select/groups/4773535/items/B7LM5ZR4)) ([pdf](zotero://open-pdf/groups/4773535/items/7I66DBIK?page=4&annotation=Z33M5FLQ))

“2.3. Indigenous Knowledge on Droughts” ([Masinde et al., 2013, p. 2](zotero://select/groups/4773535/items/M45MLGWC)) ([pdf](zotero://open-pdf/groups/4773535/items/LG6E76P4?page=2&annotation=XHU34W23))
%-----------------------------------
%	SUBSECTION 2.2
%-----------------------------------

\subsection{Drought Indices and Indicators}
% multi hazard
% longterm
% different drought difinitions

"Drought indicator" and "drought index" are two related but distinct terms used in the field of drought monitoring and assessment.

A drought indicator is a variable or set of variables that provide information on the current or potential occurrence of drought. For example, a drought indicator might be a measure of soil moisture or streamflow that can be used to assess drought conditions.

A drought index, on the other hand, is a composite measure that combines multiple drought indicators to provide a more comprehensive assessment of drought severity and duration. Drought indices are typically based on mathematical formulas that take into account various environmental factors, such as precipitation, temperature, and soil moisture.

In summary, a drought indicator is a single variable that provides information on drought conditions, while a drought index is a composite measure that combines multiple indicators to provide a more complete picture of drought severity and duration. While both terms are used in the context of drought monitoring and assessment, drought indices are generally considered to be more comprehensive and reliable measures of drought.

drought indicator (what exists so far? -> pre study)

“A. Hazard-related indicators and indices” ([RCRC, 2020, p. 15](zotero://select/groups/4773535/items/UESIQTRJ)) ([pdf](zotero://open-pdf/groups/4773535/items/P5JPVZ97?page=15&annotation=KZ5ITR9D))

“B. Vulnerability-related indicators” ([RCRC, 2020, p. 17](zotero://select/groups/4773535/items/UESIQTRJ)) ([pdf](zotero://open-pdf/groups/4773535/items/P5JPVZ97?page=17&annotation=BTLXTKJG))

% List of drought indicators
https://heigit.atlassian.net/wiki/spaces/FIS/pages/1704096/Indices

in general: https://edo.jrc.ec.europa.eu/edov2/php/index.php?id=1010

explanation and overview about drought indices “Drought prediction based on SPI and SPEI with varying timescales using LSTM recurrent neural network” ([Poornima and Pushpalatha, 2019, p. 8399](zotero://select/groups/4773535/items/NJME9MIM)) ([pdf](zotero://open-pdf/groups/4773535/items/48LYF6PR?page=1&annotation=FV2JBM9I))

SPI - standardized Precipitation Index (most widely used drought indices (DI)) 
https://climatedataguide.ucar.edu/climate-data/standardized-precipitation-index-spi
http://iridl.ldeo.columbia.edu/maproom/Global/Drought/Global/index.html

"The experimental Global Gridded Standardized Precipitation Index (SPI) is derived from the NOAA CMORPH dataset and includes timescales of 1, 3, 6 and 9 months.  The NOAA CMORPH precipitation dataset is a gridded dataset derived from combining numerous microwave-based estimates from low orbiter satellites." https://www.drought.gov/data-maps-tools/global-gridded-standardized-precipitation-index-spi-cmorph-daily + related publications

EDDI (Evaporative Demand Drought Index) ("examines how anomalous the atmospheric evaporative demand (E0; also known as "the thirst of the atmosphere) is for a given location and across a time period of interest." \& "EDDI has been shown to offer early warning of drought stress relative to current operational drought indicators, such as the U.S. Drought Monitor (USDM). A particular strength of EDDI is in capturing the precursor signals of water stress at weekly to monthly timescales, which makes EDDI a potent tool for drought preparedness at those timescales. EDDI also uses the same classification scheme as the USDM to define drought conditions, so it is easy to read EDDI maps."https://www.drought.gov/data-maps-tools/evaporative-demand-drought-index-eddi-subseasonal-forecasts)

SPEI (Standardised Precipitation-Evapotranspiration Index https://spei.csic.es/)
"The SPEI is a multiscalar drought index based on climatic data. It can be used for determining the onset, duration and magnitude of drought conditions with respect to normal conditions in a variety of natural and managed systems such as crops, ecosystems, rivers, water resources, etc."https://spei.csic.es/

key strength and key limitations https://climatedataguide.ucar.edu/climate-data/standardized-precipitation-evapotranspiration-index-spei


more indices: https://www.droughtmanagement.info/indices/

https://droughtmonitor.unl.edu/


"One example of a drought index is the Standardized Precipitation Index (SPI), which is a widely used measure of meteorological drought.

The SPI uses precipitation data to assess how much rainfall is available relative to the long-term average for a given location and time period. The SPI is calculated by standardizing the observed precipitation data over a specified time period (e.g., 3 months, 6 months, 12 months) to create a normal distribution with a mean of 0 and a standard deviation of 1.

Positive SPI values indicate wetter than average conditions, while negative SPI values indicate drier than average conditions. The magnitude of the SPI value indicates the severity of the drought, with more negative values indicating more severe drought conditions. The SPI is a useful tool for monitoring and predicting meteorological droughts, as it can be calculated using readily available precipitation data and provides a standardized measure that can be compared across different locations and time periods." ChatGPT
%----------------------------------------------------------------------------------------
%	SECTION 3 Water Scarcity
%----------------------------------------------------------------------------------------


\section{Water Access and Water Scarcity}
% human induced water shortage component

%-----------------------------------
%	SUBSECTION 3.1
%-----------------------------------
\subsection{Water Access}
“There are three main facets to urban water access: availability, quality, and affordability.” ([Mitlin et al., p. 11](zotero://select/groups/4773535/items/KAM9REZR)) ([pdf](zotero://open-pdf/groups/4773535/items/BM6BU5UR?page=11&annotation=44IJC8RC))

%-----------------------------------
%	SUBSECTION 3.2
%-----------------------------------

\subsection{Water Scarcity}

take this definition of water scarcity
“2. Defining water scarcity” ([“Coping with water scarcity: an action framework for agriculture and food security”, 2012, p. v](zotero://select/groups/4773535/items/R7VEIVF3)) ([pdf](zotero://open-pdf/groups/4773535/items/M6HRGVGP?page=23&annotation=JFSBH7ME))


Distinctions between water scarcity and drought:

“Table 1. Characteristics and impacts of water scarcity and drought Water scarcity Drought Length Long-term to permanent Temporary (weeks to multiyear) Driving forces Demand–supply imbalance, human-driven, and/or natural (overexploitation, pollution). Climate change can impact both supply and demand Natural climate variability which can be modified/amplified by climate change Potential impacts Restricted water availability, environmental degradation, desertification, exacerbated inequalities in access to water resources, potential competition Water shortages, competition, environmental degradation Measures Long-term IWRM to bring supply and demand back into sustainable balance Integrated drought management, including: (1) monitoring and early warning; (2) vulnerability and impact assessment; and (3) risk mitigation, preparedness and response Source: adapted from Hohenwallner et al. (2011) DROUGHT AND WATER SCARCITY – DEFINITIONS AND CHARACTERISTICS” ([pdf](zotero://open-pdf/groups/4773535/items/JM82W3ZF?page=9&annotation=E3EQRILA))

“Strategies for coping with droughts and water scarcity involve proactive approaches to minimize adverse effects. These responses and coping strategies should be part of overall integrated drought and water resource management strategies, including sustainable water resource development. Opportunities for proactive approaches include strengthening early alert systems, risk mitigation measures, and long-term adaptation strategies to build climate, economic, and societal resilience for the well-being of future generations. In countries and regions prone to drought and water scarcity, risk management and resilience are important for sustaining and enhancing future quality of life.” ([pdf](zotero://open-pdf/groups/4773535/items/JM82W3ZF?page=7&annotation=H4F73GR2))


“Understanding responses to climate-related water scarcity in Africa” ([Leal Filho et al., 2022, p. 1](zotero://select/groups/4773535/items/6TPZLH52)) ([pdf](zotero://open-pdf/groups/4773535/items/6LRJKBTR?page=1&annotation=4C7X6MGY))
--> impacts of water scarcity on a variety of other factors
“These findings underscore the dangers of separating (or ringfencing) responses to water scarcity from competing challenges to food security, urbanization, desertification, and human or state security.” ([Leal Filho et al., 2022, p. 11](zotero://select/groups/4773535/items/6TPZLH52)) ([pdf](zotero://open-pdf/groups/4773535/items/6LRJKBTR?page=11&annotation=DG37MVHB))
“The terms ‘water scarcity’ and ‘drought’ are often used interchangeably, despite their subtle but important differences with regards to water management.” ([pdf](zotero://open-pdf/groups/4773535/items/JM82W3ZF?page=8&annotation=8SQMLMKD))


“This means that, even if drought is a driver of water scarcity (e.g. reduction in rainfall), there is always a human dimension to the reduction in the natural water supply (European Commission, 2019).” ([pdf](zotero://open-pdf/groups/4773535/items/JM82W3ZF?page=10&annotation=UFNRBT3J))

“Water scarcity, as a supply/demand-driven and natural and/or human-made phenomenon, is one of the greatest challenges of the twenty-first century” ([pdf](zotero://open-pdf/groups/4773535/items/JM82W3ZF?page=13&annotation=WZB8I8FY))

“Likewise, water management approaches that focus only on drought when it occurs will have missed significant opportunities to reduce drought risk.” ([pdf](zotero://open-pdf/groups/4773535/items/JM82W3ZF?page=16&annotation=GVHFHKTI))


!!!!!
“Adequate and reliable weather, water, and climate data and applications are needed to monitor available water resources and provide actionable early warning for water scarcity and drought conditions.” ([pdf](zotero://open-pdf/groups/4773535/items/JM82W3ZF?page=17&annotation=67GM7DYS))
“as well as improved capacities in collecting hydrological data. Improved interaction with stakeholders is crucial to promote better tailored information products.” ([pdf](zotero://open-pdf/groups/4773535/items/JM82W3ZF?page=18&annotation=CBKCBSZV))


IWRM??

%----------------------------------------------------------------------------------------
%	SECTION 4 FbF, EAP, AA & Early Warning
%----------------------------------------------------------------------------------------


\section{FbF, EAP, AA & Early Warning + trigger}
% one possible solution to prevent impact
% FbF
% IFRC & RCRC
% EAPs
% Early Warning/Actions &  Anticipatory Actions
% triggered by forecast

%-----------------------------------
%	SUBSECTION 4.1
%-----------------------------------
\subsection{FbF and paradigm shift}
--> FbF and EAP what is what and so on --> in context of this work


%% FbA Challenges
“First, FbA for drought presents a challenge of framing and definition” ([RCRC, 2020, p. 24](zotero://select/groups/4773535/items/UESIQTRJ)) ([pdf](zotero://open-pdf/groups/4773535/items/P5JPVZ97?page=24&annotation=F2K3MP3C))

“FbA for drought is contextually challenging” ([RCRC, 2020, p. 24](zotero://select/groups/4773535/items/UESIQTRJ)) ([pdf](zotero://open-pdf/groups/4773535/items/P5JPVZ97?page=24&annotation=EQTN72CS))

“structural challenges” ([RCRC, 2020, p. 24](zotero://select/groups/4773535/items/UESIQTRJ)) ([pdf](zotero://open-pdf/groups/4773535/items/P5JPVZ97?page=24&annotation=EXGZA744))

%-----------------------------------
%	SUBSECTION 4.2
%-----------------------------------

\subsection{EAP and drought specifics}
%% other RCRC protocols and EAPs
“FORECAST-BASED FINANCING AND EARLY ACTION FOR DROUGHT” ([pdf](zotero://open-pdf/groups/4773535/items/P5JPVZ97?page=1&annotation=IL955QPI))

%% key differences between fast-onset disasters and drought.
“Uniqueness of droughts to RCRC FbA experience” ([pdf](zotero://open-pdf/groups/4773535/items/P5JPVZ97?page=19&annotation=6MZDVD5E))

building on the above mentioned blocks of drought forecasts, local knowledge co-production and water source monitoring, 

%-----------------------------------
%	SUBSECTION 4.2
%-----------------------------------

\subsection{Anticipatory Actions and Early Warning}
%% Suggested AA
The feasibility study identifies a number of high potential early actions in response to anticipated drought, including:

Water storage rehabilitation – berked repairs and re-charging can be undertaken at a reasonable cost with available skills within the lead time afforded by a drought forecast; (https://heigit.atlassian.net/wiki/spaces/FIS/pages/1704186/Background+Information#BackgroundInformation-SuggestedAA)



%% DEWS Drougth Early Warning Systems
"The study concluded that information from DEWS was of good quality although it was poorly disseminated. There is need to enhance the timeliness of information dissemination if the system is to effectively enhance community preparedness to cope the effects of drought." Akwango, D., Obaa, B. B., Turyahabwe, N., Baguma, Y., & Egeru, A. (2017). Quality and dissemination of information from a drought early warning system in Karamoja sub-region, Uganda. Journal of Arid Environments, 145, 69–80. https://doi.org/10.1016/j.jaridenv.2017.05.010 (already in zotero)




%-----------------------------------
%	SUBSECTION 4.3 Trigger selection
%-----------------------------------

\subsection{Trigger selection}
“Given the different layers of complexity with drought, different types of triggers may be required beyond what is often used in EAP development. For instance, unconventional triggers for FbA for drought could include metrics such as staple food prices, percentages of crop failure, and other elements of food security early warning systems.” ([RCRC, 2020, p. 30](zotero://select/groups/4773535/items/UESIQTRJ)) ([pdf](zotero://open-pdf/groups/4773535/items/P5JPVZ97?page=30&annotation=JZV26DPP))
% --> even the RCRC is still looking for good triggers -> maybe water levels are a good way -> reasoning for this study


%% Trigger:
“Triggers are mainly combination of hydro-meteorological forecast combined with exposure and vulnerability data” ([pdf](zotero://open-pdf/groups/4773535/items/P5JPVZ97?page=19&annotation=ILGX4MS6))

% --> water level monitoring is one trigger for in the chain of triggers (right before the last action so to say)
% compare: “Figure 5. Contrasting the timeline between fast-onset hazards and droughts” ([RCRC, 2020, p. 20](zotero://select/groups/4773535/items/UESIQTRJ)) ([pdf](zotero://open-pdf/groups/4773535/items/P5JPVZ97?page=20&annotation=Q4LRCBC4))
“As explained in the previous section, droughts do not have clear start and end dates, but the timing of below average rainfall matters deeply. Particularly for crops and forage, a dry spell at the beginning of the planting season can be particularly devastating for crop yields. As such, in order to act in anticipation, triggering systems must be based on monitoring and forecasting at the right times to capture these events and act early.” ([RCRC, 2020, p. 29](zotero://select/groups/4773535/items/UESIQTRJ)) ([pdf](zotero://open-pdf/groups/4773535/items/P5JPVZ97?page=29&annotation=KIUB7HJP))

%% critical water level
“Primary water source condition Observation of the status of the largest rainwater catchment in the area (at the end of rainy seasons) More than half-full (75%) or full Half-full (50%) at the end of rainy season Less than half-full (25%)/empty” ([Gualazzini, 2021, p. 6](zotero://select/groups/4773535/items/BWDYDL8T)) ([pdf](zotero://open-pdf/groups/4773535/items/8U5XVU5K?page=6&annotation=HMGNP355))

%% staggered triggering system
“staggered triggering system (see Annex 3)” ([RCRC, 2020, p. 26](zotero://select/groups/4773535/items/UESIQTRJ)) ([pdf](zotero://open-pdf/groups/4773535/items/P5JPVZ97?page=26&annotation=4JB7JVDV))

“However, with the longer outlook of seasonal forecasts, comes more uncertainty, less granularity and lower accuracy of the prediction.” ([RCRC, 2020, p. 26](zotero://select/groups/4773535/items/UESIQTRJ)) ([pdf](zotero://open-pdf/groups/4773535/items/P5JPVZ97?page=26&annotation=55RDIUHY))

“Notably, situations and forecasts can change throughout the seasons. This presents a challenge for the development of robust triggering systems and involves a heightened risk of false alarms, which have been shown to quickly erode the trust necessary for humanitarian presence. For these reasons, a clear and comprehensive understanding of the seasonality of the region’s climate, and of context-specific tipping points, is a fundamental first step to any FbA for drought program.” ([RCRC, 2020, p. 26](zotero://select/groups/4773535/items/UESIQTRJ)) ([pdf](zotero://open-pdf/groups/4773535/items/P5JPVZ97?page=26&annotation=IY3B3635))

--> this work -> on the ground indicators/impacts -> low risk for false alarms

%% some conditions for an effective trigger for FbA for drought
“1. Sufficient historical data on past droughts, their causes and impacts 

2. Identified drivers of rainfall predictability in the region (if forecasts are going to be used, and not triggering entirely on observations in anticipation of the impacts) or else sufficient rainfall observations 

3. Sufficient knowledge of livelihood profiles in the region and knowledge of differential impacts of drought conditions on livelihood groups.” ([RCRC, 2020, p. 25](zotero://select/groups/4773535/items/UESIQTRJ)) ([pdf](zotero://open-pdf/groups/4773535/items/P5JPVZ97?page=25&annotation=AJ4JI2QH))




“Thinking outside the box in terms of both hydro-meteorological and socio-economic indicators could be particularly useful” ([RCRC, 2020, p. 31](zotero://select/groups/4773535/items/UESIQTRJ)) ([pdf](zotero://open-pdf/groups/4773535/items/P5JPVZ97?page=31&annotation=GNZJ3FR5))


“Drought Severity = Intensity x Duration x Magnitude x Frequency” ([RCRC, 2020, p. 13](zotero://select/groups/4773535/items/UESIQTRJ)) ([pdf](zotero://open-pdf/groups/4773535/items/P5JPVZ97?page=13&annotation=MKU448SN))


good, precise definitions of drought in an easy way
“Droughts vary by intensity, duration, timing, and geographical coverage, creating conditions of limited moisture availability to a potentially damaging extent.” ([pdf](zotero://open-pdf/groups/4773535/items/JM82W3ZF?page=10&annotation=ZR4SQ4LB))

“Many practitioners and experts interviewed for this work suggested that a staggered triggering system, at different lead times and for different early actions may be the most appropriate to tackle drought” ([RCRC, 2020, p. 31](zotero://select/groups/4773535/items/UESIQTRJ)) ([pdf](zotero://open-pdf/groups/4773535/items/P5JPVZ97?page=31&annotation=X34YUJMF))


“Uniqueness of droughts to RCRC FbA experience” ([RCRC, 2020, p. 19](zotero://select/groups/4773535/items/UESIQTRJ)) ([pdf](zotero://open-pdf/groups/4773535/items/P5JPVZ97?page=19&annotation=6MZDVD5E))

“A. Long and Unclear Temporal Framing” ([RCRC, 2020, p. 20](zotero://select/groups/4773535/items/UESIQTRJ)) ([pdf](zotero://open-pdf/groups/4773535/items/P5JPVZ97?page=20&annotation=KJXT7VD5))

“The impacts of drought are therefore far reaching, arguably more so than slow-onset hazards, and arrive at different times for different groups. Sifting through these impacts to identify which ones to address through early action (and when to begin them) can be particularly difficult.” ([RCRC, 2020, p. 23](zotero://select/groups/4773535/items/UESIQTRJ)) ([pdf](zotero://open-pdf/groups/4773535/items/P5JPVZ97?page=23&annotation=XNFC6ZA8))

%----------------------------------------------------------------------------------------
%	SECTION 5 Forecasting (SPI, EDDI, SPEI) vs. impact based
%----------------------------------------------------------------------------------------


\section{Forecasting (SPI, EDDI, SPEI) vs. impact based}
% based on the triggers, those are the forecasts for now -> strengths and limitations which trigger?

% presentation of different forecasts and which are used
% well established, published by RCRC Climate Center etc.
% mostly based on satellite analysis -> limitation: global

% impact based forecasts
% difficult to assess
% possibilities and opportunities

% which trigger?

%-----------------------------------
%	SUBSECTION 5.1
%-----------------------------------
\subsection{Forecasts}
which forecasts are selected by the EAP pre-study?

current challenges for utilisation of forecasting systems: scarse coverage of weather stations and poor utilisation by the farmers often due to bad dissemination channels  (too coarse, too unreliable)
Somalia Multi-hazard Early Warning Centre under Ministry of Humanitarian Affairs & Disaster Management
based on open data sources (coverage: Somalia)

FAO Somalia Water and Land Information Management (SWALIM) coverage: all regions - source: 100 manual rainfall stations, eight synoptic weather stations and 11 automatic weather stations in Somalia

“Forecast Menu for SRCS” ([Somali Red Crescent Society, 2022, p. 43](zotero://select/groups/4773535/items/FZ6BJHJA)) ([pdf](zotero://open-pdf/groups/4773535/items/RJKNZZZ2?page=43&annotation=5A354LG7))
identified by the SRCS pre-study

“Table 1. Comparisons between indigenous knowledge-based seasonal forecasts and seasonal climate forecasts (adopted from Ziervogel and Opere 2010). Indigenous knowledge-based seasonal forecasts Seasonal climate forecasts Use biophysical indicators of the environment as well as spiritual methods Use of weather and climate models of measurable meteorological data Forecast methods are seldom documented Forecast methods are more developed and documented Up-scaling and down-scaling are usually complex Up-scaling and down-scaling are relatively simple Application of forecast output is less developed Application of forecast output is more developed Communication is usually oral Communication is usually written Explanation is based on spiritual and social values Explanation is theoretical Taught by observation and experience Taught through lectures and readings” ([Masinde and Bagula, 2012, p. 280](zotero://select/groups/4773535/items/EW9XSSZP)) ([pdf](zotero://open-pdf/groups/4773535/items/3WQ4S9PE?page=7&annotation=6XCISBM2))

“Indigenous knowledge within an early warning system for droughts” ([Masinde and Bagula, 2012, p. 282](zotero://select/groups/4773535/items/EW9XSSZP)) ([pdf](zotero://open-pdf/groups/4773535/items/3WQ4S9PE?page=9&annotation=8Z9A9AW8))

“B. Drought Forecasting in Sub-Saharan Africa” ([Masinde and Thothela, 2019, p. 304](zotero://select/groups/4773535/items/6D52T883)) ([pdf](zotero://open-pdf/groups/4773535/items/KLLQKDG2?page=2&annotation=ZQSDUEMX))

there are more! Look into it.!

%% overview about all potential meteorological drought indices
https://heigit.atlassian.net/wiki/spaces/FIS/pages/1704096/Indices

https://cdi.faoswalim.org/index/cdi
%-----------------------------------
%	SUBSECTION 5.2
%-----------------------------------

\subsection{Impact based}
% could also go into drought chapter
Why predict climate hazards if we need to understand impacts? Putting humans back into the drought equation

“There has been little effort to align the spatiotemporal granularity of socioeconomic assessments with the granularity of weather or climate monitoring.” ([Enenkel et al., 2020, p. 1161](zotero://select/groups/4773535/items/RX575C79)) ([pdf](zotero://open-pdf/groups/4773535/items/XD499UNK?page=1&annotation=QBTLFCXM))
“we highlight the need to collect and analyze environmental and socioeconomic data together and discuss novel strategies for coordinated data collection via mobile technologies from a drought risk management perspective.” ([Enenkel et al., 2020, p. 1161](zotero://select/groups/4773535/items/RX575C79)) ([pdf](zotero://open-pdf/groups/4773535/items/XD499UNK?page=1&annotation=9BUBHWNB))

“but questions related to coping capacities, migration, poverty, water supply, access to food and markets, or political conflict remain unanswered or are even decoupled from routine drought risk assessments” ([Enenkel et al., 2020, p. 1162](zotero://select/groups/4773535/items/RX575C79)) ([pdf](zotero://open-pdf/groups/4773535/items/XD499UNK?page=2&annotation=HE48ZWFA))

“The handbook of drought indicators (Svoboda et al. 2016) lists more than 50 drought indices. Not a single one of these indices connects climate anomalies to socioeconomic vulnerabilities,” ([Enenkel et al., 2020, p. 1163](zotero://select/groups/4773535/items/RX575C79)) ([pdf](zotero://open-pdf/groups/4773535/items/XD499UNK?page=3&annotation=EDKZFHJX))
--> there were more indicators in the following year, but those were limited due to availibiltiy of socioeconomic data

“In a region where migration is one of the main coping mechanisms for drought, a targeted survey focusing on the early detection of migration movements would help mobilize the timely allocation of resources by humanitarian decision-makers or even the mitigation of drought impacts.” ([Enenkel et al., 2020, p. 1167](zotero://select/groups/4773535/items/RX575C79)) ([pdf](zotero://open-pdf/groups/4773535/items/XD499UNK?page=7&annotation=N9FRDA9C))

+ humanitarian workers could focus on response rather than data acquisition (my idea)

“In addition, the impact of agricultural conditions that are not necessarily related to climate shocks, but to other factors such as pests or social conflict, could easily be monitored and used to issue early warnings and raise emergency funds before any kind of impact on crops or socioeconomic conditions is visible.” ([Enenkel et al., 2020, p. 1168](zotero://select/groups/4773535/items/RX575C79)) ([pdf](zotero://open-pdf/groups/4773535/items/XD499UNK?page=8&annotation=MDXAXIJY))

%% Impacts
“Impacts are immediately visible and localised. Impact data is collected via standardized processes such as Damage and Needs Assessments.” ([pdf](zotero://open-pdf/groups/4773535/items/P5JPVZ97?page=19&annotation=6QHQH3CN))


% what (kind of) information is needed?

“Towards drought impact-based forecasting in a multi-hazard context” ([Boult et al., 2022, p. 1](zotero://select/groups/4773535/items/B2AQSTYL)) ([pdf](zotero://open-pdf/groups/4773535/items/W9TFLH43?page=1&annotation=GL47JLV7))
The framework for drought impact-based forecasting proposed in the paper "Towards drought impact-based forecasting in a multi-hazard context" aims to provide more comprehensive and actionable information for decision-making in the context of multi-hazard risk management. The framework considers both biophysical and socioeconomic factors, including:

Exposure: The extent to which a population or system is exposed to drought-related hazards, such as the availability of water resources, the dependence on rain-fed agriculture, and the presence of infrastructure that may be vulnerable to drought-related impacts.

Vulnerability: The susceptibility of a population or system to drought-related impacts, considering factors such as the social, economic, and environmental characteristics of the affected area, as well as the capacity of the affected community or system to cope with and recover from drought-related impacts.

Coping capacities: The ability of a population or system to mitigate or adapt to drought-related impacts, including the availability of resources, the presence of social networks and support systems, and the ability to access and utilize information and technology.

The authors argue that considering these factors in the context of drought forecasting can provide more useful and actionable information for decision-making, and recommend further research on the development and implementation of drought impact-based forecasting systems, as well as their integration into existing decision-making frameworks.


“∣Potential impact (x, t)∣≡∣hazard (x, t)∣∪∣vulnerability (x, t) ∣∪∣exposure (x, t)∣” ([Boult et al., 2022, p. 2](zotero://select/groups/4773535/items/B2AQSTYL)) ([pdf](zotero://open-pdf/groups/4773535/items/W9TFLH43?page=2&annotation=6M48XVJI))

“In theory, focusing on what the weather will do, rather than what the weather will be, enables decision makers to plan and implement targeted preparatory actions to better reduce hazard impacts (Harrowsmith et al., 2020).” ([Boult et al., 2022, p. 2](zotero://select/groups/4773535/items/B2AQSTYL)) ([pdf](zotero://open-pdf/groups/4773535/items/W9TFLH43?page=2&annotation=NSLE7NL6))

“However, establishing a functional relationship can be difficult for a number of reasons” ([Boult et al., 2022, p. 3](zotero://select/groups/4773535/items/B2AQSTYL)) ([pdf](zotero://open-pdf/groups/4773535/items/W9TFLH43?page=3&annotation=ANJJD878))

% --> direct information would be better
% --> in case of multi-hazard EWS
“Drought presents a clear case for the multi-hazardapproach. Because it is a slow-onsetevent, vulnerability to drought is susceptible to the influenceof concurrent hazards and non-biophysical events, meaning multi-hazards must be considered if drought interventions are to be effectivelytargeted. Moreover, the intrinsic predictability and slow onset provides a timeframe in which early actions can be adapted in response to multi-hazard influences (Boult et al., 2020).” ([Boult et al., 2022, p. 2](zotero://select/groups/4773535/items/B2AQSTYL)) ([pdf](zotero://open-pdf/groups/4773535/items/W9TFLH43?page=2&annotation=QZ7UC62U))

“2.Challenges for drought IbF” ([Boult et al., 2022, p. 3](zotero://select/groups/4773535/items/B2AQSTYL)) ([pdf](zotero://open-pdf/groups/4773535/items/W9TFLH43?page=3&annotation=7B52FIS8)) (Impact based Forecast):
The main challenges for drought impact-based forecasting (IbF) as described in this text include:

Difficulty in directly forecasting drought impacts: It can be challenging to establish a functional relationship between drought hazards and impacts, as past humanitarian aid and development may have weakened this link, and sufficient impact data may be unavailable. In addition, the relationship between hazard severity and impact is mediated by vulnerability, which is not always fully understood.

Issues with predefined systems: Predefined IbF systems that rely on predefined trigger thresholds and pre-agreed actions may not be able to adapt to changing vulnerabilities over time. These systems also rely on expert judgement during system development, which may be subject to bias and may not accurately reflect the complexity of the system.

“Without scope to accommodate dynamic vulnerabilities, actions cannot be effectively targeted or may prove ineffective.” ([Boult et al., 2022, p. 4](zotero://select/groups/4773535/items/B2AQSTYL)) ([pdf](zotero://open-pdf/groups/4773535/items/W9TFLH43?page=4&annotation=YYURM2E3))

Lack of coordination between sectors: Drought impacts often involve multiple sectors (e.g. agriculture, water, health), but there may be a lack of coordination and integration between these sectors in the design and implementation of IbF systems. This can lead to inefficiencies and gaps in the response to drought impacts.

Limited capacity for impact-based decision making: There may be a lack of capacity or resources for impact-based decision making at the local level, particularly in low-income countries where data availability and quality may be limited.

Inability to fully capture the complexity of drought impacts: Drought impacts often involve complex and dynamic interactions between hazards, vulnerabilities, and exposures, and may be exacerbated by non-climate factors such as conflict or economic conditions. It can be difficult to fully capture and understand these interactions in IbF systems.

“Moreover, if the complex relationships linking multihazards to multi-impacts obscure attribution of particular impacts to particular hazards, hazard-focused organisations may be limited by their institutional mandate and thus unable to act.” ([Boult et al., 2022, p. 4](zotero://select/groups/4773535/items/B2AQSTYL)) ([pdf](zotero://open-pdf/groups/4773535/items/W9TFLH43?page=4&annotation=2XG6YP8D))

interesting!!! Fig. 1. A hybrid framework for multi-hazard IbF. Refer to the main text for a definition of numbers. Black arrows and numbers: components common across predefined humanitarian IbF systems. Blue arrows and numbers: real-time components. Grids represent spatially varying values. Darker reds indicate higher values of risk, vulnerability, and thresholds. In this example, despite only low to moderate risk in the southwest square, increased dynamic vulnerability lowers the threshold for action, resulting in triggering. Meanwhile, reduced vulnerability in the northern squares elevates trigger thresholds, so the northeast square no longer triggers. (For interpretation of the references to colour in this figure legend, the reader is referred to the web version of this article.)

A hydrometeorological forecast indicating the likelihood of drought occurring is combined with a predefined assessment of static vulnerability to determine risk. Where static vulnerability is higher (Fig. 1: northern squares), trigger thresholds are lower. 2) Risk is compared to agreed-upon thresholds for action. 3) If risk is greater than or equal to the threshold, early action is triggered to mitigate the worst impacts of drought.

Proposed Framework:
--> 1)“A hydrometeorological forecast indicating the likelihood of drought occurring is combined with a predefined assessment of static vulnerability to determine risk. Where static vulnerability is higher (Fig. 1: northern squares), trigger thresholds are lower.
2) Risk is compared to agreed-upon thresholds for action.
3) If risk is greater than or equal to the threshold, early action is triggered to mitigate the worst impacts of drought.” ([Boult et al., 2022, p. 5](zotero://select/groups/4773535/items/B2AQSTYL)) ([pdf](zotero://open-pdf/groups/4773535/items/W9TFLH43?page=5&annotation=KDCBHGC6))

“We then propose a number of components to account for dynamic vulnerabilities caused by concurrent hazards:
4)Expert judgement is utilised to determine dynamic vulnerabilities. For instance, conflict, pest outbreaks, or recent hydrometeorological events, may act to increase vulnerability to drought in the affected location.
5) In locations where vulnerability is elevated (Fig. 1: southwest square), the predefined forecast threshold (“danger level”) is relaxed in order to trigger for less severe droughts. This acknowledges that those with elevated vulnerabilities require support even if drought is only slight. In regions where dynamic vulnerability is lower (Fig. 1: northern squares), the predefined forecast threshold may be raised, to avoid the perception of false alarms if a less severedrought does not have significant impact on food security (the trigger threshold for northern squares is elevated to reflect reduced vulnerability). Balancing of lower thresholds for vulnerable regions against higher thresholds for less vulnerable regions reduces the need for ‘safety nets’, enabling more accurate anticipation of donor costs.
6) If risk exceeds the adjusted thresholds, early actions are triggered. Early actions may need to be adapted to account for multihazards.” ([Boult et al., 2022, p. 5](zotero://select/groups/4773535/items/B2AQSTYL)) ([pdf](zotero://open-pdf/groups/4773535/items/W9TFLH43?page=5&annotation=XV9GPPQF))


%----------------------------------------------------------------------------------------
%	SECTION 6 Crowdsensing, VGI, alternatively satellite image interpretation
%----------------------------------------------------------------------------------------


\section{Crowdsensing, Volunteersensing, VGI,  alternatively satellite image interpretation}
% alternatively -> on the ground/on site data via crowdsensing
% pros and cons of VGI
% Introduction of Crowdsourcing / sensing / Volunteersensing


%-----------------------------------
%	SUBSECTION 6.1
%-----------------------------------
% otherwise: 7.3 subsection water source mapping but I think it might be a good start here to introduce on site information
"In all, mapping presents many benefits, such as:

It makes easier to integrate data from different sources (surveys, censuses, satellites, etc.) and from different disciplines (social, economic, and environmental data). It also allows the switch to new units of analysis from, for example, administrative boundaries (e.g. state) to ecological boundaries (e.g. basin).
Maps are a powerful visual tool and are more easily understood by stakeholders, particularly in developing countries.
The spatial nature of water poverty, such as the distance to the nearest water source or the water supply infrastructure, can also be incorporated easily in a GIS database.
The allocation of resources can be improved, since geographic targeting is more efficient and cost-effective than to launch an equally expensive universal distribution programme.
Geo-referenced databases can be enriched by additional data as they become available; and new attributes, such as better details on water quality, can be incorporated into the data structure, ensuring that the relevance of the data is sustained over time.
Maps can be produced at a number of different resolutions depending on their purpose and the cost of data collection. A coarse resolution or a scale too small neglects the heterogeneity within each unit and provides insufficient detail for decision making, while a fine resolution or a scale too large increases the cost of compiling, managing, and analyzing the data."https://en.wikipedia.org/wiki/Water_point_mapping


\subsection{VGI}


%-----------------------------------
%	SUBSECTION 6.2
%-----------------------------------

\subsection{Crowdsensing / Citizen Science}


acceptance of the "crowd" is difficult to measure/identify/survey as no direct communication is possible. --> Good literature work e.g. “The basis of TAM consists of four main elements: the perceived usefulness of the application, the perceived ease of use when using the application, behavioural intention to use the application and the actual use.” ([Minkman, 2015, p. 12](zotero://select/groups/4773535/items/ZKLE6CPT)) ([pdf](zotero://open-pdf/groups/4773535/items/QMAPCSZG?page=12&annotation=IM5DGRJP))

“It was found that usefulness is the most important for behavioural intention.” ([Minkman, 2015, p. 12](zotero://select/groups/4773535/items/ZKLE6CPT)) ([pdf](zotero://open-pdf/groups/4773535/items/QMAPCSZG?page=12&annotation=75GFCEP5))

and how to best apply it -> relevant for the conceptionalization phase but generally well researched topic (in other, but possibly transferable contexts)


citizen science or rather participatory monitoring?

“How to decide whether citizen science is an appropriate means?” ([Minkman, 2015, p. 174](zotero://select/groups/4773535/items/ZKLE6CPT)) ([pdf](zotero://open-pdf/groups/4773535/items/QMAPCSZG?page=174&annotation=ZUAXHTI9))

“4 | IS REMOTE MONITORING A PATHWAY TO SUSTAINABILITY?” ([Thomson, 2021, p. 9](zotero://select/groups/4773535/items/UQLXVVYI)) ([pdf](zotero://open-pdf/groups/4773535/items/K9XBXPQD?page=9&annotation=833Q66UP))

rather than thinking water sources individually -> they can be thought of as a system when monitored more broadly

-> can change the way the water sector is funded: “With these data readily available, performance-related contracts that incentivize sustainable service delivery over short-term infrastructure investment can become the norm.” ([Thomson, 2021, p. 11](zotero://select/groups/4773535/items/UQLXVVYI)) ([pdf](zotero://open-pdf/groups/4773535/items/K9XBXPQD?page=11&annotation=JP9R4Y89))

“Our results indicate that using the phones to transmit more than just water quality data will likely improve the effectiveness and sustainability of this type of intervention.” ([Kumpel et al., 2015, p. 10846](zotero://select/groups/4773535/items/GPM4C7RJ)) ([pdf](zotero://open-pdf/groups/4773535/items/7VXVKEXK?page=1&annotation=4DJIADX2))

“USING MOBILE PHONES TO MONITOR AND MANAGE WATER SUPPLY QUALITY IN RURAL ENVIRONMNETS” ([Wilson-Jones and Rivett, 2012, p. 1](zotero://select/groups/4773535/items/ZTCP6ZDX)) ([pdf](zotero://open-pdf/groups/4773535/items/KQN2HBTG?page=1&annotation=LWS8S33I))

“MOBILE PHONE APPLICATIONS FOR WATER MANAGEMENT: CLASIFFICATION, OPPORTUNITIES AND CHALLENGES” ([Alfonso and Jonoski, 2012, p. 1](zotero://select/groups/4773535/items/4W4Q8E6B)) ([pdf](zotero://open-pdf/groups/4773535/items/PU944FGI?page=1&annotation=R8CDW3HI))

4: Start Network (2022): Integrating community voices in anticipatory action: a synthesis of complex qualitative data. Anticipatory-hub.org/news/integrating-community-voices-in-anticipatory-action-a-synthesis-of-complex-qualitative-data. [15.09.2022].

“Citizen science programmes are promising cost-efficient methods to monitor environmental resources, which make them especially suitable for low-income countries to overcome their sparse data resolution.” ([Weeser et al., 2018, p. 1598](zotero://select/groups/4773535/items/SFA2MLHC)) ([pdf](zotero://open-pdf/groups/4773535/items/GP79FHFC?page=9&annotation=4E9JCTQ5))
“Since today's citizen science studies are mostly located in high-income countries, we are enthusiastic to motivate the scientific community to conduct citizen science studies in low-income countries.” ([Weeser et al., 2018, p. 1598](zotero://select/groups/4773535/items/SFA2MLHC)) ([pdf](zotero://open-pdf/groups/4773535/items/GP79FHFC?page=9&annotation=TYD7Q2ZD))

“According to the Hyogo Framework for Action, increasing resilience to drought requires the development of a people-centered monitoring and early warning system, or in other words, a system capable of providing useful and understandable information to the community at risk.” ([Giordano et al., 2013, p. 523](zotero://select/groups/4773535/items/B7LM5ZR4)) ([pdf](zotero://open-pdf/groups/4773535/items/7I66DBIK?page=1&annotation=ELTIKEZV))

“Success factors for citizen science projects in water quality monitoring” ([San Llorente Capdevila et al., 2020, p. 1](zotero://select/groups/4773535/items/26SVUJIY)) ([pdf](zotero://open-pdf/groups/4773535/items/FHIUZZ6Y?page=1&annotation=3QTPUG53))

“After 1 y, we detect modest reductions in groundwater pumping and modest improvements in water quality and user satisfaction. Although replications are needed, the results imply that externally encouraged, community-based monitoring can improve the management of shared resources.” ([Bernedo Del Carpio et al., 2021, p. 1](zotero://select/groups/4773535/items/4ZDB6DYE)) ([pdf](zotero://open-pdf/groups/4773535/items/D7G4LR6W?page=1&annotation=HQ4A6SZV))

“Community-based monitoring and the science of water quality” ([Conrad, p. 217](zotero://select/groups/4773535/items/ZKTQ2GB5)) ([pdf](zotero://open-pdf/groups/4773535/items/3DN7838Q?page=1&annotation=G24Y695Z))

%-----------------------------------
%	SUBSECTION 6.3
%-----------------------------------

\subsection{Volunteersensing}
what about the volunteers? how many? where are they? How to they spread over the country?
“Meadow et al. (2013) recommended using trained agency staff to report drought status on a regular basis” ([Lackstrom et al., 2022, p. 27](zotero://select/groups/4773535/items/YI366LQY)) ([pdf](zotero://open-pdf/groups/4773535/items/3JTQ72UN?page=27\&annotation=AUZV7SZN))

provide hypothesis - 'integration' of local stakeholders/volunteers into drought and water source monitoring can help to get a better picture for early drought impact assessment and thus better and faster management and reactions
+ local people are engaged with the process and come into contact with 'scientific' knowledge/forecasts 
+ equal inclusion of local / indigenous knowledge and scientific forecasts can enhance the quality and make it more relevant on smaller scales (meso/mikro/local level)

knowledge co-production
\autocite{dasInteractiveInformationCrowdsourcing2016}

benefits of monitoring by external support:
“Therefore, the model of community management combined with external support has far-reaching benefits to rural water supply.” ([Huang et al., 2020, p. 144](zotero://select/groups/4773535/items/9CSBLJNJ)) ([pdf](zotero://open-pdf/groups/4773535/items/G5BEZQ7C?page=9&annotation=HU2CNZC2))

“For local communities, their needs of safe drinking water could be met and their abilities to manage and maintain water supply could be enhanced.” ([Huang et al., 2020, p. 144](zotero://select/groups/4773535/items/9CSBLJNJ)) ([pdf](zotero://open-pdf/groups/4773535/items/G5BEZQ7C?page=9&annotation=AL2NB5DK))

“For external experts, they may also benefit from community support to inform scientific processes, such as collecting data that spans across a large geographic region and having an enhanced understanding of community interests.” ([Huang et al., 2020, p. 144](zotero://select/groups/4773535/items/9CSBLJNJ)) ([pdf](zotero://open-pdf/groups/4773535/items/G5BEZQ7C?page=9&annotation=QPIKDKB9))

“Furthermore, this model could help increase scientific awareness among community members and engage the community with the environment.” ([Huang et al., 2020, p. 144](zotero://select/groups/4773535/items/9CSBLJNJ)) ([pdf](zotero://open-pdf/groups/4773535/items/G5BEZQ7C?page=9&annotation=FYU4BIKP))

“During the management of rural drinking water sources, a hybrid modality in which community management is the mainstay with supplement from external support from other organizations is highly recommended.” ([Huang et al., 2020, p. 147](zotero://select/groups/4773535/items/9CSBLJNJ)) ([pdf](zotero://open-pdf/groups/4773535/items/G5BEZQ7C?page=12&annotation=SQ6P8UBN))

“Community cultures, economies, and environments differ across countries and regions. These differences should be considered when designing hybrid management strategies, so that all actors can be appropriately enabled and the mechanism which is most effective for the given community can be identified.” ([Huang et al., 2020, p. 147](zotero://select/groups/4773535/items/9CSBLJNJ)) ([pdf](zotero://open-pdf/groups/4773535/items/G5BEZQ7C?page=12&annotation=WV5DXV5I))

“On this basis, it is essential to expand research area to study the various threats from climate variability to rural drinking water safety, and then to develop corresponding measures to address those threats to water security.” ([Huang et al., 2020, p. 147](zotero://select/groups/4773535/items/9CSBLJNJ)) ([pdf](zotero://open-pdf/groups/4773535/items/G5BEZQ7C?page=12&annotation=HCQHR7YR))

%----------------------------------------------------------------------------------------
%	SECTION 7 CBS & other tools + water related monitoring (excel)
%----------------------------------------------------------------------------------------


\section{CBS & other tools + water related monitoring (excel)}
% CBS
% water monitoring options -> can later be seen in table XY
% short introduction of other tools: Kobo, Ushahidi, NYSS, excel, etc. ..

what does 'monitoring' mean? How is it defined?

%-----------------------------------
%	SUBSECTION 7.1
%-----------------------------------
\subsection{CBS}
1: IFRC. (n.d.). Community Engagement and Accountability. https://www.ifrc.org/community-engagement-and-accountability. [15.09.2022].
2: SRCS. (2021). Measles outbreak detected by Somaliland SRCS Volunteers in Todgheer Region. https://drive.google.com/file/d/1O9PMPKKL312o1zbXELgB7FuMdokzWpic/view. [15.09.2022].
3: SRCS (2022): Feasibility Study on Potential Use of Forecast-based Financing (FbF) for SRCS Final Report. Nottawasage Institute.
4: Start Network (2022): Integrating community voices in anticipatory action: a synthesis of complex qualitative data. Anticipatory-hub.org/news/integrating-community-voices-in-anticipatory-action-a-synthesis-of-complex-qualitative-data. [15.09.2022].

%-----------------------------------
%	SUBSECTION 7.2
%-----------------------------------

\subsection{other tools}
Ushahidi
CBS/NYSS
DIPAS \& SketchMapTool (as representatives for strategic data/long-term management)

excel sheet table

“2. Key concepts of participatory early warning and monitoring systems (pEWMS)” ([“Participatory early warning and monitoring systems_ A Nordic framework for web-based flood risk management | Elsevier Enhanced Reader”, p. 1296](zotero://select/groups/4773535/items/JYH8N2BV)) ([pdf](zotero://open-pdf/groups/4773535/items/JITPV84L?page=2&annotation=B2ELHBPJ))

“Adding “bottom-up” approaches [36] to classical EWMS allows stakeholders with access to local knowledge of environments and local networks to play a stronger role in decisionmaking and risk management” ([“Participatory early warning and monitoring systems_ A Nordic framework for web-based flood risk management | Elsevier Enhanced Reader”, p. 1296](zotero://select/groups/4773535/items/JYH8N2BV)) ([pdf](zotero://open-pdf/groups/4773535/items/JITPV84L?page=2&annotation=AXZXAK9D))

%-----------------------------------
%	SUBSECTION 7.3
%-----------------------------------
\subsection{water source mapping}
"In all, mapping presents many benefits, such as:

It makes easier to integrate data from different sources (surveys, censuses, satellites, etc.) and from different disciplines (social, economic, and environmental data). It also allows the switch to new units of analysis from, for example, administrative boundaries (e.g. state) to ecological boundaries (e.g. basin).
Maps are a powerful visual tool and are more easily understood by stakeholders, particularly in developing countries.
The spatial nature of water poverty, such as the distance to the nearest water source or the water supply infrastructure, can also be incorporated easily in a GIS database.
The allocation of resources can be improved, since geographic targeting is more efficient and cost-effective than to launch an equally expensive universal distribution programme.
Geo-referenced databases can be enriched by additional data as they become available; and new attributes, such as better details on water quality, can be incorporated into the data structure, ensuring that the relevance of the data is sustained over time.
Maps can be produced at a number of different resolutions depending on their purpose and the cost of data collection. A coarse resolution or a scale too small neglects the heterogeneity within each unit and provides insufficient detail for decision making, while a fine resolution or a scale too large increases the cost of compiling, managing, and analyzing the data."https://en.wikipedia.org/wiki/Water_point_mapping

%----------------------------------------------------------------------------------------
%	SECTION 8 Case Study Area (+ application of the rest)
%----------------------------------------------------------------------------------------


\section{Case Study Area (+ application of the rest)}
% Somaliland geography
% hazards and risks: drought situation (past, present, future)
% Somaliland social and economic circumstances (past, present, future)
% Stakeholder
% resource restrictions


(establish your research territory: general information about the importance, background details to understand studies context)

Overview Confluence Background Overview:
https://heigit.atlassian.net/wiki/spaces/FIS/pages/1704186/Background+Information

https://drive.google.com/file/d/1McCpUWQPNlO-nDF0RjDW7nmrdM75ZJoP/view
%-----------------------------------
%	SUBSECTION 8.1
%-----------------------------------
\subsection{Geography}


%-----------------------------------
%	SUBSECTION 8.2
%-----------------------------------

\subsection{Hazards and risks}

“The number of affected people will be 1,200,420 persons across all the six main regions in Somaliland. The top priority needs of the people affected to date are mainly water (70{\%), Food (21\%) and Health (9\%).” ([National Drought Committee, 2022, p. 3](zotero://select/groups/4773535/items/7XJRE6LM)) ([pdf](zotero://open-pdf/groups/4773535/items/2F59E4UZ?page=3&annotation=8JZVBSM6))

%% Risk analysis

"Prioritised hazard and its historical impact.
Drought is currently the most relevant hazard in Somaliland . Droughts have occurred frequently for the last decades and presented varied impacts (The Somaliland drought rapid assessment report commissioned by the National Drought Committee in January 2022).  The occurrence of droughts has increased over the last years from one in five years to one drought every two years. The majority of the population depends on (agro) pastoral agriculture. Since this type of livelihood depends fundamentally on the natural resources of pasture and water, the local population is highly vulnerable to droughts. 
According to the International Database for Disasters ( EM DAT) recent droughts in Somaliland occurred in 2004, 2010 to 2011, 2014 to 2017, 2019, and 2021 to 2023. The drought incidents have mainly caused food shortages due to below-average crop production, livestock loss and water shortages. The impacts are also in line with the historical drought impacts of the communities in the Eastern regions of Somaliland, i.e. Sool, Sanaag and Togdheer where communities were consulted in focus group discussions as part of the ongoing FbF project.  Since 2018 there has been a drought in 2019 and the current drought (2020-2023). This is also in line with the historical Combined Drought Index (CDI) data (FAO SWALIM). The CDI is a proven and reliable indicator of drought since it incorporates precipitation, soil moisture and vegetation health. While the country has experienced periodic severe drought conditions over the CDI assessment period from 2002 to 2022, the eastern regions have experienced the most extreme drought conditions, especially in the last two years. Droughts in Somaliland lead to a variety of negative impacts. Chief among them is increased food insecurity. EM DAT  lists food shortage and famine as the principal impact of all droughts since 1987. Furthermore, based on the Food Insecurity Integrated Phase Classification (IPC) data, every drought is associated with substantial spikes in food-insecure populations. For example, the current drought led up to December 2022 to 5.6 million people in or above IPC 3 (crisis). Since 2009, Somaliland has experienced a IPC average between class 2 and 3 (between stressed and crisis), whereby the very western regions Awdal and the eastern regions (Sanaag, Sool and Togdheer) have been most affected reaching IPC classification 3 or worse in almost a quarter of assessment periods (every 2 months). 
The exact processes which lead to food insecurity due to drought are diverse. In the case of pastoral communities, failed rainy seasons result in livestock and herders having to track longer distances to water sources. These circumstances and lower forage and water availability lead to poor livestock health, increasing livestock deaths and lower birth rates, which can amount to the loss of whole herds. Milk production decreases drastically, which is the reason for the lower diet variability of herders. Furthermore, herders have to sell their animals for lower prices due to their lower body conditions and increased supply. This was the case during the 2017 drought. In some cases, whole herds were lost. The price of milk products increased significantly. At the peak of the drought in 2017, 388.000 children suffered from malnutrition and 895.000 people were internally displaced. In total 6.2 million people have been in need of humanitarian assistance.
Agropastoralist communities are affected by droughts in the same way as pastoralist communities. In addition, their livelihood suffers from increased crop failure due to poor soil moisture. This, again, causes agropastoralists to lose their income. Additionally, the local staple food prices rise, putting pressure on people who depend on maize or sorghum as the primary carbohydrate source. 

Similar to higher food prices, drought causes higher water prices as well. Besides fewer available resources to buy quantitative and qualitative adequate food, high water prices lead to poor hygiene practices. In 2017 water shortages due to two consecutive failed rainy seasons led to rising water prices. In the Eastern regions, the cost of 200 litres of water jumped from 2 USD to 7 USD. Similar water shortages during the 2016 drought caused a sharp increase in watery diarrhoea cases to 13.653 cases with a fatality rate of 3.6 %. 
Further consequences of droughts of the last 12 years in Somaliland can be seen in the table below."
https://docs.google.com/document/d/1xUEXm8RxVHTO468KqXSAoBX-cpkPwiff/edit

%% + table of the last mayor droughts

%-----------------------------------
%	SUBSECTION 8.3
%-----------------------------------

\subsection{Political, social and economic circumstances}
region is vulnerable to droughts due to its predominant reliance on pastoralism as its source of livelihood and economy. Livestock remains an essential source of employment, hard currency, export earnings and government revenue.
in case of failure of rainy season, people migrate from their locations, including pastoralists with their livestock, migrate to areas where better water and pasture are still available during the dry season
seasonal weather patterns in the region and the rest of the regions are characterized by four main seasons - two rainy seasons and two dry seasons. The main rains fall during the Gu-season (AprilJune), with lighter and more sporadic rains falling during the Deyr season (October December). Two dry seasons are characterized by one shorter, cooler season, known as the Hagaa (22nd July to 23rd September), and a long, hot, dry season, known as the Jilaal (January to 23rd March), which is the harshest season of the year


Point water sources are the primary source of water
Traditionally, natural disasters like droughts do not prevail uniformly across the entire area of Somalia but only in some hotspots.
significant increase in GBV and child abuse/abandonment during crises such as drought, flooding, conflict, etc. early action includes deploying extra social workers, training community champions, etc. UNFPA and UNICEF are currently undertaking a national assessment of these issues, and the report will likely be a useful source for identifying approaches for undertaking early action as part of the drought FbF
Hydrogeological survey and assessment of selected areas in Somaliland and Puntland (2012)
“Hydrogeological Survey and Assessment of Selected Areas in Somaliland and Puntland” ([pdf](zotero://open-pdf/groups/4773535/items/KU98BB4Z?page=1&annotation=JA8SYGKT))



%-----------------------------------
%	SUBSECTION 8.4
%-----------------------------------

\subsection{Stakeholder}

“Stakeholders interviewed for this project reinforced this and emphasized the importance of collaboration and cohesion between different methods and approaches, given the multi-faceted nature of the hazard.” ([RCRC, 2020, p. 23](zotero://select/groups/4773535/items/UESIQTRJ)) ([pdf](zotero://open-pdf/groups/4773535/items/P5JPVZ97?page=23&annotation=EQV29795))

https://heigit.atlassian.net/wiki/spaces/FIS/pages/1704213/Stakeholders+Systems+Networks

% also in subsection 7.2
“Adding “bottom-up” approaches [36] to classical EWMS allows stakeholders with access to local knowledge of environments and local networks to play a stronger role in decisionmaking and risk management” ([“Participatory early warning and monitoring systems_ A Nordic framework for web-based flood risk management | Elsevier Enhanced Reader”, p. 1296](zotero://select/groups/4773535/items/JYH8N2BV)) ([pdf](zotero://open-pdf/groups/4773535/items/JITPV84L?page=2&annotation=AXZXAK9D))

“With the increased pressure on water resources and the challenges faced with the implementation of the existing regulatory framework, a growing lack of mutual trust between water stakeholders has been observed in recent years [38].” ([“Participatory early warning and monitoring systems_ A Nordic framework for web-based flood risk management | Elsevier Enhanced Reader”, p. 1296](zotero://select/groups/4773535/items/JYH8N2BV)) ([pdf](zotero://open-pdf/groups/4773535/items/JITPV84L?page=2&annotation=HNTIH8JB))
%-----------------------------------
%	SUBSECTION 8.5
%-----------------------------------

\subsection{Water sources}
Berkads:
“Changing Pastoralism in the Ethiopian Somali National Regional State (Region 5)” ([“Changing Pastoralism in Region 5”, p. 1](zotero://select/groups/4773535/items/FXJGUTLD)) ([pdf](zotero://open-pdf/groups/4773535/items/BIAA5M57?page=1&annotation=5F9EZJYZ))

"Although birkeds cannot be considered permanent water points in the sense of permanent wells which do not rely on harvesting rainwater, clusters of birkeds represent dry season water points that they provide water throughout the dry season in most years. Today, then, distribution of water points is vastly different from a few decades ago. Map 5 shows the water points that exist today in the five districts under study. The map attempts to show the wells and boreholes as well as the main clusters of birkeds. The latter are difficult to map as there is no existing record of all locations. The map is based on sketch maps drawn by communities during the fieldwork. It is thus not meant to be accurate but to give an indication of the nature of change. It should be noted that these water points shown on the map are also the site of permanent settlements, as the tendency has been for settlements to grow up at the site of new water points."

“IMPROVED BERKAD DESIGNS BY MERCY CORPS - SOMALIA” ([pdf](zotero://open-pdf/groups/4773535/items/F3CWEKHP?page=1&annotation=KH5I94RJ))

image of a Berkad? that is a water source type typical for Somalia

"Criteria used included among others coverage, need for rehabilitation, seasonality of services, quality of water delivered, and poor management." https://en.wikipedia.org/wiki/Water_point_mapping
https://onlinelibrary.wiley.com/doi/abs/10.1111/j.1477-8947.2010.01296.x?casa_token=TPvw51virRQAAAAA:9K-6fcNcYFw9-Mny-EOIMdS6OmmSUkajSfo9qggsMNnAirGYSslUHckFyqWNP68XarnXYEgIh9eL0uGltw

"Water source:
o Out of 77 communities assessed, 49 have berkets, 26 have boreholes, 21 have shallow
wells, 28 communities reported that they receive water trucking.
o Berkads: 76\% of communities having berkets reported that all berkets are reported to be
depleted, while 22\% reported they were less half than full.
o Water trucking: Out of the 28 communities that receive water trucking, 43\% receive water
on a daily basis. 19 out of 28 communities (68\%) receiving water trucking responded that
they receive water from private suppliers. No communities, covered by this assessment,
mentioned water trucking from humanitarian partners.
o Reduction of water consumption: 54 out of 77 assessed communities (69\%) responded
that the majority of community members reduced water consumption in the past 4 weeks,
including all assessed communities in Sool." https://drive.google.com/file/d/1KWUZW0jEMV1Ijc4zeET_Yes93VJ_3qn2/view
%----------------------------------------------------------------------------------------
%	SECTION 9 Conclusion literature
%----------------------------------------------------------------------------------------


\section{Conclusion literature}
In the conclusion, you should summarize the key findings you have taken from the literature and emphasize their significance.

%-----------------------------------
%	SUBSECTION 9.1
%-----------------------------------
\subsection{key findings}


%-----------------------------------
%	SUBSECTION 9.2
%-----------------------------------
\subsection{Limitations}


%-----------------------------------
%	SUBSECTION 9.3
%-----------------------------------

\subsection{Key assumptions}



%----------------------------------------------------------------------------------------
%----------------------------------------------------------------------------------------
%----------------------------------------------------------------------------------------


\section{Notes}
% Notes: 
% the literature review is focused on providing background information and enabling historical interpretation of the subject of analysis in relation to the research problem the case is intended to address. This includes synthesizing studies that help to:

% "Place relevant works in the context of their contribution to understanding the case study being investigated. This would include summarizing studies that have used a similar subject of analysis to investigate the research problem. If there is literature using the same or a very similar case to study, you need to explain why duplicating past research is important [e.g., conditions have changed; prior studies were conducted long ago, etc.].
% Describe the relationship each work has to the others under consideration that informs the reader why this case is applicable. Your literature review should include a description of any works that support using the case to study the research problem and the underlying research questions.
% Identify new ways to interpret prior research using the case study. If applicable, review any research that has examined the research problem using a different research design. Explain how your case study design may reveal new knowledge or a new perspective or that can redirect research in an important new direction.
% Resolve conflicts amongst seemingly contradictory previous studies. This refers to synthesizing any literature that points to unresolved issues of concern about the research problem and describing how the subject of analysis that forms the case study can help resolve these existing contradictions.
% Point the way in fulfilling a need for additional research. Your review should examine any literature that lays a foundation for understanding why your case study design and the subject of analysis around which you have designed your study may reveal a new way of approaching the research problem or offer a perspective that points to the need for additional research.
% Expose any gaps that exist in the literature that the case study could help to fill. Summarize any literature that not only shows how your subject of analysis contributes to understanding the research problem, but how your case contributes to a new way of understanding the problem that prior research has failed to do.
% Locate your own research within the context of existing literature [very important!]. Collectively, your literature review should always place your case study within the larger domain of prior research about the problem. The overarching purpose of reviewing pertinent literature in a case study paper is to demonstrate that you have thoroughly identified and synthesized prior studies in the context of explaining the relevance of the case in addressing the research problem."https://libguides.pointloma.edu/c.php?g=944338&p=6806958
