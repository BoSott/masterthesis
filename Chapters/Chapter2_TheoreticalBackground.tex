% Chapter Template

\chapter{Theoretical Background} % Main chapter title

\label{Chapter2} % \ref{Chapter2}
WATCH  OUT THAT THIS STAYS THE THEORETICAL BACKGROUND AND IS NOT !!! A DISCUSSION!!!!
%----------------------------------------------------------------------------------------
%	SECTION 1
%----------------------------------------------------------------------------------------

\section{Introduction}
% The introduction should clearly establish the focus and purpose of the literature review.
% highlight research gap + emphasize the timeliness
% Locate your own research within the context of existing literature [very important!]. 
% reiterate central problem
% focus and purpose of literature review
% brief summery scholarly context
% highlight research gap
% emphasize timeliness


\subsection{reasoning}:
https://www.unwater.org/our-work/integrated-monitoring-initiative-sdg-6

"Stronger accountability: Data can communicate that work is being done and progress is happening. Data can enable greater transparency, which reduces inefficiency and corruption.
Attracting commitment and investments: Data can quantify problems and make it easier to communicate needs for political commitment and public and private investments.
Evidence-based decision-making: Data can inform policy- and decision-makers of where to focus efforts and which solutions are most effective, to ensure the greatest possible gains with existing resources.
Leaving no one behind: Disaggregated data can help identify specific groups or areas with unmet needs and higher levels of risk, to which interventions can be targeted."
https://www.unwater.org/our-work/integrated-monitoring-initiative-sdg-6/background

“Experts in trigger methodology have indicated a more appropriate strategy may be to build on tools that currently exist at the government level such as national drought monitoring systems. As such, the ideal is an iterative process with the ground level along with a technology push that creates new ways to analyse drought and drought risk.” ([RCRC, 2020, p. 28](zotero://select/groups/4773535/items/UESIQTRJ)) ([pdf](zotero://open-pdf/groups/4773535/items/P5JPVZ97?page=28&annotation=977VS8FC))

% DRM Strategic Plan General and Specific Objectives
“2.1.1 Specific Objective 1 Vulnerable communities’ resilience at SRCS target areas strengthened through anticipatory actions, response, recovery, and disaster risk reduction, and they can anticipate and effectively respond to and recover from disasters and crisis by 2026.” ([“SRCS DRM Strategic Plan_final 9thNovember 2021-converted.pdf”, p. 15](zotero://select/groups/4773535/items/LFCBRZLD)) ([pdf](zotero://open-pdf/groups/4773535/items/6IL6K72G?page=15&annotation=ZFICKZRA))

% --> even the RCRC is still looking for good triggers -> maybe water levels are a good way -> reasoning for this study (see background identical text)

“Countries in which less than 50\% of the population uses improved drinking water sources are all located in sub-Saharan Africa and Oceania 91-100\% 76-90\% 50-75\% <50\% insufficient data or not applicable Proportion of the population using improved drinking water sources in 2015 ■ 91–100\% ■ 76–90\% ■ 50–75\% ■ <50\% ■ INSUFFICIENT DATA OR NOT APPLICABLE” ([World Health Organization, 2016, p. 15](zotero://select/groups/4773535/items/KVAKZ9ZT)) ([pdf](zotero://open-pdf/groups/4773535/items/4STYK52H?page=14\&annotation=FBURDS4T))

“The methods by which the Joint Monitoring Programme (JMP) of WHO and UNICEF” ([Bartram et al., 2014, p. 8137](zotero://select/groups/4773535/items/6AWUJTW5)) ([pdf](zotero://open-pdf/groups/4773535/items/BFNSQGWS?page=1&annotation=UL4Q2I4V))
“substantial limitations: current methods do not address water quality, equity of access, or extra-household services.” ([Bartram et al., 2014, p. 8137](zotero://select/groups/4773535/items/6AWUJTW5)) ([pdf](zotero://open-pdf/groups/4773535/items/BFNSQGWS?page=1&annotation=TIPCEXGG))

request of the SRCS -> practically wanted
understanding the full scope and knowing which water sources are at what level and quality can help with management decisions and trigger certain events very locally
current challenges (?) what do I want to address? (number 1,2,3)
outline of the thesis/project

overcome limitations / incorporating recommendations: e.g. increased support/engagement of poeple who actually use the reports (e.g. SRCS officials) 
fill geographic and information gap ->

its about drought forecasting and early trigger but at the same time highly local and practical information where and which water sources are good and functioning and which are not. -> highly practical information. Some data exist but (mostly) outdated.
about getting local knowledge from SRCS Volunteers and their community as well as returning information about the bigger picture
in order to enhance the quality of data for managing severe droughts in Somaliland. (one short paragraph -> motivation)
provide number of weather stations in the area

--> problem: conclusion of existing sources, tools and forecasts - only macro/international level? or are there meso/micro forecasts available?
better understanding the forecasting and its implications on the ground are crucial. --> local information. tons of Volunteers but even more water sources. Continue with Crowdsensing? Implications?

--> highlight the pro of this work
"The highly localized information provided by observers can fill drought monitoring gaps by ground-truthing quantitative indicators and offering information in places where other monitoring tools may not exist. Overall, the research team found that strategic investments in time and funding can help fill in geographic and temporal gaps in drought monitoring information through volunteer observations."
https://www.drought.gov/news/research-confirms-role-citizen-science-contributions-drought-detection-and-monitoring and https://doi.org/10.1175/BAMS-D-21-0157.1

“Citizen science programmes are promising cost-efficient methods to monitor environmental resources, which make them especially suitable for low-income countries to overcome their sparse data resolution.” ([Weeser et al., 2018, p. 1598](zotero://select/groups/4773535/items/SFA2MLHC)) ([pdf](zotero://open-pdf/groups/4773535/items/GP79FHFC?page=9&annotation=4E9JCTQ5))
“Since today's citizen science studies are mostly located in high-income countries, we are enthusiastic to motivate the scientific community to conduct citizen science studies in low-income countries.” ([Weeser et al., 2018, p. 1598](zotero://select/groups/4773535/items/SFA2MLHC)) ([pdf](zotero://open-pdf/groups/4773535/items/GP79FHFC?page=9&annotation=TYD7Q2ZD))

% also in case study
“The number of affected people will be 1,200,420 persons across all the six main regions in Somaliland. The top priority needs of the people affected to date are mainly water (70{\%), Food (21\%) and Health (9\%).” ([National Drought Committee, 2022, p. 3](zotero://select/groups/4773535/items/7XJRE6LM)) ([pdf](zotero://open-pdf/groups/4773535/items/2F59E4UZ?page=3&annotation=8JZVBSM6))

%-----------------------------------
%	SUBSECTION 1.1
%-----------------------------------
\subsection{relevance}


\subsection{structure of this overall chapter}
relevance
Somalia is komplett am krepieren because of a multi-year long drought - ...\% of damage/conflicts etc. is based on droughts. severe shit! --> first case study introduction (geographically, socially, etc.) 
scope: of the EAP/FbF project

“1.2 DROUGHT IN SOMALIA” ([“Economics of Resilience to Drought: Somalia Analysis”, p. 9](zotero://select/groups/4773535/items/LGRWAU43)) ([pdf](zotero://open-pdf/groups/4773535/items/MBXSCVWR?page=9&annotation=K6W9FFM7))

possibly talk about water security here for the larger context

%-----------------------------------
%	SUBSECTION 1.2
%-----------------------------------

\section{key terms and concepts} % most likely skipped
In this section relevant key terms and concepts are shortly explained or summarized if they are dealt with in detail later on.


water access



risk (hazard, exposure, vulnerability)
special report 2021 p.27 + p.58 1.4 Drought risk assessment

impact

risk
https://www.preventionweb.net/understanding-disaster-risk/component-risk/disaster-risk
where vulnerability = “Risk = ƒ (Hazard, Exposure, Vulnerability), where Vulnerability = ƒ (Susceptibility to impacts, Lack of coping capacity, Lack of adaptive capacity).” ([“Special report on drought 2021”, 2021, p. 27](zotero://select/groups/4773535/items/RAAM9PVS)) ([pdf](zotero://open-pdf/groups/4773535/items/7AK5QVBL?page=29&annotation=VQABMLCT))


drought:



drought frequency


drought severity
Drought severity and drought magnitude are two different aspects of drought that are often used to describe the severity and extent of drought conditions.

Drought severity refers to the intensity of a drought and is typically characterized by the degree to which it affects various aspects of the natural and social systems, such as precipitation, temperature, soil moisture, streamflow, water supply, agriculture, and socio-economic activities. Severity can be described using qualitative or quantitative measures, such as the Palmer Drought Severity Index (PDSI), the Standardized Precipitation Index (SPI), or the Keetch-Byram Drought Index (KBDI).

On the other hand, drought magnitude refers to the overall size or extent of a drought event, typically measured in terms of its spatial and temporal coverage. Magnitude can be characterized by the duration, frequency, and spatial extent of the drought, and can be assessed using various indices, such as the Percentage of Normal (PN) or the Area-Weighted Precipitation Anomaly (AWPA).

In summary, drought severity focuses on the intensity and impacts of a drought event, while drought magnitude describes the overall size and extent of the drought event. Both severity and magnitude are important for understanding and assessing the severity of drought conditions and their impacts on various natural and social systems.

https://www.earthdata.nasa.gov/topics/atmosphere/weather-events/droughts/drought-severity#:~:text=Drought%20severity%20is%20a%20relative,drought%20is%20considered%20to%20be.

though, duration plays a role in severity as well


drought intensity


impact

“Drought impact Impact of drought on socio-economic systems (sometimes called “socio-economic drought”) and/or ecosystems Aridity Long-term dryness as feature of climate, with long-term average precipitation being much lower than potential evaporation Water scarcity Long-term imbalance between water demand and water supply, caused by high average demand, low average water availability and/or problems with water supply Water shortage/stress Acute lack of water for (social, economic, or environmental) needs, caused by lower water supply than demand Overexploitation Long-term overuse of water resources resulting in a gradual decrease in water availability (for overexploitation of groundwater, the term “depletion” is often used)” ([Van Loon et al., 2016, p. 3637](zotero://select/groups/4773535/items/T6SI6L68)) ([pdf](zotero://open-pdf/groups/4773535/items/7GTQ4QHT?page=7&annotation=SNJLBYFL))




% ChatGPT start
"what is the difference between drought severity and drought intensity?
Drought severity and drought intensity are two different but related concepts used to describe the magnitude and impact of a drought event.

Drought severity refers to the extent of the impact of a drought on various sectors such as agriculture, water resources, and socioeconomic activities. It is usually assessed by looking at the degree of damage or loss caused by the drought. The severity of a drought is typically measured using indicators such as crop yield reduction, water supply depletion, and economic losses.

Drought intensity, on the other hand, refers to the duration and magnitude of the deficit in precipitation that causes the drought. It is usually assessed by looking at the degree of deviation from normal precipitation patterns and how long it has persisted. The intensity of a drought is typically measured using indices such as the Standardized Precipitation Index (SPI) or the Palmer Drought Severity Index (PDSI).

To illustrate the difference between drought severity and drought intensity, consider a scenario where two regions experience the same precipitation deficit over a period of three months. In the first region, the drought leads to a 50\% reduction in crop yields and widespread water shortages, while in the second region, the drought has little impact on agriculture and water resources. In this case, the severity of the drought is higher in the first region, despite the same intensity of precipitation deficit in both regions.

In summary, drought severity and intensity are both important measures of drought, but they represent different aspects of the phenomenon. Severity relates to the impact of a drought on different sectors, while intensity relates to the magnitude and duration of the precipitation deficit causing the drought \autocite{rcrcFORECASTBASEDFINANCINGEARLY2020,vereintenationenSpecialReportDrought2021}.




Bosse S
how does drought severity, drought impact and risk relate to each other?

Drought severity, drought impact, and drought risk are interrelated concepts used to assess and understand the effects of drought on various sectors. Here's how they relate to each other:

Drought severity refers to the degree of impact of a drought event on various sectors such as agriculture, water resources, and socioeconomic activities. It is usually assessed by looking at the degree of damage or loss caused by the drought. Severity can be measured using indicators such as crop yield reduction, water supply depletion, and economic losses.

Drought impact refers to the direct and indirect effects of drought on different sectors of society. It can include effects on human health, wildlife, ecosystems, infrastructure, and the economy. Drought impact can be assessed using various indicators and methods, including surveys, monitoring, and modeling.

Drought risk, on the other hand, is the likelihood of a drought event occurring in a given area and the potential consequences of that event. It takes into account the severity and impact of past droughts, as well as the vulnerability of the affected region to drought. Drought risk can be assessed using various indicators such as climate patterns, soil moisture levels, and water demand.

In summary, drought severity and impact are measures of the magnitude and effects of a drought event, while drought risk is a measure of the likelihood and potential consequences of future droughts. Understanding the relationship between these concepts is important for developing effective drought management strategies, including early warning systems, drought preparedness plans, and mitigation measures. By assessing the severity, impact, and risk of droughts, policymakers, and stakeholders can make informed decisions to reduce the negative effects of drought on society and the environment.




Bosse S
difference between drought severity and drought impact
Drought severity and drought impact are two related but distinct concepts used to measure the magnitude and effects of droughts.

Drought severity refers to the degree of intensity of a drought event. It is typically assessed by looking at the extent of the precipitation deficit, the duration of the drought, and the level of dryness in the affected area. Severity can be measured using various indices such as the Standardized Precipitation Index (SPI) or the Palmer Drought Severity Index (PDSI). Drought severity is a measure of the physical characteristics of a drought event and is not directly related to the effects of the drought on various sectors.

Drought impact, on the other hand, refers to the direct and indirect effects of a drought on different sectors such as agriculture, water resources, and socioeconomic activities. It is usually assessed by looking at the degree of damage or loss caused by the drought, such as crop failure, reduced water availability, and economic losses. Drought impact is a measure of the consequences of the drought on various sectors and is not directly related to the physical characteristics of the drought event.

To illustrate the difference between drought severity and impact, consider a scenario where two regions experience the same severity of drought event, with similar precipitation deficits and dryness levels. In the first region, the drought leads to a 50\% reduction in crop yields, while in the second region, the drought has little impact on agriculture. In this case, the severity of the drought is the same in both regions, but the impact is higher in the first region due to the greater damage to crops.

In summary, drought severity and impact are both important measures of drought, but they represent different aspects of the phenomenon. Severity relates to the physical characteristics of the drought event, while impact relates to the consequences of the drought on various sectors. Understanding the difference between these concepts is crucial for developing effective drought management strategies and mitigation measures. (ChatGPT)


%----------------------------------------------------------------------------------------
%	SECTION 2 Water security, drought & water scarcity/quality/access
%----------------------------------------------------------------------------------------


\section{some other title Drought \& Water security} % irgendwas in Richtung grundlegende Konzepte oder so.. was weiß ich ^^

Water security is a theoretical construct that has emerged in the 21st century to frame the overall water objectives and goals to guide local to global water management and policy development \autocite{sadoffWaterSecurity2020a}. It "links together the web of food, energy, climate, economic growth, and human security challenges that the world economy faces over the next two decades" \autocite[5]{wefBubbleCloseBursting2009}. In more detail, it is about "the availability of an acceptable quantity and quality of water for health, livelihoods, ecosystems and production, coupled with an acceptable level of water-related risks to people, environments and economies."\autocite{greySinkSwimWater2007}.
Water security integrates therefore economic, social and environmental dimensions into an interconnected and complex system of human and natural relations by addressing risks of too much, too little or poor quality water \autocite{vanbeekWaterSecurityPutting2014, mishraWaterSecurityChanging2021}. Due to the focus of this work, emphasis is placed on factors that decrease water security due to too little water availability. Besides other factors, natural disasters such as droughts, and water scarcity are the main drivers for insufficient quantities of water \autocite{caretta2022water}. Water quality and access are briefly addressed in addition to provide a more comprehensive  understanding of water security for the following chapters.

% TODO: maybe switch the order here.. first water scarcity, then drought. -> would make more sense and it's only a small paragraph --> but takes some effort in the second paragraph of water scarcity.. maybe later on. Not that important

% nice source for water security: “A Framework for Water Security Data Gathering Strategies” ([Butte et al., 2022, p. 1](zotero://select/groups/4773535/items/QB97YZ2M)) ([pdf](zotero://open-pdf/groups/4773535/items/Q936I2JN?page=1&annotation=5G7DUDDP))

%-----------------------------------
%	SUBSECTION 2.1
%-----------------------------------

\subsection{About Drought}\label{subsec:about_drought}

\todo{TODO: change NDMC sources to knew ones}

Drought as highly complex and severe climate-related multi-hazard has far reaching, cascading and interconnected consequences affecting natural ecosystems, societies and economies \autocite{vereintenationenSpecialReportDrought2021}. Historically, droughts are a recurring feature that can occur in all climates. They can geographically extend over small areas to entire sub-continents and are slow onset events that can persist for a few weeks to several years. These high spatial and temporal variabilities make drought not only challenging to define but due to its slow onset, droughts are often only recognized when they are well advanced \autocite{idmpDroughtWaterScarcity2022,vereintenationenSpecialReportDrought2021}. While some drought conditions over large areas can be associated to some low-frequency changes in atmospheric conditions such as the El Niño, accurate cause identification can be rather challenging on smaller scales and requires many different parameters \autocite{botaiAnalysisDroughtProgression2019, vereintenationenSpecialReportDrought2021}.

% possibly include after the second sentence also adjust the first one or two sentences of the paragraph below
% in the sense of.. As can be seen in fígure xy, drought is a complex conglomeration of interrelated causes and effects,which makes definition of droughts a fairly mulit-layered undertaking.
% noice figurrrrreee
%“Figure 1.6. Schematic representation of potential interconnections among different sectors affected by droughts” ([“Special report on drought 2021”, 2021, p. 47](zotero://select/groups/4773535/items/RAAM9PVS)) ([pdf](zotero://open-pdf/groups/4773535/items/7AK5QVBL?page=49&annotation=TNA9J8ZS))


%In order to approach this complexity, drought is most often defined from four different perspectives, focussing on different manifestations and stages. These definitions are outlined in the coming sub-chapter \ref*{subsec:drought_definitions}, followed by a section addressing the necessary indicators  currently employed in practice for these definitions.
%Generally, droughts are commonly characterized by deviations or the complete failure of climate and weather systems that drive the hydrological cycle compared to normal conditions\autocite{botaiAnalysisDroughtProgression2019,idmpDroughtWaterScarcity2022,vanloonDroughtHumanmodifiedWorld2016,vereintenationenSpecialReportDrought2021}. A more in depth definition can be found in the sub-chapter \ref*{subsec:drought_definitions}.

%“Many studies characterize drought via three main aspects: (i) intensity, (ii) duration and (iii) spatial coverage (He et al.,2018;Adede et al.,2019;Dai,2011).” ([Balti et al., 2020, p. 3](zotero://select/groups/4773535/items/V9YH9QFQ)) ([pdf](zotero://open-pdf/groups/4773535/items/JC4XTVWE?page=3&annotation=WPTV8SEV))
%instead of the first sentence: This complex concept of drought commonly encompassing .... dimensions of interrelated causes ... 

The concept of drought commonly encompasses multiple temporal, spatial and thematic dimensions. This complex conglomeration of interrelated causes and effects makes definition of \textit{droughts} a fairly multi-layered undertaking \autocite{balintMonitoringDroughtCombined2013}. Several well-known definitions (in this category) are for example from the \autocite{theamericanheritagedictionaryoftheenglishlanguageDrought2022} defining drought as "a long period of abnormally low rainfall, especially one that adversely affects growing or living conditions". \autocite[2]{palmerMeteorologicalDrought1965} defines drought as "a prolonged and abnormal moisture deficiency." or \autocite{vanloonDroughtHumanmodifiedWorld2016} defines droughts simply as "an exceptional lack of water compared to normal conditions". Other drought definitions emphasize its natural and/or human origin, its special characteristics, impact and temporal duration or even understand "drought as a system of causality where the link between causes and effects is random in nature {balintMonitoringDroughtCombined2013, baltiReviewDroughtMonitoring2020, idmpDroughtWaterScarcity2022,loonDroughtAnthropocene2016, wangPropagationDroughtMeteorological2016, wilhiteUnderstandingDroughtPhenomenon1985}. Already in the 1980s, \autocite{wilhiteUnderstandingDroughtPhenomenon1985} found more than 150 published definitions of drought. Besides the categorization into a conceptual or operational category , \autocite{wilhiteUnderstandingDroughtPhenomenon1985} proposed a clustering of these definitions into four types, namely meteorological drought, agricultural drought, hydrological drought and socio-economic drought. This classification is still widespread today \autocite{balintMonitoringDroughtCombined2013, baltiReviewDroughtMonitoring2020,idmpDroughtWaterScarcity2022,vereintenationenSpecialReportDrought2021}.

The conceptual category refers to a general formulation of an idea of drought to understand its concept and identify its boundaries and is often formulated in relative terms \autocite{wilhiteUnderstandingDroughtPhenomenon1985}. Definitions in the operational category try to define how drought functions in terms of its onset, duration, severity and spatial coverage also covering how this can be measured via indices \autocite{balintMonitoringDroughtCombined2013, nationaldroughtmitigationcenterWhatDrought, wilhiteUnderstandingDroughtPhenomenon1985}. With these definitions, the current situation is usually compared to a historical average, which is usually based on a 30-year period, which presupposes the development and continuous measurement of indicators and indices that can be used. \autocite{vereintenationenSpecialReportDrought2021,wilhiteUnderstandingDroughtPhenomenon1985}.

The four types of drought are commonly conceptually defined and brought into practice by operational specifications. They can be understood as different, but complementary stages of the same process and are generally cascading in reason and time but can overlap and are difficult to completely unravel. Figure \todo{TODO:, see https://drought.unl.edu/Education/DroughtIn-depth/TypesofDrought.aspx} shows an overview about the different types, their succession and cascading elements and table \todo{TODO: see RCRC 2020 p.11} displays the four types at a glance.

\missingfigure{https://drought.unl.edu/Education/DroughtIn-depth/TypesofDrought.aspx}

The \textit{meteorological drought} is usually characterized by the duration and the degree of dryness in comparison to the normal average amount and try to conceptually understand how weather patterns can impact water availability. These definitions are specific for a regions atmospheric conditions, e.g. regions with a year-round precipitations regime such as tropical rainforest need different definitions and thresholds than e.g. climates characterized by seasonal rainfall patterns \autocite{nationaldroughtmitigationcenterTypesDrought}. Operational categorization mostly involves using precipitation, moisture, temperature and wind indicators to determine the onset, severity, and duration of the drought.

\textit{Agricultural drought} definitions establish a connection between different features of meteorological drought with their impacts on agriculture. Soil-moisture, differences between actual and potential evapotranspiration and soil water deficits are some of the operationalized indicators for monitoring this type of drought \autocite{baltiReviewDroughtMonitoring2020,nationaldroughtmitigationcenterTypesDrought,wilhiteUnderstandingDroughtPhenomenon1985}.

The type of \textit{hydrological drought} is associated with the impact of meteorological drought on surface or subsurface water resources such as rivers, lakes, and groundwater. Hydrological drought occurs when these indicators drop below normal levels \autocite{palmerMeteorologicalDrought1965}. The fastest responding indicator of this type of drought is most often the variability of streamflow. The water levels of lakes and groundwater usually lag behind the occurrence of the meteorological or agricultural drought which is why the hydrological drought is often out of phase with the previously mentioned types. The hydrological drought is commonly defined on the basis of watershed or river basin scale \autocite{baltiReviewDroughtMonitoring2020,nationaldroughtmitigationcenterTypesDrought,wilhiteUnderstandingDroughtPhenomenon1985}.

The \textit{socioeconomic drought} differs from the aforementioned types as it can also incorporate features of these types of drought to associate them with the demand and supply of some social or economic good. It therefore relates the impact of all other types of droughts on human population and its various sectors of society such as food security, health, and the economy. It is therefore sometimes also interchangeably used with drought impacts. Operational categorization involves using socioeconomic indicators such as unemployment rates and food prices to assess the severity and duration of the drought \autocite{nationaldroughtmitigationcenterTypesDrought,wilhiteUnderstandingDroughtPhenomenon1985}.

\missingfigure{This is just a test.}

The shown economic, social and environmental impacts of drought in figure \todo{TODO:} depend on the severity of, and the risk to drought. These three concepts of impact, severity and risk are interrelated concepts used to assess and understand the effects of drought on various sectors. 

\todo{TODO: insert definitions of severity, risk and impact}
% definitions of severity, risk and impact --> short! two sentences! not more -> severity is not that important, impact comes later again and risk as well so let's get going.
%severity: https://www.earthdata.nasa.gov/topics/atmosphere/weather-events/droughts/drought-severity#:~:text=Drought%20severity%20is%20a%20relative,drought%20is%20considered%20to%20be.
% https://www.e-education.psu.edu/earth111/node/900
%“Severity refers to the departure from normal of an index” ([Svoboda et al., 2016, p. 11](zotero://select/groups/4773535/items/MMM69Q42)) ([pdf](zotero://open-pdf/groups/4773535/items/E2YACAV3?page=11&annotation=4KVPWXXJ))

Thereby, in alignment with the definition of \autocite{vanloonDroughtHumanmodifiedWorld2016} it is the exceptional severity of the water shortage that distinguishes drought from aridity, a ordinarily recurrent or fully dry climate, and from water scarcity as a long-term "supply/demand and natural and/or human-made phenomenon" \autocites[7]{idmpDroughtWaterScarcity2022}{vereintenationenSpecialReportDrought2021, vanClimatologicalRiskDroughts2017}. Water scarcity is described in more detail in the following chapter \ref*{subsec:water_scarcity}.

%----------------------------------------------------------------------------------------
%	SUBSECTION 2.2 Water Scarcity
%----------------------------------------------------------------------------------------

\subsection{Water Scarcity}\label{subsec:water_scarcity}
% human induced water shortage component
% what about water security? “Water Security: A Complex Concept” ([Butte et al., 2022, p. 1](zotero://select/groups/4773535/items/QB97YZ2M)) ([pdf](zotero://open-pdf/groups/4773535/items/Q936I2JN?page=1&annotation=XD3AGTA6))
% and insecurity? “Progress in household water insecurity metrics: a crossdisciplinary approach” ([Jepson et al., 2017, p. 1](zotero://select/groups/4773535/items/HWX5JRS4)) ([pdf](zotero://open-pdf/groups/4773535/items/NHEUUZI9?page=1&annotation=SE7MN8X2))

Water scarcity, as for water security or drought, is defined in many different ways. The sixth IPCC Assessment Report defines water scarcity broadly as "a mismatch between the demand for fresh water and its availability, quantified in physical terms" \autocite[560]{caretta2022water}. Here, social and economic components are outsourced to the broader concept of water security and insecurity, focussing primarily on physical water scarcity \autocite{caretta2022water}. In contrast, the Food and Agricultural Organization of the United Nations defines water scarcity as "a gap between available supply and expressed demand of freshwater in a specified domain, under prevailing institutional arrangements (including both resource ‘pricing’ and retail charging arrangements) and infrastructural conditions" \autocite[5]{faoCopingWaterScarcity2012} further summarizing that water security is "an excess of water demand over available supply" \autocite[6]{faoCopingWaterScarcity2012}. Thus, highlighting the human dimension of this interactive and relative concept of physical and economic water scarcity. Hereby, physical water scarcity refers to a situation in which there is not enough water available in quantitative terms to meet demand whereas economic water scarcity occurs when inadequate infrastructure, institutional or financial capital obstructs access to water resources "even though water in nature is available to meet human demands" \autocites{idmpDroughtWaterScarcity2022}[11]{moldenWaterFoodWater2007}.
Water scarcity and drought are in a complex interrelationship with each other. A short overview about the key differences between water scarcity and drought are given in table \todo{TODO:“Table 1. Characteristics and impacts of water scarcity and drought” ([IDMP, 2022, p. 3](zotero://select/groups/4773535/items/LNSL8VD2)) ([pdf](zotero://open-pdf/groups/4773535/items/JM82W3ZF?page=9&annotation=QNC4A3FG))}. 

\missingfigure{differences water scarcity to drought}
% Distinctions between water scarcity and drought:

% also in Joplin

%“Table 1. Characteristics and impacts of water scarcity and drought Water scarcity Drought Length Long-term to permanent Temporary (weeks to multiyear) Driving forces Demand–supply imbalance, human-driven, and/or natural (overexploitation, pollution). Climate change can impact both supply and demand Natural climate variability which can be modified/amplified by climate change Potential impacts Restricted water availability, environmental degradation, desertification, exacerbated inequalities in access to water resources, potential competition Water shortages, competition, environmental degradation Measures Long-term IWRM to bring supply and demand back into sustainable balance Integrated drought management, including: (1) monitoring and early warning; (2quality) vulnerability and impact assessment; and (3) risk mitigation, preparedness and response Source: adapted from Hohenwallner et al. (2011) DROUGHT AND WATER SCARCITY – DEFINITIONS AND CHARACTERISTICS” ([pdf](zotero://open-pdf/groups/4773535/items/JM82W3ZF?page=9&annotation=E3EQRILA))

Furthermore, potential mutual reinforcements, climate change, increased water use and poor water management can make it sometimes difficult to clearly separate these concepts \autocite{idmpDroughtWaterScarcity2022,lealfilhoUnderstandingResponsesClimaterelated2022,liuWaterScarcityAssessments2017,rcrcFORECASTBASEDFINANCINGEARLY2020}. Nonetheless, following the definition of \autocite{faoCopingWaterScarcity2012} the concept of water scarcity always gives water shortage, understood as absolute lack of water in the current situation, a human dimension in particular on the demand side. Here, the quality of policies, planning and management is considered as critical to the overall severity of the impact of water scarcity \autocite{idmpDroughtWaterScarcity2022,faoCopingWaterScarcity2012,vereintenationenSpecialReportDrought2021}. The supply side can be influenced by human activities, but it is not a mandatory prerequisite. \autocite{idmpDroughtWaterScarcity2022}. 
Besides the already mentioned water scarcity on the basis of physical quantity and economical factors, water scarcity can also be caused by water of unacceptable quality and lack of access to water services \autocite{faoCopingWaterScarcity2012}. The recognition that insufficient water quality is an additional contributing factor to water scarcity is a relatively recent development in the literature \autocite{liuThreedimensionalWaterScarcity2020} but together with inadequate access highlights further challenges in ensuring water security \autocite{caretta2022water, mishraWaterSecurityChanging2021}. 


%-----------------------------------
%	SUBSECTION 2.3 Water Access & Water Quality
%-----------------------------------

\subsection{Water Quality \& Access}
% + evtl. human health related water borne diseases and CBS

As could be seen in the previous chapter, besides the quantitative availability of water, its accessibility and quality are crucial. Inadequate water quality can be related to numerous health and environmental issues and can further limit the availability of water for given uses \autocite{rcrcFORECASTBASEDFINANCINGEARLY2020, faoCopingWaterScarcity2012}. Unlike the previous concepts, water quality has mostly fixed indicators by which the condition can be determined but historically, and still today, water quality assessment is primarily carried out in laboratories with preceding water sampling activities. This procedure not only makes the determination of water quality a laborious and costly process, but also places high demands on equipment and personnel, so that it is not sufficient for large-scale rural assessments. \autocite{tariqOpenSourceWater2021,worldmeteorologicalorganizationPlanningWaterqualityMonitoring2013}. While simpler methods for in situ water quality monitoring exist, they are either insufficient or often still need to much investment and knowledge to conduct for widespread and frequent monitoring \autocite{worldmeteorologicalorganizationPlanningWaterqualityMonitoring2013}. Nonetheless, new solutions are being developed to simplify and scale affordable water quality assessments to rural areas e.g. \autocite{ighaloComprehensiveReviewWater2020,tariqOpenSourceWater2021}. While the direct assessment of water quality might be challenging, poor water quality can be linked to other factors. Environmental awareness, poor sanitation and hygiene conditions of people in rural areas were for example considered as major causes for contamination of water at the source \autocite{zamxakaMicrobiologicalPhysicochemicalAssessment2004}.

The definition of water access is again a rather challenging undertaking. The \autocite[254]{worldbankWorldDevelopmentReport1997} defined water access in rural areas by "access implies members of the household do not have to spend a disproportionate part of the day fetching water." While both time and distance still play a crucial role in literature when investigating water access \autocite{cassiviDrinkingWaterAccessibility2019,cassiviEvaluatingSelfreportedMeasures2021,emenikeAccessingSafeDrinking2017}, the term also gained a social component \autocite{emenikeAccessingSafeDrinking2017,mitlinUnaffordableUndrinkable}. \autocite{obeng-odoomAccessWater2012} adds four additional factors namely, affordability, quality, equitable distribution to the definition of water access to fully understand if users have access to water in daily live. \autocite{unitednations/developmentprogrammeDeepeningDemocracyFragmented2002} links these parameters to the access to an improved water source which should provide safe drinking water.
The access to improved water sources is therefore generally considered as crucial in the reaching of water security \autocite{cdcAssessingAccessWater2022}. Proactive measures to drought and water scarcity can not only potentially minimize or even neutralize impacts and are considerably more cost-efficient, early warning and anticipatory actions for drought and water scarcity impacts become ever more important \autocite{faoandun-waterProgressLevelWater2021,idmpDroughtWaterScarcity2022,worldbankHighDryClimate2016}.

%----------------------------------------------------------------------------------------
%	SECTION 4 FbF, EAP, AA & Early Warning
%----------------------------------------------------------------------------------------

\subsection{indicators with risk/vulnerability, indices and impact}\label{subsec:indicators} % yeah well... choose another title.

Indicators and Indices are often used to translate complex matters into easier to explain numbers and scales that can be measured, tracked and reasonably compared \autocite{blauveltSystematizingEnvironmentalIndicators2014,williamsUsingIndicatorsExplain2017}. This can range from capturing simple measurements to complex and detailed issues that can not only depict ecological conditions but its interactions with societies \autocite{blauveltSystematizingEnvironmentalIndicators2014,mishraWaterSecurityChanging2021}. Indicators and Indices can thus establish a clear and common understanding of a concept or parts of it in a quantifiable and more objective way.
Here, an indicator is understood as a measurable parameter that provides information on the state or trend of an issue or problem. It can be a physical, chemical, biological, or socio-economic variable, such as temperature, soil moisture or streamflow and can be measured locally or remotely. An Index is a composite measure that aggregates multiple indicators into a single value or score \autocite{unitednationsuniversityTooManyIndicators2017,williamsUsingIndicatorsExplain2017, svobodaHandbookDroughtIndicators2016}. Indices are commonly developed on regional or national level to account for the specific circumstances \autocite{unitednationsuniversityTooManyIndicators2017}. This case specification, together with different measurement and aggregation methods, partial inconsistency of definitions and differently focussed objectives on qualitative, quantitative, risk or impact scenarios can constrain their practical application and intercomparability \autocite{svobodaHandbookDroughtIndicators2016,unitednationsuniversityTooManyIndicators2017}. 
Since there is no one definition of drought, water scarcity or security, there is no one best solution to the choice between the many indicators and indices for either of those.

% all indicators are proxies -> good information -> include that somewhere
%“There exist many physical indicators of a drought that are monitored by scientists and governments to track the development of drought impacts. The complex and insidious nature of drought means that all these indicators are proxies to understand the impacts that dry conditions are having on an area. Our main suggestion here would be to examine the World Meteorological Organisation Handbook of Drought Indicators and Indices (which are classified in a traffic-light method of ease of use) identifies which indices could be available and appropriate for the context.” ([RCRC, 2020, p. 15](zotero://select/groups/4773535/items/UESIQTRJ)) ([pdf](zotero://open-pdf/groups/4773535/items/P5JPVZ97?page=15&annotation=XR5BI5Y7))


% possibly shorten this.. naming all these indices might be a little overkill, though naming none is also not feasible.
Precipitation, evapotranspiration, soil moisture, lake and groundwater levels, streamflow and vegetation water stress are among the most prominent drought indicators \autocite{europeandroughtobservatoryDroughtIndicators2017}. In order to adequately account for the different drought stages different drought indices, that aggregate these and other indicators, are applied. Among the most prominent meteorological drought indics are the Standardized Precipitation Index \textit{SPI} together with its extension the Standardized Precipitation-Evapotranspiration Index \textit{SPEI} \autocite{europeandroughtobservatoryDroughtIndicators2017,ncarStandardizedPrecipitationEvapotranspiration,ncarStandardizedPrecipitationIndex}. Agricultural drought indices like the Soil Moisture Anomaly \textit{SMA} or the Anomaly of Vegetation Condition \textit{FAPAR Anomaly} are based on soil moisture indicators and absorbed radiation fractions, respectively. By quantifying water flow volumina, the Low Flow Index \textit{LFI} belongs to the hydrological drought indices \autocite{europeandroughtobservatoryDroughtIndicators2017, svobodaHandbookDroughtIndicators2016}. In addition to these and other types of indices, such as Combined Drought Indices, the \textit{Handbook for Drought Indicators and Indices} lists over 50 drought indicators and indices. For further and more in-depth information, please refer to the interactive website of the \arcfull{IDMP} launched by the \acrfull{wmo} and \acrfull*{gwp} \autocite{idmpIndicatorsIndicesIntegrated2021}. 

All of these drought indices give a good impression about the physical side of climate anomalies, but none of the above mentioned indices link those climate anomalies to socioeconomic vulnerabilities \autocite{enenkelWhyPredictClimate2020}. \autocite{mishraWaterSecurityChanging2021} argue, that the framing of water security challenges extends beyond singular indicators. \autocite{lackstromBackyardHydroclimatologyCitizen2022} argue further, that assessments that only consider physical factors overlook the broader impact of drought on social, economic, and ecological systems.
The simple but widely used Falkenmark Indicator (Falkenmark et al. 1989) incorporates human factors by calculating a ratio between the given amount of water and the number of people living within that domain. By further categorizing this ratio to a level of water scarcity, the Falkenmark Indicator indicates the supply sides effects of water scarcity but variabilities, demand and socioeconomic factors are not represented. More dedicated indices like the \acrfull*{iwmi} Indicator and the \acrfull*{wpi} as well as other indices measuring water security give a more extensive representation of the overall situation \autocite{arreguin-cortesMunicipalLevelWater2019,liuWaterScarcityAssessments2017}. The \arcshort{wpi} for example represents the weighted average of five pre-standardized components namely, water availability, access, capacity, use and environment \autocite{sullivanWaterPovertyIndex2003}.

Determining the right set of indicators and indices for a given region to e.g. assess hazard severity depends on the objective and available data and is often a balancing act between many factors and circumstances \autocite{svobodaHandbookDroughtIndicators2016}. Besides the pure description of what certain natural or social circumstances \textit{are}, there is an growing interest to understand what these conditions will \textit{do} \autocite{boultDroughtImpactbasedForecasting2022, lackstromBackyardHydroclimatologyCitizen2022}.

The effects of these conditions on the ground are most often called the \textit{impact} of a certain weather phenomenon or climate development such as a drought hazard. Impacts can be direct or indirect and a generally difficult to quantify economically \autocite{vereintenationenSpecialReportDrought2021}. The level of impact is commonly determined based on the severity of the hazard, the exposure of the investigated elements and their respective vulnerabilities \autocite{harrowsmithFutureForecastImpact2020,svobodaHandbookDroughtIndicators2016,vereintenationenSpecialReportDrought2021}.
This concept is generally expressed by the risk equation

        \[Risk = f(Hazard, Exposure, Vulnerability)\]

    where

        \[Vulnerability = f(Level of Coping Capacity, Level of Adaptive Capacity)\]

\autocite{boultDroughtImpactbasedForecasting2022,harrowsmithFutureForecastImpact2020,vereintenationenSpecialReportDrought2021}. Drought hazard can be evaluated and described by the above mentioned indicators and indices with difficulties lying in the contextualization and setting of the threshold levels to separate between fluctuations within the normal range and extreme events. Exposure is commonly defined as social, economic, cultural or natural assets, services or resources in places that could be adversely affected by a hazard \autocite{ipccClimateChange20142014}. Exposed elements can be more ore less vulnerable to the hazard. Vulnerability conditions are determined by the sensitivity or susceptibility of a system, community or individual to physical, social, economic or environmental factors or processes \autocite{ipccClimateChange20142014}. These conditions are often further described as the level of coping and adaptive capacities. Coping capacities refer to available skills and resources of systems, organizations or individuals to address, manage and overcome unfavourable circumstances \autocite{ipccGlossaryTerms2012}. In the same manner, adaptive capacities relate to preparation, reduction and moderation of those impacts.

The establishement of a functional relationship between the hazard, exposure and vulnerability to its impact can be rather difficult for numerous reasons and is further discussed by \autocite{boultDroughtImpactbasedForecasting2022} for interested readers. Moreover, all these factors change over time, so that the quality of the calculations depends strongly on the timeliness of the data basis \autocite{harrowsmithFutureForecastImpact2020}. 

Relatively recent approaches argue for numerous benefits and reasons for greater inclusion of local knowledge and community integration in these approaches \autocite{balehegnIndigenousWeatherClimate2019,dubeFrameworkIntegrationTraditional2016,ebhuomaFrameworkIntegratingScientific2020,giordanoIntegrationLocalScientific2013a,greyIntegratingLocalIndigenous2020,hermansExploringIntegrationLocal2022a,mercerCultureDisasterRisk2012,mutasaKnowledgeApartheidDisaster2015,nyetanyaneIntegrationIndigenousKnowledge2020,nyongValueIndigenousKnowledge2007}. Another emerging area in scientific interest is the gender inequality of drought impacts \autocite{acharyaWhenRiverTalks2019,fanningDroughtDisplacementLivelihoods2018,hiwasakiLocalIndigenousKnowledge2015,mustafaGenderingFloodEarly2015,sachsRoutledgeHandbookGender2020,saniGenderOtherVulnerabilities2022}. Although these topics are of great interest, they fall largely outside the scope of this particular work.

An understanding of the severity of droughts and their current impacts enables targeted responses, as well as to allow for the development of future predictions based on current conditions. In this context, recent efforts have increasingly emphasized proactive and forward-looking measures in disaster relief initiatives. The forthcoming chapter will explore this relatively recent shift in approach and its implications for improving drought management strategies.


\section{FbF, EAP, AA & Early Warning + trigger}
% one possible solution to prevent impact
% FbF
% IFRC & RCRC
% EAPs
% Early Warning/Actions &  Anticipatory Actions
% triggered by forecast

Traditionally, disaster management efforts have primarily focused on long-term preparedness or post-disaster response, thus only providing assistance and relief to affected communities after a disaster has occurred (TODO: policy overview, hyogo framework [UNISDR], coughland et al 2015). The lack of standardized procedures for forecast-based actions led to disaster warnings often going unheard \autocite{kolenImpactsStormXynthia2013}. In the context of increasing frequency and severity of natural disasters, coupled with the impacts of climate change, the need for a more proactive approach that can reduce the impact of disasters on vulnerable communities became apparent \autocite{coughlandeperezForecastbasedFinancingApproach2015,trisosAfrica2022}. Nonetheless, financial resources were for the time being strongly directed towards post-disaster response and incentives to invest in new and complex scientific developments including relatively high uncertainties were limited \autocite{coughlandeperezActionbasedFloodForecasting2016}. This changed with the development and successful integration of several new forecast-based financing systems that utilized the opportunity gap between a forecast and the disaster to successfully reduce corresponding impact. Based on this, to "substantially increase the availability of and access to multi-hazard early warning systems and disaster risk information and assessments to people by 2030" became one of seven global targets of the Sendai for Disaster Risk Reduction 2015-2030 framework \autocites{coughlandeperezActionbasedFloodForecasting2016}[12]{undrrSendaiFrameworkDisaster}. Today, large institutions have now specialized sections for the financing of Early Actions such as the Climate Risk and Early Warning Systems Initiative \textit{(CREWS)} and the Global Risk Financing Facility \textit{(GRiF)} to support and backup \acfp{ea} \autocite{crewsClimateRiskEarly,GlobalRiskFinancing}. Forecast-based Financing \textit{(FbF)} has thus emerged as a promising approach to disaster management that enables proactive, timely, and cost-effective responses to disasters \autocite{coughlandeperezForecastbasedFinancingApproach2015} (TODO: add “FORECAST-BASED FINANCING An innovative approach” ([pdf](zotero://open-pdf/groups/4773535/items/3C2CE7BS?page=1&annotation=UKWEKCTA))).

The IFRC together with the Red Cross Climate Center \textit{(RCCC)} and German Red Cross \textit{(GRC)} have developed and improved the FbF programme to fund EAs since 2007 \autocite{ifrcForecastbasedFinancingNew2019}. 

\missingfigure[options]{“Figure 1 - FbF Diagram” ([RCRC, 2020, p. 3](zotero://select/groups/4773535/items/UESIQTRJ)) ([pdf](zotero://open-pdf/groups/4773535/items/P5JPVZ97?page=3&annotation=W3UC7H26))}

Following \autocite{coughlandeperezForecastbasedFinancingApproach2015, coughlandeperezActionbasedFloodForecasting2016} the structure of FbF can be distilled down to:
    “When forecast states that an agreed-upon probability threshold is exceeded for a hazard of a designated magnitude, then an action with an associated cost must be taken that has a desired effect and is carried out by a designated organisation.” \autocite[2]{coughlandeperezActionbasedFloodForecasting2016}.
Thus, the FbF approach involves three key components (1) triggering (2) pre-defined EAs and securing a (3) financing mechanism in advance (compare \ref{TODO: figure fbf}) (TODO: “Forecast-based Financing A new era for the humanitarian system” ([pdf](zotero://open-pdf/groups/4773535/items/KQZXSWVN?page=1&annotation=3BW2ZYST))). These components are summarized in an Early Action Protocol \textit{(EAP)} (TODO: cite policy overview “These three components are summarized in Early Action Protocols (EAPs).” ([“Forecast-based financing: A policy overview”, p. 2](zotero://select/groups/4773535/items/35XBEGJ7)) ([pdf](zotero://open-pdf/groups/4773535/items/8YZAQB5L?page=2&annotation=58UQZK6T))). 

\missingfigure{“FBF has three” ([pdf](zotero://open-pdf/groups/4773535/items/K89MIG2V?page=3&annotation=PF946AET))}

%-----------------------------------
%	SUBSECTION 4.1 Early Action Protocol
%-----------------------------------

\subsection{EAP and drought specifics}

In the \acrfull*{eap} triggers, actions to be taken and financing mechanisms are clearly outlined, thus summerizing and explicitly assigning responsibilities to the involved actors, ensuring that everyone understands their role and task in the event of activation \autocite{ruthForecastbasedFinancingPolicy2017}. This results in clear accountability and full commitment from all stakeholders, facilitating the timely and efficient implementation of the predetermined actions \autocite{ruthForecastbasedFinancingPolicy2017}.

Two types of analyses, namely the identification of forecasts and the risk assessment, form the basis for specifying the trigger, affected regions, and selected actions in the \acshort*{eap} (see Figure \ref*{TODO:}). 


\missingfigure{“Figure 2 - EAP Validation Steps” ([RCRC, 2020, p. 4](zotero://select/groups/4773535/items/UESIQTRJ)) ([pdf](zotero://open-pdf/groups/4773535/items/P5JPVZ97?page=4&annotation=6WWVYXLT))}


Both assessments are primarily based on historic data and experiences. To identify suitable forecast(s), various forecasts are compared and analyzed in terms of their capacities and performance to predict hazards. This is done mainly through a historically grounded analysis. Ultimately, a specific impact treshold based on one or a combination of several impact-based forecasts becomes the basis for triggering actions. This trigger also depends on the outcome of the risk assessment, as the impact of the hazard is highly influenced by the risk on site \autocite{ifrcFbFPractitionersManual2023,ifrcForecastbasedFinancingNew2019}.

The risk assessment is a complex analysis that takes numerous factors on scales of the hazard, and its sub-hazards, exposure, vulnerability and together with its coping and adaptive capacities, into account \autocite{ifrcFbFPractitionersManual2023}. Potential inputs depend strongly on the respective hazard and can range from records of historical events, housing location and building structures in the case of hurricanes and floods to social factors like income, demographics and school attendance. The objective being the identification of corresponding impact levels, thus determining the most effective actions and allocating resources as objectively as possible. Nonetheless, most of these parameters are proxies, as direct information about locallized impact is seldom, outdated, of low accuracy or quality \autocite{ifrcFbFPractitionersManual2023}.

Due to the majority of the implemented \acrshortpl{eap} concentrating on fast onset disasters such as floods, hurricanes or strong rains, the FbF concept were primarily focussed and developed in this regard. Here, typically a sole trigger and its associated set of actions are established, emphasizing rapid responses, given that there is often less than 48 hours between the activation and the occurrence of the disaster. \autocite{rcrcFORECASTBASEDFINANCINGEARLY2020}. Drought as a usually slow-onset hazard, on the other hand, pose unique structural challenges to the process of determining thresholds to trigger actions as impact builds up over time and is highly dependent on the context \autocite{boultDroughtImpactbasedForecasting2022}. These challenges of identifying a forecast, determining a trigger and seclting actions are further outlined in the coming chapters.

%TODO: chechen ob das auch wirklich so ist..
The specification of the financing mechanism as one of the three key components will not be covered in any further detail in this work, as the International Federation of Red Cross and Red Crescent Societies \textit{(IFRC)} has extended their Disaster Relief Emergency Fund \textit{(DREF)} with Forecast-based Action as dedicated mechanism to adequately support their increased numbers of FbF projects. Once the forecast-based trigger is met and the EAP is activated, the financing mechanism automatically assigns resources, which solves the issue of financing to a large extent and is therefore no longer of great interest to this piece of work.

%-----------------------------------
%	SUBSECTION 4.2
%-----------------------------------

\subsection{Forecasts} % melt these two together

Indicators and indices as discussed in chapter \ref*{subsec:indicators} measure the severity, duration and spatial coverage of hazard conditions based on historical and current weather data. They provide a snapshot of current conditions and serve as an indicator of the overall situation. Forecasts, on the other hand, use these indices together with climate models and weather data to predict future conditions and provide early warning of potential hazard events. Thus, forecasts extend the retrospective and current measures of indices to future prediction.
Similar to the indices, a single forecast usually only covers certain facets of a hazard. In the case of droughts, the thematic orientation commonly follows its definition classification into meteorological, hydrological and agricultural subdivisions. Furthermore, forecasts can additionally be categorized into global, continental or regional spatial scales with coarser scaling predictions mostly correlating with longer time spans and vice versa \autocite{baltiReviewDroughtMonitoring2020}. Global to continental meteorological drought forecasts with the focus on seasonal or inter-seasonal predictions are often based on same scale phenomenons such the Julian-Madden Oscillation, the ENSO cycles or the Indian Ocean Dipole \autocite{andersonMaddenJulianOscillationAffects2022,goreUnderstandingInfluenceENSO2020,yuanInfluencesIndianOcean2008}. These conditions are mostly collected through satellite and weather data often utilizing drought indices such as the SPI, SPEI and EDDI indices \autocite{kimIntegratedDroughtMonitoring2021}. Further drought prediction services such as the National Integrated Drought Information System of the US government, the European Drought Observatory (EDO) or its adaptation, the East African Drought Watch, utilize a wide range of different indices to predict hazard development and their impacts \autocite{europeandroughtobservatoryDroughtIndicators2017,icpacDroughtIndicators2023, nidisOutlooksForecasts2023}. These institutions also produce timely forecasts, but their data sources are usually based on the same remote evaluations mostly predicting what the weather and climate will be, and not what its implications on the ground will look like \autocite{enenkelWhyPredictClimate2020}. 

The transition to impact-based forecasts represents a radical shift in the way these forecasts are produced and opperationalized \autocite{ifrcFbFPractitionersManual2023}. Practically, this would change the information that a forecast would provide from predicting e.g. precipitation patterns to e.g. the magnitude and spatial coverage of crop failure \autocite{harrowsmithFutureForecastImpact2020}. The challenges of functional relationships, complex interconnected cause and effect networks and data availability mentioned in chapter \ref*{indicator} are also applicable here, but the change to impact-based information results in multiple benefits to practitioners nonetheless. Impact-based forecasts help with the identification and prioritization of areas and communities most severely impacted. They do this by supporting a transparent, evidence-based, sector- and context-specific decision-making process directly focussing the population at risk \autocite{ifrcFbFPractitionersManual2023}.

\autocite{boultDroughtImpactbasedForecasting2022} argue even further for an adapting and dynamic impact assessment process, as decadal shifts in climate variabilities, changing exposure and vulnerabilities are not incorporated in a pre-defined system. They propose a hybrid framework of multi-hazard forecasts interlinked with static vulnerability and dynamically adjusted with real-time expert vulnerability assessments. Threshold triggers are lower, where static vulnerabilities are higher. However, both the regular pre-defined impact forecast and the dynamic impact forecast must be preceded by a selection and definition of triggers and actions.


%%%%%%%%%%%%%%%%%%%%%%%%%% TRIGGER %%%%%%%%%%%%%%%%%%%%%%%%%%%%%%%%
%%%%%%%%%%%%%%%%%%%%%%%%%%%%%%%%%%%%%%%%%%%%%%%%%%%%%%%%%%%%%%%%%%%

\subsection*{Trigger definition}

%% Trigger:
“Triggers are mainly combination of hydro-meteorological forecast combined with exposure and vulnerability data” \autocite[19]{rcrcFORECASTBASEDFINANCINGEARLY2020}. There are commonly two ways to define a trigger for early actions. On one side, triggers can be consensus-based, meaning experts make real-time judgements by synthesizing information from multiple sources, or on the other side, triggers are data-driven, peer-reviewed and validated well in advance of a potential event \autocite{rcrcFORECASTBASEDFINANCINGEARLY2020}. Drought with its different layers of complexity may also benefit from a combination of these mechanisms, as e.g. the framework of \autocite{boultDroughtImpactbasedForecasting2022} proposed above shows. Generally, good conditions for effective trigger development are sufficient historical data, knowledge about local livelihoods and how diverse parts of communities are influenced differently, thorough identification of differentiated impact drivers and their correlation to magnitudes as well as trustworthy forecasts \autocite{coughlandeperezForecastbasedFinancingApproach2015,coughlandeperezActionbasedFloodForecasting2016,elisabethstephensFORECASTBASEDACTION2015,harrowsmithFutureForecastImpact2020,rcrcFORECASTBASEDFINANCINGEARLY2020}. 
Furthermore, the framing and definition of the underlying forecast, indices and indicators are paramount as data-driven triggers are "specific values of an indicator or index that initiate and/or terminate each level of a drought plan and associated mitigation and emergency management responses.” \autocites{rcrcFORECASTBASEDFINANCINGEARLY2020}[13]{svobodaHandbookDroughtIndicators2016}. This specification is highly context specific and e.g. in the case of flood can be defined as the level when the river breaches its banks and inundates the surrounding area. Though, in another area this overflow may only inundate open space and thus lead to no impact at all \autocite{elisabethstephensFORECASTBASEDACTION2015}. This circumstance is relatively easy to grasp, has a single trigger and one set of specified actions such as evacuation, transportation and early warning and is therefore well integrable and implementable (see upper illustration in figure \ref*{TODO: RCRC p.20 Figure 5}) \autocite{siahaanForecastbasedActionDREF2018}. 

\missingfigure{RCRC p.20 Figure 5}

Drought, due to its slow-onset and potentially cascading impacts that only builds up over time complexifies the process of trigger definition as \acrfullpl{aa} to some impacts may go hand in hand with active responses in some areas and be to early in others. Furthermore, forecast certainty, granularity and accuracy all decrease the more one looks into the future \autocite{rcrcFORECASTBASEDFINANCINGEARLY2020}. Deciding when to trigger is therefore a critical and challenging aspect of conceptualizing a drought \acrshort{eap} (see bottom illustration in figure TODO: RCRC p.20 Figure 5). Practitioners and experts interviewed by the \autocite{rcrcFORECASTBASEDFINANCINGEARLY2020} advocate for a staggering triggering system \autocite. Here, multiple triggers with different sets of \acrshortpl{aa} would extend the single trigger mechanism and give the opportunity to account for the different phases and the inherent complexity of the phenomenon drought. Moreover, the \autocite[30]{rcrcFORECASTBASEDFINANCINGEARLY2020} calls for the development of "unconventional triggers for \acrfull*{FbA}" as the trigger development is not yet complete.

%----------------------------------------------------------------------------------------
%	SECTION 5 Crowdsensing, VGI, alternatively satellite image interpretation
%----------------------------------------------------------------------------------------

\subsection{Anticipatory Actions}

% keep it concise. It is not complicated. No reason to blow it up.

Anticipatory Actions are at the heart of every EAP and their execution is what everything is working towards. The goal of every Anticipatory Action is to help people and communities at risk to reduce negative impacts of a hazard. The final execution is preceded by some conceptual and practical steps. The establishment process begins with the identification of contextually meaningful, suitable and locally realisable actions with special focus on stakeholders, resources and available lead-time. These are further prioritized and selected based on the risk assessment, type and magnitude of hazard, and forecasting capabilities. When a first set of \acrshort*{aa} is defined, they are worked through in detail, reflected on with stakeholders and ultimately finalised. Together with an evaluation phase, this process is often a simultaneous and iterative process which also does not stop with the operationalisation of the \acrshort{eap} \autocite{elisabethstephensFORECASTBASEDACTION2015,ifrcGlossaryTermsForecastbased2023,ifrcFbFPractitionersManual2023a,rcrcFORECASTBASEDFINANCINGEARLY2020}.
In practice, \acrlongpl{aa} are commonly split into a preparation and an activation phase. The preparation phase builds on the process described above, but also extends to actions that prepare for rapid activation, such as the prepositioning of water tablets before the rainy season \autocite{elisabethstephensFORECASTBASEDACTION2015}. The activation phase requires a constant operation of forecast monitoring and is initiated when the trigger is reached. Timely information dissemination, releasing and receiving funds, implementing of the \acrshortpl{aa} and subsequent evaluation are part of this phase \autocite{elisabethstephensFORECASTBASEDACTION2015,ifrcFbFPractitionersManual2023a}. Often, \acrshortpl{aa} are not very different from response actions except of their predictive and proactive nature. However, this foresight comes with the cost of uncertainty and forecasts may not always be accurate. The simultaneous implementation of \acrshortpl{aa} in the absence of the disaster is commonly refered to as \textit{to act in vain} \autocite{coughlandeperezForecastbasedFinancingApproach2015}. Besides financial costs, this may also manifest in reputaional costs in e.g. the case of Early Warning and evacuation if false alarms occur too frequently \autocite{elisabethstephensFORECASTBASEDACTION2015}. Albeit, a growing body of evidence suggests that the benefits of AAs outweigh the costs substantially \autocite{cabotventonEconomicsResilienceDrought2018,coughlandeperezForecastbasedFinancingApproach2015,gualazziniEWEAEarlyWarning2021}. Furthermore, the issue of \textit{acting in vain} can be lessen by staggering triggers and adjusting AAs in accordance with long-term resilience building \autocite{wfpMonitoringEvaluationAnticipatory2021}. This can allocate the actions more precisely and increases the general benefits. \autocite{ifrcGlossaryTermsForecastbased2023} makes these design adjustments the basis of its definition of \textit{acting in vain} and thus argues for the abolition of this term, since the benefits of acting should always outweigh not acting at all.


\section{Citizen Science, Crowdsensing, Volunteersensing, VGI,  alternatively satellite image interpretation}

The inclusion of local knowledge in the system of Early Warning and Anticipatory Action can result in many benefits as already mentioned in the end of chapter \ref*{subsec:indicators}. Adapting knowledge and policies to local conditions and people as well as learning from them, strengthening autonomous responses and involving local stakeholders in all stages of the processes are just some of the potential ways to improve implementations \autocite{giordanoIntegrationLocalScientific2013a,idmpDroughtWaterScarcity2022,lackstromBackyardHydroclimatologyCitizen2022,lealfilhoRoleIndigenousKnowledge2022,lealfilhoUnderstandingResponsesClimaterelated2022}. One way to include local knowledge is through Citizen Science, very broadly defined as "public participation in scientific research and knowledge production" \autocite{fraislCitizenScienceEnvironmental2022} .
Historically, the first citizen science project was possibly the Christmas Bird Count run by the National Audubon Society in the USA every year since 1900 \autocite{linkHierarchicalModelRegional2006,silvertownNewDawnCitizen2009}. Since around 2000, the number of publications in regard to Citizen Science has risen substantially and has established itself as a vibrant area of scientific interest \autocite{kirschkeCitizenScienceProjects2022}. As more and new thematic fields joined this area of interest, numerous approaches have been made to define Citizen Science more precisely \autocite{haklayWhatCitizenScience2021}. Over 30 definitions were selected by \autocite{haklayWhatCitizenScience2021} to explore their ambiguity and extend the best practice principles and characteristics of citizen science established by the European Citizen Science Association (ESCA) \autocite{escaTenPrinciplesCitizen2015,escaECSACharacteristicsCitizen2020}. Different political, scientific or societal lenses along with a variety of focal points such as (1) biology, conservation and ecology, (2) geographic data and (3) social sciences and health related issues have all contributed to the concept of Citizen Science \autocite{haklayWhatCitizenScience2021,kirschkeCitizenScienceProjects2022,kullenbergWhatCitizenScience2016}.
The first, natural research and conservation, is the orientation most frequently related to Citizen Science with overlapping concepts to community-based, volunteer and participatory monitoring. It has common interests with the second category of Volunteered Geographic Information (VGI) in topics such as crowdsourcing and data quality whereas the the third category mostly resolves around public engagement with intersections to CS in public participation \autocite{kullenbergWhatCitizenScience2016}. In order to highlight the core of Citizen Science alongside the different disciplinary orientations of the research, different frameworks, guidelines and levels of participation have been designed.\autocite{kirschkeCitizenScienceProjects2022} created a three cluster framework of design principles around \textit{citizen} and \textit{institutional} characteristics, together with their \textit{forms of interaction}. Within these categories \autocite{kirschkeCitizenScienceProjects2022} highlight various qualities and skills such as age, social status, motivation, knowledge and education of the contributing citizens, financial and human resources on the institutional side and the method and density of communication and feedback practices as important parts of interactions. Guidelines and principles further specify, expand and structure these broad topics to make them practically applicable in various contexts \autocite{citizenscience.govBasicStepsYour,escaTenPrinciplesCitizen2015,escaECSACharacteristicsCitizen2020,EUCitizenScience2023,fraislCitizenScienceEnvironmental2022,garciaFindingWhatYou2021,minkmanCitizenScienceWater2015,pocockStrategicFrameworkSupport,skarlatidouWhatVolunteersWant2019}. Citizen science projects can also be differentiated according to how engagement with participants is designed. This is referred to as the \textit{levels of participation} and is commonly structured into four levels. Increasing in participation intensity, \autocite{buckinghamshumGlobalParticipatoryPlatform2012} categorize them into (1) Crowdsourcing, (2) Distributed Intelligence, (3) Participation Science and (4) Extreme Citizen Science. Following this categorization, participants can be (1) 'sensors', (2) 'interpreters', (3) engaged in problem definition and data collection or even (4) part of the analysis. 
Depending on the level of participation and thematic orientation, Citizen Science is related to concepts of classic monitoring practices (1), transdisciplinary research emphasizing engagement of the public along the entire process (2 & 3) and participation involving "groups that are or perceive themselves as being affected by the decision" (3 & 4) \autocites{buckinghamshumGlobalParticipatoryPlatform2012}{conradReviewCitizenScience2011}{minkmanCitizenScienceWater2015}[1]{rennParticipatoryProcessesDesigning2006}. 
Current challenges and limitations in Citizen Science projects are the complex demands in the conceptualization and design process with a wide range of required skills and resources, recruiting participants and sustaining their motivation, data quality and accuracy considerations, biases in collection and analysis as well as privacy regulations \autocite{fraislCitizenScienceEnvironmental2022}. Furthermore, research and CS projects are currently unevenly distributed on a global scale with an over representation of North American countries resulting in less experiences and guidelines for other areas and contexts \autocite{kirschkeCitizenScienceProjects2022, zhengCrowdsourcingMethodsData2018}. Nonetheless, numerous studies suggest promising developments and application possibilities addressing all of the above mentioned challenges in design, participants and data related issues \autocite{buckinghamshumGlobalParticipatoryPlatform2012,buddeParticipatorySensingParticipatory2017,escaECSACharacteristicsCitizen2020,fraislCitizenScienceEnvironmental2022,lowryGrowingPainsCrowdsourced2019,pocockStrategicFrameworkSupport,ruttenHowGetKeep2017,weeserCitizenSciencePioneers2018a}. 
% come back to this in the discussion --> data quality e.g. can be 'solved' by trainings and supervision

\subsection{Community-based monitoring}\label{subsec:cbm}

% maybe add that there are more related concepts: participatory monitoring instead? or: citizen observatories, community based monitoring and participatory monitoring

\acrfull{cbm} is a sub-concept of citizen science and can be allocated to different layers of participation, depending on its definition, aspects and final implementation \autocite{westonCommunityBasedWaterMonitoring2015}. \acrshort{cbm} can encompass "a process where concerned citizens, government agencies, industry, academia, community groups and local institutions collaborate to monitor, track and respond to issues of common community concern" \autocite[410]{whitelawEstablishingCanadianCommunity2003}. The focus of \acrshort*{cbm} on monitoring is fundamental, but the monitored subject, further handling of the data and the involvement of the participants can vary widely \autocite{baptisteCommunityLedMonitoringWhen2020,conradReviewCitizenScience2011,koehlerCitizenParticipationCollaborative2008,muhamadkhairCommunitybasedMonitoringEnvironmental2021,shirkPublicParticipationScientific2012,westonCommunityBasedWaterMonitoring2015}. Within this work, \acrshort*{cbm} is understood as a combination of two main aspects. The collection part often refers to concepts of \textit{Crowdsourcing} or \textit{Crowdsensing} (see next Chapter \ref{subsec:mcs}) and a management aspect which promotes the incorporation of the generated information into community decision-making processes \autocite{conradCommunitybasedMonitoringScience2007, keoughAchievingIntegrativeCollaborative2006}. 
\acrlong*{cbm} can serve many purposes but its implementation and application is not always recommended. Therefore, many guidelines precede the design with an assessment of the feasibility of this approach \autocite{associationTenPrinciplesCitizen2015,citizenscience.govBasicStepsYour,fraislCitizenScienceEnvironmental2022,minkmanCitizenScienceWater2015, pettiboneCitizenScienceAll2016}. Here, the challenges, benefits and capabilities of the \acrshort*{cbm} approach are compared with the problem and core objectives of the project. It is emphasized that \acrshort*{cbm} should not be the goal itself, but only a means to fulfil the project goals \autocite{minkmanCitizenScienceWater2015}. Nonetheless, the diversity of this approach means that other goals can be pursued and achieved apart from the main interests (see Chapter \ref*{subsec:guidelines}). For example, enriching participants by addressing their needs, advancing their knowledge or teaching them new skills is considered as fundamental and important to achieving the main objective as it is to a successful project \autocite{fraislCitizenScienceEnvironmental2022}. 
In the following, a short overview about challenges, benefits and recommendations of \acrshort*{cbm} is given, broken down in the design phase, incorporation of participants and data concerns.

%%%%%%%%%%%%%%%%%%%%%%%%%%%
% possibly subsubsec ??
% design 
The conceptualization of CBM projects on the level of participation or the tripartite division according to characteristics of citizens, institutions and their forms of interaction have already been mentioned in connection with the broader concept of Citizen Science and are also applicable here. More concrete design factors and variables were synthesized by \autocite{kirschkeCitizenScienceProjects2022} but the systematic understanding of their influences on the success of remained unclear for now. A selection of subjects outside of the original research itself could be overall project management, communication in its various forms and with all stakeholders, community and participant recruitment, training and management, data management and analysis as well as the final implementation and operation of the project. Moreover, there is agreement that no \textit{one-size-fits-all} solution exists and different goals, resources, and contexts have considerable influence on the design from project to project \autocite{fraislCitizenScienceEnvironmental2022}. In order to account for the variety of challenges and to maximize the benefits, staged frameworks have been developed to guide the design \autocite{citizenscience.govBasicStepsYour, fraislCitizenScienceEnvironmental2022,garciaFindingWhatYou2021,minkmanCitizenScienceWater2015}. Yet, these frameworks can be relatively coarse and imprecise and are often partly tailored to specific goals and contexts, making a combination of several such frameworks and the inclusion of further guidelines and recommendations potentially necessary to tailor the design to the specific situation. 

%%%%%%%%%%%%%%%%%%%%%%%%%%%%%%%
% participants
Participants can take many roles in a \acrshort*{cbm} project based on the level of participation chosen but regardless of this, their adequate integration is seen as a cornerstone of any \acrshort*{cbm} project \autocite{land-zandstraParticipantsCitizenScience2021}. Knowledge and skills as well as other socio-economic variables can vary widely between participants and it is important to account for this to inspire and keep participants motivated to contribute \autocite{minkmanCitizenScienceWater2015,whitelawEstablishingCanadianCommunity2003}. One mayor drawback of online collaborative initiatives is often that a considerable proportion of contributors only participate once and with minimal effort while a relatively small number of participants are responsible for the majority of the work \autocite{sauermannCrowdScienceUser2015}. Understanding and thus sustaining the motivation of participants is therefore central to a successful project. The subject of what drives individuals to participate in citizen science projects has been extensively explored in literature \autocite{land-zandstraParticipantsCitizenScience2021,minkmanCitizenScienceWater2015,mloza-bandaCrowdsensingSuccessfulWater2018,ruttenHowGetKeep2017,tipaldoCitizenScienceCommunitybased2017,walkerBenefitsNegativeImpacts2021a,walkerBenefitsNegativeImpacts2021}. Motivation can be intrinsic or extrinsic and spans from the will to contribute to science and conservation over meeting and helping other potentially like minded people to learning new skills and financial compensation \autocite{minkmanCitizenScienceWater2015,rotmanDynamicChangesMotivation2012 ruttenHowGetKeep2017}. According to \autocite{rotmanDynamicChangesMotivation2012}  study, egocentric motives tended to drive new participants, whereas established participants were more motivated by altruistic reasons, such as helping others. Furthermore, the individual adaptation of the task's difficulty to each participant was suggested to positively influence motivation in order to neither bore nor overwhelm \autocite{minkmanCitizenScienceWater2015}. Other factors to inspire and sustain motivations are, among others, the expected benefits, acknowledgement and feedback culture and its perceived usefulness and integration into further processes \autocite{land-zandstraParticipantsCitizenScience2021,minkmanCitizenScienceWater2015,pettiboneCitizenScienceAll2016}. In addition to strengthening motivation, breaking down barriers to participation can also prove helpful. For this, understanding the background and circumstances of the participants is important. In their work for hydrological monitoring in Kenya, \autocite{weeserCitizenSciencePioneers2018a} could partly attribute low participation rates to the transmitting costs of 0.01 USD per text message at some station. Offsetting these costs could subsequently increase the overall participation rate significantly. \autocite{weeserCitizenSciencePioneers2018a} further discovered, that actual compensation or incentives appeared unnecessary as the intrinsic motivation of the participants proved to be adequate once financial constraints were addressed. Besides financial and resource restrictions, lack of knowledge and skills can be addressed by providing adequate training \autocite{fraislCitizenScienceEnvironmental2022,lackstromBackyardHydroclimatologyCitizen2022}.

% data
Supervision, external or mutual feedback and preceding training of participants can also address common data quality concerns \autocite{albusAccuracyLongtermVolunteer2020,baalbakiCitizenScienceLebanon2019,fraislCitizenScienceEnvironmental2022}. Besides the characteristics of the participant, the difficulty of the measurement task itself influences the quality. Simpler tasks such as gauging e.g. water levels provided high data quality in \autocite{weeserCitizenSciencePioneers2018a} study. \autocite{baalbakiCitizenScienceLebanon2019} has further found that most of the data collected by citizen scientists is comparable to that of university scientists when it comes to chemical or physical qualities of water. \autocite{albusAccuracyLongtermVolunteer2020} could support this finding, by analyzing data from the Texas Stream Team (TST) citizen science program and found an agreement of 80\% up to 90\% for DO, pH and conductivity parameters. However, \autocite{baalbakiCitizenScienceLebanon2019} also noted a disparity in the bacteriological test results between citizen and university scientists, to which they remarked, that it may be explained by the complexity of the testing process and the quality of the testing materials employed. \autocite{aceves-buenoCitizenScienceApproach2015} evaluated over 80 peer-reviewed studies of which only 11\% reported no data accuracy issues but only one study reported, that the data was unusable. Based on the aforementioned findings, ensuring data quality and accuracy through appropriate quality assurance and control measures is crucial. However, despite the reliability and accuracy challenges associated with \acrshort*{cbm} data, \autocite{aceves-buenoCitizenScienceApproach2015} noted, that these issues typically do not have a significant impact on the data's overall usefulness.

Besides the more specific challenges and benefits mentioned above, \acrlong*{cbm} approaches can benefit scientists, decision-makers, communities and participants in multiple ways. In addition to achieving the main objectives, raising awareness of the issue, the needs and the problems at hand, as well as increasing knowledge among all project stakeholders, can lead to changes in behaviour, improved management, reduced risks and a better representation of local conditions in the regional, national and international context. \autocite{huangManagementDrinkingWater2020,walkerBenefitsNegativeImpacts2021}. Output quality can be enhanced when the objective is clear, participant involvement is recognized as a high priority, enough resources for design, implementation, operation and analysis are available and the monitoring protocol is not too complex \autocite{butteFrameworkWaterSecurity2022, pocockStrategicFrameworkSupport}. 
In an attempt to scale this concept across regions or even an entire country with many physical, social and economic differences, the \acrshort*{cbm} concept has been increasingly explored with mobile, network-enabled devices. This is, together with practical examples and projects, presented in the coming chapters.


\subsection{VGI} % keep it short or completely out of this -> will come up in CS anyways

“In the field of geography, the mapping of features such as buildings, road networks, and land cover can now be undertaken by citizens as a result of advances in Web 2.0 and global positioning system (GPS)-enabled mobile technology, which has blurred the once clear-cut distinction between map producer and consumer (Coleman et al., 2009).” ([Zheng et al., 2018, p. 703](zotero://select/groups/4773535/items/LJU68CG4)) ([pdf](zotero://open-pdf/groups/4773535/items/U8QNZLI6?page=6&annotation=W66QRY3C))

“In a seminal paper published in 2007, Goodchild (2007) coined the phrase Volunteered Geographic Information (VGI). Similar to the idea of crowdsourcing, VGI refers to the idea of citizens as sensors, collecting vast amounts of georeferenced data.” ([Zheng et al., 2018, p. 703](zotero://select/groups/4773535/items/LJU68CG4)) ([pdf](zotero://open-pdf/groups/4773535/items/U8QNZLI6?page=6&annotation=73UDWLM6))

“OpenStreetMap (OSM) is an example of a highly successful VGI application (Neis & Zielstra, 2014),” ([Zheng et al., 2018, p. 703](zotero://select/groups/4773535/items/LJU68CG4)) ([pdf](zotero://open-pdf/groups/4773535/items/U8QNZLI6?page=6&annotation=LJBW7EY4))

%-----------------------------------
%	SUBSECTION 6.2
%-----------------------------------

\subsection{local knowledge (???)} %only when I still have time there are just tons of information maaaan
These \textit{unconventional triggers} could be based on local or indigeneous knowledge and data from the ground.

“External stakeholders’ attitudes towards and engagement with local knowledge in 1 disaster risk reduction: are we only paying lip service?” ([Šakić Trogrlić et al., 2021, p. 1](zotero://select/groups/4773535/items/BTRX6EIG)) ([pdf](zotero://open-pdf/groups/4773535/items/52GEZZFV?page=1&annotation=TLIV6XS7)). 

% --> local context
“Deep understanding of the local context, and the needs and wants of the targeted community would allow us to identify which drought impacts are most strongly felt by different groups of the community.” ([RCRC, 2020, p. 28](zotero://select/groups/4773535/items/UESIQTRJ)) ([pdf](zotero://open-pdf/groups/4773535/items/P5JPVZ97?page=28&annotation=EI2UMB2H))

-->
“Where they exist, these systems may be even more important for slow-onset hazards like drought, necessary but not sufficient - layers of additional (ideally) local indicators must be added to these in order to form an appropriate FbA trigger.” ([RCRC, 2020, p. 29](zotero://select/groups/4773535/items/UESIQTRJ)) ([pdf](zotero://open-pdf/groups/4773535/items/P5JPVZ97?page=29&annotation=YK99RIZB))


local on the ground impact assessment not possible with current forecast abilities --> possibly link to that via CBS and the feasibility study --> (“Early Warning/Early Action Mechanisms: EWEA is working well in cases of health emergencies/epidemics through community-based surveillance (CBS); this allows the N” ([Somali Red Crescent Society, 2022, p. 51](zotero://select/groups/4773535/items/FZ6BJHJA)) ([pdf](zotero://open-pdf/groups/4773535/items/RJKNZZZ2?page=51&annotation=4C3HL8ES))) 
thus, this and comparable approaches are investigated in the next chapter

% nonetheless, difficult to scale
“The Problem of Scale in Indigenous Knowledge: a Perspective from Northern Australia” ([Wohling, 2009, p. 1](zotero://select/groups/4773535/items/HIFJDYSG)) ([pdf](zotero://open-pdf/groups/4773535/items/BUPU6DGS?page=1&annotation=LPCCJN8Z))

different trigger --> different set of actions --> interdependent of what the actions should look like (can be an interative forth and back coming process)--> can also differ for different groups

% local knowledge --> same as with indicators, forecasting is also possible on the basis of local knowledge -> overview table
“Table 1. Comparisons between indigenous knowledge-based seasonal forecasts and seasonal climate forecasts (adopted from Ziervogel and Opere 2010). Indigenous knowledge-based seasonal forecasts Seasonal climate forecasts Use biophysical indicators of the environment as well as spiritual methods Use of weather and climate models of measurable meteorological data Forecast methods are seldom documented Forecast methods are more developed and documented Up-scaling and down-scaling are usually complex Up-scaling and down-scaling are relatively simple Application of forecast output is less developed Application of forecast output is more developed Communication is usually oral Communication is usually written Explanation is based on spiritual and social values Explanation is theoretical Taught by observation and experience Taught through lectures and readings” ([Masinde and Bagula, 2012, p. 280](zotero://select/groups/4773535/items/EW9XSSZP)) ([pdf](zotero://open-pdf/groups/4773535/items/3WQ4S9PE?page=7&annotation=6XCISBM2))

“Indigenous knowledge within an early warning system for droughts” ([Masinde and Bagula, 2012, p. 282](zotero://select/groups/4773535/items/EW9XSSZP)) ([pdf](zotero://open-pdf/groups/4773535/items/3WQ4S9PE?page=9&annotation=8Z9A9AW8))

“The Best of Both Worlds: A Decision-Making Framework for Combining Traditional and Contemporary Forecast Systems” ([Plotz et al., 2017, p. 1](zotero://select/groups/4773535/items/3SBLBZEA)) ([pdf](zotero://open-pdf/groups/4773535/items/VAUJGIFB?page=1&annotation=RRFY2UWE))

“B. Drought Forecasting in Sub-Saharan Africa” ([Masinde and Thothela, 2019, p. 304](zotero://select/groups/4773535/items/6D52T883)) ([pdf](zotero://open-pdf/groups/4773535/items/KLLQKDG2?page=2&annotation=ZQSDUEMX))

there are more! Look into it.!

“Researchers ([1], [19] and [20]) today concur that IK and modern science weather forecasts complement each other;” ([Masinde et al., 2013, p. 2](zotero://select/groups/4773535/items/M45MLGWC)) ([pdf](zotero://open-pdf/groups/4773535/items/LG6E76P4?page=2&annotation=JHQV2GYT))


“In theory, focusing on what the weather will do, rather than what the weather will be, enables decision makers to plan and implement targeted preparatory actions to better reduce hazard impacts (Harrowsmith et al., 2020).” ([Boult et al., 2022, p. 2](zotero://select/groups/4773535/items/B2AQSTYL)) ([pdf](zotero://open-pdf/groups/4773535/items/W9TFLH43?page=2&annotation=NSLE7NL6))

“Improving early warning of drought-driven food insecurity in southern Africa using operational hydrological monitoring and forecasting products” ([Shukla et al., 2020, p. 1187](zotero://select/groups/4773535/items/TE5NMA3T)) ([pdf](zotero://open-pdf/groups/4773535/items/9TNUGXSJ?page=1&annotation=TLBHC7BS))

“Moving from drought hazard to impact forecasts” ([Sutanto et al., 2019, p. 1](zotero://select/groups/4773535/items/EUC5RV7N)) ([pdf](zotero://open-pdf/groups/4773535/items/9DI9EVBF?page=1&annotation=FCQCKY5P))


%--> even though it is generally not recommend by the RCRC for a National Society to collect these local indicators by themselves

“It is important to note that local indicators cannot be collected specifically for the FbF system by RCRC national societies.” ([RCRC, 2020, p. 30](zotero://select/groups/4773535/items/UESIQTRJ)) ([pdf](zotero://open-pdf/groups/4773535/items/P5JPVZ97?page=30&annotation=LRDPV2M7))

“Indeed, collecting data on local indicators would require from the national society a team of enumerators that work continually to collect and process that information in all places where the program could possibly trigger (e.g. collect food price information for every village market). This would have extensive cost implications and likely over-burden the national society staff and volunteers.” ([RCRC, 2020, p. 30](zotero://select/groups/4773535/items/UESIQTRJ)) ([pdf](zotero://open-pdf/groups/4773535/items/P5JPVZ97?page=30&annotation=2YIIK6ZY))

“As such, the inclusion of local indicators into an FbA trigger must involve assessing what indicators are relevant for the impacts the program is trying to anticipate and identify which of those indicators are already collected (e.g. the ministry of agriculture's food price bulletin) and are available at the time they would be needed to inform a possible trigger.” ([RCRC, 2020, p. 30](zotero://select/groups/4773535/items/UESIQTRJ)) ([pdf](zotero://open-pdf/groups/4773535/items/P5JPVZ97?page=30&annotation=7X3RFGVB))


“2 Local knowledge in drought monitoring: an introduction to the literature review” ([Giordano et al., 2013, p. 526](zotero://select/groups/4773535/items/B7LM5ZR4)) ([pdf](zotero://open-pdf/groups/4773535/items/7I66DBIK?page=4&annotation=Z33M5FLQ))

%----------------------------------------------------------------------------------------
%	SECTION 7 CBS & other tools + water related monitoring (excel)
%----------------------------------------------------------------------------------------


\subsection{Mobile Crowdsensing (MCS)}\label{subsec:mcs} % practical applications of MCS CBS & other tools + water related monitoring (excel)

% CBS
% water monitoring options -> can later be seen in table XY
% short introduction of other tools: Kobo, Ushahidi, NYSS, excel, etc. ..
Originating in 2006 from an article by \autocite{howeRiseCrowdsourcing} and Mark Robinson describing Crowdsourcing as a new internet based business model in the terms of "It's not outsourcing; it's crowdsourcing", by harnessing "the creative solutions of a distributed network of individuals through what amounts to an open call for proposals" \autocite[76]{brabhamCrowdsourcingModelProblem2008}. Nowadays crowdsourcing in scientific contexts is often applied as e.g. act of "collecting data without a direct integration into the scientific process" by a generally large audience \autocite[1591]{weeserCitizenSciencePioneers2018a}. Due to the merely perceiving and transferring and not further interpreting character, \textit{Crowdsourcing} is on the lowest level of participation levels. A more specific form of \textit{Crowdsourcing} is \textit{Crowdsensing} which refers to the process of measuring and collecting data by a large mass of contributors that involves using mobile devices and/or sensors to collect information about the environment. This is also known as \acrfull{mcs} \autocite{guoParticipatorySensingMobile2014, liuSurveyMobileCrowdsensing2018}.
\acrshort*{mcs} is part of a widespread transition in the way data is gathered and managed, with a shift away from conventional methods towards incorporating mobile devices, web platforms, and apps \autocite{capponiSurveyMobileCrowdsensing2019, sanllorentecapdevilaSuccessFactorsCitizen2020}. This transition is being driven by the development and proliferation of information technology infrastructure, which includes the collection, sharing, storage, cleaning and analysis of data \autocite{fraislCitizenScienceEnvironmental2022}. These components of the information technology infrastructure can be grouped into a four-layer architecture which is described in detail in the paper by \autocite{capponiSurveyMobileCrowdsensing2019}.
The first and top layer is the \textit{application layer} concerned about high-level user, task- and overall design and organizational aspects with some examples being user recruitment's, scheduling and contribution management. The \textit{data layer} as the second layer refers to storage, processing and analysis of the received data and is followed by the \textit{communication layer} which refers to methodological and technological aspects of the reporting characteristics. These include cellular, internet or other networks and their means of transmission. The bottom layer, the centrepiece of this architecture, is the \textit{sensing layer} which includes all tools, technologies and equipment involved in the data acquisition process \autocite{capponiSurveyMobileCrowdsensing2019}. Measurements can be of different types, intentional or unintentional, at the occurrence of an event or continuous, and are based on human observation, instrumental measurements or a combination of both \autocite{zhengCrowdsourcingMethodsData2018}. In this architecture hierarchy, data flows generally from the lowest to the highest layer \autocite{aceves-buenoCitizenScienceApproach2015,capponiSurveyMobileCrowdsensing2019,zhengCrowdsourcingMethodsData2018}.
Besides generally applicable challenges of \acrlong{cbm} such as data quality and participant motivation, main challenges of \acrshort*{mcs} are seen in the socio-technical, privacy and security realms referring to hard- and software availability, reliability and usability as well as balancing access rights, anonymisation and encoding with data trustworthiness \autocite{aceves-buenoCitizenScienceApproach2015,alfonsoMOBILEPHONEAPPLICATIONS2012,capponiSurveyMobileCrowdsensing2019,liuSurveyMobileCrowdsensing2018, minkmanCitizenScienceWater2015, noureenCrowdsensingSocioTechnicalChallenges2017a}. Nonetheless, \acrshort*{mcs} also provides many opportunities and solutions to designers, operators and participants alike. Among those are the relatively good and easy scalability and increase of monitoring network density, low barriers for participation and two-way communication options as well as high potential for automatization and interoperability with other applications and frameworks \autocite{alfonsoMOBILEPHONEAPPLICATIONS2012,minkmanCitizenScienceWater2015,sanllorentecapdevilaSuccessFactorsCitizen2020,weeserCitizenSciencePioneers2018a}. In the following, practical examples of \acrshort*{cbm} and \acrshort*{mcs} or a combination of both are presented, highlighting the wide-ranging application possibilities. %together with their advantages and disadvantages.

\subsection*{Practical Examples of CBS and MCS}

The potential applications for \acrshort*{mcs}, embedded in \acrshort*{cbm} or as a stand-alone project, are, as for all Citizen Science, wide-ranging and diverse. Besides the thematic diversity, the socio-technical implementation, size and complexity can differ substantially from project to project. Established networks like the \acrfull*{cocorahs} founded in 1998 USA with over 25.000 observers facilitate the collection of daily weather observations and the sharing of written impact impressions via an online platform \autocite{cocorahsCoCoRaHSCommunityCollaborative2023,lackstromBackyardHydroclimatologyCitizen2022}. The Audubon's Christmas Bird Count (CBC) even goes back to the December of 1900 and in its 120th anniversary year over 81.000 observers counted more than 30 million individual birds \autocite{lebaron122ndChristmasBird2022}. Another major project in the realm of \textit{Crowdsourcing} and \acrshort*{mcs} is the 2004 founded OpenStreetMap Foundation. Started as a reaction to the failed release of geographic information in the United Kingdom, OSM as a collaborative community effort quickly became one of the most important sources of geographic information world wide \autocite{bennettOpenStreetMap2010, openstreetmapcontributorsOpenStreetMapBasemap2020}. Additional contemporary developments include the concept of \acrshort*{mcs} in citizen participation, Smart Cities, resource management, transport and behaviour evaluation and many more \autocite{dipasDIPASOrgDIPAS2023,europeancommissionCitizencentredApproachSmart2021, wangSurveyApplicationKey2022}. 
Other projects with a thematic focus on health, water and early warning are considered in more detail in the remaining part of this section. Health, as \acrfull*{cbs} is successfully implemented as \acrshort*{cbm} with NYSS as \acrshort*{mcs} in Somalia (FIXME: see chapter \ref*{TODO:}), and water and early warning projects, as they are thematically related to this work. Projects concerning VGI will not be discussed in depth in this context, as mapping in this project will most definitely be carried out by professional and trained personnel. 


\subsubsection*{Community-based Surveillance}
%%%%%%%%%%%%%%%%%%%%%%%%%%%%%%%%%%%%%%%%%%%%%%%%%%%%%%%%%%%%%%%%%%%%%%%%%%%%%%%%%%%%%%%%%%%%%%%%%%%
%%%%%%%%%%%%%%%%%%%%%%%%%%%%%%%%%%%%%%%%%%%%%%%%%%%%%%%%%%%%%%%%%%%%%%%%%%%%%%%%%%%%%%%%%%%%%%%%%%%
%%%%%%%%%%%%%%%%%%%%%%%%%%%%%%%%%%%%%%%%%%%%%%%%%%%%%%%%%%%%%%%%%%%%%%%%%%%%%%%%%%%%%%%%%%%%%%%%%%%
%%%%%%%%%%%%%%%%%%%%%%%%%%%%%%%%%%%%%%%%%%%%%%%%%%%%%%%%%%%%%%%%%%%%%%%%%%%%%%%%%%%%%%%%%%%%%%%%%%%
% --> cbs -> already in use and working, really only about health
% another approach with a different focus --> people focussed and already relatively established in sub-saharan Africa
%%%%%%%%%%%%%%%%%%%% CBS %%%%%%%%%%%%%%%

Conventional surveillance systems for monitoring health of animals, humans and the environment rely on information of medical professionals, health facility records, and laboratory examinations to detect abnormalities that could signify potential outbreaks and newly emerging pathogens \autocite{mcneilLandscapeParticipatorySurveillance2022a}. However, these data are not sufficiently accessible in all regions of the world  to allow adequate responses \autocite{mcneilLandscapeParticipatorySurveillance2022a,nikolayEvaluatingHospitalBasedSurveillance2017}. The strong developments and increasing availability of mobile technologies, the recognition of the value of local knowledge in health management, and recently reinforced by the COVID 19 pandemic, have led to an an increasingly widespread use of \acrshort*{cbs} \autocite{kullenbergWhatCitizenScience2016,mcneilLandscapeParticipatorySurveillance2022a}. The \autocite{technicalcontributorstothejune2018whomeetingDefinitionCommunitybasedSurveillance2019} defined \acrshort*{cbs} as "the systematic detection and reporting of events of public health significance within a community by community members". With the growing importance of the \textit{One Health} approach, these "events of public health significance" span across the domains of human, animal and ecosystem health \autocite{cdcOneHealthBasics2022}.
\autocite{mcneilLandscapeParticipatorySurveillance2022a} identified 60 different ongoing surveillance systems across five continents. These systems were covering the three domains either stand-alone or in combination, on different spatial scales and with different technical characteristics. However, all projects have used some kind of digital technology, with websites and smartphones as the most common vehicles. Furthermore, a high percentage of the surveyed projects have noted the usefulness of the \acrshort*{cbs} approach as it "improved community knowledge and understanding" (78\%) and "earlier detection" (67\%). This finding is supported by various other studies \autocite{byrneCommunitycentredApproachGlobal2020,jarrettEvaluationPopulationMobility2020,mcgowanCommunitybasedSurveillanceInfectious2022,metugeHumanitarianLedCommunitybased2021,ratnayakeEarlyDetectionCholera2020,ratnayakePeoplecentredSurveillanceNarrative2020,technicalcontributorstothejune2018whomeetingDefinitionCommunitybasedSurveillance2019}.
The \acrshort*{cbs} approach has proven to be a more advantageous complement to the conventional system, especially if certain conditions are taken into account. \autocite{gueninParticipatoryEpidemiologicalOne2022} highlights the importance of congruent definitions and their adaptation to the different actors and roles as well as the adaptation of (two-way) communication channels. Preceding suitability assessments, simple design and reasonable incorporation of technology, effective community engagement, reliable and close surveillance through supervisors of local volunteers especially in the beginning as well as evaluation and feedback opportunities have been highlighted as key drivers for success. These drivers were grouped by \autocite{mcgowanCommunitybasedSurveillanceInfectious2022} in relation to (1) surveillance workers, (2) the community, (3) case detection and reporting, and (4) integration. Most of these factors and more have already been mentioned in the \acrshort*{cbm} context. They were linked to having a decisive influence on the quality of embeddedness in existing systems, acceptance, trust and ultimately its implementation in decision-making and response. In addition to these key success factors, main challenges remain in ethical and privacy considerations, availability of resources and fast response capacities in case of an event as well as community expectation management. Furthermore, \autocite{boetzelaerEvaluationCommunityBased2020} findings indicate, that the additional benefits of \acrshort*{cbs} in already stable settings are limited as the approach is resource intensive. Nevertheless, the increasing application of \acrshort*{cbs} in low-resource or conflict-affected areas, where the full range of benefits were brought to bear. These benefits include CBSs' early warning capabilities and showed promising capacities to address current gaps in health related information and response management, especially in regard to spatial coverage and lower response times \autocite{metugeHumanitarianLedCommunitybased2021, ratnayakePeoplecentredSurveillanceNarrative2020}. \autocite{metugeHumanitarianLedCommunitybased2021} has additionally been able to fruitfully adapt \acrshort*{cbs} for related issues such as displacement and malnutrition and the SRCS is currently using CBS together with the MCS platform NYSS from the Norwegian Red Cross (NRC) in Somalia. The CBS approach has thus shown that it can be potentially adapted to other issues and that it can be successfully implemented regionally. More on the regional implementation in chapter (TODO: \ref*{CASE STUDY CBS} and \ref*{RESULTS NYSS}).

\subsubsection*{Community-based Water Monitoring and Management}
%%%%%%%%%%%%%%%%%%%%%%%%%%%%%%%%%%%%%%%%%%%%%%%%%%%%%%%%%%%%%%%%%%%%%%%%%%%%%%%%%%%%%%%%%%%%%%%%%%%
%%%%%%%%%%%%%%%%%%%%%%%%%%%%%%%%%%%%%%%%%%%%%%%%%%%%%%%%%%%%%%%%%%%%%%%%%%%%%%%%%%%%%%%%%%%%%%%%%%%
%%%%%%%%%%%%%%%%%%%%%%%%%%%%%%%%%%%%%%%%%%%%%%%%%%%%%%%%%%%%%%%%%%%%%%%%%%%%%%%%%%%%%%%%%%%%%%%%%%%
%%%%%%%%%%%%%%%%%%%%%%%%%%%%%%%%%%%%%%%%%%%%%%%%%%%%%%%%%%%%%%%%%%%%%%%%%%%%%%%%%%%%%%%%%%%%%%%%%%%
% CBWM

\acrfull*{cwbm} is an application example of \acrshort*{cbm} which gained mayor public interest particularly in North America, Europe, Australia and Southeast Asia \autocite{kirschkeCitizenScienceProjects2022, koehlerCitizenParticipationCollaborative2008, livinglakescanadaElevatingCommunityBased2018}. \acrshort*{cwbm} practices range from small monitoring projects to integrated partnerships or councils for the management of watersheds \autocite{westonCommunityBasedWaterMonitoring2015}. Just as for CBM and CBS, participant engagement, data quality control and management, sustainable funding and embedding in existing structures are key to successful integration and implementation of such projects \autocite{allenCommunityBasedWaterMonitoring2018,livinglakescanadaCommunityBasedWaterMonitoring2018,westonCommunityBasedWaterMonitoring2015}.
An overview of primarily water and weather related citizen science projects can be seen in table \ref*{TODO:}. Striking is the already mentioned globally unequal distribution of the projects with a strong emphasis on North American Countries. Furthermore, their focus is mostly on river, lake, groundwater and precipitation levels or focusses on their respective water quality. The technical solutions are mostly not freely available and not open source (FIXME: true? -> and extend). 

\missingfigure{overview table about water related CS projects with CBS/MCS aspects}
% table? table could be worth it.
% project name; topic; region/location; organization; participation level; tools, website/paper

Further noticeable are the technical requirements, which almost always require some sort of smartphone, dedicated measurement equipment or internet access. Only Weeser et al.'s approach is based on simple text messages but were limited in content to a station ID and the indicated stream water level. Here, signs explaining the monitoring and transmission process with pictures and instructions in Swahili and English were placed next to a water level indicator, encouraging passers-by to contribute \autocite{weeserCitizenSciencePioneers2018a}. \autocite[1597]{weeserCitizenSciencePioneers2018a} noted, that method of "transmitting the observations using simple cell phones and text messages turned out to be stable and reliable without major technical problems" in the context of their work in low-income rural areas in Kenya. The problem of occasionally insufficient network coverage was overcome by participants waiting until they reached a network before transmitting, making network availability not a limiting factor in this study. \autocite{wilson-jonesUsingMobilePhones2012} established and evaluated an Android mobile based system to support rural water quality monitoring in South Africa by simplifying connection between managers and operators of municipal test facilities. While all municipalities expressed the system as beneficial exemplifying the usefulness of fast, easy and low resource-intensive communication possibilities in such a context, this project does not necessarily fall within the sphere of \acrshort*{cs}, as the target group here was professional staff. Drawing on their literature review of water quality studies under climate change, \autocite[147]{huangManagementDrinkingWater2020} recommend the application of a "hybrid modality in which community management is the mainstay with supplement from external support" also considering differences in local realities and stakeholder opinions and needs. 

One approach to embed \acrshort*{cwbm} into local traditional community water management practices is proposed by \autocite{dayCommunitybasedWaterResources2009}. \autocite{dayCommunitybasedWaterResources2009} argues, that overarching concepts like the \acrfull*{iwrm} remain to large and complex to be manageable and implementable on local levels and additionally often fail to adequately include local stakeholders. Building on the decentralized and locally better opperationalisable version of \acrshort*{iwrm} called 'light IWRM' \autocite{butterworthFindingPracticalApproaches2010,moriartyIntegratedWaterResources2004} and its practical component of Water safety plans (WSP) \autocite{bartramWaterSafetyPlan2009}, \autocite{dayCommunitybasedWaterResources2009} created a community-based water resource management framework (see figure \ref*{TODO:}). 

% framework for the first couple of side goals
\missingfigure{TODO: “Figure 2. Community-based water resource manageme” ([Day, 2009, p. 59](zotero://select/groups/4773535/items/YWSNQ8A2)) ([pdf](zotero://open-pdf/groups/4773535/items/ETPCI5RI?page=14&annotation=BAMSY255))}

This framework provides the basis for monitoring by encompassing the specifics of arid regions also with regard to possible drought phases, community needs, risks and water resource assets. Furthermore, the community is seen primarily as a partner rather than a beneficiary, also taking into account internal communal heterogeneity and inequalities making it a good conceptual basis for this works water source monitoring design approach.
Further work for guiding principles in the sphere of \acrshort*{cbwm} are numerous and interested readers are referred to \autocite{westonCommunityBasedWaterMonitoring2015}.



---

Lake water quality 
“The results of this review, however, highlighted no usable preexisting system for SIMILE project, since the considered applications are too tied to specific geographic locations or too broad presenting no real connection with the lake ecosystem.” ([Carrion et al., 2020, p. 246](zotero://select/groups/4773535/items/7855TDUA)) ([pdf](zotero://open-pdf/groups/4773535/items/XI5TRN34?page=2&annotation=4P6WS72K))

%could contribute to WPI -> backslash to that (?)
%possibly include a section for the Water Poverty Index and other monitoring programmes -> everything they measure and what they recommend.
“The Water Poverty Index: Development and application at the community scale” ([Sullivan et al., 2003, p. 189](zotero://select/groups/4773535/items/3I3AYIAT)) ([pdf](zotero://open-pdf/groups/4773535/items/MSRKDWQ4?page=1&annotation=7CGSBY3V))

intro with general stuff and overview then: water, CBS, early warning stuff (?) next one
% --> other tools and projects (group them --> look at the excel sheet) mapping is part of this, but not really VGi as those will be professionals

% Even though local circumstances still need to be addressed accordingly,
%next chapter: RISK and so on..


“strengths of using mobile phone sensing: • Highly mobile and scalable; • Low-cost; • Automatic time stamp and GPS possible; • Citizens could interfere when necessary.” ([Minkman, 2015, p. 182](zotero://select/groups/4773535/items/ZKLE6CPT)) ([pdf](zotero://open-pdf/groups/4773535/items/QMAPCSZG?page=182&annotation=9GR233CN))


\subsubsection*{Participatory Early Warning and Monitoring Systems} % ???????????? or different name? -> ? Community-Based Disaster Risk Management
%%%%%%%%%%%%%%%%%%%%%%%%%%%%%%%%%%%%%%%%%%%%%%%%%%%%%%%%%%%%%%%%%%%%%%%%%%%%%%%%%%%%%%%%%%%%%%%%%%%
%%%%%%%%%%%%%%%%%%%%%%%%%%%%%%%%%%%%%%%%%%%%%%%%%%%%%%%%%%%%%%%%%%%%%%%%%%%%%%%%%%%%%%%%%%%%%%%%%%%
%%%%%%%%%%%%%%%%%%%%%%%%%%%%%%%%%%%%%%%%%%%%%%%%%%%%%%%%%%%%%%%%%%%%%%%%%%%%%%%%%%%%%%%%%%%%%%%%%%%
%early warning


as could be seen in the CBS part --> early detection -> early warning --> early response --> no outbreak

this can be translated to numerous other cases such as ....

bottom-up and top-down processes together --> benefits but (?)

general platforms like Ushahidi and Sahana Eden + Kobo + ITIKI (local context --> case study) + BRCiS (?)

+ short general challenges/opportunities/future outlook ?

% done. Keep it short. Not as important. 

% Establishing an operational waterhole monitoring system using satellite data and hydrologic modelling: Application in the pastoral regions of East Africa
https://earlywarning.usgs.gov/docs/Senay-et-al-Pastoralism-Research-Policy-and-Practice-2013.pdf


“Community-based Water Management is conceptually linked and related to the concept of Community-Based Disaster Risk Management (CBDRM). CBDRM has been used interchangeably with related concepts such as community-driven disaster risk reduction (CBDRR),” ([Vhumbunu, 2021, p. 198](zotero://select/groups/4773535/items/69UZBCA7)) ([pdf](zotero://open-pdf/groups/4773535/items/D8BBZEYJ?page=6&annotation=K7HJJP7S))

“2. Key concepts of participatory early warning and monitoring systems (pEWMS)” ([“Participatory early warning and monitoring systems_ A Nordic framework for web-based flood risk management | Elsevier Enhanced Reader”, p. 1296](zotero://select/groups/4773535/items/JYH8N2BV)) ([pdf](zotero://open-pdf/groups/4773535/items/JITPV84L?page=2&annotation=B2ELHBPJ))

“Adding “bottom-up” approaches [36] to classical EWMS allows stakeholders with access to local knowledge of environments and local networks to play a stronger role in decisionmaking and risk management” ([“Participatory early warning and monitoring systems_ A Nordic framework for web-based flood risk management | Elsevier Enhanced Reader”, p. 1296](zotero://select/groups/4773535/items/JYH8N2BV)) ([pdf](zotero://open-pdf/groups/4773535/items/JITPV84L?page=2&annotation=AXZXAK9D))
“Participatory early warning and monitoring systems: A Nordic framework for web-based flood risk management” ([Henriksen et al., 2018, p. 1295](zotero://select/groups/4773535/items/KH93MATT)) ([pdf](zotero://open-pdf/groups/4773535/items/QCMYPDEV?page=1&annotation=9ACGDGN2))

% Platforms like Ushahidi and Sahana Eden 

% ITIKI
“Development Journal 283 Forecast Indicators Integrated Weather Forecasting Application Server Indigenous Knowledge Webserver Indigenous Knowledge Database Server Indigenous Knowledge Focus Group Village Internet Kiosk 2a 3 2b Query Optimizer and Data Mining Tools PC for Uploading Data if WAP not Active Intermediary Intermediary Travels to Digital Village Source 1 Source n-1” ([Masinde and Bagula, 2012, p. 283](zotero://select/groups/4773535/items/EW9XSSZP)) ([pdf](zotero://open-pdf/groups/4773535/items/3WQ4S9PE?page=10&annotation=AXQH5Q7Z))

“ITIKI: bridge between African indigenous knowledge and modern science of drought prediction” ([Masinde and Bagula, 2012, p. 274](zotero://select/groups/4773535/items/EW9XSSZP)) ([pdf](zotero://open-pdf/groups/4773535/items/3WQ4S9PE?page=1&annotation=5BKZELQW))

“Accurate, timely and relevant drought predication information enables a community to anticipate and prepare for droughts and hence minimize the negative impacts.” ([Masinde and Bagula, 2012, p. 274](zotero://select/groups/4773535/items/EW9XSSZP)) ([pdf](zotero://open-pdf/groups/4773535/items/3WQ4S9PE?page=1&annotation=W75TQDCM))

% drought early warning.. possibly write a short paragraph about that as well.. might be a nice addition to the monitoring approach.
“Local early warning systems for drought – Could they add value to nationally disseminated seasonal climate forecasts?” ([Andersson et al., 2020, p. 1](zotero://select/groups/4773535/items/KYRIJQQB)) ([pdf](zotero://open-pdf/groups/4773535/items/NPA46JZU?page=1&annotation=NTEYIPV6))

% further classified literature reviews in the realm of geophysics: Zheng et al. 2018 p.717 Table 2 (method classification) + 3 (management) + 4 (quality assurance) + 5 (processing) + 6 (data privacy)

%+ couple of examples and others (e.g. Kobo, Ushahidi, drought in the US (relatively dedicated though))

% ein paar herausgreifen
-> CBS as CBM with NYSS as MCS or Kobo or Paper “CBS reporting methods” ([Goodermote, 2020, p. 31](zotero://select/groups/4773535/items/A35DRWSR)) ([pdf](zotero://open-pdf/groups/4773535/items/F795EXE4?page=31&annotation=PVEEL4FW))
--> NYSS as well as paper -> NYSS can be understood as the MCS part of the CBS system

% BRCiS - does not really belong here.. more of a case study thing ya know
“Data inaccuracy, access limitations, a lack of thresholds and inadequate resources for data collection resulted in non-standardised information, preventing comparison across areas and weakening strategic decision-making” ([Gualazzini, 2021, p. 4](zotero://select/groups/4773535/items/BWDYDL8T)) ([pdf](zotero://open-pdf/groups/4773535/items/8U5XVU5K?page=4&annotation=L5SELDEJ))


%%%%%%%%%%%%%%%%%%%%%%%%%%%%%%%%%%%%%%%%%%%%%%%%%%%%%%%%%%%%%%%%%%%%%%%%%%%%%%%%%%%%%%%%%%%%%%%%%%%
%%%%%%%%%%%%%%%%%%%%%%%%%%%%%%%%%%%%%%%%%%%%%%%%%%%%%%%%%%%%%%%%%%%%%%%%%%%%%%%%%%%%%%%%%%%%%%%%%%%
%%%%%%%%%%%%%%%%%%%%%%%%%%%%%%%%%%%%%%%%%%%%%%%%%%%%%%%%%%%%%%%%%%%%%%%%%%%%%%%%%%%%%%%%%%%%%%%%%%%
% or conclusion?
https://www.kobotoolbox.org/

“A major weakness of the existing tools is the emphasis on macro/international level information.” ([Masinde and Bagula, 2010, p. 390](zotero://select/groups/4773535/items/JNC4ACZS)) ([pdf](zotero://open-pdf/groups/4773535/items/IWMKDQYV?page=1&annotation=7Z44Z9L7))

“When Are Mobile Phones Useful for Water Quality Data Collection? An Analysis of Data Flows and ICT Applications among Regulated Monitoring Institutions in Sub-Saharan Africa” ([Kumpel et al., 2015, p. 10846](zotero://select/groups/4773535/items/GPM4C7RJ)) ([pdf](zotero://open-pdf/groups/4773535/items/7VXVKEXK?page=1&annotation=M5J4FFSH))

“Remote monitoring of rural water systems: A pathway to improved performance and sustainability?” ([Thomson, 2021, p. 1](zotero://select/groups/4773535/items/UQLXVVYI)) ([pdf](zotero://open-pdf/groups/4773535/items/K9XBXPQD?page=1&annotation=B59B5U68))

“The tools also tend to ignore the at risk community who happen to be host to very crucial traditional knowledge on droughts” ([Masinde and Bagula, 2010, p. 390](zotero://select/groups/4773535/items/JNC4ACZS)) ([pdf](zotero://open-pdf/groups/4773535/items/IWMKDQYV?page=1&annotation=VUYYP2LA))

“The analysis also provides a set of recommendations for citizen science program design that addresses spatial and temporal scale, data quality, costs, and effective incentives to facilitate participation and integration of findings into adaptive management.” ([Aceves-Bueno et al., 2015, p. 493](zotero://select/groups/4773535/items/YK2MKLA9)) ([pdf](zotero://open-pdf/groups/4773535/items/WGGHNGZB?page=1&annotation=3NRZ7R8Y))

“Even when receiving timely information on the ground, Members often had to wait for validation from national sources such as the Food Security and Nutrition Analysis Unit (FSNAU) to make decisions, and the resources for anticipatory action were very limited.” ([Gualazzini, 2021, p. 4](zotero://select/groups/4773535/items/BWDYDL8T)) ([pdf](zotero://open-pdf/groups/4773535/items/8U5XVU5K?page=4&annotation=CZST8S2L))

%limitations: 
“Additionally, there may be limitations related to designing and implementing citizen science projects in remote and unsafe areas, where crime levels are high and political risks exist, or where mobile network coverage is poor, access to smartphones and electricity is low and illiteracy levels among participants are high. Co-design and community-based approaches can address such challenges and ensure a high level of participant engagement20” ([Fraisl et al., 2022, p. 14](zotero://select/groups/4773535/items/FBJD7SWS)) ([pdf](zotero://open-pdf/groups/4773535/items/7WBDKYDY?page=14&annotation=MZ95QT2P))

“Current weather forecasts are still alien to African farmers, most of whom live in rural areas and struggle with illiteracy and poor communications infrastructure” ([Masinde and Bagula, 2012, p. 274](zotero://select/groups/4773535/items/EW9XSSZP)) ([pdf](zotero://open-pdf/groups/4773535/items/3WQ4S9PE?page=1&annotation=HPZC9Z65))

%%%%%%%%%%%%% OUTLOOK / Further reading 

--> clear gap --> water related but not water source related stuff

More examples of \acrshort*{mcs} were compiled and analysed by fraisl and xy
% environmental monitoring --> Zheng 2018 figure 10 and chapter 4 -> quite extensive (weather, air quality, precipitation, geography (VGI), ecology, surface water,) + table 2 overview literature for these

% further tools and platforms: “Table 2 | Examples of existing citizen science data collection platform” ([Fraisl et al., 2022, p. 5](zotero://select/groups/4773535/items/FBJD7SWS)) ([pdf](zotero://open-pdf/groups/4773535/items/7WBDKYDY?page=5&annotation=UHP9I7LS))


% mate. You know what to do. Just read through all sources and get this scientific. You know by now. It's your problem now. Cheers. I'm outta here. LOL. ok. Weird.

%\acrshort*{mcs} has great potential to cover wide areas and even extend monitoring efforts to an entire country, but local conditions must be taken into account to contribute to a successful project outcome\autocite{huangManagementDrinkingWater2020}. 


%%%%%%%%%%%%%%%%%%%%%%%
% Learnings:
“necessary to adapt the case definitions” ([Guenin et al., 2022, p. 1](zotero://select/groups/4773535/items/S7J4M8RG)) ([pdf](zotero://open-pdf/groups/4773535/items/X8FVLD63?page=1&annotation=L8RUQZ79))

% Challenges:
“Challenges and risks have related to: securing free SMS channels for alerts, time required for contextualised design, ensuring response action, and managing expectations of the scope of CBS.” ([Byrne and Nichol, 2020, p. 71](zotero://select/groups/4773535/items/R9QNVPU8)) ([pdf](zotero://open-pdf/groups/4773535/items/VFH6JVJS?page=2&annotation=KKVDV6DE))



% yeeey. could get this done this week! Woop Woop!

%----------------------------------------------------------------------------------------
%	SECTION 8 Case Study Area (+ application of the rest)
%----------------------------------------------------------------------------------------

\section{Case Study Area (+ application of the rest)} % maybe put this in the beginning of this all?
% Somaliland geography
% hazards and risks: drought situation (past, present, future)
% Somaliland social and economic circumstances (past, present, future)
% Stakeholder
% resource restrictions

here all the above chapters come together and are applied on the local context.
it start with 

geography concerning water / security / scarcity / 

hazards -> droughts and floods -> forecasts (though more in the south -> north primarily flash floods, also interesting but this won't help) + conflict and health (+CBS and NYSS)

socio-cultural-political circumstances

stakeholders in the project (GRC, IFRC, SRCS, NRC)

and then (?) good question..

Current stuff about water sources (?) -> or in the results (?) but I would need to analyze that stuff thouuuuuugh. maybe if I have time in the end?


EAP


in the following
--> apply background information on the case study By
    - presenting prevailing circumstances
    - and how this work is embedded in those
    -> point out gaps
    -> and how to resolve those


% lots of information about the local stuff

https://heigit.atlassian.net/wiki/spaces/FIS/pages/1704186/Background+Information#BackgroundInformation-AAforSomalia

“Study Area” ([Abdulkadir, 2017, p. 104172233225941010000](zotero://select/groups/4773535/items/G2RYLAC4)) ([pdf](zotero://open-pdf/groups/4773535/items/5PPIPUKD?page=1&annotation=TX2QHMDT))

(establish your research territory: general information about the importance, background details to understand studies context)

Overview Confluence Background Overview:
https://heigit.atlassian.net/wiki/spaces/FIS/pages/1704186/Background+Information

https://drive.google.com/file/d/1McCpUWQPNlO-nDF0RjDW7nmrdM75ZJoP/view

“Country context” ([Somali Red Crescent Society, 2022, p. 9](zotero://select/groups/4773535/items/FZ6BJHJA)) ([pdf](zotero://open-pdf/groups/4773535/items/RJKNZZZ2?page=13&annotation=T5MK6KYP))
%-----------------------------------
%	SUBSECTION 8.1
%-----------------------------------

\subsection{Geography & Climate}

% Extreme climatic effects hamper livestock production in Somalia
https://link.springer.com/article/10.1007/s11356-021-18114-w

% lots of good stuff here
“Assessment of Drought Recurrence in Somaliland: Causes, Impacts and Mitigations” ([Abdulkadir, 2017, p. 104172233225941010000](zotero://select/groups/4773535/items/G2RYLAC4)) ([pdf](zotero://open-pdf/groups/4773535/items/5PPIPUKD?page=1&annotation=49M6L9RG))

“Climate” ([Somali Red Crescent Society, 2022, p. 11](zotero://select/groups/4773535/items/FZ6BJHJA)) ([pdf](zotero://open-pdf/groups/4773535/items/RJKNZZZ2?page=15&annotation=K362FLLA))
%-----------------------------------
%	SUBSECTION 8.2
%-----------------------------------

\subsection{Hazards and risks}

“DROUGHT, DISPLACEMENT AND LIVELIHOODS IN SOMALIA/SOMALILAND” ([Fanning, 2018, p. 1](zotero://select/groups/4773535/items/39RLPG5R)) ([pdf](zotero://open-pdf/groups/4773535/items/JCSJW3PB?page=1&annotation=JNQU35PW))

“Spotlight on Somalia: Can we learn from failure? In Somalia in 2011, a famine was declared that, along with the complexity of the conflict situation, was responsible for thousands of deaths. At its peak, almost 4 in every 10 children in Southern Somalia were acutely malnourished, and 4 million people were estimated to be without basic food. The horror of this tragedy has since haunted the international community, who received 11 months of early warnings before a famine was declared. Beginning with La Nina forecasts almost one year in advance, FEWS-NET and others provided briefing notes and warning information to humanitarian actors in the region. Several months later, these alerts explained that rainy seasons had already failed, and that major impacts were extremely likely (Hillbruner and Moloney 2012). There has been a great deal of analysis of this event, in which several conclusions have come to light. One is that funding needs to be more readily available based on forecasted information (Lautze et al. 2012). The below graph (Hillbruner and Moloney 2012) demonstrates how large-scale funding was mobilized in the aftermath of the famine declaration, and was ultimately available after the most vulnerable had died. Secondly, the humanitarian community needs to clearly take responsibility for acting in advance of a disaster, even in complex cases like the Somali context. At the moment, such organizations are not held accountable for failure to act on early warning, as disaster response is considered business-asusual. Shouldering the responsibility to act in this critical moment between a warning and a disaster could avoid such impacts in the future (Lautze et al. 2012).” ([pdf](zotero://open-pdf/groups/4773535/items/WKFPUZCW?page=7&annotation=LIY2ABLI))

historical impact assessments:

https://heigit.atlassian.net/wiki/spaces/FIS/pages/1704784/Historical+Impact+Analysis
https://reliefweb.int/disaster/dr-2015-000134-som


https://news.un.org/en/story/2022/08/1125552

“The number of affected people will be 1,200,420 persons across all the six main regions in Somaliland. The top priority needs of the people affected to date are mainly water (70{\%), Food (21\%) and Health (9\%).” ([National Drought Committee, 2022, p. 3](zotero://select/groups/4773535/items/7XJRE6LM)) ([pdf](zotero://open-pdf/groups/4773535/items/2F59E4UZ?page=3&annotation=8JZVBSM6))

%% Risk analysis

"Prioritised hazard and its historical impact.
Drought is currently the most relevant hazard in Somaliland . Droughts have occurred frequently for the last decades and presented varied impacts (The Somaliland drought rapid assessment report commissioned by the National Drought Committee in January 2022).  The occurrence of droughts has increased over the last years from one in five years to one drought every two years. The majority of the population depends on (agro) pastoral agriculture. Since this type of livelihood depends fundamentally on the natural resources of pasture and water, the local population is highly vulnerable to droughts. 
According to the International Database for Disasters ( EM DAT) recent droughts in Somaliland occurred in 2004, 2010 to 2011, 2014 to 2017, 2019, and 2021 to 2023. The drought incidents have mainly caused food shortages due to below-average crop production, livestock loss and water shortages. The impacts are also in line with the historical drought impacts of the communities in the Eastern regions of Somaliland, i.e. Sool, Sanaag and Togdheer where communities were consulted in focus group discussions as part of the ongoing FbF project.  Since 2018 there has been a drought in 2019 and the current drought (2020-2023). This is also in line with the historical Combined Drought Index (CDI) data (FAO SWALIM). The CDI is a proven and reliable indicator of drought since it incorporates precipitation, soil moisture and vegetation health. While the country has experienced periodic severe drought conditions over the CDI assessment period from 2002 to 2022, the eastern regions have experienced the most extreme drought conditions, especially in the last two years. Droughts in Somaliland lead to a variety of negative impacts. Chief among them is increased food insecurity. EM DAT  lists food shortage and famine as the principal impact of all droughts since 1987. Furthermore, based on the Food Insecurity Integrated Phase Classification (IPC) data, every drought is associated with substantial spikes in food-insecure populations. For example, the current drought led up to December 2022 to 5.6 million people in or above IPC 3 (crisis). Since 2009, Somaliland has experienced a IPC average between class 2 and 3 (between stressed and crisis), whereby the very western regions Awdal and the eastern regions (Sanaag, Sool and Togdheer) have been most affected reaching IPC classification 3 or worse in almost a quarter of assessment periods (every 2 months). 
The exact processes which lead to food insecurity due to drought are diverse. In the case of pastoral communities, failed rainy seasons result in livestock and herders having to track longer distances to water sources. These circumstances and lower forage and water availability lead to poor livestock health, increasing livestock deaths and lower birth rates, which can amount to the loss of whole herds. Milk production decreases drastically, which is the reason for the lower diet variability of herders. Furthermore, herders have to sell their animals for lower prices due to their lower body conditions and increased supply. This was the case during the 2017 drought. In some cases, whole herds were lost. The price of milk products increased significantly. At the peak of the drought in 2017, 388.000 children suffered from malnutrition and 895.000 people were internally displaced. In total 6.2 million people have been in need of humanitarian assistance.
Agropastoralist communities are affected by droughts in the same way as pastoralist communities. In addition, their livelihood suffers from increased crop failure due to poor soil moisture. This, again, causes agropastoralists to lose their income. Additionally, the local staple food prices rise, putting pressure on people who depend on maize or sorghum as the primary carbohydrate source. 

Similar to higher food prices, drought causes higher water prices as well. Besides fewer available resources to buy quantitative and qualitative adequate food, high water prices lead to poor hygiene practices. In 2017 water shortages due to two consecutive failed rainy seasons led to rising water prices. In the Eastern regions, the cost of 200 litres of water jumped from 2 USD to 7 USD. Similar water shortages during the 2016 drought caused a sharp increase in watery diarrhoea cases to 13.653 cases with a fatality rate of 3.6 %. 
Further consequences of droughts of the last 12 years in Somaliland can be seen in the table below."
https://docs.google.com/document/d/1xUEXm8RxVHTO468KqXSAoBX-cpkPwiff/edit

“SOMALILAND DROUGHT RAPID ASSESSMENT REPORT Commissioned By National Drought Committee” ([National Drought Committee, 2022, p. 1](zotero://select/groups/4773535/items/7XJRE6LM)) ([pdf](zotero://open-pdf/groups/4773535/items/2F59E4UZ?page=1&annotation=UTRIHJIN))

% Long-Term Trend of Climate Change and Drought Assessment in the Horn of Africa
https://www.hindawi.com/journals/amete/2016/8057641/

%% + table of the last mayor droughts


https://www.climatechangenews.com/2021/11/19/can-learn-somalia-anticipating-climate-disaster/

% could also be mentioned in the TB introduction
“1.2 DROUGHT IN SOMALIA” ([“Economics of Resilience to Drought: Somalia Analysis”, p. 9](zotero://select/groups/4773535/items/LGRWAU43)) ([pdf](zotero://open-pdf/groups/4773535/items/MBXSCVWR?page=9&annotation=K6W9FFM7))


“In the case of Somalia in 2011, prolonged discussions about the appropriate action to take culminated in a famine (Hillbruner and Moloney 2012), and review of the event concluded that “Famine early warning systems have a good track record of predicting food crises but a poor track record of triggering early action” (Bailey 2013).” ([pdf](zotero://open-pdf/groups/4773535/items/WKFPUZCW?page=27&annotation=4WUQ8F7U))

“Droughts and their impacts have been studied for many decades in the Horn of Africa.” ([Balint et al., 2013, p. 2](zotero://select/groups/4773535/items/C6T6K9AP)) ([pdf](zotero://open-pdf/groups/4773535/items/ZT9SQQ2M?page=2&annotation=XYFG3XJI))

“East Africa is characterized by two main rainy seasons per year: the so-called long rains from March-April to May-June, the so-called short rains from October to December 1.” ([pdf](zotero://open-pdf/groups/4773535/items/CADECE2Y?page=4&annotation=YKNUYE6T))

%-----------------------------------
%	SUBSECTION 8.3
%-----------------------------------

\subsection{Political, social and economic circumstances}
region is vulnerable to droughts due to its predominant reliance on pastoralism as its source of livelihood and economy. Livestock remains an essential source of employment, hard currency, export earnings and government revenue.
in case of failure of rainy season, people migrate from their locations, including pastoralists with their livestock, migrate to areas where better water and pasture are still available during the dry season
seasonal weather patterns in the region and the rest of the regions are characterized by four main seasons - two rainy seasons and two dry seasons. The main rains fall during the Gu-season (AprilJune), with lighter and more sporadic rains falling during the Deyr season (October December). Two dry seasons are characterized by one shorter, cooler season, known as the Hagaa (22nd July to 23rd September), and a long, hot, dry season, known as the Jilaal (January to 23rd March), which is the harshest season of the year


Point water sources are the primary source of water
Traditionally, natural disasters like droughts do not prevail uniformly across the entire area of Somalia but only in some hotspots.
significant increase in GBV and child abuse/abandonment during crises such as drought, flooding, conflict, etc. early action includes deploying extra social workers, training community champions, etc. UNFPA and UNICEF are currently undertaking a national assessment of these issues, and the report will likely be a useful source for identifying approaches for undertaking early action as part of the drought FbF
Hydrogeological survey and assessment of selected areas in Somaliland and Puntland (2012)
“Hydrogeological Survey and Assessment of Selected Areas in Somaliland and Puntland” ([pdf](zotero://open-pdf/groups/4773535/items/KU98BB4Z?page=1&annotation=JA8SYGKT))


Pastoralism: 
"Is estimated that half of the population in Somalia depends on pastoralist and agro-pastoralist livelihoods –Around 6 million people. This population contributes to the livestock trading economy that generates around 500 million USD per year. This accounts for the 79 percent of the total export earnings in the country, and is the largest source of foreign exchange after remittances (World Bank, 2019).In Ethiopia the population of pastoralist is estimated between 12 and 15 million people (UNICEF, 2019)."
%-----------------------------------
%	SUBSECTION 8.4
%-----------------------------------

\subsection{Stakeholder}

“Stakeholders interviewed for this project reinforced this and emphasized the importance of collaboration and cohesion between different methods and approaches, given the multi-faceted nature of the hazard.” ([RCRC, 2020, p. 23](zotero://select/groups/4773535/items/UESIQTRJ)) ([pdf](zotero://open-pdf/groups/4773535/items/P5JPVZ97?page=23&annotation=EQV29795))

https://heigit.atlassian.net/wiki/spaces/FIS/pages/1704213/Stakeholders+Systems+Networks

% also in subsection 7.2
“Adding “bottom-up” approaches [36] to classical EWMS allows stakeholders with access to local knowledge of environments and local networks to play a stronger role in decisionmaking and risk management” ([“Participatory early warning and monitoring systems_ A Nordic framework for web-based flood risk management | Elsevier Enhanced Reader”, p. 1296](zotero://select/groups/4773535/items/JYH8N2BV)) ([pdf](zotero://open-pdf/groups/4773535/items/JITPV84L?page=2&annotation=AXZXAK9D))

“With the increased pressure on water resources and the challenges faced with the implementation of the existing regulatory framework, a growing lack of mutual trust between water stakeholders has been observed in recent years [38].” ([“Participatory early warning and monitoring systems_ A Nordic framework for web-based flood risk management | Elsevier Enhanced Reader”, p. 1296](zotero://select/groups/4773535/items/JYH8N2BV)) ([pdf](zotero://open-pdf/groups/4773535/items/JITPV84L?page=2&annotation=HNTIH8JB))

Selection of stakeholders:
“both information users—i.e., local decision makers—and information producers—i.e., members of the local community that can provide knowledge and information about the drought’s impacts at a local level—should be involved in the design process.” ([Giordano et al., 2013, p. 529](zotero://select/groups/4773535/items/B7LM5ZR4)) ([pdf](zotero://open-pdf/groups/4773535/items/7I66DBIK?page=7\&annotation=MNSWFWEE))

difference scientific vs. stakeholders perception of drought impact
“The comparison between stakeholders’ perception of drought impacts and scientific knowledge allowed us to draw some preliminary conclusions concerning both the drought impacts at local level and the coherence between local and scientific knowledge on drought impacts.” ([Giordano et al., 2013, p. 539](zotero://select/groups/4773535/items/B7LM5ZR4)) ([pdf](zotero://open-pdf/groups/4773535/items/7I66DBIK?page=17\&annotation=QFD3UE4C))
“the correlation degree between perception indicators and scientific indicators is high only when considering a direct impact of drought, for example the reduction of productivity for non-irrigated crops.” ([Giordano et al., 2013, p. 540](zotero://select/groups/4773535/items/B7LM5ZR4)) ([pdf](zotero://open-pdf/groups/4773535/items/7I66DBIK?page=18&annotation=6KRIS4DA))
“Firstly, stakeholders tend to oversimplify the cause-effect chain” ([Giordano et al., 2013, p. 540](zotero://select/groups/4773535/items/B7LM5ZR4)) ([pdf](zotero://open-pdf/groups/4773535/items/7I66DBIK?page=18&annotation=IBV34JQL))
“Stakeholders seem to focus exclusively on the portion of the system they perceived.” ([Giordano et al., 2013, p. 541](zotero://select/groups/4773535/items/B7LM5ZR4)) ([pdf](zotero://open-pdf/groups/4773535/items/7I66DBIK?page=19&annotation=73P565M3))
“On the one hand, this allows them to have a clear picture of the evolutionary trends of the different variables that make up that portion of the system” ([Giordano et al., 2013, p. 541](zotero://select/groups/4773535/items/B7LM5ZR4)) ([pdf](zotero://open-pdf/groups/4773535/items/7I66DBIK?page=19&annotation=ZJSACLXJ))
“On the other hand, due to the limitations of their viewpoint, stakeholders tend to neglect the existence of multiple causes for certain observed phenomena.” ([Giordano et al., 2013, p. 541](zotero://select/groups/4773535/items/B7LM5ZR4)) ([pdf](zotero://open-pdf/groups/4773535/items/7I66DBIK?page=19&annotation=8CUNSCXS))
“Secondly, the data collected shows that the stakeholders tend to aggregate the variables describing the drought impacts, for example the impacts of crop productivity vary dramatically according to the crop under consideration, whereas stakeholders perceived a general reduction in productivity.” ([Giordano et al., 2013, p. 541](zotero://select/groups/4773535/items/B7LM5ZR4)) ([pdf](zotero://open-pdf/groups/4773535/items/7I66DBIK?page=19&annotation=TT7JRW3U))


“This work also demonstrates that there is a lack of synchronism between the stakeholders’ perception and the drought onset.” ([Giordano et al., 2013, p. 541](zotero://select/groups/4773535/items/B7LM5ZR4)) ([pdf](zotero://open-pdf/groups/4773535/items/7I66DBIK?page=19&annotation=H3XRYH3Q))

“In fact, stakeholders become aware of the drought only when they recognize the impacts on their perceived environment.” ([Giordano et al., 2013, p. 541](zotero://select/groups/4773535/items/B7LM5ZR4)) ([pdf](zotero://open-pdf/groups/4773535/items/7I66DBIK?page=19&annotation=869MME8Q))
“Often, those impacts are recognizable only when the drought is at its peak” ([Giordano et al., 2013, p. 541](zotero://select/groups/4773535/items/B7LM5ZR4)) ([pdf](zotero://open-pdf/groups/4773535/items/7I66DBIK?page=19&annotation=XTTUYS8M))

“Therefore, stakeholders’ indicators cannot be used to detect the drought onset or to assess the duration of the phenomenon” ([Giordano et al., 2013, p. 541](zotero://select/groups/4773535/items/B7LM5ZR4)) ([pdf](zotero://open-pdf/groups/4773535/items/7I66DBIK?page=19&annotation=C2QMHW76))

“They are more suitable to support the evaluation of the severity of the drought, and particularly of the indirect impacts.” ([Giordano et al., 2013, p. 541](zotero://select/groups/4773535/items/B7LM5ZR4)) ([pdf](zotero://open-pdf/groups/4773535/items/7I66DBIK?page=19&annotation=K8Z77CBR))

“The results of our experience show that drought perception indicators could be useful for defining thresholds of severity rather than for evaluating drought evolutionary trends. These thresholds could support decision makers in the development of effective drought risk management strategies.” ([Giordano et al., 2013, p. 542](zotero://select/groups/4773535/items/B7LM5ZR4)) ([pdf](zotero://open-pdf/groups/4773535/items/7I66DBIK?page=20&annotation=X47HGTNQ))

%-----------------------------------
%	SUBSECTION 8.5
%-----------------------------------

\subsection{Water sources}

% I guess better leave this out and put it in the discussion section (?) 
“Using Remote Sensing to Map and Monitor Water Resources in Arid and Semiarid Regions” ([Klemas and Pieterse, 2015, p. 33](zotero://select/groups/4773535/items/BVN6IXG5)) ([pdf](zotero://open-pdf/groups/4773535/items/UPSYZXDK?page=1&annotation=4DPZD4BZ))



"In all, mapping presents many benefits, such as:

It makes easier to integrate data from different sources (surveys, censuses, satellites, etc.) and from different disciplines (social, economic, and environmental data). It also allows the switch to new units of analysis from, for example, administrative boundaries (e.g. state) to ecological boundaries (e.g. basin).
Maps are a powerful visual tool and are more easily understood by stakeholders, particularly in developing countries.
The spatial nature of water poverty, such as the distance to the nearest water source or the water supply infrastructure, can also be incorporated easily in a GIS database.
The allocation of resources can be improved, since geographic targeting is more efficient and cost-effective than to launch an equally expensive universal distribution programme.
Geo-referenced databases can be enriched by additional data as they become available; and new attributes, such as better details on water quality, can be incorporated into the data structure, ensuring that the relevance of the data is sustained over time.
Maps can be produced at a number of different resolutions depending on their purpose and the cost of data collection. A coarse resolution or a scale too small neglects the heterogeneity within each unit and provides insufficient detail for decision making, while a fine resolution or a scale too large increases the cost of compiling, managing, and analyzing the data."https://en.wikipedia.org/wiki/Water_point_mapping

Berkads:
“Changing Pastoralism in the Ethiopian Somali National Regional State (Region 5)” ([“Changing Pastoralism in Region 5”, p. 1](zotero://select/groups/4773535/items/FXJGUTLD)) ([pdf](zotero://open-pdf/groups/4773535/items/BIAA5M57?page=1&annotation=5F9EZJYZ))

"Although birkeds cannot be considered permanent water points in the sense of permanent wells which do not rely on harvesting rainwater, clusters of birkeds represent dry season water points that they provide water throughout the dry season in most years. Today, then, distribution of water points is vastly different from a few decades ago. Map 5 shows the water points that exist today in the five districts under study. The map attempts to show the wells and boreholes as well as the main clusters of birkeds. The latter are difficult to map as there is no existing record of all locations. The map is based on sketch maps drawn by communities during the fieldwork. It is thus not meant to be accurate but to give an indication of the nature of change. It should be noted that these water points shown on the map are also the site of permanent settlements, as the tendency has been for settlements to grow up at the site of new water points."

“IMPROVED BERKAD DESIGNS BY MERCY CORPS - SOMALIA” ([pdf](zotero://open-pdf/groups/4773535/items/F3CWEKHP?page=1&annotation=KH5I94RJ))

image of a Berkad? that is a water source type typical for Somalia

"Criteria used included among others coverage, need for rehabilitation, seasonality of services, quality of water delivered, and poor management." https://en.wikipedia.org/wiki/Water_point_mapping
https://onlinelibrary.wiley.com/doi/abs/10.1111/j.1477-8947.2010.01296.x?casa_token=TPvw51virRQAAAAA:9K-6fcNcYFw9-Mny-EOIMdS6OmmSUkajSfo9qggsMNnAirGYSslUHckFyqWNP68XarnXYEgIh9eL0uGltw

"Water source:
o Out of 77 communities assessed, 49 have berkets, 26 have boreholes, 21 have shallow
wells, 28 communities reported that they receive water trucking.
o Berkads: 76\% of communities having berkets reported that all berkets are reported to be
depleted, while 22\% reported they were less half than full.
o Water trucking: Out of the 28 communities that receive water trucking, 43\% receive water
on a daily basis. 19 out of 28 communities (68\%) receiving water trucking responded that
they receive water from private suppliers. No communities, covered by this assessment,
mentioned water trucking from humanitarian partners.
o Reduction of water consumption: 54 out of 77 assessed communities (69\%) responded
that the majority of community members reduced water consumption in the past 4 weeks,
including all assessed communities in Sool." https://drive.google.com/file/d/1KWUZW0jEMV1Ijc4zeET_Yes93VJ_3qn2/view

“IMPROVED BERKAD DESIGNS BY MERCY CORPS - SOMALIA” ([pdf](zotero://open-pdf/groups/4773535/items/F3CWEKHP?page=1&annotation=KH5I94RJ))

“Water source:” ([National Drought Committee, 2022, p. 8](zotero://select/groups/4773535/items/7XJRE6LM)) ([pdf](zotero://open-pdf/groups/4773535/items/2F59E4UZ?page=8&annotation=CTKWTWRE))

“77 communities assessed,” ([National Drought Committee, 2022, p. 8](zotero://select/groups/4773535/items/7XJRE6LM)) ([pdf](zotero://open-pdf/groups/4773535/items/2F59E4UZ?page=8&annotation=6AN6T6SG))

“49 have berkets” ([National Drought Committee, 2022, p. 8](zotero://select/groups/4773535/items/7XJRE6LM)) ([pdf](zotero://open-pdf/groups/4773535/items/2F59E4UZ?page=8&annotation=7BWVAK62))

“28 communities reported that they receive water trucking.” ([National Drought Committee, 2022, p. 8](zotero://select/groups/4773535/items/7XJRE6LM)) ([pdf](zotero://open-pdf/groups/4773535/items/2F59E4UZ?page=8&annotation=VCLQJBGD))

“7000 berkads in the Haud” ([Birch, 2008, p. 4](zotero://select/groups/4773535/items/X92LGIEA)) ([pdf](zotero://open-pdf/groups/4773535/items/ZWD3FAK4?page=6&annotation=V2XYRC8R))

“The explosion of permanent water points means that the Haud is now grazed all year round, leaving no space for regeneration” ([Birch, 2008, p. 4](zotero://select/groups/4773535/items/X92LGIEA)) ([pdf](zotero://open-pdf/groups/4773535/items/ZWD3FAK4?page=6&annotation=FYWWUNR9))

rather than thinking water sources individually -> they can be thought of as a system when monitored more broadly

% possibly add the definitions of water sources here (?) improved, unimproved etc. https://www.wearewater.org/en/what-do-we-mean-when-we-talk-about-access-to-water_346091

“The benefits and negative impacts of citizen science applications to water as experienced by participants and communities” ([Walker et al., 2021, p. 1](zotero://select/groups/4773535/items/F9II464K)) ([pdf](zotero://open-pdf/groups/4773535/items/QVFF3PST?page=1&annotation=DFF8PCXN))

“Global Monitoring of Water Supply and Sanitation: History, Methods and Future Challenges” ([Bartram et al., 2014, p. 8137](zotero://select/groups/4773535/items/6AWUJTW5)) ([pdf](zotero://open-pdf/groups/4773535/items/BFNSQGWS?page=1&annotation=ZWSBJVDM))


% possibly a good start for the transition from impact indicators 

\subsection{“Real-Time Risk Monitoring (RTRM)” ([Gualazzini, 2021, p. 4](zotero://select/groups/4773535/items/BWDYDL8T)) ([pdf](zotero://open-pdf/groups/4773535/items/8U5XVU5K?page=4&annotation=HZFMU84X))} % not too sure if I wanna include this or where.. maybe only in the discussion part? -> in comparison to Crowdsensing/MCS this is slow and relatively coarse.



https://brcis.shinyapps.io/EWEA_dashboard/

“RTRM FPs collect shock data monthly according to a collectively agreed protocol” ([Gualazzini, 2021, p. 8](zotero://select/groups/4773535/items/BWDYDL8T)) ([pdf](zotero://open-pdf/groups/4773535/items/8U5XVU5K?page=8&annotation=HCYJPVMQ))

“select a small group of communities (two to three) that they deemed representative of the larger population (10–12 locations) in their operation area. Within this group, FPs are then required to contact two to four community actors and integrate the shock information into a single data submission per area.” ([Gualazzini, 2021, p. 8](zotero://select/groups/4773535/items/BWDYDL8T)) ([pdf](zotero://open-pdf/groups/4773535/items/8U5XVU5K?page=8&annotation=YLEYNUY3))

“In line with the above objectives, BRCiS designed the RTRM system to provide timely, community-generated shock information on key trends to trigger early actions in BRCiS operational areas.” ([Gualazzini, 2021, p. 5](zotero://select/groups/4773535/items/BWDYDL8T)) ([pdf](zotero://open-pdf/groups/4773535/items/8U5XVU5K?page=5&annotation=LDCLNI5G))

“For this purpose, consortium technical specialists used community feedback to determine the most common shocks, stresses and coping strategies.” ([Gualazzini, 2021, p. 5](zotero://select/groups/4773535/items/BWDYDL8T)) ([pdf](zotero://open-pdf/groups/4773535/items/8U5XVU5K?page=5&annotation=4QBNWWJI))

% also in the chapter about the trigger -> makes for a good comparison to Richards levels and threshold indicators
“Primary water source condition Observation of the status of the largest rainwater catchment in the area (at the end of rainy seasons) More than half-full (75%) or full Half-full (50%) at the end of rainy season Less than half-full (25%)/empty” ([Gualazzini, 2021, p. 6](zotero://select/groups/4773535/items/BWDYDL8T)) ([pdf](zotero://open-pdf/groups/4773535/items/8U5XVU5K?page=6&annotation=HMGNP355))

“literacy was cited as a key issue hindering community leaders from further participating in digital data collection.” ([Gualazzini, 2021, p. 9](zotero://select/groups/4773535/items/BWDYDL8T)) ([pdf](zotero://open-pdf/groups/4773535/items/8U5XVU5K?page=9&annotation=J6DKMPDX))

“Half of all respondents stated that digital shock questionnaires would be too complex to handle if they were filled exclusively by community leaders, and translation to Somali language was recommended.” ([Gualazzini, 2021, p. 9](zotero://select/groups/4773535/items/BWDYDL8T)) ([pdf](zotero://open-pdf/groups/4773535/items/8U5XVU5K?page=9&annotation=BM8PXNQ5))

“Around half of survey respondents emphasised that community leaders taking part in RTRM are well aware that reported shock data may influence the probability of receiving future humanitarian assistance in their area.” ([Gualazzini, 2021, p. 9](zotero://select/groups/4773535/items/BWDYDL8T)) ([pdf](zotero://open-pdf/groups/4773535/items/8U5XVU5K?page=9&annotation=8KNB2WAI))

“This context highlights the critical importance of trustbuilding and data triangulation in community-based early warning systems.” ([Gualazzini, 2021, p. 9](zotero://select/groups/4773535/items/BWDYDL8T)) ([pdf](zotero://open-pdf/groups/4773535/items/8U5XVU5K?page=9&annotation=2D9BKJTN))

“The challenge looking forward will be determining the best methods for strengthening existing accountability structures and verification approaches to mitigate these biases.” ([Gualazzini, 2021, p. 9](zotero://select/groups/4773535/items/BWDYDL8T)) ([pdf](zotero://open-pdf/groups/4773535/items/8U5XVU5K?page=9&annotation=EI3XUNJC))

“For every red-flagged area, the respective early warning FPs contextualise the situation, interpret the indicators, triangulate with secondary data sources and propose relevant early action measures. This triangulation stage compares community and arealevel data with data at a higher geographical level (e.g., livelihood zone or district), contextualising the situation with the broader situation in other regions of Somalia.” ([Gualazzini, 2021, p. 9](zotero://select/groups/4773535/items/BWDYDL8T)) ([pdf](zotero://open-pdf/groups/4773535/items/8U5XVU5K?page=9&annotation=ST2P2WYY))

“This process includes the analysis of satellite-derived climate data and information from sources. Further third-party secondary data such as the FSNAU EWEA dashboard, WHO-Early Warning, Alert and Response Network data or other NGO assessments – also contribute to the triangulation process. District-level and macro-level data provide an evidence base to triangulate subjective information from communities, local authorities or field staff.” ([Gualazzini, 2021, p. 10](zotero://select/groups/4773535/items/BWDYDL8T)) ([pdf](zotero://open-pdf/groups/4773535/items/8U5XVU5K?page=10&annotation=69DDN2D8))

“Therefore, for slow-onset shocks captured by the BRCiS RTRM, early action decision-triggering is possible within 12 days after the data collection.” ([Gualazzini, 2021, p. 10](zotero://select/groups/4773535/items/BWDYDL8T)) ([pdf](zotero://open-pdf/groups/4773535/items/8U5XVU5K?page=10&annotation=HFQHTQKN))

“In practice, however, delays in RTRM data collection in the field and in the submission of EWEA forms have been frequent due to operational challenges and other factors. The average timeline between the submission of RTRM data and the submission of early action recommendations in the EWEA forms was 25 days.” ([Gualazzini, 2021, p. 10](zotero://select/groups/4773535/items/BWDYDL8T)) ([pdf](zotero://open-pdf/groups/4773535/items/8U5XVU5K?page=10&annotation=CL7QZEA6))

“The consortium aims to reduce this time-lapse to a maximum of 12 days. For sudden shocks such as riverine or flash floods, the time for programme managers to trigger early response was limited to one to four days following the rapid assessment of impacts within communities.” ([Gualazzini, 2021, p. 10](zotero://select/groups/4773535/items/BWDYDL8T)) ([pdf](zotero://open-pdf/groups/4773535/items/8U5XVU5K?page=10&annotation=HNG6XTD8))


% summery and learning points
“Summary of BRCiS EWEA:” ([Gualazzini, 2021, p. 18](zotero://select/groups/4773535/items/BWDYDL8T)) ([pdf](zotero://open-pdf/groups/4773535/items/8U5XVU5K?page=18&annotation=B8AT9Q75))

1. “Community leaders can play a greater role in humanitarian response.” ([Gualazzini, 2021, p. 18](zotero://select/groups/4773535/items/BWDYDL8T)) ([pdf](zotero://open-pdf/groups/4773535/items/8U5XVU5K?page=18&annotation=MED4PHDQ))

2. “2. Risk monitoring has proved highly effective and essential for future programmes.” ([Gualazzini, 2021, p. 18](zotero://select/groups/4773535/items/BWDYDL8T)) ([pdf](zotero://open-pdf/groups/4773535/items/8U5XVU5K?page=18&annotation=YM4LB2YU))


3. “3. Indicators should be capable of capturing the compound effects of multiple shocks.” ([Gualazzini, 2021, p. 18](zotero://select/groups/4773535/items/BWDYDL8T)) ([pdf](zotero://open-pdf/groups/4773535/items/8U5XVU5K?page=18&annotation=T8PZ5IZA))

“4. Vulnerable communities need improved access to climate and weather information” ([Gualazzini, 2021, p. 19](zotero://select/groups/4773535/items/BWDYDL8T)) ([pdf](zotero://open-pdf/groups/4773535/items/8U5XVU5K?page=19&annotation=4KXI6RK2))

“5. Local-level EWEA committees strengthen local capacities to prepare for and respond to shocks.” ([Gualazzini, 2021, p. 19](zotero://select/groups/4773535/items/BWDYDL8T)) ([pdf](zotero://open-pdf/groups/4773535/items/8U5XVU5K?page=19&annotation=RG55EEU2))

“6. Flexible and shock-responsive funding mechanisms successfully improved food security.” ([Gualazzini, 2021, p. 19](zotero://select/groups/4773535/items/BWDYDL8T)) ([pdf](zotero://open-pdf/groups/4773535/items/8U5XVU5K?page=19&annotation=KW32EH9F))

“7. Communities highly valued and engaged with the BRCiS integrated approach” ([Gualazzini, 2021, p. 20](zotero://select/groups/4773535/items/BWDYDL8T)) ([pdf](zotero://open-pdf/groups/4773535/items/8U5XVU5K?page=20&annotation=UTZLTZB3))

“BRCiS early warning and early action vision (2022–2026)” ([Gualazzini, 2021, p. 20](zotero://select/groups/4773535/items/BWDYDL8T)) ([pdf](zotero://open-pdf/groups/4773535/items/8U5XVU5K?page=20&annotation=QVKJRXK6))

\subsection{EAP + forecast, trigger and so on}

“CBS reporting methods” ([Goodermote, 2020, p. 31](zotero://select/groups/4773535/items/A35DRWSR)) ([pdf](zotero://open-pdf/groups/4773535/items/F795EXE4?page=31&annotation=PVEEL4FW))

% NYSS
https://www.sciencedirect.com/science/article/pii/S1201971221011280

% + short look into NYSS -> with transition to results as most of the information comes from the interviews.. (I think[??] at least parts of it and that might be enough to shift in to that section.. more results yooooo)

% CBS is often more than just monitoring --> e.g. IFRC
1: IFRC. (n.d.). Community Engagement and Accountability. https://www.ifrc.org/community-engagement-and-accountability. [15.09.2022].

https://www.cbsrc.org/

“Community-based surveillance Protocol template” ([IFRC, p. 1](zotero://select/groups/4773535/items/4I8DIUK3)) ([pdf](zotero://open-pdf/groups/4773535/items/LJ5B2HNS?page=1&annotation=3KDKKYAS))
% can also be part of anticipatory actions
2: SRCS. (2021). Measles outbreak detected by Somaliland SRCS Volunteers in Todgheer Region. https://drive.google.com/file/d/1O9PMPKKL312o1zbXELgB7FuMdokzWpic/view. [15.09.2022].
3: SRCS (2022): Feasibility Study on Potential Use of Forecast-based Financing (FbF) for SRCS Final Report. Nottawasage Institute.
4: Start Network (2022): Integrating community voices in anticipatory action: a synthesis of complex qualitative data. Anticipatory-hub.org/news/integrating-community-voices-in-anticipatory-action-a-synthesis-of-complex-qualitative-data. [15.09.2022]

% When early warning is not enough — Lessons learned from the 2011 Somalia Famine
https://www.sciencedirect.com/science/article/abs/pii/S2211912412000107



which forecasts are selected by the EAP pre-study?

current challenges for utilisation of forecasting systems: scarse coverage of weather stations and poor utilisation by the farmers often due to bad dissemination channels  (too coarse, too unreliable)
Somalia Multi-hazard Early Warning Centre under Ministry of Humanitarian Affairs & Disaster Management
based on open data sources (coverage: Somalia)

FAO Somalia Water and Land Information Management (SWALIM) coverage: all regions - source: 100 manual rainfall stations, eight synoptic weather stations and 11 automatic weather stations in Somalia

“Forecast Menu for SRCS” ([Somali Red Crescent Society, 2022, p. 43](zotero://select/groups/4773535/items/FZ6BJHJA)) ([pdf](zotero://open-pdf/groups/4773535/items/RJKNZZZ2?page=43&annotation=5A354LG7))
identified by the SRCS pre-study


% https://docs.google.com/document/d/1xUEXm8RxVHTO468KqXSAoBX-cpkPwiff/edit
"Trigger(s) statement

The  Integrated Phase Classification (IPC) food security projections (ML2) produced by FEWSNET will be used as trigger information. 
The trigger will be assessed between June and August before the Deyr (October-December) season , between August and December before the Gu (April-June) season, always with the  publication of the food security outlooks.


Trigger source is IPC food security projections produced by FEWSNET https://fews.net/. 
Projections for October-January (available between June and August) and  February-May (available from August to December) will be used as a trigger 
The trigger threshold is a projected IPC phase in the relevant period to meet or exceed 4 (emergency). 
The FEWS NET IPC includes rainfall scenarios based on the Climate and Hazards Centre (CHC) forecast, but also integrates a range of current-status information on food security. It is expected that this provides the “best guess” for projections of food security, and more relevant for the prioritised impact (food shortage) than rainfall totals alone. The skill of the IPC forecast in east africa was assessed in general in Backer & Billing (2021). The forecast has an overall accuracy of 83,64 \%. 
The trigger was set at phase 4 (“Emergency”), which is linked to very high acute malnutrition and excess mortality, or emergency livelihood strategies and asset liquidation. 
Using a phase 4 threshold since 2019 would have led to 5 triggers in Burco and 3 triggers respectively in Erigavo and Xudun, considering all forecasts available in 2 month steps. In Burco the trigger would have been reached throughout all forecasts for 2022 affecting 584.000 people. In Erigavo the trigger would have been reached in 2019 (affecting 269.000 people) and two times in 2022 (affecting first 144.000 and later on 269.000 people). The district Xudun would have been triggered in 2019 and two times 2022, affecting 41.000 people respectively.
By comparison, triggering on projection of phase 3 or higher would have led to 15 triggers in each of the pilot districts. A phase 5 threshold would have led to no triggers so far in Somaliland (FEWSNET data)."

------>>>>>> https://fews.net/ <<<<<<------

% Community-led early warning and anticipatory action in Somalia
https://www.sparc-knowledge.org/news-features/features/community-led-early-warning-and-anticipatory-action-somalia

%% overview about all potential meteorological drought indices
https://heigit.atlassian.net/wiki/spaces/FIS/pages/1704096/Indices

https://cdi.faoswalim.org/index/cdi


% --> water level monitoring is one trigger for in the chain of triggers (right before the last action so to say)
% compare: “Figure 5. Contrasting the timeline between fast-onset hazards and droughts” ([RCRC, 2020, p. 20](zotero://select/groups/4773535/items/UESIQTRJ)) ([pdf](zotero://open-pdf/groups/4773535/items/P5JPVZ97?page=20&annotation=Q4LRCBC4))


%possibly talk about this trigger as well
“BRCiS community-based real-time monitoring indicators and thresholds summary” ([Gualazzini, 2021, p. 6](zotero://select/groups/4773535/items/BWDYDL8T)) ([pdf](zotero://open-pdf/groups/4773535/items/8U5XVU5K?page=6&annotation=G26DY9D3))

%% critical water level
“Primary water source condition Observation of the status of the largest rainwater catchment in the area (at the end of rainy seasons) More than half-full (75%) or full Half-full (50%) at the end of rainy season Less than half-full (25%)/empty” ([Gualazzini, 2021, p. 6](zotero://select/groups/4773535/items/BWDYDL8T)) ([pdf](zotero://open-pdf/groups/4773535/items/8U5XVU5K?page=6&annotation=HMGNP355))

%%%%%%%%%%%%%%%%%%%%%%%%%%%%%% ACTIONS %%%%%%%%%%%%%%%%%%%%%%%%%%%%%%%%%%%%%%%%%%

“2.3. Action” ([pdf](zotero://open-pdf/groups/4773535/items/WKFPUZCW?page=15&annotation=92HW2UP9))

“water vouchers are the leveraging of existing market capacity and the leading role of CRCs in community in price negotiation, supervision of water delivery through community vouchers and monitoring of water quantity and quality.” ([Gualazzini, 2021, p. 13](zotero://select/groups/4773535/items/BWDYDL8T)) ([pdf](zotero://open-pdf/groups/4773535/items/8U5XVU5K?page=13&annotation=RXEH724K))

“The value for money (VFM) analysis showed how the communitybased negotiation approach led to savings of USD 20,173 compared to the classic water trucking modality.” ([Gualazzini, 2021, p. 13](zotero://select/groups/4773535/items/BWDYDL8T)) ([pdf](zotero://open-pdf/groups/4773535/items/8U5XVU5K?page=13&annotation=QY5PKXHV))

“Communities also reported a significant reduction in diarrhoea cases, likely due to water chlorination and increased water availability for hygiene purposes.” ([Gualazzini, 2021, p. 13](zotero://select/groups/4773535/items/BWDYDL8T)) ([pdf](zotero://open-pdf/groups/4773535/items/8U5XVU5K?page=13&annotation=57W6V43K))

“In the absence of water price regulation policy and enforcement, the price of water from wells and boreholes is abnormally high during the dry season.” ([Gualazzini, 2021, p. 13](zotero://select/groups/4773535/items/BWDYDL8T)) ([pdf](zotero://open-pdf/groups/4773535/items/8U5XVU5K?page=13&annotation=WKNG9T3Y))

“Those vulnerable households that cannot afford safe water often significantly reduce consumption or turn to unprotected water sources. This water insecurity during the dry seasons is further exacerbated by drought conditions after one or two consecutive belowaverage rainy seasons. In such instances, surface waters deplete earlier than usual, and poor harvests and pasture further diminish households’ purchasing power.” ([Gualazzini, 2021, p. 13](zotero://select/groups/4773535/items/BWDYDL8T)) ([pdf](zotero://open-pdf/groups/4773535/items/8U5XVU5K?page=13&annotation=L4VA7DR7))

“community water vouchers for safe water access during the last two months of the dry season is an early action measure to reduce the risk of disease outbreak, malnutrition and displacement in search of humanitarian assistance and water.” ([Gualazzini, 2021, p. 13](zotero://select/groups/4773535/items/BWDYDL8T)) ([pdf](zotero://open-pdf/groups/4773535/items/8U5XVU5K?page=13&annotation=YJHP3TBG))

“This approach is an appropriate alternative to the water trucking method in areas where there is a functional market of private water vendors already serving communities without permanent water sources.” ([Gualazzini, 2021, p. 13](zotero://select/groups/4773535/items/BWDYDL8T)) ([pdf](zotero://open-pdf/groups/4773535/items/8U5XVU5K?page=13&annotation=QCS5846H))

“The main characteristics of community” ([Gualazzini, 2021, p. 13](zotero://select/groups/4773535/items/BWDYDL8T)) ([pdf](zotero://open-pdf/groups/4773535/items/8U5XVU5K?page=13&annotation=79UZ7AXG))
%% Suggested AA
The feasibility study identifies a number of high potential early actions in response to anticipated drought, including:

Water storage rehabilitation – berked repairs and re-charging can be undertaken at a reasonable cost with available skills within the lead time afforded by a drought forecast; (https://heigit.atlassian.net/wiki/spaces/FIS/pages/1704186/Background+Information#BackgroundInformation-SuggestedAA)

% Anticipatory Action Plan for Drought in Somalia (OCHA)
https://heigit.atlassian.net/wiki/download/attachments/1704186/Somalia-Anticipatory-Action-Plan_vs16.pdf?version=1&modificationDate=1676385881297&cacheVersion=1&api=v2
% quick summery here: https://heigit.atlassian.net/wiki/spaces/FIS/pages/1704186/Background+Information#BackgroundInformation-AAforSomalia

% UN OCHA Anticipatory Action. Lessons from the 2020 Somalia Pilot- Emilie Gettlife
https://heigit.atlassian.net/wiki/download/attachments/1704186/WP_9_22Oct.pdf?version=1&modificationDate=1677062248719&cacheVersion=1&api=v2
% summery is also available


% possibly include just as an example
“In humanitarian practice, the term "drought" is often used to refer to some socio-meteorological combination where water shortages produce stress on human and livelihood systems. Droughts are a function of the fragility of human systems, and they become disasters where systems cannot cope with deviations from the hydro-meteorological norm. It has been argued that droughts are particularly devastating when livelihood choices are strongly determined by the climate (e.g. the decision to grow certain crops, or traditional seasonal migration patterns) - for instance, if in a given year, the weather patterns are different than normal, those livelihoods are especially vulnerable to these changes. It has also been argued that droughts pose specific challenges to income generating activities marked by low productivity that are not able to take advantage of ‘good years’ in order to provide a buffer during ‘bad years’.” ([RCRC, 2020, p. 12](zotero://select/groups/4773535/items/UESIQTRJ)) ([pdf](zotero://open-pdf/groups/4773535/items/P5JPVZ97?page=12&annotation=C622YWIR))

%% DEWS Drougth Early Warning Systems
"The study concluded that information from DEWS was of good quality although it was poorly disseminated. There is need to enhance the timeliness of information dissemination if the system is to effectively enhance community preparedness to cope the effects of drought." Akwango, D., Obaa, B. B., Turyahabwe, N., Baguma, Y., & Egeru, A. (2017). Quality and dissemination of information from a drought early warning system in Karamoja sub-region, Uganda. Journal of Arid Environments, 145, 69–80. https://doi.org/10.1016/j.jaridenv.2017.05.010 (already in zotero)


“Therefore, the impacts of drought could be mitigated by managing water demand through crop management, modifying water allocation rules during times of water scarcity, developing various water resources (such as groundwater recharge and salt water desalination), managing multiple water use, setting up water-trading mechanisms in advance of times of drought or scarcity, and physically redistributing available supplies during times of scarcity [26].” ([Wang et al., 2016, p. 2](zotero://select/groups/4773535/items/AWKLZ543)) ([pdf](zotero://open-pdf/groups/4773535/items/4YRNMM9P?page=2&annotation=S8QKSIPL))

“Improving water quality within the distribution network and promoting safer water handling practices are proposed to reduce the health risk due to consumption of contaminated water in this setting.” ([Daniel et al., 2020, p. 1](zotero://select/groups/4773535/items/ZA5MKHPC)) ([pdf](zotero://open-pdf/groups/4773535/items/CS4A7XIN?page=1&annotation=3WKEQLY4))


“Implementing early action measures helps guard against impending shocks, and early response addresses short-term humanitarian needs while protecting overarching resilience gains.” ([Gualazzini, 2021, p. 4](zotero://select/groups/4773535/items/BWDYDL8T)) ([pdf](zotero://open-pdf/groups/4773535/items/8U5XVU5K?page=4&annotation=CISMIJ8W))








"Suggested AA
The feasibility study identifies a number of high potential early actions in response to anticipated drought, including:

Water storage rehabilitation – berked repairs and re-charging can be undertaken at a reasonable cost with available skills within the lead time afforded by a drought forecast;
De-stocking – this is an effective mitigation against asset loss in the time preceding a forecast drought;
Early planting or alternative short growth crops – the MoA has experience in advising farmers on alternative short growth crops to facilitate harvesting before drought onset (Ministry can advise farmers to plant maize rather than sorghum as it has a 50% shorter growing season)
Cash distribution programs – SRCS’ long experience in cash programs make this a ready approach to mitigating suffering in coordination with the CWG;
Women/child shelters – SRCS Protection program can work with government to ensure acceleration of staffing and resources for support in the period ahead of drought onset.
→ Need to work with this Ministry in a drought-focused FbF – on both the forecasting and early response sides. On the latter, for example, in the event of a forecast drought the Ministry can advise farmers to plant maize rather than sorghum as it has a 50% shorter growing season.

→ According to the Somaliland Ministry of Livestock and Fisheries Development (Director of Planning) there are limited options for early action prior to a forecast drought as livestock owners are reluctant to de-stock, do not trust scientific forecasting, and there is no internationally approved abattoir so there is a limit to the absorptive capacity of the local market for massive de-stocking. Options such as improving water supply, pasture sequestration and promoting fodder production are more effective if resources are available.

→ While conflict is difficult to forecast accurately, its frequent upsurge in times of drought needs to be seen as a potential impact of drought. Drought early action planning, therefore, should consider actions to mitigate the impact of conflict. This notion applies to other ancillary impacts, such as migration, noted above, or food insecurity"


“On the latter, for example, in the event of a forecast drought the Ministry can advise farmers to plant maize rather than sorghum as it has a 50% shorter growing season.” ([Somali Red Crescent Society, 2022, p. 52](zotero://select/groups/4773535/items/FZ6BJHJA)) ([pdf](zotero://open-pdf/groups/4773535/items/RJKNZZZ2?page=56&annotation=3P4MD5XD))

“there are limited options for early action prior to a forecast drought as livestock owners are reluctant to de-stock” ([Somali Red Crescent Society, 2022, p. 52](zotero://select/groups/4773535/items/FZ6BJHJA)) ([pdf](zotero://open-pdf/groups/4773535/items/RJKNZZZ2?page=56&annotation=3E9VIM42))

“and there is no internationally approved abattoir so there is a limit to the absorptive capacity of the local market for massive de-stocking.” ([Somali Red Crescent Society, 2022, p. 52](zotero://select/groups/4773535/items/FZ6BJHJA)) ([pdf](zotero://open-pdf/groups/4773535/items/RJKNZZZ2?page=56&annotation=N94WKF3Q))

%----------------------------------------------------------------------------------------
%	SECTION 9 Conclusion literature
%----------------------------------------------------------------------------------------


\section{Conclusion literature}
In the conclusion, you should summarize the key findings you have taken from the literature and emphasize their significance.

%-----------------------------------
%	SUBSECTION 9.1
%-----------------------------------
\subsection{key findings}

while there are lots of hydrological monitoring stuff out there (mostly crowdsensing in the case of water + focused on health by CBS), there is none, that looks at drinking water source monitoring direktly by applying the proven sets of methods on a new topic in a new environment (not North America but resource scarce environments from the perspective of an NGO (SRCS)) 


%--> focus is on these and not on drinking water for a community
“Within these citizen science projects, monitoring mainly focuses on rivers (48%) but also considers lakes (26%), groundwater (12.57%), and estuaries (13.33%)” ([Kirschke et al., 2022, p. 4](zotero://select/groups/4773535/items/GPC3LDT5)) ([pdf](zotero://open-pdf/groups/4773535/items/AI7HRQYC?page=4&annotation=QGL5N2BA))

“Citizen scientists measure physical (34.22%, e.g., temperature, turbidity, color), chemical (34.76%, e.g., pH, nitrates, phosphates, dissolved oxygen), biological (22.99%, e.g., fecal coliform bacteria, algae, aquatic macroinvertebrates), and some other parameters (8.02%, e.g., macro/microplastic pollution, riparian habitat), demonstrating the diverse tasks citizen scientists engage in. Citizen scientists further measure these parameters rather ‘regularly’ (82.93%) than ‘sometimes only’ (17.07%), indicating continuity and thus a substantial contribution of citizens to the monitoring process.” ([Kirschke et al., 2022, p. 4](zotero://select/groups/4773535/items/GPC3LDT5)) ([pdf](zotero://open-pdf/groups/4773535/items/AI7HRQYC?page=4&annotation=6GC8UE8I))

%The findings indicate that citizen science is an economical approach for gathering critical monitoring data, and it can also generate significant levels of public participation, which are crucial for the adaptive management learning process.
“The results suggest that citizen science can be a cost-effective method to collect essential monitoring information and can also produce the high levels of citizen engagement that are vital to the adaptive management learning process.” ([Aceves-Bueno et al., 2015, p. 493](zotero://select/groups/4773535/items/YK2MKLA9)) ([pdf](zotero://open-pdf/groups/4773535/items/WGGHNGZB?page=1&annotation=ZQQBEP74))

“Furthermore it is an interesting communication tool in the light of science communication. Correspondingly water managers should be interested in participatory monitoring in the light of integrated water management.” ([Minkman, 2015, p. 199](zotero://select/groups/4773535/items/ZKLE6CPT)) ([pdf](zotero://open-pdf/groups/4773535/items/QMAPCSZG?page=199&annotation=GHI9KSDA))


The community-based monitoring and management concept has, to the best of my knowledge, not been carried out once nor scientifically investigated in regard to directly monitoring and managing of rural water sources.
%-----------------------------------
%	SUBSECTION 9.2
%-----------------------------------
\subsection{Limitations}

Although there is a possibility that a project in North America or Europe may have been missed, it is more probable that this is not the case for Sub-Saharan regions.

“4 | IS REMOTE MONITORING A PATHWAY TO SUSTAINABILITY?” ([Thomson, 2021, p. 9](zotero://select/groups/4773535/items/UQLXVVYI)) ([pdf](zotero://open-pdf/groups/4773535/items/K9XBXPQD?page=9&annotation=833Q66UP))

%-----------------------------------
%	SUBSECTION 9.3
%-----------------------------------

\subsection{Key assumptions}



%----------------------------------------------------------------------------------------
%----------------------------------------------------------------------------------------
%----------------------------------------------------------------------------------------


\section{Notes}
% Notes: 
% the literature review is focused on providing background information and enabling historical interpretation of the subject of analysis in relation to the research problem the case is intended to address. This includes synthesizing studies that help to:

% "Place relevant works in the context of their contribution to understanding the case study being investigated. This would include summarizing studies that have used a similar subject of analysis to investigate the research problem. If there is literature using the same or a very similar case to study, you need to explain why duplicating past research is important [e.g., conditions have changed; prior studies were conducted long ago, etc.].
% Describe the relationship each work has to the others under consideration that informs the reader why this case is applicable. Your literature review should include a description of any works that support using the case to study the research problem and the underlying research questions.
% Identify new ways to interpret prior research using the case study. If applicable, review any research that has examined the research problem using a different research design. Explain how your case study design may reveal new knowledge or a new perspective or that can redirect research in an important new direction.
% Resolve conflicts amongst seemingly contradictory previous studies. This refers to synthesizing any literature that points to unresolved issues of concern about the research problem and describing how the subject of analysis that forms the case study can help resolve these existing contradictions.
% Point the way in fulfilling a need for additional research. Your review should examine any literature that lays a foundation for understanding why your case study design and the subject of analysis around which you have designed your study may reveal a new way of approaching the research problem or offer a perspective that points to the need for additional research.
% Expose any gaps that exist in the literature that the case study could help to fill. Summarize any literature that not only shows how your subject of analysis contributes to understanding the research problem, but how your case contributes to a new way of understanding the problem that prior research has failed to do.
% Locate your own research within the context of existing literature [very important!]. Collectively, your literature review should always place your case study within the larger domain of prior research about the problem. The overarching purpose of reviewing pertinent literature in a case study paper is to demonstrate that you have thoroughly identified and synthesized prior studies in the context of explaining the relevance of the case in addressing the research problem."https://libguides.pointloma.edu/c.php?g=944338&p=6806958
