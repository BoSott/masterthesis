% Chapter Template

\chapter{Theoretical Background} % Main chapter title

\label{Chapter2} % \ref{Chapter2}
%----------------------------------------------------------------------------------------
%	SECTION 1
%----------------------------------------------------------------------------------------

\section{Introduction}

This chapter provides an introduction to the main concepts, theories and literature in which this thesis is embedded. Context is given by presenting current definitions together with their characteristics and differences of broad concepts such as water security, water scarcity and drought. Building on this foundation, the approach to measure and monitor these wide concepts via indicators and indices together with the ideas of risk, vulnerability and impact is introduced.
Extending the previously prevailing idea of the \acrfull*{drr} cycle of mitigation, preparation, response and recovery, the rather recently emerged concept and operationalisation of \acrfull{fbf} is described in detail. The details cover aspects of structure, decision-making based on forecasts, setting strict thresholds for when to act, and finally what to do when thresholds are reached.
Building on the realisation that the current data basis for predictions is too coarse for precise measures, another broad field is introduced, that of \acrfull*{cs}. After \acrshort{cs} is introduced, it is further extended to areas of interest specifically focussing on community collaboration, data acquisition and the integration of information technology infrastructure. \acrfull{cbs} and \acrfull*{cbwm} are given as concise examples for successful implementation in local context and thematic transferability of the approach, respectively.
Before coming to a conclusion about the literature and the state of research so far, the case study is described. Geographical and climatic conditions as well as historical and current economic and socio-cultural contexts are discussed. Furthermore, the wide-ranging concepts from the beginning will be embedded in local contexts and previous efforts and activities to implement anticipatory measures will be described.
The final conclusion will identify and present the key points of the overall chapter in terms of strengths, shortcomings and prevailing research gaps.
This work will not address the integration or applications of local and/or indigenous knowledge or the implications of \acrfull*{vgi}, although very interesting as neither is the primary focus of this work. Furthermore, the concepts mentioned, such as water security, drought or Citizen Science, are extremely complex and highly debated topics. Discussing them in detail would exceed the scope of this thesis, which is why focal points are set according to the priority of this work.

%----------------------------------------------------------------------------------------
%	SECTION 2 Water security, drought & water scarcity/quality/access
%----------------------------------------------------------------------------------------


\section{Fundamental concepts - Water Security, Water Scarcity and Drought}
% human induced water shortage component
% what about water security? “Water Security: A Complex Concept” ([Butte et al., 2022, p. 1](zotero://select/groups/4773535/items/QB97YZ2M)) Q936I2JN
% and insecurity? “Progress in household water insecurity metrics: a crossdisciplinary approach” ([Jepson et al., 2017, p. 1] HWX5JRS4

Water security is a theoretical construct that has emerged in the 21st century to frame the overall water objectives and goals to guide local to global water management and policy development \autocite{sadoffWaterSecurity2020a}. It "links together the web of food, energy, climate, economic growth, and human security challenges that the world economy faces over the next two decades" \autocite[5]{wefBubbleCloseBursting2009}. In more detail, it is about "the availability of an acceptable quantity and quality of water for health, livelihoods, ecosystems and production, coupled with an acceptable level of water-related risks to people, environments and economies."\autocite{greySinkSwimWater2007}.
Water security integrates therefore economic, social and environmental dimensions into an interconnected and complex network of human and natural relations by addressing risks of too much, too little or poor quality water \autocite{vanbeekWaterSecurityPutting2014, mishraWaterSecurityChanging2021}. Due to the focus of this work, emphasis is placed on factors that decrease water security due to too little water availability. Besides other factors, natural disasters such as droughts, and water scarcity are the main drivers for insufficient quantities of water \autocite{caretta2022water}. Water quality and access are briefly addressed in addition to provide a more comprehensive understanding of water security for the following chapters.

% TODO: maybe switch the order here.. first water scarcity, then drought. -> would make more sense and it's only a small paragraph --> but takes some effort in the second paragraph of water scarcity.. maybe later on. Not that important

% nice source for water security: “A Framework for Water Security Data Gathering Strategies” ([Butte et al., 2022, p. 1](zotero://select/groups/4773535/items/QB97YZ2M)) ([pdf](zotero://open-pdf/groups/4773535/items/Q936I2JN?page=1&annotation=5G7DUDDP))

%-----------------------------------
%	SUBSECTION 2.1
%-----------------------------------

\subsection{About Drought}\label{subsec:about_drought}

Drought as highly complex and severe climate-related multi-hazard has far reaching, cascading and interconnected consequences affecting natural ecosystems, societies and economies (see figure \ref*{TODO:})\autocite{vereintenationenSpecialReportDrought2021}. Historically, droughts are a recurring feature that can occur in all climates. They can geographically extend over small areas to entire sub-continents and are slow onset events that can persist for a few weeks to several years. These high spatial and temporal variabilities make drought not only challenging to define but due to its slow onset, droughts are often only recognized when they are well advanced \autocite{idmpDroughtWaterScarcity2022,vereintenationenSpecialReportDrought2021}. While some drought conditions over large areas can be associated to some low-frequency changes in atmospheric conditions such as the El Niño, accurate cause identification can be rather challenging on smaller scales and requires many different parameters \autocite{botaiAnalysisDroughtProgression2019, vereintenationenSpecialReportDrought2021}.


% noice figurrrrreee
%“Figure 1.6. Schematic representation of potential interconnections among different sectors affected by droughts” ([“Special report on drought 2021”, 2021, p. 47](zotero://select/groups/4773535/items/RAAM9PVS)) ([pdf](zotero://open-pdf/groups/4773535/items/7AK5QVBL?page=49&annotation=TNA9J8ZS))

\missingfigure{sick figure y<o}

%In order to approach this complexity, drought is most often defined from four different perspectives, focussing on different manifestations and stages. These definitions are outlined in the coming sub-chapter \ref*{subsec:drought_definitions}, followed by a section addressing the necessary indicators  currently employed in practice for these definitions.
%Generally, droughts are commonly characterized by deviations or the complete failure of climate and weather systems that drive the hydrological cycle compared to normal conditions\autocite{botaiAnalysisDroughtProgression2019,idmpDroughtWaterScarcity2022,vanloonDroughtHumanmodifiedWorld2016,vereintenationenSpecialReportDrought2021}. A more in depth definition can be found in the sub-chapter \ref*{subsec:drought_definitions}.
This complex conglomeration of interrelated causes and effects of multiple temporal, spatial and thematic dimensions makes the definition of \textit{drought} a fairly multi-layered undertaking \autocite{balintMonitoringDroughtCombined2013}. Several well-known definitions are, for example, from the \autocite{theamericanheritagedictionaryoftheenglishlanguageDrought2022} defining drought as "a long period of abnormally low rainfall, especially one that adversely affects growing or living conditions". \autocite[2]{palmerMeteorologicalDrought1965} defines drought as "a prolonged and abnormal moisture deficiency." or \autocite{vanloonDroughtHumanmodifiedWorld2016} defines droughts simply as "an exceptional lack of water compared to normal conditions". Other drought definitions emphasize its natural and/or human origin, its special characteristics, impact and temporal duration or even understand "drought as a system of causality where the link between causes and effects is random in nature {balintMonitoringDroughtCombined2013, baltiReviewDroughtMonitoring2020, idmpDroughtWaterScarcity2022,loonDroughtAnthropocene2016, wangPropagationDroughtMeteorological2016, wilhiteUnderstandingDroughtPhenomenon1985}. Already in the 1980s, \autocite{wilhiteUnderstandingDroughtPhenomenon1985} found more than 150 published definitions of drought. Besides the categorization into a conceptual or operational category , \autocite{wilhiteUnderstandingDroughtPhenomenon1985} proposed a clustering of these definitions into four types, namely meteorological drought, agricultural drought, hydrological drought and socio-economic drought. This classification is still widespread today \autocite{balintMonitoringDroughtCombined2013, baltiReviewDroughtMonitoring2020, idmpDroughtWaterScarcity2022,vereintenationenSpecialReportDrought2021}.

The conceptual category refers to a general formulation of an idea of drought to understand its concept and identify its boundaries and is often formulated in relative terms \autocite{wilhiteUnderstandingDroughtPhenomenon1985}. Definitions in the operational category try to define how drought functions in terms of its onset, duration, severity and spatial coverage also covering how this can be measured via indices \autocite{balintMonitoringDroughtCombined2013, nationaldroughtmitigationcenterWhatDrought, wilhiteUnderstandingDroughtPhenomenon1985}. With these definitions, the current situation is usually compared to a historical average, which is usually based on a 30-year period,  presupposing the development and continuous measurement of indicators and indices that can be used. \autocite{vereintenationenSpecialReportDrought2021,wilhiteUnderstandingDroughtPhenomenon1985}.

The four types of drought are commonly conceptually defined and brought into practice by operational specifications. They can be understood as different, but complementary stages of the same process and are generally cascading in reason and time but can overlap and are difficult to completely unravel. Table \todo{TODO: see RCRC 2020 p.11} displays the four types at a glance and figure \todo{TODO:, see https://drought.unl.edu/Education/DroughtIn-depth/TypesofDrought.aspx} shows an overview about the different types, their succession and cascading elements.

\missingfigure{table RCRC 2020 p.11}

The \textit{meteorological drought} is usually characterized by the duration and the degree of dryness in comparison to the normal average amount and tries to conceptually understand how weather patterns can impact water availability. Definitions in this category are specific for a regions atmospheric conditions. That is to say that regions with a year-round precipitations regime such as tropical rainforest need different definitions and thresholds than e.g. climates characterized by seasonal rainfall patterns \autocite{nationaldroughtmitigationcenterTypesDrought}. Operational classification mostly uses rainfall, moisture, temperature and wind indicators to determine the onset, severity and duration of drought.

\textit{Agricultural drought} definitions establish a connection between different features of meteorological drought with their impacts on agriculture. Soil-moisture, differences between actual and potential evapotranspiration and soil water deficits are some of the operationalized indicators for monitoring this type of drought \autocite{baltiReviewDroughtMonitoring2020,nationaldroughtmitigationcenterTypesDrought,wilhiteUnderstandingDroughtPhenomenon1985}.

The type of \textit{hydrological drought} is associated with the impact of meteorological drought on surface or subsurface water resources such as rivers, lakes, and groundwater. Hydrological drought occurs when these indicators drop below normal levels \autocite{palmerMeteorologicalDrought1965}. The fastest responding indicator of this type of drought is most often the variability of streamflow. The water levels of lakes and groundwater usually lag behind the occurrence of the meteorological or agricultural drought which is why the hydrological drought is often out of phase with the previously mentioned types. The hydrological drought is commonly defined on the basis of watershed or river basin scale \autocite{baltiReviewDroughtMonitoring2020,nationaldroughtmitigationcenterTypesDrought,wilhiteUnderstandingDroughtPhenomenon1985}.

The \textit{socioeconomic drought} differs from the aforementioned types as it can also incorporate features of these types of drought to associate them with the demand and supply of some social or economic good. It therefore relates the impact of all other types of droughts on human population and its various sectors of society such as food security, health, and the economy. It is therefore sometimes also interchangeably used with drought impacts. Operational categorization involves using socioeconomic indicators such as unemployment rates and food prices to assess the severity and duration of the drought \autocite{nationaldroughtmitigationcenterTypesDrought,wilhiteUnderstandingDroughtPhenomenon1985}.

\missingfigure{https://drought.unl.edu/Education/DroughtIn-depth/TypesofDrought.aspx}

The shown economic, social and environmental impacts of drought in figure \todo{TODO:} depend on the severity of, and the risk to drought. These three concepts of impact, severity and risk are interrelated concepts used to assess and understand the effects of drought on various sectors. Thereby, in alignment with the definition of \autocite{vanloonDroughtHumanmodifiedWorld2016} it is the exceptional severity of the water shortage that distinguishes drought from aridity, an ordinarily recurrent or fully dry climate, and from water scarcity as a long-term "supply/demand and natural and/or human-made phenomenon" \autocites[7]{idmpDroughtWaterScarcity2022}{vereintenationenSpecialReportDrought2021, vanClimatologicalRiskDroughts2017}. Water scarcity is described in more detail in the following chapter.

%----------------------------------------------------------------------------------------
%	SUBSECTION 2.2 Water Scarcity
%----------------------------------------------------------------------------------------

\subsection{Water Scarcity}\label{subsec:water_scarcity}

Water scarcity, as for water security or drought, is defined in many different ways. The sixth IPCC Assessment Report defines water scarcity broadly as "a mismatch between the demand for fresh water and its availability, quantified in physical terms" \autocite[560]{caretta2022water}. Here, social and economic components are outsourced to the broader concept of water security and insecurity, focussing primarily on physical water scarcity \autocite{caretta2022water}. In contrast, the \acrfull{fao} defines water scarcity as "a gap between available supply and expressed demand of freshwater in a specified domain, under prevailing institutional arrangements (including both resource ‘pricing’ and retail charging arrangements) and infrastructural conditions" \autocite[5]{faoCopingWaterScarcity2012} further summarizing that water security is "an excess of water demand over available supply" \autocite[6]{faoCopingWaterScarcity2012}. Thus, highlighting the human dimension of this interactive and relative concept of physical and economic water scarcity. Hereby, physical water scarcity refers to a situation in which there is not enough water available in quantitative terms to meet demand whereas economic water scarcity occurs when inadequate infrastructure, institutional or financial capital obstructs access to water resources "even though water in nature is available to meet human demands" \autocites{idmpDroughtWaterScarcity2022}[11]{moldenWaterFoodWater2007}.
Water scarcity and drought are in a complex interrelationship with each other. A short overview about the key differences between water scarcity and drought are given in table \todo{TODO:“Table 1. Characteristics and impacts of water scarcity and drought” ([IDMP, 2022, p. 3](zotero://select/groups/4773535/items/LNSL8VD2)) ([pdf](zotero://open-pdf/groups/4773535/items/JM82W3ZF?page=9&annotation=QNC4A3FG))}. 

\missingfigure{differences water scarcity to drought}
% Distinctions between water scarcity and drought:

% also in Joplin
%“Table 1. Characteristics and impacts of water scarcity and drought Water scarcity Drought Length Long-term to permanent Temporary (weeks to multiyear) Driving forces Demand–supply imbalance, human-driven, and/or natural (overexploitation, pollution). Climate change can impact both supply and demand Natural climate variability which can be modified/amplified by climate change Potential impacts Restricted water availability, environmental degradation, desertification, exacerbated inequalities in access to water resources, potential competition Water shortages, competition, environmental degradation Measures Long-term IWRM to bring supply and demand back into sustainable balance Integrated drought management, including: (1) monitoring and early warning; (2quality) vulnerability and impact assessment; and (3) risk mitigation, preparedness and response Source: adapted from Hohenwallner et al. (2011) DROUGHT AND WATER SCARCITY – DEFINITIONS AND CHARACTERISTICS” ([pdf](zotero://open-pdf/groups/4773535/items/JM82W3ZF?page=9&annotation=E3EQRILA))

Furthermore, potential mutual reinforcements, climate change, increased water use and poor water management can make it sometimes difficult to clearly separate these concepts \autocite{idmpDroughtWaterScarcity2022,lealfilhoUnderstandingResponsesClimaterelated2022,liuWaterScarcityAssessments2017,rcrcFORECASTBASEDFINANCINGEARLY2020}. Nonetheless, following the definition of \autocite{faoCopingWaterScarcity2012} the concept of water scarcity always gives water shortage, understood as absolute lack of water in the current situation, a human dimension in particular on the demand side. Here, the quality of policies, planning and management is considered as critical to the overall severity of the impact of water scarcity \autocite{idmpDroughtWaterScarcity2022,faoCopingWaterScarcity2012,vereintenationenSpecialReportDrought2021}. The supply side can be influenced by human activities, but it is not a mandatory prerequisite. \autocite{idmpDroughtWaterScarcity2022}. 
Besides the already mentioned water scarcity on the basis of physical quantity and economical factors, water scarcity can also be caused by water of unacceptable quality and lack of access to water services \autocite{faoCopingWaterScarcity2012}. The recognition that insufficient water quality is an additional contributing factor to water scarcity is a relatively recent development in the literature \autocite{liuThreedimensionalWaterScarcity2020} but together with inadequate access highlights further challenges in ensuring water security \autocite{caretta2022water, mishraWaterSecurityChanging2021}. 

%-----------------------------------
%	SUBSECTION 2.3 Water Access & Water Quality
%-----------------------------------

\subsection{Water Quality \& Access}
% + evtl. human health related water borne diseases and CBS

As could be seen in the previous chapter, besides the quantitative availability of water, its accessibility and quality are crucial. Inadequate water quality can be related to numerous health and environmental issues and can further limit the availability of water for given uses \autocite{rcrcFORECASTBASEDFINANCINGEARLY2020, faoCopingWaterScarcity2012}. Unlike the previous concepts, water quality has mostly fixed indicators by which the condition can be determined but historically, and still today, water quality assessment is primarily carried out in laboratories with preceding water sampling activities. This procedure not only makes the determination of water quality a laborious and costly process, but also places high demands on equipment and personnel, so that it is not viable for large-scale rural assessments in low-income areas. \autocite{tariqOpenSourceWater2021,worldmeteorologicalorganizationPlanningWaterqualityMonitoring2013}. While simpler methods for in situ water quality monitoring exist, they are either insufficient or often still need too much investment and knowledge to conduct for widespread and frequent monitoring \autocite{worldmeteorologicalorganizationPlanningWaterqualityMonitoring2013}. Nonetheless, new solutions are being developed to simplify and scale affordable water quality assessments to rural areas \autocite{ighaloComprehensiveReviewWater2020,tariqOpenSourceWater2021}. While the direct assessment of water quality might be challenging, poor water quality can be linked to other factors. Environmental awareness, poor sanitation and hygiene conditions of people in rural areas were for example considered as major causes for contamination of water at the source \autocite{zamxakaMicrobiologicalPhysicochemicalAssessment2004}.

The definition of water access is again a rather challenging undertaking. The \autocite[254]{worldbankWorldDevelopmentReport1997} defined water access in rural areas by "access implies members of the household do not have to spend a disproportionate part of the day fetching water." While both time and distance still play a crucial role in literature when investigating water access \autocite{cassiviDrinkingWaterAccessibility2019,cassiviEvaluatingSelfreportedMeasures2021,emenikeAccessingSafeDrinking2017}, the term also gained a social component \autocite{emenikeAccessingSafeDrinking2017,mitlinUnaffordableUndrinkable}. \autocite{obeng-odoomAccessWater2012} adds four additional factors namely, affordability, quality, equitable distribution to the definition of water access to fully understand if users have access to water in daily live. \autocite{unitednations/developmentprogrammeDeepeningDemocracyFragmented2002} links these parameters to the access to an improved water source which should provide safe drinking water.
The access to improved water sources is therefore generally considered as crucial in the reaching of water security \autocite{cdcAssessingAccessWater2022}. Proactive measures to drought and water scarcity can not only potentially minimize or even neutralize impacts and are considerably more cost-efficient, early warning and anticipatory actions for drought and water scarcity impacts become ever more important \autocite{faoandun-waterProgressLevelWater2021,idmpDroughtWaterScarcity2022,worldbankHighDryClimate2016}.

%----------------------------------------------------------------------------------------
%	SECTION 4 FbF, EAP, AA & Early Warning
%----------------------------------------------------------------------------------------

\subsection{Measurement and estimation approaches for impacts}\label{subsec:indicators} % or Measuring and estimating impacts

Indicators and Indices are often used to translate complex matters into easier to explain numbers and scales that can be measured, tracked and reasonably compared \autocite{blauveltSystematizingEnvironmentalIndicators2014,williamsUsingIndicatorsExplain2017}. This can range from capturing simple measurements to complex and detailed issues that can not only depict ecological conditions but its interactions with societies \autocite{blauveltSystematizingEnvironmentalIndicators2014,mishraWaterSecurityChanging2021}. Indicators and Indices can thus establish a clear and common understanding of a concept or parts of it in a quantifiable and more objective way.
Here, an indicator is understood as a measurable parameter that provides information on the state or trend of an issue or problem. It can be a physical, chemical, biological, or socio-economic variable, such as temperature, soil moisture or streamflow and can be measured locally or remotely. An index is a composite measure that aggregates multiple indicators into a single value or score \autocite{unitednationsuniversityTooManyIndicators2017,williamsUsingIndicatorsExplain2017, svobodaHandbookDroughtIndicators2016}. Indices are commonly developed on regional or national level to account for the specific circumstances \autocite{unitednationsuniversityTooManyIndicators2017}. This case specification, together with different measurement and aggregation methods, partial inconsistency of definitions and differently focussed objectives on qualitative, quantitative, risk or impact scenarios can constrain their practical application and intercomparability \autocite{svobodaHandbookDroughtIndicators2016,unitednationsuniversityTooManyIndicators2017}. 
Since there is no one definition of drought, water scarcity or security, there is no one best solution to the choice between the many indicators and indices for either of those.

% possibly shorten this.. naming all these indices might be a little overkill, though naming none is also not feasible.
Precipitation, evapotranspiration, soil moisture, lake and groundwater levels, streamflow and vegetation water stress are among the most prominent drought indicators \autocite{europeandroughtobservatoryDroughtIndicators2017}. In order to adequately account for the different drought stages different drought indices, that aggregate these and other indicators, are applied. Among the most prominent meteorological drought indics are the Standardized Precipitation Index \textit{SPI} together with its extension the Standardized Precipitation-Evapotranspiration Index \textit{SPEI} \autocite{europeandroughtobservatoryDroughtIndicators2017,ncarStandardizedPrecipitationEvapotranspiration,ncarStandardizedPrecipitationIndex}. Agricultural drought indices like the Soil Moisture Anomaly \textit{SMA} or the Anomaly of Vegetation Condition \textit{FAPAR Anomaly} are based on soil moisture indicators and absorbed radiation fractions, respectively. By quantifying water flow volumina, the Low Flow Index \textit{LFI} belongs to the hydrological drought indices \autocite{europeandroughtobservatoryDroughtIndicators2017, svobodaHandbookDroughtIndicators2016}. In addition to these and other types of indices, such as Combined Drought Indices, the \textit{Handbook for Drought Indicators and Indices} lists over 50 drought indicators and indices. For further and more in-depth information, please refer to the interactive website of the \arcfull{IDMP} launched by the \acrfull{wmo} and \acrfull*{gwp} \autocite{idmpIndicatorsIndicesIntegrated2021}. 

All of these drought indices give a good impression about the physical side of climate anomalies, but none of the above mentioned indices link those climate anomalies to socioeconomic vulnerabilities \autocite{enenkelWhyPredictClimate2020}. \autocite{mishraWaterSecurityChanging2021} argue, that the framing of water security challenges extends beyond singular indicators. \autocite{lackstromBackyardHydroclimatologyCitizen2022} argue further, that assessments that only consider physical factors overlook the broader impact of drought on social, economic, and ecological systems.
The simple but widely used Falkenmark Indicator (Falkenmark et al. 1989) incorporates human factors by calculating a ratio between the given amount of water and the number of people living within that domain. By further categorizing this ratio to a level of water scarcity, the Falkenmark Indicator indicates the supply sides effects of water scarcity but variabilities, demand and socioeconomic factors are not represented. More dedicated indices like the \acrfull*{iwmi} Indicator and the \acrfull*{wpi} as well as other indices measuring water security give a more extensive representation of the overall situation \autocite{arreguin-cortesMunicipalLevelWater2019,liuWaterScarcityAssessments2017}. The \arcshort{wpi} for example represents the weighted average of five pre-standardized components namely, water availability, access, capacity, use and environment \autocite{sullivanWaterPovertyIndex2003}.

Determining the right set of indicators and indices for a given region to e.g. assess hazard severity depends on the objective and available data and is often a balancing act between many factors and circumstances \autocite{svobodaHandbookDroughtIndicators2016}. Besides the pure description of what certain natural or social circumstances \textit{are}, there is an growing interest to understand what these conditions will \textit{do} \autocite{boultDroughtImpactbasedForecasting2022, lackstromBackyardHydroclimatologyCitizen2022}.
The effects of these conditions on the ground are most often called the \textit{impact} of a certain weather phenomenon or climate development such as a drought hazard. Impacts can be direct or indirect and a generally difficult to quantify economically \autocite{vereintenationenSpecialReportDrought2021}. The level of impact is commonly determined based on the severity of the hazard, the exposure of the investigated elements and their respective vulnerabilities \autocite{harrowsmithFutureForecastImpact2020,svobodaHandbookDroughtIndicators2016,vereintenationenSpecialReportDrought2021}.
This concept is generally expressed by the risk equation

        \[Risk = f(Hazard, Exposure, Vulnerability)\]

    where

        \[Vulnerability = f(Level of Coping Capacity, Level of Adaptive Capacity)\]

\autocite{boultDroughtImpactbasedForecasting2022,harrowsmithFutureForecastImpact2020,vereintenationenSpecialReportDrought2021}. Drought hazard can be evaluated and described by the above mentioned indicators and indices with difficulties lying in the contextualization and setting of the threshold levels to separate between fluctuations within the normal range and extreme events. Exposure is commonly defined as social, economic, cultural or natural assets, services or resources in places that could be adversely affected by a hazard \autocite{ipccClimateChange20142014}. Exposed elements can be more ore less vulnerable to the hazard. Vulnerability conditions are determined by the sensitivity or susceptibility of a system, community or individual to physical, social, economic or environmental factors or processes \autocite{ipccClimateChange20142014}. These conditions are often further described as the level of coping and adaptive capacities. Coping capacities refer to available skills and resources of systems, organizations or individuals to address, manage and overcome unfavourable circumstances \autocite{ipccGlossaryTerms2012}. In the same manner, adaptive capacities relate to preparation, reduction and moderation of those impacts.

The establishment of a functional relationship between the hazard, exposure and vulnerability to its impact can be rather difficult for numerous reasons and is further discussed by \autocite{boultDroughtImpactbasedForecasting2022} for interested readers. Moreover, all these factors change over time, so that the quality of the calculations depends strongly on the timeliness of the data basis \autocite{harrowsmithFutureForecastImpact2020}. 

Relatively recent approaches argue for numerous benefits and reasons for greater inclusion of local knowledge and community integration in these approaches \autocite{balehegnIndigenousWeatherClimate2019,dubeFrameworkIntegrationTraditional2016,ebhuomaFrameworkIntegratingScientific2020,giordanoIntegrationLocalScientific2013a,greyIntegratingLocalIndigenous2020,hermansExploringIntegrationLocal2022a,mercerCultureDisasterRisk2012,mutasaKnowledgeApartheidDisaster2015,nyetanyaneIntegrationIndigenousKnowledge2020,nyongValueIndigenousKnowledge2007}. Another emerging area in scientific interest is the gender inequality of drought impacts \autocite{acharyaWhenRiverTalks2019,fanningDroughtDisplacementLivelihoods2018,hiwasakiLocalIndigenousKnowledge2015,mustafaGenderingFloodEarly2015,sachsRoutledgeHandbookGender2020,saniGenderOtherVulnerabilities2022}. Although these topics are of great interest, they fall largely outside the scope of this particular work.

An understanding of the severity of droughts and their current impacts enables targeted responses, as well as to allow for the development of future predictions based on current conditions. In this context, recent efforts have increasingly emphasized proactive and forward-looking measures in disaster relief initiatives. The forthcoming chapter will explore this relatively recent shift in approach and its implications for improving drought management strategies.


\section{FbF, EAP, AA & Early Warning + trigger}

Traditionally, disaster management efforts have primarily focused on long-term preparedness or post-disaster response, thus only providing assistance and relief to affected communities after a disaster has occurred (TODO: policy overview, hyogo framework [UNISDR], coughland et al 2015). The lack of standardized procedures for forecast-based actions led to disaster warnings often going unheard \autocite{kolenImpactsStormXynthia2013}. In the context of increasing frequency and severity of natural disasters, coupled with the impacts of climate change, the need for a more proactive approach that can reduce the impact of disasters on vulnerable communities became apparent \autocite{coughlandeperezForecastbasedFinancingApproach2015,trisosAfrica2022}. Nonetheless, financial resources were for the time being strongly directed towards post-disaster response and incentives to invest in new and complex scientific developments including relatively high uncertainties were limited \autocite{coughlandeperezActionbasedFloodForecasting2016}. This changed with the development and successful integration of several new forecast-based financing systems that utilized the opportunity gap between a forecast and the disaster to successfully reduce corresponding impact. Based on this, to "substantially increase the availability of and access to multi-hazard early warning systems and disaster risk information and assessments to people by 2030" became one of seven global targets of the Sendai for Disaster Risk Reduction 2015-2030 framework \autocites{coughlandeperezActionbasedFloodForecasting2016}[12]{undrrSendaiFrameworkDisaster}. Today, large institutions have now specialized sections for the financing of Early Actions such as the Climate Risk and Early Warning Systems Initiative \textit{(CREWS)} and the Global Risk Financing Facility \textit{(GRiF)} to support and backup \acfp{ea} \autocite{crewsClimateRiskEarly,GlobalRiskFinancing}. Forecast-based Financing \textit{(FbF)} has thus emerged as a promising approach to disaster management that enables proactive, timely, and cost-effective responses to disasters \autocite{coughlandeperezForecastbasedFinancingApproach2015} (TODO: add “FORECAST-BASED FINANCING An innovative approach” ([pdf](zotero://open-pdf/groups/4773535/items/3C2CE7BS?page=1&annotation=UKWEKCTA))).

The \acrfull*{ifrc} together with the \acrfull*{rccc} and \acrfull*{grc} have developed and improved the FbF programme to fund EAs since 2007 \autocite{ifrcForecastbasedFinancingNew2019}. 

\missingfigure[options]{“Figure 1 - FbF Diagram” ([RCRC, 2020, p. 3](zotero://select/groups/4773535/items/UESIQTRJ)) ([pdf](zotero://open-pdf/groups/4773535/items/P5JPVZ97?page=3&annotation=W3UC7H26))}

Following \autocite{coughlandeperezForecastbasedFinancingApproach2015, coughlandeperezActionbasedFloodForecasting2016} the structure of FbF can be distilled down to:
    “When forecast states that an agreed-upon probability threshold is exceeded for a hazard of a designated magnitude, then an action with an associated cost must be taken that has a desired effect and is carried out by a designated organisation.” \autocite[2]{coughlandeperezActionbasedFloodForecasting2016}.
Thus, the FbF approach involves three key components (1) triggering (2) pre-defined EAs and securing a (3) financing mechanism in advance (compare \ref{TODO: figure fbf}) (TODO: “Forecast-based Financing A new era for the humanitarian system” ([pdf](zotero://open-pdf/groups/4773535/items/KQZXSWVN?page=1&annotation=3BW2ZYST))). These components are summarized in an Early Action Protocol \textit{(EAP)} (TODO: cite policy overview “These three components are summarized in Early Action Protocols (EAPs).” ([“Forecast-based financing: A policy overview”, p. 2](zotero://select/groups/4773535/items/35XBEGJ7)) ([pdf](zotero://open-pdf/groups/4773535/items/8YZAQB5L?page=2&annotation=58UQZK6T))). 

\missingfigure{“FBF has three” ([pdf](zotero://open-pdf/groups/4773535/items/K89MIG2V?page=3&annotation=PF946AET))}

%-----------------------------------
%	SUBSECTION 4.1 Early Action Protocol
%-----------------------------------

\subsection{EAP and drought specifics}

In the \acrfull*{eap} triggers, actions to be taken and financing mechanisms are clearly outlined, thus summarizing and explicitly assigning responsibilities to the involved actors, ensuring that everyone understands their role and task in the event of activation \autocite{ruthForecastbasedFinancingPolicy2017}. This results in clear accountability and full commitment from all stakeholders, facilitating the timely and efficient implementation of the predetermined actions \autocite{ruthForecastbasedFinancingPolicy2017}.
Two types of analyses, namely the identification of forecasts and the risk assessment, form the basis for specifying the trigger, affected regions, and selected actions in the \acshort*{eap} (see Figure \ref*{TODO:}). 

\missingfigure{“Figure 2 - EAP Validation Steps” ([RCRC, 2020, p. 4](zotero://select/groups/4773535/items/UESIQTRJ)) ([pdf](zotero://open-pdf/groups/4773535/items/P5JPVZ97?page=4&annotation=6WWVYXLT))}%TODO:

Both assessments are primarily based on historic data and experiences. To identify suitable forecast(s), various forecasts are compared and analysed in terms of their capacities and performance to predict hazards. This is done mainly through a historically grounded analysis. Ultimately, a specific impact threshold based on one or a combination of several impact-based forecasts becomes the basis for triggering actions. This trigger also depends on the outcome of the risk assessment, as the impact of the hazard is highly influenced by the risk on site \autocite{ifrcFbFPractitionersManual2023,ifrcForecastbasedFinancingNew2019}.

The risk assessment is a complex analysis that takes numerous factors on scales of the hazard, and its sub-hazards, exposure, vulnerability and together with its coping and adaptive capacities, into account \autocite{ifrcFbFPractitionersManual2023}. Potential inputs depend strongly on the respective hazard and can range from records of historical events, housing location and building structures in the case of hurricanes and floods to social factors like income, demographics and school attendance. The objective being the identification of corresponding impact levels, thus determining the most effective actions and allocating resources as objectively as possible. Nonetheless, most of these parameters are proxies, as direct information about localized impact is seldom, outdated, of low accuracy or quality \autocite{ifrcFbFPractitionersManual2023}.

Due to the majority of the implemented \acrshortpl{eap} concentrating on fast onset disasters such as floods, hurricanes or strong rains, the FbF concept were primarily focussed and developed in this regard. Here, typically a sole trigger and its associated set of actions are established, emphasizing rapid responses, given that there is often less than 48 hours between the activation and the occurrence of the disaster. \autocite{rcrcFORECASTBASEDFINANCINGEARLY2020}. Drought as a usually slow-onset hazard, on the other hand, pose unique structural challenges to the process of determining thresholds to trigger actions as impact builds up over time and is highly dependent on the context \autocite{boultDroughtImpactbasedForecasting2022}. These challenges of identifying a forecast, determining a trigger and seclting actions are further outlined in the coming chapters.

%TODO: checken ob das auch wirklich so ist..
The specification of the financing mechanism as one of the three key components will not be covered in any further detail in this work, as the \acrshort{ifrc} has extended their Disaster Relief Emergency Fund \textit{(DREF)} with Forecast-based Action as dedicated mechanism to adequately support their increased numbers of FbF projects. Once the forecast-based trigger is met and the EAP is activated, the financing mechanism automatically assigns resources, which solves the issue of financing to a large extent and is therefore no longer of great interest to this piece of work.

%-----------------------------------
%	SUBSECTION 4.2
%-----------------------------------

\subsection{Forecasts}

Indicators and indices as discussed in chapter \ref*{subsec:indicators} measure the severity, duration and spatial coverage of hazard conditions based on historical and current weather data. They provide a snapshot of current conditions and serve as an indicator of the overall situation. Forecasts, on the other hand, use these indices together with climate models and weather data to predict future conditions and provide early warning of potential hazard events. Thus, forecasts extend the retrospective and current measures of indices to future prediction.
Similar to the indices, a single forecast usually only covers certain facets of a hazard. In the case of droughts, the thematic orientation commonly follows its definition classification into meteorological, hydrological and agricultural subdivisions. Furthermore, forecasts can additionally be categorized into global, continental or regional spatial scales with coarser scaling predictions mostly correlating with longer time spans and vice versa \autocite{baltiReviewDroughtMonitoring2020}. Global to continental meteorological drought forecasts with the focus on seasonal or inter-seasonal predictions are often based on same scale phenomenons such the Julian-Madden Oscillation, the ENSO cycles or the Indian Ocean Dipole \autocite{andersonMaddenJulianOscillationAffects2022,goreUnderstandingInfluenceENSO2020,yuanInfluencesIndianOcean2008}. These conditions are mostly collected through satellite and weather data often utilizing drought indices such as the SPI, SPEI and EDDI indices \autocite{kimIntegratedDroughtMonitoring2021}. Further drought prediction services such as the National Integrated Drought Information System of the US government, the European Drought Observatory (EDO) or its adaptation, the East African Drought Watch, utilize a wide range of different indices to predict hazard development and their impacts \autocite{europeandroughtobservatoryDroughtIndicators2017,icpacDroughtIndicators2023, nidisOutlooksForecasts2023}. These institutions also produce timely forecasts, but their data sources are usually based on the same remote evaluations mostly predicting what the weather and climate will be, and not what its implications on the ground will look like \autocite{enenkelWhyPredictClimate2020}. 

The transition to impact-based forecasts represents a radical shift in the way these forecasts are produced and opperationalized \autocite{ifrcFbFPractitionersManual2023}. Practically, this would change the information that a forecast would provide from predicting e.g. precipitation patterns to e.g. the magnitude and spatial coverage of crop failure \autocite{harrowsmithFutureForecastImpact2020}. The challenges of functional relationships, complex interconnected cause and effect networks and data availability mentioned in chapter \ref*{indicator} are also applicable here, but the change to impact-based information results in multiple benefits to practitioners nonetheless. Impact-based forecasts help with the identification and prioritization of areas and communities most severely impacted. They do this by supporting a transparent, evidence-based, sector- and context-specific decision-making process directly focussing the population at risk \autocite{ifrcFbFPractitionersManual2023}.

\autocite{boultDroughtImpactbasedForecasting2022} argue even further for an adapting and dynamic impact assessment process, as decadal shifts in climate variabilities, changing exposure and vulnerabilities are not incorporated in a pre-defined system. They propose a hybrid framework of multi-hazard forecasts interlinked with static vulnerability and dynamically adjusted with real-time expert vulnerability assessments. Threshold triggers are lower, where static vulnerabilities are higher. However, both the regular pre-defined impact forecast and the dynamic impact forecast must be preceded by a selection and definition of triggers and actions.

\subsection*{Trigger definition}

“Triggers are mainly combination of hydro-meteorological forecast combined with exposure and vulnerability data” \autocite[19]{rcrcFORECASTBASEDFINANCINGEARLY2020}. There are commonly two ways to define a trigger for early actions. On one side, triggers can be consensus-based, meaning experts make real-time judgements by synthesizing information from multiple sources, or on the other side, triggers are data-driven, peer-reviewed and validated well in advance of a potential event \autocite{rcrcFORECASTBASEDFINANCINGEARLY2020}. Drought with its different layers of complexity may also benefit from a combination of these mechanisms, as e.g. the framework of \autocite{boultDroughtImpactbasedForecasting2022} proposed above shows. Generally, good conditions for effective trigger development are sufficient historical data, knowledge about local livelihoods and how diverse parts of communities are influenced differently, thorough identification of differentiated impact drivers and their correlation to magnitudes as well as trustworthy forecasts \autocite{coughlandeperezForecastbasedFinancingApproach2015,coughlandeperezActionbasedFloodForecasting2016,elisabethstephensFORECASTBASEDACTION2015,harrowsmithFutureForecastImpact2020,rcrcFORECASTBASEDFINANCINGEARLY2020}. 
Furthermore, the framing and definition of the underlying forecast, indices and indicators are paramount as data-driven triggers are "specific values of an indicator or index that initiate and/or terminate each level of a drought plan and associated mitigation and emergency management responses.” \autocites{rcrcFORECASTBASEDFINANCINGEARLY2020}[13]{svobodaHandbookDroughtIndicators2016}. This specification is highly context specific and e.g. in the case of flood can be defined as the level when the river breaches its banks and inundates the surrounding area. Though, in another area this overflow may only inundate open space and thus lead to no impact at all \autocite{elisabethstephensFORECASTBASEDACTION2015}. This circumstance is relatively easy to grasp, has a single trigger and one set of specified actions such as evacuation, transportation and early warning and is therefore well integrable and implementable (see upper illustration in figure \ref*{TODO: RCRC p.20 Figure 5}) \autocite{siahaanForecastbasedActionDREF2018}. 

\missingfigure{RCRC p.20 Figure 5}

Drought, due to its slow-onset and potentially cascading impacts that only builds up over time complexifies the process of trigger definition as \acrfullpl{aa} to some impacts may go hand in hand with active responses in some areas and be to early in others. Furthermore, forecast certainty, granularity and accuracy all decrease the more one looks into the future \autocite{rcrcFORECASTBASEDFINANCINGEARLY2020}. Deciding when to trigger is therefore a critical and challenging aspect of conceptualizing a drought \acrshort{eap} (see bottom illustration in figure TODO: RCRC p.20 Figure 5). Practitioners and experts interviewed by the \autocite{rcrcFORECASTBASEDFINANCINGEARLY2020} advocate for a staggering triggering system \autocite. Here, multiple triggers with different sets of \acrshortpl{aa} would extend the single trigger mechanism and give the opportunity to account for the different phases and the inherent complexity of the phenomenon drought. Moreover, the \autocite[30]{rcrcFORECASTBASEDFINANCINGEARLY2020} calls for the development of "unconventional triggers for \acrfull*{FbA}" as the trigger development is not yet complete.

\subsection{Anticipatory Actions}

% keep it concise. It is not complicated. No reason to blow it up.

Anticipatory Actions are at the heart of every EAP and their execution is what everything is working towards. The goal of every Anticipatory Action is to help people and communities at risk to reduce negative impacts of a hazard. The final execution is preceded by some conceptual and practical steps. The establishment process begins with the identification of contextually meaningful, suitable and locally realisable actions with special focus on stakeholders, resources and available lead-time. These are further prioritized and selected based on the risk assessment, type and magnitude of hazard, and forecasting capabilities. When a first set of \acrshort*{aa} is defined, they are worked through in detail, reflected on with stakeholders and ultimately finalised. Together with an evaluation phase, this process is often a simultaneous and iterative process which also does not stop with the operationalisation of the \acrshort{eap} \autocite{elisabethstephensFORECASTBASEDACTION2015,ifrcGlossaryTermsForecastbased2023,ifrcFbFPractitionersManual2023a,rcrcFORECASTBASEDFINANCINGEARLY2020}.
In practice, \acrlongpl{aa} are commonly split into a preparation and an activation phase. The preparation phase builds on the process described above, but also extends to actions that prepare for rapid activation, such as the prepositioning of water tablets before the rainy season \autocite{elisabethstephensFORECASTBASEDACTION2015}. The activation phase requires a constant operation of forecast monitoring and is initiated when the trigger is reached. Timely information dissemination, releasing and receiving funds, implementing of the \acrshortpl{aa} and subsequent evaluation are part of this phase \autocite{elisabethstephensFORECASTBASEDACTION2015,ifrcFbFPractitionersManual2023a}. Often, \acrshortpl{aa} are not very different from response actions except of their predictive and proactive nature. However, this foresight comes with the cost of uncertainty and forecasts may not always be accurate. The simultaneous implementation of \acrshortpl{aa} in the absence of the disaster is commonly referred to as \textit{to act in vain} \autocite{coughlandeperezForecastbasedFinancingApproach2015}. Besides financial costs, this may also manifest in reputational costs in e.g. the case of Early Warning and evacuation if false alarms occur too frequently \autocite{elisabethstephensFORECASTBASEDACTION2015}. Albeit, a growing body of evidence suggests that the benefits of AAs outweigh the costs substantially \autocite{cabotventonEconomicsResilienceDrought2018,coughlandeperezForecastbasedFinancingApproach2015,gualazziniEWEAEarlyWarning2021}. Furthermore, the issue of \textit{acting in vain} can be lessen by staggering triggers and adjusting AAs in accordance with long-term resilience building \autocite{wfpMonitoringEvaluationAnticipatory2021}. This can allocate the actions more precisely and increases the general benefits. \autocite{ifrcGlossaryTermsForecastbased2023} makes these design adjustments the basis of its definition of \textit{acting in vain} and thus argues for the abolition of this term, since the benefits of acting should always outweigh not acting at all.


\section{Citizen Science, Crowdsensing, Volunteersensing, VGI,  alternatively satellite image interpretation}

The inclusion of local knowledge in the system of Early Warning and Anticipatory Action can result in many benefits as already mentioned in the end of chapter \ref*{subsec:indicators}. Adapting knowledge and policies to local conditions and people as well as learning from them, strengthening autonomous responses and involving local stakeholders in all stages of the processes are just some of the potential ways to improve implementations \autocite{giordanoIntegrationLocalScientific2013a,idmpDroughtWaterScarcity2022,lackstromBackyardHydroclimatologyCitizen2022,lealfilhoRoleIndigenousKnowledge2022,lealfilhoUnderstandingResponsesClimaterelated2022}. One way to include local knowledge is through Citizen Science, very broadly defined as "public participation in scientific research and knowledge production" \autocite{fraislCitizenScienceEnvironmental2022} .
Historically, the first citizen science project was possibly the Christmas Bird Count run by the National Audubon Society in the USA every year since 1900 \autocite{linkHierarchicalModelRegional2006,silvertownNewDawnCitizen2009}. Since around 2000, the number of publications in regard to Citizen Science has risen substantially and has established itself as a vibrant area of scientific interest \autocite{kirschkeCitizenScienceProjects2022}. As more and new thematic fields joined this area of interest, numerous approaches have been made to define Citizen Science more precisely \autocite{haklayWhatCitizenScience2021}. Over 30 definitions were selected by \autocite{haklayWhatCitizenScience2021} to explore their ambiguity and extend the best practice principles and characteristics of citizen science established by the European Citizen Science Association (ESCA) \autocite{escaTenPrinciplesCitizen2015,escaECSACharacteristicsCitizen2020}. Different political, scientific or societal lenses along with a variety of focal points such as (1) biology, conservation and ecology, (2) geographic data and (3) social sciences and health related issues have all contributed to the concept of Citizen Science \autocite{haklayWhatCitizenScience2021,kirschkeCitizenScienceProjects2022,kullenbergWhatCitizenScience2016}.
The first, natural research and conservation, is the orientation most frequently related to Citizen Science with overlapping concepts to community-based, volunteer and participatory monitoring. It has common interests with the second category of Volunteered Geographic Information (VGI) in topics such as crowdsourcing and data quality whereas the the third category mostly resolves around public engagement with intersections to CS in public participation \autocite{kullenbergWhatCitizenScience2016}. In order to highlight the core of Citizen Science alongside the different disciplinary orientations of the research, different frameworks, guidelines and levels of participation have been designed.\autocite{kirschkeCitizenScienceProjects2022} created a three cluster framework of design principles around \textit{citizen} and \textit{institutional} characteristics, together with their \textit{forms of interaction}. Within these categories \autocite{kirschkeCitizenScienceProjects2022} highlight various qualities and skills such as age, social status, motivation, knowledge and education of the contributing citizens, financial and human resources on the institutional side and the method and density of communication and feedback practices as important parts of interactions. Guidelines and principles further specify, expand and structure these broad topics to make them practically applicable in various contexts \autocite{citizenscience.govBasicStepsYour,escaTenPrinciplesCitizen2015,escaECSACharacteristicsCitizen2020,EUCitizenScience2023,fraislCitizenScienceEnvironmental2022,garciaFindingWhatYou2021,minkmanCitizenScienceWater2015,pocockStrategicFrameworkSupport,skarlatidouWhatVolunteersWant2019}. Citizen science projects can also be differentiated according to how engagement with participants is designed. This is referred to as the \textit{levels of participation} and is commonly structured into four levels. Increasing in participation intensity, \autocite{buckinghamshumGlobalParticipatoryPlatform2012} categorize them into (1) Crowdsourcing, (2) Distributed Intelligence, (3) Participation Science and (4) Extreme Citizen Science. Following this categorization, participants can be (1) 'sensors', (2) 'interpreters', (3) engaged in problem definition and data collection or even (4) part of the analysis. 
Depending on the level of participation and thematic orientation, Citizen Science is related to concepts of classic monitoring practices (1), transdisciplinary research emphasizing engagement of the public along the entire process (2 \& 3) and participation involving "groups that are or perceive themselves as being affected by the decision" (3 \& 4) \autocites{buckinghamshumGlobalParticipatoryPlatform2012}{conradReviewCitizenScience2011}{minkmanCitizenScienceWater2015}[1]{rennParticipatoryProcessesDesigning2006}. 
Current challenges and limitations in Citizen Science projects are the complex demands in the conceptualization and design process with a wide range of required skills and resources, recruiting participants and sustaining their motivation, data quality and accuracy considerations, biases in collection and analysis as well as privacy regulations \autocite{fraislCitizenScienceEnvironmental2022}. Furthermore, research and CS projects are currently unevenly distributed on a global scale with an over representation of North American countries resulting in less experiences and guidelines for other areas and contexts \autocite{kirschkeCitizenScienceProjects2022, zhengCrowdsourcingMethodsData2018}. Nonetheless, numerous studies suggest promising developments and application possibilities addressing all of the above mentioned challenges in design, participants and data related issues \autocite{buckinghamshumGlobalParticipatoryPlatform2012,buddeParticipatorySensingParticipatory2017,escaECSACharacteristicsCitizen2020,fraislCitizenScienceEnvironmental2022,lowryGrowingPainsCrowdsourced2019,pocockStrategicFrameworkSupport,ruttenHowGetKeep2017,weeserCitizenSciencePioneers2018a}. 
% come back to this in the discussion --> data quality e.g. can be 'solved' by trainings and supervision

\subsection{Community-based monitoring}\label{subsec:cbm}

% maybe add that there are more related concepts: participatory monitoring instead? or: citizen observatories, community based monitoring and participatory monitoring

\acrfull{cbm} is a sub-concept of citizen science and can be allocated to different layers of participation, depending on its definition, aspects and final implementation \autocite{westonCommunityBasedWaterMonitoring2015}. \acrshort{cbm} can encompass "a process where concerned citizens, government agencies, industry, academia, community groups and local institutions collaborate to monitor, track and respond to issues of common community concern" \autocite[410]{whitelawEstablishingCanadianCommunity2003}. The focus of \acrshort*{cbm} on monitoring is fundamental, but the monitored subject, further handling of the data and the involvement of the participants can vary widely \autocite{baptisteCommunityLedMonitoringWhen2020,conradReviewCitizenScience2011,koehlerCitizenParticipationCollaborative2008,muhamadkhairCommunitybasedMonitoringEnvironmental2021,shirkPublicParticipationScientific2012,westonCommunityBasedWaterMonitoring2015}. Within this work, \acrshort*{cbm} is understood as a combination of two main aspects. The collection part often refers to concepts of \textit{Crowdsourcing} or \textit{Crowdsensing} (see next Chapter \ref{subsec:mcs}) and a management aspect which promotes the incorporation of the generated information into community decision-making processes \autocite{conradCommunitybasedMonitoringScience2007, keoughAchievingIntegrativeCollaborative2006}. 
\acrlong*{cbm} can serve many purposes but its implementation and application is not always recommended. Therefore, many guidelines precede the design with an assessment of the feasibility of this approach \autocite{associationTenPrinciplesCitizen2015,citizenscience.govBasicStepsYour,fraislCitizenScienceEnvironmental2022,minkmanCitizenScienceWater2015, pettiboneCitizenScienceAll2016}. Here, the challenges, benefits and capabilities of the \acrshort*{cbm} approach are compared with the problem and core objectives of the project. It is emphasized that \acrshort*{cbm} should not be the goal itself, but only a means to fulfil the project goals \autocite{minkmanCitizenScienceWater2015}. Nonetheless, the diversity of this approach means that other goals can be pursued and achieved apart from the main interests (see Chapter \ref*{subsec:guidelines}). For example, enriching participants by addressing their needs, advancing their knowledge or teaching them new skills is considered as fundamental and important to achieving the main objective as it is to a successful project \autocite{fraislCitizenScienceEnvironmental2022}. 
In the following, a short overview about challenges, benefits and recommendations of \acrshort*{cbm} is given, broken down in the design phase, incorporation of participants and data concerns.

%%%%%%%%%%%%%%%%%%%%%%%%%%%
% possibly subsubsec ??
% design 
The conceptualization of CBM projects on the level of participation or the tripartite division according to characteristics of citizens, institutions and their forms of interaction have already been mentioned in connection with the broader concept of Citizen Science and are also applicable here. More concrete design factors and variables were synthesized by \autocite{kirschkeCitizenScienceProjects2022} but the systematic understanding of their influences on the success of remained unclear for now. A selection of subjects outside of the original research itself could be overall project management, communication in its various forms and with all stakeholders, community and participant recruitment, training and management, data management and analysis as well as the final implementation and operation of the project. Moreover, there is agreement that no \textit{one-size-fits-all} solution exists and different goals, resources, and contexts have considerable influence on the design from project to project \autocite{fraislCitizenScienceEnvironmental2022}. In order to account for the variety of challenges and to maximize the benefits, staged frameworks have been developed to guide the design \autocite{citizenscience.govBasicStepsYour, fraislCitizenScienceEnvironmental2022,garciaFindingWhatYou2021,minkmanCitizenScienceWater2015}. Yet, these frameworks can be relatively coarse and imprecise and are often partly tailored to specific goals and contexts, making a combination of several such frameworks and the inclusion of further guidelines and recommendations potentially necessary to tailor the design to the specific situation. 

%%%%%%%%%%%%%%%%%%%%%%%%%%%%%%%
% participants
Participants can take many roles in a \acrshort*{cbm} project based on the level of participation chosen but regardless of this, their adequate integration is seen as a cornerstone of any \acrshort*{cbm} project \autocite{land-zandstraParticipantsCitizenScience2021}. Knowledge and skills as well as other socio-economic variables can vary widely between participants and it is important to account for this to inspire and keep participants motivated to contribute \autocite{minkmanCitizenScienceWater2015,whitelawEstablishingCanadianCommunity2003}. One mayor drawback of online collaborative initiatives is often that a considerable proportion of contributors only participate once and with minimal effort while a relatively small number of participants are responsible for the majority of the work \autocite{sauermannCrowdScienceUser2015}. Understanding and thus sustaining the motivation of participants is therefore central to a successful project. The subject of what drives individuals to participate in citizen science projects has been extensively explored in literature \autocite{land-zandstraParticipantsCitizenScience2021,minkmanCitizenScienceWater2015,mloza-bandaCrowdsensingSuccessfulWater2018,ruttenHowGetKeep2017,tipaldoCitizenScienceCommunitybased2017,walkerBenefitsNegativeImpacts2021a,walkerBenefitsNegativeImpacts2021}. Motivation can be intrinsic or extrinsic and spans from the will to contribute to science and conservation over meeting and helping other potentially like minded people to learning new skills and financial compensation \autocite{minkmanCitizenScienceWater2015,rotmanDynamicChangesMotivation2012 ruttenHowGetKeep2017}. According to \autocite{rotmanDynamicChangesMotivation2012}  study, egocentric motives tended to drive new participants, whereas established participants were more motivated by altruistic reasons, such as helping others. Furthermore, the individual adaptation of the task's difficulty to each participant was suggested to positively influence motivation in order to neither bore nor overwhelm \autocite{minkmanCitizenScienceWater2015}. Other factors to inspire and sustain motivations are, among others, the expected benefits, acknowledgement and feedback culture and its perceived usefulness and integration into further processes \autocite{land-zandstraParticipantsCitizenScience2021,minkmanCitizenScienceWater2015,pettiboneCitizenScienceAll2016}. In addition to strengthening motivation, breaking down barriers to participation can also prove helpful. For this, understanding the background and circumstances of the participants is important. In their work for hydrological monitoring in Kenya, \autocite{weeserCitizenSciencePioneers2018a} could partly attribute low participation rates to the transmitting costs of 0.01 USD per text message at some station. Offsetting these costs could subsequently increase the overall participation rate significantly. \autocite{weeserCitizenSciencePioneers2018a} further discovered, that actual compensation or incentives appeared unnecessary as the intrinsic motivation of the participants proved to be adequate once financial constraints were addressed. Besides financial and resource restrictions, lack of knowledge and skills can be addressed by providing adequate training \autocite{fraislCitizenScienceEnvironmental2022,lackstromBackyardHydroclimatologyCitizen2022}.

% data
Supervision, external or mutual feedback and preceding training of participants can also address common data quality concerns \autocite{albusAccuracyLongtermVolunteer2020,baalbakiCitizenScienceLebanon2019,fraislCitizenScienceEnvironmental2022}. Besides the characteristics of the participant, the difficulty of the measurement task itself influences the quality. Simpler tasks such as gauging e.g. water levels provided high data quality in \autocite{weeserCitizenSciencePioneers2018a} study. \autocite{baalbakiCitizenScienceLebanon2019} has further found that most of the data collected by citizen scientists is comparable to that of university scientists when it comes to chemical or physical qualities of water. \autocite{albusAccuracyLongtermVolunteer2020} could support this finding, by analyzing data from the Texas Stream Team (TST) citizen science program and found an agreement of 80\% up to 90\% for DO, pH and conductivity parameters. However, \autocite{baalbakiCitizenScienceLebanon2019} also noted a disparity in the bacteriological test results between citizen and university scientists, to which they remarked, that it may be explained by the complexity of the testing process and the quality of the testing materials employed. \autocite{aceves-buenoCitizenScienceApproach2015} evaluated over 80 peer-reviewed studies of which only 11\% reported no data accuracy issues but only one study reported, that the data was unusable. Based on the aforementioned findings, ensuring data quality and accuracy through appropriate quality assurance and control measures is crucial. However, despite the reliability and accuracy challenges associated with \acrshort*{cbm} data, \autocite{aceves-buenoCitizenScienceApproach2015} noted, that these issues typically do not have a significant impact on the data's overall usefulness.

Besides the more specific challenges and benefits mentioned above, \acrlong*{cbm} approaches can benefit scientists, decision-makers, communities and participants in multiple ways. In addition to achieving the main objectives, raising awareness of the issue, the needs and the problems at hand, as well as increasing knowledge among all project stakeholders, can lead to changes in behaviour, improved management, reduced risks and a better representation of local conditions in the regional, national and international context. \autocite{huangManagementDrinkingWater2020,walkerBenefitsNegativeImpacts2021}. Output quality can be enhanced when the objective is clear, participant involvement is recognized as a high priority, enough resources for design, implementation, operation and analysis are available and the monitoring protocol is not too complex \autocite{butteFrameworkWaterSecurity2022, pocockStrategicFrameworkSupport}. 
In an attempt to scale this concept across regions or even an entire country with many physical, social and economic differences, the \acrshort*{cbm} concept has been increasingly explored with mobile, network-enabled devices. This is, together with practical examples and projects, presented in the coming chapters.


% % \subsection{VGI} % keep it short or completely out of this -> will come up in CS anyway

% % “In the field of geography, the mapping of features such as buildings, road networks, and land cover can now be undertaken by citizens as a result of advances in Web 2.0 and global positioning system (GPS)-enabled mobile technology, which has blurred the once clear-cut distinction between map producer and consumer (Coleman et al., 2009).” ([Zheng et al., 2018, p. 703](zotero://select/groups/4773535/items/LJU68CG4)) ([pdf](zotero://open-pdf/groups/4773535/items/U8QNZLI6?page=6&annotation=W66QRY3C))

% % “In a seminal paper published in 2007, Goodchild (2007) coined the phrase Volunteered Geographic Information (VGI). Similar to the idea of crowdsourcing, VGI refers to the idea of citizens as sensors, collecting vast amounts of georeferenced data.” ([Zheng et al., 2018, p. 703](zotero://select/groups/4773535/items/LJU68CG4)) ([pdf](zotero://open-pdf/groups/4773535/items/U8QNZLI6?page=6&annotation=73UDWLM6))

% % “OpenStreetMap (OSM) is an example of a highly successful VGI application (Neis & Zielstra, 2014),” ([Zheng et al., 2018, p. 703](zotero://select/groups/4773535/items/LJU68CG4)) ([pdf](zotero://open-pdf/groups/4773535/items/U8QNZLI6?page=6&annotation=LJBW7EY4))

% %-----------------------------------
% %	SUBSECTION 6.2
% %-----------------------------------

% \subsection{local knowledge (???)} %only when I still have time there are just tons of information maaaan
% These \textit{unconventional triggers} could be based on local or indigeneous knowledge and data from the ground.

% “External stakeholders’ attitudes towards and engagement with local knowledge in 1 disaster risk reduction: are we only paying lip service?” ([Šakić Trogrlić et al., 2021, p. 1](zotero://select/groups/4773535/items/BTRX6EIG)) ([pdf](zotero://open-pdf/groups/4773535/items/52GEZZFV?page=1&annotation=TLIV6XS7)). 

% % --> local context
% “Deep understanding of the local context, and the needs and wants of the targeted community would allow us to identify which drought impacts are most strongly felt by different groups of the community.” ([RCRC, 2020, p. 28](zotero://select/groups/4773535/items/UESIQTRJ)) ([pdf](zotero://open-pdf/groups/4773535/items/P5JPVZ97?page=28&annotation=EI2UMB2H))

% -->
% “Where they exist, these systems may be even more important for slow-onset hazards like drought, necessary but not sufficient - layers of additional (ideally) local indicators must be added to these in order to form an appropriate FbA trigger.” ([RCRC, 2020, p. 29](zotero://select/groups/4773535/items/UESIQTRJ)) ([pdf](zotero://open-pdf/groups/4773535/items/P5JPVZ97?page=29&annotation=YK99RIZB))


% local on the ground impact assessment not possible with current forecast abilities --> possibly link to that via CBS and the feasibility study --> (“Early Warning/Early Action Mechanisms: EWEA is working well in cases of health emergencies/epidemics through community-based surveillance (CBS); this allows the N” ([Somali Red Crescent Society, 2022, p. 51](zotero://select/groups/4773535/items/FZ6BJHJA)) ([pdf](zotero://open-pdf/groups/4773535/items/RJKNZZZ2?page=51&annotation=4C3HL8ES))) 
% thus, this and comparable approaches are investigated in the next chapter

% % nonetheless, difficult to scale
% “The Problem of Scale in Indigenous Knowledge: a Perspective from Northern Australia” ([Wohling, 2009, p. 1](zotero://select/groups/4773535/items/HIFJDYSG)) ([pdf](zotero://open-pdf/groups/4773535/items/BUPU6DGS?page=1&annotation=LPCCJN8Z))

% different trigger --> different set of actions --> interdependent of what the actions should look like (can be an interative forth and back coming process)--> can also differ for different groups

% % local knowledge --> same as with indicators, forecasting is also possible on the basis of local knowledge -> overview table
% “Table 1. Comparisons between indigenous knowledge-based seasonal forecasts and seasonal climate forecasts (adopted from Ziervogel and Opere 2010). Indigenous knowledge-based seasonal forecasts Seasonal climate forecasts Use biophysical indicators of the environment as well as spiritual methods Use of weather and climate models of measurable meteorological data Forecast methods are seldom documented Forecast methods are more developed and documented Up-scaling and down-scaling are usually complex Up-scaling and down-scaling are relatively simple Application of forecast output is less developed Application of forecast output is more developed Communication is usually oral Communication is usually written Explanation is based on spiritual and social values Explanation is theoretical Taught by observation and experience Taught through lectures and readings” ([Masinde and Bagula, 2012, p. 280](zotero://select/groups/4773535/items/EW9XSSZP)) ([pdf](zotero://open-pdf/groups/4773535/items/3WQ4S9PE?page=7&annotation=6XCISBM2))

% “Indigenous knowledge within an early warning system for droughts” ([Masinde and Bagula, 2012, p. 282](zotero://select/groups/4773535/items/EW9XSSZP)) ([pdf](zotero://open-pdf/groups/4773535/items/3WQ4S9PE?page=9&annotation=8Z9A9AW8))

% “The Best of Both Worlds: A Decision-Making Framework for Combining Traditional and Contemporary Forecast Systems” ([Plotz et al., 2017, p. 1](zotero://select/groups/4773535/items/3SBLBZEA)) ([pdf](zotero://open-pdf/groups/4773535/items/VAUJGIFB?page=1&annotation=RRFY2UWE))

% “B. Drought Forecasting in Sub-Saharan Africa” ([Masinde and Thothela, 2019, p. 304](zotero://select/groups/4773535/items/6D52T883)) ([pdf](zotero://open-pdf/groups/4773535/items/KLLQKDG2?page=2&annotation=ZQSDUEMX))

% there are more! Look into it.!

% “Researchers ([1], [19] and [20]) today concur that IK and modern science weather forecasts complement each other;” ([Masinde et al., 2013, p. 2](zotero://select/groups/4773535/items/M45MLGWC)) ([pdf](zotero://open-pdf/groups/4773535/items/LG6E76P4?page=2&annotation=JHQV2GYT))


% “In theory, focusing on what the weather will do, rather than what the weather will be, enables decision makers to plan and implement targeted preparatory actions to better reduce hazard impacts (Harrowsmith et al., 2020).” ([Boult et al., 2022, p. 2](zotero://select/groups/4773535/items/B2AQSTYL)) ([pdf](zotero://open-pdf/groups/4773535/items/W9TFLH43?page=2&annotation=NSLE7NL6))

% “Improving early warning of drought-driven food insecurity in southern Africa using operational hydrological monitoring and forecasting products” ([Shukla et al., 2020, p. 1187](zotero://select/groups/4773535/items/TE5NMA3T)) ([pdf](zotero://open-pdf/groups/4773535/items/9TNUGXSJ?page=1&annotation=TLBHC7BS))

% “Moving from drought hazard to impact forecasts” ([Sutanto et al., 2019, p. 1](zotero://select/groups/4773535/items/EUC5RV7N)) ([pdf](zotero://open-pdf/groups/4773535/items/9DI9EVBF?page=1&annotation=FCQCKY5P))


% %--> even though it is generally not recommend by the RCRC for a National Society to collect these local indicators by themselves

% “It is important to note that local indicators cannot be collected specifically for the FbF system by RCRC national societies.” ([RCRC, 2020, p. 30](zotero://select/groups/4773535/items/UESIQTRJ)) ([pdf](zotero://open-pdf/groups/4773535/items/P5JPVZ97?page=30&annotation=LRDPV2M7))

% “Indeed, collecting data on local indicators would require from the national society a team of enumerators that work continually to collect and process that information in all places where the program could possibly trigger (e.g. collect food price information for every village market). This would have extensive cost implications and likely over-burden the national society staff and volunteers.” ([RCRC, 2020, p. 30](zotero://select/groups/4773535/items/UESIQTRJ)) ([pdf](zotero://open-pdf/groups/4773535/items/P5JPVZ97?page=30&annotation=2YIIK6ZY))

% “As such, the inclusion of local indicators into an FbA trigger must involve assessing what indicators are relevant for the impacts the program is trying to anticipate and identify which of those indicators are already collected (e.g. the ministry of agriculture's food price bulletin) and are available at the time they would be needed to inform a possible trigger.” ([RCRC, 2020, p. 30](zotero://select/groups/4773535/items/UESIQTRJ)) ([pdf](zotero://open-pdf/groups/4773535/items/P5JPVZ97?page=30&annotation=7X3RFGVB))


% “2 Local knowledge in drought monitoring: an introduction to the literature review” ([Giordano et al., 2013, p. 526](zotero://select/groups/4773535/items/B7LM5ZR4)) ([pdf](zotero://open-pdf/groups/4773535/items/7I66DBIK?page=4&annotation=Z33M5FLQ))

%----------------------------------------------------------------------------------------
%	SECTION 7 MCS & other tools + water related monitoring (excel)
%----------------------------------------------------------------------------------------

\subsection{Mobile Crowdsensing (MCS)}\label{subsec:mcs} % practical applications of MCS, CBS & other tools + water related monitoring (excel)

Originating in 2006 from an article by \autocite{howeRiseCrowdsourcing} and Mark Robinson describing Crowdsourcing as a new internet based business model in the terms of "It's not outsourcing; it's crowdsourcing", by harnessing "the creative solutions of a distributed network of individuals through what amounts to an open call for proposals" \autocite[76]{brabhamCrowdsourcingModelProblem2008}. Nowadays crowdsourcing in scientific contexts is often applied as e.g. act of "collecting data without a direct integration into the scientific process" by a generally large audience \autocite[1591]{weeserCitizenSciencePioneers2018a}. Due to the merely perceiving and transferring and not further interpreting character, \textit{Crowdsourcing} is on the lowest level of participation levels. A more specific form of \textit{Crowdsourcing} is \textit{Crowdsensing} which refers to the process of measuring and collecting data by a large mass of contributors that involves using mobile devices and/or sensors to collect information about the environment. This is also known as \acrfull{mcs} \autocite{guoParticipatorySensingMobile2014, liuSurveyMobileCrowdsensing2018}.
\acrshort*{mcs} is part of a widespread transition in the way data is gathered and managed, with a shift away from conventional methods towards incorporating mobile devices, web platforms, and apps \autocite{capponiSurveyMobileCrowdsensing2019, sanllorentecapdevilaSuccessFactorsCitizen2020}. This transition is being driven by the development and proliferation of information technology infrastructure, which includes the collection, sharing, storage, cleaning and analysis of data \autocite{fraislCitizenScienceEnvironmental2022}. These components of the information technology infrastructure can be grouped into a four-layer architecture which is described in detail in the paper by \autocite{capponiSurveyMobileCrowdsensing2019}.
The first and top layer is the \textit{application layer} concerned about high-level user, task- and overall design and organizational aspects with some examples being user recruitment's, scheduling and contribution management. The \textit{data layer} as the second layer refers to storage, processing and analysis of the received data and is followed by the \textit{communication layer} which refers to methodological and technological aspects of the reporting characteristics. These include cellular, internet or other networks and their means of transmission. The bottom layer, the centrepiece of this architecture, is the \textit{sensing layer} which includes all tools, technologies and equipment involved in the data acquisition process \autocite{capponiSurveyMobileCrowdsensing2019}. Measurements can be of different types, intentional or unintentional, at the occurrence of an event or continuous, and are based on human observation, instrumental measurements or a combination of both \autocite{zhengCrowdsourcingMethodsData2018}. In this architecture hierarchy, data flows generally from the lowest to the highest layer \autocite{aceves-buenoCitizenScienceApproach2015,capponiSurveyMobileCrowdsensing2019,zhengCrowdsourcingMethodsData2018}.
Besides generally applicable challenges of \acrlong{cbm} such as data quality and participant motivation, main challenges of \acrshort*{mcs} are seen in the socio-technical, privacy and security realms referring to hard- and software availability, reliability and usability as well as balancing access rights, anonymisation and encoding with data trustworthiness \autocite{aceves-buenoCitizenScienceApproach2015,alfonsoMOBILEPHONEAPPLICATIONS2012,capponiSurveyMobileCrowdsensing2019,liuSurveyMobileCrowdsensing2018, minkmanCitizenScienceWater2015, noureenCrowdsensingSocioTechnicalChallenges2017a}. Nonetheless, \acrshort*{mcs} also provides many opportunities and solutions to designers, operators and participants alike. Among those are the relatively good and easy scalability and increase of monitoring network density, low barriers for participation and two-way communication options as well as high potential for automatization and interoperability with other applications and frameworks \autocite{alfonsoMOBILEPHONEAPPLICATIONS2012,minkmanCitizenScienceWater2015,sanllorentecapdevilaSuccessFactorsCitizen2020,weeserCitizenSciencePioneers2018a}. In the following, practical examples of \acrshort*{cbm} and \acrshort*{mcs} or a combination of both are presented, highlighting the wide-ranging application possibilities. %together with their advantages and disadvantages.

\subsection*{Practical Examples of CBS and MCS}

The potential applications for \acrshort*{mcs}, embedded in \acrshort*{cbm} or as a stand-alone project, are, as for all Citizen Science, wide-ranging and diverse. Besides the thematic diversity, the socio-technical implementation, size and complexity can differ substantially from project to project. Established networks like the \acrfull*{cocorahs} founded in 1998 USA with over 25.000 observers facilitate the collection of daily weather observations and the sharing of written impact impressions via an online platform \autocite{cocorahsCoCoRaHSCommunityCollaborative2023,lackstromBackyardHydroclimatologyCitizen2022}. The Audubon's Christmas Bird Count (CBC) even goes back to the December of 1900 and in its 120th anniversary year over 81.000 observers counted more than 30 million individual birds \autocite{lebaron122ndChristmasBird2022}. Another major project in the realm of \textit{Crowdsourcing} and \acrshort*{mcs} is the 2004 founded OpenStreetMap Foundation. Started as a reaction to the failed release of geographic information in the United Kingdom, OSM as a collaborative community effort quickly became one of the most important sources of geographic information world wide \autocite{bennettOpenStreetMap2010, openstreetmapcontributorsOpenStreetMapBasemap2020}. Additional contemporary developments include the concept of \acrshort*{mcs} in citizen participation, Smart Cities, resource management, transport and behaviour evaluation and many more \autocite{dipasDIPASOrgDIPAS2023,europeancommissionCitizencentredApproachSmart2021, wangSurveyApplicationKey2022}. 
Other projects with a thematic focus on health, water and early warning are considered in more detail in the remaining part of this section. Health, as \acrfull*{cbs} is successfully implemented as \acrshort*{cbm} with NYSS as \acrshort*{mcs} in Somalia (FIXME: see chapter \ref*{TODO:}), and water and early warning projects, as they are thematically related to this work. Projects concerning VGI will not be discussed in depth in this context, as mapping in this project will most definitely be carried out by professional and trained personnel. 

\subsubsection*{Community-based Surveillance}\label{subsubsec:cbs}

Conventional surveillance systems for monitoring health of animals, humans and the environment rely on information of medical professionals, health facility records, and laboratory examinations to detect abnormalities that could signify potential outbreaks and newly emerging pathogens \autocite{mcneilLandscapeParticipatorySurveillance2022a}. However, these data are not sufficiently accessible in all regions of the world  to allow adequate responses \autocite{mcneilLandscapeParticipatorySurveillance2022a,nikolayEvaluatingHospitalBasedSurveillance2017}. The strong developments and increasing availability of mobile technologies, the recognition of the value of local knowledge in health management, and recently reinforced by the COVID 19 pandemic, have led to an an increasingly widespread use of \acrshort*{cbs} \autocite{kullenbergWhatCitizenScience2016,mcneilLandscapeParticipatorySurveillance2022a}. The \autocite{technicalcontributorstothejune2018whomeetingDefinitionCommunitybasedSurveillance2019} defined \acrshort*{cbs} as "the systematic detection and reporting of events of public health significance within a community by community members". With the growing importance of the \textit{One Health} approach, these "events of public health significance" span across the domains of human, animal and ecosystem health \autocite{cdcOneHealthBasics2022}.
\autocite{mcneilLandscapeParticipatorySurveillance2022a} identified 60 different ongoing surveillance systems across five continents. These systems were covering the three domains either stand-alone or in combination, on different spatial scales and with different technical characteristics. However, all projects have used some kind of digital technology, with websites and smartphones as the most common vehicles. Furthermore, a high percentage of the surveyed projects have noted the usefulness of the \acrshort*{cbs} approach as it "improved community knowledge and understanding" (78\%) and "earlier detection" (67\%). This finding is supported by various other studies \autocite{byrneCommunitycentredApproachGlobal2020,jarrettEvaluationPopulationMobility2020,mcgowanCommunitybasedSurveillanceInfectious2022,metugeHumanitarianLedCommunitybased2021,ratnayakeEarlyDetectionCholera2020,ratnayakePeoplecentredSurveillanceNarrative2020,technicalcontributorstothejune2018whomeetingDefinitionCommunitybasedSurveillance2019}.
The \acrshort*{cbs} approach has proven to be a more advantageous complement to the conventional system, especially if certain conditions are taken into account. \autocite{gueninParticipatoryEpidemiologicalOne2022} highlights the importance of congruent definitions and their adaptation to the different actors and roles as well as the adaptation of (two-way) communication channels. Preceding suitability assessments, simple design and reasonable incorporation of technology, effective community engagement, reliable and close surveillance through supervisors of local volunteers especially in the beginning as well as evaluation and feedback opportunities have been highlighted as key drivers for success. These drivers were grouped by \autocite{mcgowanCommunitybasedSurveillanceInfectious2022} in relation to (1) surveillance workers, (2) the community, (3) case detection and reporting, and (4) integration. Most of these factors and more have already been mentioned in the \acrshort*{cbm} context. They were linked to having a decisive influence on the quality of embeddedness in existing systems, acceptance, trust and ultimately its implementation in decision-making and response. In addition to these key success factors, main challenges remain in ethical and privacy considerations, availability of resources and fast response capacities in case of an event as well as community expectation management. Furthermore, \autocite{boetzelaerEvaluationCommunityBased2020} findings indicate, that the additional benefits of \acrshort*{cbs} in already stable settings are limited as the approach is resource intensive. Nevertheless, the increasing application of \acrshort*{cbs} in low-resource or conflict-affected areas, where the full range of benefits were brought to bear. These benefits include CBSs' early warning capabilities and showed promising capacities to address current gaps in health related information and response management, especially in regard to spatial coverage and lower response times \autocite{metugeHumanitarianLedCommunitybased2021, ratnayakePeoplecentredSurveillanceNarrative2020}. \autocite{metugeHumanitarianLedCommunitybased2021} has additionally been able to fruitfully adapt \acrshort*{cbs} for related issues such as displacement and malnutrition and the SRCS is currently using CBS together with the MCS platform NYSS from the Norwegian Red Cross (NRC) in Somalia. The CBS approach has thus shown that it can be potentially adapted to other issues and that it can be successfully implemented regionally. More on the regional implementation in chapter (TODO: \ref*{CASE STUDY CBS} and \ref*{RESULTS NYSS}).

\subsubsection*{Community-based Water Monitoring and Management}

\acrfull*{cbwm} is an application example of \acrshort*{cbm} which gained mayor public interest particularly in North America, Europe, Australia and Southeast Asia \autocite{kirschkeCitizenScienceProjects2022, koehlerCitizenParticipationCollaborative2008, livinglakescanadaElevatingCommunityBased2018}. \acrshort*{cbwm} practices range from small monitoring projects to integrated partnerships or councils for the management of watersheds \autocite{westonCommunityBasedWaterMonitoring2015}. Just as for CBM and CBS, participant engagement, data quality control and management, sustainable funding and embedding in existing structures are key to successful integration and implementation of such projects \autocite{allenCommunityBasedWaterMonitoring2018,livinglakescanadaCommunityBasedWaterMonitoring2018,westonCommunityBasedWaterMonitoring2015}.
An overview of primarily water and weather related citizen science projects can be seen in table \ref*{TODO:}. Striking is the already mentioned globally unequal distribution of the projects with a strong emphasis on North American Countries. Furthermore, their focus is mostly on river, lake, groundwater and precipitation levels or focusses on their respective water quality. The technical solutions are mostly not freely available and not open source (FIXME: true? -> and extend). 

\missingfigure{overview table about water related CS projects with CBS/MCS aspects}
% table? table could be worth it.
% project name; topic; region/location; organization; participation level; tools, website/paper

Further noticeable are the technical requirements, which almost always require some sort of smartphone, dedicated measurement equipment or internet access. Only Weeser et al.'s approach is based on simple text messages but were limited in content to a station ID and the indicated stream water level. Here, signs explaining the monitoring and transmission process with pictures and instructions in Swahili and English were placed next to a water level indicator, encouraging passers-by to contribute \autocite{weeserCitizenSciencePioneers2018a}. \autocite[1597]{weeserCitizenSciencePioneers2018a} noted, that method of "transmitting the observations using simple cell phones and text messages turned out to be stable and reliable without major technical problems" in the context of their work in low-income rural areas in Kenya. The problem of occasionally insufficient network coverage was overcome by participants waiting until they reached a network before transmitting, making network availability not a limiting factor in this study. \autocite{wilson-jonesUsingMobilePhones2012} established and evaluated an Android mobile based system to support rural water quality monitoring in South Africa by simplifying connection between managers and operators of municipal test facilities. While all municipalities expressed the system as beneficial exemplifying the usefulness of fast, easy and low resource-intensive communication possibilities in such a context, this project does not necessarily fall within the sphere of \acrshort*{cs}, as the target group here was professional staff. Drawing on their literature review of water quality studies under climate change, \autocite[147]{huangManagementDrinkingWater2020} recommend the application of a "hybrid modality in which community management is the mainstay with supplement from external support" also considering differences in local realities and stakeholder opinions and needs. 

One approach to embed \acrshort*{cwbm} into local traditional community water management practices is proposed by \autocite{dayCommunitybasedWaterResources2009}. \autocite{dayCommunitybasedWaterResources2009} argues, that overarching concepts like the \acrfull*{iwrm} remain to large and complex to be manageable and implementable on local levels and additionally often fail to adequately include local stakeholders. Building on the decentralized and locally better opperationalisable version of \acrshort*{iwrm} called 'light IWRM' \autocite{butterworthFindingPracticalApproaches2010,moriartyIntegratedWaterResources2004} and its practical component of Water safety plans (WSP) \autocite{bartramWaterSafetyPlan2009}, \autocite{dayCommunitybasedWaterResources2009} created a community-based water resource management framework (see figure \ref*{TODO:}). 

% framework for the first couple of side goals
\missingfigure{TODO: “Figure 2. Community-based water resource manageme” ([Day, 2009, p. 59](zotero://select/groups/4773535/items/YWSNQ8A2)) ([pdf](zotero://open-pdf/groups/4773535/items/ETPCI5RI?page=14&annotation=BAMSY255))}

This framework provides the foundation for monitoring by encompassing the specifics of arid regions also with regard to possible drought phases, community needs, risks and water resource assets. Furthermore, the community is seen primarily as a partner rather than a beneficiary, also taking into account internal communal heterogeneity and inequalities making it a good conceptual basis for this works water source monitoring design approach.
Further work for guiding principles in the sphere of \acrshort*{cbwm} are numerous and interested readers are referred to \autocite{westonCommunityBasedWaterMonitoring2015}.

\subsubsection*{Other community-based concepts and initiatives} % ???????????? or different name? -> ? Community-Based Disaster Risk Management
% done. Keep it short. Not as important. weeeellll that did not really work out though.

Potential capabilities and areas of application to apply the concept of \acrshort*{cbm} and \acrshort*{mcs} are wide-ranging and numerous. Besides health- and water related domains, Community-based Disaster Risk Reduction (CBDRR), Disaster Risk Management (CBDRM) and Early Warning Systems (CBEWS) / information dissemination are rising fields of application. While health and water-related projects can be part of the broader CBDRR or CBDRM approach, depending on their focus, many projects about CBDRR, CBDRM and CBEWS focus on natural disasters such as droughts, fires, typhoons, (flash) floods, and landslides \autocite{machereraReviewStudiesCommunity2016,manaloBellBottleTechnology2013,pinedaRedefiningCommunityBased2015,smithCommunitybasedEarlyWarning2017,tarchianiCommunityImpactBased2020,trogrlicIndigenousKnowledgeEarly2018,vhumbunuCountingDayZero2021}. Based on \autocite{unisdrUNISDRTerminologyDisaster2009}, \autocite[198]{vhumbunuCountingDayZero2021} defines CBDRM as "the involvement of potentially affected communities in disaster risk management at the local level by building their capacities to assess their vulnerability to natural disasters and develop strategies necessary to mitigate the impact of these disasters" and further states, that "at the core of these concepts is the involvement of communities in making decisions and implementing disaster risk management strategies, actions, and initiatives".
Examples for participatory Disaster Management Software are large and multi-purpose platforms like Ushahidi, Sahana Eden and Kobo \autocite{koboorganizationKoboToolbox,sahanafoundationSahanaEDEN2016,ushahidiCrowdsourcingSolutionsEmpower}. A smaller but dedicated approach to bridge indigenous knowledge and modern science by disseminating early drought information and warnings is the framework ITIKI (Information Technology and Indigenous Knowledge with Intelligence) \autocite{akanbiDevelopmentRuleBasedDrought2018,masindeEffectiveDroughtEarly2014a,masindeImplementationRoadmapDownscaling2013,masindeDownscalingAfricaDrought2018,masindeFrameworkPredictingDroughts2010a,masindeITIKIBridgeAfrican2012,masindeITIKIMobileBased2019,nyetanyaneIntegrationIndigenousKnowledge2020,thothelaSurveyIntelligentAgroclimate2021a}. This system integrates scientific and indigenous drought forecasts by combining local and expert knowledge, technical components like wireless sensors, mobile phones and artificial intelligence analysis capacities to provide micro-level forecasts to local farmers and communities. Positive effects of local drought forecast dissemination could also be confirmed by \autocite{anderssonLocalEarlyWarning2020}'s (FIXME:) study while also mentioning, that local capacities or pre-conditions often limited a positive respond to the early warning.
(FIXME: check if this is correct)\autocite{gladfelterPoliticsParticipationCommunitybased2018,inayathEARLYWARNINGSYSTEM2018,trogrlicIndigenousKnowledgeEarly2018} highlight the importance to tailor the information to the needs, capacities and social structures of communities on the ground to enable their successful implementation. Accounting for community heterogeneity is also emphasized by \autocite{gladfelterPoliticsParticipationCommunitybased2018} as people may be incapable to respond to early warnings due to a lack or resources or knowledge. In addition, \autocite[21]{inayathEARLYWARNINGSYSTEM2018} advocates that early warning messages should be "simple, timely, and encourage early action" to enable an appropriate response in the first place.
Another problem in implementing participatory early warning systems is the gap between classical top-down approaches and community-based bottom-up initiatives. Successfully bridging the gap between these two approaches by directly coordinating available technical capacities through a participatory approach is possible according to \autocite{tarchianiCommunityImpactBased2020}. This is supported by \autocite{henriksenParticipatoryEarlyWarning2018} findings, that bottom-up approaches in contrast to classical concepts better facilitate the integration of local stakeholders in processes of decision-making and risk management.Generally, \autocite{marcheziniReviewStudiesParticipatory2018} literature review indicates a shortage of research in regard to citizen science and CBEWS and \autocite{baudoinEarlyWarningSystems2014} additionally notes the need to significantly improve the design and application of early warning systems. \autocite{baudoinEarlyWarningSystems2014} advocates for an integrated cross-scale approach ensuring the involvement of the at-risk population at all stages of the management process. Further arguing for "early warning systems that are both technically systematic and people-centred" \autocite[15]{baudoinEarlyWarningSystems2014}.


% further classified literature reviews in the realm of geophysics: Zheng et al. 2018 p.717 Table 2 (method classification) + 3 (management) + 4 (quality assurance) + 5 (processing) + 6 (data privacy)

% yeeey. could get this done this week! Woop Woop! close though.

%----------------------------------------------------------------------------------------
%	SECTION 8 Case Study Area (+ application of the rest)
%----------------------------------------------------------------------------------------


\section{Case Study Area (+ application of the rest)} % maybe put this in the beginning of this all?

Northern Somalia, also known as Somaliland, is a region located in the Horn of Africa. Officially referred to as the Republic of Somaliland, it is a self-declared independent, de facto sovereign state, but it is not recognised internationally and is still considered part of Somalia. Somaliland is bordered by the Gulf of Aden to the north, Somalia to the east, the Federal Republic of Ethiopia to the south and west, and the Republic of Djibouti to the northwest. The claimed region encompasses around 177,000 km$^2$ and has an estimated population size between 4.2 to 5.5 million people, depending on the source \autocite{petrucciLandscapeLandformsNorthern2022,republicofsomaliaCountryProfile20212021,somaliredcrescentsocietyFeasibilityStudyPotential2022}. Administratively, Somaliland is divided into six regions from east to west, Awdal, Marodi jeh, Sahil, Todgheer, Sanaag and Sool with the national capital being Hargeisa in Marodi jeh (see figure \ref*{TODO: create map of Somaliland with regions https://commons.wikimedia.org/wiki/File:Regions_of_Somaliland_labelled_EN.svg}) \autocite{republicofsomaliaCountryProfile20212021}.

\missingfigure{Somaliland figure with all the fancy stuff plllllls}

This chapter will give a brief overview about the geography, economy and social conditions. It will place the above concepts in the context of past and present local conditions and elaborate on current work on early warning concepts and projects. 


\subsection{Geography \& Climate}

The geography of this region is marked by its arid and semi-arid conditions, with a diverse range of physical and environmental features that define its landscape. Topographically, Somaliland can be divided into three main zones: the coastal plain Guban, the mountain range Oogo and the plateau Hawd \autocite{republicofsomaliaCountryProfile20212021}. The Guban (Somali for 'the burnt') area is a very hot and arid region averaging less than 100\,mm rainfall per year with potential evapotranspiration exceeding rainfall by thirty times \autocite{salemTerritorialDiagnosticReport2016}. Furthermore, it is not unusual to have no rain at all for 2-3 consecutive years. The Oogo mountain ranges receive up to 500-600\,mm of rainfall annually with equal evapotranspiration potential, and annual mean temperatures of 20-24\,°C, with peaks rarely exceeding 35\,°C. Temperature conditions on the Hawd plateaus are comparable, but precipitation can be lower and the potential evapotranspiration is at a factor of about 1.5 \autocite{abdulkadirAssessmentDroughtRecurrence2017,salemTerritorialDiagnosticReport2016}. 
Somaliland's climate is typically arid to semi-arid and experiences four distinct seasons. The primary rainy season, known as Gu', takes place from April to June and contributes to about 50-60\,\% of the annual precipitation. The secondary rainy season, called Dayr, lasts from August to November and accounts for approximately 20-30\,\% of the total rainfall. The remaining two seasons are Jiilaal and Xagaa, which occur from December to March and July to August, respectively, and are characterized by dry conditions \autocite{abdulkadirAssessmentDroughtRecurrence2017,republicofsomaliaCountryProfile20212021}.
A detailed description of the geological features of Somaliland, together with many pictorial impressions can be found in \autocite{petrucciLandscapeLandformsNorthern2022}. The soil types in Somaliland are closely linked to its geomorphology and are typically marked by poor structure, high permeability, low capacity to retain moisture, and insufficient internal drainage \autocite{salemTerritorialDiagnosticReport2016}. The naturally sparse vegetation, tree cutting and overgrazing also lead to accelerated soil erosion \autocite{salemTerritorialDiagnosticReport2016}. Nomadic and transhumance pastoralism activities influence around 90\,\%, and agro-pastoralism about 2\,\% of the land with often adverse environmental effects \autocite{salemTerritorialDiagnosticReport2016}. Besides poor soils, high levels of erosion and a challenging climate, little water resources stress the local fauna, flora and human population. 

\subsection{Water Sources}

Often insufficient knowledge about hydrogeological conditions and access depths of more than 100\,m caused a very limited number of boreholes (approx. 300 in all 3 regions) \autocite{faoswalimHydrogeologicalSurveyAssessment2012, petrucciLandscapeLandformsNorthern2022,salemTerritorialDiagnosticReport2016}. As there are no permanent rivers in Somaliland, the use of surface water is primarily based on water retention structures storing part of the water supply beyond the rainy season \autocite{petrucciLandscapeLandformsNorthern2022}. Wide and open structures called \textit{balleys} can store large volumes of water, but do not last as long as \textit{berkads}.
Traditional berkads are commonly 3 to 4 meters deep, 7 to 9 meters wide and 10 to 13 meters in length. Build materials are commonly stones and clay and some are covered with organic materials such as sticks and bushes. Berkads are generally constructed in clusters and usually built on a slope to collect water during the rainy season, but are sometimes filled by man-made canals with or without impurity collection facilities \autocite{walkerChangingPastoralismEthiopian1998}. These missing mechanisms during the filling process can result in contamination of the water with organic matter, animal or human faeces etc. \autocite{mercycorpsIMPROVEDBERKADDESIGNS2017}. The same lack of separation of animals and humans can also lead to contamination when water is extracted. Improved designs exist and more sophisticated versions nowadays use concrete, are properly roofed to counteract evaporation and have adequate inflow and extraction mechanisms to prevent contamination \autocite{mercycorpsIMPROVEDBERKADDESIGNS2017, petrucciLandscapeLandformsNorthern2022}. Following \autocite{mercycorpsIMPROVEDBERKADDESIGNS2017} calculations, an improved berkad needs to have a volume of about 1000 to 1200\,cubic meters to  withstand a 3 month dry period with a monthly extraction of 288\,m$^3$. This amount would serve 240 persons (20l/day/person), 150 camels (12l/day/camel) and approximately 2000 (1.5l/day/animal) sheep and goats. Currently valid total number of Berkads for Somaliland do not exist but \autocite{walkerChangingPastoralismEthiopian1998} estimated about 12.000 berkads clustered in 126 groups in the ethiopian district in Gashaamo, which borders Somaliland in the south. \autocite{birchWeUsedSing2008} notes 7000 berkads for the Hawd region, although with an unknown number of non-operational berkads. The sheer number and reliance of pastoralists and communities on berkads mentioned by \autocite{walkerChangingPastoralismEthiopian1998} and \autocite{birchSomalilandSomaliRegion2008} illustrate their importance. Besides boreholes and berkads, shallow wells, springs and dams are types of water sources. Available datasets about all water sources but especially berkads, concerning e.g. their location, functionality, status of ownership and other factors are limited, mostly outdated and unknown in quality \autocite{FAOSWALIMSomalia}. 

% Current stuff about water sources (?) -> or in the results (?) but I would need to analyze that stuff thouuuuuugh. maybe if I have time in the end?
% https://spatial.faoswalim.org/layers/geonode:Strategic_Water_Sources_20225#/

% IF TIME: analyse SWALIM sources and create map. --> way more and more functioning but old and unreliable https://swims.faoswalim.org/livemap/view
% differences in 2018, 2020, 2022

% https://climseries.faoswalim.org/station/
% only one groundwater station

% mindmap of locals: -> difficult to compare as it does not cover the same region.
% “water points” ([“Changing Pastoralism in Region 5”, p. 8](zotero://select/groups/4773535/items/FXJGUTLD)) ([pdf](zotero://open-pdf/groups/4773535/items/BIAA5M57?page=8&annotation=J48Y7AG5))

\subsection{Political, social and economic circumstances} % + history?

After being ruled by the Ottoman Empire and subsequent British colonisation, Somaliland gained independence on 26th, June 1960. A few days later Somaliland voluntarily merged with Italian Somalia to form the Somali Republic. From 1969 until 1991, Somali Republic was controlled by a military junta, led by Siyad Barre which from a supremacy of the southern part cruelly and partly arbitrarily suppressed the northern one, Somaliland. Arrests, mining of water points and executions culminated in the genocide of thousands of members of the largest clan, the Isaaq tribe \autocite{peiferStoppingMassKillings2009,republicofsomaliaCountryProfile20212021}. Since the collapse of the Siad Barre regime in 1991, Somaliland has developed into one of the most politically stable democracies in the Horn of Africa, but challenged in recent times due to the postponement of elections \autocite{bbcSomalilandProfile2022, fortiPocketStabilityUnderstanding2011}. Though, internel conflicts and border disputes with Puntland in the east continue until today \autocite{filhoDEMOCRACYAFRICAOUTSTANDING2021}. Nowadays, Somaliland is a presidential republic, combining its traditional clan culture with modern democratic elements and structures of the House of Representatives and Elders \autocite{salemTerritorialDiagnosticReport2016}.
Somaliland has a GDP of approx. 1.5-2\$ billion, mostly based on remittances from Somalilanders working abroad and main export being livestock, per capita income is only in the hundreds of dollars \autocite{klobucistaSomalilandHornAfrica2018, republicofsomaliaCountryProfile20212021, worldbankNewWorldBank2014}. Low literacy rates (~48\% for adults above 15), a ~35\% secondary school education completion rate and high unemployment rates further complicate the situation \autocite{republicofsomaliaCountryProfile20212021,worldbankNewWorldBank2014}. Due to its reliance on pastoralism and livestock for mayor parts of its economy and food security, Somaliland is prone to natural disasters \autocite{usaidcenterforresilienceEconomicsResilienceDrought2018}.


%-----------------------------------
%	SUBSECTION 8.4
%-----------------------------------

\subsection{Hazards and risks}

Drought, flash floods, land degradation and conflict all pose risks to Somaliland's environment and society, with droughts posing the greatest threat in recent centuries \autocite{abdulkadirAssessmentDroughtRecurrence2017}. Several historical and current analyses and predictions indicate, that these phenomena will not get less but possibly intensify and become more frequent driven by large phenomena like the El Niño-Southern Oscillation and rising Sea Surface Temperatures (SST) \autocite{abdulkadirAssessmentDroughtRecurrence2017,aliMitigatingNaturalDisasters2017a, balintMonitoringDroughtCombined2013, erianGARSpecialReport2021, FAOSWALIMSomalia, museiSPEIbasedSpatialTemporal2021, nationaldroughtcommitteeSOMALILANDDROUGHTRAPID2022,trisosAfrica2022}. Population growth, deforestation and desertification, groundwater depletion and land grabbing further stresses the situation \autocite{aliMitigatingNaturalDisasters2017a}. While a rough tendency can be derived from such predictions, \autocite[10]{abdulkadirAssessmentDroughtRecurrence2017} findings indicate, that the forecast quality of global climate model simulations "show varying results and therefore remain uncertain for Somaliland". 
Geographically, the eastern regions Sanaag, Sool and Todgheer are historically the most severely impacted ones \autocite{abdulkadirAssessmentDroughtRecurrence2017, FAOSWALIMSomalia}. In the period since 1960, Somaliland experienced 17 major droughts with the most intense and widespread droughts in 1973-1974, 1984, 1991, 2010/2011, 2016/2017 and 2021 until today \autocite{abdulkadirAssessmentDroughtRecurrence2017, credEMDATInternationalDisasters2023}. The worst drought in 2010-2012 led to a famine, where more than 200.000 people died and over 2.6 million people were affected all over Somalia \autocite{srcsDRMStrategicPlan2021}.
Currently, the almost complete failures of five successive rainfall seasons, rising food prices and severe water shortages are adding up to another stressful situation putting 810.000 people in need of emergency assistance \autocite{nationaldroughtcommitteeSOMALILANDDROUGHTRAPID2022}. This number is projected to rise substantially if the current drought conditions persist \autocite{swansonNearlyMillionPeople2022}. Shallow wells and most Berkeds have dried up, leaving boreholes and expensive water trucking as the last options for water supply \autocite{nationaldroughtcommitteeSOMALILANDDROUGHTRAPID2022}.
Cascading droughts can have cascading impacts as affected people are forced into bad feedback-loops to respond to the immediate crisis, reducing their coping capacity and thus further increasing their vulnerability to future events \autocite{usaidEconomicsResilienceDrought2018}. \autocite{usaidEconomicsResilienceDrought2018} hypothesised, that these post-shock impacts can better be mitigated by early interventions than by late response. Although, \autocite{usaidEconomicsResilienceDrought2018} states, that there is very little data to support this statement and that it is primarily based upon logical deduction and not field data. Nonetheless, this assumption is also supported by \autocite{aliMitigatingNaturalDisasters2017a}, \autocite{abdulkadirAssessmentDroughtRecurrence2017} as well as by the growing Forecast based Financing practitioners \autocite{gualazziniEWEAEarlyWarning2021, harrowsmithFutureForecastImpact2020}

\subsection{EAP + forecast, trigger and so on}

The 2011 famine in Somalia was projected 11 month in advance. Despite this early warning, the international community failed to react adequately and in time to prevent the worst \autocite{elisabethstephensFORECASTBASEDACTION2015, hillbrunerWhenEarlyWarning2012}. Subsequent evaluations point to two main areas of concern. On the one hand, there was a lack of timely funding, but on the other hand, the concept of preventive action had not yet permeated the humanitarian community and response activities were still seen as the standard \autocite{elisabethstephensFORECASTBASEDACTION2015}. This failure, as well as the successive improvements in forecasting and the growing scientific interest and knowledge about the positive impact of early warning and anticipation measures, laid the foundation for the current development of the EAP for Somalia. As the project is still in progress, detailed information is not yet possible to present in all areas and the presented information is also subject to constant changes and future developments. Nevertheless, critical points for this work can be derived and the need for further developments can be elaborated.

The interest to develop an \acrshort*{eap} for a slow-onset hazard such as drought only recently started to become more popular within the RCRC as the focus laid on fast-onset disasters thus far \autocite{rcrcFORECASTBASEDFINANCINGEARLY2020}. \autocite{rcrcFORECASTBASEDFINANCINGEARLY2020} presented the first adaptation of the general manual of the \acrshort{ifrc} (see \autocite{ifrcFbFPractitionersManual2023b}), merging experiences of pilot projects to adjusted guidelines for the development of FbF and early actions in the context of drought. Currently, at least seven National Societies (Kenya, Uganda, Ethiopia, Zimbabwe, Somalia, Lesotho and Niger) are planning, developing or have recently completed a drought \acrshort*{eap} \autocite{lesothoredcrosssocietyEARLYACTIONPROTOCOL2022,nigerredcrosssocietyNigerDroughtEarly2021,rcrcFORECASTBASEDFINANCINGEARLY2020}.
The \acrfull*{srcs} has completed their preliminary \textit{Feasibility Study on Potential Use of Forecast-based Financing (FbF)} in June 2022. A pilot study shall be conducted to test practical implementation feasibility in Somaliland and potentially Puntland with emphasis on, from highest to lowest priority: droughts, health, (flash) floods, cyclones, locusts, and conflicts. Besides the detailed description and justification for each type of disaster, the assessment also confirmed the well positioning of the \acrshort*{srcs} to undertake such a FbF program and to embed it into the general \acrlong*{drm}.
The implementation of a FbF program cannot be done by a National Society alone. Besides the \acrshort*{srcs} numerous other stakeholders will take part in providing information, resources or knowledge as well as acting upon aforementioned. The landscape of actors is wide and includes many local, regional, national and international governmental and non-governmental groups, initiatives, centers and organisations. To name but a few: The Ministry of Agriculture (MoA), of Water Resources (MoWR), of Health Development (MoHD) and of Humanitarian Affairs and Disaster Management (HADMA) and others include Somaliland's state actors. (TODO:)\acrfull{brcis} and \acrfull{cdrmc} compromise local and regional NGO networks and committees. The UN (\acrshort{fao}, \acrshort{ocha}, \acrshort{undrr}, \acrshort{wfp}, \acrshort{who}, World Bank, \acrshort{wmo}, \acrshort{grc}, \acrshort{nrc} and \acrshort{ifrc} are a selection of international actors engaged in Somalia. Added to this are a number of other think tanks, climate centres and forecasting providers, making the integration of the respective actors an important but also intricate affair, especially in the light of the multi-faceted nature of droughts\autocite{rcrcFORECASTBASEDFINANCINGEARLY2020,somaliredcrescentsocietyFeasibilityStudyPotential2022}. 

Forecasts are also provided by various organisations and scales. The \acrshort*{fewsnet} releases famine warnings and reports for the entire african continent on a regular basis \autocite{fewsnetFamineEarlyWarning2023}. Regional forecasts are provided by the Climate Predictions and Applications Centre (ICPAC) based on global models for the Greater Horn of Africa region \autocite{icpacDeliveringClimateServices2023}. More small-scale prognoses are released from FAO's \acrshort*{swalim} and \acrshort*{fsnau} programs which monitor different drought indicators based on relatively few weather stations (100 manual and 10 automatic in all of Somalia) and remotely gathered and modelled climate information \autocite{faoswalimSWALIMWeatherMonitoring2014,somaliredcrescentsocietyFeasibilityStudyPotential2022}. There are two other local seasonal forecasts issued by government agencies and disseminated by the responsible agency, \acrshort*{nadfor} to stakeholders at all levels for natural hazard warnings \autocite{somaliredcrescentsocietyFeasibilityStudyPotential2022}. Besides SRCS's own disease \acrshort*{cbs} informing actions for health related issues, data of local circumstances influence forecasts only scarcely and infrequently.

Up to this point, it has not yet been decided which prediction and reaction trigger should be chosen for the SRCSs' \acrshort*{eap} but it will inevitably be based on scarce coverage and primarily large scale data, as it is the case for the \acrshortpl*{eap} in Niger and Lesotho \autocite{lesothoredcrosssocietyEARLYACTIONPROTOCOL2022,nigerredcrosssocietyNigerDroughtEarly2021}.
The trigger methodology will be a staggered trigger, following current recommendations of the \autocite{rcrcFORECASTBASEDFINANCINGEARLY2020} but its definition remains a challenge due to the currently very tense situation and the medium-term changes in weather and climate over the last 10 years. Under these conditions, it is quite difficult to determine a \textit{normal} period against which \textit{drought events} can be measured and will ultimately depend on the chosen forecast. Conceivable triggers could be the predicted failure of one or more consecutive rainy seasons or a specific classification warning for food or water insecurity but will also depend on selected actions. \autocite[19]{gettliffeOCHAAnticipatoryAction2021} found, that triggers need to be linked to their respective intervention, or otherwise will "led to significant challenges".
Identified actions by the feasibility study for the \acrshort*{eap} for drought interventions are water storage rehabilitation, de-stocking, early or alternative short growth crop planting, cash distributions, women and childeren shelters as well as water trucking \autocite{somaliredcrescentsocietyFeasibilityStudyPotential2022}. The Ministry of Livestock and Fisheries Development notes, that de-stocking will hardly be feasible due to little trust in forecasts by livestock owners as well as no internationally approved abattoir which limits the amount to local market capacities \autocite{somaliredcrescentsocietyFeasibilityStudyPotential2022}. \autocite{gualazziniEWEAEarlyWarning2021} propose water voucher as viable alternative to water trucking in regions where a functional market of private water vendors already exists. Besides \acrshortpl{aa}, adequate policies for water management, price regulations, and allocation mechanisms are seen as potential opportunities to mitigate further drought impacts \autocite{gualazziniEWEAEarlyWarning2021,wangPropagationDroughtMeteorological2016}.
Besides the mentioned forecasts of natural phenomena, SRCS has successfully set up a \acrshort*{cbs} project developing and utilizing the platform NYSS to monitor and react to disease outbreaks on community level since 2018 \autocite{jungCommunityBasedSurveillance2022}. 

%maybe add figure from here https://www.cbsrc.org/what-is-nyss

NYSS is an open-source implementation following the \acrshort*{mcs} concept and is primarily developed by the \acrshort*{nrc} \autocite{nrcNyssToolDeveloped2022}. The platform allows for high degrees of automatization in regard to collection, SMS analysis and validation as well as feedback and notification possibilities. A map and table overview about data collectors and messages facilitates supervision as well as the fast and simple escalation of warnings to health officials and other organizations through an interactive dashboard \autocite{nrcNYSSCommunitybasedSurveillance2021,nrcWhatNyss2023}. This high level of automation and good integration into existing organisational structures and actor networks enables rapid responses, often in less than 24 hours \autocite{jungCommunityBasedSurveillance2022}. NYSS maintains high standards of data security and privacy \autocite{quinnNyssDATAPROTECTION2020}.

Besides \acrshort*{srcs} and \acrshort*{ifrc}, \acrshort*{ocha} and \acrshort*{brcis} also developed anticipatory action plans for Somalia in recent years. \acrshort*{ocha} followed with their pilot study in 2020 conventional frameworks in regard to forecasts and triggers, using large scale indices with a combined trigger of pre-identified thresholds \autocite{gettliffeOCHAAnticipatoryAction2021,ochaANTICIPATORYACTIONPLAN2020}. Chosen actions comprise all major fields of food security, WASH, education, health and risk communication, often with lead times of multiple weeks to months. In their evaluation, \autocite{gettliffeOCHAAnticipatoryAction2021} synthesized many lessons learned in all areas. Highlighting the buy-in of all stakeholders, early expectation setting, the importance for parallel development of \acrshortpl{aa} together with explicit, linked and robust trigger mechanisms. Cash transfers were "identified across several clusters as the preferred action" where local markets and the operational context allow \autocites[21]{gettliffeOCHAAnticipatoryAction2021}{ochaANTICIPATORYACTIONPLAN2020}.
\acrshort*{brcis} created their own \acrfull{crtrms} to integrate local information. The \acrshort*{crtrms} is based on key informants from a selection of a small group of 2-3 communities which represent a larger population of 10-12 communities. These information are then triangulated with regional, national and international secondary information sources to ultimately propose relevant anticipatory measures. The survey together with the triangulation should allow triggering within 12 days after data collection but commonly averages on 25 days in practice. Besides the relatively long duration, key informants are well aware, that their given information may influence the amount of humanitarian assistance in the area, highlighting the importance of trustbuilding and data triangulation\autocite{gualazziniEWEAEarlyWarning2021}. Indicators and thresholds are categorized into \textit{normal}, \textit{alert}, and \textit{alarm} allowing for \textit{red-flagging} of areas based on either one very strong impact or on a pre-defined amount of cumulative impacts in multiple areas. For example, one indicator is the condition of primary water sources in communities. These are assessed at the end of rainy season and categorized based on their water level into normal \textit{(more than half-full [75\%] or full)}, alert \textit{(half-full [50\%])} or alert \textit{(less than half-full [25\%] or empty)} which allows for a seasonal prediction and corresponding  flagging.

%----------------------------------------------------------------------------------------
%	SECTION 9 Conclusion literature
%----------------------------------------------------------------------------------------

alrighty bitches. lets fking go!


\section{Conclusion literature}

The concepts of Water Security, Water Scarcity and Drought are wide ranging, complex in nature and various definitions exist for each of these concepts. Water Security links an extensive network of interrelated trans- and inter-sectoral systems together and can be seen as umbrella term for the extensive web around water availability and its many components. Water Scarcity, as it has been defined in this work, has a physical and economic aspect which refer to the availability of water resources and the various capital conditions for its extraction, respectively \autocite{faoCopingWaterScarcity2012}. Therefore, the absolute lack of water in the current situation, water shortage, always has a human (long-term) dimension, particularly on the demand side. Drought is most often considered in four stages, namely meteorological, agricultural, hydrogeological and socioeconomic drought with each having their specific sets of impacts, indicators and indices. Drought, as natural hazard, is differentiated to aridity by its short but severe nature. Conceptual definitions give an idea and set boundaries to the concept of drought while operational definitions focus on its onset, duration, severity and spatial coverage.
Measuring and predicting droughts is complex and often multiple individual or sets of indicators and indices are combined to get a full picture of the situation. Indices are themselves often composites of multiple indicators and mathematical functions. The intricate and interconnected nature of droughts leads to uncertainties in forecasts making it more difficult to define impact thresholds for anticipatory measures. Impact is generally expressed by a combination of hazard severity, the exposure of assets and their vulnerability. The latter often summarized in the term risk.
Forecast based Financing is a relatively newly emerged phenomena in the realm of humanitarian aid that promotes anticipatory action before the impact. This is based on a hazard and risk assessment to identify thresholds on which to trigger pre-defined actions that reduce the impact, documented and defined in an Early Action Protocol. For successful implementation, triggers and actions should be developed and directly linked. This is often not feasible as local information data is either missing completely or is outdated.
Citizen Science is the integration of citizens in scientific or public endeavours projects and can promote various benefits to all engaged stakeholders if implemented and operated correctly. Citizen Science, similar to the above concepts is wide-ranging and complex. Under this umbrella, \acrlong*{cbs} together with \acrlong*{mcs} provide practically realizable frameworks and guidelines for the successful application of Citizen Science. \acrlong*{cbs} and \acrlong{cbwm} demonstrate the feasibility and transferability of these concepts.
Somaliland lies within the Great Horn of Africa and can geographically and climatologically be seen as a generally arid and water scarce region with poor soils, scarce vegetation and water resources. It has historically been troubled by droughts, as well as internal and external conflicts, which regularly exacerbate the already tense situation. Somaliland is one of the poorest regions in the world but managed to develop a relatively stable democracy for the last 30 years. Currently many local, national and international organization work on mitigating and responding to a further tense situation with famine expected in mid-2023. 

Certain limitations and gaps could be identified in the above conducted literature review. The concepts of Water Security, Water Scarcity and Drought need to be clearly defined and broken down to the specific region and application. Unfortunately, the long history of conflicts and insecurities severely limited scientific research in this region in particular, which generally led to little scholarly information on the region.
While numerous international forecasts exist and local assessments start to emerge, timely, highly local and up-to-date data is not available for most areas of interest. Thus, the direct link of trigger and anticipatory action is often not given, making the implementation less effective, efficient, targeted and it takes more time. 
The concept of FbF is now generally well established, but the drought use case is new and not yet well researched, which severely limits the amount of guidelines and frameworks available for this particular application. Thus, each new project and study has, at least in part, an exploratory character.
Citizen Science projects in regard to water are geographically primarily focused on North American and European countries, relating most of the scientific findings to the respective context. Furthermore, these water related projects mainly focus on river, lake or groundwater level or water quality monitoring and not on direct community water source investigations to facilitate early actions. The review of data availability and reliability of current datasets revealed a clear need for more up-to-date and complete information on water source locations and characteristics, especially with regard to the highly important Berkad water type. 

Embedded in the above broad concepts and reasoned by the tense context, scarce data situation and current EAP development, this work will partly address found gaps in literature by building on the successful methodology of CBS for issues of health and the thematic insights in water related subjects from other projects presented above. This work will thus find new ways of combining the proven CBS approach with MCS on the topic of direct community water source monitoring for the creation of a highly local and up to date water level threshold to trigger directly related anticipatory actions in a resource scarce environment.
The community-based monitoring and management concept has, to the best of my knowledge, not been carried out once nor scientifically investigated in regard to directly monitoring and managing of rural community water sources, especially not in resource scarce environments like Somaliland.
