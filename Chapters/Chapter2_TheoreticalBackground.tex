% Chapter Template

\chapter{Theoretical Background} % Main chapter title

\label{Chapter2} % \ref{Chapter2}
WATCH  OUT THAT THIS STAYS THE THEORETICAL BACKGROUND AND IS NOT !!! A DISCUSSION!!!!
%----------------------------------------------------------------------------------------
%	SECTION 1
%----------------------------------------------------------------------------------------

\section{Introduction}
% The introduction should clearly establish the focus and purpose of the literature review.
% highlight research gap + emphasize the timeliness
% Locate your own research within the context of existing literature [very important!]. 
% reiterate central problem
% focus and purpose of literature review
% brief summery scholarly context
% highlight research gap
% emphasize timeliness


\subsection{reasoning}:
https://www.unwater.org/our-work/integrated-monitoring-initiative-sdg-6

"Stronger accountability: Data can communicate that work is being done and progress is happening. Data can enable greater transparency, which reduces inefficiency and corruption.
Attracting commitment and investments: Data can quantify problems and make it easier to communicate needs for political commitment and public and private investments.
Evidence-based decision-making: Data can inform policy- and decision-makers of where to focus efforts and which solutions are most effective, to ensure the greatest possible gains with existing resources.
Leaving no one behind: Disaggregated data can help identify specific groups or areas with unmet needs and higher levels of risk, to which interventions can be targeted."
https://www.unwater.org/our-work/integrated-monitoring-initiative-sdg-6/background

“Experts in trigger methodology have indicated a more appropriate strategy may be to build on tools that currently exist at the government level such as national drought monitoring systems. As such, the ideal is an iterative process with the ground level along with a technology push that creates new ways to analyse drought and drought risk.” ([RCRC, 2020, p. 28](zotero://select/groups/4773535/items/UESIQTRJ)) ([pdf](zotero://open-pdf/groups/4773535/items/P5JPVZ97?page=28&annotation=977VS8FC))

% DRM Strategic Plan General and Specific Objectives
“2.1.1 Specific Objective 1 Vulnerable communities’ resilience at SRCS target areas strengthened through anticipatory actions, response, recovery, and disaster risk reduction, and they can anticipate and effectively respond to and recover from disasters and crisis by 2026.” ([“SRCS DRM Strategic Plan_final 9thNovember 2021-converted.pdf”, p. 15](zotero://select/groups/4773535/items/LFCBRZLD)) ([pdf](zotero://open-pdf/groups/4773535/items/6IL6K72G?page=15&annotation=ZFICKZRA))

% --> even the RCRC is still looking for good triggers -> maybe water levels are a good way -> reasoning for this study (see background identical text)

“Countries in which less than 50\% of the population uses improved drinking water sources are all located in sub-Saharan Africa and Oceania 91-100\% 76-90\% 50-75\% <50\% insufficient data or not applicable Proportion of the population using improved drinking water sources in 2015 ■ 91–100\% ■ 76–90\% ■ 50–75\% ■ <50\% ■ INSUFFICIENT DATA OR NOT APPLICABLE” ([World Health Organization, 2016, p. 15](zotero://select/groups/4773535/items/KVAKZ9ZT)) ([pdf](zotero://open-pdf/groups/4773535/items/4STYK52H?page=14\&annotation=FBURDS4T))

“The methods by which the Joint Monitoring Programme (JMP) of WHO and UNICEF” ([Bartram et al., 2014, p. 8137](zotero://select/groups/4773535/items/6AWUJTW5)) ([pdf](zotero://open-pdf/groups/4773535/items/BFNSQGWS?page=1&annotation=UL4Q2I4V))
“substantial limitations: current methods do not address water quality, equity of access, or extra-household services.” ([Bartram et al., 2014, p. 8137](zotero://select/groups/4773535/items/6AWUJTW5)) ([pdf](zotero://open-pdf/groups/4773535/items/BFNSQGWS?page=1&annotation=TIPCEXGG))

request of the SRCS -> practically wanted
understanding the full scope and knowing which water sources are at what level and quality can help with management decisions and trigger certain events very locally
current challenges (?) what do I want to address? (number 1,2,3)
outline of the thesis/project

overcome limitations / incorporating recommendations: e.g. increased support/engagement of poeple who actually use the reports (e.g. SRCS officials) 
fill geographic and information gap ->

its about drought forecasting and early trigger but at the same time highly local and practical information where and which water sources are good and functioning and which are not. -> highly practical information. Some data exist but (mostly) outdated.
about getting local knowledge from SRCS Volunteers and their community as well as returning information about the bigger picture
in order to enhance the quality of data for managing severe droughts in Somaliland. (one short paragraph -> motivation)
provide number of weather stations in the area

--> problem: conclusion of existing sources, tools and forecasts - only macro/international level? or are there meso/micro forecasts available?
better understanding the forecasting and its implications on the ground are crucial. --> local information. tons of Volunteers but even more water sources. Continue with Crowdsensing? Implications?

--> highlight the pro of this work
"The highly localized information provided by observers can fill drought monitoring gaps by ground-truthing quantitative indicators and offering information in places where other monitoring tools may not exist. Overall, the research team found that strategic investments in time and funding can help fill in geographic and temporal gaps in drought monitoring information through volunteer observations."
https://www.drought.gov/news/research-confirms-role-citizen-science-contributions-drought-detection-and-monitoring and https://doi.org/10.1175/BAMS-D-21-0157.1

“Citizen science programmes are promising cost-efficient methods to monitor environmental resources, which make them especially suitable for low-income countries to overcome their sparse data resolution.” ([Weeser et al., 2018, p. 1598](zotero://select/groups/4773535/items/SFA2MLHC)) ([pdf](zotero://open-pdf/groups/4773535/items/GP79FHFC?page=9&annotation=4E9JCTQ5))
“Since today's citizen science studies are mostly located in high-income countries, we are enthusiastic to motivate the scientific community to conduct citizen science studies in low-income countries.” ([Weeser et al., 2018, p. 1598](zotero://select/groups/4773535/items/SFA2MLHC)) ([pdf](zotero://open-pdf/groups/4773535/items/GP79FHFC?page=9&annotation=TYD7Q2ZD))

% also in case study
“The number of affected people will be 1,200,420 persons across all the six main regions in Somaliland. The top priority needs of the people affected to date are mainly water (70{\%), Food (21\%) and Health (9\%).” ([National Drought Committee, 2022, p. 3](zotero://select/groups/4773535/items/7XJRE6LM)) ([pdf](zotero://open-pdf/groups/4773535/items/2F59E4UZ?page=3&annotation=8JZVBSM6))

%-----------------------------------
%	SUBSECTION 1.1
%-----------------------------------
\subsection{relevance}


\subsection{structure of this overall chapter}
relevance
Somalia is komplett am krepieren because of a multi-year long drought - ...\% of damage/conflicts etc. is based on droughts. severe shit! --> first case study introduction (geographically, socially, etc.) 
scope: of the EAP/FbF project

“1.2 DROUGHT IN SOMALIA” ([“Economics of Resilience to Drought: Somalia Analysis”, p. 9](zotero://select/groups/4773535/items/LGRWAU43)) ([pdf](zotero://open-pdf/groups/4773535/items/MBXSCVWR?page=9&annotation=K6W9FFM7))

possibly talk about water security here for the larger context

%-----------------------------------
%	SUBSECTION 1.2
%-----------------------------------
\Section{key terms and concepts}

In this section relevant key terms and concepts are shortly explained or summarized if they are dealt with in detail later on.






water access



risk (hazard, exposure, vulnerability)
special report 2021 p.27 + p.58 1.4 Drought risk assessment

impact

risk
https://www.preventionweb.net/understanding-disaster-risk/component-risk/disaster-risk
where vulnerability = “Risk = ƒ (Hazard, Exposure, Vulnerability), where Vulnerability = ƒ (Susceptibility to impacts, Lack of coping capacity, Lack of adaptive capacity).” ([“Special report on drought 2021”, 2021, p. 27](zotero://select/groups/4773535/items/RAAM9PVS)) ([pdf](zotero://open-pdf/groups/4773535/items/7AK5QVBL?page=29&annotation=VQABMLCT))


drought:



drought frequency


drought severity
Drought severity and drought magnitude are two different aspects of drought that are often used to describe the severity and extent of drought conditions.

Drought severity refers to the intensity of a drought and is typically characterized by the degree to which it affects various aspects of the natural and social systems, such as precipitation, temperature, soil moisture, streamflow, water supply, agriculture, and socio-economic activities. Severity can be described using qualitative or quantitative measures, such as the Palmer Drought Severity Index (PDSI), the Standardized Precipitation Index (SPI), or the Keetch-Byram Drought Index (KBDI).

On the other hand, drought magnitude refers to the overall size or extent of a drought event, typically measured in terms of its spatial and temporal coverage. Magnitude can be characterized by the duration, frequency, and spatial extent of the drought, and can be assessed using various indices, such as the Percentage of Normal (PN) or the Area-Weighted Precipitation Anomaly (AWPA).

In summary, drought severity focuses on the intensity and impacts of a drought event, while drought magnitude describes the overall size and extent of the drought event. Both severity and magnitude are important for understanding and assessing the severity of drought conditions and their impacts on various natural and social systems.

https://www.earthdata.nasa.gov/topics/atmosphere/weather-events/droughts/drought-severity#:~:text=Drought%20severity%20is%20a%20relative,drought%20is%20considered%20to%20be.

though, duration plays a role in severity as well


drought intensity


impact

“Drought impact Impact of drought on socio-economic systems (sometimes called “socio-economic drought”) and/or ecosystems Aridity Long-term dryness as feature of climate, with long-term average precipitation being much lower than potential evaporation Water scarcity Long-term imbalance between water demand and water supply, caused by high average demand, low average water availability and/or problems with water supply Water shortage/stress Acute lack of water for (social, economic, or environmental) needs, caused by lower water supply than demand Overexploitation Long-term overuse of water resources resulting in a gradual decrease in water availability (for overexploitation of groundwater, the term “depletion” is often used)” ([Van Loon et al., 2016, p. 3637](zotero://select/groups/4773535/items/T6SI6L68)) ([pdf](zotero://open-pdf/groups/4773535/items/7GTQ4QHT?page=7&annotation=SNJLBYFL))




% ChatGPT start
"what is the difference between drought severity and drought intensity?
Drought severity and drought intensity are two different but related concepts used to describe the magnitude and impact of a drought event.

Drought severity refers to the extent of the impact of a drought on various sectors such as agriculture, water resources, and socioeconomic activities. It is usually assessed by looking at the degree of damage or loss caused by the drought. The severity of a drought is typically measured using indicators such as crop yield reduction, water supply depletion, and economic losses.

Drought intensity, on the other hand, refers to the duration and magnitude of the deficit in precipitation that causes the drought. It is usually assessed by looking at the degree of deviation from normal precipitation patterns and how long it has persisted. The intensity of a drought is typically measured using indices such as the Standardized Precipitation Index (SPI) or the Palmer Drought Severity Index (PDSI).

To illustrate the difference between drought severity and drought intensity, consider a scenario where two regions experience the same precipitation deficit over a period of three months. In the first region, the drought leads to a 50\% reduction in crop yields and widespread water shortages, while in the second region, the drought has little impact on agriculture and water resources. In this case, the severity of the drought is higher in the first region, despite the same intensity of precipitation deficit in both regions.

In summary, drought severity and intensity are both important measures of drought, but they represent different aspects of the phenomenon. Severity relates to the impact of a drought on different sectors, while intensity relates to the magnitude and duration of the precipitation deficit causing the drought \autocite{rcrcFORECASTBASEDFINANCINGEARLY2020,vereintenationenSpecialReportDrought2021}.




Bosse S
how does drought severity, drought impact and risk relate to each other?

Drought severity, drought impact, and drought risk are interrelated concepts used to assess and understand the effects of drought on various sectors. Here's how they relate to each other:

Drought severity refers to the degree of impact of a drought event on various sectors such as agriculture, water resources, and socioeconomic activities. It is usually assessed by looking at the degree of damage or loss caused by the drought. Severity can be measured using indicators such as crop yield reduction, water supply depletion, and economic losses.

Drought impact refers to the direct and indirect effects of drought on different sectors of society. It can include effects on human health, wildlife, ecosystems, infrastructure, and the economy. Drought impact can be assessed using various indicators and methods, including surveys, monitoring, and modeling.

Drought risk, on the other hand, is the likelihood of a drought event occurring in a given area and the potential consequences of that event. It takes into account the severity and impact of past droughts, as well as the vulnerability of the affected region to drought. Drought risk can be assessed using various indicators such as climate patterns, soil moisture levels, and water demand.

In summary, drought severity and impact are measures of the magnitude and effects of a drought event, while drought risk is a measure of the likelihood and potential consequences of future droughts. Understanding the relationship between these concepts is important for developing effective drought management strategies, including early warning systems, drought preparedness plans, and mitigation measures. By assessing the severity, impact, and risk of droughts, policymakers, and stakeholders can make informed decisions to reduce the negative effects of drought on society and the environment.




Bosse S
difference between drought severity and drought impact
Drought severity and drought impact are two related but distinct concepts used to measure the magnitude and effects of droughts.

Drought severity refers to the degree of intensity of a drought event. It is typically assessed by looking at the extent of the precipitation deficit, the duration of the drought, and the level of dryness in the affected area. Severity can be measured using various indices such as the Standardized Precipitation Index (SPI) or the Palmer Drought Severity Index (PDSI). Drought severity is a measure of the physical characteristics of a drought event and is not directly related to the effects of the drought on various sectors.

Drought impact, on the other hand, refers to the direct and indirect effects of a drought on different sectors such as agriculture, water resources, and socioeconomic activities. It is usually assessed by looking at the degree of damage or loss caused by the drought, such as crop failure, reduced water availability, and economic losses. Drought impact is a measure of the consequences of the drought on various sectors and is not directly related to the physical characteristics of the drought event.

To illustrate the difference between drought severity and impact, consider a scenario where two regions experience the same severity of drought event, with similar precipitation deficits and dryness levels. In the first region, the drought leads to a 50\% reduction in crop yields, while in the second region, the drought has little impact on agriculture. In this case, the severity of the drought is the same in both regions, but the impact is higher in the first region due to the greater damage to crops.

In summary, drought severity and impact are both important measures of drought, but they represent different aspects of the phenomenon. Severity relates to the physical characteristics of the drought event, while impact relates to the consequences of the drought on various sectors. Understanding the difference between these concepts is crucial for developing effective drought management strategies and mitigation measures. (ChatGPT)


%----------------------------------------------------------------------------------------
%	SECTION 2 Water security, drought & water scarcity/quality/access
%----------------------------------------------------------------------------------------


\section{some other title Drought \& Water security} % irgendwas in Richtung grundlegende Konzepte oder so.. was weiß ich ^^

Water security is a theoretical construct that has emerged in the 21st century to frame the overall water objectives and goals to guide local to global water management and policy development \autocite{sadoffWaterSecurity2020a}. It "links together the web of food, energy, climate, economic growth, and human security challenges that the world economy faces over the next two decades" \autocite[5]{wefBubbleCloseBursting2009}. In more detail, it is about "the availability of an acceptable quantity and quality of water for health, livelihoods, ecosystems and production, coupled with an acceptable level of water-related risks to people, environments and economies."\autocite{greySinkSwimWater2007}.
Water security integrates therefore economic, social and environmental dimensions into an interconnected and complex system of human and natural relations by addressing risks of too much, too little or poor quality water \autocite{vanbeekWaterSecurityPutting2014, mishraWaterSecurityChanging2021}. Due to the focus of this work, emphasis is placed on factors that decrease water security due to too little water availability. Besides other factors, natural disasters such as droughts, and water scarcity are the main drivers for insufficient quantities of water \autocite{caretta2022water}. Water quality and access are briefly addressed in addition to provide a more comprehensive  understanding of water security for the following chapters.


%-----------------------------------
%	SUBSECTION 2.1
%-----------------------------------

\subsection{About Drought}\label{subsec:about_drought}

\todo{TODO: change NDMC sources to knew ones}

Drought as highly complex and severe climate-related multi-hazard has far reaching, cascading and interconnected consequences affecting natural ecosystems, societies and economies \autocite{vereintenationenSpecialReportDrought2021}. Historically, droughts are a recurring feature that can occur in all climates. They can geographically extend over small areas to entire sub-continents and are slow onset events that can persist for a few weeks to several years. These high spatial and temporal variabilities make drought not only challenging to define but due to its slow onset, droughts are often only recognized when they are well advanced \autocite{idmpDroughtWaterScarcity2022,vereintenationenSpecialReportDrought2021}. While some drought conditions over large areas can be associated to some low-frequency changes in atmospheric conditions such as the El Niño, accurate cause identification can be rather challenging on smaller scales and requires many different parameters \autocite{botaiAnalysisDroughtProgression2019, vereintenationenSpecialReportDrought2021}.

%In order to approach this complexity, drought is most often defined from four different perspectives, focussing on different manifestations and stages. These definitions are outlined in the coming sub-chapter \ref*{subsec:drought_definitions}, followed by a section addressing the necessary indicators  currently employed in practice for these definitions.
%Generally, droughts are commonly characterized by deviations or the complete failure of climate and weather systems that drive the hydrological cycle compared to normal conditions\autocite{botaiAnalysisDroughtProgression2019,idmpDroughtWaterScarcity2022,vanloonDroughtHumanmodifiedWorld2016,vereintenationenSpecialReportDrought2021}. A more in depth definition can be found in the sub-chapter \ref*{subsec:drought_definitions}.

%“Many studies characterize drought via three main aspects: (i) intensity, (ii) duration and (iii) spatial coverage (He et al.,2018;Adede et al.,2019;Dai,2011).” ([Balti et al., 2020, p. 3](zotero://select/groups/4773535/items/V9YH9QFQ)) ([pdf](zotero://open-pdf/groups/4773535/items/JC4XTVWE?page=3&annotation=WPTV8SEV))
%instead of the first sentence: This complex concept of drought commonly encompassing .... dimensions of interrelated causes ... 

The concept of drought commonly encompasses multiple temporal, spatial and thematic dimensions. This complex conglomeration of interrelated causes and effects makes definition of \textit{droughts} a fairly multi-layered undertaking \autocite{balintMonitoringDroughtCombined2013}. Several well-known definitions (in this category) are for example from the \autocite{theamericanheritagedictionaryoftheenglishlanguageDrought2022} defining drought as "a long period of abnormally low rainfall, especially one that adversely affects growing or living conditions". \autocite[2]{palmerMeteorologicalDrought1965} defines drought as "a prolonged and abnormal moisture deficiency." or \autocite{vanloonDroughtHumanmodifiedWorld2016} defines droughts simply as "an exceptional lack of water compared to normal conditions". Other drought definitions emphasize its natural and/or human origin, its special characteristics, impact and temporal duration or even understand "drought as a system of causality where the link between causes and effects is random in nature {balintMonitoringDroughtCombined2013, baltiReviewDroughtMonitoring2020, idmpDroughtWaterScarcity2022,loonDroughtAnthropocene2016, wangPropagationDroughtMeteorological2016, wilhiteUnderstandingDroughtPhenomenon1985}. Already in the 1980s, \autocite{wilhiteUnderstandingDroughtPhenomenon1985} found more than 150 published definitions of drought. Besides the categorization into a conceptual or operational category , \autocite{wilhiteUnderstandingDroughtPhenomenon1985} proposed a clustering of these definitions into four types, namely meteorological drought, agricultural drought, hydrological drought and socio-economic drought. This classification is still widespread today \autocite{balintMonitoringDroughtCombined2013, baltiReviewDroughtMonitoring2020,idmpDroughtWaterScarcity2022,vereintenationenSpecialReportDrought2021}.

The conceptual category refers to a general formulation of an idea of drought to understand its concept and identify its boundaries and is often formulated in relative terms \autocite{wilhiteUnderstandingDroughtPhenomenon1985}. Definitions in the operational category try to define how drought functions in terms of its onset, duration, severity and spatial coverage also covering how this can be measured via indices \autocite{balintMonitoringDroughtCombined2013, nationaldroughtmitigationcenterWhatDrought, wilhiteUnderstandingDroughtPhenomenon1985}. With these definitions, the current situation is usually compared to a historical average, which is usually based on a 30-year period, which presupposes the development and continuous measurement of indicators and indices that can be used. \autocite{vereintenationenSpecialReportDrought2021,wilhiteUnderstandingDroughtPhenomenon1985}.

The four types of drought are commonly conceptually defined and brought into practice by operational specifications. They can be understood as different, but complementary stages of the same process and are generally cascading in reason and time but can overlap and are difficult to completely unravel. Figure \todo{TODO:, see https://drought.unl.edu/Education/DroughtIn-depth/TypesofDrought.aspx} shows an overview about the different types, their succession and cascading elements and table \todo{TODO: see RCRC 2020 p.11} displays the four types at a glance.

\missingfigure{This is just a test.}

The \textit{meteorological drought} is usually characterized by the duration and the degree of dryness in comparison to the normal average amount and try to conceptually understand how weather patterns can impact water availability. These definitions are specific for a regions atmospheric conditions, e.g. regions with a year-round precipitations regime such as tropical rainforest need different definitions and thresholds than e.g. climates characterized by seasonal rainfall patterns \autocite{nationaldroughtmitigationcenterTypesDrought}. Operational categorization mostly involves using precipitation, moisture, temperature and wind indicators to determine the onset, severity, and duration of the drought.

\textit{Agricultural drought} definitions establish a connection between different features of meteorological drought with their impacts on agriculture. Soil-moisture, differences between actual and potential evapotranspiration and soil water deficits are some of the operationalized indicators for monitoring this type of drought \autocite{baltiReviewDroughtMonitoring2020,nationaldroughtmitigationcenterTypesDrought,wilhiteUnderstandingDroughtPhenomenon1985}.

The type of \textit{hydrological drought} is associated with the impact of meteorological drought on surface or subsurface water resources such as rivers, lakes, and groundwater. Hydrological drought occurs when these indicators drop below normal levels \autocite{palmerMeteorologicalDrought1965}. The fastest responding indicator of this type of drought is most often the variability of streamflow. The water levels of lakes and groundwater usually lag behind the occurrence of the meteorological or agricultural drought which is why the hydrological drought is often out of phase with the previously mentioned types. The hydrological drought is commonly defined on the basis of watershed or river basin scale \autocite{baltiReviewDroughtMonitoring2020,nationaldroughtmitigationcenterTypesDrought,wilhiteUnderstandingDroughtPhenomenon1985}.

The \textit{socioeconomic drought} differs from the aforementioned types as it can also incorporate features of these types of drought to associate them with the demand and supply of some social or economic good. It therefore relates the impact of all other types of droughts on human population and its various sectors of society such as food security, health, and the economy. It is therefore sometimes also interchangeably used with drought impacts. Operational categorization involves using socioeconomic indicators such as unemployment rates and food prices to assess the severity and duration of the drought \autocite{nationaldroughtmitigationcenterTypesDrought,wilhiteUnderstandingDroughtPhenomenon1985}.

\missingfigure{This is just a test.}

The shown economic, social and environmental impacts of drought in figure \todo{TODO:} depend on the severity of, and the risk to drought. These three concepts of impact, severity and risk are interrelated concepts used to assess and understand the effects of drought on various sectors. 

\todo{TODO: insert definitions of severity, risk and impact}
% definitions of severity, risk and impact --> short! two sentences! not more -> severity is not that important, impact comes later again and risk as well so let's get going.


Thereby, in alignment with the definition of \autocite{vanloonDroughtHumanmodifiedWorld2016} it is the exceptional severity of the water shortage that distinguishes drought from aridity, a ordinarily recurrent or fully dry climate, and from water scarcity as a long-term "supply/demand and natural and/or human-made phenomenon" \autocites[7]{idmpDroughtWaterScarcity2022}{vereintenationenSpecialReportDrought2021, vanClimatologicalRiskDroughts2017}. Water scarcity is described in more detail in the following chapter \ref*{subsec:water_scarcity}.

%----------------------------------------------------------------------------------------
%	SUBSECTION 2.2 Water Scarcity
%----------------------------------------------------------------------------------------


\subsection{Water Scarcity}\label{subsec:water_scarcity}
% human induced water shortage component
% what about water security? “Water Security: A Complex Concept” ([Butte et al., 2022, p. 1](zotero://select/groups/4773535/items/QB97YZ2M)) ([pdf](zotero://open-pdf/groups/4773535/items/Q936I2JN?page=1&annotation=XD3AGTA6))
% and insecurity? “Progress in household water insecurity metrics: a crossdisciplinary approach” ([Jepson et al., 2017, p. 1](zotero://select/groups/4773535/items/HWX5JRS4)) ([pdf](zotero://open-pdf/groups/4773535/items/NHEUUZI9?page=1&annotation=SE7MN8X2))

Water scarcity, as for drought or water security, is defined in many different ways. The sixth IPCC Assessment Report defines water scarcity broadly as "a mismatch between the demand for fresh water and its availability, quantified in physical terms" \autocite[560]{caretta2022water}. Here, social and economic components are outsourced to the broader concept of water security and insecurity, focussing primarily on physical water scarcity \autocite{caretta2022water}. In contrast, the Food and Agricultural Organization of the United Nations defines water scarcity as "a gap between available supply and expressed demand of freshwater in a specified domain, under prevailing institutional arrangements (including both resource ‘pricing’ and retail charging arrangements) and infrastructural conditions" \autocite[5]{faoCopingWaterScarcity2012} further summarizing that water security is "an excess of water demand over available supply" \autocite[6]{faoCopingWaterScarcity2012}. Thus, highlighting the human dimension of this interactive and relative concept of physical and economic water scarcity. Hereby, physical water scarcity refers to a situation in which there is not enough water available in quantitative terms to meet demand whereas economic water scarcity occurs when inadequate infrastructure, institutional or financial capital obstructs access to water resources "even though water in nature is available to meet human demands" \autocites{idmpDroughtWaterScarcity2022}[11]{moldenWaterFoodWater2007}.
Water scarcity and drought are in a complex interrelationship with each other. A short overview about the key differences between water scarcity and drought are given in table \todo{TODO:“Table 1. Characteristics and impacts of water scarcity and drought” ([IDMP, 2022, p. 3](zotero://select/groups/4773535/items/LNSL8VD2)) ([pdf](zotero://open-pdf/groups/4773535/items/JM82W3ZF?page=9&annotation=QNC4A3FG))}. 


\missingfigure{differences water scarcity to drought}
% Distinctions between water scarcity and drought:

% also in Joplin

%“Table 1. Characteristics and impacts of water scarcity and drought Water scarcity Drought Length Long-term to permanent Temporary (weeks to multiyear) Driving forces Demand–supply imbalance, human-driven, and/or natural (overexploitation, pollution). Climate change can impact both supply and demand Natural climate variability which can be modified/amplified by climate change Potential impacts Restricted water availability, environmental degradation, desertification, exacerbated inequalities in access to water resources, potential competition Water shortages, competition, environmental degradation Measures Long-term IWRM to bring supply and demand back into sustainable balance Integrated drought management, including: (1) monitoring and early warning; (2quality) vulnerability and impact assessment; and (3) risk mitigation, preparedness and response Source: adapted from Hohenwallner et al. (2011) DROUGHT AND WATER SCARCITY – DEFINITIONS AND CHARACTERISTICS” ([pdf](zotero://open-pdf/groups/4773535/items/JM82W3ZF?page=9&annotation=E3EQRILA))




Furthermore, potential mutual reinforcements, climate change, increased water use and poor water management can make it sometimes difficult to clearly separate these concepts \autocite{idmpDroughtWaterScarcity2022,lealfilhoUnderstandingResponsesClimaterelated2022,liuWaterScarcityAssessments2017,rcrcFORECASTBASEDFINANCINGEARLY2020}. Nonetheless, following the definition of \autocite{faoCopingWaterScarcity2012} the concept of water scarcity always gives water shortage a human dimension in particular on the demand side. The quality of policies, planning and management on the demand side can be seen as critical to the overall severity of the impact of water scarcity \autocite{idmpDroughtWaterScarcity2022,faoCopingWaterScarcity2012,vereintenationenSpecialReportDrought2021}. The supply side can be influenced by human activities, but it is not a mandatory prerequisite. \autocite{idmpDroughtWaterScarcity2022}. 

Besides the already mentioned water scarcity on the basis of physical quantity and economical factors, water scarcity can be caused by water of unacceptable quality and lack of access to water services \autocite{faoCopingWaterScarcity2012}. 


Acknowledging water quality induced water scarcity as an additional factor is only a relatively new development in literature \autocite{liuThreedimensionalWaterScarcity2020} but together with inadequate access highlights further challenges in ensuring water security \autocite{caretta2022water, mishraWaterSecurityChanging2021}


%-----------------------------------
%	SUBSECTION 2.3 Water Access & Water Quality
%-----------------------------------

\subsection{Water Access \& Water Quality}
% + evtl. human health related water borne diseases and CBS

“Economic water scarcity can also occur where infrastructure exists, but water distribution is inequitable (Jaeger et al., 2017).” ([Caretta et al., 2022, p. 560](zotero://select/groups/4773535/items/W8C83PVC)) ([pdf](zotero://open-pdf/groups/4773535/items/VDJIEIJT?page=10&annotation=D4DJHC8Y))


“There are three main facets to urban water access: availability, quality, and affordability.” ([Mitlin et al., p. 11](zotero://select/groups/4773535/items/KAM9REZR)) ([pdf](zotero://open-pdf/groups/4773535/items/BM6BU5UR?page=11&annotation=44IJC8RC))

“Household-Reported Availability of Drinking Water in Africa: A Systematic Review” ([Thomas et al., 2020, p. 1](zotero://select/groups/4773535/items/PICD9PFW)) ([pdf](zotero://open-pdf/groups/4773535/items/P4AM2TVM?page=1&annotation=V4LFXHU8))

“Drinking water accessibility and quantity in low and middle-income countries: A systematic review” ([Cassivi et al., 2019, p. 1011](zotero://select/groups/4773535/items/Y2T649J9)) ([pdf](zotero://open-pdf/groups/4773535/items/42UK29GK?page=1&annotation=6RW4YRLY))

“Going beyond basic access to improved water sources: Towards deriving a water accessibility index” ([Lester and Rhiney, 2018, p. 129](zotero://select/groups/4773535/items/FFZUMH2N)) ([pdf](zotero://open-pdf/groups/4773535/items/AY9B3FVS?page=1&annotation=72KSD29V))

“Countries in which less than 50\% of the population uses improved drinking water sources are all located in sub-Saharan Africa and Oceania” ([World Health Organization, 2016, p. 15](zotero://select/groups/4773535/items/KVAKZ9ZT)) ([pdf](zotero://open-pdf/groups/4773535/items/4STYK52H?page=14&annotation=PSKH53UZ))

“► Equitable access to safe, reliable, and affordable water is a human right. Urban water provision is a social good, but one that will become increasingly difficult for cities and water utilities to provide due to climate change and population growth.” ([Mitlin et al., p. 1](zotero://select/groups/4773535/items/KAM9REZR)) ([pdf](zotero://open-pdf/groups/4773535/items/BM6BU5UR?page=1&annotation=W3NNXNJA))

“Unaffordable and Undrinkable: Rethinking Urban Water Access in the Global South” ([Mitlin et al., p. 1](zotero://select/groups/4773535/items/KAM9REZR)) ([pdf](zotero://open-pdf/groups/4773535/items/BM6BU5UR?page=1&annotation=HSCX925W))

!!!!!
“Adequate and reliable weather, water, and climate data and applications are needed to monitor available water resources and provide actionable early warning for water scarcity and drought conditions.” ([pdf](zotero://open-pdf/groups/4773535/items/JM82W3ZF?page=17&annotation=67GM7DYS))
“as well as improved capacities in collecting hydrological data. Improved interaction with stakeholders is crucial to promote better tailored information products.” ([pdf](zotero://open-pdf/groups/4773535/items/JM82W3ZF?page=18&annotation=CBKCBSZV))


“as well as improved capacities in collecting hydrological data. Improved interaction with stakeholders is crucial to promote better tailored information products.” ([pdf](zotero://open-pdf/groups/4773535/items/JM82W3ZF?page=18&annotation=CBKCBSZV))

“Additionally, several of the existing tools were found to be very specific and limited physical and hydrological aspects, failing to incorporate important factors such as socio-economic and political characteristics.” ([Butte et al., 2022, p. 17](zotero://select/groups/4773535/items/QB97YZ2M)) ([pdf](zotero://open-pdf/groups/4773535/items/Q936I2JN?page=17&annotation=KWSSFHK2))

“These findings underscore the dangers of separating (or ringfencing) responses to water scarcity from competing challenges to food security, urbanization, desertification, and human or state security.” ([Leal Filho et al., 2022, p. 11](zotero://select/groups/4773535/items/CBFHWKLP)) ([pdf](zotero://open-pdf/groups/4773535/items/CG4XVTAI?page=11&annotation=DG37MVHB))

“One of main challenges for adaptation for the coming decade is to extend planned adaptation at the local level and better integrate projected risk of climate change and variability into local autonomous responses.” ([Leal Filho et al., 2022, p. 11](zotero://select/groups/4773535/items/CBFHWKLP)) ([pdf](zotero://open-pdf/groups/4773535/items/CG4XVTAI?page=11&annotation=K4DUGJBX))

“The discrepancy between physical water scarcity and overall water insecurity is a function of socioeconomic vulnerabilities and governance gaps. Therefore, improving societal aspects of water management will be key in adapting to climate change-driven increases in water scarcity in the future (high confidence).” ([Caretta et al., 2022, p. 563](zotero://select/groups/4773535/items/W8C83PVC)) ([pdf](zotero://open-pdf/groups/4773535/items/VDJIEIJT?page=13&annotation=62FBCSBU))

conceptual basis for the next chapter, how to prevent/anticipate this in order to prevent rather than react

whatever source of water they can find is what they have. (2023-03-04_Beledi, Pos. 20)

“Water, sanitation, hygiene and rural poverty: Issues of sector monitoring and the role of aggregated indicators” ([Gine and Pérez-Foguet, 2013, p. 1](zotero://select/groups/4773535/items/32UPYQC7)) ([pdf](zotero://open-pdf/groups/4773535/items/FMQ5Y4BP?page=1&annotation=D4XLEESS))


% possibly add the definitions of water sources here (?) improved, unimproved etc. https://www.wearewater.org/en/what-do-we-mean-when-we-talk-about-access-to-water_346091

%----------------------------------------------------------------------------------------
%	SECTION 4 FbF, EAP, AA & Early Warning
%----------------------------------------------------------------------------------------


\section{FbF, EAP, AA & Early Warning + trigger}
% one possible solution to prevent impact
% FbF
% IFRC & RCRC
% EAPs
% Early Warning/Actions &  Anticipatory Actions
% triggered by forecast

%-----------------------------------
%	SUBSECTION 4.1
%-----------------------------------
\subsection{FbF and paradigm shift}
--> FbF and EAP what is what and so on --> in context of this work


%% FbA Challenges
“First, FbA for drought presents a challenge of framing and definition” ([RCRC, 2020, p. 24](zotero://select/groups/4773535/items/UESIQTRJ)) ([pdf](zotero://open-pdf/groups/4773535/items/P5JPVZ97?page=24&annotation=F2K3MP3C))

“FbA for drought is contextually challenging” ([RCRC, 2020, p. 24](zotero://select/groups/4773535/items/UESIQTRJ)) ([pdf](zotero://open-pdf/groups/4773535/items/P5JPVZ97?page=24&annotation=EQTN72CS))

“structural challenges” ([RCRC, 2020, p. 24](zotero://select/groups/4773535/items/UESIQTRJ)) ([pdf](zotero://open-pdf/groups/4773535/items/P5JPVZ97?page=24&annotation=EXGZA744))

“Early warning Early Action (EWEA) mitigates the risk of vulnerable people adopting extreme negative coping strategies or becoming dependent on aid. Additionally, a growing body of evidence demonstrates that early action is far more cost-effective than late humanitarian response. Two recent studies on early response and disaster resilience in Ethiopia, Kenya and Somalia showed that even if there is uncertainty that a high-magnitude drought will occur, the cost difference is so significant that investing in early action measures is still more cost-effective.1,2” ([Gualazzini, 2021, p. 3](zotero://select/groups/4773535/items/BWDYDL8T)) ([pdf](zotero://open-pdf/groups/4773535/items/8U5XVU5K?page=3&annotation=BGQSB3HK))

“Implementing early action measures helps guard against impending shocks, and early response addresses short-term humanitarian needs while protecting overarching resilience gains.” ([Gualazzini, 2021, p. 4](zotero://select/groups/4773535/items/BWDYDL8T)) ([pdf](zotero://open-pdf/groups/4773535/items/8U5XVU5K?page=4&annotation=CISMIJ8W))
%-----------------------------------
%	SUBSECTION 4.2
%-----------------------------------



\subsection{EAP and drought specifics}
%% other RCRC protocols and EAPs
“FORECAST-BASED FINANCING AND EARLY ACTION FOR DROUGHT” ([pdf](zotero://open-pdf/groups/4773535/items/P5JPVZ97?page=1&annotation=IL955QPI))

%% key differences between fast-onset disasters and drought.
“Uniqueness of droughts to RCRC FbA experience” ([pdf](zotero://open-pdf/groups/4773535/items/P5JPVZ97?page=19&annotation=6MZDVD5E))

building on the above mentioned blocks of drought forecasts, local knowledge co-production and water source monitoring, 

“FORECAST-BASED FINANCING AND EARLY ACTION FOR DROUGHT” ([RCRC, 2020, p. 1](zotero://select/groups/4773535/items/UESIQTRJ)) ([pdf](zotero://open-pdf/groups/4773535/items/P5JPVZ97?page=1&annotation=IL955QPI))

“2.Challenges for drought IbF” ([Boult et al., 2022, p. 3](zotero://select/groups/4773535/items/B2AQSTYL)) ([pdf](zotero://open-pdf/groups/4773535/items/W9TFLH43?page=3&annotation=7B52FIS8))

“Towards the Development of a Rule-based Drought Early Warning Expert Systems using Indigenous Knowledge” ([Akanbi and Masinde, 2018, p. 2](zotero://select/groups/4773535/items/7EGUL4GK)) ([pdf](zotero://open-pdf/groups/4773535/items/5K6KJRLZ?page=2&annotation=F4XUAKFF))

“Feasibility Study on Potential Use of Forecast-based Financing (FbF) for SRCS Final Report” ([Somali Red Crescent Society, 2022, pp. -3](zotero://select/groups/4773535/items/FZ6BJHJA)) ([pdf](zotero://open-pdf/groups/4773535/items/RJKNZZZ2?page=1&annotation=KHCH33GX))
%-----------------------------------
%	SUBSECTION 4.2
%-----------------------------------

\subsection{Anticipatory Actions and Early Warning}
%% Suggested AA
The feasibility study identifies a number of high potential early actions in response to anticipated drought, including:

Water storage rehabilitation – berked repairs and re-charging can be undertaken at a reasonable cost with available skills within the lead time afforded by a drought forecast; (https://heigit.atlassian.net/wiki/spaces/FIS/pages/1704186/Background+Information#BackgroundInformation-SuggestedAA)

“water vouchers are the leveraging of existing market capacity and the leading role of CRCs in community in price negotiation, supervision of water delivery through community vouchers and monitoring of water quantity and quality.” ([Gualazzini, 2021, p. 13](zotero://select/groups/4773535/items/BWDYDL8T)) ([pdf](zotero://open-pdf/groups/4773535/items/8U5XVU5K?page=13&annotation=RXEH724K))

“The value for money (VFM) analysis showed how the communitybased negotiation approach led to savings of USD 20,173 compared to the classic water trucking modality.” ([Gualazzini, 2021, p. 13](zotero://select/groups/4773535/items/BWDYDL8T)) ([pdf](zotero://open-pdf/groups/4773535/items/8U5XVU5K?page=13&annotation=QY5PKXHV))

“Communities also reported a significant reduction in diarrhoea cases, likely due to water chlorination and increased water availability for hygiene purposes.” ([Gualazzini, 2021, p. 13](zotero://select/groups/4773535/items/BWDYDL8T)) ([pdf](zotero://open-pdf/groups/4773535/items/8U5XVU5K?page=13&annotation=57W6V43K))

“In the absence of water price regulation policy and enforcement, the price of water from wells and boreholes is abnormally high during the dry season.” ([Gualazzini, 2021, p. 13](zotero://select/groups/4773535/items/BWDYDL8T)) ([pdf](zotero://open-pdf/groups/4773535/items/8U5XVU5K?page=13&annotation=WKNG9T3Y))

“Those vulnerable households that cannot afford safe water often significantly reduce consumption or turn to unprotected water sources. This water insecurity during the dry seasons is further exacerbated by drought conditions after one or two consecutive belowaverage rainy seasons. In such instances, surface waters deplete earlier than usual, and poor harvests and pasture further diminish households’ purchasing power.” ([Gualazzini, 2021, p. 13](zotero://select/groups/4773535/items/BWDYDL8T)) ([pdf](zotero://open-pdf/groups/4773535/items/8U5XVU5K?page=13&annotation=L4VA7DR7))

“community water vouchers for safe water access during the last two months of the dry season is an early action measure to reduce the risk of disease outbreak, malnutrition and displacement in search of humanitarian assistance and water.” ([Gualazzini, 2021, p. 13](zotero://select/groups/4773535/items/BWDYDL8T)) ([pdf](zotero://open-pdf/groups/4773535/items/8U5XVU5K?page=13&annotation=YJHP3TBG))

“This approach is an appropriate alternative to the water trucking method in areas where there is a functional market of private water vendors already serving communities without permanent water sources.” ([Gualazzini, 2021, p. 13](zotero://select/groups/4773535/items/BWDYDL8T)) ([pdf](zotero://open-pdf/groups/4773535/items/8U5XVU5K?page=13&annotation=QCS5846H))

“The main characteristics of community” ([Gualazzini, 2021, p. 13](zotero://select/groups/4773535/items/BWDYDL8T)) ([pdf](zotero://open-pdf/groups/4773535/items/8U5XVU5K?page=13&annotation=79UZ7AXG))

% possibly include just as an example
“In humanitarian practice, the term "drought" is often used to refer to some socio-meteorological combination where water shortages produce stress on human and livelihood systems. Droughts are a function of the fragility of human systems, and they become disasters where systems cannot cope with deviations from the hydro-meteorological norm. It has been argued that droughts are particularly devastating when livelihood choices are strongly determined by the climate (e.g. the decision to grow certain crops, or traditional seasonal migration patterns) - for instance, if in a given year, the weather patterns are different than normal, those livelihoods are especially vulnerable to these changes. It has also been argued that droughts pose specific challenges to income generating activities marked by low productivity that are not able to take advantage of ‘good years’ in order to provide a buffer during ‘bad years’.” ([RCRC, 2020, p. 12](zotero://select/groups/4773535/items/UESIQTRJ)) ([pdf](zotero://open-pdf/groups/4773535/items/P5JPVZ97?page=12&annotation=C622YWIR))

%% DEWS Drougth Early Warning Systems
"The study concluded that information from DEWS was of good quality although it was poorly disseminated. There is need to enhance the timeliness of information dissemination if the system is to effectively enhance community preparedness to cope the effects of drought." Akwango, D., Obaa, B. B., Turyahabwe, N., Baguma, Y., & Egeru, A. (2017). Quality and dissemination of information from a drought early warning system in Karamoja sub-region, Uganda. Journal of Arid Environments, 145, 69–80. https://doi.org/10.1016/j.jaridenv.2017.05.010 (already in zotero)


“Therefore, the impacts of drought could be mitigated by managing water demand through crop management, modifying water allocation rules during times of water scarcity, developing various water resources (such as groundwater recharge and salt water desalination), managing multiple water use, setting up water-trading mechanisms in advance of times of drought or scarcity, and physically redistributing available supplies during times of scarcity [26].” ([Wang et al., 2016, p. 2](zotero://select/groups/4773535/items/AWKLZ543)) ([pdf](zotero://open-pdf/groups/4773535/items/4YRNMM9P?page=2&annotation=S8QKSIPL))

%-----------------------------------
%	SUBSECTION 4.3 Trigger selection
%-----------------------------------

\subsection{Trigger selection}


what is a trigger? definitions and thresholds + introduction for indicators

“Given the different layers of complexity with drought, different types of triggers may be required beyond what is often used in EAP development. For instance, unconventional triggers for FbA for drought could include metrics such as staple food prices, percentages of crop failure, and other elements of food security early warning systems.” ([RCRC, 2020, p. 30](zotero://select/groups/4773535/items/UESIQTRJ)) ([pdf](zotero://open-pdf/groups/4773535/items/P5JPVZ97?page=30&annotation=JZV26DPP))
% --> even the RCRC is still looking for good triggers -> maybe water levels are a good way -> reasoning for this study


%% Trigger:
“Triggers are mainly combination of hydro-meteorological forecast combined with exposure and vulnerability data” ([pdf](zotero://open-pdf/groups/4773535/items/P5JPVZ97?page=19&annotation=ILGX4MS6))

% --> water level monitoring is one trigger for in the chain of triggers (right before the last action so to say)
% compare: “Figure 5. Contrasting the timeline between fast-onset hazards and droughts” ([RCRC, 2020, p. 20](zotero://select/groups/4773535/items/UESIQTRJ)) ([pdf](zotero://open-pdf/groups/4773535/items/P5JPVZ97?page=20&annotation=Q4LRCBC4))
“As explained in the previous section, droughts do not have clear start and end dates, but the timing of below average rainfall matters deeply. Particularly for crops and forage, a dry spell at the beginning of the planting season can be particularly devastating for crop yields. As such, in order to act in anticipation, triggering systems must be based on monitoring and forecasting at the right times to capture these events and act early.” ([RCRC, 2020, p. 29](zotero://select/groups/4773535/items/UESIQTRJ)) ([pdf](zotero://open-pdf/groups/4773535/items/P5JPVZ97?page=29&annotation=KIUB7HJP))

%possibly talk about this trigger as well
“BRCiS community-based real-time monitoring indicators and thresholds summary” ([Gualazzini, 2021, p. 6](zotero://select/groups/4773535/items/BWDYDL8T)) ([pdf](zotero://open-pdf/groups/4773535/items/8U5XVU5K?page=6&annotation=G26DY9D3))

%% critical water level
“Primary water source condition Observation of the status of the largest rainwater catchment in the area (at the end of rainy seasons) More than half-full (75%) or full Half-full (50%) at the end of rainy season Less than half-full (25%)/empty” ([Gualazzini, 2021, p. 6](zotero://select/groups/4773535/items/BWDYDL8T)) ([pdf](zotero://open-pdf/groups/4773535/items/8U5XVU5K?page=6&annotation=HMGNP355))

%% staggered triggering system
“staggered triggering system (see Annex 3)” ([RCRC, 2020, p. 26](zotero://select/groups/4773535/items/UESIQTRJ)) ([pdf](zotero://open-pdf/groups/4773535/items/P5JPVZ97?page=26&annotation=4JB7JVDV))

“However, with the longer outlook of seasonal forecasts, comes more uncertainty, less granularity and lower accuracy of the prediction.” ([RCRC, 2020, p. 26](zotero://select/groups/4773535/items/UESIQTRJ)) ([pdf](zotero://open-pdf/groups/4773535/items/P5JPVZ97?page=26&annotation=55RDIUHY))

“Notably, situations and forecasts can change throughout the seasons. This presents a challenge for the development of robust triggering systems and involves a heightened risk of false alarms, which have been shown to quickly erode the trust necessary for humanitarian presence. For these reasons, a clear and comprehensive understanding of the seasonality of the region’s climate, and of context-specific tipping points, is a fundamental first step to any FbA for drought program.” ([RCRC, 2020, p. 26](zotero://select/groups/4773535/items/UESIQTRJ)) ([pdf](zotero://open-pdf/groups/4773535/items/P5JPVZ97?page=26&annotation=IY3B3635))

--> this work -> on the ground indicators/impacts -> low risk for false alarms

%% some conditions for an effective trigger for FbA for drought
“1. Sufficient historical data on past droughts, their causes and impacts 

2. Identified drivers of rainfall predictability in the region (if forecasts are going to be used, and not triggering entirely on observations in anticipation of the impacts) or else sufficient rainfall observations 

3. Sufficient knowledge of livelihood profiles in the region and knowledge of differential impacts of drought conditions on livelihood groups.” ([RCRC, 2020, p. 25](zotero://select/groups/4773535/items/UESIQTRJ)) ([pdf](zotero://open-pdf/groups/4773535/items/P5JPVZ97?page=25&annotation=AJ4JI2QH))




“Thinking outside the box in terms of both hydro-meteorological and socio-economic indicators could be particularly useful” ([RCRC, 2020, p. 31](zotero://select/groups/4773535/items/UESIQTRJ)) ([pdf](zotero://open-pdf/groups/4773535/items/P5JPVZ97?page=31&annotation=GNZJ3FR5))


“Drought Severity = Intensity x Duration x Magnitude x Frequency” ([RCRC, 2020, p. 13](zotero://select/groups/4773535/items/UESIQTRJ)) ([pdf](zotero://open-pdf/groups/4773535/items/P5JPVZ97?page=13&annotation=MKU448SN))


good, precise definitions of drought in an easy way
“Droughts vary by intensity, duration, timing, and geographical coverage, creating conditions of limited moisture availability to a potentially damaging extent.” ([pdf](zotero://open-pdf/groups/4773535/items/JM82W3ZF?page=10&annotation=ZR4SQ4LB))

“Many practitioners and experts interviewed for this work suggested that a staggered triggering system, at different lead times and for different early actions may be the most appropriate to tackle drought” ([RCRC, 2020, p. 31](zotero://select/groups/4773535/items/UESIQTRJ)) ([pdf](zotero://open-pdf/groups/4773535/items/P5JPVZ97?page=31&annotation=X34YUJMF))


“Uniqueness of droughts to RCRC FbA experience” ([RCRC, 2020, p. 19](zotero://select/groups/4773535/items/UESIQTRJ)) ([pdf](zotero://open-pdf/groups/4773535/items/P5JPVZ97?page=19&annotation=6MZDVD5E))

“A. Long and Unclear Temporal Framing” ([RCRC, 2020, p. 20](zotero://select/groups/4773535/items/UESIQTRJ)) ([pdf](zotero://open-pdf/groups/4773535/items/P5JPVZ97?page=20&annotation=KJXT7VD5))

“The impacts of drought are therefore far reaching, arguably more so than slow-onset hazards, and arrive at different times for different groups. Sifting through these impacts to identify which ones to address through early action (and when to begin them) can be particularly difficult.” ([RCRC, 2020, p. 23](zotero://select/groups/4773535/items/UESIQTRJ)) ([pdf](zotero://open-pdf/groups/4773535/items/P5JPVZ97?page=23&annotation=XNFC6ZA8))

%----------------------------------------------------------------------------------------
%	SECTION 5 Forecasting (SPI, EDDI, SPEI) vs. impact based
%----------------------------------------------------------------------------------------


\section{Drought monitoring, Forecasting (SPI, EDDI, SPEI) vs. impact based}
% based on the triggers, those are the forecasts for now -> strengths and limitations which trigger?

% presentation of different forecasts and which are used
% well established, published by RCRC Climate Center etc.
% mostly based on satellite analysis -> limitation: global

% impact based forecasts
% difficult to assess
% possibilities and opportunities

% which trigger?

\subsection{Drought Indicators and Indices}
% multi hazard
% longterm
% different drought difinitions
% + RISK and vulnerability indicators “Figure 1.10. Differing relevance of vulnerability indicators to drought impacts on agricultural systems and water supply” ([“Special report on drought 2021”, 2021, p. 66](zotero://select/groups/4773535/items/RAAM9PVS)) ([pdf](zotero://open-pdf/groups/4773535/items/7AK5QVBL?page=68&annotation=DBU4VJVQ))

“Summary of the Characteristics of Water Scarcity Indicators” ([Liu et al., 2017, p. 546](zotero://select/groups/4773535/items/6LCA67BF)) ([pdf](zotero://open-pdf/groups/4773535/items/XDWCZPNG?page=2&annotation=Q5M3BKG5))



current monitoring approaches e.g. (“the extent to which volunteers’ assessments of dry-to-wet conditions correspond to objective drought indicators (EDDI, SPI, SPEI) typically employed for monitoring drought” ([Lackstrom et al., 2022, p. 2](zotero://select/groups/4773535/items/YI366LQY)) ([pdf](zotero://open-pdf/groups/4773535/items/3JTQ72UN?page=2\&annotation=DQ4ZNIRS)))
“quantitative indicators (namely the SPEI, SPI, and EDDI).” ([Lackstrom et al., 2022, p. 26](zotero://select/groups/4773535/items/YI366LQY)) ([pdf](zotero://open-pdf/groups/4773535/items/3JTQ72UN?page=26\&annotation=G4KIJYUR))

“Monitoring Drought with the Combined Drought Index in Kenya” ([Balint et al., 2013, p. 1](zotero://select/groups/4773535/items/C6T6K9AP)) ([pdf](zotero://open-pdf/groups/4773535/items/ZT9SQQ2M?page=1\&annotation=DMSJ4J63))
introduction is good.

“2 Local knowledge in drought monitoring: an introduction to the literature review” ([Giordano et al., 2013, p. 526](zotero://select/groups/4773535/items/B7LM5ZR4)) ([pdf](zotero://open-pdf/groups/4773535/items/7I66DBIK?page=4&annotation=Z33M5FLQ))

“The Palmer Drought Severity Index (PDSI), developed in 1965 by W. C. Palmer [21], is probably the best-known meteorologic drought definition in the United States and is well known internationally.” ([Wilhite and Glantz, 1985, p. 5](zotero://select/groups/4773535/items/W7CX6KJP)) ([pdf](zotero://open-pdf/groups/4773535/items/X6YHDSPK?page=6&annotation=R6QLIRDZ))


“Drought is a slowly reoccurring, natural phenomenon with a complex development pattern. This natural hazard is often characterized as a deviation of weather and climate variations (and change) as manifested in parameters such as precipitation, soil moisture, groundwater, and streamflow from normal conditions. It exhibits spatial and temporal characteristics that vary significantly from one region to another. Unlike other hazards (e.g., floods), drought gradually develops over a broader area for a longer period of time and only becomes recognized after it is well developed [1]” ([Botai et al., 2019, p. 1](zotero://select/groups/4773535/items/MG66Q545)) ([pdf](zotero://open-pdf/groups/4773535/items/AEMF65U7?page=1&annotation=LXX6TW6N))



“1.2.2 Drought indicators” ([“Special report on drought 2021”, 2021, p. 28](zotero://select/groups/4773535/items/RAAM9PVS)) ([pdf](zotero://open-pdf/groups/4773535/items/7AK5QVBL?page=30&annotation=ZR477ZFL))

"Drought indicator" and "drought index" are two related but distinct terms used in the field of drought monitoring and assessment.

A drought indicator is a variable or set of variables that provide information on the current or potential occurrence of drought. For example, a drought indicator might be a measure of soil moisture or streamflow that can be used to assess drought conditions.

A drought index, on the other hand, is a composite measure that combines multiple drought indicators to provide a more comprehensive assessment of drought severity and duration. Drought indices are typically based on mathematical formulas that take into account various environmental factors, such as precipitation, temperature, and soil moisture.

In summary, a drought indicator is a single variable that provides information on drought conditions, while a drought index is a composite measure that combines multiple indicators to provide a more complete picture of drought severity and duration. While both terms are used in the context of drought monitoring and assessment, drought indices are generally considered to be more comprehensive and reliable measures of drought.

drought indicator (what exists so far? -> pre study)

“A. Hazard-related indicators and indices” ([RCRC, 2020, p. 15](zotero://select/groups/4773535/items/UESIQTRJ)) ([pdf](zotero://open-pdf/groups/4773535/items/P5JPVZ97?page=15&annotation=KZ5ITR9D))

“B. Vulnerability-related indicators” ([RCRC, 2020, p. 17](zotero://select/groups/4773535/items/UESIQTRJ)) ([pdf](zotero://open-pdf/groups/4773535/items/P5JPVZ97?page=17&annotation=BTLXTKJG))

% List of drought indicators
https://heigit.atlassian.net/wiki/spaces/FIS/pages/1704096/Indices

in general: https://edo.jrc.ec.europa.eu/edov2/php/index.php?id=1010

explanation and overview about drought indices “Drought prediction based on SPI and SPEI with varying timescales using LSTM recurrent neural network” ([Poornima and Pushpalatha, 2019, p. 8399](zotero://select/groups/4773535/items/NJME9MIM)) ([pdf](zotero://open-pdf/groups/4773535/items/48LYF6PR?page=1&annotation=FV2JBM9I))

SPI - standardized Precipitation Index (most widely used drought indices (DI)) 
https://climatedataguide.ucar.edu/climate-data/standardized-precipitation-index-spi
http://iridl.ldeo.columbia.edu/maproom/Global/Drought/Global/index.html

"The experimental Global Gridded Standardized Precipitation Index (SPI) is derived from the NOAA CMORPH dataset and includes timescales of 1, 3, 6 and 9 months.  The NOAA CMORPH precipitation dataset is a gridded dataset derived from combining numerous microwave-based estimates from low orbiter satellites." https://www.drought.gov/data-maps-tools/global-gridded-standardized-precipitation-index-spi-cmorph-daily + related publications

EDDI (Evaporative Demand Drought Index) ("examines how anomalous the atmospheric evaporative demand (E0; also known as "the thirst of the atmosphere) is for a given location and across a time period of interest." \& "EDDI has been shown to offer early warning of drought stress relative to current operational drought indicators, such as the U.S. Drought Monitor (USDM). A particular strength of EDDI is in capturing the precursor signals of water stress at weekly to monthly timescales, which makes EDDI a potent tool for drought preparedness at those timescales. EDDI also uses the same classification scheme as the USDM to define drought conditions, so it is easy to read EDDI maps."https://www.drought.gov/data-maps-tools/evaporative-demand-drought-index-eddi-subseasonal-forecasts)

SPEI (Standardised Precipitation-Evapotranspiration Index https://spei.csic.es/)
"The SPEI is a multiscalar drought index based on climatic data. It can be used for determining the onset, duration and magnitude of drought conditions with respect to normal conditions in a variety of natural and managed systems such as crops, ecosystems, rivers, water resources, etc."https://spei.csic.es/

key strength and key limitations https://climatedataguide.ucar.edu/climate-data/standardized-precipitation-evapotranspiration-index-spei


more indices: https://www.droughtmanagement.info/indices/

https://droughtmonitor.unl.edu/


"One example of a drought index is the Standardized Precipitation Index (SPI), which is a widely used measure of meteorological drought.

The SPI uses precipitation data to assess how much rainfall is available relative to the long-term average for a given location and time period. The SPI is calculated by standardizing the observed precipitation data over a specified time period (e.g., 3 months, 6 months, 12 months) to create a normal distribution with a mean of 0 and a standard deviation of 1.

Positive SPI values indicate wetter than average conditions, while negative SPI values indicate drier than average conditions. The magnitude of the SPI value indicates the severity of the drought, with more negative values indicating more severe drought conditions. The SPI is a useful tool for monitoring and predicting meteorological droughts, as it can be calculated using readily available precipitation data and provides a standardized measure that can be compared across different locations and time periods." ChatGPT


“Droughts and their impacts have been studied for many decades in the Horn of Africa.” ([Balint et al., 2013, p. 2](zotero://select/groups/4773535/items/C6T6K9AP)) ([pdf](zotero://open-pdf/groups/4773535/items/ZT9SQQ2M?page=2&annotation=XYFG3XJI))

“Most of these studies have been largely based on one or two variables that are linked to the characterization of the climate such as rainfall, temperature, vegetation and soil moisture (Shanko and Camberlin, 1998; Ntale and Gan, 2003; Hastenrat et al., 2007)” ([Balint et al., 2013, p. 2](zotero://select/groups/4773535/items/C6T6K9AP)) ([pdf](zotero://open-pdf/groups/4773535/items/ZT9SQQ2M?page=2&annotation=SYNA96CD))

“None of these approaches have been able to fully trace the drought footprints and offer reliable prediction in the Horn of Africa.” ([Balint et al., 2013, p. 2](zotero://select/groups/4773535/items/C6T6K9AP)) ([pdf](zotero://open-pdf/groups/4773535/items/ZT9SQQ2M?page=2&annotation=LPNE3XVS))

“drought is a composite natural phenomenon whose monitoring methodology should be able to include at least six variables simultaneously, namely, rainfall deficit and its persistence, soil moisture deficit and its persistence, and temperature excess and its persistence.” ([Balint et al., 2013, p. 2](zotero://select/groups/4773535/items/C6T6K9AP)) ([pdf](zotero://open-pdf/groups/4773535/items/ZT9SQQ2M?page=2&annotation=HNF3IDCP))

Drought indices vs. drought forecast
Drought indices and drought forecasts are two different types of tools used to assess and manage drought conditions. The main differences between the two are as follows:
Definition and purpose: Drought indices are statistical measures that describe the severity and duration of drought conditions based on historical and current weather data. They provide a snapshot of current conditions and serve as an indicator of the overall drought situation. Drought forecasts, on the other hand, use climate models and weather data to predict future drought conditions and provide early warning of potential drought events.
Timeframe: Drought indices provide a retrospective analysis of current drought conditions, while drought forecasts look ahead to potential drought conditions. Drought indices are typically calculated over a period of weeks or months, while drought forecasts can extend months or even years into the future.
Data requirements: Drought indices rely on historical and current weather data to calculate indicators of drought severity, such as soil moisture, streamflow, and precipitation. Drought forecasts, on the other hand, use complex climate models and a range of environmental data sources to generate probabilistic predictions of future drought conditions.
Usefulness: Drought indices are useful for tracking the progression of drought over time and can provide information to policymakers and resource managers to make decisions about water allocation and conservation measures. Drought forecasts are useful for predicting the likelihood and severity of future drought events, which can help policymakers and resource managers to plan for drought response and mitigation measures.
In summary, drought indices are retrospective measures of current drought conditions, while drought forecasts are predictive tools that look ahead to potential future drought events. Both tools are important for managing drought conditions and informing drought response and mitigation measures. (ChatGPT)


“III. Impacts and indicators of drought” ([RCRC, 2020, p. 15](zotero://select/groups/4773535/items/UESIQTRJ)) ([pdf](zotero://open-pdf/groups/4773535/items/P5JPVZ97?page=15&annotation=MFH29MUH))

“Forecasting the impacts of drought requires knowledge about climate sensitivity and resilience of local food systems networks, livelihood strategies and diversity, and many other context-specific variables. Indeed, droughts have a range of physical and socio-economic impacts. The following section lists some of the main indicators used by academics and practitioners to identify that a drought is occurring and monitor its progress.” ([RCRC, 2020, p. 15](zotero://select/groups/4773535/items/UESIQTRJ)) ([pdf](zotero://open-pdf/groups/4773535/items/P5JPVZ97?page=15&annotation=JQ42LDWM))


“A. Hazard-related indicators and indices” ([RCRC, 2020, p. 15](zotero://select/groups/4773535/items/UESIQTRJ)) ([pdf](zotero://open-pdf/groups/4773535/items/P5JPVZ97?page=15&annotation=KZ5ITR9D))

“There exist many physical indicators of a drought that are monitored by scientists and governments to track the development of drought impacts. The complex and insidious nature of drought means that all these indicators are proxies to understand the impacts that dry conditions are having on an area. Our main suggestion here would be to examine the World Meteorological Organisation Handbook of Drought Indicators and Indices (which are classified in a traffic-light method of ease of use) identifies which indices could be available and appropriate for the context.” ([RCRC, 2020, p. 15](zotero://select/groups/4773535/items/UESIQTRJ)) ([pdf](zotero://open-pdf/groups/4773535/items/P5JPVZ97?page=15&annotation=XR5BI5Y7))

%good tabular overview about drought indicators

“Standard Precipitation Index (SPI) The standard precipitation index (SPI) is one of the most used metrics to quantify meteorological drought. This calculates deviation of rainfall from norm over a defined period (Sutanto et al. 2019). The SPI takes the record of observed precipitation, fitted to a probability distribution that is then transformed to a normal distribution. Negative values indicate lower than average amounts of precipitation. The magnitude of drought can be calculated through the positive sum of the SPI over the season (AMS, 2020). There are also conversion tables that exist to calculate the return periods of droughts based on the SPI (these can be particularly useful for FbA for drought triggering systems) (WMO, 2012, p.11). An extension of the SPI is the SPEI, the Standard Precipitation and potential Evapotranspiration Index, that adds the impact of evapotranspiration to the metric, and provides a better indication of water stress.” ([RCRC, 2020, p. 17](zotero://select/groups/4773535/items/UESIQTRJ)) ([pdf](zotero://open-pdf/groups/4773535/items/P5JPVZ97?page=17&annotation=RUXA6YMT))

%% overview about all potential meteorological drought indices --> Indices possibly only short here and extensive in the forecasting section?
https://heigit.atlassian.net/wiki/spaces/FIS/pages/1704096/Indices



% drought monitoring tool for somalia
https://cdi.faoswalim.org/index/rfe-maps/2022

%-----------------------------------
%	SUBSECTION 5.1
%-----------------------------------

\subsection{Forecasts}

“A review of Drought Monitoring with Big Data: Issues, Methods, Challenges and Research Directions” ([Balti et al., 2020, p. 1](zotero://select/groups/4773535/items/V9YH9QFQ)) ([pdf](zotero://open-pdf/groups/4773535/items/JC4XTVWE?page=1&annotation=9W786QNF))

“Integrated Drought Monitoring and Evaluation through Multi-Sensor Satellite-Based Statistical Simulation” ([Kim et al., 2021](zotero://select/groups/4773535/items/3I6ZZJXC)) ([pdf](zotero://open-pdf/groups/4773535/items/AR55UZYK?page=1&annotation=HAWQQLMG))

which forecasts are selected by the EAP pre-study?

current challenges for utilisation of forecasting systems: scarse coverage of weather stations and poor utilisation by the farmers often due to bad dissemination channels  (too coarse, too unreliable)
Somalia Multi-hazard Early Warning Centre under Ministry of Humanitarian Affairs & Disaster Management
based on open data sources (coverage: Somalia)

FAO Somalia Water and Land Information Management (SWALIM) coverage: all regions - source: 100 manual rainfall stations, eight synoptic weather stations and 11 automatic weather stations in Somalia

“Forecast Menu for SRCS” ([Somali Red Crescent Society, 2022, p. 43](zotero://select/groups/4773535/items/FZ6BJHJA)) ([pdf](zotero://open-pdf/groups/4773535/items/RJKNZZZ2?page=43&annotation=5A354LG7))
identified by the SRCS pre-study

“Table 1. Comparisons between indigenous knowledge-based seasonal forecasts and seasonal climate forecasts (adopted from Ziervogel and Opere 2010). Indigenous knowledge-based seasonal forecasts Seasonal climate forecasts Use biophysical indicators of the environment as well as spiritual methods Use of weather and climate models of measurable meteorological data Forecast methods are seldom documented Forecast methods are more developed and documented Up-scaling and down-scaling are usually complex Up-scaling and down-scaling are relatively simple Application of forecast output is less developed Application of forecast output is more developed Communication is usually oral Communication is usually written Explanation is based on spiritual and social values Explanation is theoretical Taught by observation and experience Taught through lectures and readings” ([Masinde and Bagula, 2012, p. 280](zotero://select/groups/4773535/items/EW9XSSZP)) ([pdf](zotero://open-pdf/groups/4773535/items/3WQ4S9PE?page=7&annotation=6XCISBM2))

“Indigenous knowledge within an early warning system for droughts” ([Masinde and Bagula, 2012, p. 282](zotero://select/groups/4773535/items/EW9XSSZP)) ([pdf](zotero://open-pdf/groups/4773535/items/3WQ4S9PE?page=9&annotation=8Z9A9AW8))

“The Best of Both Worlds: A Decision-Making Framework for Combining Traditional and Contemporary Forecast Systems” ([Plotz et al., 2017, p. 1](zotero://select/groups/4773535/items/3SBLBZEA)) ([pdf](zotero://open-pdf/groups/4773535/items/VAUJGIFB?page=1&annotation=RRFY2UWE))

“B. Drought Forecasting in Sub-Saharan Africa” ([Masinde and Thothela, 2019, p. 304](zotero://select/groups/4773535/items/6D52T883)) ([pdf](zotero://open-pdf/groups/4773535/items/KLLQKDG2?page=2&annotation=ZQSDUEMX))

there are more! Look into it.!

“∣Potential impact (x, t)∣≡∣hazard (x, t)∣∪∣vulnerability (x, t) ∣∪∣exposure (x, t)∣” ([Boult et al., 2022, p. 2](zotero://select/groups/4773535/items/B2AQSTYL)) ([pdf](zotero://open-pdf/groups/4773535/items/W9TFLH43?page=2&annotation=6M48XVJI))

“In theory, focusing on what the weather will do, rather than what the weather will be, enables decision makers to plan and implement targeted preparatory actions to better reduce hazard impacts (Harrowsmith et al., 2020).” ([Boult et al., 2022, p. 2](zotero://select/groups/4773535/items/B2AQSTYL)) ([pdf](zotero://open-pdf/groups/4773535/items/W9TFLH43?page=2&annotation=NSLE7NL6))

“Improving early warning of drought-driven food insecurity in southern Africa using operational hydrological monitoring and forecasting products” ([Shukla et al., 2020, p. 1187](zotero://select/groups/4773535/items/TE5NMA3T)) ([pdf](zotero://open-pdf/groups/4773535/items/9TNUGXSJ?page=1&annotation=TLBHC7BS))

“Moving from drought hazard to impact forecasts” ([Sutanto et al., 2019, p. 1](zotero://select/groups/4773535/items/EUC5RV7N)) ([pdf](zotero://open-pdf/groups/4773535/items/9DI9EVBF?page=1&annotation=FCQCKY5P))

“IKON-OWL: Using Ontologies for Knowledge Representation of Local Indigenous Knowledge on Drought” ([Akanbi and Masinde, 2018, p. 1](zotero://select/groups/4773535/items/ETGPPTN4)) ([pdf](zotero://open-pdf/groups/4773535/items/5DN7H3YM?page=2&annotation=EFZHAUKA))

%% overview about all potential meteorological drought indices
https://heigit.atlassian.net/wiki/spaces/FIS/pages/1704096/Indices

https://cdi.faoswalim.org/index/cdi

“Researchers ([1], [19] and [20]) today concur that IK and modern science weather forecasts complement each other;” ([Masinde et al., 2013, p. 2](zotero://select/groups/4773535/items/M45MLGWC)) ([pdf](zotero://open-pdf/groups/4773535/items/LG6E76P4?page=2&annotation=JHQV2GYT))
%-----------------------------------
%	SUBSECTION 5.2
%-----------------------------------

\subsection{Impact \& risk? based}
%there needs to be a paragraph about risk (==> vulnerability, exposure and coping capacity (generally))
“Section 1.4 provides a detailed discussion of the different components of the risk equation and their relationships, as well as the concepts and methodologies for assessment of the resulting risk, including a framework for understanding the relationships among various factors.” ([“Special report on drought 2021”, 2021, p. 27](zotero://select/groups/4773535/items/RAAM9PVS)) ([pdf](zotero://open-pdf/groups/4773535/items/7AK5QVBL?page=29&annotation=3U8V6ZXN))

% could also go into drought chapter
“Why predict climate hazards if we need to understand impacts? Putting humans back into the drought equation” ([Enenkel et al., 2020, p. 1161](zotero://select/groups/4773535/items/RX575C79)) ([pdf](zotero://open-pdf/groups/4773535/items/XD499UNK?page=1&annotation=UVBCTUE6))

“There has been little effort to align the spatiotemporal granularity of socioeconomic assessments with the granularity of weather or climate monitoring.” ([Enenkel et al., 2020, p. 1161](zotero://select/groups/4773535/items/RX575C79)) ([pdf](zotero://open-pdf/groups/4773535/items/XD499UNK?page=1&annotation=QBTLFCXM))
“we highlight the need to collect and analyze environmental and socioeconomic data together and discuss novel strategies for coordinated data collection via mobile technologies from a drought risk management perspective.” ([Enenkel et al., 2020, p. 1161](zotero://select/groups/4773535/items/RX575C79)) ([pdf](zotero://open-pdf/groups/4773535/items/XD499UNK?page=1&annotation=9BUBHWNB))

“but questions related to coping capacities, migration, poverty, water supply, access to food and markets, or political conflict remain unanswered or are even decoupled from routine drought risk assessments” ([Enenkel et al., 2020, p. 1162](zotero://select/groups/4773535/items/RX575C79)) ([pdf](zotero://open-pdf/groups/4773535/items/XD499UNK?page=2&annotation=HE48ZWFA))

“The handbook of drought indicators (Svoboda et al. 2016) lists more than 50 drought indices. Not a single one of these indices connects climate anomalies to socioeconomic vulnerabilities,” ([Enenkel et al., 2020, p. 1163](zotero://select/groups/4773535/items/RX575C79)) ([pdf](zotero://open-pdf/groups/4773535/items/XD499UNK?page=3&annotation=EDKZFHJX))
--> there were more indicators in the following year, but those were limited due to availibiltiy of socioeconomic data

“In a region where migration is one of the main coping mechanisms for drought, a targeted survey focusing on the early detection of migration movements would help mobilize the timely allocation of resources by humanitarian decision-makers or even the mitigation of drought impacts.” ([Enenkel et al., 2020, p. 1167](zotero://select/groups/4773535/items/RX575C79)) ([pdf](zotero://open-pdf/groups/4773535/items/XD499UNK?page=7&annotation=N9FRDA9C))

+ humanitarian workers could focus on response rather than data acquisition (my idea)

“In addition, the impact of agricultural conditions that are not necessarily related to climate shocks, but to other factors such as pests or social conflict, could easily be monitored and used to issue early warnings and raise emergency funds before any kind of impact on crops or socioeconomic conditions is visible.” ([Enenkel et al., 2020, p. 1168](zotero://select/groups/4773535/items/RX575C79)) ([pdf](zotero://open-pdf/groups/4773535/items/XD499UNK?page=8&annotation=MDXAXIJY))

% impacts may vary depending on variuous factors such as ...
“(a) meteorological droughts evolve collectively into a prolonged hydrological drought (pooling);” ([Wang et al., 2016, p. 2](zotero://select/groups/4773535/items/AWKLZ543)) ([pdf](zotero://open-pdf/groups/4773535/items/4YRNMM9P?page=2&annotation=NAJPZFFV))


%% Impacts
“Impacts are immediately visible and localised. Impact data is collected via standardized processes such as Damage and Needs Assessments.” ([pdf](zotero://open-pdf/groups/4773535/items/P5JPVZ97?page=19&annotation=6QHQH3CN))

% lots and lots and lots of information here
“Drought impacts” ([“Special report on drought 2021”, 2021, p. 42](zotero://select/groups/4773535/items/RAAM9PVS)) ([pdf](zotero://open-pdf/groups/4773535/items/7AK5QVBL?page=44&annotation=5LETGXSN))

% what (kind of) information is needed?

“Towards drought impact-based forecasting in a multi-hazard context” ([Boult et al., 2022, p. 1](zotero://select/groups/4773535/items/B2AQSTYL)) ([pdf](zotero://open-pdf/groups/4773535/items/W9TFLH43?page=1&annotation=GL47JLV7))
The framework for drought impact-based forecasting proposed in the paper "Towards drought impact-based forecasting in a multi-hazard context" aims to provide more comprehensive and actionable information for decision-making in the context of multi-hazard risk management. The framework considers both biophysical and socioeconomic factors, including:

Exposure: The extent to which a population or system is exposed to drought-related hazards, such as the availability of water resources, the dependence on rain-fed agriculture, and the presence of infrastructure that may be vulnerable to drought-related impacts.

Vulnerability: The susceptibility of a population or system to drought-related impacts, considering factors such as the social, economic, and environmental characteristics of the affected area, as well as the capacity of the affected community or system to cope with and recover from drought-related impacts.

Coping capacities: The ability of a population or system to mitigate or adapt to drought-related impacts, including the availability of resources, the presence of social networks and support systems, and the ability to access and utilize information and technology.

The authors argue that considering these factors in the context of drought forecasting can provide more useful and actionable information for decision-making, and recommend further research on the development and implementation of drought impact-based forecasting systems, as well as their integration into existing decision-making frameworks.


“∣Potential impact (x, t)∣≡∣hazard (x, t)∣∪∣vulnerability (x, t) ∣∪∣exposure (x, t)∣” ([Boult et al., 2022, p. 2](zotero://select/groups/4773535/items/B2AQSTYL)) ([pdf](zotero://open-pdf/groups/4773535/items/W9TFLH43?page=2&annotation=6M48XVJI))

“In theory, focusing on what the weather will do, rather than what the weather will be, enables decision makers to plan and implement targeted preparatory actions to better reduce hazard impacts (Harrowsmith et al., 2020).” ([Boult et al., 2022, p. 2](zotero://select/groups/4773535/items/B2AQSTYL)) ([pdf](zotero://open-pdf/groups/4773535/items/W9TFLH43?page=2&annotation=NSLE7NL6))

“However, establishing a functional relationship can be difficult for a number of reasons” ([Boult et al., 2022, p. 3](zotero://select/groups/4773535/items/B2AQSTYL)) ([pdf](zotero://open-pdf/groups/4773535/items/W9TFLH43?page=3&annotation=ANJJD878))

% --> direct information would be better
% --> in case of multi-hazard EWS
“Drought presents a clear case for the multi-hazardapproach. Because it is a slow-onsetevent, vulnerability to drought is susceptible to the influenceof concurrent hazards and non-biophysical events, meaning multi-hazards must be considered if drought interventions are to be effectivelytargeted. Moreover, the intrinsic predictability and slow onset provides a timeframe in which early actions can be adapted in response to multi-hazard influences (Boult et al., 2020).” ([Boult et al., 2022, p. 2](zotero://select/groups/4773535/items/B2AQSTYL)) ([pdf](zotero://open-pdf/groups/4773535/items/W9TFLH43?page=2&annotation=QZ7UC62U))

“2.Challenges for drought IbF” ([Boult et al., 2022, p. 3](zotero://select/groups/4773535/items/B2AQSTYL)) ([pdf](zotero://open-pdf/groups/4773535/items/W9TFLH43?page=3&annotation=7B52FIS8)) (Impact based Forecast):
The main challenges for drought impact-based forecasting (IbF) as described in this text include:

Difficulty in directly forecasting drought impacts: It can be challenging to establish a functional relationship between drought hazards and impacts, as past humanitarian aid and development may have weakened this link, and sufficient impact data may be unavailable. In addition, the relationship between hazard severity and impact is mediated by vulnerability, which is not always fully understood.

Issues with predefined systems: Predefined IbF systems that rely on predefined trigger thresholds and pre-agreed actions may not be able to adapt to changing vulnerabilities over time. These systems also rely on expert judgement during system development, which may be subject to bias and may not accurately reflect the complexity of the system.

“Without scope to accommodate dynamic vulnerabilities, actions cannot be effectively targeted or may prove ineffective.” ([Boult et al., 2022, p. 4](zotero://select/groups/4773535/items/B2AQSTYL)) ([pdf](zotero://open-pdf/groups/4773535/items/W9TFLH43?page=4&annotation=YYURM2E3))

Lack of coordination between sectors: Drought impacts often involve multiple sectors (e.g. agriculture, water, health), but there may be a lack of coordination and integration between these sectors in the design and implementation of IbF systems. This can lead to inefficiencies and gaps in the response to drought impacts.

Limited capacity for impact-based decision making: There may be a lack of capacity or resources for impact-based decision making at the local level, particularly in low-income countries where data availability and quality may be limited.

Inability to fully capture the complexity of drought impacts: Drought impacts often involve complex and dynamic interactions between hazards, vulnerabilities, and exposures, and may be exacerbated by non-climate factors such as conflict or economic conditions. It can be difficult to fully capture and understand these interactions in IbF systems.

“Moreover, if the complex relationships linking multihazards to multi-impacts obscure attribution of particular impacts to particular hazards, hazard-focused organisations may be limited by their institutional mandate and thus unable to act.” ([Boult et al., 2022, p. 4](zotero://select/groups/4773535/items/B2AQSTYL)) ([pdf](zotero://open-pdf/groups/4773535/items/W9TFLH43?page=4&annotation=2XG6YP8D))

interesting!!! Fig. 1. A hybrid framework for multi-hazard IbF. Refer to the main text for a definition of numbers. Black arrows and numbers: components common across predefined humanitarian IbF systems. Blue arrows and numbers: real-time components. Grids represent spatially varying values. Darker reds indicate higher values of risk, vulnerability, and thresholds. In this example, despite only low to moderate risk in the southwest square, increased dynamic vulnerability lowers the threshold for action, resulting in triggering. Meanwhile, reduced vulnerability in the northern squares elevates trigger thresholds, so the northeast square no longer triggers. (For interpretation of the references to colour in this figure legend, the reader is referred to the web version of this article.)

A hydrometeorological forecast indicating the likelihood of drought occurring is combined with a predefined assessment of static vulnerability to determine risk. Where static vulnerability is higher (Fig. 1: northern squares), trigger thresholds are lower. 2) Risk is compared to agreed-upon thresholds for action. 3) If risk is greater than or equal to the threshold, early action is triggered to mitigate the worst impacts of drought.

Proposed Framework:
--> 1)“A hydrometeorological forecast indicating the likelihood of drought occurring is combined with a predefined assessment of static vulnerability to determine risk. Where static vulnerability is higher (Fig. 1: northern squares), trigger thresholds are lower.
2) Risk is compared to agreed-upon thresholds for action.
3) If risk is greater than or equal to the threshold, early action is triggered to mitigate the worst impacts of drought.” ([Boult et al., 2022, p. 5](zotero://select/groups/4773535/items/B2AQSTYL)) ([pdf](zotero://open-pdf/groups/4773535/items/W9TFLH43?page=5&annotation=KDCBHGC6))

“We then propose a number of components to account for dynamic vulnerabilities caused by concurrent hazards:
4)Expert judgement is utilised to determine dynamic vulnerabilities. For instance, conflict, pest outbreaks, or recent hydrometeorological events, may act to increase vulnerability to drought in the affected location.
5) In locations where vulnerability is elevated (Fig. 1: southwest square), the predefined forecast threshold (“danger level”) is relaxed in order to trigger for less severe droughts. This acknowledges that those with elevated vulnerabilities require support even if drought is only slight. In regions where dynamic vulnerability is lower (Fig. 1: northern squares), the predefined forecast threshold may be raised, to avoid the perception of false alarms if a less severedrought does not have significant impact on food security (the trigger threshold for northern squares is elevated to reflect reduced vulnerability). Balancing of lower thresholds for vulnerable regions against higher thresholds for less vulnerable regions reduces the need for ‘safety nets’, enabling more accurate anticipation of donor costs.
6) If risk exceeds the adjusted thresholds, early actions are triggered. Early actions may need to be adapted to account for multihazards.” ([Boult et al., 2022, p. 5](zotero://select/groups/4773535/items/B2AQSTYL)) ([pdf](zotero://open-pdf/groups/4773535/items/W9TFLH43?page=5&annotation=XV9GPPQF))


“According to the Hyogo Framework for Action, increasing resilience to drought requires the development of a people-centered monitoring and early warning system, or in other words, a system capable of providing useful and understandable information to the community at risk.” ([Giordano et al., 2013, p. 523](zotero://select/groups/4773535/items/B7LM5ZR4)) ([pdf](zotero://open-pdf/groups/4773535/items/7I66DBIK?page=1&annotation=ELTIKEZV))

“should include both expert and local knowledge” ([Giordano et al., 2013, p. 523](zotero://select/groups/4773535/items/B7LM5ZR4)) ([pdf](zotero://open-pdf/groups/4773535/items/7I66DBIK?page=1&annotation=25R42QM8))

“Although several benefits can be obtained, the integration of local and scientific knowledge to support drought monitoring is still far from being the standard in drought monitoring and early warning” ([Giordano et al., 2013, p. 523](zotero://select/groups/4773535/items/B7LM5ZR4)) ([pdf](zotero://open-pdf/groups/4773535/items/7I66DBIK?page=1&annotation=5QMA2EZR))

“many reasons, that is, the reciprocal skepticism of local communities and decision makers, and the limits in the capacity to understand and assess the complex web of drought impacts.” ([Giordano et al., 2013, p. 523](zotero://select/groups/4773535/items/B7LM5ZR4)) ([pdf](zotero://open-pdf/groups/4773535/items/7I66DBIK?page=1&annotation=43QUIG2F))

“A validation phase was carried out comparing the perceived indicators of drought and the physical indicators” ([Giordano et al., 2013, p. 523](zotero://select/groups/4773535/items/B7LM5ZR4)) ([pdf](zotero://open-pdf/groups/4773535/items/7I66DBIK?page=1&annotation=JR8Z5JJH))

impacts can also be different based on gender “Box 1.5. Drought impacts and gender imbalance” ([“Special report on drought 2021”, 2021, p. 57](zotero://select/groups/4773535/items/RAAM9PVS)) ([pdf](zotero://open-pdf/groups/4773535/items/7AK5QVBL?page=59&annotation=VBN9S6RL)) x

“In humanitarian practice, the term "drought" is often used to refer to some socio-meteorological combination where water shortages produce stress on human and livelihood systems. Droughts are a function of the fragility of human systems, and they become disasters where systems cannot cope with deviations from the hydro-meteorological norm. It has been argued that droughts are particularly devastating when livelihood choices are strongly determined by the climate (e.g. the decision to grow certain crops, or traditional seasonal migration patterns) - for instance, if in a given year, the weather patterns are different than normal, those livelihoods are especially vulnerable to these changes. It has also been argued that droughts pose specific challenges to income generating activities marked by low productivity that are not able to take advantage of ‘good years’ in order to provide a buffer during ‘bad years’.” ([RCRC, 2020, p. 12](zotero://select/groups/4773535/items/UESIQTRJ)) ([pdf](zotero://open-pdf/groups/4773535/items/P5JPVZ97?page=12&annotation=C622YWIR))

% impact
what drought is (definition)
drought monitoring but not only physical indicators but socials as well (water source accessibility) “However, assessments focused only on physical variables and processes fail to capture why drought matters, in other words, how social, economic, and ecological systems are affected (i.e., impacts) (Redmond 2002; Van Loon et al. 2016; Wilhite and Glantz 1985).” ([Lackstrom et al., 2022, p. 3](zotero://select/groups/4773535/items/YI366LQY)) ([pdf](zotero://open-pdf/groups/4773535/items/3JTQ72UN?page=3\&annotation=72WYIE7B))



%----------------------------------------------------------------------------------------
%	SECTION 6 Crowdsensing, VGI, alternatively satellite image interpretation
%----------------------------------------------------------------------------------------


\section{Crowdsensing, Volunteersensing, VGI,  alternatively satellite image interpretation}
% alternatively -> on the ground/on site data via crowdsensing
% pros and cons of VGI
% Introduction of Crowdsourcing / sensing / Volunteersensing


%-----------------------------------
%	SUBSECTION 6.1
%-----------------------------------

% otherwise: 7.3 subsection water source mapping but I think it might be a good start here to introduce on site information
"In all, mapping presents many benefits, such as:

It makes easier to integrate data from different sources (surveys, censuses, satellites, etc.) and from different disciplines (social, economic, and environmental data). It also allows the switch to new units of analysis from, for example, administrative boundaries (e.g. state) to ecological boundaries (e.g. basin).
Maps are a powerful visual tool and are more easily understood by stakeholders, particularly in developing countries.
The spatial nature of water poverty, such as the distance to the nearest water source or the water supply infrastructure, can also be incorporated easily in a GIS database.
The allocation of resources can be improved, since geographic targeting is more efficient and cost-effective than to launch an equally expensive universal distribution programme.
Geo-referenced databases can be enriched by additional data as they become available; and new attributes, such as better details on water quality, can be incorporated into the data structure, ensuring that the relevance of the data is sustained over time.
Maps can be produced at a number of different resolutions depending on their purpose and the cost of data collection. A coarse resolution or a scale too small neglects the heterogeneity within each unit and provides insufficient detail for decision making, while a fine resolution or a scale too large increases the cost of compiling, managing, and analyzing the data."https://en.wikipedia.org/wiki/Water_point_mapping

% possibly a good start for the transition from impact indicators to Crowdsensing because this is slow and relatively coarse.
\subsection{“Real-Time Risk Monitoring (RTRM)” ([Gualazzini, 2021, p. 4](zotero://select/groups/4773535/items/BWDYDL8T)) ([pdf](zotero://open-pdf/groups/4773535/items/8U5XVU5K?page=4&annotation=HZFMU84X))}33

https://brcis.shinyapps.io/EWEA_dashboard/

“RTRM FPs collect shock data monthly according to a collectively agreed protocol” ([Gualazzini, 2021, p. 8](zotero://select/groups/4773535/items/BWDYDL8T)) ([pdf](zotero://open-pdf/groups/4773535/items/8U5XVU5K?page=8&annotation=HCYJPVMQ))

“select a small group of communities (two to three) that they deemed representative of the larger population (10–12 locations) in their operation area. Within this group, FPs are then required to contact two to four community actors and integrate the shock information into a single data submission per area.” ([Gualazzini, 2021, p. 8](zotero://select/groups/4773535/items/BWDYDL8T)) ([pdf](zotero://open-pdf/groups/4773535/items/8U5XVU5K?page=8&annotation=YLEYNUY3))

“In line with the above objectives, BRCiS designed the RTRM system to provide timely, community-generated shock information on key trends to trigger early actions in BRCiS operational areas.” ([Gualazzini, 2021, p. 5](zotero://select/groups/4773535/items/BWDYDL8T)) ([pdf](zotero://open-pdf/groups/4773535/items/8U5XVU5K?page=5&annotation=LDCLNI5G))

“For this purpose, consortium technical specialists used community feedback to determine the most common shocks, stresses and coping strategies.” ([Gualazzini, 2021, p. 5](zotero://select/groups/4773535/items/BWDYDL8T)) ([pdf](zotero://open-pdf/groups/4773535/items/8U5XVU5K?page=5&annotation=4QBNWWJI))

% also in the chapter about the trigger -> makes for a good comparison to Richards levels and threshold indicators
“Primary water source condition Observation of the status of the largest rainwater catchment in the area (at the end of rainy seasons) More than half-full (75%) or full Half-full (50%) at the end of rainy season Less than half-full (25%)/empty” ([Gualazzini, 2021, p. 6](zotero://select/groups/4773535/items/BWDYDL8T)) ([pdf](zotero://open-pdf/groups/4773535/items/8U5XVU5K?page=6&annotation=HMGNP355))

“literacy was cited as a key issue hindering community leaders from further participating in digital data collection.” ([Gualazzini, 2021, p. 9](zotero://select/groups/4773535/items/BWDYDL8T)) ([pdf](zotero://open-pdf/groups/4773535/items/8U5XVU5K?page=9&annotation=J6DKMPDX))

“Half of all respondents stated that digital shock questionnaires would be too complex to handle if they were filled exclusively by community leaders, and translation to Somali language was recommended.” ([Gualazzini, 2021, p. 9](zotero://select/groups/4773535/items/BWDYDL8T)) ([pdf](zotero://open-pdf/groups/4773535/items/8U5XVU5K?page=9&annotation=BM8PXNQ5))

“Around half of survey respondents emphasised that community leaders taking part in RTRM are well aware that reported shock data may influence the probability of receiving future humanitarian assistance in their area.” ([Gualazzini, 2021, p. 9](zotero://select/groups/4773535/items/BWDYDL8T)) ([pdf](zotero://open-pdf/groups/4773535/items/8U5XVU5K?page=9&annotation=8KNB2WAI))

“This context highlights the critical importance of trustbuilding and data triangulation in community-based early warning systems.” ([Gualazzini, 2021, p. 9](zotero://select/groups/4773535/items/BWDYDL8T)) ([pdf](zotero://open-pdf/groups/4773535/items/8U5XVU5K?page=9&annotation=2D9BKJTN))

“The challenge looking forward will be determining the best methods for strengthening existing accountability structures and verification approaches to mitigate these biases.” ([Gualazzini, 2021, p. 9](zotero://select/groups/4773535/items/BWDYDL8T)) ([pdf](zotero://open-pdf/groups/4773535/items/8U5XVU5K?page=9&annotation=EI3XUNJC))

“For every red-flagged area, the respective early warning FPs contextualise the situation, interpret the indicators, triangulate with secondary data sources and propose relevant early action measures. This triangulation stage compares community and arealevel data with data at a higher geographical level (e.g., livelihood zone or district), contextualising the situation with the broader situation in other regions of Somalia.” ([Gualazzini, 2021, p. 9](zotero://select/groups/4773535/items/BWDYDL8T)) ([pdf](zotero://open-pdf/groups/4773535/items/8U5XVU5K?page=9&annotation=ST2P2WYY))

“This process includes the analysis of satellite-derived climate data and information from sources. Further third-party secondary data such as the FSNAU EWEA dashboard, WHO-Early Warning, Alert and Response Network data or other NGO assessments – also contribute to the triangulation process. District-level and macro-level data provide an evidence base to triangulate subjective information from communities, local authorities or field staff.” ([Gualazzini, 2021, p. 10](zotero://select/groups/4773535/items/BWDYDL8T)) ([pdf](zotero://open-pdf/groups/4773535/items/8U5XVU5K?page=10&annotation=69DDN2D8))

“Therefore, for slow-onset shocks captured by the BRCiS RTRM, early action decision-triggering is possible within 12 days after the data collection.” ([Gualazzini, 2021, p. 10](zotero://select/groups/4773535/items/BWDYDL8T)) ([pdf](zotero://open-pdf/groups/4773535/items/8U5XVU5K?page=10&annotation=HFQHTQKN))

“In practice, however, delays in RTRM data collection in the field and in the submission of EWEA forms have been frequent due to operational challenges and other factors. The average timeline between the submission of RTRM data and the submission of early action recommendations in the EWEA forms was 25 days.” ([Gualazzini, 2021, p. 10](zotero://select/groups/4773535/items/BWDYDL8T)) ([pdf](zotero://open-pdf/groups/4773535/items/8U5XVU5K?page=10&annotation=CL7QZEA6))

“The consortium aims to reduce this time-lapse to a maximum of 12 days. For sudden shocks such as riverine or flash floods, the time for programme managers to trigger early response was limited to one to four days following the rapid assessment of impacts within communities.” ([Gualazzini, 2021, p. 10](zotero://select/groups/4773535/items/BWDYDL8T)) ([pdf](zotero://open-pdf/groups/4773535/items/8U5XVU5K?page=10&annotation=HNG6XTD8))


% summery and learning points
“Summary of BRCiS EWEA:” ([Gualazzini, 2021, p. 18](zotero://select/groups/4773535/items/BWDYDL8T)) ([pdf](zotero://open-pdf/groups/4773535/items/8U5XVU5K?page=18&annotation=B8AT9Q75))

1. “Community leaders can play a greater role in humanitarian response.” ([Gualazzini, 2021, p. 18](zotero://select/groups/4773535/items/BWDYDL8T)) ([pdf](zotero://open-pdf/groups/4773535/items/8U5XVU5K?page=18&annotation=MED4PHDQ))

2. “2. Risk monitoring has proved highly effective and essential for future programmes.” ([Gualazzini, 2021, p. 18](zotero://select/groups/4773535/items/BWDYDL8T)) ([pdf](zotero://open-pdf/groups/4773535/items/8U5XVU5K?page=18&annotation=YM4LB2YU))


3. “3. Indicators should be capable of capturing the compound effects of multiple shocks.” ([Gualazzini, 2021, p. 18](zotero://select/groups/4773535/items/BWDYDL8T)) ([pdf](zotero://open-pdf/groups/4773535/items/8U5XVU5K?page=18&annotation=T8PZ5IZA))

“4. Vulnerable communities need improved access to climate and weather information” ([Gualazzini, 2021, p. 19](zotero://select/groups/4773535/items/BWDYDL8T)) ([pdf](zotero://open-pdf/groups/4773535/items/8U5XVU5K?page=19&annotation=4KXI6RK2))

“5. Local-level EWEA committees strengthen local capacities to prepare for and respond to shocks.” ([Gualazzini, 2021, p. 19](zotero://select/groups/4773535/items/BWDYDL8T)) ([pdf](zotero://open-pdf/groups/4773535/items/8U5XVU5K?page=19&annotation=RG55EEU2))

“6. Flexible and shock-responsive funding mechanisms successfully improved food security.” ([Gualazzini, 2021, p. 19](zotero://select/groups/4773535/items/BWDYDL8T)) ([pdf](zotero://open-pdf/groups/4773535/items/8U5XVU5K?page=19&annotation=KW32EH9F))

“7. Communities highly valued and engaged with the BRCiS integrated approach” ([Gualazzini, 2021, p. 20](zotero://select/groups/4773535/items/BWDYDL8T)) ([pdf](zotero://open-pdf/groups/4773535/items/8U5XVU5K?page=20&annotation=UTZLTZB3))

“BRCiS early warning and early action vision (2022–2026)” ([Gualazzini, 2021, p. 20](zotero://select/groups/4773535/items/BWDYDL8T)) ([pdf](zotero://open-pdf/groups/4773535/items/8U5XVU5K?page=20&annotation=QVKJRXK6))


\subsection{VGI}


%-----------------------------------
%	SUBSECTION 6.2
%-----------------------------------

\subsection{local knowledge (???)}
noe

\subsection{Crowdsensing / Citizen Science}
Design Factors of citizen Science projects
though, their systematic impact is still unclear “most pressing research question is a systematic understanding of the effects of design variables, factors, and clusters on project success, including data quantity and quality, outcome variables, and actual solutions to resource management problems.” ([Kirschke et al., 2022, p. 8](zotero://select/groups/4773535/items/GPC3LDT5)) ([pdf](zotero://open-pdf/groups/4773535/items/AI7HRQYC?page=8&annotation=QHPJDGUV))

“Citizen science projects in freshwater monitoring. From individual design to clusters?” ([Kirschke et al., 2022, p. 1](zotero://select/groups/4773535/items/GPC3LDT5)) ([pdf](zotero://open-pdf/groups/4773535/items/AI7HRQYC?page=1&annotation=6IJERY2C))

“2.3. Description of the SMS-server” ([Weeser et al., 2018, p. 1593](zotero://select/groups/4773535/items/SFA2MLHC)) ([pdf](zotero://open-pdf/groups/4773535/items/GP79FHFC?page=4&annotation=V6IYDSJ9))

“The findings confirm other studies on technology acceptance, as the mobile crowd sensing technology should be useful rather than free of effort.” ([Minkman, 2015, p. 12](zotero://select/groups/4773535/items/ZKLE6CPT)) ([pdf](zotero://open-pdf/groups/4773535/items/QMAPCSZG?page=12&annotation=IQEBUPYG))


“Crowdsourcing as a Model for Problem Solving” ([“Crowdsourcing as a Model for Problem Solving”, p. 75](zotero://select/groups/4773535/items/NT56MWYM)) ([pdf](zotero://open-pdf/groups/4773535/items/TZI4NDBJ?page=1&annotation=FJTS2A7L))

“2 Local knowledge in drought monitoring: an introduction to the literature review” ([Giordano et al., 2013, p. 526](zotero://select/groups/4773535/items/B7LM5ZR4)) ([pdf](zotero://open-pdf/groups/4773535/items/7I66DBIK?page=4&annotation=Z33M5FLQ))

citizen science the right tool?
--> if help to achieve desired results AND benefit participants --> YES

+ check research question, spatial and temporal scale, type and amount of data, level of expertise required to collect the data, training and coodination efforts needed, target groups, 

funding: review resources available --> human resources (skills, equipment, travel, necessary training)

“To set up a successful citizen science project, alignment of goals between water authorities and citizens is important and this thesis’s findings can provide initial understanding of citizen motivations.” ([Minkman, 2015, p. 11](zotero://select/groups/4773535/items/ZKLE6CPT)) ([pdf](zotero://open-pdf/groups/4773535/items/QMAPCSZG?page=11&annotation=F2MJEUJY))
% 

“The goal is to understand whether involving citizen science participants will help to achieve the desired results, while at the same time benefiting participants by addressing their needs” ([Fraisl et al., 2022, p. 2](zotero://select/groups/4773535/items/FBJD7SWS)) ([pdf](zotero://open-pdf/groups/4773535/items/7WBDKYDY?page=2&annotation=ISWIIMXQ))
“Examples of projects that are suitable for citizen science approaches are [...] monitoring water or air quality” ([Fraisl et al., 2022, p. 3](zotero://select/groups/4773535/items/FBJD7SWS)) ([pdf](zotero://open-pdf/groups/4773535/items/7WBDKYDY?page=3&annotation=ZHSWV5QJ))

origin: https://www.wired.com/2006/06/crowds/

“Towards an integrated crowdsourcing definition” ([Estellés-Arolas and González-Ladrón-de-Guevara, 2012, p. 189](zotero://select/groups/4773535/items/XYSTHUYY)) ([pdf](zotero://open-pdf/groups/4773535/items/AHADBEP2?page=1&annotation=EUKRSKE7))

“Crowdsourcing as a Model for Problem Solving” ([“Crowdsourcing as a Model for Problem Solving”, p. 75](zotero://select/groups/4773535/items/NT56MWYM)) ([pdf](zotero://open-pdf/groups/4773535/items/TZI4NDBJ?page=1&annotation=FJTS2A7L))


“Numerous comparison studies of volunteer vs. professional water resources data suggest that volunteer data are generally comparable to professional data for chemical (Obrecht et al., 1998; Loperfido et al., 2010), physical (Rodrigues and Castro, 2008), and biological (Fore et al., 2001; Vail et al., 2003; Gowan et al., 2007; Stepenuck et al., 2011) monitoring.” ([Lowry et al., 2019, p. 4](zotero://select/groups/4773535/items/3TVRMCP5)) ([pdf](zotero://open-pdf/groups/4773535/items/TDWAUJIY?page=4&annotation=UYNVBR7F))
“Notably, in all of these studies, volunteers were trained to carry out the monitoring in which they were engaged.” ([Lowry et al., 2019, p. 4](zotero://select/groups/4773535/items/3TVRMCP5)) ([pdf](zotero://open-pdf/groups/4773535/items/TDWAUJIY?page=4&annotation=ZNP3EUNT))
“Limited validation in the CrowdHydrology project, using a co-located pressure transducer (Lowry and Fienen, 2013), revealed root mean square error of participant data versus researcher data of about 0.02 feet – roughly the resolution of a class A staff gauge.” ([Lowry et al., 2019, p. 4](zotero://select/groups/4773535/items/3TVRMCP5)) ([pdf](zotero://open-pdf/groups/4773535/items/TDWAUJIY?page=4&annotation=RAZTZSKF))


“Crowdsourcing as a Model for Problem Solving” ([“Crowdsourcing as a Model for Problem Solving”, p. 75](zotero://select/groups/4773535/items/NT56MWYM)) ([pdf](zotero://open-pdf/groups/4773535/items/TZI4NDBJ?page=1&annotation=FJTS2A7L))


acceptance of the "crowd" is difficult to measure/identify/survey as no direct communication is possible. --> Good literature work e.g. “The basis of TAM consists of four main elements: the perceived usefulness of the application, the perceived ease of use when using the application, behavioural intention to use the application and the actual use.” ([Minkman, 2015, p. 12](zotero://select/groups/4773535/items/ZKLE6CPT)) ([pdf](zotero://open-pdf/groups/4773535/items/QMAPCSZG?page=12&annotation=IM5DGRJP))

“It was found that usefulness is the most important for behavioural intention.” ([Minkman, 2015, p. 12](zotero://select/groups/4773535/items/ZKLE6CPT)) ([pdf](zotero://open-pdf/groups/4773535/items/QMAPCSZG?page=12&annotation=75GFCEP5))

and how to best apply it -> relevant for the conceptionalization phase but generally well researched topic (in other, but possibly transferable contexts)


citizen science or rather participatory monitoring?

“How to decide whether citizen science is an appropriate means?” ([Minkman, 2015, p. 174](zotero://select/groups/4773535/items/ZKLE6CPT)) ([pdf](zotero://open-pdf/groups/4773535/items/QMAPCSZG?page=174&annotation=ZUAXHTI9))

“4 | IS REMOTE MONITORING A PATHWAY TO SUSTAINABILITY?” ([Thomson, 2021, p. 9](zotero://select/groups/4773535/items/UQLXVVYI)) ([pdf](zotero://open-pdf/groups/4773535/items/K9XBXPQD?page=9&annotation=833Q66UP))

rather than thinking water sources individually -> they can be thought of as a system when monitored more broadly

-> can change the way the water sector is funded: “With these data readily available, performance-related contracts that incentivize sustainable service delivery over short-term infrastructure investment can become the norm.” ([Thomson, 2021, p. 11](zotero://select/groups/4773535/items/UQLXVVYI)) ([pdf](zotero://open-pdf/groups/4773535/items/K9XBXPQD?page=11&annotation=JP9R4Y89))

“Our results indicate that using the phones to transmit more than just water quality data will likely improve the effectiveness and sustainability of this type of intervention.” ([Kumpel et al., 2015, p. 10846](zotero://select/groups/4773535/items/GPM4C7RJ)) ([pdf](zotero://open-pdf/groups/4773535/items/7VXVKEXK?page=1&annotation=4DJIADX2))

“USING MOBILE PHONES TO MONITOR AND MANAGE WATER SUPPLY QUALITY IN RURAL ENVIRONMNETS” ([Wilson-Jones and Rivett, 2012, p. 1](zotero://select/groups/4773535/items/ZTCP6ZDX)) ([pdf](zotero://open-pdf/groups/4773535/items/KQN2HBTG?page=1&annotation=LWS8S33I))

“MOBILE PHONE APPLICATIONS FOR WATER MANAGEMENT: CLASIFFICATION, OPPORTUNITIES AND CHALLENGES” ([Alfonso and Jonoski, 2012, p. 1](zotero://select/groups/4773535/items/4W4Q8E6B)) ([pdf](zotero://open-pdf/groups/4773535/items/PU944FGI?page=1&annotation=R8CDW3HI))

4: Start Network (2022): Integrating community voices in anticipatory action: a synthesis of complex qualitative data. Anticipatory-hub.org/news/integrating-community-voices-in-anticipatory-action-a-synthesis-of-complex-qualitative-data. [15.09.2022].

“Citizen science programmes are promising cost-efficient methods to monitor environmental resources, which make them especially suitable for low-income countries to overcome their sparse data resolution.” ([Weeser et al., 2018, p. 1598](zotero://select/groups/4773535/items/SFA2MLHC)) ([pdf](zotero://open-pdf/groups/4773535/items/GP79FHFC?page=9&annotation=4E9JCTQ5))
“Since today's citizen science studies are mostly located in high-income countries, we are enthusiastic to motivate the scientific community to conduct citizen science studies in low-income countries.” ([Weeser et al., 2018, p. 1598](zotero://select/groups/4773535/items/SFA2MLHC)) ([pdf](zotero://open-pdf/groups/4773535/items/GP79FHFC?page=9&annotation=TYD7Q2ZD))

“According to the Hyogo Framework for Action, increasing resilience to drought requires the development of a people-centered monitoring and early warning system, or in other words, a system capable of providing useful and understandable information to the community at risk.” ([Giordano et al., 2013, p. 523](zotero://select/groups/4773535/items/B7LM5ZR4)) ([pdf](zotero://open-pdf/groups/4773535/items/7I66DBIK?page=1&annotation=ELTIKEZV))

“Success factors for citizen science projects in water quality monitoring” ([San Llorente Capdevila et al., 2020, p. 1](zotero://select/groups/4773535/items/26SVUJIY)) ([pdf](zotero://open-pdf/groups/4773535/items/FHIUZZ6Y?page=1&annotation=3QTPUG53))

“After 1 y, we detect modest reductions in groundwater pumping and modest improvements in water quality and user satisfaction. Although replications are needed, the results imply that externally encouraged, community-based monitoring can improve the management of shared resources.” ([Bernedo Del Carpio et al., 2021, p. 1](zotero://select/groups/4773535/items/4ZDB6DYE)) ([pdf](zotero://open-pdf/groups/4773535/items/D7G4LR6W?page=1&annotation=HQ4A6SZV))

“Community-based monitoring and the science of water quality” ([Conrad, p. 217](zotero://select/groups/4773535/items/ZKTQ2GB5)) ([pdf](zotero://open-pdf/groups/4773535/items/3DN7838Q?page=1&annotation=G24Y695Z))

“This is because citizen  science is most successful when:   
• the aim/questions are clear;  
• engagement with people is given a high priority;  
• sufficient resources are available to begin and continue the project until its completion;  
• scale of sampling is relatively large (because it is often not cost‐efficient to use a citizen  science approach at small spatio‐temporal scales);  
• the protocol required for data collection is not too complex.”
([Pocock et al., p. 3](zotero://select/groups/4773535/items/XDEG6K4D)) ([pdf](zotero://open-pdf/groups/4773535/items/GWG6VVIR?page=4&annotation=RPCPV6R4))

“Three main benefits of citizen science are identified:” ([Minkman, 2015, p. 173](zotero://select/groups/4773535/items/ZKLE6CPT)) ([pdf](zotero://open-pdf/groups/4773535/items/QMAPCSZG?page=173&annotation=MSPY93QA))
1. Low cost data collection method
2. Flexible monitoring scheme
3. Involves Citizens


“Citizen Science is a broad term encompassing many different types of environmental monitoring and observation projects that are conducted by non-governmental organizations and individuals, often in partnership with research institutions and government agencies (Cornell Lab of Ornithology, 2015). Community-based water monitoring (CBWM) is a specific faction of citizen science that has become increasingly sophisticated and widespread, notably in Canada, the United States, Australia, the United Kingdom, the Netherlands, and Southeast Asia (Koehler & Koontz, 2008).” ([Weston and Conrad, 2015, p. 1](zotero://select/groups/4773535/items/49HXDHSH)) ([pdf](zotero://open-pdf/groups/4773535/items/CCHM5SNH?page=1&annotation=F2WVK6BV))

% is volunteer monitoring any good?
“Results from the statewide TST dataset include 82 separate station/year ANOVAs and demonstrate that large-scale, existing volunteer and professional data with unpaired samples can show agreement of~80\% for all analyzed parameters (DO =77\%, pH =79\%, conductivity = 85\%). Inaddition, toassess whether limiting variation within the source datasets increased the level ofagreement between volunteers and professionals, data were analyzed atalocal scale. Data from asingle partner city, with increased controls on sampling times and locations and correction ofasystematic bias inDO, confirmed this by showing an even greater agreement of91\% overall from 2009–2017 (DO =91\%, pH =83\%, conductivity =100\%).” ([Albus et al., 2020, p. 1](zotero://select/groups/4773535/items/DMCGSGUU)) ([pdf](zotero://open-pdf/groups/4773535/items/PRI5ALTR?page=1\&annotation=63J3WN2P))

“The results suggest that citizen science can be a cost-effective method to collect essential monitoring information and can also produce the high levels of citizen engagement that are vital to the adaptive management learning process.” ([Aceves-Bueno et al., 2015, p. 493](zotero://select/groups/4773535/items/YK2MKLA9)) ([pdf](zotero://open-pdf/groups/4773535/items/WGGHNGZB?page=1&annotation=ZQQBEP74))

“Citizen science in environmental and ecological sciences” ([Fraisl et al., 2022, p. 1](zotero://select/groups/4773535/items/FBJD7SWS)) ([pdf](zotero://open-pdf/groups/4773535/items/7WBDKYDY?page=1&annotation=5HMYC85E))
% -------------------------------------------------------------------------------------------
\subsubsection{Volunteersensing}
what about the volunteers? how many? where are they? How to they spread over the country?
“Meadow et al. (2013) recommended using trained agency staff to report drought status on a regular basis” ([Lackstrom et al., 2022, p. 27](zotero://select/groups/4773535/items/YI366LQY)) ([pdf](zotero://open-pdf/groups/4773535/items/3JTQ72UN?page=27\&annotation=AUZV7SZN))

provide hypothesis - 'integration' of local stakeholders/volunteers into drought and water source monitoring can help to get a better picture for early drought impact assessment and thus better and faster management and reactions
+ local people are engaged with the process and come into contact with 'scientific' knowledge/forecasts 
+ equal inclusion of local / indigenous knowledge and scientific forecasts can enhance the quality and make it more relevant on smaller scales (meso/mikro/local level)

knowledge co-production
\autocite{dasInteractiveInformationCrowdsourcing2016}

benefits of monitoring by external support:
“Therefore, the model of community management combined with external support has far-reaching benefits to rural water supply.” ([Huang et al., 2020, p. 144](zotero://select/groups/4773535/items/9CSBLJNJ)) ([pdf](zotero://open-pdf/groups/4773535/items/G5BEZQ7C?page=9&annotation=HU2CNZC2))

“For local communities, their needs of safe drinking water could be met and their abilities to manage and maintain water supply could be enhanced.” ([Huang et al., 2020, p. 144](zotero://select/groups/4773535/items/9CSBLJNJ)) ([pdf](zotero://open-pdf/groups/4773535/items/G5BEZQ7C?page=9&annotation=AL2NB5DK))

“For external experts, they may also benefit from community support to inform scientific processes, such as collecting data that spans across a large geographic region and having an enhanced understanding of community interests.” ([Huang et al., 2020, p. 144](zotero://select/groups/4773535/items/9CSBLJNJ)) ([pdf](zotero://open-pdf/groups/4773535/items/G5BEZQ7C?page=9&annotation=QPIKDKB9))

“Furthermore, this model could help increase scientific awareness among community members and engage the community with the environment.” ([Huang et al., 2020, p. 144](zotero://select/groups/4773535/items/9CSBLJNJ)) ([pdf](zotero://open-pdf/groups/4773535/items/G5BEZQ7C?page=9&annotation=FYU4BIKP))

“During the management of rural drinking water sources, a hybrid modality in which community management is the mainstay with supplement from external support from other organizations is highly recommended.” ([Huang et al., 2020, p. 147](zotero://select/groups/4773535/items/9CSBLJNJ)) ([pdf](zotero://open-pdf/groups/4773535/items/G5BEZQ7C?page=12&annotation=SQ6P8UBN))

“Community cultures, economies, and environments differ across countries and regions. These differences should be considered when designing hybrid management strategies, so that all actors can be appropriately enabled and the mechanism which is most effective for the given community can be identified.” ([Huang et al., 2020, p. 147](zotero://select/groups/4773535/items/9CSBLJNJ)) ([pdf](zotero://open-pdf/groups/4773535/items/G5BEZQ7C?page=12&annotation=WV5DXV5I))

“On this basis, it is essential to expand research area to study the various threats from climate variability to rural drinking water safety, and then to develop corresponding measures to address those threats to water security.” ([Huang et al., 2020, p. 147](zotero://select/groups/4773535/items/9CSBLJNJ)) ([pdf](zotero://open-pdf/groups/4773535/items/G5BEZQ7C?page=12&annotation=HCQHR7YR))

% -------------------------------------------------------------------------------------------
\subsubsection{Motivation of Volunteers}

“we conclude that the active participation is not depending on the actual education level but rather induced by their personal perception of and dependency on their environment. Especially citizens who depend on local water resources are expected to be interested in increasing their understanding of their environment and to participate in local political decisions to ensure a sustainable use of their resources (Overdevest et al., 2004).” ([Weeser et al., 2018, p. 1596](zotero://select/groups/4773535/items/SFA2MLHC)) ([pdf](zotero://open-pdf/groups/4773535/items/GP79FHFC?page=7&annotation=58MAGPAQ))

“At the same time, low participation rates at some stations can be attributed partly to the transmitting cost of 0.01 USD per text message, which was paid by the volunteers. Especially in rural areas, participants expressed that they might be unable to participate due to costs.” ([Weeser et al., 2018, p. 1596](zotero://select/groups/4773535/items/SFA2MLHC)) ([pdf](zotero://open-pdf/groups/4773535/items/GP79FHFC?page=7&annotation=86AIS5QQ))

“Buytaert et al. (2014) described that observers in low-income countries often derive an income from their engagement in citizen-science projects. These authors argue, that the concept of sending data voluntarily is not well developed, and that it may be necessary to reward people at local wages for motivation.” ([Weeser et al., 2018, p. 1596](zotero://select/groups/4773535/items/SFA2MLHC)) ([pdf](zotero://open-pdf/groups/4773535/items/GP79FHFC?page=7&annotation=JUZWKKNE))

“We found that paying a small reward that covers the costs significantly increases the overall participation rate.” ([Weeser et al., 2018, p. 1596](zotero://select/groups/4773535/items/SFA2MLHC)) ([pdf](zotero://open-pdf/groups/4773535/items/GP79FHFC?page=7&annotation=PTWU32DL))
--> seven times larger with reimbursement system in place


or: collective coverage of the costs

“Instead of a reimbursement centrally paid by the project, interested water users organised an own reward system by collecting a contribution from several users to reimburse one person recording the water level data. However, a real payment or reward was not necessary, since the intrinsic motivation of the participants seemed to be sufficient when lack of money was overcome.” ([Weeser et al., 2018, p. 1597](zotero://select/groups/4773535/items/SFA2MLHC)) ([pdf](zotero://open-pdf/groups/4773535/items/GP79FHFC?page=8&annotation=S4IKR9K9))


%-----------------------------------
%	SUBSECTION 6.3
%-----------------------------------

\subsection{Mobile Crowdsensing (MCS)}

participatory monitoring instead? or: citizen observatories, community based monitoring and participatory monitoring
“Participatory monitoring, as I will call it from now, can be useful to increase the density of a monitoring network.” ([Minkman, 2015, p. 199](zotero://select/groups/4773535/items/ZKLE6CPT)) ([pdf](zotero://open-pdf/groups/4773535/items/QMAPCSZG?page=199&annotation=XP2RRKYQ))
“Furthermore it is an interesting communication tool in the light of science communication. Correspondingly water managers should be interested in participatory monitoring in the light of integrated water management.” ([Minkman, 2015, p. 199](zotero://select/groups/4773535/items/ZKLE6CPT)) ([pdf](zotero://open-pdf/groups/4773535/items/QMAPCSZG?page=199&annotation=GHI9KSDA))


“Citizen science in environmental and ecological sciences” ([Fraisl et al., 2022, p. 1](zotero://select/groups/4773535/items/FBJD7SWS)) ([pdf](zotero://open-pdf/groups/4773535/items/7WBDKYDY?page=1&annotation=5HMYC85E))

“MCS is a special application of citizen science, where a mobile device, often smartphones, supports data collection and transmission.” ([Minkman, 2015, p. 11](zotero://select/groups/4773535/items/ZKLE6CPT)) ([pdf](zotero://open-pdf/groups/4773535/items/QMAPCSZG?page=11&annotation=VL9SHGYM))

“It is suggested, based on literature, that water authorities might prefer MCS to wireless sensor networks, for its mobility, lower costs and scalability.” ([Minkman, 2015, p. 11](zotero://select/groups/4773535/items/ZKLE6CPT)) ([pdf](zotero://open-pdf/groups/4773535/items/QMAPCSZG?page=11&annotation=RM5PX55M))

“The most prominent challenge is to balance privacy issues with data trustworthiness.” ([Minkman, 2015, p. 11](zotero://select/groups/4773535/items/ZKLE6CPT)) ([pdf](zotero://open-pdf/groups/4773535/items/QMAPCSZG?page=11&annotation=CCDRFEEH))

!!!!!!!!!!!!!!!!!!!!!!!!!!!!!!!!!!!!!!!!!!!!!!!!!!!!!!!!!!!!!!!!!!!!!!!!!!!!!!!!!!!!!!!!!!!!!!!!!!!!!!!!!!!!!!!!!!!!!!!!!!!!!!!!!!!!!!!!!!!!!!!!!!!!!!!!!!!!!
“A Survey on Mobile Crowdsensing Systems: Challenges, Solutions, and Opportunities” ([Capponi et al., 2019, p. 2419](zotero://select/groups/4773535/items/ZWU5RVDR)) ([pdf](zotero://open-pdf/groups/4773535/items/LY8KZ2MZ?page=1&annotation=ELYSHCCA))
!!!!!!!!!!!!!!!!!!!!!!!!!!!!!!!!!!!!!!!!!!!!!!!!!!!!!!!!!!!!!!!!!!!!!!!!!!!!!!!!!!!!!!!!!!!!!!!!!!!!!!!!!!!!!!!!!!!!!!!!!!!!!!!!!!!!!!!!!!!!!!!!!!!!!!!!!!!!!

timeline and extensive insights:
“B. Timeline” ([Capponi et al., 2019, p. 2423](zotero://select/groups/4773535/items/ZWU5RVDR)) ([pdf](zotero://open-pdf/groups/4773535/items/LY8KZ2MZ?page=5&annotation=SE5W8NPH))
“III. MOBILE CROWDSENSING IN A NUTSHELL” ([Capponi et al., 2019, p. 2426](zotero://select/groups/4773535/items/ZWU5RVDR)) ([pdf](zotero://open-pdf/groups/4773535/items/LY8KZ2MZ?page=8&annotation=2H7S77GL))

“Participatory Sensing or Participatory Nonsense? — Mitigating the Effect of Human Error on Data Quality in Citizen Science” ([Budde et al., 2017, p. 391](zotero://select/groups/4773535/items/8422CFKA)) ([pdf](zotero://open-pdf/groups/4773535/items/5N8N23KN?page=1&annotation=VAW3B38X))

“MOBILE PHONE APPLICATIONS FOR WATER MANAGEMENT: CLASIFFICATION, OPPORTUNITIES AND CHALLENGES” ([Alfonso and Jonoski, 2012, p. 2](zotero://select/groups/4773535/items/4W4Q8E6B)) ([pdf](zotero://open-pdf/groups/4773535/items/PU944FGI?page=2&annotation=QV3HATMM))

“Building a Platform to Collect Crowdsensing Data. Preliminary Considerations” ([George et al., 2017, p. 2](zotero://select/groups/4773535/items/GCM2RQZ2)) ([pdf](zotero://open-pdf/groups/4773535/items/RQ6QQ58V?page=2&annotation=3NXULEJJ))

“A Survey on Mobile Crowdsensing Systems: Challenges, Solutions, and Opportunities” ([Capponi et al., 2019, p. 2419](zotero://select/groups/4773535/items/ZWU5RVDR)) ([pdf](zotero://open-pdf/groups/4773535/items/LY8KZ2MZ?page=1&annotation=ELYSHCCA))

“Building a Platform to Collect Crowdsensing Data. Preliminary Considerations” ([George et al., 2017, p. 2](zotero://select/groups/4773535/items/GCM2RQZ2)) ([pdf](zotero://open-pdf/groups/4773535/items/RQ6QQ58V?page=2&annotation=3NXULEJJ))

“Transmitting the observations using simple cell phones and text messages turned out to be stable and reliable without major technical problems.” ([Weeser et al., 2018, p. 1597](zotero://select/groups/4773535/items/SFA2MLHC)) ([pdf](zotero://open-pdf/groups/4773535/items/GP79FHFC?page=8&annotation=R57CUAHW))

“Furthermore, the participants were able to send text messages without additional training.” ([Weeser et al., 2018, p. 1597](zotero://select/groups/4773535/items/SFA2MLHC)) ([pdf](zotero://open-pdf/groups/4773535/items/GP79FHFC?page=8&annotation=TNVKTF4W))

“The analysis also provides a set of recommendations for citizen science program design that addresses spatial and temporal scale, data quality, costs, and effective incentives to facilitate participation and integration of findings into adaptive management.” ([Aceves-Bueno et al., 2015, p. 493](zotero://select/groups/4773535/items/YK2MKLA9)) ([pdf](zotero://open-pdf/groups/4773535/items/WGGHNGZB?page=1&annotation=3NRZ7R8Y))

“When Are Mobile Phones Useful for Water Quality Data Collection? An Analysis of Data Flows and ICT Applications among Regulated Monitoring Institutions in Sub-Saharan Africa” ([Kumpel et al., 2015, p. 10846](zotero://select/groups/4773535/items/GPM4C7RJ)) ([pdf](zotero://open-pdf/groups/4773535/items/7VXVKEXK?page=1&annotation=M5J4FFSH))

%----------------------------------------------------------------------------------------
%	SECTION 7 CBS & other tools + water related monitoring (excel)
%----------------------------------------------------------------------------------------


\section{CBS & other tools + water related monitoring (excel)}
% CBS
% water monitoring options -> can later be seen in table XY
% short introduction of other tools: Kobo, Ushahidi, NYSS, excel, etc. ..

what does 'monitoring' mean? How is it defined?

%-----------------------------------
%	SUBSECTION 7.1
%-----------------------------------
\subsection{CBS}

“Conducting an Assessment for Community-based surveillance (CBS)” ([Goodermote, 2020, p. 1](zotero://select/groups/4773535/items/A35DRWSR)) ([pdf](zotero://open-pdf/groups/4773535/items/F795EXE4?page=1&annotation=XPIMQ7SP))

1: IFRC. (n.d.). Community Engagement and Accountability. https://www.ifrc.org/community-engagement-and-accountability. [15.09.2022].
2: SRCS. (2021). Measles outbreak detected by Somaliland SRCS Volunteers in Todgheer Region. https://drive.google.com/file/d/1O9PMPKKL312o1zbXELgB7FuMdokzWpic/view. [15.09.2022].
3: SRCS (2022): Feasibility Study on Potential Use of Forecast-based Financing (FbF) for SRCS Final Report. Nottawasage Institute.
4: Start Network (2022): Integrating community voices in anticipatory action: a synthesis of complex qualitative data. Anticipatory-hub.org/news/integrating-community-voices-in-anticipatory-action-a-synthesis-of-complex-qualitative-data. [15.09.2022].

“Community-based surveillance Protocol template” ([IFRC, p. 1](zotero://select/groups/4773535/items/4I8DIUK3)) ([pdf](zotero://open-pdf/groups/4773535/items/LJ5B2HNS?page=1&annotation=3KDKKYAS))


%-----------------------------------
%	SUBSECTION 7.2
%-----------------------------------

\subsection{community based water monitoring}
see sources zotero

“Community-based water monitoring (CBWM) is a specific faction of citizen science that has become increasingly sophisticated and widespread, notably in Canada, the United States, Australia, the United Kingdom, the Netherlands, and Southeast Asia (Koehler & Koontz, 2008).” ([Weston and Conrad, 2015, p. 1](zotero://select/groups/4773535/items/49HXDHSH)) ([pdf](zotero://open-pdf/groups/4773535/items/CCHM5SNH?page=1&annotation=7FA4WQ2Y))

mostly concerned about water quality

“Data inaccuracy, access limitations, a lack of thresholds and inadequate resources for data collection resulted in non-standardised information, preventing comparison across areas and weakening strategic decision-making” ([Gualazzini, 2021, p. 4](zotero://select/groups/4773535/items/BWDYDL8T)) ([pdf](zotero://open-pdf/groups/4773535/items/8U5XVU5K?page=4&annotation=L5SELDEJ))

“Even when receiving timely information on the ground, Members often had to wait for validation from national sources such as the Food Security and Nutrition Analysis Unit (FSNAU) to make decisions, and the resources for anticipatory action were very limited.” ([Gualazzini, 2021, p. 4](zotero://select/groups/4773535/items/BWDYDL8T)) ([pdf](zotero://open-pdf/groups/4773535/items/8U5XVU5K?page=4&annotation=CZST8S2L))

“USING MOBILE PHONES TO MONITOR AND MANAGE WATER SUPPLY QUALITY IN RURAL ENVIRONMNETS” ([Wilson-Jones and Rivett, 2012, p. 1](zotero://select/groups/4773535/items/ZTCP6ZDX)) ([pdf](zotero://open-pdf/groups/4773535/items/KQN2HBTG?page=1&annotation=LWS8S33I))

“Management of Drinking Water Source in Rural Communities under Climate Change” ([Huang et al., 2020, p. 136](zotero://select/groups/4773535/items/9CSBLJNJ)) ([pdf](zotero://open-pdf/groups/4773535/items/G5BEZQ7C?page=1&annotation=92FTSBXQ))

\subsection{other tools}

Ushahidi
CBS/NYSS
DIPAS \& SketchMapTool (as representatives for strategic data/long-term management)

excel sheet table

“2. Key concepts of participatory early warning and monitoring systems (pEWMS)” ([“Participatory early warning and monitoring systems_ A Nordic framework for web-based flood risk management | Elsevier Enhanced Reader”, p. 1296](zotero://select/groups/4773535/items/JYH8N2BV)) ([pdf](zotero://open-pdf/groups/4773535/items/JITPV84L?page=2&annotation=B2ELHBPJ))

“Adding “bottom-up” approaches [36] to classical EWMS allows stakeholders with access to local knowledge of environments and local networks to play a stronger role in decisionmaking and risk management” ([“Participatory early warning and monitoring systems_ A Nordic framework for web-based flood risk management | Elsevier Enhanced Reader”, p. 1296](zotero://select/groups/4773535/items/JYH8N2BV)) ([pdf](zotero://open-pdf/groups/4773535/items/JITPV84L?page=2&annotation=AXZXAK9D))

Lake water quality 
“The results of this review, however, highlighted no usable preexisting system for SIMILE project, since the considered applications are too tied to specific geographic locations or too broad presenting no real connection with the lake ecosystem.” ([Carrion et al., 2020, p. 246](zotero://select/groups/4773535/items/7855TDUA)) ([pdf](zotero://open-pdf/groups/4773535/items/XI5TRN34?page=2&annotation=4P6WS72K))

“Moreover, the majority of the explored solutions are not free and open source.” ([Carrion et al., 2020, p. 246](zotero://select/groups/4773535/items/7855TDUA)) ([pdf](zotero://open-pdf/groups/4773535/items/XI5TRN34?page=2&annotation=EY6HDG2D))

“Transmitting the observations using simple cell phones and text messages turned out to be stable and reliable without major technical problems.” ([Weeser et al., 2018, p. 1597](zotero://select/groups/4773535/items/SFA2MLHC)) ([pdf](zotero://open-pdf/groups/4773535/items/GP79FHFC?page=8&annotation=R57CUAHW))

“Text messages are a common way of communication and significantly lowered the technical barrier to contribute and send data.” ([Weeser et al., 2018, p. 1597](zotero://select/groups/4773535/items/SFA2MLHC)) ([pdf](zotero://open-pdf/groups/4773535/items/GP79FHFC?page=8&annotation=NH68HN6C))

“Furthermore, the feedback loop allowed participants to identify whether their observation was correctly received.” ([Weeser et al., 2018, p. 1597](zotero://select/groups/4773535/items/SFA2MLHC)) ([pdf](zotero://open-pdf/groups/4773535/items/GP79FHFC?page=8&annotation=BM5TYHNZ))

“We occasionally faced phone network coverage issues.” ([Weeser et al., 2018, p. 1597](zotero://select/groups/4773535/items/SFA2MLHC)) ([pdf](zotero://open-pdf/groups/4773535/items/GP79FHFC?page=8&annotation=K6NJJJY9))
“However, those stations with restricted network availability did not turn out as a limited factor for data contribution.” ([Weeser et al., 2018, p. 1597](zotero://select/groups/4773535/items/SFA2MLHC)) ([pdf](zotero://open-pdf/groups/4773535/items/GP79FHFC?page=8&annotation=NF5R6J5H))
“Observers took the readings of the water level and waited until they reached an area with network coverage to send their messages.” ([Weeser et al., 2018, p. 1597](zotero://select/groups/4773535/items/SFA2MLHC)) ([pdf](zotero://open-pdf/groups/4773535/items/GP79FHFC?page=8&annotation=D659LB5J))

“In comparison to more sophisticated methods, like using smartphones, we believe that this approach produces more and, in turn, more reliable results in a low-income country because wrong data and outliers become obvious.” ([Weeser et al., 2018, p. 1597](zotero://select/groups/4773535/items/SFA2MLHC)) ([pdf](zotero://open-pdf/groups/4773535/items/GP79FHFC?page=8&annotation=VP5QT8UJ))

“A major weakness of the existing tools is the emphasis on macro/international level information.” ([Masinde and Bagula, 2010, p. 390](zotero://select/groups/4773535/items/JNC4ACZS)) ([pdf](zotero://open-pdf/groups/4773535/items/IWMKDQYV?page=1&annotation=7Z44Z9L7))

“The tools also tend to ignore the at risk community who happen to be host to very crucial traditional knowledge on droughts” ([Masinde and Bagula, 2010, p. 390](zotero://select/groups/4773535/items/JNC4ACZS)) ([pdf](zotero://open-pdf/groups/4773535/items/IWMKDQYV?page=1&annotation=VUYYP2LA))

“The analysis also provides a set of recommendations for citizen science program design that addresses spatial and temporal scale, data quality, costs, and effective incentives to facilitate participation and integration of findings into adaptive management.” ([Aceves-Bueno et al., 2015, p. 493](zotero://select/groups/4773535/items/YK2MKLA9)) ([pdf](zotero://open-pdf/groups/4773535/items/WGGHNGZB?page=1&annotation=3NRZ7R8Y))

“Moving beyond the Technology: A Socio-technical Roadmap for Low-Cost Water Sensor Network Applications” ([Mao et al., 2020, p. 9145](zotero://select/groups/4773535/items/VEJUB8D9)) ([pdf](zotero://open-pdf/groups/4773535/items/KMCIFXB8?page=1&annotation=2A7QA3V3))
“Participatory early warning and monitoring systems: A Nordic framework for web-based flood risk management” ([Henriksen et al., 2018, p. 1295](zotero://select/groups/4773535/items/KH93MATT)) ([pdf](zotero://open-pdf/groups/4773535/items/QCMYPDEV?page=1&annotation=9ACGDGN2))

“Smartphone‑based System for water quality analysis” ([Srivastava et al., 2018, p. 1](zotero://select/groups/4773535/items/DJPVRP9R)) ([pdf](zotero://open-pdf/groups/4773535/items/QT5ZAARM?page=1&annotation=6W6EFD7R))

“CWDAT—An Open-Source Tool for the Visualization and Analysis of Community-Generated Water Quality Data” ([Gray et al., 2021, p. 1](zotero://select/groups/4773535/items/ZZL9GTV7)) ([pdf](zotero://open-pdf/groups/4773535/items/JSQU75CF?page=1&annotation=8G4ZQ4PD))

“Remote monitoring of rural water systems: A pathway to improved performance and sustainability?” ([Thomson, 2021, p. 1](zotero://select/groups/4773535/items/UQLXVVYI)) ([pdf](zotero://open-pdf/groups/4773535/items/K9XBXPQD?page=1&annotation=B59B5U68))

“Global Monitoring of Water Supply and Sanitation: History, Methods and Future Challenges” ([Bartram et al., 2014, p. 8137](zotero://select/groups/4773535/items/6AWUJTW5)) ([pdf](zotero://open-pdf/groups/4773535/items/BFNSQGWS?page=1&annotation=ZWSBJVDM))

“When Are Mobile Phones Useful for Water Quality Data Collection? An Analysis of Data Flows and ICT Applications among Regulated Monitoring Institutions in Sub-Saharan Africa” ([Kumpel et al., 2015, p. 10846](zotero://select/groups/4773535/items/GPM4C7RJ)) ([pdf](zotero://open-pdf/groups/4773535/items/7VXVKEXK?page=1&annotation=M5J4FFSH))

% drought early warning.. possibly write a short paragraph about that as well.. might be a nice addition to the monitoring approach.
“Local early warning systems for drought – Could they add value to nationally disseminated seasonal climate forecasts?” ([Andersson et al., 2020, p. 1](zotero://select/groups/4773535/items/KYRIJQQB)) ([pdf](zotero://open-pdf/groups/4773535/items/NPA46JZU?page=1&annotation=NTEYIPV6))

\subsubsection{CBS}
https://www.cbsrc.org/

\subsubsection{NYSS}

\subsubsection{CoCoRaHS}
“Backyard Hydroclimatology: Citizen Scientists Contribute to Drought Detection and Monitoring” ([Lackstrom et al., 2022, p. 1](zotero://select/groups/4773535/items/YI366LQY)) ([pdf](zotero://open-pdf/groups/4773535/items/3JTQ72UN?page=1&annotation=XH76JMTR))

\subsubsection{Ushahidi} %ggf. # Sahana Eden (just as a project what is done in the open-source software world)

\subsubsection{Social.Water}
foundation for CrowdHydrology
\subsubsection{CrowdHydrology (?)} 
“The CrowdHydrology network is dominated by one-time participants who submitted just under half of all observations received into the system” ([Lowry et al., 2019, p. 4](zotero://select/groups/4773535/items/3TVRMCP5)) ([pdf](zotero://open-pdf/groups/4773535/items/TDWAUJIY?page=4&annotation=I7D2QPWW))
“Modifying the methodology in these two ways could fix perceived issues of lack of information, interest, and feedback among onetime participants.” ([Lowry et al., 2019, p. 6](zotero://select/groups/4773535/items/3TVRMCP5)) ([pdf](zotero://open-pdf/groups/4773535/items/TDWAUJIY?page=6&annotation=4XPYEMTQ))
“Based on opportunistic conversations with participants we have learned that citizen scientists would like feedback that their message has been received. This is consistent with others’ findings as well (Devlin et al., 2001; Rotman et al., 2014).” ([Lowry et al., 2019, p. 7](zotero://select/groups/4773535/items/3TVRMCP5)) ([pdf](zotero://open-pdf/groups/4773535/items/TDWAUJIY?page=7&annotation=PPU3INB5))
“When we initiated the project, we were concerned that a response message might trigger privacy concerns among participants, but subsequent literature has suggested that the value to participation would likely eclipse such privacy concerns, and in our response, we can include an opt-out option.” ([Lowry et al., 2019, p. 7](zotero://select/groups/4773535/items/3TVRMCP5)) ([pdf](zotero://open-pdf/groups/4773535/items/TDWAUJIY?page=7&annotation=7C3X7EBN))
“While altruistic reasons have been observed to be a motivation for volunteers early in a program, we know that building personal networks is cited as a valued outcome and motivator for continued participation over time (Ryan et al., 2001; Gooch, 2005). This deserves future attention.” ([Lowry et al., 2019, p. 8](zotero://select/groups/4773535/items/3TVRMCP5)) ([pdf](zotero://open-pdf/groups/4773535/items/TDWAUJIY?page=8&annotation=PH9MDNBM))

---
very similar approach:
+ “Citizen science pioneers in Kenya – A crowdsourced approach for hydrological monitoring” ([Weeser et al., 2018, p. 1590](zotero://select/groups/4773535/items/SFA2MLHC)) ([pdf](zotero://open-pdf/groups/4773535/items/GP79FHFC?page=1&annotation=R92YZEVU))

\subsubsection{ITIKI}
“Development Journal 283 Forecast Indicators Integrated Weather Forecasting Application Server Indigenous Knowledge Webserver Indigenous Knowledge Database Server Indigenous Knowledge Focus Group Village Internet Kiosk 2a 3 2b Query Optimizer and Data Mining Tools PC for Uploading Data if WAP not Active Intermediary Intermediary Travels to Digital Village Source 1 Source n-1” ([Masinde and Bagula, 2012, p. 283](zotero://select/groups/4773535/items/EW9XSSZP)) ([pdf](zotero://open-pdf/groups/4773535/items/3WQ4S9PE?page=10&annotation=AXQH5Q7Z))

“ITIKI: bridge between African indigenous knowledge and modern science of drought prediction” ([Masinde and Bagula, 2012, p. 274](zotero://select/groups/4773535/items/EW9XSSZP)) ([pdf](zotero://open-pdf/groups/4773535/items/3WQ4S9PE?page=1&annotation=5BKZELQW))

“Accurate, timely and relevant drought predication information enables a community to anticipate and prepare for droughts and hence minimize the negative impacts.” ([Masinde and Bagula, 2012, p. 274](zotero://select/groups/4773535/items/EW9XSSZP)) ([pdf](zotero://open-pdf/groups/4773535/items/3WQ4S9PE?page=1&annotation=W75TQDCM))

“Current weather forecasts are still alien to African farmers, most of whom live in rural areas and struggle with illiteracy and poor communications infrastructure” ([Masinde and Bagula, 2012, p. 274](zotero://select/groups/4773535/items/EW9XSSZP)) ([pdf](zotero://open-pdf/groups/4773535/items/3WQ4S9PE?page=1&annotation=HPZC9Z65))

%-----------------------------------
%	SUBSECTION 7.3
%-----------------------------------

\subsection{water source mapping}
"In all, mapping presents many benefits, such as:

It makes easier to integrate data from different sources (surveys, censuses, satellites, etc.) and from different disciplines (social, economic, and environmental data). It also allows the switch to new units of analysis from, for example, administrative boundaries (e.g. state) to ecological boundaries (e.g. basin).
Maps are a powerful visual tool and are more easily understood by stakeholders, particularly in developing countries.
The spatial nature of water poverty, such as the distance to the nearest water source or the water supply infrastructure, can also be incorporated easily in a GIS database.
The allocation of resources can be improved, since geographic targeting is more efficient and cost-effective than to launch an equally expensive universal distribution programme.
Geo-referenced databases can be enriched by additional data as they become available; and new attributes, such as better details on water quality, can be incorporated into the data structure, ensuring that the relevance of the data is sustained over time.
Maps can be produced at a number of different resolutions depending on their purpose and the cost of data collection. A coarse resolution or a scale too small neglects the heterogeneity within each unit and provides insufficient detail for decision making, while a fine resolution or a scale too large increases the cost of compiling, managing, and analyzing the data."https://en.wikipedia.org/wiki/Water_point_mapping

“Core questions for drinking water” ([United Nations Children’s Fund (UNICEF) and World Health and World Health Organization, 2018, p. 8](zotero://select/groups/4773535/items/YRLCKKJZ)) ([pdf](zotero://open-pdf/groups/4773535/items/DE3SXICX?page=8&annotation=CIWX3HLI))

“Using Remote Sensing to Map and Monitor Water Resources in Arid and Semiarid Regions” ([Klemas and Pieterse, 2015, p. 33](zotero://select/groups/4773535/items/BVN6IXG5)) ([pdf](zotero://open-pdf/groups/4773535/items/UPSYZXDK?page=1&annotation=4DPZD4BZ))

%----------------------------------------------------------------------------------------
%	SECTION 8 Case Study Area (+ application of the rest)
%----------------------------------------------------------------------------------------


\section{Case Study Area (+ application of the rest)}
% Somaliland geography
% hazards and risks: drought situation (past, present, future)
% Somaliland social and economic circumstances (past, present, future)
% Stakeholder
% resource restrictions

% lots of information about the local stuff
“Study Area” ([Abdulkadir, 2017, p. 104172233225941010000](zotero://select/groups/4773535/items/G2RYLAC4)) ([pdf](zotero://open-pdf/groups/4773535/items/5PPIPUKD?page=1&annotation=TX2QHMDT))


(establish your research territory: general information about the importance, background details to understand studies context)

Overview Confluence Background Overview:
https://heigit.atlassian.net/wiki/spaces/FIS/pages/1704186/Background+Information

https://drive.google.com/file/d/1McCpUWQPNlO-nDF0RjDW7nmrdM75ZJoP/view
%-----------------------------------
%	SUBSECTION 8.1
%-----------------------------------
\subsection{Geography & Climate}

% lots of good stuff here
“Assessment of Drought Recurrence in Somaliland: Causes, Impacts and Mitigations” ([Abdulkadir, 2017, p. 104172233225941010000](zotero://select/groups/4773535/items/G2RYLAC4)) ([pdf](zotero://open-pdf/groups/4773535/items/5PPIPUKD?page=1&annotation=49M6L9RG))
%-----------------------------------
%	SUBSECTION 8.2
%-----------------------------------

\subsection{Hazards and risks}

https://news.un.org/en/story/2022/08/1125552

“The number of affected people will be 1,200,420 persons across all the six main regions in Somaliland. The top priority needs of the people affected to date are mainly water (70{\%), Food (21\%) and Health (9\%).” ([National Drought Committee, 2022, p. 3](zotero://select/groups/4773535/items/7XJRE6LM)) ([pdf](zotero://open-pdf/groups/4773535/items/2F59E4UZ?page=3&annotation=8JZVBSM6))

%% Risk analysis

"Prioritised hazard and its historical impact.
Drought is currently the most relevant hazard in Somaliland . Droughts have occurred frequently for the last decades and presented varied impacts (The Somaliland drought rapid assessment report commissioned by the National Drought Committee in January 2022).  The occurrence of droughts has increased over the last years from one in five years to one drought every two years. The majority of the population depends on (agro) pastoral agriculture. Since this type of livelihood depends fundamentally on the natural resources of pasture and water, the local population is highly vulnerable to droughts. 
According to the International Database for Disasters ( EM DAT) recent droughts in Somaliland occurred in 2004, 2010 to 2011, 2014 to 2017, 2019, and 2021 to 2023. The drought incidents have mainly caused food shortages due to below-average crop production, livestock loss and water shortages. The impacts are also in line with the historical drought impacts of the communities in the Eastern regions of Somaliland, i.e. Sool, Sanaag and Togdheer where communities were consulted in focus group discussions as part of the ongoing FbF project.  Since 2018 there has been a drought in 2019 and the current drought (2020-2023). This is also in line with the historical Combined Drought Index (CDI) data (FAO SWALIM). The CDI is a proven and reliable indicator of drought since it incorporates precipitation, soil moisture and vegetation health. While the country has experienced periodic severe drought conditions over the CDI assessment period from 2002 to 2022, the eastern regions have experienced the most extreme drought conditions, especially in the last two years. Droughts in Somaliland lead to a variety of negative impacts. Chief among them is increased food insecurity. EM DAT  lists food shortage and famine as the principal impact of all droughts since 1987. Furthermore, based on the Food Insecurity Integrated Phase Classification (IPC) data, every drought is associated with substantial spikes in food-insecure populations. For example, the current drought led up to December 2022 to 5.6 million people in or above IPC 3 (crisis). Since 2009, Somaliland has experienced a IPC average between class 2 and 3 (between stressed and crisis), whereby the very western regions Awdal and the eastern regions (Sanaag, Sool and Togdheer) have been most affected reaching IPC classification 3 or worse in almost a quarter of assessment periods (every 2 months). 
The exact processes which lead to food insecurity due to drought are diverse. In the case of pastoral communities, failed rainy seasons result in livestock and herders having to track longer distances to water sources. These circumstances and lower forage and water availability lead to poor livestock health, increasing livestock deaths and lower birth rates, which can amount to the loss of whole herds. Milk production decreases drastically, which is the reason for the lower diet variability of herders. Furthermore, herders have to sell their animals for lower prices due to their lower body conditions and increased supply. This was the case during the 2017 drought. In some cases, whole herds were lost. The price of milk products increased significantly. At the peak of the drought in 2017, 388.000 children suffered from malnutrition and 895.000 people were internally displaced. In total 6.2 million people have been in need of humanitarian assistance.
Agropastoralist communities are affected by droughts in the same way as pastoralist communities. In addition, their livelihood suffers from increased crop failure due to poor soil moisture. This, again, causes agropastoralists to lose their income. Additionally, the local staple food prices rise, putting pressure on people who depend on maize or sorghum as the primary carbohydrate source. 

Similar to higher food prices, drought causes higher water prices as well. Besides fewer available resources to buy quantitative and qualitative adequate food, high water prices lead to poor hygiene practices. In 2017 water shortages due to two consecutive failed rainy seasons led to rising water prices. In the Eastern regions, the cost of 200 litres of water jumped from 2 USD to 7 USD. Similar water shortages during the 2016 drought caused a sharp increase in watery diarrhoea cases to 13.653 cases with a fatality rate of 3.6 %. 
Further consequences of droughts of the last 12 years in Somaliland can be seen in the table below."
https://docs.google.com/document/d/1xUEXm8RxVHTO468KqXSAoBX-cpkPwiff/edit

“SOMALILAND DROUGHT RAPID ASSESSMENT REPORT Commissioned By National Drought Committee” ([National Drought Committee, 2022, p. 1](zotero://select/groups/4773535/items/7XJRE6LM)) ([pdf](zotero://open-pdf/groups/4773535/items/2F59E4UZ?page=1&annotation=UTRIHJIN))

%% + table of the last mayor droughts


% could also be mentioned in the TB introduction
“1.2 DROUGHT IN SOMALIA” ([“Economics of Resilience to Drought: Somalia Analysis”, p. 9](zotero://select/groups/4773535/items/LGRWAU43)) ([pdf](zotero://open-pdf/groups/4773535/items/MBXSCVWR?page=9&annotation=K6W9FFM7))
%-----------------------------------
%	SUBSECTION 8.3
%-----------------------------------

\subsection{Political, social and economic circumstances}
region is vulnerable to droughts due to its predominant reliance on pastoralism as its source of livelihood and economy. Livestock remains an essential source of employment, hard currency, export earnings and government revenue.
in case of failure of rainy season, people migrate from their locations, including pastoralists with their livestock, migrate to areas where better water and pasture are still available during the dry season
seasonal weather patterns in the region and the rest of the regions are characterized by four main seasons - two rainy seasons and two dry seasons. The main rains fall during the Gu-season (AprilJune), with lighter and more sporadic rains falling during the Deyr season (October December). Two dry seasons are characterized by one shorter, cooler season, known as the Hagaa (22nd July to 23rd September), and a long, hot, dry season, known as the Jilaal (January to 23rd March), which is the harshest season of the year


Point water sources are the primary source of water
Traditionally, natural disasters like droughts do not prevail uniformly across the entire area of Somalia but only in some hotspots.
significant increase in GBV and child abuse/abandonment during crises such as drought, flooding, conflict, etc. early action includes deploying extra social workers, training community champions, etc. UNFPA and UNICEF are currently undertaking a national assessment of these issues, and the report will likely be a useful source for identifying approaches for undertaking early action as part of the drought FbF
Hydrogeological survey and assessment of selected areas in Somaliland and Puntland (2012)
“Hydrogeological Survey and Assessment of Selected Areas in Somaliland and Puntland” ([pdf](zotero://open-pdf/groups/4773535/items/KU98BB4Z?page=1&annotation=JA8SYGKT))



%-----------------------------------
%	SUBSECTION 8.4
%-----------------------------------

\subsection{Stakeholder}

“Stakeholders interviewed for this project reinforced this and emphasized the importance of collaboration and cohesion between different methods and approaches, given the multi-faceted nature of the hazard.” ([RCRC, 2020, p. 23](zotero://select/groups/4773535/items/UESIQTRJ)) ([pdf](zotero://open-pdf/groups/4773535/items/P5JPVZ97?page=23&annotation=EQV29795))

https://heigit.atlassian.net/wiki/spaces/FIS/pages/1704213/Stakeholders+Systems+Networks

% also in subsection 7.2
“Adding “bottom-up” approaches [36] to classical EWMS allows stakeholders with access to local knowledge of environments and local networks to play a stronger role in decisionmaking and risk management” ([“Participatory early warning and monitoring systems_ A Nordic framework for web-based flood risk management | Elsevier Enhanced Reader”, p. 1296](zotero://select/groups/4773535/items/JYH8N2BV)) ([pdf](zotero://open-pdf/groups/4773535/items/JITPV84L?page=2&annotation=AXZXAK9D))

“With the increased pressure on water resources and the challenges faced with the implementation of the existing regulatory framework, a growing lack of mutual trust between water stakeholders has been observed in recent years [38].” ([“Participatory early warning and monitoring systems_ A Nordic framework for web-based flood risk management | Elsevier Enhanced Reader”, p. 1296](zotero://select/groups/4773535/items/JYH8N2BV)) ([pdf](zotero://open-pdf/groups/4773535/items/JITPV84L?page=2&annotation=HNTIH8JB))

Selection of stakeholders:
“both information users—i.e., local decision makers—and information producers—i.e., members of the local community that can provide knowledge and information about the drought’s impacts at a local level—should be involved in the design process.” ([Giordano et al., 2013, p. 529](zotero://select/groups/4773535/items/B7LM5ZR4)) ([pdf](zotero://open-pdf/groups/4773535/items/7I66DBIK?page=7\&annotation=MNSWFWEE))

difference scientific vs. stakeholders perception of drought impact
“The comparison between stakeholders’ perception of drought impacts and scientific knowledge allowed us to draw some preliminary conclusions concerning both the drought impacts at local level and the coherence between local and scientific knowledge on drought impacts.” ([Giordano et al., 2013, p. 539](zotero://select/groups/4773535/items/B7LM5ZR4)) ([pdf](zotero://open-pdf/groups/4773535/items/7I66DBIK?page=17\&annotation=QFD3UE4C))
“the correlation degree between perception indicators and scientific indicators is high only when considering a direct impact of drought, for example the reduction of productivity for non-irrigated crops.” ([Giordano et al., 2013, p. 540](zotero://select/groups/4773535/items/B7LM5ZR4)) ([pdf](zotero://open-pdf/groups/4773535/items/7I66DBIK?page=18&annotation=6KRIS4DA))
“Firstly, stakeholders tend to oversimplify the cause-effect chain” ([Giordano et al., 2013, p. 540](zotero://select/groups/4773535/items/B7LM5ZR4)) ([pdf](zotero://open-pdf/groups/4773535/items/7I66DBIK?page=18&annotation=IBV34JQL))
“Stakeholders seem to focus exclusively on the portion of the system they perceived.” ([Giordano et al., 2013, p. 541](zotero://select/groups/4773535/items/B7LM5ZR4)) ([pdf](zotero://open-pdf/groups/4773535/items/7I66DBIK?page=19&annotation=73P565M3))
“On the one hand, this allows them to have a clear picture of the evolutionary trends of the different variables that make up that portion of the system” ([Giordano et al., 2013, p. 541](zotero://select/groups/4773535/items/B7LM5ZR4)) ([pdf](zotero://open-pdf/groups/4773535/items/7I66DBIK?page=19&annotation=ZJSACLXJ))
“On the other hand, due to the limitations of their viewpoint, stakeholders tend to neglect the existence of multiple causes for certain observed phenomena.” ([Giordano et al., 2013, p. 541](zotero://select/groups/4773535/items/B7LM5ZR4)) ([pdf](zotero://open-pdf/groups/4773535/items/7I66DBIK?page=19&annotation=8CUNSCXS))
“Secondly, the data collected shows that the stakeholders tend to aggregate the variables describing the drought impacts, for example the impacts of crop productivity vary dramatically according to the crop under consideration, whereas stakeholders perceived a general reduction in productivity.” ([Giordano et al., 2013, p. 541](zotero://select/groups/4773535/items/B7LM5ZR4)) ([pdf](zotero://open-pdf/groups/4773535/items/7I66DBIK?page=19&annotation=TT7JRW3U))


“This work also demonstrates that there is a lack of synchronism between the stakeholders’ perception and the drought onset.” ([Giordano et al., 2013, p. 541](zotero://select/groups/4773535/items/B7LM5ZR4)) ([pdf](zotero://open-pdf/groups/4773535/items/7I66DBIK?page=19&annotation=H3XRYH3Q))

“In fact, stakeholders become aware of the drought only when they recognize the impacts on their perceived environment.” ([Giordano et al., 2013, p. 541](zotero://select/groups/4773535/items/B7LM5ZR4)) ([pdf](zotero://open-pdf/groups/4773535/items/7I66DBIK?page=19&annotation=869MME8Q))
“Often, those impacts are recognizable only when the drought is at its peak” ([Giordano et al., 2013, p. 541](zotero://select/groups/4773535/items/B7LM5ZR4)) ([pdf](zotero://open-pdf/groups/4773535/items/7I66DBIK?page=19&annotation=XTTUYS8M))

“Therefore, stakeholders’ indicators cannot be used to detect the drought onset or to assess the duration of the phenomenon” ([Giordano et al., 2013, p. 541](zotero://select/groups/4773535/items/B7LM5ZR4)) ([pdf](zotero://open-pdf/groups/4773535/items/7I66DBIK?page=19&annotation=C2QMHW76))

“They are more suitable to support the evaluation of the severity of the drought, and particularly of the indirect impacts.” ([Giordano et al., 2013, p. 541](zotero://select/groups/4773535/items/B7LM5ZR4)) ([pdf](zotero://open-pdf/groups/4773535/items/7I66DBIK?page=19&annotation=K8Z77CBR))

“The results of our experience show that drought perception indicators could be useful for defining thresholds of severity rather than for evaluating drought evolutionary trends. These thresholds could support decision makers in the development of effective drought risk management strategies.” ([Giordano et al., 2013, p. 542](zotero://select/groups/4773535/items/B7LM5ZR4)) ([pdf](zotero://open-pdf/groups/4773535/items/7I66DBIK?page=20&annotation=X47HGTNQ))

%-----------------------------------
%	SUBSECTION 8.5
%-----------------------------------

\subsection{Water sources}
Berkads:
“Changing Pastoralism in the Ethiopian Somali National Regional State (Region 5)” ([“Changing Pastoralism in Region 5”, p. 1](zotero://select/groups/4773535/items/FXJGUTLD)) ([pdf](zotero://open-pdf/groups/4773535/items/BIAA5M57?page=1&annotation=5F9EZJYZ))

"Although birkeds cannot be considered permanent water points in the sense of permanent wells which do not rely on harvesting rainwater, clusters of birkeds represent dry season water points that they provide water throughout the dry season in most years. Today, then, distribution of water points is vastly different from a few decades ago. Map 5 shows the water points that exist today in the five districts under study. The map attempts to show the wells and boreholes as well as the main clusters of birkeds. The latter are difficult to map as there is no existing record of all locations. The map is based on sketch maps drawn by communities during the fieldwork. It is thus not meant to be accurate but to give an indication of the nature of change. It should be noted that these water points shown on the map are also the site of permanent settlements, as the tendency has been for settlements to grow up at the site of new water points."

“IMPROVED BERKAD DESIGNS BY MERCY CORPS - SOMALIA” ([pdf](zotero://open-pdf/groups/4773535/items/F3CWEKHP?page=1&annotation=KH5I94RJ))

image of a Berkad? that is a water source type typical for Somalia

"Criteria used included among others coverage, need for rehabilitation, seasonality of services, quality of water delivered, and poor management." https://en.wikipedia.org/wiki/Water_point_mapping
https://onlinelibrary.wiley.com/doi/abs/10.1111/j.1477-8947.2010.01296.x?casa_token=TPvw51virRQAAAAA:9K-6fcNcYFw9-Mny-EOIMdS6OmmSUkajSfo9qggsMNnAirGYSslUHckFyqWNP68XarnXYEgIh9eL0uGltw

"Water source:
o Out of 77 communities assessed, 49 have berkets, 26 have boreholes, 21 have shallow
wells, 28 communities reported that they receive water trucking.
o Berkads: 76\% of communities having berkets reported that all berkets are reported to be
depleted, while 22\% reported they were less half than full.
o Water trucking: Out of the 28 communities that receive water trucking, 43\% receive water
on a daily basis. 19 out of 28 communities (68\%) receiving water trucking responded that
they receive water from private suppliers. No communities, covered by this assessment,
mentioned water trucking from humanitarian partners.
o Reduction of water consumption: 54 out of 77 assessed communities (69\%) responded
that the majority of community members reduced water consumption in the past 4 weeks,
including all assessed communities in Sool." https://drive.google.com/file/d/1KWUZW0jEMV1Ijc4zeET_Yes93VJ_3qn2/view

“IMPROVED BERKAD DESIGNS BY MERCY CORPS - SOMALIA” ([pdf](zotero://open-pdf/groups/4773535/items/F3CWEKHP?page=1&annotation=KH5I94RJ))

“Water source:” ([National Drought Committee, 2022, p. 8](zotero://select/groups/4773535/items/7XJRE6LM)) ([pdf](zotero://open-pdf/groups/4773535/items/2F59E4UZ?page=8&annotation=CTKWTWRE))

“77 communities assessed,” ([National Drought Committee, 2022, p. 8](zotero://select/groups/4773535/items/7XJRE6LM)) ([pdf](zotero://open-pdf/groups/4773535/items/2F59E4UZ?page=8&annotation=6AN6T6SG))

“49 have berkets” ([National Drought Committee, 2022, p. 8](zotero://select/groups/4773535/items/7XJRE6LM)) ([pdf](zotero://open-pdf/groups/4773535/items/2F59E4UZ?page=8&annotation=7BWVAK62))

“28 communities reported that they receive water trucking.” ([National Drought Committee, 2022, p. 8](zotero://select/groups/4773535/items/7XJRE6LM)) ([pdf](zotero://open-pdf/groups/4773535/items/2F59E4UZ?page=8&annotation=VCLQJBGD))

“7000 berkads in the Haud” ([Birch, 2008, p. 4](zotero://select/groups/4773535/items/X92LGIEA)) ([pdf](zotero://open-pdf/groups/4773535/items/ZWD3FAK4?page=6&annotation=V2XYRC8R))

“The explosion of permanent water points means that the Haud is now grazed all year round, leaving no space for regeneration” ([Birch, 2008, p. 4](zotero://select/groups/4773535/items/X92LGIEA)) ([pdf](zotero://open-pdf/groups/4773535/items/ZWD3FAK4?page=6&annotation=FYWWUNR9))
%----------------------------------------------------------------------------------------
%	SECTION 9 Conclusion literature
%----------------------------------------------------------------------------------------


\section{Conclusion literature}
In the conclusion, you should summarize the key findings you have taken from the literature and emphasize their significance.

%-----------------------------------
%	SUBSECTION 9.1
%-----------------------------------
\subsection{key findings}


%-----------------------------------
%	SUBSECTION 9.2
%-----------------------------------
\subsection{Limitations}


%-----------------------------------
%	SUBSECTION 9.3
%-----------------------------------

\subsection{Key assumptions}



%----------------------------------------------------------------------------------------
%----------------------------------------------------------------------------------------
%----------------------------------------------------------------------------------------


\section{Notes}
% Notes: 
% the literature review is focused on providing background information and enabling historical interpretation of the subject of analysis in relation to the research problem the case is intended to address. This includes synthesizing studies that help to:

% "Place relevant works in the context of their contribution to understanding the case study being investigated. This would include summarizing studies that have used a similar subject of analysis to investigate the research problem. If there is literature using the same or a very similar case to study, you need to explain why duplicating past research is important [e.g., conditions have changed; prior studies were conducted long ago, etc.].
% Describe the relationship each work has to the others under consideration that informs the reader why this case is applicable. Your literature review should include a description of any works that support using the case to study the research problem and the underlying research questions.
% Identify new ways to interpret prior research using the case study. If applicable, review any research that has examined the research problem using a different research design. Explain how your case study design may reveal new knowledge or a new perspective or that can redirect research in an important new direction.
% Resolve conflicts amongst seemingly contradictory previous studies. This refers to synthesizing any literature that points to unresolved issues of concern about the research problem and describing how the subject of analysis that forms the case study can help resolve these existing contradictions.
% Point the way in fulfilling a need for additional research. Your review should examine any literature that lays a foundation for understanding why your case study design and the subject of analysis around which you have designed your study may reveal a new way of approaching the research problem or offer a perspective that points to the need for additional research.
% Expose any gaps that exist in the literature that the case study could help to fill. Summarize any literature that not only shows how your subject of analysis contributes to understanding the research problem, but how your case contributes to a new way of understanding the problem that prior research has failed to do.
% Locate your own research within the context of existing literature [very important!]. Collectively, your literature review should always place your case study within the larger domain of prior research about the problem. The overarching purpose of reviewing pertinent literature in a case study paper is to demonstrate that you have thoroughly identified and synthesized prior studies in the context of explaining the relevance of the case in addressing the research problem."https://libguides.pointloma.edu/c.php?g=944338&p=6806958
