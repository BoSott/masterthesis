% Chapter Template

\chapter{Discussion and Reasoning} % Main chapter title

\label{chapter5} % Change X to a consecutive number; for referencing this chapter elsewhere, use \ref{ChapterX}
% Interpretation of the results that go further than the research questions. This can, e.g., contain implications for software development.


% (in the discussion)
% -> Reiterate the Research Problem/State the Major Findings
% -> Explain the Meaning of the Findings and Why They are Important
% -> Relate the Findings to Similar Studies
% -> Consider Alternative Explanations of the Findings
% -> Acknowledge the Study's Limitations
% evaluate and interpret results
% -> focus on stuff that is directly related to the research aim
% don't report ANY new findings

%----------------------------------------------------------------------------------------
% Step 1: restate your research problem, aim, hypotheses and questions
%----------------------------------------------------------------------------------------
% introduction + overall findings/aim

% method section --> use of frameworks and literature to create a framework + prc
% pros and cons of ssf and slmc + why not others and how it relates to those
% e.g. several other .. have also a staged, iterative process design (.gov, butte)
% prc --> why and how it worked together with the ssf - how it compares to others (does it?)
% + a couple of special cases or highlights e.g. areas with no or only a few indicators/data points that might have been interesting
% e.g. discuss specific products or activities that highlight something + here and there.. 
% quality of the impacts --> risk assessment and correlation --> overlying EAP development (highlight this)
% -> e.g. clear and simple guidelines may be inherently unlikely due to the underlying complexity of each case study and project
% no directly comparable projects -> comparability lacks. Compares well with cbs etc. but it is also build on those recommendations --> name this!
% are there trade-offs?
% e.g. what the focus is comes with inherent drawbacks in other areas
% synergies?
% could add more and more and more but: would loose clarity and focus -> only included what is really necessary and useful
% -> the decomposition of complex situations into specific categories naturally entails limitations -> particularly true for interrelated and influences over time

% "However, the reduction to essential parts also allows clarity and focus that would otherwise not be possible. An attempt was made to name the various advantages and disadvantages and to show that the methodology applied is appropriate in this con-
% text." myself, BA

% application --> case study also used from others for similar attempts
% basically worked well - inherent limitations -> just a research method with cons and pros
% reason why not include the mapping and what could be done about it.

% The augmentation of both the theoretical and empirical underpinnings of this endeavour has served to enhance its overall depth and rigor.

% “Experts in trigger methodology have indicated a more appropriate strategy may be to build on tools that currently exist at the government level such as national drought monitoring systems. As such, the ideal is an iterative process with the ground level along with a technology push that creates new ways to analyse drought and drought risk.” ([RCRC, 2020, p. 28](zotero://select/groups/4773535/items/UESIQTRJ)) ([pdf](zotero://open-pdf/groups/4773535/items/P5JPVZ97?page=28&annotation=977VS8FC))

%! am Ende noch überarbeiten
This study aimed to design and test an approach for community-based participatory mapping and monitoring of water sources in a water-scarce and resource-limited setting in collaboration with the \acrlong{srcs}. The ultimate goal was to facilitate respective \acrlongpl{aa} in the context of \acrlong{fbf} and to improve water management and accessibility in underserved communities. To achieve this aim, four research objectives were formulated, including a comprehensive literature review to identify and evaluate principles for community-based participatory mapping and monitoring, assessing the feasibility of the approach in the given context, developing a replicable and adaptable framework based on the identified guidelines, and applying the framework to create a roadmap for implementation.\newline

The literature and data analysis revealed the high complexity of the context and could determine gaps in the data situation on water sources as well as the project and framework landscape in regard to \acrlong{cs} approaches in the given context for the implementation in a \acrshort{fbf} project. However, the general feasibility of the approach for the project was suggested through further analysis. Building on this positive assessment, the identified frameworks and guidelines were adapted and expanded to ultimately lead to the development of a new replicable and adaptable framework for a community-based participatory water source mapping and monitoring in the context of \acrlong{fbf}. Its application on this specific case area resulted in a roadmap for the practical implementation of the project. This roadmap includes goals and sub-goals, required products and respective activities.\newline

In this discussion chapter, the focus is on reflecting on the main findings and contributions of this study and discuss their implications for further developments and practical applications. In detail, each research objective is addressed in turn and its relevance to the research aim is discussed. Finally, limitations and challenges encountered during the research process are named and considered. 
% The successful application of the developed framework indicates its usability and usefulness.
%----------------------------------------------------------------------------------------
% Step 2: summarise your key findings
%----------------------------------------------------------------------------------------
% themes and relationships (qualitative) and correlations and causality (quantitative)
% -> highlight overall key findings
% -> one or two paragraphs -> be concise

% e.g.
% The data suggest that…
% The data support/oppose the theory that…
% The analysis identifies…

%----------------------------------------------------------------------------------------
% Step 3: interpret your results
%----------------------------------------------------------------------------------------
% unpack the findings (no new information!)
% ,“Climate information presented as early warnings are only as valuable as the actions that are taken in response to the information, even if the information is a perfect warning of future events.” ([Mariani et al., 2015, p. 8](zotero://select/groups/4773535/items/8THVVJVK)) ([pdf](zotero://open-pdf/groups/4773535/items/GYUFNK32?page=8&annotation=Y9DG5FSE))

% -> follow a similar structure as in the result chapter
% or research questions/hypotheses
% or theoretical framework

% how does those results compare to existing research ?! -> lit review

% contrasts are often the most interesting findings --> why? significant?

% - How do your results relate with those of previous studies?
% - If you get results that differ from those of previous studies, why may this be the case?
% - What do your results contribute to your field of research?
% - What other explanations could there be for your findings?

% don't draw conclusions, that aren't substantiated -> everything need to be backed up by something

%?%%%%%%%%%%%%%%%%%%%%%%%%%%%%%%%%%%%%%%%%%%%%%%%%%%%%%%%%%%%%%%%%%%%%%%%%%%%%%%%%%%%%%%%%%%%%%%%%%%%
%  How can a replicable and adaptable framework for community-based participatory water source mapping and monitoring with the aim of facilitating AA in the context of \acrlong{fbf} be developed?
1. RQ
%?%%%%%%%%%%%%%%%%%%%%%%%%%%%%%%%%%%%%%%%%%%%%%%%%%%%%%%%%%%%%%%%%%%%%%%%%%%%%%%%%%%%%%%%%%%%%%%%%%%%
% but those are expected to be solved in future work..
\section{Literature, Project and Data Analysis}
% % data analysis
% An in-depth analysis of existing data sets of water source point and feature information was considered, but  discarded after a first assessment. The very limited reliability, completeness and actuality of the available data sets had already been reviewed and stated by the project team before the start of this work. The lack of data was the main reason for this work to start with, and a short analysis via QGIS3 was able to confirm these statements. 

% was considered, but was discarded at a very early stage. The very limited reliability, completeness and actuality of the available data sets had already been reviewed and stated by the project team before the start of this work. This lack of data was the reason for this work, but a short analysis was able to confirm these statements and thus, due to relevance and time constraints, 
% and thus, due to relevance and time constraints, the focus shifted to the design approach.
% --> poor quality for water source data analysis --> was the reason for this work % nice transitions for reasons for other work

% zweiter Teil noch nicht im Methodenteil genannt
% the focus was more on the design to create exactly such data sets for further analyses. This can be seen as a limitation, but can also be another reason to do exactly this work."


% important: name the mapping campaign! planned as SRCS staff -> educated personal, no rookies, no volunteers.
% biases.. important? mention? discussion? possibly discussion. Have fun old me!

% The identification of the conceptual context along with the literature and \acrshort{cs} project analysis could successfully build a sound foundation for this work and specify relevant aspects and gaps for addressing the first and subsequent research objectives.


\section{A Replicable and Adaptable Framework}

\begin{quote}
    "All models are wrong, but some are useful" George E.P. Box
\end{quote}

While this statement was made in regard to statistical models, the consideration of the trade-off between generalisation and specialisation is also crucial in the design of frameworks. Highly general principles and characteristics up to highly specialised projects can be found in the literature (see section \ref{sec:cs}). The development of the \acrshort{ssdr} has tried to find a balance between the focus on drought, FbF and citizen involvement in Somaliland while also staying adaptable to other, yet comparable projects.\newline
% surprising
Surprising was, that while a manifold of general guidelines, characteristics and quality criteria for \acrlong{cs} projects exist, no grouped and ordered requirements list along potential \acrshort{cs} goals could be found. While this is unexpected, as it is no radically new insight, but merely a different framing of more or less the same information, it could be explained by the limited time of practitioners to publish concrete information. This lack of time for publication was also mentioned by I2 but no peer-reviewed study could be found to either underline or falsify this assumption. However, an interweaving of the more often encountered process-oriented approach with a specific, yet adaptable requirements catalogue was found to be manageable and, as also later discussed in more detail, well applicable.\newline
% chapter structure
In the following, the general development of the \acrshort{ssdr} is discussed and the challenges encountered and potential solutions are considered in more detail, looking first at the \acrshort{ssf} and then at the \acrshort{prc}. When considering the \acrshort{ssdr} and \acrshort{prc} frameworks, it is crucial to acknowledge that they provide only limited perspectives on the complex reality of design processes. This research primarily adopted a process- and requirements-oriented approach in designing and conceptualising the design roadmap. Other perspectives, such as resource, behavioural network or stakeholder networks, cultural norms and values, as well as the communication network perspective may play a role in certain aspects, but are of secondary importance in this work.\newline
% challenges information availability
Challenges in conceptualising the new framework primarily laid in information availability and transferability. Several \acrshort{cbm}, \acrshort{mcs}, \acrshort{cbs}, \acrshort{cbwm} and other risk related \acrshort{cs} frameworks and respective guidelines could be identified but none of them exactly matched the intended application (see section \ref{sec:cs}). While "there is no one-size-fits-all approach" \autocite[2]{fraislCitizenScienceEnvironmental2022}, the existing frameworks either focussed on different thematics, contexts, had different participation levels, different goals or a combination of the above (see sections \ref{subsubsec:cbwm}, \ref{subsubsec:cbs} and \ref{subsec:cbc}). This is consistent with \autocite{butteFrameworkWaterSecurity2022}'s and \autocite{carrionCROWDSOURCINGWATERQUALITY2020}'s findings that existing frameworks guiding the development of water security data collection projects are often very specific and limited to certain factors, in many cases also not taking socio-economic factors into account. At the same time, frameworks like the on from \autocite{butteFrameworkWaterSecurity2022,eu-citizen.scienceEUCitizenScience,citizenscience.govBasicStepsYour} and others were too broad, to be more than general guidelines.
% FbF lack of information
This lack of information was also present in a lessened form in relation to Drought FbF. In addition to these case study related domains, there are currently further gaps in knowledge when in comes to the application of the \acrshort{fbf} approach on the slow-onset hazard of drought. Generally, the concept of \acrshort{fbf} is now well established in regard to fast-onset disasters, but the drought use case is relatively new (2020) and not yet well researched, which severely limits the amount of guidelines and frameworks available for this particular application (see section \ref{subsec:eap}). Thus the \acrshort{fbf} approach for drought is still in its infancy itself and while it was to be expected that the literature on Somaliland would be limited, that it would be so severely limited was still somewhat surprising (see section \ref{sec:case_area}). The lack of local and directly related information was overcome by transferring the above information sources through interpolation with experiences from preliminary work on other, roughly comparable local projects. While such a transfer does not replace direct local knowledge, it can give a first approximation.\newline
% novelty and breadth
Besides the information availability, the novelty and breadth of the \acrshort{cs} field led to further challenges which emerged in this work in terms of blurred variables and definitional acuity. While many principles, characteristics and guidelines cover a multitude of design variables, \autocite{kirschkeCitizenScienceProjects2022} highlight, that the concrete influence and inter-relations of these has not yet been studied in much detail. This leads to a limited understanding of their influences and impacts for success. This lack of understanding also became apparent in this work and prevented more accurate attribution in the design phase. Though, this might not be as relevant once the project is implemented because then it is then more important that it works and not primarily why.\newline
% positive vs negative constraints
Most of the guidance identified in the literature analysis and later also integrated suggest primarily positive constraints  (see sections \ref{sec:cs} and \ref{sec:ssdr}). The assessment guidelines of the \autocite{ifrcCommunityBasedSurveillanceGuiding2017} were the only identified guideline formulating concrete \textit{red flags}. This was unexpected, as negative constraints can clearly enhance the formulation of conceptual and practical boundaries. These \textit{red flags} were included in the assessment in Stage 2, as they represent a stronger barrier than positive constraints and thus support a more careful assessment. However, \autocite[1]{escaECSACharacteristicsCitizen2020} argues, that because of the various fields of application, disciplinaries and cultural contexts, defining a "universal set of rules for exclusion or inclusion is difficult, and might even limit the advancement of the field". Besides the integration of the \textit{red flags}, this was taken into account by keeping the \acrshort{ssf} relatively general and mainly implementing more precise requirements for the applied case in the \acrshort{prc}.\newline

% ssf
In the following, major design decision for the \acrshort{ssdr} are shortly outlined and reasoned. The decision to build on \autocite{fraislCitizenScienceEnvironmental2022}'s \acrlong{ssf} was primarily driven by its timeliness, comprehensiveness and focus on environmental issues as it was clear, that a more social and local component can be integrated from the \acrshort{srcs}'s experiences with \acrshort{cbs}. The results indicate, that the interpolation of these two approaches was useful, especially in consideration of personal data. While observing natural phenomena at the level of data collection did not raise too many privacy concerns for \autocite{fraislCitizenScienceEnvironmental2022}, this was almost the opposite for CBS \autocite{ifrcCommunityBasedSurveillanceGuiding2017}. Applying these contrasting perspectives to the issue of water sources was thus able to address both the physical and social components well by considering trade-offs between the two 'extremes'. This observation was further supported over the course of this work, when the iterative integration of other guidelines from several divergent foci into the existing framework could be implemented smoothly and only minor revisions had to be made. This goes along \autocite{mcgowanCommunitybasedSurveillanceInfectious2022} findings, that the success factors of \acrshort{cbs} are closely linked to the general principles of participatory community engagement and may therefore be transferred to other participatory surveillance preparedness activities.\newline
% ssf adjustements
In the application of the \acrshort{ssf} as basis for this design and implementation roadmap creation, some adjustments were made. The main overall adaptation were the shortening of the iteration cycle by the first two stages. It became clear, that the exploration and assessment stages do not need to be regularly integrated in the iterative design once the third Stage is reached (see figure \ref{TODO: tha juicy figuuuuure}). Nonetheless, when new fundamental findings or discoveries are made, it may be necessary to partly go back to Stage 1. The same also applies to Stage 2, when one of the defined \textit{red flags} is violated in the further course of the work. Further adaptation were made in the integration of the feasibility assessment and \textit{red flags} of the \acrshort{ifrc} in Stage 2, the integration of the \acrshort{prc} and \acrshort{iwrm} framework in Stage 3 and the focus on iterative improvements in Stage 6. The applicability of these changes could not be evaluated due to time and resource constraints but all adjustments were based on experiences and studies of already conducted or peer-reviewed work and integrated well with the overall framework (see section \ref{sec:ssdr}).\newline

% PRC
The reasons, specifics and their implications of the \acrlong{prc} are discussed in this final part, of the section which addresses the first research question. The development and integration of the \acrshort{prc} attempted to address some of the shortcomings of the process-oriented \acrshort{ssdr}. These shortcomings became apparent right at the beginning of the application in the third stage. It was increasingly difficult to keep an overview of the actual project requirements and their interdependencies in terms of subject matter and temporal constraints (see section \ref{subsubsec:knowledge}). Furthermore, \acrshort{cbs}, \acrshort{cbwm} and other approaches have strongly emphasised the importance of embedding the project into prevailing social and decision-making conditions and procedures, which became apparent to be under-represented in the \acrshort{ssf} (see section \ref{subsubsec:groundwork}). The results of the CBS analysis also highlighted the high time and resource requirements, which were needed for the development and adaptation of methods and techniques to start with the \acrshort{cbs} project in Somaliland. This goes along with \autocite{garciaFindingWhatYou2021}'s findings, that some adjustments and tailoring always need to be done when implementing a new project (see section \ref{subsubsec:innovations}). Together with the emerging need to structure smaller developments and create an overview of decision dependencies, a fourth area of management became apparent that needed to be addressed (see section \ref{subsubsec:management}).\newline
The emphasis of the top most layers, the \textit{Goal-, Products-, and Activities-Layer} (see section \ref{subsec:slmc}) is reasoned by overall time and resource constraints along with the realisation that the latter four layers \textit{methods, techniques, tools and scripts} are too detailed for a relatively general framework. However, it needs to be acknowledged, that the thematic focus of the \acrshort{slmc} is not on \acrshort{cs} and that the goals derived by \autocite{minkmanCitizenScienceWater2015} were primarily focussed on being potential goals of the project itself, and not meant to guide the conceptual phase. Nonetheless, the overall design pattern of the \acrshort{slmc} together with the formulated goals could support the conceptualisation considerably. The close integration of Minkman's goals in this conceptual way may also bring about their 'automatic' consideration during the design, which might lead to a greater breadth of output.\newline
The \acrshort{prc} structure is closer related to the case study area as the SSF but should still be adaptable to other contexts as the high level products mostly relate to general parts of the project and not to concrete techniques or tools. However, the \acrshort{prc} should not be separated from the \acrshort{ssdr} as many products are addressed by activities mentioned in this framework, which are not specifically mentioned again in the \acrshort{prc}. Its final applicability can only be evaluated in practices but generally, formulating project requirements in detail is nothing new and should also benefit a \acrshort{cs} project approach \autocite{wiegersSoftwareRequirements2013,youngEffectiveRequirementsPractices2001,youngProjectRequirementsGuide2006}. Due to the generally iterative nature of this framework, both classical and agile development practices can be applied, with the latter possibly having the advantage \autocite{confortoCanAgileProject2014,ManifestoAgileSoftware}.\newline
% ! hier bin ich noch nicht ganz zufrieden mit.. da geht doch noch was.

Besides the above mentioned challenges in the design phase and the general limitations named in the last section of this chapter, the work will also encounter challenges in the implementation and operation phases. In the (practical) application of this framework, some adjustments will be necessary and as it was the case of this work, time and resource constraints will be imposed by overarching projects or conditions, making some compromises inevitable. This is discussed in the next section in the case of creating a roadmap for implementation in Somaliland.

% Integrating a well thought through \acrshort{iwrm} (see groundwork \ref{subsubsec:groundwork} ) may be a project on its own and data quality, assurance and privacy procedures can be limitless (see section \ref{subsec:stage5_design}) but besides all these design and implementation issues, it should not be forgotten, why a project is started and that the project itself should not be the goal. Therefore, over consideration of issues and best practices may also majorly hinder the implementation and undergo its use. This danger is also mentioned by multiple other projects and studies (sources yo).

% knowledge: A-D more or less generic - though also rather unspecific. Could need to be more detailed.. A has at least many activities
% E-F specific for monitoring stuff (initial and regular) but this could be adapted
% G \& H AA and trigger: FbF context. leave it or take it. Generalisation should easily be possible

% Groundwork
% e.g. groundwork -> focussed on IWRM and day, but really: everything could be included here
% together with community -> realm of local knowledge
% first find the important stuff with com. leaders and key stakeholders and then disseminate --> maybe also in a case study
% giving community so much leverage may not always be wanted by everyone and could be quite tough in the beginning - though good for the embedding

% Policy: these will also limit the freedom of some - that should be accounted for.

% Innovations:
% - new developments may be difficult for a standalone project without scientific accompaniment
% - but adjustments need to be considered. Which are the biggest? what could go wrong? what could be done now, to later integrate improved methods?

% Management:
% - overview about developments/derivations and decisions
% - what else could influence this? also over time and across industries? Who makes decisions about this project in the first place? overlying projects? _> those boundaries should have been addressed in stages 1 and 2 and the beginning of stage 3

% in the end. Really comes done to its execution.








% ggf nur kurz nennen und an genrelles Projektmanagement verweisen

% ! framework

% how did the lit compare to the CBS and others?
% the findings/recommendations of the SRCS were only new in two major aspects
% -> assessment
% -> redflags
% anything else?

% key points: 


was könnte schiefgehen?
über Zeit stabil? --> jein. Mit stage 6 schon.
metaperspektive
konnte selbst unter dem Fokus auf Somalia nicht letztlich zugezurrt werden - viel allgemein gehalten. Einfach ein riesen topic. not surprising aber wichtig.

https://data.jrc.ec.europa.eu/dataset/jrc-citsci-10004
ja well. auch wenige Wasser data 
so far so good. Lot's of it can also be derived by logical thinking, but: special juicy findings!

vor und nachteile dieser 6 stages Aufteilung?
-> iteration was more between the last four. not all 6. no other requirements catalogue could be found in the sources. some lists, some hints, but nothing really structured
-> use my fking systematic thinking man.
complex adaptive systems - critical parts? critical stakeholders? critical stages? could all be different for every project.
funding? may also work or not. 


% ! SSF
open data. Its own huge discussion.
data quality
relate to other studies. Look in the book -> some stuff CBDRR, CBWM and so on..
questions from other studies sparked this? e.g. global inequality of studies



% Overall. This work tries to incorporate up to date and future outlooks of several studies.
“Acknowledging the importance of bidirectionality of information, these systems simultaneously share findings back with the users.” ([McNeil et al., 2022, p. 1](zotero://select/groups/4773535/items/LVLYX8N5)) ([pdf](zotero://open-pdf/groups/4773535/items/4YG35TC6?page=1&annotation=IC6W8R3B))


The first RQ is addressed in this section by discussing the literature and project analysis along with the \acrshort{ssdr} and \acrshort{prc}. The development of a replicable and adaptable framework for community-based participatory water source mapping and monitoring with the aim of facilitating AA in the context of \acrlong{fbf} 


\section{Feasibility Assessment}
%%%%%%%%%%%%%%%%%%%%%%%%%%%%%%%%%%%%%%%%%%%%%%%%%%%%%%%%%%%%%%%%%%%%%%%%%%%%%%%%%%%%%%%%%%%%%
% feasibility assessment + challenges, opportunities and solutions
2. Objective + 2. RQ
%%%%%%%%%%%%%%%%%%%%%%%%%%%%%%%%%%%%%%%%%%%%%%%%%%%%%%%%%%%%%%%%%%%%%%%%%%%%%%%%%%%%%%%%%%%%%
% especially the extensions with red flags was new here
Results indicated that.. the ministry might be.. because of.. based on experiences of.. 
Unexpected findings in this application were the issue about the server location in Ireland, the competition between the NGOs and the MoH's initially negative attitude towards the inclusion of CBS due to oversupply by international NGOs. Furthermore, the highlighted heterogeneity in the community and the stakeholders themselves in regard to the response but also to the implementation of the project. Results indicated, that local stakeholders such as the private berkad owners or private water vendors may not be in favour of this project. The inclusion of financial interests of all parties involved was also mentioned by Minkman. However, it was only mentioned by her and may therefore not be common knowledge. In general, the bias of the more applied guidelines seems to have some dangers, as can be illustrated by this example. %! gerade den bias finde ich wichtig zu erwähnen.

in regard to some details, interview were unexpected. 

Nonetheless, good project management will be required to account for initial adjustments, trade-off considerations and changes over time. 

clear: kein Selbstläufer. Ein großer Haufen an Arbeit.
how influential is the culture?

Stage 6:
output, quality, participant experience and impact
-> evaluation practices may be difficult. Bias towards positive results

privacy concerns for the open data pledge.. not everybody will be a friend..

The buy in off all stakeholders but specifically of the government bodies is often highlighted 
server

many other organisations

contradicting stuff


Since the feasibility had to be determined before this work could move on to address the other research objectives and questions, the second objective to \textit{assess the feasibility of the \acrlong{cs} approach in the given context by identifying potential challenges and opportunities for successful implementation, and to propose recommendations for addressing these challenges} was an interim result of the work. Based on the developed framework in section \ref{subsec:stage2_design} the feasibility was already assessed in sections \ref{subsec:stage1_appl} and \ref{subsec:stage2_appl}. This assessment combined and applied general, international guidance from many projects and studies with local experiences with the \acrshort{cbs} program. It is believed that, even though no dedicated pilot study could be conducted, this combination and interpolation of experiences can reasonably suggest the feasibility of the \acrshort{cs} concept for this application. However, this claim can ultimately only be verified or falsified by a pilot study on site. Furthermore, several challenges such as the embedding into local decision-making and processes, actual tailoring to local conditions and clarifying financial capacities could not further be investigated due to the limited amount of interviews with local stakeholders and ongoing developments of the superordinate project.\newline
Due to the already conducted discussion in section \ref{subsec:stage2_appl} and challenges that cannot be investigated further in this context, the remaining part of this section focusses more on how, why and in what order this assessment was realised as it is believed that this holds more value to the reader than iterating over the discussion again.\newline
Since, to the best of my knowledge, no work has been conducted with the combination of methods, goals and context of this work, there was no concrete existing guidance to assess the feasibility of this approach to achieve the research's aim in the first place. The lack of suitable frameworks for this project made it thus necessary to work on the development of the framework and its application step by step and not only chronologically, at least to some extent. This was facilitated by the iterative working approach, which made it possible to first sketch out possible solutions and then deepen them when the conditions were met accordingly. This was also the case in addressing the second objective and the early conduction of the feasibility assessment is also recommended by multiple other guidelines \autocite{citizenscience.govBasicStepsYour,garciaFindingWhatYou2021,ifrcCommunityBasedSurveillanceGuiding2017,ifrcFbFPractitionersManual2023b,minkmanCitizenScienceWater2015}.\newline
The \acrlong{ssf} and \acrlong{slmc} were adopted at an early stage of the work to have a general direction for the development. To conduct the assessment, the third research objective had to be somewhat anticipated in order to provide an initial framework for the structured feasibility assessment. This framework, now conceptually integrated in the second stage of the design roadmap (see section \ref{subsec:stage2_design}) was in the beginning primarily a combination of the \acrshort{ssf}'s second stage and the feasibility assessment of the \acrshort{cbs} of the \acrshort{ifrc}. The final feasibility assessment took place on the current basis, which was further underpinned with some additional guidelines, best practices and knowledge of the interviewees over the course of multiple iterations.\newline
When designing a framework for or directly assessing the feasibility of \acrshort{cs}, it becomes clear that \textit{feasibility} depends on a variety of factors, but also that there are no clear rules that must be followed. Each \acrshort{cs} project is somewhat special and the flexible concept also allows for several adaptations (see section \ref{sec:cs}). Therefore, the feasibility is not assessed by a specific set of rules, but rather how well it relates to general principles and factors of success. This makes sense in the way, that what specifically works in e.g. \autocite{minkmanCitizenScienceWater2015}'s approach in the Netherlands may not be feasible in Somaliland, e.g. the use of smartphone sensors as the rural population in Somaliland has few smartphones and internet coverage is poor. Assessing challenges and opportunities is thus a highly specific and local task and depends on many factors.\newline
Nonetheless, the \acrshort{ecsa} along with many other associations and studies developed \acrshort{cs} principles and characteristics that support the successful design, implementation and operation of a \acrshort{cs} project. Furthermore, a \acrshort{cbs} project was already successfully implemented and in operation for several years within the context and the \acrshort{srcs} but focused on a different topic. This, again highlights the thorough analysis of local comparable projects, mentioned in stage 1, section \ref{subsec:stage1_design}. The actual feasibility assessment therefore focussed primarily on the differences between the \acrshort{cbs} and the potential water source mapping and monitoring project.\newline

%%%%%%%%%%%%%%%%%%%%%%%%%%%%%%%%%%%%%%%%%%%%%%%%%%%%%%%%%%%%%%%%%%%%%%%%%%%%%%%%%%%%%%%%%%%%%%%%%%%
% specific application of the framework to the context
4. Objective + 4. RQ
%%%%%%%%%%%%%%%%%%%%%%%%%%%%%%%%%%%%%%%%%%%%%%%%%%%%%%%%%%%%%%%%%%%%%%%%%%%%%%%%%%%%%%%%%%%%%%%%%%%
\section{Application}


any surprising findings? 
--> e.g. there might be people, that have an interest in this project not working. Private vendors of water trucking e.g.? what about those.
more power to the people -> less power to others. Problems?

interesting: 

server location in Ireland

publication of the water source locations -> in regard to the history. it is, in the end a critical infrastructure

“This landscape demonstrated the breadth of applicability of participatory surveillance, from tick identification in photographs, to One Health apps used by community members, to trained volunteers reporting invasive plant pests,” ([McNeil et al., 2022, p. 8](zotero://select/groups/4773535/items/LVLYX8N5)) ([pdf](zotero://open-pdf/groups/4773535/items/4YG35TC6?page=8&annotation=AQ46ELFM))



The fourth research objective to \textit{apply the adapted and developed framework to establish a roadmap for the implementation of a water source mapping and monitoring approach to trigger appropriate anticipatory actions to address water shortages} has already been partially addressed under the first two objectives. Stage 1 was primarily addressed by the first research objective which explored much of the underlying concepts, context, prevailing problems and also carved out potential solutions. Stage 2, the feasibility assessment, was addressed with the second research objective. Therefore, this section concentrates on the stages three to six.\newline
In stage 3, the subdivision into the \acrshort{prc} was helpful to reduce cognitive overload and highlight chronological and thematic (inter-) dependencies. In terms of knowledge, the \acrshort{prc} helped to structure the identified information from stages 1 and 2, which additionally helped to make knowledge gaps, such as e.g. missing detailed knowledge about local decision-making procedures, obvious. As it is was not feasible to gather these information in the scope of this work, it was therefore simplified to concentrate on those areas, that could be tackled. For example, it became clear that the initial mapping, which includes gathering other key information about the berkad, cannot be done by local volunteers as the knowledge and technical equipment requirements are too high for most. Thus the initial mapping needs to be conducted by the \acrshort{srcs} professionals who are already experienced and don't need further guidance for the process. Nonetheless, gathering the information that is initially required was feasible in the context of the work and thus focussed on. The knowledge gathered was thus more broad than deep and in most cases requires further investigation, especially in relation to local conditions. For example, in the case of the \acrshort{aa} of water trucking, see figure \ref{TODO:} in section \ref{subsubsec:assemblage}, requirements could be listed, but their actual specification is only possible on the ground with local stakeholders.\newline
Close and early cooperation with other local actors in the area of embedding the project in conceptual water management practice was also suggested as important by the interviewees (see section \ref{subsubsec:groundwork_appl}). This compares well with common recommendations for \acrshort{cs} projects (see section \ref{sec:cs}). To facilitate this, a light \acrshort{iwrm} framework which was also already tested in other local circumstances could be identified in \autocite{dayCommunitybasedWaterResources2009}'s adaptation. Yet, the same limitations apply here, as the actual feasibility can only assessed on the ground. Nonetheless, the willingness and experience of local managers to implement those concepts could be identified, which suggests at least a good initial situation for the successful embedding of the project into local management practices. The great importance of deep local embedding is also highlighted by \autocite{gualazziniEWEAEarlyWarning2021} because even if the information gathered is good and timely, it still needs to be incorporated in decision-making and acted upon.\newline
Besides the conceptual groundwork directly on site, innovations for the determination and collection of water level thresholds are required. The gathered information suggest, that there are two potential ways to assess the water level (see section \ref{subsubsec:innovations_appl}). The technical measuring and transmission of the actual water height would require the knowledge about the exact capacity and size of the berkad to assess the water level. Although this method would provide a more objective measurement, local knowledge of the potential duration of water supply was also found to be good with a Berkad (see section \ref{subsubsec:assemblage}). Both approaches do not contradict each other and could also be used together. This would also allow a good basis over time for evaluating the quality of the assessment of local which could what could then improve local water management. In addition to the quantity of water, its quality was also considered very important, but no locally feasible approach to assessing quality could be identified. This supports the importance of providing a sound knowledge foundation about contamination prevention and water management practices to the community. This is also supported by several other studies \autocite{danielAssessingDrinkingWater2020,huangManagementDrinkingWater2020,tariqOpenSourceWater2021,wmoPlanningWaterqualityMonitoring2013}. Research in this field is still ongoing \autocite{tariqOpenSourceWater2021} and \autocite{delaireHowMuchWill2017} cost estimations suggest, that even with current equipment, costs are minimal in relation to achieving the SDG 6.1 of safe water for all.\newline
While no management decision or concrete developments could be made in the scope of this work, the results suggest some additional considerations. In the case of deciding for a specific water source monitoring strategy, all accessible water sources in a community should be monitored, as the largest, e.g. a ballay, is not necessarily the one that can withstand a period of drought the longest. Physical as well as social access factors need to be considered in terms of actual water withdrawal and monitoring when deciding on the actual monitoring routine see section \ref{subsubsec:assemblage}. Furthermore, the results support the feasibility and usefulness of a staggered trigger as proposed by \autocite{rcrcFORECASTBASEDFINANCINGEARLY2020} for triggering on water level thresholds as both, a seasonal and a short-term assessment are possible. In terms of \acrshortpl{aa}, the results mostly supported the feasibility of water trucking and cash transfer \acrshortpl{aa} (see section \ref{subsec:case_eap} and \ref{subsubsec:assemblage}), which compares well with \autocite{gettliffeOCHAAnticipatoryAction2021} findings. Yet, when comparing this finding with the statements of the interviewees from section \ref{subsec:stage1_appl} that water is often over prized in times of scarcity and with the statement of \autocite{ochaANTICIPATORYACTIONPLAN2020} that markets need to be operational to permit this handling of the demand, distributions of cash or water vouchers may not always be feasible as \acrshort{aa}. Besides the social factors \autocite{birchSomalilandSomaliRegion2008} further highlights that impact on natural ecosystems through water availability for animals need to be considered when addressing water shortages. 
% “Additionally, there may be limitations related to designing and implementing citizen science projects in remote and unsafe areas, where crime levels are high and political risks exist, or where mobile network coverage is poor, access to smartphones and electricity is low and illiteracy levels among participants are high. Co-design and community-based approaches can address such challenges and ensure a high level of participant engagement20” ([Fraisl et al., 2022, p. 14](zotero://select/groups/4773535/items/FBJD7SWS)) ([pdf](zotero://open-pdf/groups/4773535/items/7WBDKYDY?page=14&annotation=MZ95QT2P))
The community building aspect in stage 4 was mainly focussed on assessing the capacities of the \acrshort{srcs}. The findings suggested good capacities and high experiences in the area of community engagement as well as volunteer training and supervision (see section \ref{subsec:stage4_appl} and \ref{subsec:stage2_appl}). This was to be expected, as the \acrshort{srcs} already implemented a comparable project and was also found to be performing well within the framework of the overarching \acrshort{fbf} project. Findings of Stage 5 \textit{Data management} support the technical practicability of the project. The solution of the implemented \acrshort{cbs} project with the NYSS platform was found to be very dedicated and technically adaptable to the new requirements. Currently, discussions about the adaptation are ongoing on management level. In addition to this very automated and already well-rehearsed system, however, several other potential systems could be identified which, although not quite as advanced or tailored, would also suffice for the application for the time being (see section \ref{subsec:stage5_appl}and \ref{subsec:mcs}).\newline
A major concern of data collected by \acrshort{cs} is their quality and accuracy. This is well addressed in the current \acrshort{cbs} program with initial and refresher trainings, close supervision and further verifications when necessary. Together with the built-in automatic checks in NYSS, measures of \acrshort{qc} and \acrshort{qa} are well considered. It is expected that these mechanisms can be translated to the new project largely addressing quality concerns right from the start. Constructed validity needs to be addressed by further data triangulation, as the simple measurement of the water level itself does not proof, that the underlying reason for low water levels is indeed a drought but could also be reasoned by social factors (see section \ref{subsec:water_scarcity}). Evaluation practices are integrated in every stage but due to no actual implementation and operation, evaluation could not be conducted. Nonetheless, the design process, due to its iterative processes underwent several evaluations and piecemeal improvements itself suggesting good adaptability and upgradeability.\newline
The overall application of the developed framework worked well and the combination of the \acrshort{ssf} together with the \acrshort{prc} based on the \acrshort{slmc} could provide good guidance while also remaining flexible to incorporate new, unexpected findings. Yet, the entire power of the \acrshort{slmc} could not be exploited as the coming layers were too detailed and most of those need to be determined in closer collaboration with the team and the local stakeholders. Nonetheless, based on the positive experiences with the first three stages, it is believed that the following layers will also prove fruitful to potential future developments.\newline
The application also supports the findings of \autocite{garciaFindingWhatYou2021} and \autocite{conradCommunityBasedMonitoringFrameworks2008} that a framework should be used but also highlights the need to adapt this framework the actual projects conditions and goal. Furthermore, the work suggests that a respective implementation of a \acrlong{cs} project is not only theoretically feasible but practically implementable. While an end to end establishment of an implementation roadmap was not feasible in the context of this work due to several information, resource and time constraints, a sound foundation could be laid for further practical implementation in the scope of a pilot study. Yet, important questions need to be answered on management level and not all indicators are in favour of a practical application. For example, the \acrshort{rcrc} is generally not recommending its National Societies to implement their own data gathering strategies as this would, under normal circumstances over-burden and exceed costs \autocite{rcrcFORECASTBASEDFINANCINGEARLY2020}. Therefore, \autocite{rcrcFORECASTBASEDFINANCINGEARLY2020} generally recommends to found the triggers and information on already gathered information by other stakeholders or international organisations. As suggested by the results, this is not feasible in this context. Furthermore, the analysis of the \acrshort{cbs} approach and other projects together with the feasibility assessment and conducted application of this work suggest that \acrshort{cs} can be reasonable and cost-effective approach to gather relevant information. This is also supported by \autocite{aceves-buenoCitizenScienceApproach2015} and \autocite{minkmanCitizenScienceWater2015} findings. Timely and accurate data can support the appropriate \acrlongpl{aa} to be tailored, making mitigation and response potentially more streamlined, efficient and effective.\newline

% % alternative designs
% % I guess better leave this out and put it in the discussion section (?) 
% “Using Remote Sensing to Map and Monitor Water Resources in Arid and Semiarid Regions” ([Klemas and Pieterse, 2015, p. 33](zotero://select/groups/4773535/items/BVN6IXG5)) ([pdf](zotero://open-pdf/groups/4773535/items/UPSYZXDK?page=1&annotation=4DPZD4BZ))
% % Establishing an operational waterhole monitoring system using satellite data and hydrologic modelling: Application in the pastoral regions of East Africa
% https://earlywarning.usgs.gov/docs/Senay-et-al-Pastoralism-Research-Policy-and-Practice-2013.pdf
% “Global Monitoring of Water Supply and Sanitation: History, Methods and Future Challenges” ([Bartram et al., 2014, p. 8137](zotero://select/groups/4773535/items/6AWUJTW5)) ([pdf](zotero://open-pdf/groups/4773535/items/BFNSQGWS?page=1&annotation=ZWSBJVDM))
% “Remote monitoring of rural water systems: A pathway to improved performance and sustainability?” ([Thomson, 2021, p. 1](zotero://select/groups/4773535/items/UQLXVVYI)) ([pdf](zotero://open-pdf/groups/4773535/items/K9XBXPQD?page=1&annotation=B59B5U68))

%% --> not gonna happen but still interesting

%----------------------------------------------------------------------------------------
% Step 4: Acknowledge the limitations of the study
%----------------------------------------------------------------------------------------
% - can cover any part of the study (theory up to methods or sample)
% e.g. small sample -> no generalisation possible nono
% + possible improvements, but don't undermine the research

Besides the high demands and complexity of the \acrshort{cs} project design itself, this process also highlighted, that \acrshort{cs} is not a silver bullet itself but comes with various advantages and disadvantages. While the final application may seem appealing, as a lot of the work is done by the contributing citizens, particularly the design and implementation poses a lot of requirements in time, skill, and resources. This complexity is also mentioned by \autocite{fraislCitizenScienceEnvironmental2022} and \autocite{minkmanCitizenScienceWater2015} highlighting the statement, that Citizen Science one of many methods and its used should be well considered.



\section{Limitations}
% theoretical foundation
In any research project, limitations are an important aspect to consider and some were already addressed in the above discussion. Yet, there are further limitations that need to be acknowledged. The literature review did not follow a strict formal structure and that comparable projects may have been overlooked, although unlikely, cannot be ruled out. Nonetheless, this exploratory approach also allowed for the discovery of many, formerly unknown aspects and contributed many insights to the study. The subsequent in depth literature review, although not formal, was detailed and extended and was able to identify and close some gaps. However, the generally sparse literature on Somaliland limited the desk-based collection of information about local conditions. Furthermore, this work has also not addressed the integration or application of local and/or indigenous knowledge or the further use of VGI, to the extent that this would have been possible in principle. Although both areas are very interesting, this was either not the focus of this work or, the case of available water source datasets and VGI an in-depth analysis had been deemed unsuitable due to the poor quality identified early on. Nevertheless, insights could be gained from the data using more refined methods in future work. On the theoretical level of the contextual basis, the concepts mentioned, such as water security, drought or Citizen Science, are extremely complex and highly debated topics. Discussing them in detail would have exceeded the scope of this thesis, which is why focal points were set according to the priority of this work.\newline
% exploratory approach
The inclusive nature of the exploratory approach, was thus tried to be addressed by information triangulation from other studies but it made for a generally more consensual work and fewer contradictory findings. This, together with the inclusion of most relatable studies and projects into the framework itself, also made for a relatively homogeneous discussion due to the lack of contradictory findings. It is expected that this comparison with other work will be possible in the future as more CS projects are carried out in a similar context and with similar objectives.\newline
% case study
The critique of the case study research type, its challenge to execute, significant documentation efforts and complex nature also had certain shares in this work. Nevertheless, its strength of being rich, detailed and contextual also contributed positively. The generalisability of the understanding gained through actual application can be considered low, but the applicability of the framework developed can be expected to be transferable to other comparable contexts.\newline
% validities and biases
An attempt was made to improve internal validity through the iterative design, reciprocal reviews and triangulation of multiple sources of information, but despite great efforts, it is hardly possible to establish causal relationships in such a complex environment. Constructed validity of the framework is believed to be reasonable due the extensive triangulation of resources but can only be tested in a practical examination. The importance of data triangulation was also noted and integrated in the actual application. Interviews always add a human factor which can complicate repeatability but clear procedures and documentation were established to account for this as best as possible. The expert and snowball sampling strategy itself worked well, but was severely limited by other factors. The interviews always had to be arranged and signed off by senior managers, and the already tense situation with ongoing response activities and parallel \acrshort{fbf} development in Somaliland made the availability of interviewees even scarcer. This resulted in a relatively low sample of interviewees, limiting the otherwise strength of a case study to incorporate a multitude of perspectives on one area of interest. Here, interviews especially with representatives of \acrshort{nadfor}, the \acrshort{mowr} and \acrshort{moh} as well as \acrshort{brcis} and \acrshort{ocha} mights have been fruitful. The conversion of one interview into a questionnaire hindered direct clarification and follow-up questions in the interview, but this could be compensated for by a second questionnaire. Since a lot of information could be drawn from the answers and the interview could not have taken place otherwise, this can ultimately be seen as a good compromise.\newline
% overall
Overall, the study was conducted under difficult conditions in a case study area known for its complex and difficult environment. The generally limited time and resources available in the context of a Master's thesis further constrained the study and the focus on just the development of the framework might have been beneficial. Nevertheless, it is believed that the study conducted was ultimately able to provide a good theoretical and structural basis for a potential practical implementation of this approach. Furthermore, this study contributes to the general ongoing discourse of \acrlong{cs}-projects by adding a work from a currently underrepresented region. It is to be expected that in the further process of this discourse many of the limitations mentioned here can be addressed and overcome.

% 
% “Our results indicate that using the phones to transmit more than just water quality data will likely improve the effectiveness and sustainability of this type of intervention.” ([Kumpel et al., 2015, p. 10846](zotero://select/groups/4773535/items/GPM4C7RJ)) ([pdf](zotero://open-pdf/groups/4773535/items/7VXVKEXK?page=1&annotation=4DJIADX2))

% keeping data up to date is crucial in ensuring correct vulnerability and exposure data 
% “Vulnerability and exposure changes over time, particularly after an extreme weather or climate event. Datasets must be kept up to date to ensure the impact-based forecast or warning using this data is reliable. Recognise that many official governmental data sources, such as a national census or demographic and health surveys, are updated infrequently – every five or ten years.” ([Harrowsmith et al., 2020, p. 28](zotero://select/groups/4773535/items/QJ397Y54)) ([pdf](zotero://open-pdf/groups/4773535/items/2GS362N5?page=28&annotation=5XVAQCTY)
% 

% recommended to combine the BRCiS appraoch with I1 categorization. one more extensive long-term assessment at the end of the rain season together with the other rarely changing indicators (e.g. number of animals and people) but also combined with an triangulation with stakeholder and thorough analysis and interpretation of data --> integration into decision-making processes. --> seasonal warning and identification of vulnerability + short-term facilitation of immediate action with the short-term water level warnings

% -	Water levels in berkeds could be a good indicator, however it cannot be a stand alone indicator. This has to be combined by meteorological forecasts and local knowledge as well.
% - Regarding Anticipatory actions, there has been any actions yet due to the fact that there is no water monitoring and trigger mechanism in place.
% Beyond data collection, \autocite{gualazziniEWEAEarlyWarning2021} highlights the possibility to adopt a two-way communication capability to also distribute climate and weather forecasts to participants. Thus "communities have direct access to short-term weather forecasts and climate risk alerts to mitigate risk and protect livelihoods, while also providing information from the site" \autocite[20]{gualazziniEWEAEarlyWarning2021}. % ITIKI may also contribute to this statement

% “Current weather forecasts are still alien to African farmers, most of whom live in rural areas and struggle with illiteracy and poor communications infrastructure” ([Masinde and Bagula, 2012, p. 274](zotero://select/groups/4773535/items/EW9XSSZP)) ([pdf](zotero://open-pdf/groups/4773535/items/3WQ4S9PE?page=1&annotation=HPZC9Z65))

% 
% The combination of both mechanisms (short and seasonal) might be the best choice for the creation of a staggered trigger.



% further information about the water source - e.g. accessibility could be surveyed and potential integrations via codes need to be delineated

% regular reporting with event based reporting as water in Berkads can e.g. dry up quicker than predicted or turn bad or what ever else..



% % discuss these other research fields and how it could relate to it.. could water level also be a good proxy for other stuff?
% “but questions related to coping capacities, migration, poverty, water supply, access to food and markets, or political conflict remain unanswered or are even decoupled from routine drought risk assessments” ([Enenkel et al., 2020, p. 1162](zotero://select/groups/4773535/items/RX575C79)) ([pdf](zotero://open-pdf/groups/4773535/items/XD499UNK?page=2&annotation=HE48ZWFA))






