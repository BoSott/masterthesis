% Chapter Template

\chapter{Discussion and Reasoning} % Main chapter title

\label{chapter5} % Change X to a consecutive number; for referencing this chapter elsewhere, use \ref{ChapterX}
% Interpretation of the results that go further than the research questions. This can, e.g., contain implications for software development.


% (in the discussion)
% -> Reiterate the Research Problem/State the Major Findings
% -> Explain the Meaning of the Findings and Why They are Important
% -> Relate the Findings to Similar Studies
% -> Consider Alternative Explanations of the Findings
% -> Acknowledge the Study's Limitations
% evaluate and interpret results
% -> focus on stuff that is directly related to the research aim
% don't report ANY new findings

%----------------------------------------------------------------------------------------
% Step 1: restate your research problem, aim, hypotheses and questions
%----------------------------------------------------------------------------------------
% introduction + overall findings/aim

% method section --> use of frameworks and literature to create a framework + prc
% pros and cons of ssf and slmc + why not others and how it relates to those
% e.g. several other .. have also a staged, iterative process design (.gov, butte)
% prc --> why and how it worked together with the ssf - how it compares to others (does it?)
% + a couple of special cases or highlights e.g. areas with no or only a few indicators/data points that might have been interesting
% e.g. discuss specific products or activities that highlight something + here and there.. 
% quality of the impacts --> risk assessment and correlation --> overlying EAP development (highlight this)
% -> e.g. clear and simple guidelines may be inherently unlikely due to the underlying complexity of each case study and project
% no directly comparable projects -> comparability lacks. Compares well with cbs etc. but it is also build on those recommendations --> name this!
% are there trade-offs?
% e.g. what the focus is comes with inherent drawbacks in other areas
% synergies?
% could add more and more and more but: would loose clarity and focus -> only included what is really necessary and useful
% -> the decomposition of complex situations into specific categories naturally entails limitations -> particularly true for interrelated and influences over time

% "However, the reduction to essential parts also allows clarity and focus that would otherwise not be possible. An attempt was made to name the various advantages and disadvantages and to show that the methodology applied is appropriate in this con-
% text." myself, BA

% application --> case study also used from others for similar attempts
% basically worked well - inherent limitations -> just a research method with cons and pros
% reason why not include the mapping and what could be done about it.

% The augmentation of both the theoretical and empirical underpinnings of this endeavour has served to enhance its overall depth and rigor.

% “Experts in trigger methodology have indicated a more appropriate strategy may be to build on tools that currently exist at the government level such as national drought monitoring systems. As such, the ideal is an iterative process with the ground level along with a technology push that creates new ways to analyse drought and drought risk.” ([RCRC, 2020, p. 28](zotero://select/groups/4773535/items/UESIQTRJ)) ([pdf](zotero://open-pdf/groups/4773535/items/P5JPVZ97?page=28&annotation=977VS8FC))

%! am Ende noch überarbeiten
This study aimed to design and test an approach for community-based participatory mapping and monitoring of water sources in a water-scarce and resource-limited setting in collaboration with the \acrlong{srcs}. The ultimate goal was to facilitate respective \acrlongpl{aa} in the context of \acrlong{fbf} and to improve water management and accessibility in underserved communities. To achieve this aim, four research objectives were formulated, including a comprehensive literature review to identify and evaluate principles for community-based participatory mapping and monitoring, assessing the feasibility of the approach in the given context, developing a replicable and adaptable framework based on the identified guidelines, and applying the framework to create a roadmap for implementation.\newline

The literature and data analysis revealed the high complexity of the context and could determine gaps in the data situation on water sources as well as the project and framework landscape in regard to \acrlong{cs} approaches in the given context for the implementation in a \acrshort{fbf} project. However, the general feasibility of the approach for the project was suggested through further analysis. Building on this positive assessment, the identified frameworks and guidelines were adapted and expanded to ultimately lead to the development of a new replicable and adaptable framework for a community-based participatory water source mapping and monitoring in the context of \acrlong{fbf}. Its application on this specific case area resulted in a roadmap for the practical implementation of the project. This roadmap includes goals and sub-goals, required products and respective activities.\newline

In this discussion chapter, the focus is on reflecting on the main findings and contributions of this study and discuss their implications for further developments and practical applications. In detail, each research objective is addressed in turn and its relevance to the research aim is discussed. Finally, limitations and challenges encountered during the research process are named and considered. 
% The successful application of the developed framework indicates its usability and usefulness.
%----------------------------------------------------------------------------------------
% Step 2: summarise your key findings
%----------------------------------------------------------------------------------------
% themes and relationships (qualitative) and correlations and causality (quantitative)
% -> highlight overall key findings
% -> one or two paragraphs -> be concise

% e.g.
% The data suggest that…
% The data support/oppose the theory that…
% The analysis identifies…

%----------------------------------------------------------------------------------------
% Step 3: interpret your results
%----------------------------------------------------------------------------------------
% unpack the findings (no new information!)
% ,“Climate information presented as early warnings are only as valuable as the actions that are taken in response to the information, even if the information is a perfect warning of future events.” ([Mariani et al., 2015, p. 8](zotero://select/groups/4773535/items/8THVVJVK)) ([pdf](zotero://open-pdf/groups/4773535/items/GYUFNK32?page=8&annotation=Y9DG5FSE))

% -> follow a similar structure as in the result chapter
% or research questions/hypotheses
% or theoretical framework

% how does those results compare to existing research ?! -> lit review

% contrasts are often the most interesting findings --> why? significant?

% - How do your results relate with those of previous studies?
% - If you get results that differ from those of previous studies, why may this be the case?
% - What do your results contribute to your field of research?
% - What other explanations could there be for your findings?

% don't draw conclusions, that aren't substantiated -> everything need to be backed up by something

%?%%%%%%%%%%%%%%%%%%%%%%%%%%%%%%%%%%%%%%%%%%%%%%%%%%%%%%%%%%%%%%%%%%%%%%%%%%%%%%%%%%%%%%%%%%%%%%%%%%%
%  How can a replicable and adaptable framework for community-based participatory water source mapping and monitoring with the aim of facilitating AA in the context of \acrlong{fbf} be developed?
% 1. RQ
%?%%%%%%%%%%%%%%%%%%%%%%%%%%%%%%%%%%%%%%%%%%%%%%%%%%%%%%%%%%%%%%%%%%%%%%%%%%%%%%%%%%%%%%%%%%%%%%%%%%%

\section{A Replicable and Adaptable Framework}

\begin{quote}
    "All models are wrong, but some are useful" George E.P. Box
\end{quote}

While this statement was made in regard to statistical models, the consideration of the trade-off between generalisation and specialisation is also crucial in the design of frameworks. Highly general principles and characteristics up to highly specialised projects can be found in the literature (see section \ref{sec:cs}). The development of the \acrshort{ssdr} has tried to find a balance between the focus on drought, FbF and citizen involvement in Somaliland while also staying adaptable to other, yet comparable projects.\newline
% surprising
Surprising was, that while a manifold of general guidelines, characteristics and quality criteria for \acrlong{cs} projects exist, no grouped and ordered requirements list along potential \acrshort{cs} goals could be found. While this is unexpected, as it is no radically new insight, but merely a different framing of more or less the same information, it could be explained by the limited time of practitioners to publish concrete information. This lack of time for publication was also mentioned by I2 but no peer-reviewed study could be found to either underline or falsify this assumption. However, an interweaving of the more often encountered process-oriented approach with a specific, yet adaptable requirements catalogue was found to be manageable and, as also later discussed in more detail, well applicable.\newline
% chapter structure
In the following, the general development of the \acrshort{ssdr} is discussed and the challenges encountered and potential solutions are considered in more detail, looking first at the \acrshort{ssf} and then at the \acrshort{prc}. When considering the \acrshort{ssdr} and \acrshort{prc} frameworks, it is crucial to acknowledge that they provide only limited perspectives on the complex reality of design processes. This research primarily adopted a process- and requirements-oriented approach in designing and conceptualising the design roadmap. Other perspectives, such as resource, behavioural network or stakeholder networks, cultural norms and values, as well as the communication network perspective may play a role in certain aspects, but are of secondary importance in this work.\newline
% challenges information availability
Challenges in conceptualising the new framework primarily laid in information availability and transferability. Several \acrshort{cbm}, \acrshort{mcs}, \acrshort{cbs}, \acrshort{cbwm} and other risk related \acrshort{cs} frameworks and respective guidelines could be identified but none of them exactly matched the intended application (see section \ref{sec:cs}). While "there is no one-size-fits-all approach" \autocite[2]{fraislCitizenScienceEnvironmental2022}, the existing frameworks either focussed on different thematics, contexts, had different participation levels, different goals or a combination of the above (see sections \ref{subsubsec:cbwm}, \ref{subsubsec:cbs} and \ref{subsec:cbc}). This is consistent with \autocite{butteFrameworkWaterSecurity2022}'s and \autocite{carrionCROWDSOURCINGWATERQUALITY2020}'s findings that existing frameworks guiding the development of water security data collection projects are often very specific and limited to certain factors, in many cases also not taking socio-economic factors into account. At the same time, frameworks like the on from \autocite{butteFrameworkWaterSecurity2022,eu-citizen.scienceEUCitizenScience,citizenscience.govBasicStepsYour} and others were too broad, to be more than general guidelines.
% FbF lack of information
This lack of information was also present in a lessened form in relation to Drought FbF. In addition to these case study related domains, there are currently further gaps in knowledge when in comes to the application of the \acrshort{fbf} approach on the slow-onset hazard of drought. Generally, the concept of \acrshort{fbf} is now well established in regard to fast-onset disasters, but the drought use case is relatively new (2020) and not yet well researched, which severely limits the amount of guidelines and frameworks available for this particular application (see section \ref{subsec:eap}). Thus the \acrshort{fbf} approach for drought is still in its infancy itself and while it was to be expected that the literature on Somaliland would be limited, that it would be so severely limited was still somewhat surprising (see section \ref{sec:case_area}). The lack of local and directly related information was overcome by transferring the above information sources through interpolation with experiences from preliminary work on other, roughly comparable local projects. While such a transfer does not replace direct local knowledge, it can give a first approximation.\newline
% novelty and breadth
Besides the information availability, the novelty and breadth of the \acrshort{cs} field led to further challenges which emerged in this work in terms of blurred variables and definitional acuity. While many principles, characteristics and guidelines cover a multitude of design variables, \autocite{kirschkeCitizenScienceProjects2022} highlight, that the concrete influence and inter-relations of these has not yet been studied in much detail. This leads to a limited understanding of their influences and impacts for success. This lack of understanding also became apparent in this work and prevented more accurate attribution in the design phase. Though, this might not be as relevant once the project is implemented because then it is then more important that it works and not primarily why.\newline
% positive vs negative constraints
Most of the guidance identified in the literature analysis and later also integrated suggest primarily positive constraints  (see sections \ref{sec:cs} and \ref{sec:ssdr}). The assessment guidelines of the \autocite{ifrcCommunityBasedSurveillanceGuiding2017} were the only identified guideline formulating concrete \textit{red flags}. This was unexpected, as negative constraints can clearly enhance the formulation of conceptual and practical boundaries. These \textit{red flags} were included in the assessment in Stage 2, as they represent a stronger barrier than positive constraints and thus support a more careful assessment. However, \autocite[1]{escaECSACharacteristicsCitizen2020} argues, that because of the various fields of application, disciplinaries and cultural contexts, defining a "universal set of rules for exclusion or inclusion is difficult, and might even limit the advancement of the field". Besides the integration of the \textit{red flags}, this was taken into account by keeping the \acrshort{ssf} relatively general and mainly implementing more precise requirements for the applied case in the \acrshort{prc}.\newline

% ssf
In the following, major design decision for the \acrshort{ssdr} are shortly outlined and reasoned. The decision to build on \autocite{fraislCitizenScienceEnvironmental2022}'s \acrlong{ssf} was primarily driven by its timeliness, comprehensiveness and focus on environmental issues as it was clear, that a more social and local component can be integrated from the \acrshort{srcs}'s experiences with \acrshort{cbs}. The results indicate, that the interpolation of these two approaches was useful, especially in consideration of personal data. While observing natural phenomena at the level of data collection did not raise too many privacy concerns for \autocite{fraislCitizenScienceEnvironmental2022}, this was almost the opposite for CBS \autocite{ifrcCommunityBasedSurveillanceGuiding2017}. Applying these contrasting perspectives to the issue of water sources was thus able to address both the physical and social components well by considering trade-offs between the two 'extremes'. This observation was further supported over the course of this work, when the iterative integration of other guidelines from several divergent foci into the existing framework could be implemented smoothly and only minor revisions had to be made. This goes along \autocite{mcgowanCommunitybasedSurveillanceInfectious2022} findings, that the success factors of \acrshort{cbs} are closely linked to the general principles of participatory community engagement and may therefore be transferred to other participatory surveillance preparedness activities.\newline
% ssf adjustements
In the application of the \acrshort{ssf} as basis for this design and implementation roadmap creation, some adjustments were made. The main overall adaptation were the shortening of the iteration cycle by the first two stages. It became clear, that the exploration and assessment stages do not need to be regularly integrated in the iterative design once the third Stage is reached (see figure \ref{TODO: tha juicy figuuuuure}). Nonetheless, when new fundamental findings or discoveries are made, it may be necessary to partly go back to Stage 1. The same also applies to Stage 2, when one of the defined \textit{red flags} is violated in the further course of the work. Further adaptation were made in the integration of the feasibility assessment and \textit{red flags} of the \acrshort{ifrc} in Stage 2, the integration of the \acrshort{prc} and \acrshort{iwrm} framework in Stage 3 and the focus on iterative improvements in Stage 6. The applicability of these changes could not be evaluated due to time and resource constraints but all adjustments were based on experiences and studies of already conducted or peer-reviewed work and integrated well with the overall framework (see section \ref{sec:ssdr}).\newline

% PRC
The reasons, specifics and their implications of the \acrlong{prc} are discussed in this final part, of the section which addresses the first research question. The development and integration of the \acrshort{prc} attempted to address some of the shortcomings of the process-oriented \acrshort{ssdr}. These shortcomings became apparent right at the beginning of the application in the third stage. It was increasingly difficult to keep an overview of the actual project requirements and their interdependencies in terms of subject matter and temporal constraints (see section \ref{subsubsec:knowledge}). Furthermore, \acrshort{cbs}, \acrshort{cbwm} and other approaches have strongly emphasised the importance of embedding the project into prevailing social and decision-making conditions and procedures, which became apparent to be under-represented in the \acrshort{ssf} (see section \ref{subsubsec:groundwork}). The results of the CBS analysis also highlighted the high time and resource requirements, which were needed for the development and adaptation of methods and techniques to start with the \acrshort{cbs} project in Somaliland. This goes along with \autocite{garciaFindingWhatYou2021}'s findings, that some adjustments and tailoring always need to be done when implementing a new project (see section \ref{subsubsec:innovations}). Together with the emerging need to structure smaller developments and create an overview of decision dependencies, a fourth area of management became apparent that needed to be addressed (see section \ref{subsubsec:management}).\newline
The emphasis of the top most layers, the \textit{Goal-, Products-, and Activities-Layer} (see section \ref{subsec:slmc}) is reasoned by overall time and resource constraints along with the realisation that the latter four layers \textit{methods, techniques, tools and scripts} are too detailed for a relatively general framework. However, it needs to be acknowledged, that the thematic focus of the \acrshort{slmc} is not on \acrshort{cs} and that the goals derived by \autocite{minkmanCitizenScienceWater2015} were primarily focussed on being potential goals of the project itself, and not meant to guide the conceptual phase. Nonetheless, the overall design pattern of the \acrshort{slmc} together with the formulated goals could support the conceptualisation considerably. The close integration of Minkman's goals in this conceptual way may also bring about their 'automatic' consideration during the design, which might lead to a greater breadth of output.\newline
The \acrshort{prc} structure is closer related to the case study area as the SSF but should still be adaptable to other contexts as the high level products mostly relate to general parts of the project and not to concrete techniques or tools. However, the \acrshort{prc} should not be separated from the \acrshort{ssdr} as many products are addressed by activities mentioned in this framework, which are not specifically mentioned again in the \acrshort{prc}. Its final applicability can only be evaluated in practices but generally, formulating project requirements in detail is nothing new and should also benefit a \acrshort{cs} project approach \autocite{wiegersSoftwareRequirements2013,youngEffectiveRequirementsPractices2001,youngProjectRequirementsGuide2006}. Due to the generally iterative nature of this framework, both classical and agile development practices can be applied, with the latter possibly having the advantage \autocite{confortoCanAgileProject2014,ManifestoAgileSoftware}.\newline
% ! hier bin ich noch nicht ganz zufrieden mit.. da geht doch noch was.

Besides the above mentioned challenges in the design phase and the general limitations named in the last section of this chapter, the work will also encounter challenges in the implementation and operation phases. In the (practical) application of this framework, some adjustments will be necessary and as it was the case of this work, time and resource constraints will be imposed by overarching projects or conditions, making some compromises inevitable. This is discussed in the next section in the case of creating a roadmap for implementation in Somaliland.

% ggf nur kurz nennen und an genrelles Projektmanagement verweisen

% ! framework
% how did the lit compare to the CBS and others?
% the findings/recommendations of the SRCS were only new in two major aspects
% -> assessment
% -> redflags
% anything else?
% key points: 
% was könnte schiefgehen?
% über Zeit stabil? --> jein. Mit stage 6 schon.
% metaperspektive
% konnte selbst unter dem Fokus auf Somalia nicht letztlich zugezurrt werden - viel allgemein gehalten. Einfach ein riesen topic. not surprising aber wichtig.
% https://data.jrc.ec.europa.eu/dataset/jrc-citsci-10004
% ja well. auch wenige Wasser data 
% so far so good. Lot's of it can also be derived by logical thinking, but: special juicy findings!
% vor und nachteile dieser 6 stages Aufteilung?
% -> iteration was more between the last four. not all 6. no other requirements catalogue could be found in the sources. some lists, some hints, but nothing really structured
% -> use my fking systematic thinking man.
% complex adaptive systems - critical parts? critical stakeholders? critical stages? could all be different for every project.
% funding? may also work or not. 
% % ! SSF
% open data. Its own huge discussion.
% data quality
% relate to other studies. Look in the book -> some stuff CBDRR, CBWM and so on..
% questions from other studies sparked this? e.g. global inequality of studies
% The first RQ is addressed in this section by discussing the literature and project analysis along with the \acrshort{ssdr} and \acrshort{prc}. The development of a replicable and adaptable framework for community-based participatory water source mapping and monitoring with the aim of facilitating AA in the context of \acrlong{fbf} 
%%%%%%%%%%%%%%%%%%%%%%%%%%%%%%%%%%%%%%%%%%%%%%%%%%%%%%%%%%%%%%%%%%%%%%%%%%%%%%%%%%%%%%%%%%%%%
% An in-depth analysis of existing data sets of water source point and feature information was considered, but  discarded after a first assessment. The very limited reliability, completeness and actuality of the available data sets had already been reviewed and stated by the project team before the start of this work. The lack of data was the main reason for this work to start with, and a short analysis via QGIS3 was able to confirm these statements. 
% was considered, but was discarded at a very early stage. The very limited reliability, completeness and actuality of the available data sets had already been reviewed and stated by the project team before the start of this work. This lack of data was the reason for this work, but a short analysis was able to confirm these statements and thus, due to relevance and time constraints, 
% and thus, due to relevance and time constraints, the focus shifted to the design approach.
% --> poor quality for water source data analysis --> was the reason for this work % nice transitions for reasons for other work
%%%%%%%%%%%%%%%%%%%%%%%%%%%%%%%%%%%%%%%%%%%%%%%%%%%%%%%%%%%%%%%%%%%%%%%%%%%%%%%%%%%%%%%%%%%%%%%%%%%
% specific application of the framework to the context
%? 2. RQ
%?%%%%%%%%%%%%%%%%%%%%%%%%%%%%%%%%%%%%%%%%%%%%%%%%%%%%%%%%%%%%%%%%%%%%%%%%%%%%%%%%%%%%%%%%%%%%%%%%%%
\section{Application}

%! intro
This section discusses the conducted application of the \acrshort{ssdr} and \acrshort{prc} for the creation of an implementation roadmap in the case study area. It is structured according to the six stages. Unexpected findings in this application were the issue about the server location in Ireland, the competition between the NGOs, and the MoH's initially negative attitude towards the inclusion of CBS due to oversupply by international NGOs. Furthermore, the heterogeneity in the community and stakeholders potentially contrary attitudes to the projects as highlighted by the interviewees in regard to the implementation was also only rarely mentioned in the considered guidelines. Specifically, results indicate, that local stakeholders such as private berkad owners or private water vendors may not only be in favour of this project. To be aware of financial motivations of all parties was only mentioned by \autocite{minkmanCitizenScienceWater2015} in the context of the literature considered in this thesis. However, this work was not a literature review about motivational factors of stakeholders, therefore, this can only be regarded as an unexpected first impression and does not imply more.\newline
% ! irgendwas zu stage 1?
% stage 1:
The exploration Stage 1 allowed for an open search in all directions and supported the identification and specification of the problem, context, project boundaries and to formulate first draft solutions. The situation on the ground was identified as severe and complex, with many interrelated and partly specific conditions and indicated, that the combination of topics is new and under-researched. Supported by these findings, the research type case study, in combination with the iterative mixed-method approach of literature and project analysis along expert interviews can in retrospective reasonably be assumed as fit for purpose. Further discussion of the methods is addressed in the coming section about limitations.\newline
The topic of community-based participatory water monitoring was found to be closely related to the already established \acrshort{cbs} project, which supported the transfer of knowledge to this new project. However, water source data set explorations could not support a further analysis due to the identified poor quality and incompleteness. This finding corresponds well with the impressions of the higher-level \acrshort{eap} development team and \autocite{harrowsmithFutureForecastImpact2020}. Another identified challenge was the current allocation mechanisms and risk assessment in regard to water trucking. However, this was not the aim of this work and will be addressed by the \acrshort{eap}.
% ! feasibilitiy assessment
The conducted feasibility assessment suggests the applicability of the project in the context of Somaliland. However, not all \textit{red flags} could be addressed due to this works constraints. In particular, one \textit{red flag} in regard to information sharing about exact water source locations may need more thought and consultations in the light of the relatively recent history of Somaliland where information about this critical infrastructure was used against the general public \autocite{republicofsomalilandSomalilandCountryProfile2021}. Findings also suggest the importance of an additional technical support as the capacities of \acrshort{srcs} are particularly limited in this regard.\newline
%! Stage 3 
In stage 3, the subdivision into the \acrshort{prc} was helpful to reduce cognitive overload and highlight chronological and thematic (inter-) dependencies. In terms of knowledge, the \acrshort{prc} helped to structure the information identified in Stages 1 and 2,  which additionally helped to make knowledge gaps, such as lack of detailed knowledge about local decision-making procedures, more visible.As it is was not feasible to gather these information in the scope of this work, it was therefore simplified to concentrate on those areas, that could be addressed. For example, it became clear that the initial mapping, which also includes the gathering of other key information about the berkad, cannot be done by local volunteers as the knowledge and technical equipment requirements are too high for most. Thus the initial mapping needs to be conducted by \acrshort{srcs} professionals who already meet the requirements. Nonetheless, gathering the information that is initially required for the mapping campaign was feasible in the context of the work and was thus focussed on. The knowledge gathered was thus more broad than deep and in most cases requires further investigation, especially in relation to local conditions. In the case of the \acrshort{aa} of water trucking e.g., see figure \ref{TODO:} in section \ref{subsubsec:assemblage}, requirements could be listed, but their actual specification is only possible on the ground with local stakeholders.\newline
% ! Groundwork
The findings suggest, that a sound foundation for general groundwork for this project is already laid (see section \ref{subsubsec:groundwork_appl}) as volunteers are well embedded in the social network and communities already manage water regulations themselves. Furthermore, the already conducted health related \acrshortpl{aa} of awareness raising about water borne diseases and water tablet dissemination suggests the assumption, that general familiarity with the topic of water management also already exists. Therefore, while the integration of an entire \acrshort{iwrm} may be a project on its own, at least aspects of this can potentially be well and easily integrated within the already prevailing procedures. However, contrary opinions must not be disregarded here either. The synthesis of knowledge will need to be discussed with local community leaders and key stakeholders and will strongly depend on the local context. Nonetheless, findings indicated willingness and experience of local managers to implement those concepts, which suggests at least a good initial situation for the successful embedding of the project into local management practices.
%! Innovations: water level + water quality -  add: water level as a proxy?
Besides the conceptual groundwork directly on site, innovations for the determination and collection of water level thresholds are required. The gathered information suggest, that there are two potential ways to assess the water level (see section \ref{subsubsec:innovations_appl}). The more technical measuring and transmission of the actual water height would require knowledge about the exact capacity and size of the berkad to assess the remaining water capacity. Although this method would provide a more objective measurement, local knowledge of the potential duration of water supply was also found to be good (see section \ref{subsubsec:assemblage}). Both approaches do not contradict each other and could also be used together. This would also allow a good basis over time for evaluating the quality of the assessment of local knowledge what could then improve local water monitoring methods and management. Potential codes were developed, but need to be evaluated in practice. In addition to the quantity of water, its quality was also considered very important, but no locally feasible approach to assessing quality could be identified. This supports the importance of providing a sound knowledge foundation about contamination prevention and water management practices to the community. This is also supported by several other studies \autocite{danielAssessingDrinkingWater2020,huangManagementDrinkingWater2020,tariqOpenSourceWater2021,wmoPlanningWaterqualityMonitoring2013}. Research in this field is still ongoing \autocite{tariqOpenSourceWater2021} and \autocite{delaireHowMuchWill2017} cost estimations suggest, that even with current water quality monitoring equipment, costs are minimal in relation to achieving the SDG 6.1 of safe water for all.\newline 
% Innovations:
% - new developments may be difficult for a standalone project without scientific accompaniment
% - but adjustments need to be considered. Which are the biggest? what could go wrong? what could be done now, to later integrate improved methods?  
%! water monitoring
While no management decision or developments could be made or evaluated in practice in the scope of this work, the findings suggest some additional considerations. \autocite{gualazziniEWEAEarlyWarning2021}highlighted that local actors may be aware that their answers can influence the subsequently provided help but also mentioned, that this can be addressed by good supervision structures and preliminary training. This is also supported by experiences from the \acrshort{cbs} approach which also suggest that these measures can greatly mitigate reporting biases.\newline
In the case of deciding for a specific water source monitoring strategy, all accessible water sources in a community should be monitored. The gathered information suggests, that the largest source, e.g. a ballay, is not necessarily the water source that can withstand a period of drought the longest. Physical as well as social access factors need to be considered in terms of actual water withdrawal and monitoring when deciding on the actual monitoring routine see sections \ref{subsubsec:assemblage} and \ref{subsubsec:innovations_appl}.\newline
It can be acknowledge that the water level of berkads is not the end of a potential monitoring routine. \autocite{unitednationschildrensfundunicefandworldhealthCoreQuestionsWater2018} created an extensive questionnaire for their interview surveys and \autocite{enenkelWhyPredictClimate2020} formulated questions that are relatively simple and remotely monitorable and give a bigger picture of the situation by aligning socio-economic elements with weather and climate monitoring. Their work may give additional insights for the extension of this approach. Though, the results of this work suggest, that the integration of even some of these questions may already go far beyond the scope of this project. Nevertheless, certain aspects can provide suggestions for practically implementable adjustments, e.g. the simple integration of precipitation measurements, which could already significantly improve the data situation. Possibly, a correlation could also be made with other measurements to see if the water level can also serve as a proxy for other parameters.\newline
%! staggered trigger + AA
The findings support the feasibility and usefulness of a staggered trigger as proposed by \autocite{rcrcFORECASTBASEDFINANCINGEARLY2020} for triggering on water level thresholds as both, a seasonal and a short-term assessment are possible. In terms of \acrshortpl{aa}, the results supported the feasibility of water trucking and cash transfer \acrshortpl{aa} (see sections \ref{subsec:case_eap} and \ref{subsubsec:assemblage}), which compares well with \autocite{gettliffeOCHAAnticipatoryAction2021} findings. Yet, when comparing this finding with the statements of the interviewees from section \ref{subsec:stage1_appl} that water is often over prized in times of scarcity and with the statement of \autocite{ochaANTICIPATORYACTIONPLAN2020} that markets need to be operational to permit the adequate and reasonable handling of the demand, distributions of cash or water vouchers may not always be feasible as \acrshortpl{aa}. These findings compare well with \acrshortpl{eap} in Lesotho and Niger. Direct cash or voucher distributions are also part of these \acrshort{eap}'s \acrshortpl{aa} and the \acrshort{eap} in Niger also accounts for higher market prices inits food security measures by including subsidies for cereal prizes.
%! Stage 4: community building + NYSS
The community building aspect in Stage 4 was primarily focussed on assessing the capacities of the \acrshort{srcs}. The findings suggest good capacities and high experiences in the area of community engagement as well as volunteer training and supervision (see section \ref{subsec:stage4_appl} and \ref{subsec:stage2_appl}). This was to be expected, as the \acrshort{srcs} already successfully implemented a comparable project and is also found to be performing well within the framework of the overarching \acrshort{fbf} project. Community selection of volunteers, which reinforces altruistic and communal values for lasting participation, is also indicated in the study by \autocite{rotmanDynamicChangesMotivation2012}.Unexpected was the finding that volunteers are primarily women, as this contrasts with the trend in \acrshort{cs} participation generally found in the literature \autocite{ibrahimGenderImbalanceSpatiotemporal2021,patemanDiversityParticipantsEnvironmental2021}.\newline
% ! Stage 5
Findings of Stage 5 \textit{Data management} support the technical practicability of the project. The solution of the implemented \acrshort{cbs} project with the NYSS platform is identified to be very dedicated and an adaptation is technically possible to the new requirements. Currently, discussions about the adaptation are ongoing on management level. However, the results also suggest that this project can be carried out using less specialised approaches than the NYSS platform presented, which could potentially simplify the technical aspects (see sections \ref{subsec:stage5_appl} and \ref{subsec:mcs}).\newline
% ! Quality and Accuracy - das braucht mehr Lieeeeeebe. Zu results lastig.
A major concern of data collected by \acrshort{cs} is their quality and accuracy. The findings indicate that these are well addressed in the current \acrshort{cbs} by the initial and refresher trainings of the participants, close supervision and further verifications when necessary. Further data triangulation with third party information may also improve confidence in the data. These \acrshort{qc} and \acrshort{qa} measures and the relatively simple task of water level monitoring are also identified in current literature as adequate \acrshort{cs} applications with good results \autocite{albusAccuracyLongtermVolunteer2020,baalbakiCitizenScienceLebanon2019,fraislCitizenScienceEnvironmental2022}. Furthermore, \autocite{aceves-buenoCitizenScienceApproach2015} findings suggest, that usefulness of the data may also be given, even though data quality issues persist to some extent. However, the greater constraint than quality in this case might be the limited number  of total codes per message. Here, the results indicate that no more than three codes for weekly monitoring are reasonably feasible without making major compromises in the choice of volunteers. Nevertheless, the \acrshort{cbs} project has already achieved some success with this restriction.\newline
Using more dedicated sensors such as smartphones, sensors or remote imagery are solutions considered in other projects, but the findings indicate that the rural population rarely has smartphones, sensors usually need some kind of network which may not be given, and berkads are often roofed and too small for satellite imagery \autocite{bartramGlobalMonitoringWater2014,klemasUsingRemoteSensing2015,maoMovingTechnologySociotechnical2020,masindeITIKIMobileBased2019,mcneilLandscapeParticipatorySurveillance2022a,senayEstablishingOperationalWaterhole2013,thomsonRemoteMonitoringRural2021}. The problem of network coverage is circumvented in this project as the volunteer is mobile and can go to a place with network reception, making the SMS coding solution the only viable solution at present.\newline
% Stage 6:
% output, quality, participant experience and impact
% -> evaluation practices may be difficult. Bias towards positive results
Evaluation practices are integrated in every stage but due to no actual implementation and operation, evaluation could not be conducted. Nonetheless, the design process, due to its iterative processes underwent several evaluations and piecemeal improvements itself suggesting good adaptability and upgradeability. One big open question that remains for the overarching EAP is the definition of success metrics for evaluation, which can significantly influence further developments.\newline

% ! Overall: Review of the application process and the SSDR
The overall application of the developed framework worked well and the combination of the \acrshort{ssf} together with the \acrshort{prc} based on the \acrshort{slmc} could provide good guidance while also remaining flexible to incorporate new and unexpected findings. Yet, the entire power of the \acrshort{slmc} could not be exploited as the coming layers were too detailed and most of those need to be determined in closer collaboration with the \acrshort{eap} team and the local stakeholders. Nonetheless, based on the positive experiences with the first three stages, it can reasonably be assumed that the following layers will also prove fruitful to potential future developments.\newline

The application also supports the findings of \autocite{garciaFindingWhatYou2021} and \autocite{conradCommunityBasedMonitoringFrameworks2008} that a framework should be used in designing a \acrshort{cs} project but also highlights the need to adapt this framework to the actual projects conditions and goals. Furthermore, the work suggests that a respective implementation of a \acrlong{cs} project is not only theoretically feasible but practically implementable. While an end to end establishment of an implementation roadmap was not feasible in the context of this work due to given limitations, it can reasonably be assumed, that a sound foundation could be laid for further practical implementation in the scope of a pilot study. Yet, important questions need to be answered on management level and not all indicators are in favour of a practical application. For example, the \acrshort{rcrc} is generally not recommending its National Societies to implement their own data gathering strategies as this would, under normal circumstances, over-burden and exceed costs \autocite{rcrcFORECASTBASEDFINANCINGEARLY2020}. Therefore, \autocite{rcrcFORECASTBASEDFINANCINGEARLY2020} generally recommends to found the triggers and information on already gathered information by other stakeholders or international organisations. However, as suggested by the results, this is not feasible in this context. Furthermore, the analysis of the \acrshort{cbs} approach and other projects together with the feasibility assessment and conducted application of this work suggest that \acrshort{cs} can be a reasonable and cost-effective approach to gather relevant information for triggering \acrshortpl{aa}. This is also supported by \autocite{aceves-buenoCitizenScienceApproach2015} and \autocite{minkmanCitizenScienceWater2015} findings. 
Finally, the results allow for the assumption that this timely and accurate data will be able to support appropriate \acrlongpl{aa}, potentially making mitigation and responses more streamlined, efficient and effective.\newline
Besides the high demands and complexity of the \acrshort{cs} project design itself, this process also highlighted, that \acrshort{cs} is not a silver bullet itself but comes with various advantages and disadvantages. While the final application may seem appealing, as a lot of the work is done by the contributing citizens, particularly the design and implementation poses a lot of requirements in time, skill, and resources. This complexity is also mentioned by \autocite{fraislCitizenScienceEnvironmental2022} and \autocite{minkmanCitizenScienceWater2015} highlighting the statement, that Citizen Science is one of many methods and its deployment should be well considered.

%----------------------------------------------------------------------------------------
% Step 4: Acknowledge the limitations of the study
%----------------------------------------------------------------------------------------

\section{Limitations}
% theoretical foundation
In any research project, limitations are an important aspect to consider and some were already addressed in the above discussion. Yet, there are further limitations that need to be acknowledged. The literature review did not follow a strict formal structure and that comparable projects may have been overlooked, although unlikely, cannot be ruled out. Nonetheless, this exploratory approach also allowed for the discovery of many, formerly unknown aspects and contributed many insights to the study. The subsequent in depth literature analysis, although not formal, was detailed, extended and was able to identify and address some gaps. However, the generally sparse literature on Somaliland limited the desk-based collection of information about local conditions. Furthermore, this work has also not addressed the integration or application of local and/or indigenous knowledge or the further use of VGI, to the extent that this would have been possible in principle. Although both areas are very interesting, this was either not the focus of this work or, in the case of available water source datasets and VGI, in-depth analysis was deemed inappropriate due to the poor quality identified early on. Nevertheless, insights could be gained from the data using more refined methods in future work. On the theoretical level of the contextual basis, the concepts mentioned, such as water security, drought or Citizen Science, are extremely complex and highly debated topics. Discussing them in detail would have exceeded the scope of this thesis, which is why focal points were set according to the priority of this work.\newline
% exploratory approach
The inclusive nature of the exploratory approach, was thus tried to be addressed by information triangulation from other studies but it made for a generally more consensual work and fewer contradictory findings. This, together with the inclusion of most relatable studies and projects into the framework itself, also made for a relatively homogeneous discussion due to the lack of contradictory findings. It is expected that this comparison with other work will be possible in the future as more CS projects are carried out in a similar context and with similar objectives.\newline
% case study
The critique of the case study research type, its challenge to execute, significant documentation efforts and complex nature also had certain shares in this work. Nevertheless, its strength of being rich, detailed and contextual also contributed positively. The generalisability of the understanding gained through actual application can be considered low, but the applicability of the framework developed can be expected to be transferable to other comparable contexts.\newline
% validities and biases
An attempt was made to improve internal validity through the iterative design, reciprocal reviews and triangulation of multiple sources of information, but despite great efforts, it is hardly possible to establish causal relationships in such a complex environment in just one case study. Constructed validity of the framework is believed to be reasonable due the extensive triangulation of resources but can only be evaluated in a practical examination. The importance of data triangulation was also noted and integrated in the actual application. Interviews always add a human factor which can complicate repeatability but clear procedures and documentation were established to account for this as best as possible. The expert and snowball sampling strategy itself worked well, but was severely limited by other factors. The interviews always had to be arranged and signed off by senior managers, and the already tense situation with ongoing response activities and parallel \acrshort{fbf} development in Somaliland made the availability of interviewees even scarcer. This resulted in a relatively low sample of interviewees, limiting the otherwise strength of a case study to incorporate a multitude of perspectives on one area of interest. Here, interviews especially with representatives of \acrshort{nadfor}, the \acrshort{mowr} and \acrshort{moh} as well as \acrshort{brcis} and \acrshort{ocha} might have been fruitful. The conversion of one interview into a questionnaire hindered direct clarification and follow-up questions in the interview, but this could be compensated for by a second questionnaire. Since a lot of information could be drawn from the answers and the interview could not have taken place otherwise, this can ultimately be seen as a good compromise.\newline
% overall
Overall, the study was conducted under difficult conditions in a case study area known for its complex and difficult environment. The generally limited time and resources available in the context of a Master's thesis further constrained the study and the focus on just the development of the framework might have been beneficial. Nevertheless, it is believed that the study conducted was ultimately able to provide a good theoretical and structural basis for a potential practical implementation of this approach. Furthermore, this study contributes to the general ongoing discourse of \acrlong{cs}-projects by adding a work from a currently underrepresented region. It is to be expected that in the further process of this discourse many of the limitations mentioned here can be addressed and overcome.
