% Chapter Template

\chapter{Discussion} % Main chapter title

\label{Chapter5} % Change X to a consecutive number; for referencing this chapter elsewhere, use \ref{ChapterX}

%----------------------------------------------------------------------------------------
%	SECTION 1
%----------------------------------------------------------------------------------------

\section{Main Section 1}

%-----------------------------------
%	SUBSECTION 1
%-----------------------------------
\subsection{Subsection 1}
application design 
“Citizen Science experiments have demonstrated that at least three issues must be addressed: commitment of people participating in data collection (were social component is key), the validation of data collected and sent by citizens and privacy concerns about the collection of personal information.” ([Alfonso and Jonoski, 2012, p. 6](zotero://select/groups/4773535/items/4W4Q8E6B)) ([pdf](zotero://open-pdf/groups/4773535/items/PU944FGI?page=6\&annotation=LZXCJDFB))

research vs. reality

other perspectives than process oriented -> 7-layer model self-critisism (resource, behavioral perspectives, + value and communication network design approaches)
%-----------------------------------
%	SUBSECTION 2
%-----------------------------------

\subsection{Subsection 2}

%----------------------------------------------------------------------------------------
%	SECTION 2
%----------------------------------------------------------------------------------------

\section{Main Section 2}

There is a need for research exploring if marginalized perspectives are excluded in crowdsourcing and self-reporting approaches, the recall bias and measurement error in self-reporting (Bell et al. 2019), and if incentives, memory triggers, or other mechanisms can be used to address these issues.



%----------------------------------------------------------------------------------------
%	SECTION 2 Future outlook
%----------------------------------------------------------------------------------------

“Connecting top-down weather and climate data with bottom-up socioeconomic data via machine learning” ([Enenkel et al., 2020, p. 1166](zotero://select/groups/4773535/items/RX575C79)) ([pdf](zotero://open-pdf/groups/4773535/items/XD499UNK?page=6&annotation=BCGJNKZB))

