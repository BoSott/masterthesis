% Chapter Template

\chapter{Discussion} % Main chapter title

\label{Chapter5} % Change X to a consecutive number; for referencing this chapter elsewhere, use \ref{ChapterX}

%----------------------------------------------------------------------------------------
%	SECTION 1
%----------------------------------------------------------------------------------------

\section{Main Section 1}
as: preparatory study (??)
%-----------------------------------
%	SUBSECTION 1
%-----------------------------------
"in your discussion, include a paragraph on strengths and weaknesses that you have discovered as a result of carrying out the research, say what you would do differently, and what would lend itself to further research."


\subsection{Discussion indices and forecasts}

Nonetheless, it needs to be mentioned, that most indices will always be proxies for other conditions, which can not or only very arduously be measured. 

“A major weakness of the existing tools is the emphasis on macro/international level information.” ([Masinde and Bagula, 2010, p. 390](zotero://select/groups/4773535/items/JNC4ACZS)) ([pdf](zotero://open-pdf/groups/4773535/items/IWMKDQYV?page=1&annotation=7Z44Z9L7))

“While food security forecasts may be an acceptable means for mobilizing global funding earlier, this does not imply that the underlying crisis is strictly a food security crisis.” ([OCHA, 2020, p. 6](zotero://select/groups/4773535/items/NYQL45SA)) ([pdf](zotero://open-pdf/groups/4773535/items/3PG9LF86?page=7&annotation=JD7JZTTN))

“. In fact, needs are multi-dimensional, interrelated and complex” ([OCHA, 2020, p. 6](zotero://select/groups/4773535/items/NYQL45SA)) ([pdf](zotero://open-pdf/groups/4773535/items/3PG9LF86?page=7&annotation=SMBIBQ3G))

localization of forecasts and so on.. why it is difficult to use SPEI, SPI and so on and others.. --> more data to operationalize is very important, even though it should not be the task of the SRCS --> high costs but a lot of trust in the communities and CBS is also working great.
drought indicator (what exists so far? -> pre study)

% drought monitoring tool for somalia
https://cdi.faoswalim.org/index/rfe-maps/2022

% Drought Assessment and Monitoring using some drought indicators in the semi-arid Puntland state of Somalia
https://d1wqtxts1xzle7.cloudfront.net/61927859/DroughtAssessmentandMonitoringUsingSomeDroughtIndicatorsintheSemi-AridPuntlandStateofSomalia_FEB_2019NovemberIssue11A20200129-7600-1cwi1.pdf?1580293969=&response-content-disposition=inline%3B+filename%3DDROUGHT_ASSESSMENT_AND_MONITORING_USING.pdf&Expires=1678353893&Signature=U3ArqyQZMaz2Pbm6iwGv-0qMl0meKe6028igIkC5qjjTYaQ4OtNvKRiAKBzGNVWYCOBNxMmOtgriFg7Mavekgov3BRt38tG-e7tD86OYU4qS0zQfshEmQxBD9tM~CqqO45XBASpGpuQFl5Atks6ADcjQef03Cds7bRoTCrwyz-rOJvHo3nyHk~lbfupqWrgYzZ6oE~YTmUC6iQ7xNtKf-XNcjW4z6R7Rtvu1NXvoX~YWLFRrMa50O3kTUISwEEsD-Rw7QiVNPF~VBOa2~PepbkueXRRGuCRolvb1q95DQVqHXQkaS6jG1UUvJ-vrmqH0wAX-zmo-r31yuGQLbkQ5UQ__&Key-Pair-Id=APKAJLOHF5GGSLRBV4ZA

but: 
“Scale is critical in assessing water security [31]. National level assessments make it difficult to take action at operationalization level.” ([Mishra et al., 2021, p. 8](zotero://select/groups/4773535/items/MD2Z2HTF)) ([pdf](zotero://open-pdf/groups/4773535/items/366Z36U7?page=8&annotation=6IAXCXUB))

and: “Creating and using indicators for water security has to be directed towards some management control or assessment action.” ([Mishra et al., 2021, p. 8](zotero://select/groups/4773535/items/MD2Z2HTF)) ([pdf](zotero://open-pdf/groups/4773535/items/366Z36U7?page=8&annotation=P72LT9Y8))


% maybeeeeee and maybe not.. tend not to include it
“2 Local knowledge in drought monitoring: an introduction to the literature review” ([Giordano et al., 2013, p. 526](zotero://select/groups/4773535/items/B7LM5ZR4)) ([pdf](zotero://open-pdf/groups/4773535/items/7I66DBIK?page=4&annotation=Z33M5FLQ))

“There has been little effort to align the spatiotemporal granularity of socioeconomic assessments with the granularity of weather or climate monitoring.” ([Enenkel et al., 2020, p. 1161](zotero://select/groups/4773535/items/RX575C79)) ([pdf](zotero://open-pdf/groups/4773535/items/XD499UNK?page=1&annotation=QBTLFCXM))
“we highlight the need to collect and analyze environmental and socioeconomic data together and discuss novel strategies for coordinated data collection via mobile technologies from a drought risk management perspective.” ([Enenkel et al., 2020, p. 1161](zotero://select/groups/4773535/items/RX575C79)) ([pdf](zotero://open-pdf/groups/4773535/items/XD499UNK?page=1&annotation=9BUBHWNB))

“but questions related to coping capacities, migration, poverty, water supply, access to food and markets, or political conflict remain unanswered or are even decoupled from routine drought risk assessments” ([Enenkel et al., 2020, p. 1162](zotero://select/groups/4773535/items/RX575C79)) ([pdf](zotero://open-pdf/groups/4773535/items/XD499UNK?page=2&annotation=HE48ZWFA))

“The handbook of drought indicators (Svoboda et al. 2016) lists more than 50 drought indices. Not a single one of these indices connects climate anomalies to socioeconomic vulnerabilities,” ([Enenkel et al., 2020, p. 1163](zotero://select/groups/4773535/items/RX575C79)) ([pdf](zotero://open-pdf/groups/4773535/items/XD499UNK?page=3&annotation=EDKZFHJX))
--> there were more indicators in the following year, but those were limited due to availibiltiy of socioeconomic data


% List of drought indicators
https://heigit.atlassian.net/wiki/spaces/FIS/pages/1704096/Indices


% e.g. SPEI has difficulties because of the scarce hydro-meteorological monitoring network
%SPEI-based spatial and temporal evaluation of drought in Somalia
http://www.repository.embuni.ac.ke/bitstream/handle/embuni/3705/SPEI-based%20spatial%20and%20temporal%20evaluation%20of%20drought%20in%20Somalia.pdf?sequence=1&isAllowed=y
% notes from the team
evaluate the spatio-temporal variations of drought occurrences in Somalia for the period between 1980 and 2015 as quantified by Standardized Precipitation Evapotranspiration Index (SPEI)
The temporal variations in drought showed decreasing trends in severity and increasing trends in drought duration as the SPEI timescales increas
However, the study was limited by the low security levels in Somalia that makes
the country largely inaccessible. This has complicated possibilities for
ground-truthing available information. Besides, the 1991 civil war in
Somalia led to the collapse of hydro-meteorological monitoring network
(Muchiri, 2007). The current stations established by Food and Agricul-
ture Organization Somalia Water and Land Information Management
(FAO SWALIM) from 2002 do not have uniform spatial distribution
throughout the country and therefore cannot be useful in computing
reference SPE

% this level of impact can then.. 
one can access the impact of a certain weather phenomenon or climate development on the situation on site. These information can then facilitate exact and precise assessments of the conditions and help to react accordingly (policy, management, etc. ) 
% possibly too much of a discussion.. 


% later on
However, it requires complex and various information and is thus often difficult to implement \autocite{liuWaterScarcityAssessments2017}.

... in the light of
While all of these more dedicated indicators may present great value, relatively simple Indices such as the SPEI is not possible for Somalia due to the lack of a good enough weather monitoring network. 

% this level of impact can then.. 
one can access the impact of a certain weather phenomenon or climate development on the situation on site. These information can then facilitate exact and precise assessments of the conditions and help to react accordingly (policy, management, etc. ) 
% possibly too much of a discussion.. 


... in the light of
While all of these more dedicated indicators may present great value, relatively simple Indices such as the SPEI is not possible for Somalia due to the lack of a good enough weather monitoring network. 



research vs. reality

other perspectives than process oriented -> 7-layer model self-critisism (resource, behavioral perspectives, + value and communication network design approaches)

“It is important to note that local indicators cannot be collected specifically for the FbF system by RCRC national societies.” ([RCRC, 2020, p. 30](zotero://select/groups/4773535/items/UESIQTRJ)) ([pdf](zotero://open-pdf/groups/4773535/items/P5JPVZ97?page=30&annotation=LRDPV2M7))

“Indeed, collecting data on local indicators would require from the national society a team of enumerators that work continually to collect and process that information in all places where the program could possibly trigger (e.g. collect food price information for every village market). This would have extensive cost implications and likely over-burden the national society staff and volunteers.” ([RCRC, 2020, p. 30](zotero://select/groups/4773535/items/UESIQTRJ)) ([pdf](zotero://open-pdf/groups/4773535/items/P5JPVZ97?page=30&annotation=2YIIK6ZY))

“As such, the inclusion of local indicators into an FbA trigger must involve assessing what indicators are relevant for the impacts the program is trying to anticipate and identify which of those indicators are already collected (e.g. the ministry of agriculture's food price bulletin) and are available at the time they would be needed to inform a possible trigger.” ([RCRC, 2020, p. 30](zotero://select/groups/4773535/items/UESIQTRJ)) ([pdf](zotero://open-pdf/groups/4773535/items/P5JPVZ97?page=30&annotation=7X3RFGVB))

“Thinking outside the box in terms of both hydro-meteorological and socio-economic indicators could be particularly useful” ([RCRC, 2020, p. 31](zotero://select/groups/4773535/items/UESIQTRJ)) ([pdf](zotero://open-pdf/groups/4773535/items/P5JPVZ97?page=31&annotation=GNZJ3FR5))


The \textit{Drought Monitoring Tool for Somalia} FAOSWALIM is currently also looking for alternative indicators for their Combined Drought Index (CDI) since the current proxy for soil moisture, NDVI does not correlate well with ground information anymore.  https://cdi.faoswalim.org/index/cdi

% stakeholder perceptions and limitations
difference scientific vs. stakeholders perception of drought impact
“The comparison between stakeholders’ perception of drought impacts and scientific knowledge allowed us to draw some preliminary conclusions concerning both the drought impacts at local level and the coherence between local and scientific knowledge on drought impacts.” ([Giordano et al., 2013, p. 539](zotero://select/groups/4773535/items/B7LM5ZR4)) ([pdf](zotero://open-pdf/groups/4773535/items/7I66DBIK?page=17\&annotation=QFD3UE4C))

“the correlation degree between perception indicators and scientific indicators is high only when considering a direct impact of drought, for example the reduction of productivity for non-irrigated crops.” ([Giordano et al., 2013, p. 540](zotero://select/groups/4773535/items/B7LM5ZR4)) ([pdf](zotero://open-pdf/groups/4773535/items/7I66DBIK?page=18&annotation=6KRIS4DA))
“Firstly, stakeholders tend to oversimplify the cause-effect chain” ([Giordano et al., 2013, p. 540](zotero://select/groups/4773535/items/B7LM5ZR4)) ([pdf](zotero://open-pdf/groups/4773535/items/7I66DBIK?page=18&annotation=IBV34JQL))
“Stakeholders seem to focus exclusively on the portion of the system they perceived.” ([Giordano et al., 2013, p. 541](zotero://select/groups/4773535/items/B7LM5ZR4)) ([pdf](zotero://open-pdf/groups/4773535/items/7I66DBIK?page=19&annotation=73P565M3))
“On the one hand, this allows them to have a clear picture of the evolutionary trends of the different variables that make up that portion of the system” ([Giordano et al., 2013, p. 541](zotero://select/groups/4773535/items/B7LM5ZR4)) ([pdf](zotero://open-pdf/groups/4773535/items/7I66DBIK?page=19&annotation=ZJSACLXJ))
“On the other hand, due to the limitations of their viewpoint, stakeholders tend to neglect the existence of multiple causes for certain observed phenomena.” ([Giordano et al., 2013, p. 541](zotero://select/groups/4773535/items/B7LM5ZR4)) ([pdf](zotero://open-pdf/groups/4773535/items/7I66DBIK?page=19&annotation=8CUNSCXS))
“Secondly, the data collected shows that the stakeholders tend to aggregate the variables describing the drought impacts, for example the impacts of crop productivity vary dramatically according to the crop under consideration, whereas stakeholders perceived a general reduction in productivity.” ([Giordano et al., 2013, p. 541](zotero://select/groups/4773535/items/B7LM5ZR4)) ([pdf](zotero://open-pdf/groups/4773535/items/7I66DBIK?page=19&annotation=TT7JRW3U))

% lessons learned from OCHA, while this is interesting, IFRC has way more experience already, making this somewhat superficial.. nonetheless, may be an alright transition
“UN OCHA ANTICIPATORY ACTION. LESSONS FROM THE 2020 SOMALIA PILOT” ([Gettliffe, 2021, p. 1](zotero://select/groups/4773535/items/2DG8MZPF)) ([pdf](zotero://open-pdf/groups/4773535/items/WQU78KAC?page=1&annotation=AHGXJ6KY))



%%%%%%%%%%%%%%%%%%%%%%%%%%%%%%%%%%%%%%%%%%%%%%%%%%%%%%%%%%%%%%%%%%%%%%%%%%%%%%%%%%%%%%%%%%%%%%%%%%%
%%%%%%%%%%%%%%%%%%%%%%%%%%%%%%%%%%%%%%%%%%%%%%%%%%%%%%%%%%%%%%%%%%%%%%%%%%%%%%%%%%%%%%%%%%%%%%%%%%%
%%%%%%%%%%%%%%%%%%%%%%%%%%%%%%%%%%%%% !!!CITIZEN SCIENCE !!! %%%%%%%%%%%%%%%%%%%%%%%%%%%%%%%%%%%%%%
%%%%%%%%%%%%%%%%%%%%%%%%%%%%%%%%%%%%%%%%%%%%%%%%%%%%%%%%%%%%%%%%%%%%%%%%%%%%%%%%%%%%%%%%%%%%%%%%%%%
%%%%%%%%%%%%%%%%%%%%%%%%%%%%%%%%%%%%%%%%%%%%%%%%%%%%%%%%%%%%%%%%%%%%%%%%%%%%%%%%%%%%%%%%%%%%%%%%%%%
%%%%%%%%%%%%%%%%%%%%%%%%%%%%%%%%%%%%%%%%%%%%%%%%%%%%%%%%%%%%%%%%%%%%%%%%%%%%%%%%%%%%%%%%%%%%%%%%%%%
\subsection{Citizen Science}




%\acrshort*{mcs} has great potential to cover wide areas and even extend monitoring efforts to an entire country, but local conditions must be taken into account to contribute to a successful project outcome\autocite{huangManagementDrinkingWater2020}. 

% Challenges:
“Challenges and risks have related to: securing free SMS channels for alerts, time required for contextualised design, ensuring response action, and managing expectations of the scope of CBS.” ([Byrne and Nichol, 2020, p. 71](zotero://select/groups/4773535/items/R9QNVPU8)) ([pdf](zotero://open-pdf/groups/4773535/items/VFH6JVJS?page=2&annotation=KKVDV6DE))

% “strengths of using mobile phone sensing: • Highly mobile and scalable; • Low-cost; • Automatic time stamp and GPS possible; • Citizens could interfere when necessary.” ([Minkman, 2015, p. 182](zotero://select/groups/4773535/items/ZKLE6CPT)) ([pdf](zotero://open-pdf/groups/4773535/items/QMAPCSZG?page=182&annotation=9GR233CN))


“The results suggest that citizen science can be a cost-effective method to collect essential monitoring information and can also produce the high levels of citizen engagement that are vital to the adaptive management learning process.” ([Aceves-Bueno et al., 2015, p. 493](zotero://select/groups/4773535/items/YK2MKLA9)) ([pdf](zotero://open-pdf/groups/4773535/items/WGGHNGZB?page=1&annotation=ZQQBEP74))

“Furthermore it is an interesting communication tool in the light of science communication. Correspondingly water managers should be interested in participatory monitoring in the light of integrated water management.” ([Minkman, 2015, p. 199](zotero://select/groups/4773535/items/ZKLE6CPT)) ([pdf](zotero://open-pdf/groups/4773535/items/QMAPCSZG?page=199&annotation=GHI9KSDA))


“The Water Poverty Index: Development and application at the community scale” ([Sullivan et al., 2003, p. 189](zotero://select/groups/4773535/items/3I3AYIAT)) ([pdf](zotero://open-pdf/groups/4773535/items/MSRKDWQ4?page=1&annotation=7CGSBY3V))

“When Are Mobile Phones Useful for Water Quality Data Collection? An Analysis of Data Flows and ICT Applications among Regulated Monitoring Institutions in Sub-Saharan Africa” ([Kumpel et al., 2015, p. 10846](zotero://select/groups/4773535/items/GPM4C7RJ)) ([pdf](zotero://open-pdf/groups/4773535/items/7VXVKEXK?page=1&annotation=M5J4FFSH))

“Remote monitoring of rural water systems: A pathway to improved performance and sustainability?” ([Thomson, 2021, p. 1](zotero://select/groups/4773535/items/UQLXVVYI)) ([pdf](zotero://open-pdf/groups/4773535/items/K9XBXPQD?page=1&annotation=B59B5U68))

“The tools also tend to ignore the at risk community who happen to be host to very crucial traditional knowledge on droughts” ([Masinde and Bagula, 2010, p. 390](zotero://select/groups/4773535/items/JNC4ACZS)) ([pdf](zotero://open-pdf/groups/4773535/items/IWMKDQYV?page=1&annotation=VUYYP2LA))

“The analysis also provides a set of recommendations for citizen science program design that addresses spatial and temporal scale, data quality, costs, and effective incentives to facilitate participation and integration of findings into adaptive management.” ([Aceves-Bueno et al., 2015, p. 493](zotero://select/groups/4773535/items/YK2MKLA9)) ([pdf](zotero://open-pdf/groups/4773535/items/WGGHNGZB?page=1&annotation=3NRZ7R8Y))

“Even when receiving timely information on the ground, Members often had to wait for validation from national sources such as the Food Security and Nutrition Analysis Unit (FSNAU) to make decisions, and the resources for anticipatory action were very limited.” ([Gualazzini, 2021, p. 4](zotero://select/groups/4773535/items/BWDYDL8T)) ([pdf](zotero://open-pdf/groups/4773535/items/8U5XVU5K?page=4&annotation=CZST8S2L))


%limitations: 
“Additionally, there may be limitations related to designing and implementing citizen science projects in remote and unsafe areas, where crime levels are high and political risks exist, or where mobile network coverage is poor, access to smartphones and electricity is low and illiteracy levels among participants are high. Co-design and community-based approaches can address such challenges and ensure a high level of participant engagement20” ([Fraisl et al., 2022, p. 14](zotero://select/groups/4773535/items/FBJD7SWS)) ([pdf](zotero://open-pdf/groups/4773535/items/7WBDKYDY?page=14&annotation=MZ95QT2P))

“Current weather forecasts are still alien to African farmers, most of whom live in rural areas and struggle with illiteracy and poor communications infrastructure” ([Masinde and Bagula, 2012, p. 274](zotero://select/groups/4773535/items/EW9XSSZP)) ([pdf](zotero://open-pdf/groups/4773535/items/3WQ4S9PE?page=1&annotation=HPZC9Z65))



More examples of \acrshort*{mcs} were compiled and analysed by fraisl and xy
% environmental monitoring --> Zheng 2018 figure 10 and chapter 4 -> quite extensive (weather, air quality, precipitation, geography (VGI), ecology, surface water,) + table 2 overview literature for these

% further tools and platforms: “Table 2 | Examples of existing citizen science data collection platform” ([Fraisl et al., 2022, p. 5](zotero://select/groups/4773535/items/FBJD7SWS)) ([pdf](zotero://open-pdf/groups/4773535/items/7WBDKYDY?page=5&annotation=UHP9I7LS))



“4 | IS REMOTE MONITORING A PATHWAY TO SUSTAINABILITY?” ([Thomson, 2021, p. 9](zotero://select/groups/4773535/items/UQLXVVYI)) ([pdf](zotero://open-pdf/groups/4773535/items/K9XBXPQD?page=9&annotation=833Q66UP))

“The results of this review, however, highlighted no usable preexisting system for SIMILE project, since the considered applications are too tied to specific geographic locations or too broad presenting no real connection with the lake ecosystem.” ([Carrion et al., 2020, p. 246](zotero://select/groups/4773535/items/7855TDUA)) ([pdf](zotero://open-pdf/groups/4773535/items/XI5TRN34?page=2&annotation=4P6WS72K))

“Other questions raised include; how to combine relevant science, technology and local knowledge” ([pdf](zotero://open-pdf/groups/4773535/items/SH3998CR?page=15&annotation=4DI7BR57))




%-----------------------------------
%	SUBSECTION 2
%-----------------------------------

%%%%%%%%%%%%%%%%%%%%%%%%%%%%%%%%%%%%%%%%%%%%%%%%%%%%%%%%%%%%%%%%%%%%%%%%%%%%%%%%%%%%%%%%%%%%%%%%%%%
%%%%%%%%%%%%%%%%%%%%%%%%%%%%%%%%%%%%%%%%%%%%%%%%%%%%%%%%%%%%%%%%%%%%%%%%%%%%%%%%%%%%%%%%%%%%%%%%%%%
%%%%%%%%%%%%%%%%%%%%%%%%%%%%%%%%%%%%% !!!  FbF and EAP  !!! %%%%%%%%%%%%%%%%%%%%%%%%%%%%%%%%%%%%%%%
%%%%%%%%%%%%%%%%%%%%%%%%%%%%%%%%%%%%%%%%%%%%%%%%%%%%%%%%%%%%%%%%%%%%%%%%%%%%%%%%%%%%%%%%%%%%%%%%%%%
%%%%%%%%%%%%%%%%%%%%%%%%%%%%%%%%%%%%%%%%%%%%%%%%%%%%%%%%%%%%%%%%%%%%%%%%%%%%%%%%%%%%%%%%%%%%%%%%%%%
%%%%%%%%%%%%%%%%%%%%%%%%%%%%%%%%%%%%%%%%%%%%%%%%%%%%%%%%%%%%%%%%%%%%%%%%%%%%%%%%%%%%%%%%%%%%%%%%%%%
\subsection{FbF, EAP and actions}

% takes long (25 days) is super resource intensive (key informant interviews -> 3 communities, 4 people each, lots of work.) + trust + triangulation + barely any automatization and rather coarse -> no detailed risk and vulnerability assessments (though this point may not be too valid.. -> the people know the area)
“Real-Time Risk Monitoring (RTRM)” ([Gualazzini, 2021, p. 4](zotero://select/groups/4773535/items/BWDYDL8T)) ([pdf](zotero://open-pdf/groups/4773535/items/8U5XVU5K?page=4&annotation=HZFMU84X))} % not too sure if I wanna include this or where.. maybe only in the discussion part? -> in comparison to Crowdsensing/MCS this is slow and relatively coarse. -> OCHA 2022 learning as well

% Community-led early warning and anticipatory action in Somalia
https://www.sparc-knowledge.org/news-features/features/community-led-early-warning-and-anticipatory-action-somalia
The \acrshort*{brcis} network 

,“Climate information presented as early warnings are only as valuable as the actions that are taken in response to the information, even if the information is a perfect warning of future events.” ([Mariani et al., 2015, p. 8](zotero://select/groups/4773535/items/8THVVJVK)) ([pdf](zotero://open-pdf/groups/4773535/items/GYUFNK32?page=8&annotation=Y9DG5FSE))



%During the development of an \acrshort{eap} more and complex challenges need to be addressed such as the lack of coordination and integration between sectors, limited capacities on the ground to decide and react accordingly due to data and resource shortages as well as the reality that they often work in areas where they have to operate in a complex network of actors and conflicting interests.

% --> language is important. --> staggered trigger -> cannot call everything an action! There is a difference between small and large repsonsen etc.. especially in the perception of people
“We also found that the phrases we had been using to trigger action, Be Aware, Be Prepared and Take Action, were not driving the actions we had hoped. We used Take Action exclusively for high impact warnings, but we need people to be taking action for low and medium impact warnings too, if the impacts are to be mitigated. We have replaced these phrases with detailed impact information and linked our warnings to advice and guidance from our partner organisations.” ([Harrowsmith et al., 2020, p. 62](zotero://select/groups/4773535/items/QJ397Y54)) ([pdf](zotero://open-pdf/groups/4773535/items/2GS362N5?page=62&annotation=KZK3TSK3))


%%%%%%%%%%%%%%%%%%%%%%%%%%%%%%%%%%%%%%%%%%%%%%%%%
%%%%%%%%%%%% trigger:
“Given the different layers of complexity with drought, different types of triggers may be required beyond what is often used in EAP development. For instance, unconventional triggers for FbA for drought could include metrics such as staple food prices, percentages of crop failure, and other elements of food security early warning systems.” ([RCRC, 2020, p. 30](zotero://select/groups/4773535/items/UESIQTRJ)) ([pdf](zotero://open-pdf/groups/4773535/items/P5JPVZ97?page=30&annotation=JZV26DPP))
% --> even the RCRC is still looking for good triggers -> maybe water levels are a good way -> reasoning for this study

“Intervening early to respond to spikes in need – i.e. before negative coping strategies are employed - can deliver significant gains and should be prioritized.” ([USAID, 2018, p. 6](zotero://select/groups/4773535/items/LGRWAU43)) ([pdf](zotero://open-pdf/groups/4773535/items/MBXSCVWR?page=6&annotation=C47BGB9V))

\subsection{Citizen Science -> CBS and Tools}
This works follows the “general public, typically as part of a collaborative project with professional scientists” ([Kirschke et al., 2022, p. 2](zotero://select/groups/4773535/items/GPC3LDT5)) ([pdf](zotero://open-pdf/groups/4773535/items/AI7HRQYC?page=2&annotation=BTE2JFKV)) which is basically “community-based monitoring, which also explicitly focuses on the public ’s involvement in monitoring processes” ([Kirschke et al., 2022, p. 2](zotero://select/groups/4773535/items/GPC3LDT5)) ([pdf](zotero://open-pdf/groups/4773535/items/AI7HRQYC?page=2&annotation=K4URWWLV))

“Local citizens are seen as essential participants in collaborative environmental management because they can provide vital information about the area’s natural and sociopolitical systems as well as support for measures to address non–point source pollution” ([Koehler and Koontz, 2008, p. 143](zotero://select/groups/4773535/items/2PKW9CYX)) ([pdf](zotero://open-pdf/groups/4773535/items/YYG4JK3R?page=1&annotation=MV6JAERB))


“Furthermore it is an interesting communication tool in the light of science communication. Correspondingly water managers should be interested in participatory monitoring in the light of integrated water management.” ([Minkman, 2015, p. 199](zotero://select/groups/4773535/items/ZKLE6CPT)) ([pdf](zotero://open-pdf/groups/4773535/items/QMAPCSZG?page=199&annotation=GHI9KSDA))

% in comparison to sensor networks:
“It is suggested, based on literature, that water authorities might prefer MCS to wireless sensor networks, for its mobility, lower costs and scalability.” ([Minkman, 2015, p. 11](zotero://select/groups/4773535/items/ZKLE6CPT)) ([pdf](zotero://open-pdf/groups/4773535/items/QMAPCSZG?page=11&annotation=RM5PX55M))

%%%%%%%%%%%%%%%%%%%%%%%%%%%%% water scarcity and rural water sources
“Community cultures, economies, and environments differ across countries and regions. These differences should be considered when designing hybrid management strategies, so that all actors can be appropriately enabled and the mechanism which is most effective for the given community can be identified.” ([Huang et al., 2020, p. 147](zotero://select/groups/4773535/items/9CSBLJNJ)) ([pdf](zotero://open-pdf/groups/4773535/items/G5BEZQ7C?page=12&annotation=WV5DXV5I))

“On this basis, it is essential to expand research area to study the various threats from climate variability to rural drinking water safety, and then to develop corresponding measures to address those threats to water security.” ([Huang et al., 2020, p. 147](zotero://select/groups/4773535/items/9CSBLJNJ)) ([pdf](zotero://open-pdf/groups/4773535/items/G5BEZQ7C?page=12&annotation=HCQHR7YR))

-> can change the way the water sector is funded: “With these data readily available, performance-related contracts that incentivize sustainable service delivery over short-term infrastructure investment can become the norm.” ([Thomson, 2021, p. 11](zotero://select/groups/4773535/items/UQLXVVYI)) ([pdf](zotero://open-pdf/groups/4773535/items/K9XBXPQD?page=11&annotation=JP9R4Y89))


“Our results indicate that using the phones to transmit more than just water quality data will likely improve the effectiveness and sustainability of this type of intervention.” ([Kumpel et al., 2015, p. 10846](zotero://select/groups/4773535/items/GPM4C7RJ)) ([pdf](zotero://open-pdf/groups/4773535/items/7VXVKEXK?page=1&annotation=4DJIADX2))
%%%%%%%%%%%%%%%%%%%%%%%%%%%%%%%%%%%%%%%%%%%%%%%%%%%%%%%%%%%%%%%%%%%%%%%%%%%%%%%%%%%%%%%%%%%%%%%%%%%
%%%%%%%%%%%%%%%%%%%%%%%%%%%%%%%%%%%%%%%%%%%%%%%%%%%%%%%%%%%%%%%%%%%%%%%%%%%%%%%%%%%%%%%%%%%%%%%%%%%
%%%%%%%%%%%%%%%%%%%%%%%%%%%%%%%%%%%%% !!!  OWN RESULTS  !!! %%%%%%%%%%%%%%%%%%%%%%%%%%%%%%%%%%%%%%%
%%%%%%%%%%%%%%%%%%%%%%%%%%%%%%%%%%%%%%%%%%%%%%%%%%%%%%%%%%%%%%%%%%%%%%%%%%%%%%%%%%%%%%%%%%%%%%%%%%%
%%%%%%%%%%%%%%%%%%%%%%%%%%%%%%%%%%%%%%%%%%%%%%%%%%%%%%%%%%%%%%%%%%%%%%%%%%%%%%%%%%%%%%%%%%%%%%%%%%%
%%%%%%%%%%%%%%%%%%%%%%%%%%%%%%%%%%%%%%%%%%%%%%%%%%%%%%%%%%%%%%%%%%%%%%%%%%%%%%%%%%%%%%%%%%%%%%%%%%%
\subsection{Own results and how the relate to all of this shizzle}








%%%%%%%%%%%%%%%%%%%%%%%%%%%%%%%%%%%%%%%%%%%%%%%%%%%%%%%%%%%%%%%%%%%%%%%%%%%%%%%%%%%%%%%%%%%%%%%%%%%
%%%%%%%%%%%%%%%%%%%%%%%%%%%%%%%%%%%%%%%%%%%%%%%%%%%%%%%%%%%%%%%%%%%%%%%%%%%%%%%%%%%%%%%%%%%%%%%%%%%
%%%%%%%%%%%%%%%%%%%%%%%%%%%%%%%%%%%%% !!! WATER  SOURCES !!! %%%%%%%%%%%%%%%%%%%%%%%%%%%%%%%%%%%%%%
%%%%%%%%%%%%%%%%%%%%%%%%%%%%%%%%%%%%%%%%%%%%%%%%%%%%%%%%%%%%%%%%%%%%%%%%%%%%%%%%%%%%%%%%%%%%%%%%%%%
%%%%%%%%%%%%%%%%%%%%%%%%%%%%%%%%%%%%%%%%%%%%%%%%%%%%%%%%%%%%%%%%%%%%%%%%%%%%%%%%%%%%%%%%%%%%%%%%%%%
%%%%%%%%%%%%%%%%%%%%%%%%%%%%%%%%%%%%%%%%%%%%%%%%%%%%%%%%%%%%%%%%%%%%%%%%%%%%%%%%%%%%%%%%%%%%%%%%%%%
\subsection{water source/access monitoring etc.}


“Within these citizen science projects, monitoring mainly focuses on rivers (48%) but also considers lakes (26%), groundwater (12.57%), and estuaries (13.33%)” ([Kirschke et al., 2022, p. 4](zotero://select/groups/4773535/items/GPC3LDT5)) ([pdf](zotero://open-pdf/groups/4773535/items/AI7HRQYC?page=4&annotation=QGL5N2BA))
% while there are lots of hydrological monitoring stuff out there (mostly crowdsensing in the case of water + focused on health by CBS), there is none, that looks at drinking water source monitoring direktly by applying the proven sets of methods on a new topic in a new environment (not North America but resource scarce environments from the perspective of an NGO (SRCS)) 
“Citizen scientists measure physical (34.22%, e.g., temperature, turbidity, color), chemical (34.76%, e.g., pH, nitrates, phosphates, dissolved oxygen), biological (22.99%, e.g., fecal coliform bacteria, algae, aquatic macroinvertebrates), and some other parameters (8.02%, e.g., macro/microplastic pollution, riparian habitat), demonstrating the diverse tasks citizen scientists engage in. Citizen scientists further measure these parameters rather ‘regularly’ (82.93%) than ‘sometimes only’ (17.07%), indicating continuity and thus a substantial contribution of citizens to the monitoring process.” ([Kirschke et al., 2022, p. 4](zotero://select/groups/4773535/items/GPC3LDT5)) ([pdf](zotero://open-pdf/groups/4773535/items/AI7HRQYC?page=4&annotation=6GC8UE8I))

% from case study EAP stuff --> here is interesting, that the largest aka the ballay (open reservoir) may be the largest, though definitely not the one which holds water for the longest time.. so not a very good trigger + berkads are often build in clusters, making an average of these more important + no ownership -> access is not observed.. shitty shit shit.
For example, the condition of the primary water source is assessed at the end of rainy season and based on its water level categorized into normal (more than half-full [75\%] or full), alert (half-full [50\%]) or alert (less than half-full [25\%] or empty).


%core paper:
“Community-based water resources management” ([Day, 2009, p. 47](zotero://select/groups/4773535/items/YWSNQ8A2)) ([pdf](zotero://open-pdf/groups/4773535/items/ETPCI5RI?page=2&annotation=RQLJMKL7))

“Households’ perceptions of their drinking water quality were mostly influenced by the water’s visual appearance, and these perceptions in general motivated their use of HWT.” ([Daniel et al., 2020, p. 1](zotero://select/groups/4773535/items/ZA5MKHPC)) ([pdf](zotero://open-pdf/groups/4773535/items/CS4A7XIN?page=1&annotation=KHJHJFPM))
“Improving water quality within the distribution network and promoting safer water handling practices are proposed to reduce the health risk due to consumption of contaminated water in this setting.” ([Daniel et al., 2020, p. 1](zotero://select/groups/4773535/items/ZA5MKHPC)) ([pdf](zotero://open-pdf/groups/4773535/items/CS4A7XIN?page=1&annotation=3WKEQLY4))

%Berkad construction not only helpful --> also lead to overgrazing “The explosion of permanent water points means that the Haud is now grazed all year round, leaving no space for regeneration” ([Birch, 2008, p. 4](zotero://select/groups/4773535/items/X92LGIEA)) ([pdf](zotero://open-pdf/groups/4773535/items/ZWD3FAK4?page=6&annotation=FYWWUNR9))


% I guess better leave this out and put it in the discussion section (?) 
“Using Remote Sensing to Map and Monitor Water Resources in Arid and Semiarid Regions” ([Klemas and Pieterse, 2015, p. 33](zotero://select/groups/4773535/items/BVN6IXG5)) ([pdf](zotero://open-pdf/groups/4773535/items/UPSYZXDK?page=1&annotation=4DPZD4BZ))

% Establishing an operational waterhole monitoring system using satellite data and hydrologic modelling: Application in the pastoral regions of East Africa
https://earlywarning.usgs.gov/docs/Senay-et-al-Pastoralism-Research-Policy-and-Practice-2013.pdf


“Global Monitoring of Water Supply and Sanitation: History, Methods and Future Challenges” ([Bartram et al., 2014, p. 8137](zotero://select/groups/4773535/items/6AWUJTW5)) ([pdf](zotero://open-pdf/groups/4773535/items/BFNSQGWS?page=1&annotation=ZWSBJVDM))

generally: one should look out for eurocentric views and perceptions / attitudes
Afrotopia

%%%%%%%%%%%%%%% CBS important papers: 2 systematic reviews
“Community-based surveillance of infectious diseases: a systematic review of drivers of success” ([McGowan et al., 2022, p. 1](zotero://select/groups/4773535/items/P74WDM6C)) ([pdf](zotero://open-pdf/groups/4773535/items/NP79EIIE?page=1&annotation=XJRZ7YIR))
“The Landscape of Participatory Surveillance Systems Across the One Health Spectrum: Systematic Review” ([McNeil et al., 2022, p. 1](zotero://select/groups/4773535/items/LVLYX8N5)) ([pdf](zotero://open-pdf/groups/4773535/items/4YG35TC6?page=1&annotation=IMCGBJLB))

-> "map closely to principles of participatory community engagement" (McGowan 2022 p.1)
+ “Other factors included: strong supervision and training, a strong sense of responsibility for community health, effective engagement of community informants, close proximity of surveillance workers to communities, the use of simple and adaptable case definitions, quality assurance, effective use of technology, and the use of data for real-time decision-making.” ([McGowan et al., 2022, p. 1](zotero://select/groups/4773535/items/P74WDM6C)) ([pdf](zotero://open-pdf/groups/4773535/items/NP79EIIE?page=1&annotation=FLMKXLLM))
-> tech

% compare CS with sensor networks
“Moving beyond the Technology: A Socio-technical Roadmap for Low-Cost Water Sensor Network Applications” ([Mao et al., 2020, p. 9145](zotero://select/groups/4773535/items/VEJUB8D9)) ([pdf](zotero://open-pdf/groups/4773535/items/KMCIFXB8?page=1&annotation=2A7QA3V3))


\subsection{Subsection 2}

application design 
“Citizen Science experiments have demonstrated that at least three issues must be addressed: commitment of people participating in data collection (were social component is key), the validation of data collected and sent by citizens and privacy concerns about the collection of personal information.” ([Alfonso and Jonoski, 2012, p. 6](zotero://select/groups/4773535/items/4W4Q8E6B)) ([pdf](zotero://open-pdf/groups/4773535/items/PU944FGI?page=6\&annotation=LZXCJDFB))

“Households’ perceptions of their drinking water quality were mostly influenced by the water’s visual appearance, and these perceptions in general motivated their use of HWT.” ([Daniel et al., 2020, p. 1](zotero://select/groups/4773535/items/ZA5MKHPC)) ([pdf](zotero://open-pdf/groups/4773535/items/CS4A7XIN?page=1&annotation=KHJHJFPM))

“The findings confirm other studies on technology acceptance, as the mobile crowd sensing technology should be useful rather than free of effort.” ([Minkman, 2015, p. 12](zotero://select/groups/4773535/items/ZKLE6CPT)) ([pdf](zotero://open-pdf/groups/4773535/items/QMAPCSZG?page=12&annotation=IQEBUPYG))



“For external experts, they may also benefit from community support to inform scientific processes, such as collecting data that spans across a large geographic region and having an enhanced understanding of community interests.” ([Huang et al., 2020, p. 144](zotero://select/groups/4773535/items/9CSBLJNJ)) ([pdf](zotero://open-pdf/groups/4773535/items/G5BEZQ7C?page=9&annotation=QPIKDKB9))
“Management of Drinking Water Source in Rural Communities under Climate Change” ([Huang et al., 2020, p. 136](zotero://select/groups/4773535/items/9CSBLJNJ)) ([pdf](zotero://open-pdf/groups/4773535/items/G5BEZQ7C?page=1&annotation=92FTSBXQ))
“For local communities, their needs of safe drinking water could be met and their abilities to manage and maintain water supply could be enhanced.” ([Huang et al., 2020, p. 144](zotero://select/groups/4773535/items/9CSBLJNJ)) ([pdf](zotero://open-pdf/groups/4773535/items/G5BEZQ7C?page=9&annotation=AL2NB5DK))
“Furthermore, this model could help increase scientific awareness among community members and engage the community with the environment.” ([Huang et al., 2020, p. 144](zotero://select/groups/4773535/items/9CSBLJNJ)) ([pdf](zotero://open-pdf/groups/4773535/items/G5BEZQ7C?page=9&annotation=FYU4BIKP))
“During the management of rural drinking water sources, a hybrid modality in which community management is the mainstay with supplement from external support from other organizations is highly recommended.” ([Huang et al., 2020, p. 147](zotero://select/groups/4773535/items/9CSBLJNJ)) ([pdf](zotero://open-pdf/groups/4773535/items/G5BEZQ7C?page=12&annotation=SQ6P8UBN)) % may also be a good transition to CBWM
% --> water related stuff --> subject is generally well researched and tried out in other circumstances and contexts




% --> further comparison though possibly also good in the discussion section (?)
“When Are Mobile Phones Useful for Water Quality Data Collection? An Analysis of Data Flows and ICT Applications among Regulated Monitoring Institutions in Sub-Saharan Africa” ([Kumpel et al., 2015, p. 10846](zotero://select/groups/4773535/items/GPM4C7RJ)) ([pdf](zotero://open-pdf/groups/4773535/items/7VXVKEXK?page=1&annotation=M5J4FFSH))


%----------------------------------------------------------------------------------------
%	SECTION 2
%----------------------------------------------------------------------------------------

\section{Main Section 2}

There is a need for research exploring if marginalized perspectives are excluded in crowdsourcing and self-reporting approaches, the recall bias and measurement error in self-reporting (Bell et al. 2019), and if incentives, memory triggers, or other mechanisms can be used to address these issues.


local knowledge
gender inequalities

etc. not really mentioned but have huge potential to improve all of this. Big question is HOW!
%----------------------------------------------------------------------------------------
%	SECTION 2 Future outlook
%----------------------------------------------------------------------------------------

“5.1.3. Challenges and Future Directions” ([Zheng et al., 2018, p. 721](zotero://select/groups/4773535/items/LJU68CG4)) ([pdf](zotero://open-pdf/groups/4773535/items/U8QNZLI6?page=24&annotation=D54SWR46))

“Connecting top-down weather and climate data with bottom-up socioeconomic data via machine learning” ([Enenkel et al., 2020, p. 1166](zotero://select/groups/4773535/items/RX575C79)) ([pdf](zotero://open-pdf/groups/4773535/items/XD499UNK?page=6&annotation=BCGJNKZB))

% thus nice in theory, but not useful in practice
% in regard to drought and water scarcity
% and while there is an extensive body of literature about these topics, the minor details are not of great interest to this work but the general conclusion, that physical, large scale drought or water scarcity indicators do not capture the required level of detail and impact that is needed to operationally act upon. Also, while the complexity of these concepts is due to the level of complexity of the surveyed phenomenon, its application and comparison is hindered. Thus a method to assess local impact, that builds and incorporates these concepts in a practically applicable manner is needed to adequately address this detrimental topic. 

% self surveying is generally not recommended:
“Indeed, collecting data on local indicators would require from the national society a team of enumerators that work continually to collect and process that information in all places where the program could possibly trigger (e.g. collect food price information for every village market). This would have extensive cost implications and likely over-burden the national society staff and volunteers.” ([RCRC, 2020, p. 30](zotero://select/groups/4773535/items/UESIQTRJ)) ([pdf](zotero://open-pdf/groups/4773535/items/P5JPVZ97?page=30&annotation=2YIIK6ZY))

“As such, the inclusion of local indicators into an FbA trigger must involve assessing what indicators are relevant for the impacts the program is trying to anticipate and identify which of those indicators are already collected (e.g. the ministry of agriculture's food price bulletin) and are available at the time they would be needed to inform a possible trigger.” ([RCRC, 2020, p. 30](zotero://select/groups/4773535/items/UESIQTRJ)) ([pdf](zotero://open-pdf/groups/4773535/items/P5JPVZ97?page=30&annotation=7X3RFGVB))


% --> thus - monthly review of users is necessary
“In a region where migration is one of the main coping mechanisms for drought, a targeted survey focusing on the early detection of migration movements would help mobilize the timely allocation of resources by humanitarian decision-makers or even the mitigation of drought impacts.” ([Enenkel et al., 2020, p. 1167](zotero://select/groups/4773535/items/RX575C79)) ([pdf](zotero://open-pdf/groups/4773535/items/XD499UNK?page=7&annotation=N9FRDA9C))



% keeping data up to date is crucial in ensuring correct vulnerability and exposure data 
“Vulnerability and exposure changes over time, particularly after an extreme weather or climate event. Datasets must be kept up to date to ensure the impact-based forecast or warning using this data is reliable. Recognise that many official governmental data sources, such as a national census or demographic and health surveys, are updated infrequently – every five or ten years.” ([Harrowsmith et al., 2020, p. 28](zotero://select/groups/4773535/items/QJ397Y54)) ([pdf](zotero://open-pdf/groups/4773535/items/2GS362N5?page=28&annotation=5XVAQCTY))





% VULNERABILITY
%Understanding vulnerability and resilience in Somalia
https://www.ncbi.nlm.nih.gov/pmc/articles/PMC7768599/pdf/JAMBA-12-856.pdf

“Without scope to accommodate dynamic vulnerabilities, actions cannot be effectively targeted or may prove ineffective.” ([Boult et al., 2022, p. 4](zotero://select/groups/4773535/items/B2AQSTYL)) ([pdf](zotero://open-pdf/groups/4773535/items/W9TFLH43?page=4&annotation=YYURM2E3))




\subsection{Limitations}
“Citizen science has several limitations, including the wide range of required skills outside the research subject, sustaining engagement, biases related to data collection and analysis, sensor calibration issues and varying data privacy regulations around the world, among others.” ([Fraisl et al., 2022, p. 13](zotero://select/groups/4773535/items/FBJD7SWS)) ([pdf](zotero://open-pdf/groups/4773535/items/7WBDKYDY?page=13&annotation=WNQZRHRF))
