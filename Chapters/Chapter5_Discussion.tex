% Chapter Template

\chapter{Discussion} % Main chapter title

\label{Chapter5} % Change X to a consecutive number; for referencing this chapter elsewhere, use \ref{ChapterX}

%----------------------------------------------------------------------------------------
%	SECTION 1
%----------------------------------------------------------------------------------------

\section{Main Section 1}

%-----------------------------------
%	SUBSECTION 1
%-----------------------------------
\subsection{Subsection 1}
application design 
“Citizen Science experiments have demonstrated that at least three issues must be addressed: commitment of people participating in data collection (were social component is key), the validation of data collected and sent by citizens and privacy concerns about the collection of personal information.” ([Alfonso and Jonoski, 2012, p. 6](zotero://select/groups/4773535/items/4W4Q8E6B)) ([pdf](zotero://open-pdf/groups/4773535/items/PU944FGI?page=6\&annotation=LZXCJDFB))

“Households’ perceptions of their drinking water quality were mostly influenced by the water’s visual appearance, and these perceptions in general motivated their use of HWT.” ([Daniel et al., 2020, p. 1](zotero://select/groups/4773535/items/ZA5MKHPC)) ([pdf](zotero://open-pdf/groups/4773535/items/CS4A7XIN?page=1&annotation=KHJHJFPM))


research vs. reality

other perspectives than process oriented -> 7-layer model self-critisism (resource, behavioral perspectives, + value and communication network design approaches)
%-----------------------------------
%	SUBSECTION 2
%-----------------------------------

\subsection{Subsection 2}

%----------------------------------------------------------------------------------------
%	SECTION 2
%----------------------------------------------------------------------------------------

\section{Main Section 2}

There is a need for research exploring if marginalized perspectives are excluded in crowdsourcing and self-reporting approaches, the recall bias and measurement error in self-reporting (Bell et al. 2019), and if incentives, memory triggers, or other mechanisms can be used to address these issues.



%----------------------------------------------------------------------------------------
%	SECTION 2 Future outlook
%----------------------------------------------------------------------------------------

“Connecting top-down weather and climate data with bottom-up socioeconomic data via machine learning” ([Enenkel et al., 2020, p. 1166](zotero://select/groups/4773535/items/RX575C79)) ([pdf](zotero://open-pdf/groups/4773535/items/XD499UNK?page=6&annotation=BCGJNKZB))

% thus nice in theory, but not useful in practice
% in regard to drought and water scarcity
% and while there is an extensive body of literature about these topics, the minor details are not of great interest to this work but the general conclusion, that physical, large scale drought or water scarcity indicators do not capture the required level of detail and impact that is needed to operationally act upon. Also, while the complexity of these concepts is due to the level of complexity of the surveyed phenomenon, its application and comparison is hindered. Thus a method to assess local impact, that builds and incorporates these concepts in a practically applicable manner is needed to adequately address this detrimental topic. 