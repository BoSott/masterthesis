% Chapter Template

\chapter{Discussion} % Main chapter title

\label{Chapter5} % Change X to a consecutive number; for referencing this chapter elsewhere, use \ref{ChapterX}

%----------------------------------------------------------------------------------------
%	SECTION 1
%----------------------------------------------------------------------------------------

\section{Main Section 1}

%-----------------------------------
%	SUBSECTION 1
%-----------------------------------
\subsection{Discussion indices and forecasts}

localization of forecasts and so on.. why it is difficult to use SPEI, SPI and so on and others.. --> more data to operationalize is very important, even though it should not be the task of the SRCS --> high costs but a lot of trust in the communities and CBS is also working great.
drought indicator (what exists so far? -> pre study)

% drought monitoring tool for somalia
https://cdi.faoswalim.org/index/rfe-maps/2022

% Drought Assessment and Monitoring using some drought indicators in the semi-arid Puntland state of Somalia
https://d1wqtxts1xzle7.cloudfront.net/61927859/DroughtAssessmentandMonitoringUsingSomeDroughtIndicatorsintheSemi-AridPuntlandStateofSomalia_FEB_2019NovemberIssue11A20200129-7600-1cwi1.pdf?1580293969=&response-content-disposition=inline%3B+filename%3DDROUGHT_ASSESSMENT_AND_MONITORING_USING.pdf&Expires=1678353893&Signature=U3ArqyQZMaz2Pbm6iwGv-0qMl0meKe6028igIkC5qjjTYaQ4OtNvKRiAKBzGNVWYCOBNxMmOtgriFg7Mavekgov3BRt38tG-e7tD86OYU4qS0zQfshEmQxBD9tM~CqqO45XBASpGpuQFl5Atks6ADcjQef03Cds7bRoTCrwyz-rOJvHo3nyHk~lbfupqWrgYzZ6oE~YTmUC6iQ7xNtKf-XNcjW4z6R7Rtvu1NXvoX~YWLFRrMa50O3kTUISwEEsD-Rw7QiVNPF~VBOa2~PepbkueXRRGuCRolvb1q95DQVqHXQkaS6jG1UUvJ-vrmqH0wAX-zmo-r31yuGQLbkQ5UQ__&Key-Pair-Id=APKAJLOHF5GGSLRBV4ZA

but: 
“Scale is critical in assessing water security [31]. National level assessments make it difficult to take action at operationalization level.” ([Mishra et al., 2021, p. 8](zotero://select/groups/4773535/items/MD2Z2HTF)) ([pdf](zotero://open-pdf/groups/4773535/items/366Z36U7?page=8&annotation=6IAXCXUB))

and: “Creating and using indicators for water security has to be directed towards some management control or assessment action.” ([Mishra et al., 2021, p. 8](zotero://select/groups/4773535/items/MD2Z2HTF)) ([pdf](zotero://open-pdf/groups/4773535/items/366Z36U7?page=8&annotation=P72LT9Y8))


% maybeeeeee and maybe not.. tend not to include it
“2 Local knowledge in drought monitoring: an introduction to the literature review” ([Giordano et al., 2013, p. 526](zotero://select/groups/4773535/items/B7LM5ZR4)) ([pdf](zotero://open-pdf/groups/4773535/items/7I66DBIK?page=4&annotation=Z33M5FLQ))

“There has been little effort to align the spatiotemporal granularity of socioeconomic assessments with the granularity of weather or climate monitoring.” ([Enenkel et al., 2020, p. 1161](zotero://select/groups/4773535/items/RX575C79)) ([pdf](zotero://open-pdf/groups/4773535/items/XD499UNK?page=1&annotation=QBTLFCXM))
“we highlight the need to collect and analyze environmental and socioeconomic data together and discuss novel strategies for coordinated data collection via mobile technologies from a drought risk management perspective.” ([Enenkel et al., 2020, p. 1161](zotero://select/groups/4773535/items/RX575C79)) ([pdf](zotero://open-pdf/groups/4773535/items/XD499UNK?page=1&annotation=9BUBHWNB))

“but questions related to coping capacities, migration, poverty, water supply, access to food and markets, or political conflict remain unanswered or are even decoupled from routine drought risk assessments” ([Enenkel et al., 2020, p. 1162](zotero://select/groups/4773535/items/RX575C79)) ([pdf](zotero://open-pdf/groups/4773535/items/XD499UNK?page=2&annotation=HE48ZWFA))

“The handbook of drought indicators (Svoboda et al. 2016) lists more than 50 drought indices. Not a single one of these indices connects climate anomalies to socioeconomic vulnerabilities,” ([Enenkel et al., 2020, p. 1163](zotero://select/groups/4773535/items/RX575C79)) ([pdf](zotero://open-pdf/groups/4773535/items/XD499UNK?page=3&annotation=EDKZFHJX))
--> there were more indicators in the following year, but those were limited due to availibiltiy of socioeconomic data


% List of drought indicators
https://heigit.atlassian.net/wiki/spaces/FIS/pages/1704096/Indices


% e.g. SPEI has difficulties because of the scarce hydro-meteorological monitoring network
%SPEI-based spatial and temporal evaluation of drought in Somalia
http://www.repository.embuni.ac.ke/bitstream/handle/embuni/3705/SPEI-based%20spatial%20and%20temporal%20evaluation%20of%20drought%20in%20Somalia.pdf?sequence=1&isAllowed=y
% notes from the team
evaluate the spatio-temporal variations of drought occurrences in Somalia for the period between 1980 and 2015 as quantified by Standardized Precipitation Evapotranspiration Index (SPEI)
The temporal variations in drought showed decreasing trends in severity and increasing trends in drought duration as the SPEI timescales increas
However, the study was limited by the low security levels in Somalia that makes
the country largely inaccessible. This has complicated possibilities for
ground-truthing available information. Besides, the 1991 civil war in
Somalia led to the collapse of hydro-meteorological monitoring network
(Muchiri, 2007). The current stations established by Food and Agricul-
ture Organization Somalia Water and Land Information Management
(FAO SWALIM) from 2002 do not have uniform spatial distribution
throughout the country and therefore cannot be useful in computing
reference SPE

% this level of impact can then.. 
one can access the impact of a certain weather phenomenon or climate development on the situation on site. These information can then facilitate exact and precise assessments of the conditions and help to react accordingly (policy, management, etc. ) 
% possibly too much of a discussion.. 


% later on
However, it requires complex and various information and is thus often difficult to implement \autocite{liuWaterScarcityAssessments2017}.

... in the light of
While all of these more dedicated indicators may present great value, relatively simple Indices such as the SPEI is not possible for Somalia due to the lack of a good enough weather monitoring network. 

% this level of impact can then.. 
one can access the impact of a certain weather phenomenon or climate development on the situation on site. These information can then facilitate exact and precise assessments of the conditions and help to react accordingly (policy, management, etc. ) 
% possibly too much of a discussion.. 


... in the light of
While all of these more dedicated indicators may present great value, relatively simple Indices such as the SPEI is not possible for Somalia due to the lack of a good enough weather monitoring network. 



research vs. reality

other perspectives than process oriented -> 7-layer model self-critisism (resource, behavioral perspectives, + value and communication network design approaches)
%-----------------------------------
%	SUBSECTION 2
%-----------------------------------

\subsection{FbF and EAP}



\subsection{CBS and Tools}


\subsection{Own results and how the relate to all of this shizzle}


\subsection{Subsection 2}


\subsection{Subsection 2}


\subsection{Subsection 2}

application design 
“Citizen Science experiments have demonstrated that at least three issues must be addressed: commitment of people participating in data collection (were social component is key), the validation of data collected and sent by citizens and privacy concerns about the collection of personal information.” ([Alfonso and Jonoski, 2012, p. 6](zotero://select/groups/4773535/items/4W4Q8E6B)) ([pdf](zotero://open-pdf/groups/4773535/items/PU944FGI?page=6\&annotation=LZXCJDFB))

“Households’ perceptions of their drinking water quality were mostly influenced by the water’s visual appearance, and these perceptions in general motivated their use of HWT.” ([Daniel et al., 2020, p. 1](zotero://select/groups/4773535/items/ZA5MKHPC)) ([pdf](zotero://open-pdf/groups/4773535/items/CS4A7XIN?page=1&annotation=KHJHJFPM))


%----------------------------------------------------------------------------------------
%	SECTION 2
%----------------------------------------------------------------------------------------

\section{Main Section 2}

There is a need for research exploring if marginalized perspectives are excluded in crowdsourcing and self-reporting approaches, the recall bias and measurement error in self-reporting (Bell et al. 2019), and if incentives, memory triggers, or other mechanisms can be used to address these issues.


local knowledge
gender inequalities

etc. not really mentioned but have huge potential to improve all of this. Big question is HOW!
%----------------------------------------------------------------------------------------
%	SECTION 2 Future outlook
%----------------------------------------------------------------------------------------

“Connecting top-down weather and climate data with bottom-up socioeconomic data via machine learning” ([Enenkel et al., 2020, p. 1166](zotero://select/groups/4773535/items/RX575C79)) ([pdf](zotero://open-pdf/groups/4773535/items/XD499UNK?page=6&annotation=BCGJNKZB))

% thus nice in theory, but not useful in practice
% in regard to drought and water scarcity
% and while there is an extensive body of literature about these topics, the minor details are not of great interest to this work but the general conclusion, that physical, large scale drought or water scarcity indicators do not capture the required level of detail and impact that is needed to operationally act upon. Also, while the complexity of these concepts is due to the level of complexity of the surveyed phenomenon, its application and comparison is hindered. Thus a method to assess local impact, that builds and incorporates these concepts in a practically applicable manner is needed to adequately address this detrimental topic. 

% self surveying is generally not recommended:
“Indeed, collecting data on local indicators would require from the national society a team of enumerators that work continually to collect and process that information in all places where the program could possibly trigger (e.g. collect food price information for every village market). This would have extensive cost implications and likely over-burden the national society staff and volunteers.” ([RCRC, 2020, p. 30](zotero://select/groups/4773535/items/UESIQTRJ)) ([pdf](zotero://open-pdf/groups/4773535/items/P5JPVZ97?page=30&annotation=2YIIK6ZY))

“As such, the inclusion of local indicators into an FbA trigger must involve assessing what indicators are relevant for the impacts the program is trying to anticipate and identify which of those indicators are already collected (e.g. the ministry of agriculture's food price bulletin) and are available at the time they would be needed to inform a possible trigger.” ([RCRC, 2020, p. 30](zotero://select/groups/4773535/items/UESIQTRJ)) ([pdf](zotero://open-pdf/groups/4773535/items/P5JPVZ97?page=30&annotation=7X3RFGVB))


% --> thus - monthly review of users is necessary
“In a region where migration is one of the main coping mechanisms for drought, a targeted survey focusing on the early detection of migration movements would help mobilize the timely allocation of resources by humanitarian decision-makers or even the mitigation of drought impacts.” ([Enenkel et al., 2020, p. 1167](zotero://select/groups/4773535/items/RX575C79)) ([pdf](zotero://open-pdf/groups/4773535/items/XD499UNK?page=7&annotation=N9FRDA9C))



% keeping data up to date is crucial in ensuring correct vulnerability and exposure data 
“Vulnerability and exposure changes over time, particularly after an extreme weather or climate event. Datasets must be kept up to date to ensure the impact-based forecast or warning using this data is reliable. Recognise that many official governmental data sources, such as a national census or demographic and health surveys, are updated infrequently – every five or ten years.” ([Harrowsmith et al., 2020, p. 28](zotero://select/groups/4773535/items/QJ397Y54)) ([pdf](zotero://open-pdf/groups/4773535/items/2GS362N5?page=28&annotation=5XVAQCTY))





% VULNERABILITY
%Understanding vulnerability and resilience in Somalia
https://www.ncbi.nlm.nih.gov/pmc/articles/PMC7768599/pdf/JAMBA-12-856.pdf

“Without scope to accommodate dynamic vulnerabilities, actions cannot be effectively targeted or may prove ineffective.” ([Boult et al., 2022, p. 4](zotero://select/groups/4773535/items/B2AQSTYL)) ([pdf](zotero://open-pdf/groups/4773535/items/W9TFLH43?page=4&annotation=YYURM2E3))