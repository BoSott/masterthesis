% Chapter Template

\chapter{Results} % Main chapter title

\label{chapter4}


% "The first step is to craft a brief introduction to the chapter. This intro is vital as it provides some context for your findings. In your introduction, you should begin by reiterating your problem statement and research questions and highlight the purpose of your research. Make sure that you spell this out for the reader so that the rest of your chapter is well contextualised.
% The next step is to briefly outline the structure of your results chapter. In other words, explain what’s included in the chapter and what the reader can expect. In the results chapter, you want to tell a story that is coherent, flows logically, and is easy to follow, so make sure that you plan your structure out well and convey that structure (at a high level), so that your reader is well oriented." 
%----------------------------------------------------------------------------------------
%	SECTION 1
%----------------------------------------------------------------------------------------

The first aim of this thesis was to answer a deductive hypothesis, namely whether CBS and MCS are potentially viable approaches to answer the primary inductive research question of this thesis. The main aim was to create a roadmap for the development of a community-based participatory monitoring approach using simple transmission protocols such as SMS in the context of developing an EAP in a resource-limited setting. While the first aim could already be partially addressed in the preceding chapters (see \ref*{subsec:practical_examples} and \ref*{subsec:water_sources}), the hypothesis is further supported by the results presented below, particularly in Stages 1 and 2.
The stages follow the SSF (see chapter \ref*{subsec:ssf}) and are presented in the coming sections. Stages 1 and 2 are concerned with the context, problem definition and feasibility assessment. Stage 3 \textit{Designing the Project} incorporates the \acrshort*{slmc} and is the central element of this chapter by presenting the roadmap in detail. Stage 4 to 6 contain brief descriptions of the SRCS's community building and volunteer recruitment, data management practices with an emphasis on NYSS and the evaluation procedures which are currently in place, respectively. The results are based on the conducted interviews, the project analysis and the literature analysis. The full interview and questionnaire transcripts are presented in Appendix \ref*{TODO:} and the protocol to the data analysis in Appendix \ref*{TODO:}. The chapter is concluded with a summary of the key findings.

Introduction

grouped into 6 stages

only the first 3 are really considered
4 -> not really necessary
5 -> NYSS and others -> just went too  far
6 -> Evaluation - nothing to evaluate really --> evaluation of this process -> discussion



%%%%%%%%%%%%%%%%%%%%%%%%%%%%%%%%%%%%%%%%%%%%%%%%%%%%%%%%%%%%%%%%%%%%%%%%%%%%%%%%%%%%%%%%%%%%%%%%%%%
%%%%%%%%%%%%%%%%%%%%%%%%%%%%%%%%%%%%%%%%%%%%%%%%%%%%%%%%%%%%%%%%%%%%%%%%%%%%%%%%%%%%%%%%%%%%%%%%%%%
%%%%%%%%%%%%%%%%%%%%%%%%%%%%%%%%%%%%%%% !!! SECTION 1 !!! %%%%%%%%%%%%%%%%%%%%%%%%%%%%%%%%%%%%%%%%%
%%%%%%%%%%%%%%%%%%%%%%%%%%%%%%%%%%%%%%%%%%%%%%%%%%%%%%%%%%%%%%%%%%%%%%%%%%%%%%%%%%%%%%%%%%%%%%%%%%%
%%%%%%%%%%%%%%%%%%%%%%%%%%%%%%%%%%%%%%%%%%%%%%%%%%%%%%%%%%%%%%%%%%%%%%%%%%%%%%%%%%%%%%%%%%%%%%%%%%%
%%%%%%%%%%%%%%%%%%%%%%%%%%%%%%%%%%%%%%%%%%%%%%%%%%%%%%%%%%%%%%%%%%%%%%%%%%%%%%%%%%%%%%%%%%%%%%%%%%%
\section{Stage 1: Context and Problem identification}

The brief water source data analysis along with given context, resource restriction, stakeholders and comparable projects as well as problem and goal statements of the interviewees are covered in this first stage. It also builds on the preceding case study area section \ref{sec:case_area} in chapter \ref{chapter2}.\newline
% current situation
The current drought and water scarcity situation in Somaliland has now lasted five years and has greatly impacted the water sector in Somaliland in terms of quantity and quality (I1). I1 describes the current crisis as \textit{"huge and response activities are being overwhelmed by the need"} which will lead to \textit{"commercialization and overpricing of fresh water"} further exacerbating the situation. This is underlined by I3 who describes the current water situation of the rural population as \textit{"whatever source of water they can find is what they have"}. I3 further mentions, that the people sometimes \textit{won't have enough water to wash their hands"} or for other necessary things and that \textit{"they [then] don't think of what kind of water they can get, whether it's bad or something like that"} but only focus on having at least something to drink. Increased water shortage because of bad quality was also reported in the literature (see section \ref*{subsec:water_quality}). Water can potentially be contaminated at all stages of the water collection process, from the initial abstraction of water, through transport and storage, to the use of the water (I3). Water quality is difficult to assess on site, as the the colour is not a good indicator and other parameters can only be determined technically which required equipment and training (I1, I3). Furthermore, I3 confirms the reports in literature that contamination of water in berkads, through their shared use with animals, can happen even before water withdrawal. This depends very much on the construction method and how it is used. The rehabilitation as well as training how to adequately use a berkad are already activities of the \acrshort*{srcs} (I3) and besides the water quality, the quantity also depends on the kind of construction and amount of withdrawal. According to I3, supply period can range from one month to half a year, so information on these parameters is crucial for estimating the potential duration of supply (I3).\newline
Water monitoring and management is not a problem in urban areas as there is an agency responsible for water supply but the problem is primarily in rural and nomad areas, where 70\,\% of the people live, according to I3. \autocite{republicofsomaliaCountryProfile20212021}, on the other hand, estimates an urban and semi-urban, sedentary population rate of 53\,\%.\newline
The selection of beneficiaries for response activities of the \acrshort*{srcs} are currently conducted on the basis of a preceding joint priority setting with the government. This prioritisation is based on \textit{"assumed vulnerability per community based on Number of \acrfull{idp} camps in the area, number of women headed families, predicted IPC classifications etc."} (I1.2). \acrfullpl{aa} have not been implemented due to the "already prevailing crisis" where the "needs are [already] dire and the current \acrshort*{srcs}s focus is on response mechanisms to address the already visible impacts of drought" (I1.2). Additionally, "there has been any actions yet due to the fact that there is no water monitoring and trigger mechanism in place" (I1).The monitoring is itself hampered by the fact that \textit{"Berkads location data is currently missing"} (I1). This statement compares well with the experience of the current project team working on the EAP implementation and the assessment of available data sets (A. Schulze-Eckel \& A. Schauss, personal communication, November 2022).\newline
Table \ref*{TODO} shows all available data sets about water sources in Somaliland, provided by \acrshort*{swalim}. % and OSM (?)

\missingfigure{table swalim water sources} % see joplin
% maybe add OSM data as well.. quite some work though as not one single tag can be used.. could be a good comparison though. Only one data provider is a bit mehh.. and no OSM is hard to explain and I got the python scripts already.. need to refine the methodology though. Should be worth it. + add settlements to it

The spatial distribution of the datasets across Somalia is relatively balanced, with focal points in the regions with many or larger settlements. Based on the SWALIM settlement data set, there are currently 2123 settlements in Somaliland. These settlement data are mostly from the years 2002 and 2006. The total number of water sources varies widely between and over time of the other data sets. The timeliness of the data also has a wide range, from relatively few pieces of data from 2019 in the 2020 dataset to data from the 1980s in the same dataset is much represented. The 2022 dataset misses information about timeliness altogether and the other datasets all have many blank entries as well. Furthermore, many water sources are labelled as 'abandoned' or 'non-functional', e.g. in the 2022 dataset those are 147 out of 685.\newline
The data sets are fed by many sources and actors e.g. FAOSWALIM or other UN organisation, \acrshort*{mowr}, \acrshort*{nadfor} and other NGOs which constructed some water sources in some communities (I1). These actors, along with the community and their elders, local government representatives, SRCS and their volunteers and private berkad owners are also the potential stakeholders of the mapping and monitoring of berkads. Here, I1 notes, that besides the \acrshort*{srcs}, the \acrshort*{mowr}, \acrshort*{nadfor} and the constructing NGOs are the most important stakeholders. The \acrshort*{mowr} and NGOs have the technical expertise in construction, rehabilitation and monitoring and \acrshort*{nadfor} has a comparable community level programme for monitoring \textit{"livestock body condition, market prices as well as weather variables"} (I1). 
Other comparable programmes exist from \acrshort*{ocha}, \acrshort*{brcis} and the \acrshort*{cbs} programme run by the Ministry of Health and the \acrshort*{nrc}. While these projects may broadly be comparable, none of the Interviewees know of a project that conducts similar things to this works approach (I1, I2, I3). I2 also suggests that the projects are close enough to each other to pass on experiences and recommendations, e.g. from the MoH to the MoRW, in order to overcome initial scepticism and reluctance.\newline
Challenges, limitations and requirements are mentioned in areas of privately owned berkads, community expectation handling and the dissolution of misconceptions as well as potentially already overstretched SRCS staff and volunteers (I1). I1 mentioned, that private owners of berkads may prevent the volunteer form gaining access to their berkad which would result in less information but also in tension in the community. Giving information from the community to someone else may also generally require some explanation (I2). Furthermore, some \textit{"information on past details per particular geographical areas"}(I1.2) can be difficult to access, as \textit{"Somalis are highly mobile communities"} (I1.2). The monitoring could furthermore develop \textit{"hugh expectations from the communities as there is the ongoing drought. Whenever there is monitoring of resources communities believe this should be followed up by instant aid"} (I1). 
Addressing some of the challenges mentioned above, the \textit{"community elders should be engaged before the start of the mapping and monitoring as they will help dispel misconceptions about the project"} (I1) and the \textit{"ministry of water resources should be in the loop during the entire project duration"} (I1). Nonetheless, the \textit{"community and SRCS goals match as both focus on closing the knowledge [gap] currently existing"} (I1) in regard to the number, status of ownership, location and capacity of the berkads per community, district and regional level. This will \textit{"inform decision makers on the priority areas to focus on"} (I1). Therefore, I1 expects that this information from the site triangulated with weather forecasts can help to form robust triggers to take appropriate and informed \acrlongpl*{aa} before critical water levels are reached.









%%%%%%%%%%%%%%%%%%%%%%%%%%%%%%%%%%%%%%%%%%%%%%%%%%%%%%%%%%%%%%%%%%%%%%%%%%%%%%%%%%%%%%%%%%%%%%%%%%%
%%%%%%%%%%%%%%%%%%%%%%%%%%%%%%%%%%%%%%%%%%%%%%%%%%%%%%%%%%%%%%%%%%%%%%%%%%%%%%%%%%%%%%%%%%%%%%%%%%%
%%%%%%%%%%%%%%%%%%%%%%%%%%%%%%%%%%%%%%% !!! SECTION 2 !!! %%%%%%%%%%%%%%%%%%%%%%%%%%%%%%%%%%%%%%%%%
%%%%%%%%%%%%%%%%%%%%%%%%%%%%%%%%%%%%%%%%%%%%%%%%%%%%%%%%%%%%%%%%%%%%%%%%%%%%%%%%%%%%%%%%%%%%%%%%%%%
%%%%%%%%%%%%%%%%%%%%%%%%%%%%%%%%%%%%%%%%%%%%%%%%%%%%%%%%%%%%%%%%%%%%%%%%%%%%%%%%%%%%%%%%%%%%%%%%%%%
%%%%%%%%%%%%%%%%%%%%%%%%%%%%%%%%%%%%%%%%%%%%%%%%%%%%%%%%%%%%%%%%%%%%%%%%%%%%%%%%%%%%%%%%%%%%%%%%%%%
\section{Stage 2: Determining if Citizen Science is the right Approach}

in stage 2 assessment of SRCS is capable of handling such a project under these circumstances.



--> NYSS worked
--> other project worked

-> CBS / NYSS (design phase: similar data collection formats ? -> reuse)
-> table water CBMs/MCSs
-> BRCiS and OCHA
this should also work --> meets all criteria 


% page 10852 about other forms of data collection -> not as automatically 
“When Are Mobile Phones Useful for Water Quality Data Collection? An Analysis of Data Flows and ICT Applications among Regulated Monitoring Institutions in Sub-Saharan Africa” ([Kumpel et al., 2015, p. 10846](zotero://select/groups/4773535/items/GPM4C7RJ)) ([pdf](zotero://open-pdf/groups/4773535/items/7VXVKEXK?page=1&annotation=M5J4FFSH))

also feasibility study of SRCS -> EAP works -> SCRS key points


\subsection{CBS and NYSS} (NYSS qustionnaire was already developed, but never used (see Appendix XYZ))
-> no data collection tool (!!) 
% I need to include some technical details and stuff somewhere.. or at least name the alternatives to NYSS.. could do that in the result part as Julia Jung talked about that stuff as well



%%%%%%%%%%%%%%%%%%%%%%%%%%%%%%%%%%% INTERVIEW I1 RICHARD %%%%%%%%%%%%%%%%%%%%%%%%%%%%%%%%%%%%%%%%%%





%%%%%%%%%%%%%%%%%%%%%%%%%%%%%%%%%%% INTERVIEW I1.2 RICHARD %%%%%%%%%%%%%%%%%%%%%%%%%%%%%%%%%%%%%%%%






%%%%%%%%%%%%%%%%%%%%%%%%%%%%%%%%%%% INTERVIEW I2 JUNG %%%%%%%%%%%%%%%%%%%%%%%%%%%%%%%%%%%%%%%%%%%%%
-	(75, 95) Acceptance and understanding of the MoH for the introduction of NYSS is often an issue. Also because a lot of organizations want to integrate their own tools and instruments – may be just too many.

(23) before CBS, there is always an assessment
(103) Julia Jung does not know of other organizations, which would conduct similar things. Except MSF is active in Somalia/Somaliland but focussed on direct health responses. Further information should be provided by the Ministry or latest by the community itself.
(7) Would fit well with the current thematic focus and technical expertise
(26) Weekly updates about the water level from the volunteer should work.
(28, 29) It should be communicated, that reports will be checked by the supervisor in order to prevent false reports in hope of more water. If this happens frequently, a solution must be conceptualized.
(103) Not recommended to implement things, when they already exist in some form, even if those are worse. 

(104) MoH has good experiences with NYSS in the health sector. These experiences should be translated and widened on other topics.
-	(106) The assessment of other projects and stakeholders should be thoroughly as it could otherwise lead to problems.

-	(109) SRCS invests a lot into communication and feedback with the communities.
-	(113) SRCS are no rookies. They know how to communicate – big part of their culture.


% I2 good experiences with the method are available
CBS
(1) Nothing new in itself, often used in health contexts
(2) Can be a problem to send information about community members – possibly also about water sources? 
(5) CBS does not generally have a negative connotation in Somalia, but could be (?) -> talk to MoH to translate experiences to MoWR
(104) MoH has good experiences with NYSS in the health sector. These experiences should be translated and widened on other topics.
(36) Monitoring water sources would work well with the CBS system and fit well with the overall theme of health risks.
-	(71) MoH in the loop right from the beginning to find out what else is going on, can often be combined with an interview for the assessment.
-	(72) MoH is constantly in the loop to decide together what, how, when, who.
-	(75, 95) Acceptance and understanding of the MoH for the introduction of NYSS is often an issue. Also because a lot of organizations want to integrate their own tools and instruments – may be just too many.
-	(97) MoH discussed a lot and still discusses a lot.
-	(98) Trust in the MoH is not high. That was one result of the evaluation.
-	(102) MoH is part of the ToT (Training of Trainers)
(103) Julia Jung does not know of other organizations, which would conduct similar things. Except MSF is active in Somalia/Somaliland but focussed on direct health responses. Further information should be provided by the Ministry or latest by the community itself.
(23) before CBS, there is always an assessment

Stakeholder > Ministry of Health
-	(71) MoH in the loop right from the beginning to find out what else is going on, can often be combined with an interview for the assessment.
-	(72) MoH is constantly in the loop to decide together what, how, when, who.
-	(75, 95) Acceptance and understanding of the MoH for the introduction of NYSS is often an issue. Also because a lot of organizations want to integrate their own tools and instruments – may be just too many.
-	(97) MoH discussed a lot and still discusses a lot.
-	(98) Trust in the MoH is not high. That was one result of the evaluation.
-	(25, 99) Might be beneficial to bring the MoH in contact with the MoWR in order to transfer their experiences with CBS to the new approach.
-	(102) MoH is part of the ToT (Training of Trainers)
%%%%%%%%%%%%%%%%%%%%%%%%%%%%%%%%%%% INTERVIEW I3 BELEDI %%%%%%%%%%%%%%%%%%%%%%%%%%%%%%%%%%%%%%%%%%%
(1) CBS started in 2018 in Burao with 75 community volunteers because of the 2017 cholera outbreak in the same region
-(5, 6, 7) Oral Rehydration Points are in the community and run by volunteers of the community who get trained and then support the community by education and promotion of e.g. WASH activities and going to the actual treatment centers.
-(8, 9) After the pilot was successful, they scaled CBS and are reaching now all 6 regions in Somaliland.
-(11, 14) Response happens together with other partners such as the government, MoH, WHO, and other sister RCRC organisations.
-(25) The idea to implement CBS came from the will to identify the cases in the community early enough to respond at community level and stop an outbreak immediately.”
NYSS
-(33) NYSS was developed over the course of a year with a lot of feedback from the SRCS and is now “very effective and very supportive.”
-(34) NYSS is highly automatized and alerts are given when thresholds are reached based on geographical location.
% NYSS Fortsetzung in Stage 5 Data Management




-(43) SRCS has a good connection to the local communities and thus they get direct feedback about water shortages.
-(8, 9) After the pilot was successful, they scaled CBS and are reaching now all 6 regions in Somaliland.
-(10) SRCS only focusses on hot spot areas where they expect outbreaks to happen. They do not cover all of the country.
-(11, 14) Response happens together with other partners such as the government, MoH, WHO, and other sister RCRC organisations.
-(13) Reports are verified by the regional officer.
-(15) “SRCS has mobile teams who can be deployed immediately within hours so they can do the response.”
-(76, 77) Thus far, information come from SRCS assessments, community themselves, government or FSNAU (Food Security and Nutrition Analysis Unit) (https://fsnau.org/)






generally FbF and EAP Feasibility --> Feasibility study -> yes, it is.

RCRC Assessment + Fraisl questions + Minkman and others

-> volunteers and community would benefit and scientists/practitioners as well




TODO:TODO:TODO:TODO:TODO:TODO:TODO:TODO:TODO:TODO:TODO:TODO:TODO:TODO:TODO:TODO:TODO:TODO:TODO:TODO:TODO:
%%%%%%%%%%%%%%%%%%%%%%%%%%%%%%%%%%%%%%%%%%%%%%%%%%%%%%%%%%%%%%%%%%%%%%%%%%%%%%%%%%%%%%%%%%%%%%%%%%%
%%%%%%%%%%%%%%%%%%%%%%%%%%%%%%%%%%%%%%%%%%%%%%%%%%%%%%%%%%%%%%%%%%%%%%%%%%%%%%%%%%%%%%%%%%%%%%%%%%%
%%%%%%%%%%%%%%%%%%%%%%%%%%%%%%%%%%%%%%% !!! SECTION 3 !!! %%%%%%%%%%%%%%%%%%%%%%%%%%%%%%%%%%%%%%%%%
%%%%%%%%%%%%%%%%%%%%%%%%%%%%%%%%%%%%%%%%%%%%%%%%%%%%%%%%%%%%%%%%%%%%%%%%%%%%%%%%%%%%%%%%%%%%%%%%%%%
%%%%%%%%%%%%%%%%%%%%%%%%%%%%%%%%%%%%%%%%%%%%%%%%%%%%%%%%%%%%%%%%%%%%%%%%%%%%%%%%%%%%%%%%%%%%%%%%%%%
%%%%%%%%%%%%%%%%%%%%%%%%%%%%%%%%%%%%%%%%%%%%%%%%%%%%%%%%%%%%%%%%%%%%%%%%%%%%%%%%%%%%%%%%%%%%%%%%%%%
\section{Stage 3: Designing the Project}
%%%%%%%%%%%%%%%%%%%%%%%%%%%%%%%%%%% INTERVIEW I1 RICHARD %%%%%%%%%%%%%%%%%%%%%%%%%%%%%%%%%%%%%%%%%%

- WANTED RESULTS
•	What results must be achieved in order to reach the goals? Please list them.
o	i) Location data of the berkads (coordinates),
o	ii)Volunteer orientation on water resources monitoring
o	iii) determining the ownership status of each berkad
o	iv) community sensitization to dispel misconceptions about the mapping and water monitoring exercise
o	v) water levels monitoring
o	vi) triggering action based on water levels
o	vii) Determining the water level trigger

- most critical ones:
o	Location data as this will enable determine the serving capacity of each berkad i.e the total number of communities dependant on each berkad. One important result is also community sensitisation to dispel misconceptions within the communtiies. The community will need to understand why the SRCS will be monitoring water bodies. Triggering for action is also a key result as the end goal will be to counter water shortages so as to mitigate water shortages

- chronological order:
o	i) Volunteer orientation on water resources monitoring
o	ii) community sensitization to dispel misconceptions about the mapping and water monitoring exercise
o	iii) Location data of the berkads (coordinates)
o	iv) determining the ownership status of each berkad
o	v) Determining the water level trigger
o	vi) water levels monitoring
o	vii) triggering action based on water levels

- specifically for the design phase:
o	Determining the water levels to trigger action.

-results for the implementation of AAs:
o	i) Determining the water level to trigger action
o	ii) water levels monitoring
o	iii) triggering action based on water levels




%%%%%%%%%%%%%%%%%%%%%%%%%%%%%%%%%%% INTERVIEW I1.2 RICHARD %%%%%%%%%%%%%%%%%%%%%%%%%%%%%%%%%%%%%%%%






%%%%%%%%%%%%%%%%%%%%%%%%%%%%%%%%%%% INTERVIEW I2 JUNG %%%%%%%%%%%%%%%%%%%%%%%%%%%%%%%%%%%%%%%%%%%%%







%%%%%%%%%%%%%%%%%%%%%%%%%%%%%%%%%%% INTERVIEW I3 BELEDI %%%%%%%%%%%%%%%%%%%%%%%%%%%%%%%%%%%%%%%%%%%







Intro

define objectives in detail
learn from other projects --> already done in stage 1
stuff about data storage, quality, analysis, privacy and security 
sampling bias (not that important)
training strategies
communication plan
define participant tasks and benefits in detail 

goals were ordered in this order --> I1: chronological order and focus due to focus on EAP and AA and constraints of the work itself





--> to order design phase --> as mentioned, SLMC
--> roadmap -> focus was the identification of goals, sub-goals and objectives and respective products and activities and for e and f a bit more. 


% goals a-d party necessary, party added benefit/bonus --> embeddedness is a fundamental requirement for CBS projects.

Each subsection has an overview diagram, in which the goal, products and activities are displayed in a structured tree-diagram. This diagram, together with potential implications and results, is described in detail in the respective section. Sections to the goals a-d are kept concise, with sections (e) and (f) being more extensive.

% order has meaning as well bottom to top.


each with tree diagram % mal sehen wie man das genau gestalten kann..
and then top-down description of the roadmap
+ results from interviews and literature

+ some integration of interview or literature analysis results 

the main objective is the mapping and monitoring of berkeds BUT: side goals may exist and benefit the project an the participants as well % already mentioned
thus --> the following sub-chapters will each be concerned with one of these six goals. As the main objective of this work falls into the scope of the goals (e) knowledge building (mapping) and (f) management improvements (monitoring) these were therefore also the focus of the work. The first 4 objectives are therefore somewhat shorter and not dealt with in as much detail. % already mentioned in the methodology, be concise.

presented for each of the minkman goals --> goal (+objective)
(a) awareness raising, (b) public education, (c) policy development, (d) method improvements, (e) knowledge building, and (f) management improvements








% pull the two three goals together to one? each is rather small and hard to separate anayway..
% in the way of: Build a foundation for mapping and monitoring.
% --> Determine and establish preceding procedures including awareness raising, public education, policy development and method improvements

TODO:TODO:TODO:TODO:TODO:TODO:TODO:TODO:TODO:TODO:TODO:TODO:TODO:TODO:TODO:TODO:TODO:TODO:TODO:TODO:TODO:
\subsection{Awareness Raising \& Public Education}

%%%%%%%%%%%%%%%%%%%%%%%%%%%%%%%%%%% INTERVIEW I1 RICHARD %%%%%%%%%%%%%%%%%%%%%%%%%%%%%%%%%%%%%%%%%%
o	The project might create huge expectations from the communities as there is the ongoing drought. Whenever there is monitoring of resources communities believe this should be followed up by instant aid. Private berkad owners might not be willing to contribute to the project. They might bar Volunteers from accessing their berkads thus creating tension between community volunteers and berkad owners.

o	ii) Community elders should be engaged before the start of the mapping and monitoring as they will help dispel misconceptions about the project
o	iii) the ministry of water resources should be in the loop during the entire project duration

o	From the community/volunteer perspective the main goal would be for them to know the existing water resources within their vicinity, as well as the capacity of these water bodies. The main goal being to ascertain whether these water bodies are able to withstand the demand during drought periods.

o	The community and SRCS goals match as both focus on closing the knowledge currently existing regarding berkads numbers per district, community and regional level




%%%%%%%%%%%%%%%%%%%%%%%%%%%%%%%%%%% INTERVIEW I1.2 RICHARD %%%%%%%%%%%%%%%%%%%%%%%%%%%%%%%%%%%%%%%%

Awareness raising and information dissemination should be more on informing the communities on how to improve water quality at local level e,g boiling before drinking. Involving private berked owners is also feasible however their involvement could be limited as they are more concerned about their business models i.e selling of the water and preserving their berkeds than being part of the overall response/Anticipatory action mechanism. Nevertheless there is the potential to work closely with the private berked owners. This can be done through rehabilitation of their privately owned berkeds in return for their involvement in response and anticipatory action activities related to addressing water shortages.

-	Utilise the community based SRCS volunteers to engage communities and sensitise the communoties on the riole the SRCS plays. Also establishing a robust feedback and Complaints mechanism that ensures communities can easily relay their feedback.


%%%%%%%%%%%%%%%%%%%%%%%%%%%%%%%%%%% INTERVIEW I2 JUNG %%%%%%%%%%%%%%%%%%%%%%%%%%%%%%%%%%%%%%%%%%%%%
(4, 5) It must be certain, that something happens after a report is send or at least communicated in which situations they have to handle it themselves. This should be communicated, discussed and decided beforehand. Otherwise, it could fall back negatively on the SRCS and the Volunteer (33).
(24) SRCS will meet the elders before the start and will explain what, who and when things happen. They continue this throughout in roughly monthly or quarterly intervals.
(35) Maybe communicate a timeframe how long it will take until water arrives so they can plan for themselves.

Stakeholder > Communities
-	(88) Reasons for reporting must be explained in detail before the start of it.
-	(90) They will have expectations of this project.


%%%%%%%%%%%%%%%%%%%%%%%%%%%%%%%%%%% INTERVIEW I3 BELEDI %%%%%%%%%%%%%%%%%%%%%%%%%%%%%%%%%%%%%%%%%%%






\subsubsection{Integrated Water Management (Goal and Groundwork for mapping/monitoring)}


one of the goals of public education and policy development

Integrating communities into Water managemend practices is needed!

ground work -> public education --> inform community about water shortages (?) -> create "agreement on water priorization during times of acute drought" + "a common shared agreement to prevent water resource contamination, as well as mitigating over-abstraction." (Day, 2009: 52)

BUT:
“Communities may not always represent a homogeneous, consenting group” ([Day, 2009, p. 52](zotero://select/groups/4773535/items/YWSNQ8A2)) ([pdf](zotero://open-pdf/groups/4773535/items/ETPCI5RI?page=7&annotation=NMA3CWDS))
--> different interests, resources, knowledge, access rights, own hierarchy + status level

how about private water sources?
“Strikingly, the traditional influence of sheikhs in water management does not, however, extend to supervision of private wells or boreholes operated by farmers or individual landowners. Sh” ([Day, 2009, p. 55](zotero://select/groups/4773535/items/YWSNQ8A2)) ([pdf](zotero://open-pdf/groups/4773535/items/ETPCI5RI?page=10&annotation=FXZW8RBU))

AA -> manage water and prioritize effectively (e.g. start with it now)
“This is significant because in drought-prone environments little at tempt is made to inform communities about their available ground water resources and there is minimal emphasis or preparation for monitoring groundwater fluctuations, prioritizing water usage dur ing periods of hardship or assisting communities to develop basic contingency plans with relief agencies or local authorities acting as a back-stop to provide support during periods of acute” ([Day, 2009, p. 51](zotero://select/groups/4773535/items/YWSNQ8A2)) ([pdf](zotero://open-pdf/groups/4773535/items/ETPCI5RI?page=6&annotation=6C4BAHSE))

“hardship. When describing community water projects, 'sustainability' is often referred to. In reality the immediate challenge is to introduce sound steward ship of water resources to assist communities to resist and recover from drought or low and variable rainfall” ([Day, 2009, p. 51](zotero://select/groups/4773535/items/YWSNQ8A2)) ([pdf](zotero://open-pdf/groups/4773535/items/ETPCI5RI?page=6&annotation=DA9BJCDN))

“r of distinct advantages of engaging in commu nity-based water resource mana” ([Day, 2009, p. 51](zotero://select/groups/4773535/items/YWSNQ8A2)) ([pdf](zotero://open-pdf/groups/4773535/items/ETPCI5RI?page=6&annotation=85TSGZK9))

FIGURE of the framework: “Figure 2. Community-based water resource manageme” ([Day, 2009, p. 59](zotero://select/groups/4773535/items/YWSNQ8A2)) ([pdf](zotero://open-pdf/groups/4773535/items/ETPCI5RI?page=14&annotation=BAMSY255))
%% --> this framework as 'base'/foundation/Early Action/AA?

“Local water users often possess detailed indigenous knowledge related to water resources, water needs and historical change that has occurred related to water use. 
Water users recognize that water is a fundamental component of their subsistence-based livelihoods, which helps to weave rela tionships between water users. 
Communities are able to monitor agreed water usage on a daily basis, as part of their daily activities. 
Communities often have historical mechanisms for conflict and dispute resolution related to water resource management, which may require continued support and assistance to evolve and adapt to global challenges. 
Effective water management requires community participation; this principle is well understood in development li” ([Day, 2009, p. 52](zotero://select/groups/4773535/items/YWSNQ8A2)) ([pdf](zotero://open-pdf/groups/4773535/items/ETPCI5RI?page=7&annotation=4RVGKM5R))

“However, responsible planning for drought mitigation at community level is often omitted.” ([Day, 2009, p. 47](zotero://select/groups/4773535/items/YWSNQ8A2)) ([pdf](zotero://open-pdf/groups/4773535/items/ETPCI5RI?page=2&annotation=S4ACQRPL))

“Communities frequently remain excluded from any basic capacity building, centred on water resource management, as part of a localized Integrated Water Resources Management (IWRM) programme.” ([Day, 2009, p. 47](zotero://select/groups/4773535/items/YWSNQ8A2)) ([pdf](zotero://open-pdf/groups/4773535/items/ETPCI5RI?page=2&annotation=WBL45NKR))

“Community-based water resources management” ([Day, 2009, p. 47](zotero://select/groups/4773535/items/YWSNQ8A2)) ([pdf](zotero://open-pdf/groups/4773535/items/ETPCI5RI?page=2&annotation=RQLJMKL7))



% to facilitate this and further implement this roadmap --> policy (social frameworks) and methods (pracitcal techniques and tools) need to be developed.



TODO:TODO:TODO:TODO:TODO:TODO:TODO:TODO:TODO:TODO:TODO:TODO:TODO:TODO:TODO:TODO:TODO:TODO:TODO:TODO:TODO:

\subsection{Policy Development \& Method Improvements}
TODO:TODO:TODO:TODO:TODO:TODO:TODO:TODO:TODO:TODO:TODO:TODO:TODO:TODO:TODO:TODO:TODO:TODO:TODO:TODO:TODO:
%%%%%%%%%%%%%%%%%%%%%%%%%%%%%%%%%%% INTERVIEW I1 RICHARD %%%%%%%%%%%%%%%%%%%%%%%%%%%%%%%%%%%%%%%%%%





%%%%%%%%%%%%%%%%%%%%%%%%%%%%%%%%%%% INTERVIEW I1.2 RICHARD %%%%%%%%%%%%%%%%%%%%%%%%%%%%%%%%%%%%%%%%






%%%%%%%%%%%%%%%%%%%%%%%%%%%%%%%%%%% INTERVIEW I2 JUNG %%%%%%%%%%%%%%%%%%%%%%%%%%%%%%%%%%%%%%%%%%%%%







%%%%%%%%%%%%%%%%%%%%%%%%%%%%%%%%%%% INTERVIEW I3 BELEDI %%%%%%%%%%%%%%%%%%%%%%%%%%%%%%%%%%%%%%%%%%%






TODO:TODO:TODO:TODO:TODO:TODO:TODO:TODO:TODO:TODO:TODO:TODO:TODO:TODO:TODO:TODO:TODO:TODO:TODO:TODO:TODO:
\subsection{Knowledge Building}
with preceding procedures including awareness raising, public education, policy development and method improvements determined and accomplished now proceed to the actual mapping task of Berkeds.
%%%%%%%%%%%%%%%%%%%%%%%%%%%%%%%%%%% INTERVIEW I1 RICHARD %%%%%%%%%%%%%%%%%%%%%%%%%%%%%%%%%%%%%%%%%%






%%%%%%%%%%%%%%%%%%%%%%%%%%%%%%%%%%% INTERVIEW I1.2 RICHARD %%%%%%%%%%%%%%%%%%%%%%%%%%%%%%%%%%%%%%%%
Regarding Anticipatory actions, there has been any actions yet due to the fact that there is no water monitoring and trigger mechanism in place % though there is of course some response

The SRCS in consultation with the government select target communities based on a pre existing selection /vulnerability criteria based on either number of IDP camps etc

BERKADS - important indicators besides •	the location,•	ownership,•	total number of people or communities dependant on the berkad,•	its water storage capacity and•	functioning

Other information might included the year it was built, the last time it was rehabilitated etc. However this kind of information might be missing as you need people with community/institutional memory to provide such kind of information. Somalis are highly mobile communities and it will be difficult to get information on past details per particular geographical area.


% local knowledge could be well utilized
% Does the community have an idea of how long the water of their water sources will last? How good is this prediction usually?
-	Yes they have an idea. These kinds of predictions are good as communities usually have their own control measures to ensure equitable distribution of water e,g how many containers per family etc. The berkeds are usually locked to ensure there is controlled acess to the water stored.




%%%%%%%%%%%%%%%%%%%%%%%%%%%%%%%%%%% INTERVIEW I2 JUNG %%%%%%%%%%%%%%%%%%%%%%%%%%%%%%%%%%%%%%%%%%%%%


(16) fusion with other datasets can be challenging and is a lot of work




%%%%%%%%%%%%%%%%%%%%%%%%%%%%%%%%%%% INTERVIEW I3 BELEDI %%%%%%%%%%%%%%%%%%%%%%%%%%%%%%%%%%%%%%%%%%%






TODO:TODO:TODO:TODO:TODO:TODO:TODO:TODO:TODO:TODO:TODO:TODO:TODO:TODO:TODO:TODO:TODO:TODO:TODO:TODO:TODO:
\subsection{Management Improvements}
%one example how to execute the entire SLMC tree (see Miro)
%%%%%%%%%%%%%%%%%%%%%%%%%%%%%%%%%%% INTERVIEW I1 RICHARD %%%%%%%%%%%%%%%%%%%%%%%%%%%%%%%%%%%%%%%%%%



%%%%%%%%%%%%%%%%%%%%%%%%%%%%%%%%%%% INTERVIEW I1.2 RICHARD %%%%%%%%%%%%%%%%%%%%%%%%%%%%%%%%%%%%%%%%
berked rehabilitation (SRCS), water trucking (other agencies), distribution of water purification tablets (SRCS), multi purpose cash, awareness campaigns related to hygiene promotion (SRCS). Regarding Anticipatory actions, there has been any actions yet due to the fact that there is no water monitoring and trigger mechanism in place.

Assistance/response is based on the initial prioritization of target areas that SRCS conducts. The prioritization is based on assumed vulnerability per community based on Number of IDP camps in the area, number of women headed families, predicted IPC classifications etc.
% limitations
The SRCS in consultation with the government select target communities based on a pre existing selection /vulnerability criteria based on either number of IDP camps etc


Activities such as berked rehabilitation are done in consultations with the communities and SRCS branches who flag/identify berkeds in need of repairing. Repairing may consists of re roofing/ roofing, and brickwork to strengthen the structure. The berkeds are meant to capture run off water in case of rainfall incidences. In cases where there hasnt been rains for a prolonged time then water trucks are deployed to deliver water to the communities. Cash has been an imoortattn modality to adress the water shortages. In the current prevailing drought, water and food insecurity crisis, water is now being sold by private players. So the cash has come in handy to t least enable the communities to buy fresh water for drinking. Water sources such as dug wells are often contaminated as livestock i,e camels, goats also drink water from those same water bodies as well.

Water levels in berkeds could be a good indicator, however it cannot be a stand alone indicator. This has to be combined by meteorological forecasts and local knowledge as well.
% triangulation with other sources. -> trust and quality as always

-proposed AA:
-	Awareness raising and information dissemination should be more on informing the communities on how to improve water quality at local level e,g boiling before drinking. Involving private berked owners is also feasible however their involvement could be limited as they are more concerned about their business models i.e selling of the water and preserving their berkeds than being part of the overall response/Anticipatory action mechanism. Nevertheless there is the potential to work closely with the private berked owners. This can be done through rehabilitation of their privately owned berkeds in return for their involvement in response and anticipatory action activities related to addressing water shortages.

yes, i) timely distribution of cash to enable communities to buy and stock fresh water
ii) timely distribution of water purification tablets
iii) timely rehabilitation of other water sources such as boreholes

- TRIGGER
These water levels are ideal i.e 
o	Empty (no water at all) 
o	Critical (1 day of water supply remaining), 
o	Low (3 days of water supply remaining), 
o	Middle (5 days of water supply remianing) 
o	High (full capacity)

-	Low category should trigger AA (he possibly meant water trucking and/or cash)

- MONITORING INTERVAL
•	Water level (daily monitoring)
•	Berked condition (annually)
•	Number of people accessing the water form the berked (weekly/monthly)

- WATER QUALITY
Water quality is difficult to monitor at community level as it is a technical activity. Unless if the SRCS through the branch staff are equipped with water testing equipment as well as training them on the water parameters to be tested.
+ not aware of any feasible water quality tests on the ground.. (should be further investigated though..)


% No Answer to Water Trucking



%%%%%%%%%%%%%%%%%%%%%%%%%%%%%%%%%%% INTERVIEW I2 JUNG %%%%%%%%%%%%%%%%%%%%%%%%%%%%%%%%%%%%%%%%%%%%%
(25) Might be beneficial to bring the MoH in contact with the MoWR in order to transfer their experiences with CBS to the new approach.
(37) One to three codes for regular monitoring should be alright but not more, as more codes make it more complicated and will narrow down the choice which Volunteer to take. 

(30, 31) Sending photos would also be possible, though smartphones and internet are often not available. Julia Jung is not supporting the distribution of smartphones for ‘several reasons’

(35) Maybe communicate a timeframe how long it will take until water arrives so they can plan for themselves.

(9, 49) in special occasion up to 7 numbers
(10) small notes that explain the codes and their order
(27) Other or more information besides the codes could be clarified via phone and inserted by the supervisor manually.

%%%%%%%%%%%%%%%%%%%%%%%%%%%%%%%%%%% INTERVIEW I3 BELEDI %%%%%%%%%%%%%%%%%%%%%%%%%%%%%%%%%%%%%%%%%%%
-(10) SRCS only focusses on hot spot areas where they expect outbreaks to happen. They do not cover all of the country.
-(11, 14) Response happens together with other partners such as the government, MoH, WHO, and other sister RCRC organisations.
-(13) Reports are verified by the regional officer.
-(15) “SRCS has mobile teams who can be deployed immediately within hours so they can do the response.”

-(70) Water monitoring in urban areas is not a problem. There is an agency that is responsible for water supply. “So we don't have any issue with that.”
-(71) Contrary, “when it comes to the rural areas, and nomad areas, is the way we have the problem”
-(72, 62) water can get contaminated, “For example, when you are taking from the source, and again, when you are traveling with the water, and again, when you are storing the water in your home, or when you are using even the water”
-(73) “water may be clear in colour, but when we are using it, it is contaminated. So we cannot decide by the colour”
-(26) “Whatever source of water they can find is what they have.”
-(43) SRCS has a good connection to the local communities and thus they get direct feedback about water shortages.

% WATER TRUCKING
-(74) before putting water in the berked, they initially clean it.
-(74) water trucking “can take a long distance to the, for example, main cities” and is thus costly and requires lots of time.
-(75) but still, “When it comes to the water trucking, it depends” à no universal solution will work
-(76, 77) Thus far, information come from SRCS assessments, community themselves, government or FSNAU (Food Security and Nutrition Analysis Unit) (https://fsnau.org/)
-(78) Prioritisation is based on the government + “all these efforts together they decide where these resources” are going including SRCS. -> Most vulnerable in the area are prioritised and that mostly depends.
-See (81) FbF
-(41) “Sometimes there was water trucking and, you know, the SRCS or the other organizations, even the commercial or trade people, they were supporting to the communities who are in need”

% FbF
-(7) Volunteers provide Oral Rehydration Salts and aquatabs.
-(18) If the water level of berkeds becomes less or scarce, the volunteers provide hygiene and health promotion activities.
-(79) Sometimes the people/community/some households buy water trucking themselves by bringing their money together. They do this in the initial phase.
% !!
-(80) “people they are depending on their livestock” but “if there is a drought the livestock become weak or die” and the people cannot afford to buy water (trucking). “[...] this is the time they talk to the other NGOs or the government and say we need support [...]”
-(81) Water trucking is not only financed by the government and NGOs but also by normal people (“good willers” who then try to gather money and buy water. They “distribute according to the need” and look at the magnitude of the problem and where it exists - refer this water to the to the community”.
-(19, 20, 21) In case of water shortage, volunteers tell people how to prevent waterborne diseases by providing hygiene and health promotion as well as water purification and water boiling actions.






%%%%%%%%%%%%%%%%%%%%%%%%%%%%%%%%%%%%%%%%%%%%%%%%%%%%%%%%%%%%%%%%%%%%%%%%%%%%%%%%%%%%%%%%%%%%%%%%%%%
%%%%%%%%%%%%%%%%%%%%%%%%%%%%%%%%%%%%%%%%%%%%%%%%%%%%%%%%%%%%%%%%%%%%%%%%%%%%%%%%%%%%%%%%%%%%%%%%%%%
%%%%%%%%%%%%%%%%%%%%%%%%%%%%%%%%%%%%%%% !!! SECTION 4-6 !!! %%%%%%%%%%%%%%%%%%%%%%%%%%%%%%%%%%%%%%%
%%%%%%%%%%%%%%%%%%%%%%%%%%%%%%%%%%%%%%%%%%%%%%%%%%%%%%%%%%%%%%%%%%%%%%%%%%%%%%%%%%%%%%%%%%%%%%%%%%%
%%%%%%%%%%%%%%%%%%%%%%%%%%%%%%%%%%%%%%%%%%%%%%%%%%%%%%%%%%%%%%%%%%%%%%%%%%%%%%%%%%%%%%%%%%%%%%%%%%%
%%%%%%%%%%%%%%%%%%%%%%%%%%%%%%%%%%%%%%%%%%%%%%%%%%%%%%%%%%%%%%%%%%%%%%%%%%%%%%%%%%%%%%%%%%%%%%%%%%%
\section{Stages 4 to 6}
These sections were not the main focus of this work as feasibility assessment and subsequent design of a roadmap was the primary focus. Furthermore, for stage 4 and 5, groundwork is completed otherwise and therefore a theoretical and practical foundation is already well established. The volunteer recruitment process and community building procedure of the \acrshort*{srcs} is outlined in Stage 4 and potential data management tools are described in Stage 5. Locally implemented procedures and strategies for Stage 6 Evaluation are briefly presented thereafter. A short summary of the results concludes this chapter.



\subsection{Stage 4: Community Building} % volunteers + SRCS + training - etc.
%%%%%%%%%%%%%%%%%%%%%%%%%%%%%%%%%%% INTERVIEW I1 RICHARD %%%%%%%%%%%%%%%%%%%%%%%%%%%%%%%%%%%%%%%%%%
paid employees does the SRCS have in total?
o	249
Anticipatory Actions?
o	30
•	How many volunteers does the SRCS have?
o	1500
Volunteers spread across the country? 
-> Volunteers are evenly spread across the country.
However some regiosn have inactive volunteers due to less activities there whilst some regions have active volunteers due to the amount of project work being undertaken there



%%%%%%%%%%%%%%%%%%%%%%%%%%%%%%%%%%% INTERVIEW I1.2 RICHARD %%%%%%%%%%%%%%%%%%%%%%%%%%%%%%%%%%%%%%%%






%%%%%%%%%%%%%%%%%%%%%%%%%%%%%%%%%%% INTERVIEW I2 JUNG %%%%%%%%%%%%%%%%%%%%%%%%%%%%%%%%%%%%%%%%%%%%%

-	(66) Meanwhile, refreshers are not conducted monthly anymore as the volunteers know their business by now. 
-	(112, 119) Volunteers are from within the community and chosen by the elders. They do not get incentives from the SRCS. Their incentives are helping their community and the travels for the trainings.
-	Preliminary trainings, supervision and regular refreshers are important and useful.
-	(120) Volunteers are mostly women as they stay in the community and do not travel as much as men.

Stakeholder > Communities
-	(88) Reasons for reporting must be explained in detail before the start of it.
-	(90) They will have expectations of this project.

(44) No money is given either to the MoH nor to Volunteers.

(53) Supervisor can validate reports e.g. vie phone.
(54) Great success factors are the training, supervision and regular refreshers by the supervisors.


%%%%%%%%%%%%%%%%%%%%%%%%%%%%%%%%%%% INTERVIEW I3 BELEDI %%%%%%%%%%%%%%%%%%%%%%%%%%%%%%%%%%%%%%%%%%%
-(5, 6, 7) Oral Rehydration Points are in the community and run by volunteers of the community who get trained and then support the community by education and promotion of e.g. WASH activities and going to the actual treatment centers.
-(19, 20, 21) In case of water shortage, volunteers tell people how to prevent waterborne diseases by providing hygiene and health promotion as well as water purification and water boiling actions.
-(23) They are trained “not to wait until the people become fall sick, but in a professional mechanism”
-(24) Volunteers “do awareness raising, hygiene and health promotion sessions by doing, for example, group sessions by visiting house to house, visiting to meeting and all these things.”
-(22) SRCS distributes aqua tablets per month “to the volunteers so that they can manage at community level if there is a case.”

-(19, 20, 21) In case of water shortage, volunteers tell people how to prevent waterborne diseases by providing hygiene and health promotion as well as water purification and water boiling actions.
-(23) They are trained “not to wait until the people become fall sick, but in a professional mechanism”
-(24) Volunteers “do awareness raising, hygiene and health promotion sessions by doing, for example, group sessions by visiting house to house, visiting to meeting and all these things.”
-(22) SRCS distributes aqua tablets per month “to the volunteers so that they can manage at community level if there is a case.”
-(43) SRCS has good relations with the communities.
-(44) “community leaders are the one who tells the SRCS or other partners or the government that there is a water shortage”
-(45) “Somalis they support a lot each other when it comes to the disasters or something like that.“ – “everybody in the community whether in the urban or rural areas is participating to support each other”
-(46)”SRCS has a good reputation and image at community levels” “it is one of the most trusted organization in the country. So it is one of the most trusted organization in the country. So there is a strong relation at community level. So that it helps us also to do this program as community level.”
-(48) Response happens together with the MoH.
-(49) SRCS spends a lot of time on community bond building.
-(50) “SRCS, what we do is to provide any necessary support at the community level.”
Volunteers
-„when we are recruiting, I can't say, I cannot say recruiting. When we are you know going to get volunteers of that community, we go to the community.“
-(52) The person must be willing to be a volunteer as the SRCS is not paying them.
-(52, 53) Volunteers have a good reputation in the community and are selected by the community or the committees in that community based on their criteria’s.
-(54, 55, 56) After the selection, SRCS is doing a small assessment about e.g. reading and writing skills and then provide training to them on the basis of the CBS program (health promotions, coding, etc.).
-(57) After the training, the volunteers are send back to their communities and start working there.
-(59) In the community the community leaders, community committees and also the community health committees are important. “They are the one who are supporting” the SRCS on site. There is also a good collaboration between these groups, the volunteer and the SRCS.
-Volunteers teach mothers, and communities about health, how to prevent water contamination, etc.



\subsection{Stage 5: Data Management}
%%%%%%%%%%%%%%%%%%%%%%%%%%%%%%%%%%% INTERVIEW I1 RICHARD %%%%%%%%%%%%%%%%%%%%%%%%%%%%%%%%%%%%%%%%%%





%%%%%%%%%%%%%%%%%%%%%%%%%%%%%%%%%%% INTERVIEW I1.2 RICHARD %%%%%%%%%%%%%%%%%%%%%%%%%%%%%%%%%%%%%%%%






%%%%%%%%%%%%%%%%%%%%%%%%%%%%%%%%%%% INTERVIEW I2 JUNG %%%%%%%%%%%%%%%%%%%%%%%%%%%%%%%%%%%%%%%%%%%%%

GOAL OF NYSS:
(41) The goal of NYSS was the provision of a simple data collection tool
(44, 46) The goal was not the collection of data nor forecasting, the goal was to create a platform for Early Warning.
(76) goal: investigation and response from the MoH.


(2) Can be a problem to send information about community members – possibly also about water sources? 
(36) Monitoring water sources would work well with the CBS system and fit well with the overall theme of health risks.

(28, 29) It should be communicated, that reports will be checked by the supervisor in order to prevent false reports in hope of more water. If this happens frequently, a solution must be conceptualized.

(30, 31) Sending photos would also be possible, though smartphones and internet are often not available. Julia Jung is not supporting the distribution of smartphones for ‘several reasons’

-	(3, 12, 20) Excel, SMS and manual insert into excel works as well. NYSS biggest advantage: lots of automatization
-	(40) Kobo would possibly be the best alternative if it doesn’t work with NYSS.

-	(105) WHO is interested in NYSS.
-	(93) IFRC used and uses Kobo.
-	(94) IFRC decides about the development of NYSS.

(8) difference to most other tools: possible with a basic phone. (19) which is necessary because CBS is often used in regions, where no Smartphones are accessible. More complex input requires also more educated volunteers.
(12) The good side on NYSS is, that everything is automated
(43) NYSS itself is relatively new. March 2020 was the first time it was used in its current form.

(16) fusion with other datasets can be challenging and is a lot of work
(87) integration of this into NYSS is work and it needs to be discussed who does it and who pays for it.

-	(49) Automatic integration with other data, e.g. from the Ministry is laborious and can be complicated. Though it’s doable.

-	(58) The platform is not made for individual, personal data.
-	(64) data collection in itself is not possible via NYSS.
-	(56) Server from NYSS is in Ireland for data protection reasons and its easier to maintain for the developers.
-	(57) The server location can be an issue for the Ministry as they do not have control over the data. The ownership of the data lies with the National Society. 



%%%%%%%%%%%%%%%%%%%%%%%%%%%%%%%%%%% INTERVIEW I3 BELEDI %%%%%%%%%%%%%%%%%%%%%%%%%%%%%%%%%%%%%%%%%%%
-(33) NYSS was developed over the course of a year with a lot of feedback from the SRCS and is now “very effective and very supportive.”
-(34) NYSS is highly automatized and alerts are given when thresholds are reached based on geographical location.
-(36) “So any mobile you can use it.”
-(37) “No need to have a smartphone but you are using SMS.”
-(38) must be a network in that area and again SMS Eagle.”
-(39) Before the automatization, they downloaded the data from the NYSS platform and analyzed it via Excel.
-(58) Feedback on wrong reports is given by the regional supervisor very timely and they support the volunteers to send the report in the right format.






\subsection{Stage 6: Evaluation}
-> community meetings with the elders every month or so
-> feedback messages NYSS and so on
-> fraisl: ongoing effort + metrics for measuring success --> implemented in NYSS -> reliance, completeness and quality of the contributions and the activity of the volunteers


%%%%%%%%%%%%%%%%%%%%%%%%%%%%%%%%%%% INTERVIEW I1 RICHARD %%%%%%%%%%%%%%%%%%%%%%%%%%%%%%%%%%%%%%%%%%





%%%%%%%%%%%%%%%%%%%%%%%%%%%%%%%%%%% INTERVIEW I1.2 RICHARD %%%%%%%%%%%%%%%%%%%%%%%%%%%%%%%%%%%%%%%%

-	Utilise the community based SRCS volunteers to engage communities and sensitise the communoties on the riole the SRCS plays. Also establishing a robust feedback and Complaints mechanism that ensures communities can easily relay their feedback.




%%%%%%%%%%%%%%%%%%%%%%%%%%%%%%%%%%% INTERVIEW I2 JUNG %%%%%%%%%%%%%%%%%%%%%%%%%%%%%%%%%%%%%%%%%%%%%

"Usually, volunteers are chosen by the elders and the main criterium is not them being the smartest."
(28, 29) It should be communicated, that reports will be checked by the supervisor in order to prevent false reports in hope of more water. If this happens frequently, a solution must be conceptualized.

(32) Should work with prior training, supervision and feedback.

(34) Additional communication via phone is useful and necessary especially for details and instant feedback e.g. to communicate lack in response and its reasons.

-	(109) SRCS invests a lot into communication and feedback with the communities.
-	(113) SRCS are no rookies. They know how to communicate – big part of their culture.
(27) Other or more information besides the codes could be clarified via phone and inserted by the supervisor manually.

-	The option for feedback messages comes from discussions with the SRCS
-	(60) An evaluation has been conducted but not yet made public.
-	(66) SRCS knows their business. They are no rookies.

(53) Supervisor can validate reports e.g. vie phone.

%%%%%%%%%%%%%%%%%%%%%%%%%%%%%%%%%%% INTERVIEW I3 BELEDI %%%%%%%%%%%%%%%%%%%%%%%%%%%%%%%%%%%%%%%%%%%
-(13) Reports are verified by the regional officer.
-(58) Feedback on wrong reports is given by the regional supervisor very timely and they support the volunteers to send the report in the right format.
-(43) SRCS has a good connection to the local communities and thus they get direct feedback about water shortages.
-(43) SRCS has good relations with the communities.
-(44) “community leaders are the one who tells the SRCS or other partners or the government that there is a water shortage”



\section{Summary Results + key lessons learned (?)}
%%%%%%%%%%%%%%%%%%%%%%%%%%%%%%%%%%% INTERVIEW I1 RICHARD %%%%%%%%%%%%%%%%%%%%%%%%%%%%%%%%%%%%%%%%%%





%%%%%%%%%%%%%%%%%%%%%%%%%%%%%%%%%%% INTERVIEW I1.2 RICHARD %%%%%%%%%%%%%%%%%%%%%%%%%%%%%%%%%%%%%%%%






%%%%%%%%%%%%%%%%%%%%%%%%%%%%%%%%%%% INTERVIEW I2 JUNG %%%%%%%%%%%%%%%%%%%%%%%%%%%%%%%%%%%%%%%%%%%%%







%%%%%%%%%%%%%%%%%%%%%%%%%%%%%%%%%%% INTERVIEW I3 BELEDI %%%%%%%%%%%%%%%%%%%%%%%%%%%%%%%%%%%%%%%%%%%













%%%%%%%%%%%%%%%%%%%%%%%%%%%%%%%%%%%%%%%%%%%%%%%%%%%%
summary
--> + review to the deductive hypothesis --> could the lit review and the interviews answer this? here or discussion?




% The concluding summary is very important because it summarises your key findings and lays the foundation for the discussion chapter. Keep in mind that some readers may skip directly to this section (from the introduction section), so make sure that it can be read and understood well in isolation.
% In this section, you need to remind the reader of the key findings. That is, the results that directly relate to your research questions and that you will build upon in your discussion chapter. Remember, your reader has digested a lot of information in this chapter, so you need to use this section to remind them of the most important takeaways.



























\section{Main Section 1}

% put this in the result section.
intro to Somalia EAP: https://docs.google.com/document/d/1xUEXm8RxVHTO468KqXSAoBX-cpkPwiff/edit
https://heigit.atlassian.net/wiki/spaces/FIS/pages/1704096/Indices

current EAP stage in somalia
--> Somalia so far. But: still under development.

“Hazards Exposure and Vulnerability” ([Somali Red Crescent Society, 2022, p. 13](zotero://select/groups/4773535/items/FZ6BJHJA)) ([pdf](zotero://open-pdf/groups/4773535/items/RJKNZZZ2?page=17&annotation=IRH526LN))

“Feasibility Study on Potential Use of Forecast-based Financing (FbF) for SRCS Final Report” ([Somali Red Crescent Society, 2022, pp. -3](zotero://select/groups/4773535/items/FZ6BJHJA)) ([pdf](zotero://open-pdf/groups/4773535/items/RJKNZZZ2?page=1&annotation=KHCH33GX)





% Quality criteria for Early Action Protocols
https://heigit.atlassian.net/wiki/download/attachments/1704186/FbA-EAP-criteria-May-2022.docx?version=1&modificationDate=1677660171372&cacheVersion=1&api=v2
(summery available -> confluence)


%-----------------------------------
%	SUBSECTION 2
%-----------------------------------







%----------------------------------------------------------------------------------------
%	SECTION 2
%----------------------------------------------------------------------------------------

\section{}


%----------------------------------------------------------------------------------------
%	SECTION 1
%----------------------------------------------------------------------------------------

\section{Case study protocol}

%----------------------------------------------------------------------------------------
%	SECTION 1
%----------------------------------------------------------------------------------------

\section{Main Section 1}

%----------------------------------------------------------------------------------------
%	SECTION 1
%----------------------------------------------------------------------------------------

\section{Main Section 1}

%----------------------------------------------------------------------------------------
%	SECTION 1
%----------------------------------------------------------------------------------------



"5.3 Types of water resources monitoring
As it has been indicated above, water resources monitoring provides information on the state and trends of quantitative and qualitative characteristics of the monitored object, level and distribution of anthropogenic loads, state of ecosystems and a degree of possibility of satisfaction of various needs in water, municipal and economic.

There are three main types of water resources monitoring, used in the water resources management system:
%% main points:
local monitoring, performed for solution of specific local problems on a limited part of the water body or the territory;
global (background) monitoring, performed at man-impact free sites, or on sites with low level of anthropogenic influence. Such monitoring is performed for acquisition of information on steady natural characteristics of environmental components. The background monitoring of water bodies is used for evaluation and/or prognostication of shifts in their state caused by economic activities;
comprehensive (regime) monitoring performed at the water body observation network for determination of the actual state of the water body, for decision making on efficient use, protection, and restoration of water resources;
critical or alarm monitoring, performed at sites of high risk for immediate warning about unfavourable situations caused primarily by human activities.
In the water management system there is also a special type of monitoring for wastewater discharges to the water body."https://echo2.epfl.ch/VICAIRE/mod_4/chapt_5/main.htm

"As a rule, the following parameters are always monitored:

Water Temperature.
Transparency or Turbidity.
pH.
Conductivity.
Dissolved oxygen (DO).
Total phosphorus.
Total nitrogen.
Nitrogen, Ammonia.
Nitrogen, Nitrate.
Soluble Reactive Phosphorus.
Faecal coliform bacteria." https://echo2.epfl.ch/VICAIRE/mod_4/chapt_5/main.htm
%% --> not gonna happen but still interesting

https://www.oxfamwash.org/water/cbwrm/Oxfam%20CBWRM%20Companion,%202009.pdf
%----------------------------------------------------------------------------------------
%	SECTION 1
%----------------------------------------------------------------------------------------

\section{Main Section 1}

%----------------------------------------------------------------------------------------
%	SECTION 1
%----------------------------------------------------------------------------------------

\section{NOTES}

discussion: the SLMC was good until general activities - then: it has to become quite detailed --> not feasible in the scope of this work. but a good framework to continue the work with.
-> size of Berkad is detrimental! --> Richard water levels are for a week. Improved ones are for a couple of months --> water level is only one variable! needs to be correlated with the extraction and storage size

- BRCiS: "Cash transfers were "identified across several clusters as the preferred action" where local markets and the operational context allow" (TB) --> but only there and market prices are very high when drought sets in
--> Beledi highlights help throughout the society -> but water prices rise very high.. 

- better comparison between the comparable projects of NADFOR, MoWR and OCHA was not possible due to lack of raw data --> no interviews



% APPEDNIX
% include questions, questionnaires, transcripts and codes
The first interview came about through existing contacts of the project in which this work is embedded, and the interviewee was the project leader of the FbF approach in the \acrshort{srcs} (I1). In the further course, the CBS project manager on the Norwegian Red Cross side (I2) and the CBS manager on the Somali side (I3) were also interviewed. Between these two interviews, there was a second interview with the project manager of the \acrshort{srcs}' \acrshort{fbf} team (I1.2).