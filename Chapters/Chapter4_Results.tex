% Chapter Template

\chapter{Design and Application} % Main chapter title

\label{chapter4}


% "The first step is to craft a brief introduction to the chapter. This intro is vital as it provides some context for your findings. In your introduction, you should begin by reiterating your problem statement and research questions and highlight the purpose of your research. Make sure that you spell this out for the reader so that the rest of your chapter is well contextualised.
% The next step is to briefly outline the structure of your results chapter. In other words, explain what’s included in the chapter and what the reader can expect. In the results chapter, you want to tell a story that is coherent, flows logically, and is easy to follow, so make sure that you plan your structure out well and convey that structure (at a high level), so that your reader is well oriented." 
%----------------------------------------------------------------------------------------
%	SECTION 1
%----------------------------------------------------------------------------------------
%using simple transmission protocols such as SMS

% not only results - a bit of reasoning included, as it is common in case study literature --> short reasoning, e.g. why each group of the project requirements exist

% “Each of the stages presented in this section are interconnected and one step does not necessarily need to end for another to begin (fIg. 1).” ([Fraisl et al., 2022, p. 2](zotero://select/groups/4773535/items/FBJD7SWS)) ([pdf](zotero://open-pdf/groups/4773535/items/7WBDKYDY?page=2&annotation=MQDZ6JNX))

% “All stages and steps should be reviewed throughout the project cycle to actively incorporate changing factors, lessons learned and participant feedback.” ([Fraisl et al., 2022, p. 2](zotero://select/groups/4773535/items/FBJD7SWS)) ([pdf](zotero://open-pdf/groups/4773535/items/7WBDKYDY?page=2&annotation=QGFBERT3))
% “there is no one-size-fits-all approach” ([Fraisl et al., 2022, p. 2](zotero://select/groups/4773535/items/FBJD7SWS)) ([pdf](zotero://open-pdf/groups/4773535/items/7WBDKYDY?page=2&annotation=TVGHJZWV))

% two-fold? raod map is partly described and reasoned, its application is not (?)

The first aim of this thesis was to answer a deductive hypothesis, namely whether CBS and MCS are potentially viable approaches to answer the primary inductive research question of this thesis. The main aim was to create a roadmap for the development of a community-based participatory monitoring approach in the context of developing an EAP in a resource-limited setting. While the first aim could already be partially addressed in the preceding chapters (see \ref*{subsec:practical_examples} and \ref*{subsec:water_sources}), the hypothesis is further supported by the results presented below, particularly in Stages 1 and 2. %adjust. main aim roadmap -> applicability in the second half of the results
% this is the intro to the second half, partly at least - need to rewrite afterwards
The stages follow the SSF (see chapter \ref*{subsec:ssf}) and are presented in the coming sections. Stages 1 and 2 are concerned with the context, problem definition and feasibility assessment. A short conclusion is attached to each of these stages, justifying further progress to the next phase. Stage 3 \textit{Designing the Project} incorporates the \acrshort*{slmc} and is the central element of this chapter by presenting the roadmap in detail. Stage 4 to 6 contain brief descriptions of the SRCS's community building and volunteer recruitment, data management practices with an emphasis on NYSS and the evaluation procedures which are currently in place, respectively. The results are based on the conducted interviews, the project analysis and the literature analysis. The full interview and questionnaire transcripts are presented in Appendix \ref*{TODO:} and the protocol to the data analysis in Appendix \ref*{TODO:}. The chapter is concluded with a summary of the key findings.


In the second part, insights are presented that could be determined in the course of this work for the respective activities or products.


\section{Design Roadmap}
% today:::
% 1. Methodology angleichen (afterwards)
% 2. this chapter --> pull all guidelines together + double check my own stuff in stage 3 SLMC
% If I get all of this done by today, it's really good.
% Saturday:
% write the second half of the results
% Sunday:
% reread methodology, adjust introduction and write result summary possibly prepare discussion

% discussion:
% read some guidelines mate. just read some guidelines. Bed time. can't think anymore.
% last big chapter! Woop Woop!!
% discuss context e.g. water security, water scarcity and drought
% discuss FbF and CS
% 




Design of the roadmap --> SSF, SLMC, Guidelines + projects - literature review and interviews

While the first hypothesis could be verified (if applicable) the deductive/inductive roadmap design is the focus of this part -

second half: application of this roadmap with a focus on Stages 1-3, whereas in stage 3, the focus lies on the identification of applicable and fruitful water level trigger development focussed on the water source type berkad.  

% it is common in case studies to integrate the results with the discussion as the context is very important and changes during the study often influence the results during the work. This work separates the chapters in principle but some brief explanations are given within the result chapter in order to be able to follow along with its content and decisions on the development of the framework. (?needed?)

% this grouping was based on multiple findings.
In the first part
--> roadmap design, 

1. Project requirements -> could be grouped into four groups -> Foundation, Innovations, Knowledge and Management 
% -> Innovation -> it is believed, that every context is special and that some sort of adjustments and innovations must be made every time - varying extend but nonetheless (not many papers talked about this though)
% foundation: transition of CBS --> no CBS without thorough embeddedness into the context with a good and well laid out basis - policy development is somewhat in between innovation and foundation. but e.g. the framework of Day 2009 lies a good foundation for this -> though, not that new and mana IWRM concepts exist as well. Though, as already discussed, often way too complicated to be applicable in practice --> Day found to be extensive but also limited enough to be realisable.
% knowledge, gathers everything that needs to be known for the entire project. Is answered across all stages 
% management -> what the goal is: initial and regular information gathering --> good to know what the knowledge is needed for
2. adjustments/expansions of the six stages





geared towards EAP and FbF integration -- or better it grows out of this. --> therefore, a lot of groundwork has already happened and structures are in place

Don't integrate in stage 1 \& 2 --> no bias/not limit into the exploration phase and feasibility phase --> project requirements are for the design, and not for the exploration stage or feasibility assessment in stage 2.


\subsection{Stage 1: Context and Problem identification}
%%%%%%%%%%%%%%%%%%%%%%%%%%%%%%%%%%%%%%%%%%%%%%%%%%%%%%%%%%%%%%%%%%%%%%%%%%%%%%%%%%%%%%%%%%%%%%%%%%%
%%%%%%%%%%%%%%%%%%%%%%%%%%%%%%%%%%%%%%%%%%%%%%%%%%%%%%%%%%%%%%%%%%%%%%%%%%%%%%%%%%%%%%%%%%%%%%%%%%%
%%%%%%%%%%%%%%%%%%%%%%%%%%%%%%%%%%%%%%% !!! SECTION 1 !!! %%%%%%%%%%%%%%%%%%%%%%%%%%%%%%%%%%%%%%%%%
%%%%%%%%%%%%%%%%%%%%%%%%%%%%%%%%%%%%%%%%%%%%%%%%%%%%%%%%%%%%%%%%%%%%%%%%%%%%%%%%%%%%%%%%%%%%%%%%%%%
%%%%%%%%%%%%%%%%%%%%%%%%%%%%%%%%%%%%%%%%%%%%%%%%%%%%%%%%%%%%%%%%%%%%%%%%%%%%%%%%%%%%%%%%%%%%%%%%%%%
%%%%%%%%%%%%%%%%%%%%%%%%%%%%%%%%%%%%%%%%%%%%%%%%%%%%%%%%%%%%%%%%%%%%%%%%%%%%%%%%%%%%%%%%%%%%%%%%%%%
This first stage is the exploration phase of the overall project \autocite{citizenscience.govBasicStepsYour}. This is where the environment of this project is established, in which itself is embedded. It is aimed at identifying prevailing conditions in all areas that may be covered or touched by the project. Even if this stage does not go into too much detail, the identification efforts must be thorough and as complete as possible. Oversights in this stage can have serious consequences in later stages. To enable this identification, project boundaries must first be defined by the overall objective and the problems to be solved, which also take into account challenges, positive and negative constraints as well as resource requirements. In addition, potential key stakeholders should be involved from the beginning and comparable projects and datasets need to be carefully identified and analysed to avoid duplication. \autocite{citizenscience.govBasicStepsYour,fraislCitizenScienceEnvironmental2022,minkmanCitizenScienceWater2015}. Based on this information, possible solutions can be derived and hypotheses or research questions formulated \autocite{fraislCitizenScienceEnvironmental2022, silvertownNewDawnCitizen2009}. Additionally, evaluation practices and sustainability considerations should be integrated into the project as early as possible although they are only defined in detail at a later stage \autocite{fraislCitizenScienceEnvironmental2022}.

\subsection{Stage 2: Assess the feasibility of the Citizen Science approach.}
%%%%%%%%%%%%%%%%%%%%%%%%%%%%%%%%%%%%%%%%%%%%%%%%%%%%%%%%%%%%%%%%%%%%%%%%%%%%%%%%%%%%%%%%%%%%%%%%%%%
%%%%%%%%%%%%%%%%%%%%%%%%%%%%%%%%%%%%%%%%%%%%%%%%%%%%%%%%%%%%%%%%%%%%%%%%%%%%%%%%%%%%%%%%%%%%%%%%%%%
%%%%%%%%%%%%%%%%%%%%%%%%%%%%%%%%%%%%%%% !!! SECTION 2 !!! %%%%%%%%%%%%%%%%%%%%%%%%%%%%%%%%%%%%%%%%%
%%%%%%%%%%%%%%%%%%%%%%%%%%%%%%%%%%%%%%%%%%%%%%%%%%%%%%%%%%%%%%%%%%%%%%%%%%%%%%%%%%%%%%%%%%%%%%%%%%%
%%%%%%%%%%%%%%%%%%%%%%%%%%%%%%%%%%%%%%%%%%%%%%%%%%%%%%%%%%%%%%%%%%%%%%%%%%%%%%%%%%%%%%%%%%%%%%%%%%%
%%%%%%%%%%%%%%%%%%%%%%%%%%%%%%%%%%%%%%%%%%%%%%%%%%%%%%%%%%%%%%%%%%%%%%%%%%%%%%%%%%%%%%%%%%%%%%%%%%%

The \acrlong{cs} approach is not feasible for all kinds of projects. Certain criteria should be met and the feasibility of design, implementation and operation must be ensured and tailored to the decision-making processes that the project aims to influence. Fundamentally, a \acrshort{cs} a project must contribute to achieving the defined objectives and solving the problems, while also providing benefits to the participants, e.g. in terms of knowledge, community or recreational value \autocite{escaeuropeancitizenscienceassociationTenPrinciplesCitizen2015,fraislCitizenScienceEnvironmental2022}. The feasibility assessment needs to consider various factors and constraints in more detail than in stage 1 to identify information and management gaps accordingly. The goal should be clarified along with potential sub-goals and related products which need to match the derived gaps and the capacity of the implementing organisation \autocite{ifrcCommunityBasedSurveillanceGuiding2017,minkmanCitizenScienceWater2015}. This organisational capacity depends on financial, human and technical resources, knowledge and experiences, embedding in decision-making networks and structures, and the organisational commitment and dedication of those involved. \autocite{fraislCitizenScienceEnvironmental2022,ifrcCommunityBasedSurveillanceGuiding2017}. The importance of securing (long-term) funding is also highlighted by many guidelines \autocite{cervoniImplementingIntegratedWater2008,minkmanCitizenScienceWater2015,sharpeCommunityBasedEcological2006, whitelawEstablishingCanadianCommunity2003}. Existing datasets and potential tools need to be analysed and assessed for suitability. In addition to the positive constraints, the \autocite{ ifrcCommunityBasedSurveillanceGuiding2017} has defined negative \textit{red flag} constraints which, when they occur, should stop further design developments until they can be resolved appropriately. These \textit{red flags are:}

\begin{itemize}
    \item A need does not exist
    \item The community does not want the project
    \item Barriers and fears of: information usage, data sharing, applied technology and different cultural beliefs
    \item Insufficient capacities regarding financial and human resources, knowledge, experience and phone coverage
    \item No support of key stakeholders
    \item No or insufficient response possibilities
\end{itemize}

\subsection{Stage 3: Designing the Project}
%%%%%%%%%%%%%%%%%%%%%%%%%%%%%%%%%%%%%%%%%%%%%%%%%%%%%%%%%%%%%%%%%%%%%%%%%%%%%%%%%%%%%%%%%%%%%%%%%%%
%%%%%%%%%%%%%%%%%%%%%%%%%%%%%%%%%%%%%%%%%%%%%%%%%%%%%%%%%%%%%%%%%%%%%%%%%%%%%%%%%%%%%%%%%%%%%%%%%%%
%%%%%%%%%%%%%%%%%%%%%%%%%%%%%%%%%%%%%%% !!! SECTION 3 !!! %%%%%%%%%%%%%%%%%%%%%%%%%%%%%%%%%%%%%%%%%
%%%%%%%%%%%%%%%%%%%%%%%%%%%%%%%%%%%%%%%%%%%%%%%%%%%%%%%%%%%%%%%%%%%%%%%%%%%%%%%%%%%%%%%%%%%%%%%%%%%
%%%%%%%%%%%%%%%%%%%%%%%%%%%%%%%%%%%%%%%%%%%%%%%%%%%%%%%%%%%%%%%%%%%%%%%%%%%%%%%%%%%%%%%%%%%%%%%%%%%
%%%%%%%%%%%%%%%%%%%%%%%%%%%%%%%%%%%%%%%%%%%%%%%%%%%%%%%%%%%%%%%%%%%%%%%%%%%%%%%%%%%%%%%%%%%%%%%%%%%

This stage builds on the identified context and conditions of stage 1 and the feasibility study of stage 2 and creates the broader framework for more specific work in stage 4, 5, and 6. The goals and research questions are considered again and finally specified and formulated in alignment with the projects, participants and stakeholders interests and aims \autocite{conradReviewCitizenScience2011,fraislCitizenScienceEnvironmental2022,minkmanCitizenScienceWater2015}. Previous assumptions should be backed up as much as possible and made explicit \autocite{silvertownNewDawnCitizen2009} and biases need to be addressed \autocite{escaeuropeancitizenscienceassociationTenPrinciplesCitizen2015,fraislCitizenScienceEnvironmental2022}. "Legal and ethical issues surrounding copyright, intellectual property, data sharing agreements, confidentiality, attribution, and the environmental impact of any activities" need to be considered \autocite{escaeuropeancitizenscienceassociationTenPrinciplesCitizen2015}. The design need to be thoroughly embedded in the context and anchored by policies, preferably in an \acrlong{iwrm} initiative \autocite{cervoniImplementingIntegratedWater2008,sharpeCommunityBasedEcological2006} and ongoing \acrshort{fbf} implementations and efforts. The 'light' \acrshort{iwrm} \textit{community-based water resource management} framework by \autocite{dayCommunitybasedWaterResources2009} is recommended as a starting point as it is geared towards practical feasibility (see section \ref*{subsubsec:foundation}). The integration into the \acrshort{fbf} is done by targeting data management (stage 5) on indicators which support intended triggers and respective anticipatory actions \autocite{ifrcCommunityBasedSurveillanceGuiding2017}. Adequate and scientifically justified thresholds and monitoring methods may need to be developed and triangulation data sets identified, assessed and integrated. \newline
A structured interconnected foundation needs to be created for the integration of community building (Stage 4), data management (stage 5) and evaluation and iterative improvement procedures (stage 6) in the current project. Community building encompasses recruitment, training, task specifications and participant benefits, motivations, feedback mechanisms and stakeholder acknowledgements. It is also concerned with the broader frame of partaking and collaborating non-governmental organisations and government bodies at all levels \autocite{conradReviewCitizenScience2011}. Data management practices should be oriented on already proven concepts of comparable projects, identify and define data collection, transmission, storage and analysis aims, formats, and types \autocite{fraislCitizenScienceEnvironmental2022,gualazziniEWEAEarlyWarning2021,ifrcCommunityBasedSurveillanceGuiding2017}. Consideration also needs to be given to the ways in which the new data from this project can be publicly displayed, accessed and used to improve completeness, timeliness and overall quality of information and decision-making processes \autocite{conradMeaningfulCommunityBasedEcological2006}. Evaluation and iterative improvement procedures are concerned with pre-defining success metrics which should be considered during the entire project design and operation.\newline
In this third stage, the project requirements catalogue presented in the coming section \ref*{subsec:project_requirements} is recommended for integration to reduce mental load in the further design process.

\subsection{Stage 4: Community Building}
%%%%%%%%%%%%%%%%%%%%%%%%%%%%%%%%%%%%%%%%%%%%%%%%%%%%%%%%%%%%%%%%%%%%%%%%%%%%%%%%%%%%%%%%%%%%%%%%%%%
%%%%%%%%%%%%%%%%%%%%%%%%%%%%%%%%%%%%%%%%%%%%%%%%%%%%%%%%%%%%%%%%%%%%%%%%%%%%%%%%%%%%%%%%%%%%%%%%%%%
%%%%%%%%%%%%%%%%%%%%%%%%%%%%%%%%%%%%%%% !!! SECTION 4 !!! %%%%%%%%%%%%%%%%%%%%%%%%%%%%%%%%%%%%%%%%%
%%%%%%%%%%%%%%%%%%%%%%%%%%%%%%%%%%%%%%%%%%%%%%%%%%%%%%%%%%%%%%%%%%%%%%%%%%%%%%%%%%%%%%%%%%%%%%%%%%%
%%%%%%%%%%%%%%%%%%%%%%%%%%%%%%%%%%%%%%%%%%%%%%%%%%%%%%%%%%%%%%%%%%%%%%%%%%%%%%%%%%%%%%%%%%%%%%%%%%%
%%%%%%%%%%%%%%%%%%%%%%%%%%%%%%%%%%%%%%%%%%%%%%%%%%%%%%%%%%%%%%%%%%%%%%%%%%%%%%%%%%%%%%%%%%%%%%%%%%%
This section pertains to the identification and establishment of all relevant factors associated with the participants, community, network, and organizational and governmental decision-makers. Understanding participants characteristics and motivations as the primary data collectors and contributors to the project is detrimental to the sustained success of the project. Their characteristics include, among others, the educational level, skills and demographics \autocite{cervoniImplementingIntegratedWater2008,fraislCitizenScienceEnvironmental2022}. The motivation aspects comprise elements of interests, engagement, acknowledgements and overall gained benefits. The first set of characteristics can be addressed by training, supervision and provision of feedback, especially for new participants \autocite{escaeuropeancitizenscienceassociationTenPrinciplesCitizen2015,fraislCitizenScienceEnvironmental2022,minkmanCitizenScienceWater2015,sharpeCommunityBasedEcological2006}. Providing feedback to the actual contributions but also in terms of how the contributions influence planning and management decision-making processes and outcomes can positively influence the motivational aspects \autocite{conradMeaningfulCommunityBasedEcological2006,conradReviewCitizenScience2011,whitelawEstablishingCanadianCommunity2003}. Creating wider public engagement and interest can further enhance motivational factors such as recognition and community building \autocite{conradMeaningfulCommunityBasedEcological2006}. Community events and networking bring further social benefits, trust and belonging and open up opportunities to engage directly with decision-makers and make them aware of what this project is and why it exists \autocite{conradMeaningfulCommunityBasedEcological2006,fraislCitizenScienceEnvironmental2022,sharpeCommunityBasedEcological2006}. Decision-makers, such as respective water and risk related government ministries and agencies, should be integrated in the process and design right from the beginning as especially local and regional leaders can help to implement and operate the project on site \autocite{gualazziniEWEAEarlyWarning2021,ifrcCommunityBasedSurveillanceGuiding2017}. They can furthermore help to sensitize the community, manage expectations and inform about and support in dealing with oppositely motivated stakeholders (I1). \autocite{conradMeaningfulCommunityBasedEcological2006} encourages the perspective of integrating the project into the management as an opportunity and not as a threat. Inclusion of legal and ethical guidelines should also happen in this stage, but has its focus in the upcoming stage 5 \autocite{fraislCitizenScienceEnvironmental2022, ifrcCommunityBasedSurveillanceGuiding2017,minkmanCitizenScienceWater2015}.

\subsection{Stage 5: Data Management}
%%%%%%%%%%%%%%%%%%%%%%%%%%%%%%%%%%%%%%%%%%%%%%%%%%%%%%%%%%%%%%%%%%%%%%%%%%%%%%%%%%%%%%%%%%%%%%%%%%%
%%%%%%%%%%%%%%%%%%%%%%%%%%%%%%%%%%%%%%%%%%%%%%%%%%%%%%%%%%%%%%%%%%%%%%%%%%%%%%%%%%%%%%%%%%%%%%%%%%%
%%%%%%%%%%%%%%%%%%%%%%%%%%%%%%%%%%%%%%% !!! SECTION 5 !!! %%%%%%%%%%%%%%%%%%%%%%%%%%%%%%%%%%%%%%%%%
%%%%%%%%%%%%%%%%%%%%%%%%%%%%%%%%%%%%%%%%%%%%%%%%%%%%%%%%%%%%%%%%%%%%%%%%%%%%%%%%%%%%%%%%%%%%%%%%%%%
%%%%%%%%%%%%%%%%%%%%%%%%%%%%%%%%%%%%%%%%%%%%%%%%%%%%%%%%%%%%%%%%%%%%%%%%%%%%%%%%%%%%%%%%%%%%%%%%%%%
%%%%%%%%%%%%%%%%%%%%%%%%%%%%%%%%%%%%%%%%%%%%%%%%%%%%%%%%%%%%%%%%%%%%%%%%%%%%%%%%%%%%%%%%%%%%%%%%%%%

Legal and ethical laws, guidelines, and standards especially in terms of privacy and data security need to be respected. This also includes taking into account informal, community and cultural practices during all phases of data management \autocite[017]{ifrcCommunityBasedSurveillanceGuiding2017}. These phases encompass the planning and design, the collection on site, the transmission, storage, \acrfull{qa} and \acrfull{qc} as well as subsequent analysis, presentation and dissemination of the outcomes \autocite{fraislCitizenScienceEnvironmental2022}.\newline
All of these phases need to match the capacities of the organisation and of the contributing participants \autocite{ifrcCommunityBasedSurveillanceGuiding2017,minkmanCitizenScienceWater2015}. Furthermore, all practices should focus on the end use and application of the data in supporting decision-making and follow the principle of data minimisation \autocite{edpsGlossaryEuropeanData2023,ifrcCommunityBasedSurveillanceGuiding2017,minkmanCitizenScienceWater2015}. In this stage, the planning of the data management procedures enters its detailed phase, is based on the established structure in stage 3 and results in precise methods, techniques, protocols and scripts. The methods should be simple, well-designed, peer-reviewed and standardised, while being fit for purpose \autocite{fraislCitizenScienceEnvironmental2022,ifrcCommunityBasedSurveillanceGuiding2017,silvertownNewDawnCitizen2009,whitelawEstablishingCanadianCommunity2003}. \acrshort{qa} and \acrshort{qc} procedures should ideally be integrated in every phase and follow the same high standards as the methods \autocite{fraislCitizenScienceEnvironmental2022,mackechnieRoleBigSociety2011,sharpeCommunityBasedEcological2006,silvertownNewDawnCitizen2009}. Financial and human investments and resources need to be specified and parameters about the technical infrastructure such as architecture, storage, analysis, transmission and collection protocols and methods need to be defined \autocite{fraislCitizenScienceEnvironmental2022,sharpeCommunityBasedEcological2006}. For data collection "the least intrusive and most cost-effective method available" is recommended \autocite[27]{ifrcCommunityBasedSurveillanceGuiding2017}. The applied tools for data collection and transmission need to meet the available resources and technical abilities of the participant on site \autocite{ifrcCommunityBasedSurveillanceGuiding2017,minkmanCitizenScienceWater2015}. Network coverage should be taken into account when implementing SMS or other remote devices but the \autocite[26]{ifrcCommunityBasedSurveillanceGuiding2017} notes, that "it is now increasingly rare to have absolutely no network access, but a bicycle messenger or another local communication system will also work". An automated, technical remote solution should be the preferred solution, but simple SMS and phone calls directly to the relevant manager or via traditional communication networks can also work, especially in cases where transmission speed is not of utmost importance \autocite{gualazziniEWEAEarlyWarning2021,ifrcCommunityBasedSurveillanceGuiding2017}. The requirements for data storage solutions include secure storage, good maintenance options and high up-times. The position of the servers and the ownership of the data can lead to disagreements with local stakeholders and should be well communicated (I2). Before the analysis of the data, robust \acrshort{qa} and \acrshort{qc} measures should ensure high quality of the collected data \autocite{fraislCitizenScienceEnvironmental2022,sharpeCommunityBasedEcological2006}. The integration of other data sources for information triangulation is recommended. The analysis is the centrepiece of the data management and should follow the objectives of the overall project. The outcomes should be made publicly available unless prevented by security or privacy concerns \autocite{escaeuropeancitizenscienceassociationTenPrinciplesCitizen2015,sharpeCommunityBasedEcological2006}. The type of presentation can show aggregated data, but should take into account the information needs of decision-makers. As with the issue of ownership of data, the procedures for sharing and presentation can also lead to disagreements and need to be  managed sensitively. (I3)\autocite{ifrcCommunityBasedSurveillanceGuiding2017}. %TODO: check if this is not a () ().

\subsection{Stage 6: Evaluation and iterative improvements}
%%%%%%%%%%%%%%%%%%%%%%%%%%%%%%%%%%%%%%%%%%%%%%%%%%%%%%%%%%%%%%%%%%%%%%%%%%%%%%%%%%%%%%%%%%%%%%%%%%%
%%%%%%%%%%%%%%%%%%%%%%%%%%%%%%%%%%%%%%%%%%%%%%%%%%%%%%%%%%%%%%%%%%%%%%%%%%%%%%%%%%%%%%%%%%%%%%%%%%%
%%%%%%%%%%%%%%%%%%%%%%%%%%%%%%%%%%%%%%% !!! SECTION 6 !!! %%%%%%%%%%%%%%%%%%%%%%%%%%%%%%%%%%%%%%%%%
%%%%%%%%%%%%%%%%%%%%%%%%%%%%%%%%%%%%%%%%%%%%%%%%%%%%%%%%%%%%%%%%%%%%%%%%%%%%%%%%%%%%%%%%%%%%%%%%%%%
%%%%%%%%%%%%%%%%%%%%%%%%%%%%%%%%%%%%%%%%%%%%%%%%%%%%%%%%%%%%%%%%%%%%%%%%%%%%%%%%%%%%%%%%%%%%%%%%%%%
%%%%%%%%%%%%%%%%%%%%%%%%%%%%%%%%%%%%%%%%%%%%%%%%%%%%%%%%%%%%%%%%%%%%%%%%%%%%%%%%%%%%%%%%%%%%%%%%%%%
Evaluation is an ongoing effort and should be considered at all phases of the project. The structure of the project should allow for evaluation procedures and subsequent implementation of the derived recommendations at all design stages and during operation \autocite{fraislCitizenScienceEnvironmental2022,ifrcCommunityBasedSurveillanceGuiding2017}. Success metrics as well as the response upon those need to be defined and agreed upon before the implementation of the project \autocite{fraislCitizenScienceEnvironmental2022,gualazziniEWEAEarlyWarning2021}. The ninth principle of \autocite{escaeuropeancitizenscienceassociationTenPrinciplesCitizen2015} states, that "citizen science programmes are evaluated for their scientific output, data quality, participant experience and wider societal or policy impact". Underlying structures, practices and efforts can thus be configured and adapted into a more effective and efficient design and process.The integration of these assessments thus helps to progressively improve the use of data by decision-makers as well as the methods used \autocite{fraislCitizenScienceEnvironmental2022}.


\section{Project requirements}\label{subsec:project_requirements}
%%%%%%%%%%%%%%%%%%%%%%%%%%%%%%%%%%%%%%%%%%%%%%%%%%%%%%%%%%%%%%%%%%%%%%%%%%%%%%%%%%%%%%%%%%%%%%%%%%%
%%%%%%%%%%%%%%%%%%%%%%%%%%%%%%%%%%%%%%%%%%%%%%%%%%%%%%%%%%%%%%%%%%%%%%%%%%%%%%%%%%%%%%%%%%%%%%%%%%%
%%%%%%%%%%%%%%%%%%%%%%%%%%%%%%%%%% !!! PROJECT REQUIREMENTS !!! %%%%%%%%%%%%%%%%%%%%%%%%%%%%%%%%%%%
%%%%%%%%%%%%%%%%%%%%%%%%%%%%%%%%%%%%%%%%%%%%%%%%%%%%%%%%%%%%%%%%%%%%%%%%%%%%%%%%%%%%%%%%%%%%%%%%%%%
%%%%%%%%%%%%%%%%%%%%%%%%%%%%%%%%%%%%%%%%%%%%%%%%%%%%%%%%%%%%%%%%%%%%%%%%%%%%%%%%%%%%%%%%%%%%%%%%%%%
%%%%%%%%%%%%%%%%%%%%%%%%%%%%%%%%%%%%%%%%%%%%%%%%%%%%%%%%%%%%%%%%%%%%%%%%%%%%%%%%%%%%%%%%%%%%%%%%%%%

The stages of the roadmap outlined above give a good overview of what needs to be done and in what order. However, the process-oriented structure makes it difficult not to overlook important information, as there is no more detailed and grouped listing of this. This catalogue attempts to close this gap. The catalogue of project requirements presented here is divided into four groups of information that include \autocite{minkmanCitizenScienceWater2015}'s derived goals and sub-goals and follows the layer structure of the \acrshort{slmc}. For each goal, products along with their required activities have been defined in line with the above roadmap. \newline
The first group \textbf{Knowledge Base}, see figure \ref*{TODO:,} contains everything that should be known for the project. It thereby serves as a knowledge base being filled throughout the above-mentioned stages. Activities are primarily about identification and information collection. The second group \textbf{Groundwork} contains everything on which the actual implementation of the project must be based on (see figure \ref*{TODO:}). It includes the goals \textit{Awareness Raising, Public Education and Policy Development}. The first two goals relate to products of information and sensitization measures for the target community, contributing volunteers and involved stakeholders. Activities focus on consulting with local elders and key stakeholders, and then informing and raising awareness among the whole community before the project starts. The third goal, \textit{Policy Development} is about \acrshort{iwrm} measures, required products and their development. The activities aim to embed these into existing local management practices and procedures through identification and agreements with all affected and involved stakeholders. In order to facilitate this, new policy developments and adjustments may be required which overlap with the third group of \textbf{Innovations}, see figure \ref*{TODO:}. New developments of suitable and scientifically sound methods to meet specific challenges of the project tasks are summarised under the goal of \textit{Method Improvements}. The last group \textbf{Management} contains all that needs to be done to successfully develop and implement a \acrshort{cs} water source mapping and monitoring project (see figure \ref*{TODO:}). The goal \textit{Management Improvements} is further subdivided into products that are initially necessary once, recurrently and products that are required to embed the project in the context. The focus of the activities is on making decisions, developing procedures based on the \textbf{Groundwork} and \textbf{Knowledge Base} groups and implement developed methods from the \textbf{Innovations} group.\newline
The order of the sections does not necessarily reflect the processing order but rather indicates dependencies whereas e.g. not everything need to be known in order to proceed with method development. The same applies to the order of products and activities (from top to bottom). However, to stay in this example, everything that, directly or indirectly, influences or is influenced by this certain method should be known. Therefore, stages 1 and 2 of the design roadmap should be completed before proceeding further with this catalogue in order to have a profound understanding of the local circumstances and their interconnectedness and interdependences.\newline
The following sections briefly reason and describe the overall idea and content of each group. The activities are mostly already outlined in the above-described 6 stage roadmap, therefore the focus lies on the products of each goal. 

\subsection{The Knowledge Base: What needs to be known}
%%%%%%%%%%%%%%%%%%%%%%%%%%%%%%%%%%%%%%%%%%%%%%%%%%%%%%%%%%%%%%%%%%%%%%%%%%%%%%%%%%%%%%%%%%%%%%%%%%%
%%%%%%%%%%%%%%%%%%%%%%%%%%%%%%%%%%%%% !!! SUBSECTION 1 !!! %%%%%%%%%%%%%%%%%%%%%%%%%%%%%%%%%%%%%%%%
%%%%%%%%%%%%%%%%%%%%%%%%%%%%%%%%%%%%%%%%%%%%%%%%%%%%%%%%%%%%%%%%%%%%%%%%%%%%%%%%%%%%%%%%%%%%%%%%%%%
This Knowledge Base covers information and knowledge of all stages and is sub-divided into multiple products that each covers certain aspects of the project. 
The products (A), (B), (C) and (D) cover the baseline information of all stages. (A) includes information obtained in Stages 1 and 2 and, if this project is developed under an EAP, also the information from the overarching EAP assessment and feasibility study. (B) is the repository for all information related to the \acrlong{cs} community and participant management, (C) covers all topics of data management and \acrshort{qc} and \acrshort{qa} integration and (D) relates to evaluation and improvement procedures. These products consciously resemble Stages 4, 5 and 6, and should be included right at the beginning of Stage 3. The structural basis for each of the subsequent stages is laid in Stage 3, and by bundling the information from the beginning, follow-up in the subsequent stages is simplified and streamlined.\newline
Products (E) to (H) are more specific for the mapping and monitoring of water sources in an anticipatory context. (E) integrates all information for the first phase of mapping and collection of key information on each water source including corresponding methods, while (F) includes all knowledge and methods on regular indicator monitoring. Together, this information enables actions on products (G) and (H). All potential \acrshortpl{aa} are first collected in product(G) and then narrowed down to those that can be triggered by thresholds related to water sources. The information about these thresholds along with triangulation data for the respective triggers is comprised in product (H).

\missingfigure{Knowledge tree diagram with listing}

\subsection{The Groundwork: What needs to be built on}\label{subsubsec:groundwork}%%%%%%%%%%%%%%%%%%%%%%%%%%%%%%%%%%%%%%%%%%%%%%%%%%%%%%%%%%%%%%%%%%%%%%%%%%%%%%%%%%%%%%%%%%%%%%%%%%%
%%%%%%%%%%%%%%%%%%%%%%%%%%%%%%%%%%%%% !!! SUBSECTION 1 !!! %%%%%%%%%%%%%%%%%%%%%%%%%%%%%%%%%%%%%%%%
%%%%%%%%%%%%%%%%%%%%%%%%%%%%%%%%%%%%%%%%%%%%%%%%%%%%%%%%%%%%%%%%%%%%%%%%%%%%%%%%%%%%%%%%%%%%%%%%%%%
The Groundwork relates to everything, that needs to be in place, before the actual mapping and monitoring of the water sources starts. It is about giving the community all the information and knowledge to decide, manage and act on their own, only limited by the available physical and financial resources. Important is, that knowledge should not opposed on the communities but rather developed in close cooperation with the affected actors, their knowledge and practices. Therefore, the first product (A) of the goals \textit{Awareness Raising \& Public Educatin} (Fig. \ref*{TODO:}) is about providing all the necessary background information to the community in order to allow them to make profound and informed decisions and contributions. Based on this, product (B) summarizes activities to address misunderstandings, reservations as well as expectations and concern handling. For this, the early involvement of community leaders or elders may be beneficial (I1). Product (C) refers to the sharing of information gathered in other phases with the community to keep their knowledge of prevailing and evolving circumstances up to date and to enable informed decision-making. In order to act on the information, product (D) summarizes the identification and transfer of information about water management and saving opportunities. Product (E) highlights the consideration, that even a single community is not homogeneous and that various groups with different interests exist within. Knowledge and measures to account for this are collected under this product.\newline

\missingfigure{Awareness Raising \& Public Educatin}

Products A to E of goal \textit{Policy Implementation}, figure \ref*{TODO:}, represent \autocite{dayCommunitybasedWaterResources2009}'s light community-based \acrshort{iwrm} framework. The framework starts with the products (A) \& (B) by identifying and assessing prevailing circumstances and requirements. From this, community water usage targets are derived and agreed upon in (C). Management guidelines, and priorisation plans are pre-defined and implemented in products (D) and (E). Product (F) goes beyond the framework and addresses measures related to data security and privacy.

\missingfigure{Policy Developments}

\subsection{The Innovations: What needs to be invented}%%%%%%%%%%%%%%%%%%%%%%%%%%%%%%%%%%%%%%%%%%%%%%%%%%%%%%%%%%%%%%%%%%%%%%%%%%%%%%%%%%%%%%%%%%%%%%%%%%%
%%%%%%%%%%%%%%%%%%%%%%%%%%%%%%%%%%%%% !!! SUBSECTION 1 !!! %%%%%%%%%%%%%%%%%%%%%%%%%%%%%%%%%%%%%%%%
%%%%%%%%%%%%%%%%%%%%%%%%%%%%%%%%%%%%%%%%%%%%%%%%%%%%%%%%%%%%%%%%%%%%%%%%%%%%%%%%%%%%%%%%%%%%%%%%%%%
New innovations may be required to successfully achieve the project objectives under the given conditions and in the given context. These developments are grouped separately because the development of appropriate, scientifically sound and context-aware methods can require a great deal of financial, material and human resources and should therefore receive specific consideration. In addition, the development of new methods is often a separate process that runs alongside the actual design and implementation efforts. However, there are now also many projects, guidelines and frameworks from which experience and best practices can be transferred. The developments can therefore also be an amalgamation of the old, tailored to new circumstances.\newline
The products represent potential areas of such required innovations or adaptations. These can include areas of the initial and regular data collection, transmission, storage and analysis as well as, the determination of suitable thresholds, and their categorization for respective triggers.

\missingfigure{INNOVATIONS}

\subsection{The Management: What needs to be done}%%%%%%%%%%%%%%%%%%%%%%%%%%%%%%%%%%%%%%%%%%%%%%%%%%%%%%%%%%%%%%%%%%%%%%%%%%%%%%%%%%%%%%%%%%%%%%%%%%%
%%%%%%%%%%%%%%%%%%%%%%%%%%%%%%%%%%%%% !!! SUBSECTION 1 !!! %%%%%%%%%%%%%%%%%%%%%%%%%%%%%%%%%%%%%%%%
%%%%%%%%%%%%%%%%%%%%%%%%%%%%%%%%%%%%%%%%%%%%%%%%%%%%%%%%%%%%%%%%%%%%%%%%%%%%%%%%%%%%%%%%%%%%%%%%%%%

The group around management tasks is primarily concerned with the initial processes and responsibilities at the start of the project, tasks that need to be done regularly during operation, and specifications for solidly embedding the project in prevailing processes and practices. These tasks are about evaluating and processing identified and collected information to develop and make decisions about what elements, practices or structures to implement. The initial areas of concern (A) range from handling specific context related circumstances, to the development of training materials and evaluation procedures. Activities contributing to regular required products (B) are mostly concerned about providing training and supervision and implementing improvements derived from evaluations and feedback. Product (C) includes both initial and ongoing tasks, but is focused on embedding the project in the local context. It addresses the implementation of the \textit{Groundwork} on community level and its integrations into prevailing local, regional and national structures, practices and stakeholder networks to make it relevant for decision-making.

\missingfigure{MANAGEMENT STUFF}


\section{Framework Application: Focus Somaliland}

In this part, the above outlined roadmap and project requirements are applied to the FIXME:(last) research question and context of this work. Therefore, the results in this section are primarily geared towards the aim to identify and implement adequate water level threshold monitoring of the water source type of berkads in order to trigger respective \acrshortpl{aa}. The third stage is structured following the project requirement catalogue. The stages 4 to 6 refer their findings to the catalogue. However, because of their specific foci, they are not structured according to the catalogue. The above already displayed tree-diagrams can also additionally be found in the respective GitHub repository in Appendix \ref*{TODO:}. This provides the reader with a simultaneous reference option, as the products and activities will only be mentioned by their abbreviations and not by their full name, to enhance readability.

%begin of second half
\subsection{Stage 1: Context and Problem Identification}\label{subsec:stage1_appl}
%%%%%%%%%%%%%%%%%%%%%%%%%%%%%%%%%%%%%%%%%%%%%%%%%%%%%%%%%%%%%%%%%%%%%%%%%%%%%%%%%%%%%%%%%%%%%%%%%%%
%%%%%%%%%%%%%%%%%%%%%%%%%%%%%%%%%%%%%%%%%%%%%%%%%%%%%%%%%%%%%%%%%%%%%%%%%%%%%%%%%%%%%%%%%%%%%%%%%%%
%%%%%%%%%%%%%%%%%%%%%%%%%%%%%%%%%%%%%%% !!! SECTION 1 !!! %%%%%%%%%%%%%%%%%%%%%%%%%%%%%%%%%%%%%%%%%
%%%%%%%%%%%%%%%%%%%%%%%%%%%%%%%%%%%%%%%%%%%%%%%%%%%%%%%%%%%%%%%%%%%%%%%%%%%%%%%%%%%%%%%%%%%%%%%%%%%
%%%%%%%%%%%%%%%%%%%%%%%%%%%%%%%%%%%%%%%%%%%%%%%%%%%%%%%%%%%%%%%%%%%%%%%%%%%%%%%%%%%%%%%%%%%%%%%%%%%
%%%%%%%%%%%%%%%%%%%%%%%%%%%%%%%%%%%%%%%%%%%%%%%%%%%%%%%%%%%%%%%%%%%%%%%%%%%%%%%%%%%%%%%%%%%%%%%%%%%

The brief water source data analysis along with given context, resource restriction, stakeholders and comparable projects as well as problem and goal statements of the interviewees are covered in this first stage. It also builds on the preceding case study area section \ref{sec:case_area} in chapter \ref{chapter2}.\newline
% current situation
The current drought and water scarcity situation in Somaliland has now lasted five years and has greatly impacted the water sector in Somaliland in terms of quantity and quality (I1). I1 describes the current crisis as \textit{"huge and response activities are being overwhelmed by the need"} which will lead to \textit{"commercialization and overpricing of fresh water"} further exacerbating the situation. This was underlined by I3 who describes the current water situation of the rural population as \textit{"whatever source of water they can find is what they have"}. I3 further mentions, that the people sometimes \textit{won't have enough water to wash their hands"} or for other necessary things and that \textit{"they [then] don't think of what kind of water they can get, whether it's bad or something like that"} but only focus on having at least something to drink. Increased water shortage because of bad quality was also reported in the literature (see section \ref*{subsec:water_quality}). Water can potentially be contaminated at all stages of the water collection process, from the initial abstraction of water, through transport and storage, to the use of the water (I3). Water quality is difficult to assess on site, as the the colour is not a good indicator and other parameters can only be determined technically which required equipment and training (I1, I3). Furthermore, I3 confirms the reports in literature that contamination of water in berkads, through their shared use with animals, can happen even before water withdrawal. This depends very much on the construction method and how it is used. The rehabilitation as well as training how to adequately use a berkad are already activities of the \acrshort*{srcs} (I3) and besides the water quality, the quantity also depends on the kind of construction and of withdrawal. According to I3, supply period can range from one month to half a year, so information on these parameters is crucial for estimating the potential duration of supply (I3).\newline
Water monitoring and management is not seen as a problem in urban areas as there is an agency responsible for water supply but the problem is primarily located in rural and nomad areas, where 70\,\% of the people live, according to I3. \autocite{republicofsomaliaCountryProfile20212021}, on the other hand, estimates an urban and semi-urban, sedentary population rate of 53\,\%.\newline
% problem
The selection of beneficiaries for response activities of the \acrshort*{srcs} are currently conducted on the basis of a preceding joint priority setting with the government. This prioritisation is based on \textit{"assumed vulnerability per community based on Number of \acrfull{idp} camps in the area, number of women headed families, predicted IPC classifications etc."} (I1.2). \acrfullpl{aa} have not been implemented due to the \textit{"already prevailing crisis"} where the \textit{"needs are [already] dire and the current \acrshort*{srcs}s focus is on response mechanisms to address the already visible impacts of drought"} (I1.2). Thus, the overall \textit{"end goal [of this project] will be to counter water shortages"} (I1) proactively but \textit{"there has been any actions yet due to the fact that there is no water monitoring and trigger mechanism in place"} (I1). The monitoring was itself hampered by the fact that \textit{"Berkads location data is currently missing"} (I1). This statement compares well with the experience of the current project team working on the EAP implementation and the assessment of available data sets (A. Schulze-Eckel \& A. Schauss, personal communication, November 2022).\newline
Table \ref*{TODO} shows all available data sets of water sources in Somaliland, provided by \acrshort*{swalim}. % and OSM (?)

\missingfigure{table swalim water sources} % see joplin
% maybe add OSM data as well.. quite some work though as not one single tag can be used.. could be a good comparison though. Only one data provider is a bit mehh.. and no OSM is hard to explain and I got the python scripts already.. need to refine the methodology though. Should be worth it. + add settlements to it

The spatial distribution of the datasets across Somalia is relatively balanced, with focal points in the regions with many or larger settlements. Based on the \acrshort{swalim} settlement data set, there are currently 2123 settlements in Somaliland. These settlement data are mostly from the years 2002 and 2006. The total number of water sources varies widely between and over time of the other data sets. The timeliness of the data also has a wide range, from relatively few pieces of data from 2019 in the 2020 dataset to data from the 1980s in the same dataset is much represented. The 2022 dataset misses information about timeliness altogether and the other datasets all have many blank entries as well. Furthermore, many water sources are labelled as 'abandoned' or 'non-functional', e.g. in the 2022 dataset those are 147 out of 685.\newline
The data sets are fed by many sources and actors e.g. FAOSWALIM or other UN organisation, \acrshort*{mowr}, \acrshort*{nadfor} and other NGOs which constructed some water sources in some communities (I1). These actors, along with the community and their elders, local government representatives, SRCS and their volunteers and private berkad owners are also the potential stakeholders of the mapping and monitoring of berkads. Here, I1 notes, that besides the \acrshort*{srcs}, the \acrshort*{mowr}, \acrshort*{nadfor} and the constructing NGOs are the most important stakeholders. The \acrshort*{mowr} and NGOs have the technical expertise in construction, rehabilitation and monitoring and \acrshort*{nadfor} has a comparable community level programme for monitoring \textit{"livestock body condition, market prices as well as weather variables"} (I1).\newline
Other comparable programmes exist from \acrshort*{ocha}, \acrshort*{brcis} and the \acrshort*{cbs} programme run by the Ministry of Health, the \acrshort{srcs} and the \acrshort*{nrc}. While these projects may broadly be comparable in terms of focus on \acrshortpl{aa}, none of the Interviewees know of a project that conducts similar things to this works approach (I1, I2, I3). I2 also suggests that the projects are close enough to each other to pass on experiences and recommendations, e.g. from the \acrshort{moh} to the \acrshort{mowr}, in order to overcome initial scepticism and reluctance.\newline
Challenges, limitations and requirements are mentioned in areas of privately owned berkads, community expectation handling and the dissolution of misconceptions as well as potentially already overstretched SRCS staff and volunteers (I1). I1 mentioned, that private owners of berkads may prevent the volunteer from gaining access to their berkad which would result in less information on the one hand but also in tension in the community on the other. Giving information from the community to someone else may also generally require some explanation (I2). Furthermore, some \textit{"information on past details per particular geographical areas"}(I1.2) can be difficult to access, as \textit{"Somalis are highly mobile communities"} (I1.2). The monitoring could furthermore develop \textit{"hugh expectations from the communities as there is the ongoing drought. Whenever there is monitoring of resources, communities believe this should be followed up by instant aid"} (I1).\newline
Addressing some of the challenges mentioned above, the \textit{"community elders should be engaged before the start of the mapping and monitoring as they will help dispel misconceptions about the project"} (I1) and the \textit{"ministry of water resources should be in the loop during the entire project duration"} (I1). Nonetheless, the \textit{"community and SRCS goals match as both focus on closing the knowledge [gap] currently existing"} (I1) in regard to the number, status of ownership, location and capacity of the berkads per community, district and regional level. This will \textit{"inform decision-makers on the priority areas to focus on"} (I1). Therefore, I1 expects that this information from the site triangulated with weather forecasts can help to form robust triggers to take appropriate and informed \acrlongpl{aa} before critical water levels are reached.

% RQ / hypothesis is not formulated --> overall RQ is already there.. 

\subsection{Stage 2: Assess the feasibility of the Citizen Science approach.}\label{subsec:stage2_appl}
%%%%%%%%%%%%%%%%%%%%%%%%%%%%%%%%%%%%%%%%%%%%%%%%%%%%%%%%%%%%%%%%%%%%%%%%%%%%%%%%%%%%%%%%%%%%%%%%%%%
%%%%%%%%%%%%%%%%%%%%%%%%%%%%%%%%%%%%%%%%%%%%%%%%%%%%%%%%%%%%%%%%%%%%%%%%%%%%%%%%%%%%%%%%%%%%%%%%%%%
%%%%%%%%%%%%%%%%%%%%%%%%%%%%%%%%%%%%%%% !!! SECTION 2 !!! %%%%%%%%%%%%%%%%%%%%%%%%%%%%%%%%%%%%%%%%%
%%%%%%%%%%%%%%%%%%%%%%%%%%%%%%%%%%%%%%%%%%%%%%%%%%%%%%%%%%%%%%%%%%%%%%%%%%%%%%%%%%%%%%%%%%%%%%%%%%%
%%%%%%%%%%%%%%%%%%%%%%%%%%%%%%%%%%%%%%%%%%%%%%%%%%%%%%%%%%%%%%%%%%%%%%%%%%%%%%%%%%%%%%%%%%%%%%%%%%%
%%%%%%%%%%%%%%%%%%%%%%%%%%%%%%%%%%%%%%%%%%%%%%%%%%%%%%%%%%%%%%%%%%%%%%%%%%%%%%%%%%%%%%%%%%%%%%%%%%%

In this stage, the practical capacities, and applicability and suitability of the \acrshort*{cbm} and \acrshort*{mcs} approach for community-based water monitoring were examined in this context. The SRCS has 249 paid employees, of which 30 work in the risk management and \acrlong*{aa} domain and an additional 1500 volunteers are "evenly spread" across the country (I1). I3 emphasises the \textit{"good relations and good reputation"} that \acrshort*{srcs} has within the communities, making them \textit{"one of the most trusted organizations in the country"} which helps to do programs at community level. The 'feasibility study on Potential Use of \acrshort*{fbf} for SRCS' \autocite[44]{somaliredcrescentsocietyFeasibilityStudyPotential2022} recognizes a \textit{strong national organization} with a \textit{strong volunteer base at community level} that \textit{provides monitoring and hazard warning capacities}. Furthermore, \textit{highly skilled and experienced management staff at coordination and Branch levels} is stated. Nonetheless, \textit{minimal domestic resource mobilization} and a \textit{lack of meteorological, geo-spatial analysis, data management and IT staff} has also been detected.
% NYSS
This lack of resources and digital capacities was addressed in the \acrshort*{cbs} project by the \acrshort*{nrc} and their NYSS platform. Generally, \acrshort{cbs} is \textit{nothing new in itself and often used in health contexts} (I2). \acrshort*{cbs} in Somaliland started in Burao in 2018 with 75 community volunteers, as cholera had broken out in the same region in 2017 (I3). After the pilot was successful, \acrshort*{cbs} has since been extended to all regions but \textit{\acrshort*{srcs} only focusses on hot spot areas where they expect outbreaks to happen} (I3). This development took place over the course of a year with much feedback from the \acrshort{srcs} and NYSS is now \textit{"very effective and very supportive"} (I3). The \acrfull*{moh} was and is \textit{constantly in the loop to decide together what, how, when and who} and could gain good experiences with NYSS over the years (I2). By now, NYSS is well embedded in the local conditions and \textit{"mobile teams [...] can be deployed immediately within hours so they can do the response"} (I3) in collaboration with other partners such as the \textit{"government, \acrshort{moh}, \acrshort{who}, and other sister \acrshort{rcrc} organisations} (I3).\newline
% reasoning why this approach works
I2 mentions, that the monitoring of water sources \textit{would fit well thematically, because it [low and poor water quality] is a health risk.} and that although it would require some adjustments and considerations, it would organically expand technical expertise and functionality as it is not \textit{radically new}. Besides NYSS, being methodologically similar but different in thematic orientation, several other projects could be identified in literature, that are oriented towards the same topic but differ in their implementation and operation procedures (compare section \ref*{subsubsec:cbwm}). It can be deduced from this that \acrshort{cbm} and \acrshort{mcs} can in principle be applied to the thematic issue. Furthermore, \autocite{fraislCitizenScienceEnvironmental2022} themselves describe approaches that focus on the "in situ monitoring" of water resources and at the same time benefit the respective participant as adequate for the use of a \acrlong{cs} approach. \acrshort{srcs} does not pay their volunteers but provides training and covers travel expenses as incentives. Moreover, volunteers are generally well regarded and are selected by the community itself. Intrinsic motivation is therefore present (I2, I3).\newline
\autocite{fraislCitizenScienceEnvironmental2022} prerequisites are thus fulfilled, and the key objectives of the \acrshort{rcrc}'s \acrshort{cbs} preliminary assessment can be answered positively. The \acrshort{srcs} is an adequate partner with sufficient capacities and experiences to implement the mapping and monitoring approach and the project can thus be further developed in stage 3. 

\subsection{Stage 3: Designing the Project}\label{subsec:stage3_appl}
TODO:TODO:TODO:TODO:TODO:TODO:TODO:TODO:TODO:TODO:TODO:TODO:TODO:TODO:TODO:TODO:TODO:TODO:TODO:TODO:TODO:
%%%%%%%%%%%%%%%%%%%%%%%%%%%%%%%%%%%%%%%%%%%%%%%%%%%%%%%%%%%%%%%%%%%%%%%%%%%%%%%%%%%%%%%%%%%%%%%%%%%
%%%%%%%%%%%%%%%%%%%%%%%%%%%%%%%%%%%%%%%%%%%%%%%%%%%%%%%%%%%%%%%%%%%%%%%%%%%%%%%%%%%%%%%%%%%%%%%%%%%
%%%%%%%%%%%%%%%%%%%%%%%%%%%%%%%%%%%%%%% !!! SECTION 3 !!! %%%%%%%%%%%%%%%%%%%%%%%%%%%%%%%%%%%%%%%%%
%%%%%%%%%%%%%%%%%%%%%%%%%%%%%%%%%%%%%%%%%%%%%%%%%%%%%%%%%%%%%%%%%%%%%%%%%%%%%%%%%%%%%%%%%%%%%%%%%%%
%%%%%%%%%%%%%%%%%%%%%%%%%%%%%%%%%%%%%%%%%%%%%%%%%%%%%%%%%%%%%%%%%%%%%%%%%%%%%%%%%%%%%%%%%%%%%%%%%%%
%%%%%%%%%%%%%%%%%%%%%%%%%%%%%%%%%%%%%%%%%%%%%%%%%%%%%%%%%%%%%%%%%%%%%%%%%%%%%%%%%%%%%%%%%%%%%%%%%%%

This stage is grouped into the four project requirement groups and lays the structure for the coming stages. The overall project end goal to counter water shortages proactively is majorly hampered by missing information about the water source type berkad. Therefore, up-to-date information on water sources in Somaliland is needed. In particular, there is a lack of information on the important Berkad water sources (compare stage \ref*{subsec:stage1_appl}). Therefore, the focus of this project will be on gathering information about this specific water source type (I1). See chapter \ref*{subsec:water_sources} for further information about this water source type. I1 also named and chronologically ordered the most important fields of action that need to be realised in order to achieve this goal:

\begin{enumerate}
    \item Volunteer briefing and training
    \item Community sensitization
    \item Locating and gathering key information about berkads
    \item Determination of the water level threshold 
    \item Determining respective \acrlongpl{aa} (own addition)
    \item Monitoring of the water level
    \item Triggering \acrshortpl{aa} based on pre-defined threshold    
\end{enumerate}

I1 highlights the importance of the actions 2, 3 and 7 as critical for the overall success of the project. For the design stage, I1 emphasises the determination of the threshold, \acrlongpl{aa} and respective trigger. As I1 draws his judgement from his work as the local program manager and coordinator, this project followed these priorities during the design process. This process is grouped by the project requirement catalogue in the following. The structure of each group follows the order of the products and respective activities, emphasis is given to the activities which contribute directly to the above mentioned priorities. To minimise repetition in the scope of this work, already covered areas are only referred to and not again outlined in detail. 

% biases.. important? mention? discussion? possibly discussion. Have fun old me!

% The design of the roadmap for implementing the research aim is presented in detail in this section. This stage, as described in chapter \ref*{sec:design_framework} integrates the \acrshort{slmc} to better structure the design process. The following division of the sections is a grouping of the defined (sub-)goals of \autocite{minkmanCitizenScienceWater2015} and represents the first level of the SLMC (see Fig. \ref{TODO:}). This categorisation of the goals was supported by statements from I1 and arranged chronologically. The order of the goals in this chapter represents this timeline. The foundation of the design is represented in the first two goals \textit{Awareness Raising \& Public Education}. The sub-goals \textit{Policy Development \& Method Improvements} describe technical, social and political prerequisites, necessities and recommendations which will facilitate the central goals of this project \textit{Knowledge Building} and \textit{Management Improvements}. 

\subsubsection{The Assemblage: Knowledge Building}\label{subsubsec:assemblage}

The majority of the activities of product (A) were already covered by previous stages and chapters. (A1) was extensively outlined in section \ref*{sec:fundamental_concepts}. (A2) and (A3) were covered by stage 1 \ref*{subsec:stage1_appl}, \ref*{subsec:stage2_appl} and section \ref*{sec:case_area}. (A4) on the one hand will influence decision-making in the \acrshort{srcs} but further integration into e.g. governmental procedures could not be covered by this work. Activities A5 and A6 could also only be touched upon, but especially the topic of integration of local knowledge holds a lot of potential (see \ref*{subsec:indicators}).\newline
Product (B) could partly be covered by stage 2 \ref*{subsec:stage2_appl}. Results so far suggest that the network and individual volunteers are adequately trained, motivated and managed for the monitoring task. This sets the framework for stage 4. In terms of adequate data management (C), feasible technical solutions could be identified from other projects and their practical applicability was demonstrated by the successful \acrshort{cbs} program of the \acrshort{srcs} (C1). Further exploration in Stage 5 is thus possible. (D) current evaluation and improvement procedures could be identified and are further described in stage 6.\newline
(E1) initially important information about each berkad is summarized in figure \ref*{TODO:}. This compilation is based on section \ref*{subsec:water_sources}, and knowledge of I1, I2, and I3. The left, bold side are the information highlighted by the interviewees and summarizes the key information. The right side displays information, that may be nice to know for further analysis, but is not critical for this project.

\missingfigure{List of initially necessary berkad information}

In terms of (E2), location, storage capacity and construction information need to be identified initially and might be need to be updated when e.g. the berkad is rehabilitated. These information will be gathered by \acrshort{srcs} professionals and therefore does not need to be included in the regular monitoring routine. (E2) a report about the condition of the berkad may only be necessary once a year (I1.2) while the number of people and animals may need a weekly or monthly reporting interval, depending on the fluctuation strength (I1.2). This information should be kept comparatively up to date, as the high mobility of Somalis means that this number can change relatively regularly and has a great influence on the amount of water abstraction (I1.2).\newline
(F1) the water level of the berkad was named as constantly changing indicator which should be monitored in a weekly interval (I1.2). The realisability and adequacy of this reporting frequency was also supported by I2 (F2). (F3) The data sets for the data triangulation are adopted from the overarching \acrshort{eap} and have not yet been determined at the time of writing.\newline
Potential AAs that can be triggered in correspondence with a the surveyed information and certain water level threshold are listed below (G). 

\begin{itemize}
    \item Informing about water rationalisation and saving opportunities
    \item Information dissemination of climate and weather forecasts
    \item Distribution of drought-resistant crops
    \item Rehabilitation of berkads before the rainy season 
    \item Compensate private berkad owners to access their water
    \item Timely distribution of cash to enable communities to buy and stock fresh water
    \item Timely distribution of water purification tablets 
    \item Water trucking
\end{itemize}

Raising awareness and information dissemination need to be the foundation of this (see Groundwork \ref*{TODO:}, I1.2). The distribution of drought-resistant crops and other agricultural related actions need to be coordinated with the \acrfull{moad}. The rehabilitation of berkads before the rainy season needs to be related to seasonal triggers as this actions will help to store available rain water and won't directly help in times of acute water shortages. I1.2 notes, that the involvement of private berkad owners \textit{"could be limited as they are more concerned about their business models i.e selling of the water and preserving their berkads than being part of the overall response/Anticipatory action mechanism"}. Nevertheless, I1.2 sees potential in working with private berkad owners and suggests e.g. the rehabilitation of their berkads \textit{"in return for their involvement in response and anticipatory action activities"} as viable \acrshort{aa}. The distribution of cash is a widely applied AA in \acrshort{fbf} project and can help in many cases. Distribution of water vouchers is an alternative to direct cash and has already been used successfully in Somaliland (see section \ref*{subsec:case_eap}). Water purification tables and information for waterborne disease prevention are already disseminated by \acrshort{srcs} volunteers together with hygiene and health promotion activities but could be better targeted by more timely and localised information (I3). Direct water transport is the last solution to mitigate water scarcity. The required lead time, tangible and intangible resources, information requirements and involved roles (G2 \& G3) of these \acrshortpl{aa} are illustrated in figure \ref*{TODO:}. 

\missingfigure{AA requirements}

This list is not comprehensive and needs to be refined for each \acrshort{aa}, which is illustrated in figure \ref*{TODO:} for the \acrshort{aa} \textit{water trucking}.

\missingfigure{Water trucking tree-diagram}

Water trucking is a common measure to cope with acute water shortage in Somaliland. Information for water water trucking comes, thus far, from \acrshort{srcs} assessments, from the community themselves, government agencies, \acrshort{fsnau} or other NGOs (I3). This type of information transmission can be timely and may be incomplete. The water transport itself can take a long time and be relatively expensive due to the distance and high demand (I1, I3). It is financed by various stakeholders, including the community themselves and private donors (I3). Self-financed water trucking is especially common in the beginning of the initial phase of drought but if the people cannot afford to buy water any more and their livestock becomes weak or dies, \textit{“[...] this is the time they talk to the other NGOs or the government and say we need support [...]”} (I3). Currently, the following prioritisation of water trucking by the regional and national stakeholders is primarily based on government decisions and focusses on the most vulnerable communities (I3). This decision-making process could not be explained in more detail by (I3) other than it is a \textit{joint effort by all stakeholders}. The \acrshort{srcs} does not truck water themselves.\newline
(H1) potential water level thresholds were suggested by I1.
\begin{itemize}
    \item Empty (no water at all)
    \item Critical (1 day of water supply remaining)
    \item Low (3 days of water supply remaining)
    \item Middle (5 days of water supply remaining)
    \item Hihg (full capacity)
\end{itemize}
I1 further specified the \textit{Low} category to trigger the \acrshort{aa} of \textit{water trucking} (H2). These water levels either require the local knowledge about how long the water will last or require the analysis of the exact or categorized water level with the berkads capacity in the backend. The first option would outsource the triangulation of available resources and amount abstraction to the communities predictions. I1.2 notes, that \textit{"these kinds of predictions are good as communities usually have their own control measures to ensure equitable distribution of water e.g. how many containers per family etc. The berkads are usually locked to ensure there is controlled access to the water stored."}. The second requires good information about the berkad itself and a feasible method to interpolate this with the regularly measured information. \autocite{gualazziniEWEAEarlyWarning2021} however, proposed more seasonal focussed threshold levels for berkads (see chapter \ref*{subsec:case_eap}). (H3) the short-term thresholds may be feasible to short and fast \acrshortpl{aa}, whereas seasonal information may trigger \acrshortpl{aa} such as the rehabilitation of berkads and information campaigns.

\subsubsection{The Groundwork: Laying the foundation}
TODO:TODO:TODO:TODO:TODO:TODO:TODO:TODO:TODO:TODO:TODO:TODO:TODO:TODO:TODO:TODO:TODO:TODO:TODO:TODO:TODO:

Volunteer briefing and training together with following sensitization of the community are the first two actions that lay the foundation for the implementation of this project (see \ref*{subsec:stage3_appl}, I1). (A1) major challenges could be identified in community expectation handling and the involvement of private berkad owners (I1, I2). I1 suggested, that the \textit{"community elders should be engaged before the start of the mapping and monitoring as they will help dispel misconceptions"} (I1) and that the \acrshort{mowr} \textit{"should be in the loop during the entire project duration"}.Awareness-raising activities and dissemination of information may include knowledge of water quality improvement techniques, water conservation strategies, early warnings and a detailed explanation of the reasons for reporting before the start of the project (I1, I2). It also needs to be \textit{communicated, discussed and decided beforehand},  what happens in cases when thresholds are reached but no response is possible (I2). \textit{Otherwise, it could fall back negatively on the SRCS and the Volunteer} (I2). Furthermore, I1.2 highlights the importance, to establish \textit{"a robust feedback and complaints mechanism that ensures communities can easily relay their feedback.} right from the start. The development and implementation of product (A2), (B), (C), (D), and (E) must happen in close collaboration with local stakeholders and were thus out of scope of this work. Nonetheless, the work with the community should be fruitful as their goal to \textit{"ascertain whether these water bodies are able to withstand the demand during drought periods"} overlap with the project goals (I1).\newline
The integration of the light \acrshort{iwrm} framework developed by \autocite{dayCommunitybasedWaterResources2009} was presented in section \ref*{subsubsec:cbwm} and needs to be further discussed with local stakeholders.

% to facilitate this and further implement this roadmap --> policy (social frameworks) and methods (practical techniques and tools) need to be developed.

% As with most activities, that need close collaboration on site, this was out of scope of this study.

% \subsubsection{Integrated Water Management (Goal and Groundwork for mapping/monitoring)}

% ground work -> public education --> inform community about water shortages (?) -> create "agreement on water priorization during times of acute drought" + "a common shared agreement to prevent water resource contamination, as well as mitigating over-abstraction." (Day, 2009: 52)

% BUT:
% “Communities may not always represent a homogeneous, consenting group” ([Day, 2009, p. 52](zotero://select/groups/4773535/items/YWSNQ8A2)) ([pdf](zotero://open-pdf/groups/4773535/items/ETPCI5RI?page=7&annotation=NMA3CWDS))
% --> different interests, resources, knowledge, access rights, own hierarchy + status level

% how about private water sources?
% “Strikingly, the traditional influence of sheikhs in water management does not, however, extend to supervision of private wells or boreholes operated by farmers or individual landowners. Sh” ([Day, 2009, p. 55](zotero://select/groups/4773535/items/YWSNQ8A2)) ([pdf](zotero://open-pdf/groups/4773535/items/ETPCI5RI?page=10&annotation=FXZW8RBU))

% AA -> manage water and prioritize effectively (e.g. start with it now)
% “This is significant because in drought-prone environments little at tempt is made to inform communities about their available ground water resources and there is minimal emphasis or preparation for monitoring groundwater fluctuations, prioritizing water usage dur ing periods of hardship or assisting communities to develop basic contingency plans with relief agencies or local authorities acting as a back-stop to provide support during periods of acute” ([Day, 2009, p. 51](zotero://select/groups/4773535/items/YWSNQ8A2)) ([pdf](zotero://open-pdf/groups/4773535/items/ETPCI5RI?page=6&annotation=6C4BAHSE))

% “hardship. When describing community water projects, 'sustainability' is often referred to. In reality the immediate challenge is to introduce sound steward ship of water resources to assist communities to resist and recover from drought or low and variable rainfall” ([Day, 2009, p. 51](zotero://select/groups/4773535/items/YWSNQ8A2)) ([pdf](zotero://open-pdf/groups/4773535/items/ETPCI5RI?page=6&annotation=DA9BJCDN))

% “r of distinct advantages of engaging in commu nity-based water resource mana” ([Day, 2009, p. 51](zotero://select/groups/4773535/items/YWSNQ8A2)) ([pdf](zotero://open-pdf/groups/4773535/items/ETPCI5RI?page=6&annotation=85TSGZK9))

% FIGURE of the framework: “Figure 2. Community-based water resource manageme” ([Day, 2009, p. 59](zotero://select/groups/4773535/items/YWSNQ8A2)) ([pdf](zotero://open-pdf/groups/4773535/items/ETPCI5RI?page=14&annotation=BAMSY255))
% %% --> this framework as 'base'/foundation/Early Action/AA?

% “Local water users often possess detailed indigenous knowledge related to water resources, water needs and historical change that has occurred related to water use. 
% Water users recognize that water is a fundamental component of their subsistence-based livelihoods, which helps to weave rela tionships between water users. 
% Communities are able to monitor agreed water usage on a daily basis, as part of their daily activities. 
% Communities often have historical mechanisms for conflict and dispute resolution related to water resource management, which may require continued support and assistance to evolve and adapt to global challenges. 
% Effective water management requires community participation; this principle is well understood in development li” ([Day, 2009, p. 52](zotero://select/groups/4773535/items/YWSNQ8A2)) ([pdf](zotero://open-pdf/groups/4773535/items/ETPCI5RI?page=7&annotation=4RVGKM5R))

% “However, responsible planning for drought mitigation at community level is often omitted.” ([Day, 2009, p. 47](zotero://select/groups/4773535/items/YWSNQ8A2)) ([pdf](zotero://open-pdf/groups/4773535/items/ETPCI5RI?page=2&annotation=S4ACQRPL))

% “Communities frequently remain excluded from any basic capacity building, centred on water resource management, as part of a localized Integrated Water Resources Management (IWRM) programme.” ([Day, 2009, p. 47](zotero://select/groups/4773535/items/YWSNQ8A2)) ([pdf](zotero://open-pdf/groups/4773535/items/ETPCI5RI?page=2&annotation=WBL45NKR))

% “Community-based water resources management” ([Day, 2009, p. 47](zotero://select/groups/4773535/items/YWSNQ8A2)) ([pdf](zotero://open-pdf/groups/4773535/items/ETPCI5RI?page=2&annotation=RQLJMKL7))


\subsubsection{The Innovations: Developments and Improvements}
TODO:TODO:TODO:TODO:TODO:TODO:TODO:TODO:TODO:TODO:TODO:TODO:TODO:TODO:TODO:TODO:TODO:TODO:TODO:TODO:TODO:

Besides the identification of a way to integrate \acrshort{iwrm} practices into local procedures and structures, more technical solutions were also required to be developed or adapted. The actual outcome are part of stage 5. (A1) important primary information about the water source berkad could be identified by the method of expert interviews. (B1) and (C1) could be identified through interviews and literature analysis. However, a thorough assessment and subsequent adjustment of the practical suitability will only be possible in a pilot study. This is especially true for their evaluation (A2) and (B1). Several data management methods could be identified in literature, see section \ref*{subsec:mcs} and section \ref{subsec:case_eap}





--> here: methods are developed for the goals e and f --> not done yet, therefore no methods in those goals. Here not methods and so on, as it was not the focus of this work (?)

scientifically sound methods

--> transfer from other projects

water level measuring
-> different options
-> folding rule or rule with categorizations (3 or 5) --> categorization is kinda key (is it?)
-> or local knowledge (how long will it last) (outlook) b

method: generally: coded SMS

%%%%%%%%%%%%%%%%%%%%%%%%%%%%%%%%%%% INTERVIEW I1 RICHARD %%%%%%%%%%%%%%%%%%%%%%%%%%%%%%%%%%%%%%%%%%
RESULTS
o	v) Determining the water level trigger
specifically important:
o	Determining the water levels to trigger action.



%%%%%%%%%%%%%%%%%%%%%%%%%%%%%%%%%%% INTERVIEW I1.2 RICHARD %%%%%%%%%%%%%%%%%%%%%%%%%%%%%%%%%%%%%%%%






%%%%%%%%%%%%%%%%%%%%%%%%%%%%%%%%%%% INTERVIEW I2 JUNG %%%%%%%%%%%%%%%%%%%%%%%%%%%%%%%%%%%%%%%%%%%%%

(26) Weekly updates about the water level from the volunteer should work.
(28, 29) It should be communicated, that reports will be checked by the supervisor in order to prevent false reports in hope of more water. If this happens frequently, a solution must be conceptualized.





%%%%%%%%%%%%%%%%%%%%%%%%%%%%%%%%%%% INTERVIEW I3 BELEDI %%%%%%%%%%%%%%%%%%%%%%%%%%%%%%%%%%%%%%%%%%%








\subsubsection{The Management: Mapping \& Monitoring}
TODO:TODO:TODO:TODO:TODO:TODO:TODO:TODO:TODO:TODO:TODO:TODO:TODO:TODO:TODO:TODO:TODO:TODO:TODO:TODO:TODO:
%one example how to execute the entire SLMC tree (see Miro)
% important: name the mapping campaign! planned as SRCS staff -> educated personal, no rookies, no volunteers.

with preceding procedures including awareness raising, public education, policy development and method improvements determined and accomplished now proceed to the actual mapping task of Berkeds.


The combination of both mechanisms might be the best choice for the kreeirung of a staggered trigger.

The trigger will follow the overall trigger methodology of the \acrshort{eap} and 
trigger

%%%%%%%%%%%%%%%%%%%%%%%%%%%%%%%%%%% INTERVIEW I1 RICHARD %%%%%%%%%%%%%%%%%%%%%%%%%%%%%%%%%%%%%%%%%%
RESULTS:
o	vi) water levels monitoring
o	vii) triggering action based on water levels


%%%%%%%%%%%%%%%%%%%%%%%%%%%%%%%%%%% INTERVIEW I1.2 RICHARD %%%%%%%%%%%%%%%%%%%%%%%%%%%%%%%%%%%%%%%%


I1.2
-	Water levels in berkeds could be a good indicator, however it cannot be a stand alone indicator. This has to be combined by meteorological forecasts and local knowledge as well.



berked rehabilitation (SRCS), water trucking (other agencies), distribution of water purification tablets (SRCS), multi purpose cash, awareness campaigns related to hygiene promotion (SRCS). Regarding Anticipatory actions, there has been any actions yet due to the fact that there is no water monitoring and trigger mechanism in place.

Assistance/response is based on the initial prioritization of target areas that SRCS conducts. The prioritization is based on assumed vulnerability per community based on Number of IDP camps in the area, number of women headed families, predicted IPC classifications etc.
% limitations
The SRCS in consultation with the government select target communities based on a pre existing selection /vulnerability criteria based on either number of IDP camps etc


Activities such as berked rehabilitation are done in consultations with the communities and SRCS branches who flag/identify berkeds in need of repairing. Repairing may consists of re roofing/ roofing, and brickwork to strengthen the structure. The berkeds are meant to capture run off water in case of rainfall incidences. In cases where there hasnt been rains for a prolonged time then water trucks are deployed to deliver water to the communities. Cash has been an important modality to address the water shortages. In the current prevailing drought, water and food insecurity crisis, water is now being sold by private players. So the cash has come in handy to t least enable the communities to buy fresh water for drinking. Water sources such as dug wells are often contaminated as livestock i,e camels, goats also drink water from those same water bodies as well.

Water levels in berkeds could be a good indicator, however it cannot be a stand alone indicator. This has to be combined by meteorological forecasts and local knowledge as well.
% triangulation with other sources. -> trust and quality as always

-proposed AA:
-	Awareness raising and information dissemination should be more on informing the communities on how to improve water quality at local level e,g boiling before drinking. Involving private berked owners is also feasible however their involvement could be limited as they are more concerned about their business models i.e selling of the water and preserving their berkeds than being part of the overall response/Anticipatory action mechanism. Nevertheless there is the potential to work closely with the private berked owners. This can be done through rehabilitation of their privately owned berkeds in return for their involvement in response and anticipatory action activities related to addressing water shortages.

yes, i) timely distribution of cash to enable communities to buy and stock fresh water
ii) timely distribution of water purification tablets
iii) timely rehabilitation of other water sources such as boreholes

- TRIGGER
These water levels are ideal i.e 
o	Empty (no water at all) 
o	Critical (1 day of water supply remaining), 
o	Low (3 days of water supply remaining), 
o	Middle (5 days of water supply remianing) 
o	High (full capacity)

-	Low category should trigger AA (he possibly meant water trucking and/or cash)

- MONITORING INTERVAL
•	Water level (daily monitoring)
•	Berked condition (annually)
•	Number of people accessing the water form the berked (weekly/monthly)

- WATER QUALITY
Water quality is difficult to monitor at community level as it is a technical activity. Unless if the SRCS through the branch staff are equipped with water testing equipment as well as training them on the water parameters to be tested.
+ not aware of any feasible water quality tests on the ground.. (should be further investigated though..)


% No Answer to Water Trucking

%%%%%%%%%%%%%%%%%%%%%%%%%%%%%%%%%%% INTERVIEW I2 JUNG %%%%%%%%%%%%%%%%%%%%%%%%%%%%%%%%%%%%%%%%%%%%%
(16) fusion with other datasets can be challenging and is a lot of work


(25) Might be beneficial to bring the MoH in contact with the MoWR in order to transfer their experiences with CBS to the new approach.
(37) One to three codes for regular monitoring should be alright but not more, as more codes make it more complicated and will narrow down the choice which Volunteer to take. 

(30, 31) Sending photos would also be possible, though smartphones and internet are often not available. Julia Jung is not supporting the distribution of smartphones for ‘several reasons’

(35) Maybe communicate a timeframe how long it will take until water arrives so they can plan for themselves.

(9, 49) in special occasion up to 7 numbers
(10) small notes that explain the codes and their order
(27) Other or more information besides the codes could be clarified via phone and inserted by the supervisor manually.

(26) Weekly updates about the water level from the volunteer should work.
(28, 29) It should be communicated, that reports will be checked by the supervisor in order to prevent false reports in hope of more water. If this happens frequently, a solution must be conceptualized.

%%%%%%%%%%%%%%%%%%%%%%%%%%%%%%%%%%% INTERVIEW I3 BELEDI %%%%%%%%%%%%%%%%%%%%%%%%%%%%%%%%%%%%%%%%%%%
-(10) SRCS only focusses on hot spot areas where they expect outbreaks to happen. They do not cover all of the country.
-(11, 14) Response happens together with other partners such as the government, MoH, WHO, and other sister RCRC organisations.
-(13) Reports are verified by the regional officer.
-(15) “SRCS has mobile teams who can be deployed immediately within hours so they can do the response.”

-(70) Water monitoring in urban areas is not a problem. There is an agency that is responsible for water supply. “So we don't have any issue with that.”
-(71) Contrary, “when it comes to the rural areas, and nomad areas, is the way we have the problem”
-(72, 62) water can get contaminated, “For example, when you are taking from the source, and again, when you are traveling with the water, and again, when you are storing the water in your home, or when you are using even the water”
-(73) “water may be clear in colour, but when we are using it, it is contaminated. So we cannot decide by the colour”
-(26) “Whatever source of water they can find is what they have.”
-(43) SRCS has a good connection to the local communities and thus they get direct feedback about water shortages.

% WATER TRUCKING
-(74) before putting water in the berked, they initially clean it.
-(74) water trucking “can take a long distance to the, for example, main cities” and is thus costly and requires lots of time.
-(75) but still, “When it comes to the water trucking, it depends” à no universal solution will work
-(76, 77) Thus far, information come from SRCS assessments, community themselves, government or FSNAU (Food Security and Nutrition Analysis Unit) (https://fsnau.org/)
-(78) Prioritisation is based on the government + “all these efforts together they decide where these resources” are going including SRCS. -> Most vulnerable in the area are prioritised and that mostly depends.
-See (81) FbF
-(41) “Sometimes there was water trucking and, you know, the SRCS or the other organizations, even the commercial or trade people, they were supporting to the communities who are in need”

% FbF
-(7) Volunteers provide Oral Rehydration Salts and aquatabs.
-(18) If the water level of berkeds becomes less or scarce, the volunteers provide hygiene and health promotion activities.
-(79) Sometimes the people/community/some households buy water trucking themselves by bringing their money together. They do this in the initial phase.
% !!
-(80) “people they are depending on their livestock” but “if there is a drought the livestock become weak or die” and the people cannot afford to buy water (trucking). “[...] this is the time they talk to the other NGOs or the government and say we need support [...]”
-(81) Water trucking is not only financed by the government and NGOs but also by normal people (“good willers” who then try to gather money and buy water. They “distribute according to the need” and look at the magnitude of the problem and where it exists - refer this water to the to the community”.
-(19, 20, 21) In case of water shortage, volunteers tell people how to prevent waterborne diseases by providing hygiene and health promotion as well as water purification and water boiling actions.




%%%%%%%%%%%%%%%%%%%%%%%%%%%%%%%%%%%%%%%%%%%%%%%%%%%%%%%%%%%%%%%%%%%%%%%%%%%%%%%%%%%%%%%%%%%%%%%%%%%
%%%%%%%%%%%%%%%%%%%%%%%%%%%%%%%%%%%%%%%%%%%%%%%%%%%%%%%%%%%%%%%%%%%%%%%%%%%%%%%%%%%%%%%%%%%%%%%%%%%
%%%%%%%%%%%%%%%%%%%%%%%%%%%%%%%%%%%%%%% !!! SECTION 4-6 !!! %%%%%%%%%%%%%%%%%%%%%%%%%%%%%%%%%%%%%%%
%%%%%%%%%%%%%%%%%%%%%%%%%%%%%%%%%%%%%%%%%%%%%%%%%%%%%%%%%%%%%%%%%%%%%%%%%%%%%%%%%%%%%%%%%%%%%%%%%%%
%%%%%%%%%%%%%%%%%%%%%%%%%%%%%%%%%%%%%%%%%%%%%%%%%%%%%%%%%%%%%%%%%%%%%%%%%%%%%%%%%%%%%%%%%%%%%%%%%%%
%%%%%%%%%%%%%%%%%%%%%%%%%%%%%%%%%%%%%%%%%%%%%%%%%%%%%%%%%%%%%%%%%%%%%%%%%%%%%%%%%%%%%%%%%%%%%%%%%%%

\subsection{Stage 4: Community Building} % volunteers + SRCS + training - etc.

These sections were not the main focus of this work as feasibility assessment and subsequent design of a roadmap was the primary focus. Furthermore, for stage 4 and 5, groundwork is completed otherwise and therefore a theoretical and practical foundation is already well established. The volunteer recruitment process and community building procedure of the \acrshort*{srcs} is outlined in Stage 4 and potential data management tools are described in Stage 5. Locally implemented procedures and strategies for Stage 6 Evaluation are briefly presented thereafter. A short summary of the results concludes this chapter.

% skip this and make stage 4-6 sections of their own. Does not fit to the rest of the structure otherwise.. and it's not necessary to go in to too much detail

%%%%%%%%%%%%%%%%%%%%%%%%%%%%%%%%%%% INTERVIEW I1 RICHARD %%%%%%%%%%%%%%%%%%%%%%%%%%%%%%%%%%%%%%%%%%


paid employees does the SRCS have in total?
o	249
Anticipatory Actions?
o	30
•	How many volunteers does the SRCS have?
o	1500
Volunteers spread across the country? 
-> Volunteers are evenly spread across the country.
However some regiosn have inactive volunteers due to less activities there whilst some regions have active volunteers due to the amount of project work being undertaken there



%%%%%%%%%%%%%%%%%%%%%%%%%%%%%%%%%%% INTERVIEW I1.2 RICHARD %%%%%%%%%%%%%%%%%%%%%%%%%%%%%%%%%%%%%%%%






%%%%%%%%%%%%%%%%%%%%%%%%%%%%%%%%%%% INTERVIEW I2 JUNG %%%%%%%%%%%%%%%%%%%%%%%%%%%%%%%%%%%%%%%%%%%%%

-	(66) Meanwhile, refreshers are not conducted monthly anymore as the volunteers know their business by now. 
-	(112, 119) Volunteers are from within the community and chosen by the elders. They do not get incentives from the SRCS. Their incentives are helping their community and the travels for the trainings.
-	Preliminary trainings, supervision and regular refreshers are important and useful.
-	(120) Volunteers are mostly women as they stay in the community and do not travel as much as men.

Stakeholder > Communities
-	(88) Reasons for reporting must be explained in detail before the start of it.
-	(90) They will have expectations of this project.

(44) No money is given either to the MoH nor to Volunteers.

(53) Supervisor can validate reports e.g. vie phone.
(54) Great success factors are the training, supervision and regular refreshers by the supervisors.


%%%%%%%%%%%%%%%%%%%%%%%%%%%%%%%%%%% INTERVIEW I3 BELEDI %%%%%%%%%%%%%%%%%%%%%%%%%%%%%%%%%%%%%%%%%%%
-(5, 6, 7) Oral Rehydration Points are in the community and run by volunteers of the community who get trained and then support the community by education and promotion of e.g. WASH activities and going to the actual treatment centers.
-(19, 20, 21) In case of water shortage, volunteers tell people how to prevent waterborne diseases by providing hygiene and health promotion as well as water purification and water boiling actions.
-(23) They are trained “not to wait until the people become fall sick, but in a professional mechanism”
-(24) Volunteers “do awareness raising, hygiene and health promotion sessions by doing, for example, group sessions by visiting house to house, visiting to meeting and all these things.”
-(22) SRCS distributes aqua tablets per month “to the volunteers so that they can manage at community level if there is a case.”

-(19, 20, 21) In case of water shortage, volunteers tell people how to prevent waterborne diseases by providing hygiene and health promotion as well as water purification and water boiling actions.
-(23) They are trained “not to wait until the people become fall sick, but in a professional mechanism”
-(24) Volunteers “do awareness raising, hygiene and health promotion sessions by doing, for example, group sessions by visiting house to house, visiting to meeting and all these things.”
-(22) SRCS distributes aqua tablets per month “to the volunteers so that they can manage at community level if there is a case.”
-(43) SRCS has good relations with the communities.
-(44) “community leaders are the one who tells the SRCS or other partners or the government that there is a water shortage”
-(45) “Somalis they support a lot each other when it comes to the disasters or something like that.“ – “everybody in the community whether in the urban or rural areas is participating to support each other”
-(46)”SRCS has a good reputation and image at community levels” “it is one of the most trusted organization in the country. So it is one of the most trusted organization in the country. So there is a strong relation at community level. So that it helps us also to do this program as community level.”
-(48) Response happens together with the MoH.
-(49) SRCS spends a lot of time on community bond building.
-(50) “SRCS, what we do is to provide any necessary support at the community level.”
Volunteers
-„when we are recruiting, I can't say, I cannot say recruiting. When we are you know going to get volunteers of that community, we go to the community.“
-(52) The person must be willing to be a volunteer as the SRCS is not paying them.
-(52, 53) Volunteers have a good reputation in the community and are selected by the community or the committees in that community based on their criteria’s.
-(54, 55, 56) After the selection, SRCS is doing a small assessment about e.g. reading and writing skills and then provide training to them on the basis of the CBS program (health promotions, coding, etc.).
-(57) After the training, the volunteers are send back to their communities and start working there.
-(59) In the community the community leaders, community committees and also the community health committees are important. “They are the one who are supporting” the SRCS on site. There is also a good collaboration between these groups, the volunteer and the SRCS.
-Volunteers teach mothers, and communities about health, how to prevent water contamination, etc.

\subsection{Stage 5: Data Management}

Some local conditions and requirements, such as the importance of the education level of the volunteer, e.g. in regard to literacy, could 


\subsection{CBS and NYSS} (NYSS questionnaire was already developed, but never used (see Appendix XYZ))
-> no data collection tool (!!) 
% I need to include some technical details and stuff somewhere.. or at least name the alternatives to NYSS.. could do that in the result part as Julia Jung talked about that stuff as well

%%%%%%%%%%%%%%%%%%%%%%%%%%%%%%%%%%% INTERVIEW I1 RICHARD %%%%%%%%%%%%%%%%%%%%%%%%%%%%%%%%%%%%%%%%%%





%%%%%%%%%%%%%%%%%%%%%%%%%%%%%%%%%%% INTERVIEW I1.2 RICHARD %%%%%%%%%%%%%%%%%%%%%%%%%%%%%%%%%%%%%%%%






%%%%%%%%%%%%%%%%%%%%%%%%%%%%%%%%%%% INTERVIEW I2 JUNG %%%%%%%%%%%%%%%%%%%%%%%%%%%%%%%%%%%%%%%%%%%%%



GOAL OF NYSS:
(41) The goal of NYSS was the provision of a simple data collection tool
(44, 46) The goal was not the collection of data nor forecasting, the goal was to create a platform for Early Warning.
(76) goal: investigation and response from the MoH.

% I2 good experiences with the method are available
CBS
(1) Nothing new in itself, often used in health contexts
(2) Can be a problem to send information about community members – possibly also about water sources? 
(5) CBS does not generally have a negative connotation in Somalia, but could be (?) -> talk to MoH to translate experiences to MoWR
(104) MoH has good experiences with NYSS in the health sector. These experiences should be translated and widened on other topics.
(36) Monitoring water sources would work well with the CBS system and fit well with the overall theme of health risks.


-	(71) MoH in the loop right from the beginning to find out what else is going on, can often be combined with an interview for the assessment.


-	(75, 95) Acceptance and understanding of the MoH for the introduction of NYSS is often an issue. Also because a lot of organizations want to integrate their own tools and instruments – may be just too many.
-	(97) MoH discussed a lot and still discusses a lot.
-	(98) Trust in the MoH is not high. That was one result of the evaluation.
-	(102) MoH is part of the ToT (Training of Trainers)
(103) Julia Jung does not know of other organizations, which would conduct similar things. Except MSF is active in Somalia/Somaliland but focussed on direct health responses. Further information should be provided by the Ministry or latest by the community itself.
(23) before CBS, there is always an assessment

Stakeholder > Ministry of Health
-	(71) MoH in the loop right from the beginning to find out what else is going on, can often be combined with an interview for the assessment.

-	(25, 99) Might be beneficial to bring the MoH in contact with the MoWR in order to transfer their experiences with CBS to the new approach.
-	(102) MoH is part of the ToT (Training of Trainers)

(2) Can be a problem to send information about community members – possibly also about water sources? 
(36) Monitoring water sources would work well with the CBS system and fit well with the overall theme of health risks.

(28, 29) It should be communicated, that reports will be checked by the supervisor in order to prevent false reports in hope of more water. If this happens frequently, a solution must be conceptualized.

(30, 31) Sending photos would also be possible, though smartphones and internet are often not available. Julia Jung is not supporting the distribution of smartphones for ‘several reasons’

-	(3, 12, 20) Excel, SMS and manual insert into excel works as well. NYSS biggest advantage: lots of automatization
-	(40) Kobo would possibly be the best alternative if it doesn’t work with NYSS.

-	(105) WHO is interested in NYSS.
-	(93) IFRC used and uses Kobo.
-	(94) IFRC decides about the development of NYSS.

(8) difference to most other tools: possible with a basic phone. (19) which is necessary because CBS is often used in regions, where no Smartphones are accessible. More complex input requires also more educated volunteers.
(12) The good side on NYSS is, that everything is automated
(43) NYSS itself is relatively new. March 2020 was the first time it was used in its current form.

(16) fusion with other datasets can be challenging and is a lot of work
(87) integration of this into NYSS is work and it needs to be discussed who does it and who pays for it.

-	(49) Automatic integration with other data, e.g. from the Ministry is laborious and can be complicated. Though it’s doable.

-	(58) The platform is not made for individual, personal data.
-	(64) data collection in itself is not possible via NYSS.
-	(56) Server from NYSS is in Ireland for data protection reasons and its easier to maintain for the developers.
-	(57) The server location can be an issue for the Ministry as they do not have control over the data. The ownership of the data lies with the National Society. 



%%%%%%%%%%%%%%%%%%%%%%%%%%%%%%%%%%% INTERVIEW I3 BELEDI %%%%%%%%%%%%%%%%%%%%%%%%%%%%%%%%%%%%%%%%%%%

(1) CBS started in 2018 in Burao with 75 community volunteers because of the 2017 cholera outbreak in the same region
-(5, 6, 7) Oral Rehydration Points are in the community and run by volunteers of the community who get trained and then support the community by education and promotion of e.g. WASH activities and going to the actual treatment centers.
-(11, 14) Response happens together with other partners such as the government, MoH, WHO, and other sister RCRC organisations.
-(25) The idea to implement CBS came from the will to identify the cases in the community early enough to respond at community level and stop an outbreak immediately.”
-(33) NYSS was developed over the course of a year with a lot of feedback from the SRCS and is now “very effective and very supportive.”
-(34) NYSS is highly automatized and alerts are given when thresholds are reached based on geographical location.
-(36) “So any mobile you can use it.”
-(37) “No need to have a smartphone but you are using SMS.”
-(38) must be a network in that area and again SMS Eagle.”
-(39) Before the automatization, they downloaded the data from the NYSS platform and analyzed it via Excel.
-(58) Feedback on wrong reports is given by the regional supervisor very timely and they support the volunteers to send the report in the right format.






\subsection{Stage 6: Evaluation}
-> community meetings with the elders every month or so
-> feedback messages NYSS and so on
-> fraisl: ongoing effort + metrics for measuring success --> implemented in NYSS -> reliance, completeness and quality of the contributions and the activity of the volunteers




%%%%%%%%%%%%%%%%%%%%%%%%%%%%%%%%%%% INTERVIEW I1 RICHARD %%%%%%%%%%%%%%%%%%%%%%%%%%%%%%%%%%%%%%%%%%





%%%%%%%%%%%%%%%%%%%%%%%%%%%%%%%%%%% INTERVIEW I1.2 RICHARD %%%%%%%%%%%%%%%%%%%%%%%%%%%%%%%%%%%%%%%%

-	Utilise the community based SRCS volunteers to engage communities and sensitise the communoties on the riole the SRCS plays. Also establishing a robust feedback and Complaints mechanism that ensures communities can easily relay their feedback.




%%%%%%%%%%%%%%%%%%%%%%%%%%%%%%%%%%% INTERVIEW I2 JUNG %%%%%%%%%%%%%%%%%%%%%%%%%%%%%%%%%%%%%%%%%%%%%

"Usually, volunteers are chosen by the elders and the main criterium is not them being the smartest."
(28, 29) It should be communicated, that reports will be checked by the supervisor in order to prevent false reports in hope of more water. If this happens frequently, a solution must be conceptualized.

(32) Should work with prior training, supervision and feedback.

(34) Additional communication via phone is useful and necessary especially for details and instant feedback e.g. to communicate lack in response and its reasons.

-	(109) SRCS invests a lot into communication and feedback with the communities.
-	(113) SRCS are no rookies. They know how to communicate – big part of their culture.
(27) Other or more information besides the codes could be clarified via phone and inserted by the supervisor manually.

-	The option for feedback messages comes from discussions with the SRCS
-	(60) An evaluation has been conducted but not yet made public.
-	(66) SRCS knows their business. They are no rookies.

(53) Supervisor can validate reports e.g. vie phone.

-	(109) SRCS invests a lot into communication and feedback with the communities.
-	(113) SRCS are no rookies. They know how to communicate – big part of their culture.



%%%%%%%%%%%%%%%%%%%%%%%%%%%%%%%%%%% INTERVIEW I3 BELEDI %%%%%%%%%%%%%%%%%%%%%%%%%%%%%%%%%%%%%%%%%%%
-(13) Reports are verified by the regional officer.
-(58) Feedback on wrong reports is given by the regional supervisor very timely and they support the volunteers to send the report in the right format.
-(43) SRCS has a good connection to the local communities and thus they get direct feedback about water shortages.
-(43) SRCS has good relations with the communities.
-(44) “community leaders are the one who tells the SRCS or other partners or the government that there is a water shortage”


\section{Summary Results + key lessons learned (?)}
%%%%%%%%%%%%%%%%%%%%%%%%%%%%%%%%%%% INTERVIEW I1 RICHARD %%%%%%%%%%%%%%%%%%%%%%%%%%%%%%%%%%%%%%%%%%





%%%%%%%%%%%%%%%%%%%%%%%%%%%%%%%%%%% INTERVIEW I1.2 RICHARD %%%%%%%%%%%%%%%%%%%%%%%%%%%%%%%%%%%%%%%%






%%%%%%%%%%%%%%%%%%%%%%%%%%%%%%%%%%% INTERVIEW I2 JUNG %%%%%%%%%%%%%%%%%%%%%%%%%%%%%%%%%%%%%%%%%%%%%







%%%%%%%%%%%%%%%%%%%%%%%%%%%%%%%%%%% INTERVIEW I3 BELEDI %%%%%%%%%%%%%%%%%%%%%%%%%%%%%%%%%%%%%%%%%%%













%%%%%%%%%%%%%%%%%%%%%%%%%%%%%%%%%%%%%%%%%%%%%%%%%%%%
summary
--> + review to the deductive hypothesis --> could the lit review and the interviews answer this? here or discussion?




% The concluding summary is very important because it summarises your key findings and lays the foundation for the discussion chapter. Keep in mind that some readers may skip directly to this section (from the introduction section), so make sure that it can be read and understood well in isolation.
% In this section, you need to remind the reader of the key findings. That is, the results that directly relate to your research questions and that you will build upon in your discussion chapter. Remember, your reader has digested a lot of information in this chapter, so you need to use this section to remind them of the most important takeaways.



























\section{Main Section 1}

% put this in the result section.
intro to Somalia EAP: https://docs.google.com/document/d/1xUEXm8RxVHTO468KqXSAoBX-cpkPwiff/edit
https://heigit.atlassian.net/wiki/spaces/FIS/pages/1704096/Indices

current EAP stage in somalia
--> Somalia so far. But: still under development.

“Hazards Exposure and Vulnerability” ([Somali Red Crescent Society, 2022, p. 13](zotero://select/groups/4773535/items/FZ6BJHJA)) ([pdf](zotero://open-pdf/groups/4773535/items/RJKNZZZ2?page=17&annotation=IRH526LN))

“Feasibility Study on Potential Use of Forecast-based Financing (FbF) for SRCS Final Report” ([Somali Red Crescent Society, 2022, pp. -3](zotero://select/groups/4773535/items/FZ6BJHJA)) ([pdf](zotero://open-pdf/groups/4773535/items/RJKNZZZ2?page=1&annotation=KHCH33GX)





% Quality criteria for Early Action Protocols
https://heigit.atlassian.net/wiki/download/attachments/1704186/FbA-EAP-criteria-May-2022.docx?version=1&modificationDate=1677660171372&cacheVersion=1&api=v2
(summery available -> confluence)


%-----------------------------------
%	SUBSECTION 2
%-----------------------------------







%----------------------------------------------------------------------------------------
%	SECTION 2
%----------------------------------------------------------------------------------------

\section{}


%----------------------------------------------------------------------------------------
%	SECTION 1
%----------------------------------------------------------------------------------------

\section{Case study protocol}

%----------------------------------------------------------------------------------------
%	SECTION 1
%----------------------------------------------------------------------------------------

\section{Main Section 1}

%----------------------------------------------------------------------------------------
%	SECTION 1
%----------------------------------------------------------------------------------------

\section{Main Section 1}

%----------------------------------------------------------------------------------------
%	SECTION 1
%----------------------------------------------------------------------------------------



"5.3 Types of water resources monitoring
As it has been indicated above, water resources monitoring provides information on the state and trends of quantitative and qualitative characteristics of the monitored object, level and distribution of anthropogenic loads, state of ecosystems and a degree of possibility of satisfaction of various needs in water, municipal and economic.

There are three main types of water resources monitoring, used in the water resources management system:
%% main points:
local monitoring, performed for solution of specific local problems on a limited part of the water body or the territory;
global (background) monitoring, performed at man-impact free sites, or on sites with low level of anthropogenic influence. Such monitoring is performed for acquisition of information on steady natural characteristics of environmental components. The background monitoring of water bodies is used for evaluation and/or prognostication of shifts in their state caused by economic activities;
comprehensive (regime) monitoring performed at the water body observation network for determination of the actual state of the water body, for decision making on efficient use, protection, and restoration of water resources;
critical or alarm monitoring, performed at sites of high risk for immediate warning about unfavourable situations caused primarily by human activities.
In the water management system there is also a special type of monitoring for wastewater discharges to the water body."https://echo2.epfl.ch/VICAIRE/mod_4/chapt_5/main.htm

"As a rule, the following parameters are always monitored:

Water Temperature.
Transparency or Turbidity.
pH.
Conductivity.
Dissolved oxygen (DO).
Total phosphorus.
Total nitrogen.
Nitrogen, Ammonia.
Nitrogen, Nitrate.
Soluble Reactive Phosphorus.
Faecal coliform bacteria." https://echo2.epfl.ch/VICAIRE/mod_4/chapt_5/main.htm
%% --> not gonna happen but still interesting

https://www.oxfamwash.org/water/cbwrm/Oxfam%20CBWRM%20Companion,%202009.pdf
%----------------------------------------------------------------------------------------
%	SECTION 1
%----------------------------------------------------------------------------------------

\section{Main Section 1}

%----------------------------------------------------------------------------------------
%	SECTION 1
%----------------------------------------------------------------------------------------

\section{NOTES}

discussion: the SLMC was good until general activities - then: it has to become quite detailed --> not feasible in the scope of this work. but a good framework to continue the work with.
-> size of Berkad is detrimental! --> Richard water levels are for a week. Improved ones are for a couple of months --> water level is only one variable! needs to be correlated with the extraction and storage size

- BRCiS: "Cash transfers were "identified across several clusters as the preferred action" where local markets and the operational context allow" (TB) --> but only there and market prices are very high when drought sets in
--> Beledi highlights help throughout the society -> but water prices rise very high.. 

- better comparison between the comparable projects of NADFOR, MoWR and OCHA was not possible due to lack of raw data --> no interviews

% I1
-results for the implementation of AAs:
o	i) Determining the water level to trigger action
o	ii) water levels monitoring
o	iii) triggering action based on water levels

% I2
(104) MoH has good experiences with NYSS in the health sector. These experiences should be translated and widened on other topics.

% zu stage 2
The funding of a pilot study as a decisive factor could not be clarified in this framework, but does not affect the further conception. --> though there is an ongoing proposal which would build on top of this work (if granted).

% APPEDNIX
% include questions, questionnaires, transcripts and codes
The first interview came about through existing contacts of the project in which this work is embedded, and the interviewee was the project leader of the FbF approach in the \acrshort{srcs} (I1). In the further course, the CBS project manager on the Norwegian Red Cross side (I2) and the CBS manager on the Somali side (I3) were also interviewed. Between these two interviews, there was a second interview with the project manager of the \acrshort{srcs}' \acrshort{fbf} team (I1.2).


% about SLMC
-> sub-sub-activities or products could also be labelled differently.. not all levels were clear and distinct

--> Methods, techniques, tools and scripts need to be created/developed -> therefore sub-goal instead of following the SLMC --> believed to: better overview, prerequisites -> important



% while multiple other frameworks and guidelines already exist, their numrous number is also explained/reflected by the need, that the frameworks need to be focussed on a specific topic, region and environment in order to give meaningful advice and not only generic information that is too coarse to be of great use. Therefore, -> new development and adjustments --> fraisl was e.g. focused on environmental and ecological stuff, CBS was too heavy on the social side e.g. patient privacy and case handling is way less important in this case

% funding
Funding is not considered in this work, as it is generally out of scope of a Master Thesis to account for that. This is the task of the project management and should be covered by some grant.


% recommended to combine the BRCiS appraoch with I1 categorization. one more extensive long-term assessment at the end of the rain season together with the other rarely changing indicators (e.g. number of animals and people) but also combined with an triangulation with stakeholder and thorough analysis and interpretation of data --> integration into decision-making processes. --> seasonal warning and identification of vulnerability + short-term facilitation of immediate action with the short-term water level warnings

% regular reporting with event based reporting as water in Berkads can e.g. dry up quicker than predicted or turn bad or what ever else..