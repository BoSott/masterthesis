% Chapter Template

\chapter{Chapter Title Here} % Main chapter title

\label{ChapterX} % Change X to a consecutive number; for referencing this chapter elsewhere, use \ref{ChapterX}

%----------------------------------------------------------------------------------------
%	SECTION 1
%----------------------------------------------------------------------------------------

\section{Main Section 1}

% put this in the result section.
intro to Somalia EAP: https://docs.google.com/document/d/1xUEXm8RxVHTO468KqXSAoBX-cpkPwiff/edit
https://heigit.atlassian.net/wiki/spaces/FIS/pages/1704096/Indices

current EAP stage in somalia
--> Somalia so far. But: still under development.

“Hazards Exposure and Vulnerability” ([Somali Red Crescent Society, 2022, p. 13](zotero://select/groups/4773535/items/FZ6BJHJA)) ([pdf](zotero://open-pdf/groups/4773535/items/RJKNZZZ2?page=17&annotation=IRH526LN))

“Feasibility Study on Potential Use of Forecast-based Financing (FbF) for SRCS Final Report” ([Somali Red Crescent Society, 2022, pp. -3](zotero://select/groups/4773535/items/FZ6BJHJA)) ([pdf](zotero://open-pdf/groups/4773535/items/RJKNZZZ2?page=1&annotation=KHCH33GX)





% Quality criteria for Early Action Protocols
https://heigit.atlassian.net/wiki/download/attachments/1704186/FbA-EAP-criteria-May-2022.docx?version=1&modificationDate=1677660171372&cacheVersion=1&api=v2
(summery available -> confluence)

%-----------------------------------
%	SUBSECTION 1
%-----------------------------------
\subsection{Subsection 1}

%-----------------------------------
%	SUBSECTION 2
%-----------------------------------

\subsection{Subsection 2}

%----------------------------------------------------------------------------------------
%	SECTION 2
%----------------------------------------------------------------------------------------

\section{Main Section 2}

%----------------------------------------------------------------------------------------
%	SECTION 1
%----------------------------------------------------------------------------------------

\section{Main Section 1}

%----------------------------------------------------------------------------------------
%	SECTION 1
%----------------------------------------------------------------------------------------

\section{Main Section 1}

%----------------------------------------------------------------------------------------
%	SECTION 1
%----------------------------------------------------------------------------------------

\section{Main Section 1}

%----------------------------------------------------------------------------------------
%	SECTION 1
%----------------------------------------------------------------------------------------

\section{Integrated Water Management (Goal and Groundwork for mapping/monitoring)}
one of the goals of public education and policy development


Integrating communities into Water managemend practices is needed!

ground work -> public education --> inform community about water shortages (?) -> create "agreement on water priorization during times of acute drought" + "a common shared agreement to prevent water resource contamination, as well as mitigating over-abstraction." (Day, 2009: 52)

BUT:
“Communities may not always represent a homogeneous, consenting group” ([Day, 2009, p. 52](zotero://select/groups/4773535/items/YWSNQ8A2)) ([pdf](zotero://open-pdf/groups/4773535/items/ETPCI5RI?page=7&annotation=NMA3CWDS))
--> different interests, resources, knowledge, access rights, own hierarchy + status level

how about private water sources?
“Strikingly, the traditional influence of sheikhs in water management does not, however, extend to supervision of private wells or boreholes operated by farmers or individual landowners. Sh” ([Day, 2009, p. 55](zotero://select/groups/4773535/items/YWSNQ8A2)) ([pdf](zotero://open-pdf/groups/4773535/items/ETPCI5RI?page=10&annotation=FXZW8RBU))

AA -> manage water and prioritize effectively (e.g. start with it now)
“This is significant because in drought-prone environments little at tempt is made to inform communities about their available ground water resources and there is minimal emphasis or preparation for monitoring groundwater fluctuations, prioritizing water usage dur ing periods of hardship or assisting communities to develop basic contingency plans with relief agencies or local authorities acting as a back-stop to provide support during periods of acute” ([Day, 2009, p. 51](zotero://select/groups/4773535/items/YWSNQ8A2)) ([pdf](zotero://open-pdf/groups/4773535/items/ETPCI5RI?page=6&annotation=6C4BAHSE))

“hardship. When describing community water projects, 'sustainability' is often referred to. In reality the immediate challenge is to introduce sound steward ship of water resources to assist communities to resist and recover from drought or low and variable rainfall” ([Day, 2009, p. 51](zotero://select/groups/4773535/items/YWSNQ8A2)) ([pdf](zotero://open-pdf/groups/4773535/items/ETPCI5RI?page=6&annotation=DA9BJCDN))


“r of distinct advantages of engaging in commu nity-based water resource mana” ([Day, 2009, p. 51](zotero://select/groups/4773535/items/YWSNQ8A2)) ([pdf](zotero://open-pdf/groups/4773535/items/ETPCI5RI?page=6&annotation=85TSGZK9))

FIGURE of the framework: “Figure 2. Community-based water resource manageme” ([Day, 2009, p. 59](zotero://select/groups/4773535/items/YWSNQ8A2)) ([pdf](zotero://open-pdf/groups/4773535/items/ETPCI5RI?page=14&annotation=BAMSY255))
%% --> this framework as 'base'/foundation/Early Action/AA?

“Local water users often possess detailed indigenous knowledge related to water resources, water needs and historical change that has occurred related to water use. 
Water users recognize that water is a fundamental component of their subsistence-based livelihoods, which helps to weave rela tionships between water users. 
Communities are able to monitor agreed water usage on a daily basis, as part of their daily activities. 
Communities often have historical mechanisms for conflict and dispute resolution related to water resource management, which may require continued support and assistance to evolve and adapt to global challenges. 
Effective water management requires community participation; this principle is well understood in development li” ([Day, 2009, p. 52](zotero://select/groups/4773535/items/YWSNQ8A2)) ([pdf](zotero://open-pdf/groups/4773535/items/ETPCI5RI?page=7&annotation=4RVGKM5R))





“However, responsible planning for drought mitigation at community level is often omitted.” ([Day, 2009, p. 47](zotero://select/groups/4773535/items/YWSNQ8A2)) ([pdf](zotero://open-pdf/groups/4773535/items/ETPCI5RI?page=2&annotation=S4ACQRPL))

“Communities frequently remain excluded from any basic capacity building, centred on water resource management, as part of a localized Integrated Water Resources Management (IWRM) programme.” ([Day, 2009, p. 47](zotero://select/groups/4773535/items/YWSNQ8A2)) ([pdf](zotero://open-pdf/groups/4773535/items/ETPCI5RI?page=2&annotation=WBL45NKR))

“Community-based water resources management” ([Day, 2009, p. 47](zotero://select/groups/4773535/items/YWSNQ8A2)) ([pdf](zotero://open-pdf/groups/4773535/items/ETPCI5RI?page=2&annotation=RQLJMKL7))

"5.3 Types of water resources monitoring
As it has been indicated above, water resources monitoring provides information on the state and trends of quantitative and qualitative characteristics of the monitored object, level and distribution of anthropogenic loads, state of ecosystems and a degree of possibility of satisfaction of various needs in water, municipal and economic.

There are three main types of water resources monitoring, used in the water resources management system:
%% main points:
local monitoring, performed for solution of specific local problems on a limited part of the water body or the territory;
global (background) monitoring, performed at man-impact free sites, or on sites with low level of anthropogenic influence. Such monitoring is performed for acquisition of information on steady natural characteristics of environmental components. The background monitoring of water bodies is used for evaluation and/or prognostication of shifts in their state caused by economic activities;
comprehensive (regime) monitoring performed at the water body observation network for determination of the actual state of the water body, for decision making on efficient use, protection, and restoration of water resources;
critical or alarm monitoring, performed at sites of high risk for immediate warning about unfavourable situations caused primarily by human activities.
In the water management system there is also a special type of monitoring for wastewater discharges to the water body."https://echo2.epfl.ch/VICAIRE/mod_4/chapt_5/main.htm

"As a rule, the following parameters are always monitored:

Water Temperature.
Transparency or Turbidity.
pH.
Conductivity.
Dissolved oxygen (DO).
Total phosphorus.
Total nitrogen.
Nitrogen, Ammonia.
Nitrogen, Nitrate.
Soluble Reactive Phosphorus.
Faecal coliform bacteria." https://echo2.epfl.ch/VICAIRE/mod_4/chapt_5/main.htm
%% --> not gonna happen but still interesting

https://www.oxfamwash.org/water/cbwrm/Oxfam%20CBWRM%20Companion,%202009.pdf
%----------------------------------------------------------------------------------------
%	SECTION 1
%----------------------------------------------------------------------------------------

\section{Main Section 1}

%----------------------------------------------------------------------------------------
%	SECTION 1
%----------------------------------------------------------------------------------------

\section{Main Section 1}
