% Chapter Template

\chapter{Introduction} % Main chapter title

\label{Chapter1} % Change X to a consecutive number; for referencing this chapter elsewhere, use \ref{ChapterX}
%----------------------------------------------------------------------------------------

% Define some commands to keep the formatting separated from the content 
\newcommand{\keyword}[1]{\textbf{#1}}
\newcommand{\tabhead}[1]{\textbf{#1}}
\newcommand{\code}[1]{\texttt{#1}}
\newcommand{\file}[1]{\texttt{\bfseries#1}}
\newcommand{\option}[1]{\texttt{\itshape#1}}

%----------------------------------------------------------------------------------------
%	SECTION 1
%----------------------------------------------------------------------------------------

what drought is (definition)
drought monitoring but not only physical indicators but socials as well (water source accessibility) “However, assessments focused only on physical variables and processes fail to capture why drought matters, in other words, how social, economic, and ecological systems are affected (i.e., impacts) (Redmond 2002; Van Loon et al. 2016; Wilhite and Glantz 1985).” ([Lackstrom et al., 2022, p. 3](zotero://select/groups/4773535/items/YI366LQY)) ([pdf](zotero://open-pdf/groups/4773535/items/3JTQ72UN?page=3\&annotation=72WYIE7B))
current monitoring approaches e.g. (“the extent to which volunteers’ assessments of dry-to-wet conditions correspond to objective drought indicators (EDDI, SPI, SPEI) typically employed for monitoring drought” ([Lackstrom et al., 2022, p. 2](zotero://select/groups/4773535/items/YI366LQY)) ([pdf](zotero://open-pdf/groups/4773535/items/3JTQ72UN?page=2\&annotation=DQ4ZNIRS)))
“quantitative indicators (namely the SPEI, SPI, and EDDI).” ([Lackstrom et al., 2022, p. 26](zotero://select/groups/4773535/items/YI366LQY)) ([pdf](zotero://open-pdf/groups/4773535/items/3JTQ72UN?page=26\&annotation=G4KIJYUR))
which forecasts are selected by the EAP pre-study?
request of the SRCS -> practically wanted
understanding the full scope and knowing which water sources are at what level and quality can help with management decisions and trigger certain events very locally
current challenges (?) what do I want to address? (number 1,2,3)
outline of the thesis/project

what about the volunteers? how many? where are they? How to they spread over the country?
“Meadow et al. (2013) recommended using trained agency staff to report drought status on a regular basis” ([Lackstrom et al., 2022, p. 27](zotero://select/groups/4773535/items/YI366LQY)) ([pdf](zotero://open-pdf/groups/4773535/items/3JTQ72UN?page=27\&annotation=AUZV7SZN))

overcome limitations / incorporating recommendations: e.g. increased support/engagement of poeple who actually use the reports (e.g. SRCS officials) 
fill geographic and information gap ->

its about drought forecasting and early trigger but at the same time highly local and pracitcal information where and which water sources are good and functioning and which are not. -> highly practical information. Some data exist but (mostly) outdated.
about getting local knowledge from SRCS Volunteers and their community as well as returning information about the bigger picture
in order to enhance the quality of data for managing severe droughts in Somaliland. (one short paragraph -> motivation)
provide number of weather stations in the area

relevance
Somalia is komplett am krepieren because of a multi-year long drought - ...\% of damage/conflicts etc. is based on droughts. severe shit! --> first case study introduction (geographically, socially, etc.) 
scope: of the EAP/FbF project

--> FbF and EAP what is what and so on --> in context of this work -> drought indicator (what exists so far? -> pre study)
Somalia red Crescent Society scope etc. 

--> problem: conclusion of existing sources, tools and forecasts - only macro/international level? or are there meso/micro forecasts available?
better understanding the forecasting and its implications on the ground are crucial. --> local information. tons of Volunteers but even more water sources. Continue with Crowdsensing? Implications?

--> highlight the pro of this work
"The highly localized information provided by observers can fill drought monitoring gaps by ground-truthing quantitative indicators and offering information in places where other monitoring tools may not exist. Overall, the research team found that strategic investments in time and funding can help fill in geographic and temporal gaps in drought monitoring information through volunteer observations."
https://www.drought.gov/news/research-confirms-role-citizen-science-contributions-drought-detection-and-monitoring and https://doi.org/10.1175/BAMS-D-21-0157.1

key aims and objectives + research questions

incorporating local knowledge? - if yes, how?
how to account for gender inequalities? Even possible?

current challenges for utilisation of forecasting systems: scarse coverage of weather stations and poor utilisation by the farmers often due to bad dissemination channels  (too coarse, too unreliable)

provide hypothesis - 'integration' of local stakeholders/volunteers into drought and water source monitoring can help to get a better picture for early drought impact assessment and thus better and faster management and reactions
+ local people are engaged with the process and come into contact with 'scientific' knowledge/forecasts 
+ equal inclusion of local / indigenous knowledge and scientific forecasts can enhance the quality and make it more relevant on smaller scales (meso/mikro/local level)


how it was done: methods are qualitative, half-structured interviews, literature review and conceptualizing a tool based on current best practices

outline structure

knowledge co-production
\autocite{dasInteractiveInformationCrowdsourcing2016}

conceptualization: evtl. den 4er Zirkel?


%-----------------------------------
%	SUBSECTION 1
%-----------------------------------
\section{Section 1}
Definitions of terminology
drought
local knowledge
\subsection{Subsection 1}


%-----------------------------------
%	SUBSECTION 2
%-----------------------------------

\subsection{Subsection 2}

%----------------------------------------------------------------------------------------
%	SECTION 2
%----------------------------------------------------------------------------------------

\section{Main Section 2}

10\% (8 pages)

establish your research territory: general information about the importance, background details to understand studies context
justifying your niche - why the research is needed. -> how Gap
explain the significance of your study -> how the research was conducted -> value of the study

---
what is the topic of your thesis?
what are the objectives
what is the outline
methods?
thesis statement

Introduce the topic of the study.
Provide some general background info about your study.
Give a short overview of the literature review (we use the word short because the main lit review will be in Chapter Two: Literature Review.
Bring out the general idea of the study or the scope.
Provide the details of the current situation about the problem.
Describe the relevance of the research that you are going to present (note that you are introducing a study that you have already completed).
Outline the key aims and objectives of the dissertation.
Bring out the research questions or problems of the study.
Provide your hypothesis.
Outline the structure of your dissertation.
Highlight the methodology that you used to do the study.


acknowledge former studies
structure 
surprise (gender and the role of "integration")
best examples and best literature review
write a draft

Motivation of the study.
Description of the study topic.
Explanation of the relevance of the study.
Explanation of the scope of the study.
Demonstration of how the study was done.
Your dissertation outline.