% Chapter Template

\chapter{Literature Review} % Main chapter title

\label{Chapter2} % Change X to a consecutive number; for referencing this chapter elsewhere, use \ref{ChapterX}

%----------------------------------------------------------------------------------------
%	SECTION 1
%----------------------------------------------------------------------------------------

Summarize and synthesize: give an overview of the main points of each source and combine them into a coherent whole
Analyze and interpret: don’t just paraphrase other researchers—add your own interpretations where possible, discussing the significance of findings in relation to the literature as a whole
Critically evaluate: mention the strengths and weaknesses of your sources
Write in well-structured paragraphs: use transition words and topic sentences to draw connections, comparisons and contrasts

II.  Literature Review

The literature review for a case study research paper is generally structured the same as it is for any college-level research paper. The difference, however, is that the literature review is focused on providing background information and enabling historical interpretation of the subject of analysis in relation to the research problem the case is intended to address. This includes synthesizing studies that help to:

"Place relevant works in the context of their contribution to understanding the case study being investigated. This would include summarizing studies that have used a similar subject of analysis to investigate the research problem. If there is literature using the same or a very similar case to study, you need to explain why duplicating past research is important [e.g., conditions have changed; prior studies were conducted long ago, etc.].
Describe the relationship each work has to the others under consideration that informs the reader why this case is applicable. Your literature review should include a description of any works that support using the case to study the research problem and the underlying research questions.
Identify new ways to interpret prior research using the case study. If applicable, review any research that has examined the research problem using a different research design. Explain how your case study design may reveal new knowledge or a new perspective or that can redirect research in an important new direction.
Resolve conflicts amongst seemingly contradictory previous studies. This refers to synthesizing any literature that points to unresolved issues of concern about the research problem and describing how the subject of analysis that forms the case study can help resolve these existing contradictions.
Point the way in fulfilling a need for additional research. Your review should examine any literature that lays a foundation for understanding why your case study design and the subject of analysis around which you have designed your study may reveal a new way of approaching the research problem or offer a perspective that points to the need for additional research.
Expose any gaps that exist in the literature that the case study could help to fill. Summarize any literature that not only shows how your subject of analysis contributes to understanding the research problem, but how your case contributes to a new way of understanding the problem that prior research has failed to do.
Locate your own research within the context of existing literature [very important!]. Collectively, your literature review should always place your case study within the larger domain of prior research about the problem. The overarching purpose of reviewing pertinent literature in a case study paper is to demonstrate that you have thoroughly identified and synthesized prior studies in the context of explaining the relevance of the case in addressing the research problem."https://libguides.pointloma.edu/c.php?g=944338&p=6806958

1. look for relevant scholarly resources
2. evaluate the resources 
    What problem is addressed in the resource?
    How has the author defined the main concepts?
    What theories and methods are used in the resource?
    What is the conclusion of the resource?
    What is the relationship between the resource and other resources?
    How does the resource contribute to knowledge about the topic?

3. identify debates and gaps in these resources
    Patterns and trends, especially in theories, methods, and results.
    Debates, major conflicts, and contradictions.
    Themes.
    Gaps on what is missing in the literature.
    Pivotal publications.


4. develop your outline
order: chronological, thematic, methodological, theoretical
    

5. write literature review
    Introduction
        give focus of the literature review
    body
        finer details.
        summerize, analyze, interpret
        evaluate comprehensively
        write carefully in properly structured and easy to read paragraphs
        
    conclusion

%-----------------------------------
%	SUBSECTION 1
%-----------------------------------

\section{Droughts}
“Monitoring Drought with the Combined Drought Index in Kenya” ([Balint et al., 2013, p. 1](zotero://select/groups/4773535/items/C6T6K9AP)) ([pdf](zotero://open-pdf/groups/4773535/items/ZT9SQQ2M?page=1\&annotation=DMSJ4J63))
introduction is good.

drought definitions, drought types (meteorological, agricultural, hydrological and estimation)





\subsection{Drought Indicators}


in general: https://edo.jrc.ec.europa.eu/edov2/php/index.php?id=1010

explanation and overview about drought indices “Drought prediction based on SPI and SPEI with varying timescales using LSTM recurrent neural network” ([Poornima and Pushpalatha, 2019, p. 8399](zotero://select/groups/4773535/items/NJME9MIM)) ([pdf](zotero://open-pdf/groups/4773535/items/48LYF6PR?page=1&annotation=FV2JBM9I))

SPI - standardized Precipitation Index (most widely used drought indices (DI)) 
https://climatedataguide.ucar.edu/climate-data/standardized-precipitation-index-spi
http://iridl.ldeo.columbia.edu/maproom/Global/Drought/Global/index.html

"The experimental Global Gridded Standardized Precipitation Index (SPI) is derived from the NOAA CMORPH dataset and includes timescales of 1, 3, 6 and 9 months.  The NOAA CMORPH precipitation dataset is a gridded dataset derived from combining numerous microwave-based estimates from low orbiter satellites." https://www.drought.gov/data-maps-tools/global-gridded-standardized-precipitation-index-spi-cmorph-daily + related publications

EDDI (Evaporative Demand Drought Index) ("examines how anomalous the atmospheric evaporative demand (E0; also known as "the thirst of the atmosphere) is for a given location and across a time period of interest." \& "EDDI has been shown to offer early warning of drought stress relative to current operational drought indicators, such as the U.S. Drought Monitor (USDM). A particular strength of EDDI is in capturing the precursor signals of water stress at weekly to monthly timescales, which makes EDDI a potent tool for drought preparedness at those timescales. EDDI also uses the same classification scheme as the USDM to define drought conditions, so it is easy to read EDDI maps."https://www.drought.gov/data-maps-tools/evaporative-demand-drought-index-eddi-subseasonal-forecasts)



SPEI (Standardised Precipitation-Evapotranspiration Index https://spei.csic.es/)
"The SPEI is a multiscalar drought index based on climatic data. It can be used for determining the onset, duration and magnitude of drought conditions with respect to normal conditions in a variety of natural and managed systems such as crops, ecosystems, rivers, water resources, etc."https://spei.csic.es/

key strength and key limitations https://climatedataguide.ucar.edu/climate-data/standardized-precipitation-evapotranspiration-index-spei


more indices: https://www.droughtmanagement.info/indices/
%-----------------------------------
%	SUBSECTION 2
%-----------------------------------

\subsection{forecasts}
Somalia Multi-hazard Early Warning Centre under Ministry of Humanitarian Affairs & Disaster Management
based on open data sources (coverage: Somalia)

FAO Somalia Water and Land Information Management (SWALIM) coverage: all regions - source: 100 manual rainfall stations, eight synoptic weather stations and 11 automatic weather stations in Somalia

“Forecast Menu for SRCS” ([Somali Red Crescent Society, 2022, p. 43](zotero://select/groups/4773535/items/FZ6BJHJA)) ([pdf](zotero://open-pdf/groups/4773535/items/RJKNZZZ2?page=43&annotation=5A354LG7))
identified by the SRCS pre-study

there are more! Look into it.!
%----------------------------------------------------------------------------------------
%	SECTION 2
%----------------------------------------------------------------------------------------

\section{Main Section 2}
