% Chapter Template

\chapter{Literature Review} % Main chapter title

\label{Chapter2} % Change X to a consecutive number; for referencing this chapter elsewhere, use \ref{ChapterX}

Summarize and synthesize: give an overview of the main points of each source and combine them into a coherent whole
Analyze and interpret: don’t just paraphrase other researchers—add your own interpretations where possible, discussing the significance of findings in relation to the literature as a whole
Critically evaluate: mention the strengths and weaknesses of your sources
Write in well-structured paragraphs: use transition words and topic sentences to draw connections, comparisons and contrasts

II.  Literature Review

The literature review for a case study research paper is generally structured the same as it is for any college-level research paper. The difference, however, is that the literature review is focused on providing background information and enabling historical interpretation of the subject of analysis in relation to the research problem the case is intended to address. This includes synthesizing studies that help to:

"Place relevant works in the context of their contribution to understanding the case study being investigated. This would include summarizing studies that have used a similar subject of analysis to investigate the research problem. If there is literature using the same or a very similar case to study, you need to explain why duplicating past research is important [e.g., conditions have changed; prior studies were conducted long ago, etc.].
Describe the relationship each work has to the others under consideration that informs the reader why this case is applicable. Your literature review should include a description of any works that support using the case to study the research problem and the underlying research questions.
Identify new ways to interpret prior research using the case study. If applicable, review any research that has examined the research problem using a different research design. Explain how your case study design may reveal new knowledge or a new perspective or that can redirect research in an important new direction.
Resolve conflicts amongst seemingly contradictory previous studies. This refers to synthesizing any literature that points to unresolved issues of concern about the research problem and describing how the subject of analysis that forms the case study can help resolve these existing contradictions.
Point the way in fulfilling a need for additional research. Your review should examine any literature that lays a foundation for understanding why your case study design and the subject of analysis around which you have designed your study may reveal a new way of approaching the research problem or offer a perspective that points to the need for additional research.
Expose any gaps that exist in the literature that the case study could help to fill. Summarize any literature that not only shows how your subject of analysis contributes to understanding the research problem, but how your case contributes to a new way of understanding the problem that prior research has failed to do.
Locate your own research within the context of existing literature [very important!]. Collectively, your literature review should always place your case study within the larger domain of prior research about the problem. The overarching purpose of reviewing pertinent literature in a case study paper is to demonstrate that you have thoroughly identified and synthesized prior studies in the context of explaining the relevance of the case in addressing the research problem."https://libguides.pointloma.edu/c.php?g=944338&p=6806958

1. look for relevant scholarly resources
2. evaluate the resources 
    What problem is addressed in the resource?
    How has the author defined the main concepts?
    What theories and methods are used in the resource?
    What is the conclusion of the resource?
    What is the relationship between the resource and other resources?
    How does the resource contribute to knowledge about the topic?

3. identify debates and gaps in these resources
    Patterns and trends, especially in theories, methods, and results.
    Debates, major conflicts, and contradictions.
    Themes.
    Gaps on what is missing in the literature.
    Pivotal publications.


4. develop your outline
order: chronological, thematic, methodological, theoretical
    

5. write literature review
    Introduction
        give focus of the literature review
    body
        finer details.
        summerize, analyze, interpret
        evaluate comprehensively
        write carefully in properly structured and easy to read paragraphs
        
    conclusion








use this as beginning?
“Chapter 13 Synthesis – product” ([Minkman, 2015, p. 173](zotero://select/groups/4773535/items/ZKLE6CPT)) ([pdf](zotero://open-pdf/groups/4773535/items/QMAPCSZG?page=173&annotation=5FECLF3E))

Guidelines:
- Consider the design elements to make informed decisions
    - Design tips:
        - Comprehend what citizen science is and decide whether it is suitable as a means
            -> formulate clear goals of the project ([Minkman, 2015, p. 177](zotero://select/groups/4773535/items/ZKLE6CPT)) ([pdf](zotero://open-pdf/groups/4773535/items/QMAPCSZG?page=177\&annotation=DERHMEYN))
            -> formulate sub-goals and related products (e.g. list of things that should be known, ...)
            -> determine which activities are most suitable to deliver the products
        - Collaborate with partners
            -> Identify partners (research institutes, official agencies, organizations, interest groups)
            -> be aware of their interests and goals + who has an interest in a non-functioning system?
        - Beware of financial motivations
            -> infrastructure, development, coordination, training, equipment, etc. are non-negligible costs
        - Key success-factor: Match patterns of collaboration with organizational capacity (regarding coordination and support)
            -> determine patterns of collaboration: governance and level of involvement (-> contributory, collaborative, or co-created)
        - Match techniques with organizational capacity (regarding data processing)
            -> MCS (Mobile Crowd Sensing)
                - Pros:
                    * highly mobile and scalable
                    * low-cost
                    * automatic time stamp and GPS possible
                    * citizens could interfere when necessary
                - Cons:
                    * devices are not specifically developed for the sensing task
                    * the target audience may not be familiar with smartphones
                    * difficult to ensure data trustworthiness
                    * high risk of privacy invasion
            -> An MCS application should be useful rather than easy-to-use
                	-> TAM (Technology Acceptance Model): ease of use, usefulness, behavioural intention to use a product
        - Organize a pilot or have a trial period
        - align organization interest/motivation with citizen interest/motivation
            -> motivation overview p.167 + create ownership
        - make sure data us used and provide feedback
        - constantly recruit new participants
 

- is citizen science suitable?
- collaborate with strategic partners
- care about the associated costs (development, infrastructure, coordination, training, etc.)
- techniques and citizen science method should match organizational capacities
- take (social) scientific knowledge into account
- start with a pilot project
- keep the capacity of the target audience, users and authorities, in mind. balance data trustworthiness and privacy is the main challenge
- 

4




%----------------------------------------------------------------------------------------
%	SECTION 1 FbF, EAP & SRCS (+internal organization)
%----------------------------------------------------------------------------------------
%-----------------------------------
%	SUBSECTION 1 what (kind of) information is needed?

“Towards drought impact-based forecasting in a multi-hazard context” ([Boult et al., 2022, p. 1](zotero://select/groups/4773535/items/B2AQSTYL)) ([pdf](zotero://open-pdf/groups/4773535/items/W9TFLH43?page=1&annotation=GL47JLV7))
The framework for drought impact-based forecasting proposed in the paper "Towards drought impact-based forecasting in a multi-hazard context" aims to provide more comprehensive and actionable information for decision-making in the context of multi-hazard risk management. The framework considers both biophysical and socioeconomic factors, including:

Exposure: The extent to which a population or system is exposed to drought-related hazards, such as the availability of water resources, the dependence on rain-fed agriculture, and the presence of infrastructure that may be vulnerable to drought-related impacts.

Vulnerability: The susceptibility of a population or system to drought-related impacts, considering factors such as the social, economic, and environmental characteristics of the affected area, as well as the capacity of the affected community or system to cope with and recover from drought-related impacts.

Coping capacities: The ability of a population or system to mitigate or adapt to drought-related impacts, including the availability of resources, the presence of social networks and support systems, and the ability to access and utilize information and technology.

The authors argue that considering these factors in the context of drought forecasting can provide more useful and actionable information for decision-making, and recommend further research on the development and implementation of drought impact-based forecasting systems, as well as their integration into existing decision-making frameworks.


“∣Potential impact (x, t)∣≡∣hazard (x, t)∣∪∣vulnerability (x, t) ∣∪∣exposure (x, t)∣” ([Boult et al., 2022, p. 2](zotero://select/groups/4773535/items/B2AQSTYL)) ([pdf](zotero://open-pdf/groups/4773535/items/W9TFLH43?page=2&annotation=6M48XVJI))

“In theory, focusing on what the weather will do, rather than what the weather will be, enables decision makers to plan and implement targeted preparatory actions to better reduce hazard impacts (Harrowsmith et al., 2020).” ([Boult et al., 2022, p. 2](zotero://select/groups/4773535/items/B2AQSTYL)) ([pdf](zotero://open-pdf/groups/4773535/items/W9TFLH43?page=2&annotation=NSLE7NL6))

“However, establishing a functional relationship can be difficult for a number of reasons” ([Boult et al., 2022, p. 3](zotero://select/groups/4773535/items/B2AQSTYL)) ([pdf](zotero://open-pdf/groups/4773535/items/W9TFLH43?page=3&annotation=ANJJD878))

--> direct information would be better
--> in case of multi-hazard EWS
“Drought presents a clear case for the multi-hazardapproach. Because it is a slow-onsetevent, vulnerability to drought is susceptible to the influenceof concurrent hazards and non-biophysical events, meaning multi-hazards must be considered if drought interventions are to be effectivelytargeted. Moreover, the intrinsic predictability and slow onset provides a timeframe in which early actions can be adapted in response to multi-hazard influences (Boult et al., 2020).” ([Boult et al., 2022, p. 2](zotero://select/groups/4773535/items/B2AQSTYL)) ([pdf](zotero://open-pdf/groups/4773535/items/W9TFLH43?page=2&annotation=QZ7UC62U))

“2.Challenges for drought IbF” ([Boult et al., 2022, p. 3](zotero://select/groups/4773535/items/B2AQSTYL)) ([pdf](zotero://open-pdf/groups/4773535/items/W9TFLH43?page=3&annotation=7B52FIS8)) (Impact based Forecast):
The main challenges for drought impact-based forecasting (IbF) as described in this text include:

Difficulty in directly forecasting drought impacts: It can be challenging to establish a functional relationship between drought hazards and impacts, as past humanitarian aid and development may have weakened this link, and sufficient impact data may be unavailable. In addition, the relationship between hazard severity and impact is mediated by vulnerability, which is not always fully understood.

Issues with predefined systems: Predefined IbF systems that rely on predefined trigger thresholds and pre-agreed actions may not be able to adapt to changing vulnerabilities over time. These systems also rely on expert judgement during system development, which may be subject to bias and may not accurately reflect the complexity of the system.

“Without scope to accommodate dynamic vulnerabilities, actions cannot be effectively targeted or may prove ineffective.” ([Boult et al., 2022, p. 4](zotero://select/groups/4773535/items/B2AQSTYL)) ([pdf](zotero://open-pdf/groups/4773535/items/W9TFLH43?page=4&annotation=YYURM2E3))

Lack of coordination between sectors: Drought impacts often involve multiple sectors (e.g. agriculture, water, health), but there may be a lack of coordination and integration between these sectors in the design and implementation of IbF systems. This can lead to inefficiencies and gaps in the response to drought impacts.

Limited capacity for impact-based decision making: There may be a lack of capacity or resources for impact-based decision making at the local level, particularly in low-income countries where data availability and quality may be limited.

Inability to fully capture the complexity of drought impacts: Drought impacts often involve complex and dynamic interactions between hazards, vulnerabilities, and exposures, and may be exacerbated by non-climate factors such as conflict or economic conditions. It can be difficult to fully capture and understand these interactions in IbF systems.

“Moreover, if the complex relationships linking multihazards to multi-impacts obscure attribution of particular impacts to particular hazards, hazard-focused organisations may be limited by their institutional mandate and thus unable to act.” ([Boult et al., 2022, p. 4](zotero://select/groups/4773535/items/B2AQSTYL)) ([pdf](zotero://open-pdf/groups/4773535/items/W9TFLH43?page=4&annotation=2XG6YP8D))

interesting!!! Fig. 1. A hybrid framework for multi-hazard IbF. Refer to the main text for a definition of numbers. Black arrows and numbers: components common across predefined humanitarian IbF systems. Blue arrows and numbers: real-time components. Grids represent spatially varying values. Darker reds indicate higher values of risk, vulnerability, and thresholds. In this example, despite only low to moderate risk in the southwest square, increased dynamic vulnerability lowers the threshold for action, resulting in triggering. Meanwhile, reduced vulnerability in the northern squares elevates trigger thresholds, so the northeast square no longer triggers. (For interpretation of the references to colour in this figure legend, the reader is referred to the web version of this article.)

A hydrometeorological forecast indicating the likelihood of drought occurring is combined with a predefined assessment of static vulnerability to determine risk. Where static vulnerability is higher (Fig. 1: northern squares), trigger thresholds are lower. 2) Risk is compared to agreed-upon thresholds for action. 3) If risk is greater than or equal to the threshold, early action is triggered to mitigate the worst impacts of drought.

Proposed Framework:
--> 1)“A hydrometeorological forecast indicating the likelihood of drought occurring is combined with a predefined assessment of static vulnerability to determine risk. Where static vulnerability is higher (Fig. 1: northern squares), trigger thresholds are lower.
2) Risk is compared to agreed-upon thresholds for action.
3) If risk is greater than or equal to the threshold, early action is triggered to mitigate the worst impacts of drought.” ([Boult et al., 2022, p. 5](zotero://select/groups/4773535/items/B2AQSTYL)) ([pdf](zotero://open-pdf/groups/4773535/items/W9TFLH43?page=5&annotation=KDCBHGC6))

“We then propose a number of components to account for dynamic vulnerabilities caused by concurrent hazards:
4)Expert judgement is utilised to determine dynamic vulnerabilities. For instance, conflict, pest outbreaks, or recent hydrometeorological events, may act to increase vulnerability to drought in the affected location.
5) In locations where vulnerability is elevated (Fig. 1: southwest square), the predefined forecast threshold (“danger level”) is relaxed in order to trigger for less severe droughts. This acknowledges that those with elevated vulnerabilities require support even if drought is only slight. In regions where dynamic vulnerability is lower (Fig. 1: northern squares), the predefined forecast threshold may be raised, to avoid the perception of false alarms if a less severedrought does not have significant impact on food security (the trigger threshold for northern squares is elevated to reflect reduced vulnerability). Balancing of lower thresholds for vulnerable regions against higher thresholds for less vulnerable regions reduces the need for ‘safety nets’, enabling more accurate anticipation of donor costs.
6) If risk exceeds the adjusted thresholds, early actions are triggered. Early actions may need to be adapted to account for multihazards.” ([Boult et al., 2022, p. 5](zotero://select/groups/4773535/items/B2AQSTYL)) ([pdf](zotero://open-pdf/groups/4773535/items/W9TFLH43?page=5&annotation=XV9GPPQF))

%-----------------------------------
%   SUBSECTION 3.0 Water Access
%-----------------------------------
“There are three main facets to urban water access: availability, quality, and affordability.” ([Mitlin et al., p. 11](zotero://select/groups/4773535/items/KAM9REZR)) ([pdf](zotero://open-pdf/groups/4773535/items/BM6BU5UR?page=11&annotation=44IJC8RC))


%-----------------------------------
%	SUBSECTION 3 Water Scarcity
%-----------------------------------
take this definition of water scarcity
“2. Defining water scarcity” ([“Coping with water scarcity: an action framework for agriculture and food security”, 2012, p. v](zotero://select/groups/4773535/items/R7VEIVF3)) ([pdf](zotero://open-pdf/groups/4773535/items/M6HRGVGP?page=23&annotation=JFSBH7ME))


Distinctions between water scarcity and drought:

“Table 1. Characteristics and impacts of water scarcity and drought Water scarcity Drought Length Long-term to permanent Temporary (weeks to multiyear) Driving forces Demand–supply imbalance, human-driven, and/or natural (overexploitation, pollution). Climate change can impact both supply and demand Natural climate variability which can be modified/amplified by climate change Potential impacts Restricted water availability, environmental degradation, desertification, exacerbated inequalities in access to water resources, potential competition Water shortages, competition, environmental degradation Measures Long-term IWRM to bring supply and demand back into sustainable balance Integrated drought management, including: (1) monitoring and early warning; (2) vulnerability and impact assessment; and (3) risk mitigation, preparedness and response Source: adapted from Hohenwallner et al. (2011) DROUGHT AND WATER SCARCITY – DEFINITIONS AND CHARACTERISTICS” ([pdf](zotero://open-pdf/groups/4773535/items/JM82W3ZF?page=9&annotation=E3EQRILA))

“Strategies for coping with droughts and water scarcity involve proactive approaches to minimize adverse effects. These responses and coping strategies should be part of overall integrated drought and water resource management strategies, including sustainable water resource development. Opportunities for proactive approaches include strengthening early alert systems, risk mitigation measures, and long-term adaptation strategies to build climate, economic, and societal resilience for the well-being of future generations. In countries and regions prone to drought and water scarcity, risk management and resilience are important for sustaining and enhancing future quality of life.” ([pdf](zotero://open-pdf/groups/4773535/items/JM82W3ZF?page=7&annotation=H4F73GR2))


“Understanding responses to climate-related water scarcity in Africa” ([Leal Filho et al., 2022, p. 1](zotero://select/groups/4773535/items/6TPZLH52)) ([pdf](zotero://open-pdf/groups/4773535/items/6LRJKBTR?page=1&annotation=4C7X6MGY))
--> impacts of water scarcity on a variety of other factors
“These findings underscore the dangers of separating (or ringfencing) responses to water scarcity from competing challenges to food security, urbanization, desertification, and human or state security.” ([Leal Filho et al., 2022, p. 11](zotero://select/groups/4773535/items/6TPZLH52)) ([pdf](zotero://open-pdf/groups/4773535/items/6LRJKBTR?page=11&annotation=DG37MVHB))
“The terms ‘water scarcity’ and ‘drought’ are often used interchangeably, despite their subtle but important differences with regards to water management.” ([pdf](zotero://open-pdf/groups/4773535/items/JM82W3ZF?page=8&annotation=8SQMLMKD))


“This means that, even if drought is a driver of water scarcity (e.g. reduction in rainfall), there is always a human dimension to the reduction in the natural water supply (European Commission, 2019).” ([pdf](zotero://open-pdf/groups/4773535/items/JM82W3ZF?page=10&annotation=UFNRBT3J))

“Water scarcity, as a supply/demand-driven and natural and/or human-made phenomenon, is one of the greatest challenges of the twenty-first century” ([pdf](zotero://open-pdf/groups/4773535/items/JM82W3ZF?page=13&annotation=WZB8I8FY))

“Likewise, water management approaches that focus only on drought when it occurs will have missed significant opportunities to reduce drought risk.” ([pdf](zotero://open-pdf/groups/4773535/items/JM82W3ZF?page=16&annotation=GVHFHKTI))


!!!!!
“Adequate and reliable weather, water, and climate data and applications are needed to monitor available water resources and provide actionable early warning for water scarcity and drought conditions.” ([pdf](zotero://open-pdf/groups/4773535/items/JM82W3ZF?page=17&annotation=67GM7DYS))
“as well as improved capacities in collecting hydrological data. Improved interaction with stakeholders is crucial to promote better tailored information products.” ([pdf](zotero://open-pdf/groups/4773535/items/JM82W3ZF?page=18&annotation=CBKCBSZV))


IWRM??


%-----------------------------------
%	SUBSECTION 3 Water Source Mapping / Water point mapping
%-----------------------------------
"In all, mapping presents many benefits, such as:

It makes easier to integrate data from different sources (surveys, censuses, satellites, etc.) and from different disciplines (social, economic, and environmental data). It also allows the switch to new units of analysis from, for example, administrative boundaries (e.g. state) to ecological boundaries (e.g. basin).
Maps are a powerful visual tool and are more easily understood by stakeholders, particularly in developing countries.
The spatial nature of water poverty, such as the distance to the nearest water source or the water supply infrastructure, can also be incorporated easily in a GIS database.
The allocation of resources can be improved, since geographic targeting is more efficient and cost-effective than to launch an equally expensive universal distribution programme.
Geo-referenced databases can be enriched by additional data as they become available; and new attributes, such as better details on water quality, can be incorporated into the data structure, ensuring that the relevance of the data is sustained over time.
Maps can be produced at a number of different resolutions depending on their purpose and the cost of data collection. A coarse resolution or a scale too small neglects the heterogeneity within each unit and provides insufficient detail for decision making, while a fine resolution or a scale too large increases the cost of compiling, managing, and analyzing the data."https://en.wikipedia.org/wiki/Water_point_mapping


%----------------------------------------------------------------------------------------
%	SECTION 2 Case Study
%----------------------------------------------------------------------------------------
%-----------------------------------
%	SUBSECTION 1 Water Points / Wells / Berkads
Berkads:
“Changing Pastoralism in the Ethiopian Somali National Regional State (Region 5)” ([“Changing Pastoralism in Region 5”, p. 1](zotero://select/groups/4773535/items/FXJGUTLD)) ([pdf](zotero://open-pdf/groups/4773535/items/BIAA5M57?page=1&annotation=5F9EZJYZ))

"Although birkeds cannot be considered permanent water points in the sense of permanent wells which do not rely on harvesting rainwater, clusters of birkeds represent dry season water points that they provide water throughout the dry season in most years. Today, then, distribution of water points is vastly different from a few decades ago. Map 5 shows the water points that exist today in the five districts under study. The map attempts to show the wells and boreholes as well as the main clusters of birkeds. The latter are difficult to map as there is no existing record of all locations. The map is based on sketch maps drawn by communities during the fieldwork. It is thus not meant to be accurate but to give an indication of the nature of change. It should be noted that these water points shown on the map are also the site of permanent settlements, as the tendency has been for settlements to grow up at the site of new water points."



"Criteria used included among others coverage, need for rehabilitation, seasonality of services, quality of water delivered, and poor management." https://en.wikipedia.org/wiki/Water_point_mapping
https://onlinelibrary.wiley.com/doi/abs/10.1111/j.1477-8947.2010.01296.x?casa_token=TPvw51virRQAAAAA:9K-6fcNcYFw9-Mny-EOIMdS6OmmSUkajSfo9qggsMNnAirGYSslUHckFyqWNP68XarnXYEgIh9eL0uGltw
%-----------------------------------
%-----------------------------------
%	SUBSECTION 2 Is monitoring the right way? Benefits and Costs
%-----------------------------------
“How to decide whether citizen science is an appropriate means?” ([Minkman, 2015, p. 174](zotero://select/groups/4773535/items/ZKLE6CPT)) ([pdf](zotero://open-pdf/groups/4773535/items/QMAPCSZG?page=174&annotation=ZUAXHTI9))

“4 | IS REMOTE MONITORING A PATHWAY TO SUSTAINABILITY?” ([Thomson, 2021, p. 9](zotero://select/groups/4773535/items/UQLXVVYI)) ([pdf](zotero://open-pdf/groups/4773535/items/K9XBXPQD?page=9&annotation=833Q66UP))

rather than thinking water sources individually -> they can be thought of as a system when monitored more broadly

-> can change the way the water sector is funded: “With these data readily available, performance-related contracts that incentivize sustainable service delivery over short-term infrastructure investment can become the norm.” ([Thomson, 2021, p. 11](zotero://select/groups/4773535/items/UQLXVVYI)) ([pdf](zotero://open-pdf/groups/4773535/items/K9XBXPQD?page=11&annotation=JP9R4Y89))

“Our results indicate that using the phones to transmit more than just water quality data will likely improve the effectiveness and sustainability of this type of intervention.” ([Kumpel et al., 2015, p. 10846](zotero://select/groups/4773535/items/GPM4C7RJ)) ([pdf](zotero://open-pdf/groups/4773535/items/7VXVKEXK?page=1&annotation=4DJIADX2))

“USING MOBILE PHONES TO MONITOR AND MANAGE WATER SUPPLY QUALITY IN RURAL ENVIRONMNETS” ([Wilson-Jones and Rivett, 2012, p. 1](zotero://select/groups/4773535/items/ZTCP6ZDX)) ([pdf](zotero://open-pdf/groups/4773535/items/KQN2HBTG?page=1&annotation=LWS8S33I))

%-----------------------------------
%	SUBSECTION 3
%-----------------------------------


%----------------------------------------------------------------------------------------
%	SECTION 3 Key Term definitions (could also be integrated with section 4 & 6)
%----------------------------------------------------------------------------------------
“2. Key concepts of participatory early warning and monitoring systems (pEWMS)” ([“Participatory early warning and monitoring systems_ A Nordic framework for web-based flood risk management | Elsevier Enhanced Reader”, p. 1296](zotero://select/groups/4773535/items/JYH8N2BV)) ([pdf](zotero://open-pdf/groups/4773535/items/JITPV84L?page=2&annotation=B2ELHBPJ))

“Adding “bottom-up” approaches [36] to classical EWMS allows stakeholders with access to local knowledge of environments and local networks to play a stronger role in decisionmaking and risk management” ([“Participatory early warning and monitoring systems_ A Nordic framework for web-based flood risk management | Elsevier Enhanced Reader”, p. 1296](zotero://select/groups/4773535/items/JYH8N2BV)) ([pdf](zotero://open-pdf/groups/4773535/items/JITPV84L?page=2&annotation=AXZXAK9D))

“With the increased pressure on water resources and the challenges faced with the implementation of the existing regulatory framework, a growing lack of mutual trust between water stakeholders has been observed in recent years [38].” ([“Participatory early warning and monitoring systems_ A Nordic framework for web-based flood risk management | Elsevier Enhanced Reader”, p. 1296](zotero://select/groups/4773535/items/JYH8N2BV)) ([pdf](zotero://open-pdf/groups/4773535/items/JITPV84L?page=2&annotation=HNTIH8JB))

citizen science or rather participatory monitoring?

%-----------------------------------
%	SUBSECTION 1 Drought, Participatory Early Warning, Anticipatory Actions, Monitoring Systems, Water Sources (evtl. VGI, Crowdsourcing/-sensing / Volunteersensing)
%-----------------------------------

what does 'monitoring' means? How is it defined?

"Water source:
o Out of 77 communities assessed, 49 have berkets, 26 have boreholes, 21 have shallow
wells, 28 communities reported that they receive water trucking.
o Berkads: 76\% of communities having berkets reported that all berkets are reported to be
depleted, while 22\% reported they were less half than full.
o Water trucking: Out of the 28 communities that receive water trucking, 43\% receive water
on a daily basis. 19 out of 28 communities (68\%) receiving water trucking responded that
they receive water from private suppliers. No communities, covered by this assessment,
mentioned water trucking from humanitarian partners.
o Reduction of water consumption: 54 out of 77 assessed communities (69\%) responded
that the majority of community members reduced water consumption in the past 4 weeks,
including all assessed communities in Sool." https://drive.google.com/file/d/1KWUZW0jEMV1Ijc4zeET_Yes93VJ_3qn2/view

%-----------------------------------
%	SUBSECTION 3
%-----------------------------------

\section{Droughts}
“Monitoring Drought with the Combined Drought Index in Kenya” ([Balint et al., 2013, p. 1](zotero://select/groups/4773535/items/C6T6K9AP)) ([pdf](zotero://open-pdf/groups/4773535/items/ZT9SQQ2M?page=1\&annotation=DMSJ4J63))
introduction is good.

drought definitions, drought types (meteorological, agricultural, hydrological and estimation)

%----------------------------------------------------------------------------------------
%	SECTION 4 Drought Index & Forecasts & Triggers
%----------------------------------------------------------------------------------------
good, precise definitions of drought in an easy way
“Droughts vary by intensity, duration, timing, and geographical coverage, creating conditions of limited moisture availability to a potentially damaging extent.” ([pdf](zotero://open-pdf/groups/4773535/items/JM82W3ZF?page=10&annotation=ZR4SQ4LB))
%-----------------------------------
%	SUBSECTION 1
%-----------------------------------

%-----------------------------------
%	SUBSECTION 2 Local Knowledge
%-----------------------------------

“2 Local knowledge in drought monitoring: an introduction to the literature review” ([Giordano et al., 2013, p. 526](zotero://select/groups/4773535/items/B7LM5ZR4)) ([pdf](zotero://open-pdf/groups/4773535/items/7I66DBIK?page=4&annotation=Z33M5FLQ))

“2.3. Indigenous Knowledge on Droughts” ([Masinde et al., 2013, p. 2](zotero://select/groups/4773535/items/M45MLGWC)) ([pdf](zotero://open-pdf/groups/4773535/items/LG6E76P4?page=2&annotation=XHU34W23))

%-----------------------------------
%	SUBSECTION 2
%-----------------------------------

\subsection{Drought Indicators}

in general: https://edo.jrc.ec.europa.eu/edov2/php/index.php?id=1010

explanation and overview about drought indices “Drought prediction based on SPI and SPEI with varying timescales using LSTM recurrent neural network” ([Poornima and Pushpalatha, 2019, p. 8399](zotero://select/groups/4773535/items/NJME9MIM)) ([pdf](zotero://open-pdf/groups/4773535/items/48LYF6PR?page=1&annotation=FV2JBM9I))

SPI - standardized Precipitation Index (most widely used drought indices (DI)) 
https://climatedataguide.ucar.edu/climate-data/standardized-precipitation-index-spi
http://iridl.ldeo.columbia.edu/maproom/Global/Drought/Global/index.html

"The experimental Global Gridded Standardized Precipitation Index (SPI) is derived from the NOAA CMORPH dataset and includes timescales of 1, 3, 6 and 9 months.  The NOAA CMORPH precipitation dataset is a gridded dataset derived from combining numerous microwave-based estimates from low orbiter satellites." https://www.drought.gov/data-maps-tools/global-gridded-standardized-precipitation-index-spi-cmorph-daily + related publications

EDDI (Evaporative Demand Drought Index) ("examines how anomalous the atmospheric evaporative demand (E0; also known as "the thirst of the atmosphere) is for a given location and across a time period of interest." \& "EDDI has been shown to offer early warning of drought stress relative to current operational drought indicators, such as the U.S. Drought Monitor (USDM). A particular strength of EDDI is in capturing the precursor signals of water stress at weekly to monthly timescales, which makes EDDI a potent tool for drought preparedness at those timescales. EDDI also uses the same classification scheme as the USDM to define drought conditions, so it is easy to read EDDI maps."https://www.drought.gov/data-maps-tools/evaporative-demand-drought-index-eddi-subseasonal-forecasts)



SPEI (Standardised Precipitation-Evapotranspiration Index https://spei.csic.es/)
"The SPEI is a multiscalar drought index based on climatic data. It can be used for determining the onset, duration and magnitude of drought conditions with respect to normal conditions in a variety of natural and managed systems such as crops, ecosystems, rivers, water resources, etc."https://spei.csic.es/

key strength and key limitations https://climatedataguide.ucar.edu/climate-data/standardized-precipitation-evapotranspiration-index-spei


more indices: https://www.droughtmanagement.info/indices/

https://droughtmonitor.unl.edu/

%-----------------------------------
%	SUBSECTION Forecasts
%-----------------------------------

\subsection{forecasts}
Somalia Multi-hazard Early Warning Centre under Ministry of Humanitarian Affairs & Disaster Management
based on open data sources (coverage: Somalia)

FAO Somalia Water and Land Information Management (SWALIM) coverage: all regions - source: 100 manual rainfall stations, eight synoptic weather stations and 11 automatic weather stations in Somalia

“Forecast Menu for SRCS” ([Somali Red Crescent Society, 2022, p. 43](zotero://select/groups/4773535/items/FZ6BJHJA)) ([pdf](zotero://open-pdf/groups/4773535/items/RJKNZZZ2?page=43&annotation=5A354LG7))
identified by the SRCS pre-study

“Table 1. Comparisons between indigenous knowledge-based seasonal forecasts and seasonal climate forecasts (adopted from Ziervogel and Opere 2010). Indigenous knowledge-based seasonal forecasts Seasonal climate forecasts Use biophysical indicators of the environment as well as spiritual methods Use of weather and climate models of measurable meteorological data Forecast methods are seldom documented Forecast methods are more developed and documented Up-scaling and down-scaling are usually complex Up-scaling and down-scaling are relatively simple Application of forecast output is less developed Application of forecast output is more developed Communication is usually oral Communication is usually written Explanation is based on spiritual and social values Explanation is theoretical Taught by observation and experience Taught through lectures and readings” ([Masinde and Bagula, 2012, p. 280](zotero://select/groups/4773535/items/EW9XSSZP)) ([pdf](zotero://open-pdf/groups/4773535/items/3WQ4S9PE?page=7&annotation=6XCISBM2))

“Indigenous knowledge within an early warning system for droughts” ([Masinde and Bagula, 2012, p. 282](zotero://select/groups/4773535/items/EW9XSSZP)) ([pdf](zotero://open-pdf/groups/4773535/items/3WQ4S9PE?page=9&annotation=8Z9A9AW8))

“B. Drought Forecasting in Sub-Saharan Africa” ([Masinde and Thothela, 2019, p. 304](zotero://select/groups/4773535/items/6D52T883)) ([pdf](zotero://open-pdf/groups/4773535/items/KLLQKDG2?page=2&annotation=ZQSDUEMX))

there are more! Look into it.!



%----------------------------------------------------------------------------------------
%	SECTION 5 Watersource Monitoring 
%----------------------------------------------------------------------------------------
“Success factors for citizen science projects in water quality monitoring” ([San Llorente Capdevila et al., 2020, p. 1](zotero://select/groups/4773535/items/26SVUJIY)) ([pdf](zotero://open-pdf/groups/4773535/items/FHIUZZ6Y?page=1&annotation=3QTPUG53))
%-----------------------------------
%	SUBSECTION 1
%-----------------------------------
acceptance of the "crowd" is difficult to measure/identify/survey as no direct communication is possible. --> Good literature work e.g. “The basis of TAM consists of four main elements: the perceived usefulness of the application, the perceived ease of use when using the application, behavioural intention to use the application and the actual use.” ([Minkman, 2015, p. 12](zotero://select/groups/4773535/items/ZKLE6CPT)) ([pdf](zotero://open-pdf/groups/4773535/items/QMAPCSZG?page=12&annotation=IM5DGRJP))

“It was found that usefulness is the most important for behavioural intention.” ([Minkman, 2015, p. 12](zotero://select/groups/4773535/items/ZKLE6CPT)) ([pdf](zotero://open-pdf/groups/4773535/items/QMAPCSZG?page=12&annotation=75GFCEP5))

and how to best apply it -> relevant for the conceptionalization phase but generally well researched topic (in other, but possibly transferable contexts)


%-----------------------------------
%	SUBSECTION 2
%-----------------------------------
%-----------------------------------
%	SUBSECTION 3
%-----------------------------------

%----------------------------------------------------------------------------------------
%	SECTION 6 Crowdsourcing and Mobile Crowdsensing Systems 
%----------------------------------------------------------------------------------------

participatory monitoring instead? or: citizen observatories, community based monitoring and participatory monitoring
“Participatory monitoring, as I will call it from now, can be useful to increase the density of a monitoring network.” ([Minkman, 2015, p. 199](zotero://select/groups/4773535/items/ZKLE6CPT)) ([pdf](zotero://open-pdf/groups/4773535/items/QMAPCSZG?page=199&annotation=XP2RRKYQ))
“Furthermore it is an interesting communication tool in the light of science communication. Correspondingly water managers should be interested in participatory monitoring in the light of integrated water management.” ([Minkman, 2015, p. 199](zotero://select/groups/4773535/items/ZKLE6CPT)) ([pdf](zotero://open-pdf/groups/4773535/items/QMAPCSZG?page=199&annotation=GHI9KSDA))


“Citizen science in environmental and ecological sciences” ([Fraisl et al., 2022, p. 1](zotero://select/groups/4773535/items/FBJD7SWS)) ([pdf](zotero://open-pdf/groups/4773535/items/7WBDKYDY?page=1&annotation=5HMYC85E))

“MCS is a special application of citizen science, where a mobile device, often smartphones, supports data collection and transmission.” ([Minkman, 2015, p. 11](zotero://select/groups/4773535/items/ZKLE6CPT)) ([pdf](zotero://open-pdf/groups/4773535/items/QMAPCSZG?page=11&annotation=VL9SHGYM))

“It is suggested, based on literature, that water authorities might prefer MCS to wireless sensor networks, for its mobility, lower costs and scalability.” ([Minkman, 2015, p. 11](zotero://select/groups/4773535/items/ZKLE6CPT)) ([pdf](zotero://open-pdf/groups/4773535/items/QMAPCSZG?page=11&annotation=RM5PX55M))

“The most prominent challenge is to balance privacy issues with data trustworthiness.” ([Minkman, 2015, p. 11](zotero://select/groups/4773535/items/ZKLE6CPT)) ([pdf](zotero://open-pdf/groups/4773535/items/QMAPCSZG?page=11&annotation=CCDRFEEH))


%-----------------------------------
%	SUBSECTION Crowdsourcing
%-----------------------------------

citizen science the right tool?
--> if help to achieve desired results AND benefit participants --> YES

+ check research question, spatial and temporal scale, type and amount of data, level of expertise required to collect the data, training and coodination efforts needed, target groups, 

funding: review resources available --> human resources (skills, equipment, travel, necessary training)



“The goal is to understand whether involving citizen science participants will help to achieve the desired results, while at the same time benefiting participants by addressing their needs” ([Fraisl et al., 2022, p. 2](zotero://select/groups/4773535/items/FBJD7SWS)) ([pdf](zotero://open-pdf/groups/4773535/items/7WBDKYDY?page=2&annotation=ISWIIMXQ))
“Examples of projects that are suitable for citizen science approaches are [...] monitoring water or air quality” ([Fraisl et al., 2022, p. 3](zotero://select/groups/4773535/items/FBJD7SWS)) ([pdf](zotero://open-pdf/groups/4773535/items/7WBDKYDY?page=3&annotation=ZHSWV5QJ))

origin: https://www.wired.com/2006/06/crowds/

“Towards an integrated crowdsourcing definition” ([Estellés-Arolas and González-Ladrón-de-Guevara, 2012, p. 189](zotero://select/groups/4773535/items/XYSTHUYY)) ([pdf](zotero://open-pdf/groups/4773535/items/AHADBEP2?page=1&annotation=EUKRSKE7))

“Crowdsourcing as a Model for Problem Solving” ([“Crowdsourcing as a Model for Problem Solving”, p. 75](zotero://select/groups/4773535/items/NT56MWYM)) ([pdf](zotero://open-pdf/groups/4773535/items/TZI4NDBJ?page=1&annotation=FJTS2A7L))


“Numerous comparison studies of volunteer vs. professional water resources data suggest that volunteer data are generally comparable to professional data for chemical (Obrecht et al., 1998; Loperfido et al., 2010), physical (Rodrigues and Castro, 2008), and biological (Fore et al., 2001; Vail et al., 2003; Gowan et al., 2007; Stepenuck et al., 2011) monitoring.” ([Lowry et al., 2019, p. 4](zotero://select/groups/4773535/items/3TVRMCP5)) ([pdf](zotero://open-pdf/groups/4773535/items/TDWAUJIY?page=4&annotation=UYNVBR7F))
“Notably, in all of these studies, volunteers were trained to carry out the monitoring in which they were engaged.” ([Lowry et al., 2019, p. 4](zotero://select/groups/4773535/items/3TVRMCP5)) ([pdf](zotero://open-pdf/groups/4773535/items/TDWAUJIY?page=4&annotation=ZNP3EUNT))
“Limited validation in the CrowdHydrology project, using a co-located pressure transducer (Lowry and Fienen, 2013), revealed root mean square error of participant data versus researcher data of about 0.02 feet – roughly the resolution of a class A staff gauge.” ([Lowry et al., 2019, p. 4](zotero://select/groups/4773535/items/3TVRMCP5)) ([pdf](zotero://open-pdf/groups/4773535/items/TDWAUJIY?page=4&annotation=RAZTZSKF))


“Crowdsourcing as a Model for Problem Solving” ([“Crowdsourcing as a Model for Problem Solving”, p. 75](zotero://select/groups/4773535/items/NT56MWYM)) ([pdf](zotero://open-pdf/groups/4773535/items/TZI4NDBJ?page=1&annotation=FJTS2A7L))

!!!!!!!!!!!!!!!!!!!!!!!!!!!!!!!!!!!!!!!!!!!!!!!!!!!!!!!!!!!!!!!!!!!!!!!!!!!!!!!!!!!!!!!!!!!!!!!!!!!!!!!!!!!!!!!!!!!!!!!!!!!!!!!!!!!!!!!!!!!!!!!!!!!!!!!!!!!!!
“A Survey on Mobile Crowdsensing Systems: Challenges, Solutions, and Opportunities” ([Capponi et al., 2019, p. 2419](zotero://select/groups/4773535/items/ZWU5RVDR)) ([pdf](zotero://open-pdf/groups/4773535/items/LY8KZ2MZ?page=1&annotation=ELYSHCCA))
!!!!!!!!!!!!!!!!!!!!!!!!!!!!!!!!!!!!!!!!!!!!!!!!!!!!!!!!!!!!!!!!!!!!!!!!!!!!!!!!!!!!!!!!!!!!!!!!!!!!!!!!!!!!!!!!!!!!!!!!!!!!!!!!!!!!!!!!!!!!!!!!!!!!!!!!!!!!!


“This is because citizen  science is most successful when:   
• the aim/questions are clear;  
• engagement with people is given a high priority;  
• sufficient resources are available to begin and continue the project until its completion;  
• scale of sampling is relatively large (because it is often not cost‐efficient to use a citizen  science approach at small spatio‐temporal scales);  
• the protocol required for data collection is not too complex.”
([Pocock et al., p. 3](zotero://select/groups/4773535/items/XDEG6K4D)) ([pdf](zotero://open-pdf/groups/4773535/items/GWG6VVIR?page=4&annotation=RPCPV6R4))

“Three main benefits of citizen science are identified:” ([Minkman, 2015, p. 173](zotero://select/groups/4773535/items/ZKLE6CPT)) ([pdf](zotero://open-pdf/groups/4773535/items/QMAPCSZG?page=173&annotation=MSPY93QA))
1. Low cost data collection method
2. Flexible monitoring scheme
3. Involves Citizens


%-----------------------------------
%	SUBSECTION 2 Is volunteer monitoring any good?
%-----------------------------------

“Results from the statewide TST dataset include 82 separate station/year ANOVAs and demonstrate that large-scale, existing volunteer and professional data with unpaired samples can show agreement of~80\% for all analyzed parameters (DO =77\%, pH =79\%, conductivity = 85\%). Inaddition, toassess whether limiting variation within the source datasets increased the level ofagreement between volunteers and professionals, data were analyzed atalocal scale. Data from asingle partner city, with increased controls on sampling times and locations and correction ofasystematic bias inDO, confirmed this by showing an even greater agreement of91\% overall from 2009–2017 (DO =91\%, pH =83\%, conductivity =100\%).” ([Albus et al., 2020, p. 1](zotero://select/groups/4773535/items/DMCGSGUU)) ([pdf](zotero://open-pdf/groups/4773535/items/PRI5ALTR?page=1\&annotation=63J3WN2P))

“The results suggest that citizen science can be a cost-effective method to collect essential monitoring information and can also produce the high levels of citizen engagement that are vital to the adaptive management learning process.” ([Aceves-Bueno et al., 2015, p. 493](zotero://select/groups/4773535/items/YK2MKLA9)) ([pdf](zotero://open-pdf/groups/4773535/items/WGGHNGZB?page=1&annotation=ZQQBEP74))




%-----------------------------------
%	SUBSECTION 3
%-----------------------------------

\section{Crowdsensing and Mobile Phones}
timeline and extensive insights:
“B. Timeline” ([Capponi et al., 2019, p. 2423](zotero://select/groups/4773535/items/ZWU5RVDR)) ([pdf](zotero://open-pdf/groups/4773535/items/LY8KZ2MZ?page=5&annotation=SE5W8NPH))
“III. MOBILE CROWDSENSING IN A NUTSHELL” ([Capponi et al., 2019, p. 2426](zotero://select/groups/4773535/items/ZWU5RVDR)) ([pdf](zotero://open-pdf/groups/4773535/items/LY8KZ2MZ?page=8&annotation=2H7S77GL))

“Participatory Sensing or Participatory Nonsense? — Mitigating the Effect of Human Error on Data Quality in Citizen Science” ([Budde et al., 2017, p. 391](zotero://select/groups/4773535/items/8422CFKA)) ([pdf](zotero://open-pdf/groups/4773535/items/5N8N23KN?page=1&annotation=VAW3B38X))

“MOBILE PHONE APPLICATIONS FOR WATER MANAGEMENT: CLASIFFICATION, OPPORTUNITIES AND CHALLENGES” ([Alfonso and Jonoski, 2012, p. 2](zotero://select/groups/4773535/items/4W4Q8E6B)) ([pdf](zotero://open-pdf/groups/4773535/items/PU944FGI?page=2&annotation=QV3HATMM))

“Building a Platform to Collect Crowdsensing Data. Preliminary Considerations” ([George et al., 2017, p. 2](zotero://select/groups/4773535/items/GCM2RQZ2)) ([pdf](zotero://open-pdf/groups/4773535/items/RQ6QQ58V?page=2&annotation=3NXULEJJ))

%----------------------------------------------------------------------------------------
%	SECTION 7 Previous Tool and Studies 
%----------------------------------------------------------------------------------------
Lake water quality 
“The results of this review, however, highlighted no usable preexisting system for SIMILE project, since the considered applications are too tied to specific geographic locations or too broad presenting no real connection with the lake ecosystem.” ([Carrion et al., 2020, p. 246](zotero://select/groups/4773535/items/7855TDUA)) ([pdf](zotero://open-pdf/groups/4773535/items/XI5TRN34?page=2&annotation=4P6WS72K))

“Moreover, the majority of the explored solutions are not free and open source.” ([Carrion et al., 2020, p. 246](zotero://select/groups/4773535/items/7855TDUA)) ([pdf](zotero://open-pdf/groups/4773535/items/XI5TRN34?page=2&annotation=EY6HDG2D))

“Transmitting the observations using simple cell phones and text messages turned out to be stable and reliable without major technical problems.” ([Weeser et al., 2018, p. 1597](zotero://select/groups/4773535/items/SFA2MLHC)) ([pdf](zotero://open-pdf/groups/4773535/items/GP79FHFC?page=8&annotation=R57CUAHW))

“Text messages are a common way of communication and significantly lowered the technical barrier to contribute and send data.” ([Weeser et al., 2018, p. 1597](zotero://select/groups/4773535/items/SFA2MLHC)) ([pdf](zotero://open-pdf/groups/4773535/items/GP79FHFC?page=8&annotation=NH68HN6C))

“Furthermore, the feedback loop allowed participants to identify whether their observation was correctly received.” ([Weeser et al., 2018, p. 1597](zotero://select/groups/4773535/items/SFA2MLHC)) ([pdf](zotero://open-pdf/groups/4773535/items/GP79FHFC?page=8&annotation=BM5TYHNZ))

“We occasionally faced phone network coverage issues.” ([Weeser et al., 2018, p. 1597](zotero://select/groups/4773535/items/SFA2MLHC)) ([pdf](zotero://open-pdf/groups/4773535/items/GP79FHFC?page=8&annotation=K6NJJJY9))
“However, those stations with restricted network availability did not turn out as a limited factor for data contribution.” ([Weeser et al., 2018, p. 1597](zotero://select/groups/4773535/items/SFA2MLHC)) ([pdf](zotero://open-pdf/groups/4773535/items/GP79FHFC?page=8&annotation=NF5R6J5H))
“Observers took the readings of the water level and waited until they reached an area with network coverage to send their messages.” ([Weeser et al., 2018, p. 1597](zotero://select/groups/4773535/items/SFA2MLHC)) ([pdf](zotero://open-pdf/groups/4773535/items/GP79FHFC?page=8&annotation=D659LB5J))

“In comparison to more sophisticated methods, like using smartphones, we believe that this approach produces more and, in turn, more reliable results in a low-income country because wrong data and outliers become obvious.” ([Weeser et al., 2018, p. 1597](zotero://select/groups/4773535/items/SFA2MLHC)) ([pdf](zotero://open-pdf/groups/4773535/items/GP79FHFC?page=8&annotation=VP5QT8UJ))

“A major weakness of the existing tools is the emphasis on macro/international level information.” ([Masinde and Bagula, 2010, p. 390](zotero://select/groups/4773535/items/JNC4ACZS)) ([pdf](zotero://open-pdf/groups/4773535/items/IWMKDQYV?page=1&annotation=7Z44Z9L7))

“The tools also tend to ignore the at risk community who happen to be host to very crucial traditional knowledge on droughts” ([Masinde and Bagula, 2010, p. 390](zotero://select/groups/4773535/items/JNC4ACZS)) ([pdf](zotero://open-pdf/groups/4773535/items/IWMKDQYV?page=1&annotation=VUYYP2LA))

“The analysis also provides a set of recommendations for citizen science program design that addresses spatial and temporal scale, data quality, costs, and effective incentives to facilitate participation and integration of findings into adaptive management.” ([Aceves-Bueno et al., 2015, p. 493](zotero://select/groups/4773535/items/YK2MKLA9)) ([pdf](zotero://open-pdf/groups/4773535/items/WGGHNGZB?page=1&annotation=3NRZ7R8Y))

%-----------------------------------
%	SUBSECTION Tools
%-----------------------------------

\subsubsection{CBS}
https://www.cbsrc.org/

\subsubsection{NYSS}

\subsubsection{CoCoRaHS}

\subsubsection{Ushahidi} %ggf. # Sahana Eden (just as a project what is done in the open-source software world)

\subsubsection{Social.Water}
foundation for CrowdHydrology
\subsubsection{CrowdHydrology (?)} 
“The CrowdHydrology network is dominated by one-time participants who submitted just under half of all observations received into the system” ([Lowry et al., 2019, p. 4](zotero://select/groups/4773535/items/3TVRMCP5)) ([pdf](zotero://open-pdf/groups/4773535/items/TDWAUJIY?page=4&annotation=I7D2QPWW))
“Modifying the methodology in these two ways could fix perceived issues of lack of information, interest, and feedback among onetime participants.” ([Lowry et al., 2019, p. 6](zotero://select/groups/4773535/items/3TVRMCP5)) ([pdf](zotero://open-pdf/groups/4773535/items/TDWAUJIY?page=6&annotation=4XPYEMTQ))
“Based on opportunistic conversations with participants we have learned that citizen scientists would like feedback that their message has been received. This is consistent with others’ findings as well (Devlin et al., 2001; Rotman et al., 2014).” ([Lowry et al., 2019, p. 7](zotero://select/groups/4773535/items/3TVRMCP5)) ([pdf](zotero://open-pdf/groups/4773535/items/TDWAUJIY?page=7&annotation=PPU3INB5))
“When we initiated the project, we were concerned that a response message might trigger privacy concerns among participants, but subsequent literature has suggested that the value to participation would likely eclipse such privacy concerns, and in our response, we can include an opt-out option.” ([Lowry et al., 2019, p. 7](zotero://select/groups/4773535/items/3TVRMCP5)) ([pdf](zotero://open-pdf/groups/4773535/items/TDWAUJIY?page=7&annotation=7C3X7EBN))
“While altruistic reasons have been observed to be a motivation for volunteers early in a program, we know that building personal networks is cited as a valued outcome and motivator for continued participation over time (Ryan et al., 2001; Gooch, 2005). This deserves future attention.” ([Lowry et al., 2019, p. 8](zotero://select/groups/4773535/items/3TVRMCP5)) ([pdf](zotero://open-pdf/groups/4773535/items/TDWAUJIY?page=8&annotation=PH9MDNBM))

---
very similar approach:
+ “Citizen science pioneers in Kenya – A crowdsourced approach for hydrological monitoring” ([Weeser et al., 2018, p. 1590](zotero://select/groups/4773535/items/SFA2MLHC)) ([pdf](zotero://open-pdf/groups/4773535/items/GP79FHFC?page=1&annotation=R92YZEVU))

\subsubsection{ITIKI}
“Development Journal 283 Forecast Indicators Integrated Weather Forecasting Application Server Indigenous Knowledge Webserver Indigenous Knowledge Database Server Indigenous Knowledge Focus Group Village Internet Kiosk 2a 3 2b Query Optimizer and Data Mining Tools PC for Uploading Data if WAP not Active Intermediary Intermediary Travels to Digital Village Source 1 Source n-1” ([Masinde and Bagula, 2012, p. 283](zotero://select/groups/4773535/items/EW9XSSZP)) ([pdf](zotero://open-pdf/groups/4773535/items/3WQ4S9PE?page=10&annotation=AXQH5Q7Z))

\subsubsection{}


%-----------------------------------
%	SUBSECTION Motivation of Citizen Scientists
“we conclude that the active participation is not depending on the actual education level but rather induced by their personal perception of and dependency on their environment. Especially citizens who depend on local water resources are expected to be interested in increasing their understanding of their environment and to participate in local political decisions to ensure a sustainable use of their resources (Overdevest et al., 2004).” ([Weeser et al., 2018, p. 1596](zotero://select/groups/4773535/items/SFA2MLHC)) ([pdf](zotero://open-pdf/groups/4773535/items/GP79FHFC?page=7&annotation=58MAGPAQ))

“At the same time, low participation rates at some stations can be attributed partly to the transmitting cost of 0.01 USD per text message, which was paid by the volunteers. Especially in rural areas, participants expressed that they might be unable to participate due to costs.” ([Weeser et al., 2018, p. 1596](zotero://select/groups/4773535/items/SFA2MLHC)) ([pdf](zotero://open-pdf/groups/4773535/items/GP79FHFC?page=7&annotation=86AIS5QQ))

“Buytaert et al. (2014) described that observers in low-income countries often derive an income from their engagement in citizen-science projects. These authors argue, that the concept of sending data voluntarily is not well developed, and that it may be necessary to reward people at local wages for motivation.” ([Weeser et al., 2018, p. 1596](zotero://select/groups/4773535/items/SFA2MLHC)) ([pdf](zotero://open-pdf/groups/4773535/items/GP79FHFC?page=7&annotation=JUZWKKNE))

“We found that paying a small reward that covers the costs significantly increases the overall participation rate.” ([Weeser et al., 2018, p. 1596](zotero://select/groups/4773535/items/SFA2MLHC)) ([pdf](zotero://open-pdf/groups/4773535/items/GP79FHFC?page=7&annotation=PTWU32DL))
--> seven times larger with reimbursement system in place


or: collective coverage of the costs

“Instead of a reimbursement centrally paid by the project, interested water users organised an own reward system by collecting a contribution from several users to reimburse one person recording the water level data. However, a real payment or reward was not necessary, since the intrinsic motivation of the participants seemed to be sufficient when lack of money was overcome.” ([Weeser et al., 2018, p. 1597](zotero://select/groups/4773535/items/SFA2MLHC)) ([pdf](zotero://open-pdf/groups/4773535/items/GP79FHFC?page=8&annotation=S4IKR9K9))

%-----------------------------------
%-----------------------------------
%	SUBSECTION 3 response
%-----------------------------------
The most frequently identified actors responding to water scarcity include individuals or households (32\%), local government (15\%) and national government (15\%), while the most common types of response are behavioural and cultural (30\%), technological and infrastructural (27\%), ecosystem-based (25\%) and institutional (18\%).


